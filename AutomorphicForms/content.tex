%% Delete this \nocite command invocation to make the references section only list out the bibitems that are actually cited.
\nocite{*}

\section{Modular forms}

\begin{definition} \label{definition:congruence_subgroup_of_SL_2_Z}
    \TODO{define congruence subgroups for more general arithmetic groups and describe how these are instances of those}
Let $N$ be a positive integer.
\begin{itemize}
    \item The \hldef{principal congruence subgroup of $\mathrm{SL}_2(\bbZ)$ of level $N$} is the group \hl{$\Gamma(N)$} defined by
    $$ \Gamma(N) := \left\{ \gamma \in \mathrm{SL}_2(\mathbb{Z}) : \gamma \equiv I_2 \pmod{N} \right\}.  $$
    \item The subgroup \hl{$\Gamma_1(N)$} of $\SL_2(\bbZ)$ is defined by
    $$ \Gamma_1(N) := \left\{ \gamma = \begin{pmatrix} a & b \\ c & d \end{pmatrix} \in \mathrm{SL}_2(\mathbb{Z}) : a \equiv d \equiv 1 \pmod{N},\; c \equiv 0 \pmod{N} \right\}.  $$
    \item The subgroup \hl{$\Gamma_0(N)$} of $\SL_2(\bbZ)$ is defined by
    $$ \Gamma_0(N) := \left\{ \gamma = \begin{pmatrix} a & b \\ c & d \end{pmatrix} \in \mathrm{SL}_2(\mathbb{Z}) : c \equiv 0 \pmod{N} \right\}.  $$
    \item A \hldef{congruence subgroup of $\mathrm{SL}_2(\mathbb{Z})$} is any subgroup $\Gamma \leq \mathrm{SL}_2(\mathbb{Z})$ that contains $\Gamma(N)$ for some $N \geq 1$.
\end{itemize}
Note that
$$\Gamma(N) \subset \Gamma_1(N) \subset \Gamma_0(N) \subset \SL_2(\bbZ).$$
Moreover, the natural reduction homomorphism
$$\SL_2(\bbZ) \to \SL_2(\bbZ/N \bbZ)$$
is surjective with kernel $\Gamma(N)$. Therefore, we have an isomorphism 
$$\SL_2(\bbZ) / \Gamma(N) \to \SL_2(\bbZ/N \bbZ).$$
In particular, $\Gamma(N)$ and hence $\Gamma_1(N)$ and $\Gamma_0(N)$ are all of finite index in $\SL_2(\bbZ / N \bbZ)$.
\end{definition}

\begin{lemma} \label{lemma:every_congruence_subgroup_of_SL_2_Z_has_a_translation_matrix}
    Let $\Gamma \subset \SL_2(\bbZ)$ be a \CrefAndHyperref{definition:congruence_subgroup_of_SL_2_Z}{congruence subgroup of $\SL_2(\bbZ)$}. There is some integer $h \in \bbZ_{> 0}$ such that $\begin{pmatrix} 1 & h \\ 0 & 1 \end{pmatrix} \in \Gamma$.
\end{lemma}


\begin{definition}[Modular group action] \label{definition:modular_action_of_a_subgroup_of_SL_2_Z_on_the_upper_half_plane}
    Let $\Gamma \subseteq \mathrm{SL}_2(\mathbb{Z})$ be a subgroup. For $\gamma = \begin{pmatrix} a & b \\ c & d \end{pmatrix}\in \Gamma$ and $\tau \in \mathbb{H} := \{ z \in \mathbb{C} \mid \operatorname{Im}(z) > 0 \}$, define the action
    $$\hlin{\gamma \cdot \tau := \frac{a\tau + b}{c\tau + d}.}$$
    This is called the \hldef{modular action of $\Gamma$ on the upper half-plane $\mathbb{H}$}.
\end{definition}

\begin{definition}[Weight-$k$ slash operator] \label{definition:weight_k_operator_on_meromorphic_functions_by_elements}
Let $k \in \mathbb{Z}$. For $\gamma = \begin{pmatrix} a & b \\ c & d \end{pmatrix}\in \SL_2(\bbZ)$ and any function $f: \mathbb{H} \to \mathbb{C}$, the \hldef{weight-$k$ slash operator} (or simply the \hldef{weight-$k$ operator}) is defined as
$$\hlin{(f|_k\gamma)(\tau) := (c\tau+d)^{-k} f\!\left(\frac{a\tau+b}{c\tau+d}\right).}$$
$f|_k \gamma$ is also often denoted by \hl{$f[\gamma]_k$}.
\end{definition}

\begin{definition}[Weakly modular form] \label{definition:weakly_modular_form_with_respect_to_a_subgroup_of_SL_2_Z}
Let $k \in \mathbb{Z}$, and let $\Gamma \subseteq \mathrm{SL}_2(\mathbb{Z})$ be a subgroup. A meromorphic function $f:\mathbb{H}\to\mathbb{C}$ is called a \hldef{weakly modular form of weight $k$ with respect to $\Gamma$} if
\[ f|_k\gamma = f \quad \text{for all } \gamma \in \Gamma.  \]
(\Cref{definition:weight_k_operator_on_meromorphic_functions_by_elements}).
\end{definition}

\begin{lemma} \label{lemma:weakly_modular_form_for_any_congruence_subgroup_of_SL_2_Z_is_periodic}
    Let $\Gamma \subset \SL_2(\bbZ)$ be a \CrefAndHyperref{definition:congruence_subgroup_of_SL_2_Z}{congruence subgroup of $\SL_2(\bbZ)$}.
    Let $h \in \bbZ_{> 0}$ be such that $\begin{pmatrix} 1 & h \\ 0 & 1 \end{pmatrix}$ by \Cref{lemma:every_congruence_subgroup_of_SL_2_Z_has_a_translation_matrix}. 
    Every \CrefAndHyperref{definition:weakly_modular_form_with_respect_to_a_subgroup_of_SL_2_Z}{weakly modular form $f$} of some weight $k \in \bbZ$ is $h\bbZ$-periodic, i.e. $f(z+h) = f(z)$ for all $z \in \bbH$ \TODO{cite the upper half plane}.

    In particular, there exists a function $g: D ' \to \bbC$ where $D' \subset \bbC$ is the punctured unit disk such that $f(z) = g(q_h)$ where $q_h = e^{2 \pi i z/h}$.
\end{lemma}

\begin{definition} \label{definition:holomorphic_at_infinity_for_a_meromorphic_function_on_the_upper_half_plane}
    Let $\Gamma \subset \SL_2(\bbZ)$ be a \CrefAndHyperref{definition:congruence_subgroup_of_SL_2_Z}{congruence subgroup of $\SL_2(\bbZ)$}.
    Let $h \in \bbZ_{> 0}$ be minimal such that $\begin{pmatrix} 1 & h \\ 0 & 1 \end{pmatrix}$ by \Cref{lemma:every_congruence_subgroup_of_SL_2_Z_has_a_translation_matrix}. 

    Let $f: \bbH \to \bbC$ be a meromorphic function of the form $f(z) = g(q_h)$ for some meromorphic function $g: D' \to \bbC$ on the punctured unit disk $D'$ where $q_h = e^{2 \pi i z/h}$ (e.g. \CrefAndHyperref{definition:weakly_modular_form_with_respect_to_a_subgroup_of_SL_2_Z}{weakly modular forms} satisfy this condition, see \Cref{lemma:weakly_modular_form_for_any_congruence_subgroup_of_SL_2_Z_is_periodic}). In particular, $g$ has a Laurent expansion near $q_h = 0$ (which corresponds to $\operatorname{Im}(z) \to \infty$) \TODO{cite this as a complex analytic fact} and hence $f$ has a Fourier expansion: 

    $$f(z) = \sum_{n=-\infty}^\infty a_n q_h^n.$$
    
    The values $a_n \in \bbC$ are called the \hldef{Fourier coefficients of $f$}.

    We say that $f$ is \hldef{meromorphic at $\infty$} if $g$ is meromorphic at $q_h = 0$, i.e. the Fourier expansion of $f$ has the form
    $$f(z) = \sum_{n=n_0}^\infty a_n q_h^n.$$
    for some $n_0 \in \bbZ$. We say that $f$ is \hldef{holomorphic at $\infty$} if $g$ is holomorphic at $q_h = 0$, i.e. the Fourier expansion of $f$ has the form
    $$f(z) = \sum_{n=0}^\infty a_n q_h^n.$$
    If $f$ is holomorphic at $\infty$ and we additionally have $a_0 = 0$, then we say that $f$ has a \hldef{cusp at $\infty$}.
\end{definition}


\begin{definition} \label{definition:modular_form_of_weight_k_with_respect_to_a_congruence_subgroup_of_SL_2_Z}
    \TODO{define holomorphic function, upper half plane}
    \TODO{define cusps of $\Gamma$}
Let $k$ be an integer and $\Gamma \leq \mathrm{SL}_2(\mathbb{Z})$ a \CrefAndHyperref{definition:congruence_subgroup_of_SL_2_Z}{congruence subgroup}. A \hldef{modular form of weight $k$ with respect to $\Gamma$} is a function $f : \mathbb{H} \to \mathbb{C}$, where $\mathbb{H}$ is the upper half-plane, such that 
\begin{itemize}
    \item The function $f$ is holomorphic on $\mathbb{H}$.
    \item The funciton $f$ is a \CrefAndHyperref{definition:weakly_modular_form_with_respect_to_a_subgroup_of_SL_2_Z}{weakly modular form of weight $k$ with respect to $\Gamma$}. Equivalently, for all $\gamma = \begin{pmatrix} a & b \\ c & d \end{pmatrix}\in \Gamma$ and all $z \in \mathbb{H}$, we have
    $$ f\left( \frac{a z + b}{c z + d} \right) = (c z + d)^k f(z).  $$
    \item $f$ is holomorphic at the cusps of $\Gamma$ (i.e., its Fourier expansion at all cusps of $\Gamma$ is holomorphic). More precisely, \CrefAndHyperref{definition:weight_k_operator_on_meromorphic_functions_by_elements}{$f|_k \alpha$} is \CrefAndHyperref{definition:holomorphic_at_infinity_for_a_meromorphic_function_on_the_upper_half_plane}{holomorphic at $\infty$} for all $\alpha \in \Gamma$.
\end{itemize}
A \hldef{cusp form} is a modular form $f$ whose Fourier expansion at every cusp of $\Gamma$ has constant term $0$, i.e. $f|_k \alpha$ has a cusp at $\infty$ for every $\alpha \in \Gamma$.
\end{definition}


\begin{definition} \label{definition:automorphic_form_of_weight_k_with_respect_to_a_congruence_subgroup_of_SL_2_Z}
    \TODO{define holomorphic function, upper half plane}
    \TODO{define cusps of $\Gamma$}
Let $k$ be an integer and $\Gamma \leq \mathrm{SL}_2(\mathbb{Z})$ a \CrefAndHyperref{definition:congruence_subgroup_of_SL_2_Z}{congruence subgroup}. A \hldef{modular form of weight $k$ with respect to $\Gamma$} is a function $f : \mathbb{H} \to \mathbb{C}$, where $\mathbb{H}$ is the upper half-plane, such that 
\begin{itemize}
    \item The function $f$ is meromorphic on $\mathbb{H}$.
    \item The funciton $f$ is a \CrefAndHyperref{definition:weakly_modular_form_with_respect_to_a_subgroup_of_SL_2_Z}{weakly modular form of weight $k$ with respect to $\Gamma$}. Equivalently, for all $\gamma = \begin{pmatrix} a & b \\ c & d \end{pmatrix}\in \Gamma$ and all $z \in \mathbb{H}$, we have
    $$ f\left( \frac{a z + b}{c z + d} \right) = (c z + d)^k f(z).  $$
    \item $f$ is meromorphic at the cusps of $\Gamma$ (i.e., its Fourier expansion at all cusps of $\Gamma$ is holomorphic). More precisely, \CrefAndHyperref{definition:weight_k_operator_on_meromorphic_functions_by_elements}{$f|_k \alpha$} is \CrefAndHyperref{definition:holomorphic_at_infinity_for_a_meromorphic_function_on_the_upper_half_plane}{meromorphic at $\infty$} for all $\alpha \in \Gamma$.
\end{itemize}
\end{definition}

We define the $L$-function of a modular form by putting the Fourier coefficients of the modular form into a Dirichlet series.

\begin{definition}[L-function of a modular form] \label{definition:L_function_of_a_modular_form}
Let $k \in \mathbb{Z}_{>0}$, and let $f:\mathbb{H}\to\mathbb{C}$ be a \CrefAndHyperref{definition:modular_form_of_weight_k_with_respect_to_a_congruence_subgroup_of_SL_2_Z}{modular form} of weight $k$ with respect to a \CrefAndHyperref{definition:congruence_subgroup_of_SL_2_Z}{congruence subgroup} $\Gamma \subseteq \mathrm{SL}_2(\mathbb{Z})$. Let
$$ f(\tau) = \sum_{n=0}^\infty a_n e^{\frac{2\pi i n \tau}{h}} $$
be the \CrefAndHyperref{definition:holomorphic_at_infinity_for_a_meromorphic_function_on_the_upper_half_plane}{Fourier expansion of $f$} where $h > 0$ is an appropriate integer. 

The \hldef{L-function of the modular form $f$} is the complex function defined for $s \in \mathbb{C}$ with sufficiently large $\operatorname{Re}(s)$ by the Dirichlet series
$$ L(f,s) := \sum_{n=1}^\infty \frac{a_n}{n^s}.  $$
\TODO{define the completed $L$-function}
\end{definition}

The construction of the $L$-function of a modular form is not arbitrary --- a Fourier transform involving the modular form yields the completed $L$-function.

\begin{theorem}
Let $k \in \mathbb{Z}_{>0}$, and let $f:\mathbb{H}\to\mathbb{C}$ be a \CrefAndHyperref{definition:modular_form_of_weight_k_with_respect_to_a_congruence_subgroup_of_SL_2_Z}{modular form} of weight $k$ with respect to a \CrefAndHyperref{definition:congruence_subgroup_of_SL_2_Z}{congruence subgroup} $\Gamma \subseteq \mathrm{SL}_2(\mathbb{Z})$. Let
$$ f(\tau) = \sum_{n=0}^\infty a_n e^{\frac{2\pi i n \tau}{h}} $$
be the \CrefAndHyperref{definition:holomorphic_at_infinity_for_a_meromorphic_function_on_the_upper_half_plane}{Fourier expansion of $f$} where $h > 0$ is an appropriate integer. 

The Mellin transform of $f(i\tau)$ equals the completed $L$-function $\Lambda(f,s)$.
\end{theorem}

\begin{proof}
    Let \(f: \mathbb{H} \to \mathbb{C}\) be a meromorphic function periodic with period \(h > 0\), so
    \[
    f(z+h) = f(z).
    \]
    This implies that \(f\) factors through the variable
    \[
    q_h := e^{2\pi i z / h}.
    \]
    Define
    \[
    \Phi(t) := f(i t), \quad t > 0.
    \]

    Using the periodicity, \(f\) has a Fourier expansion in \(q_h\):
    \[
    f(z) = \sum_{n=n_0}^\infty a_n q_h^n = \sum_{n=n_0}^\infty a_n e^{2 \pi i n z / h}.
    \]

    Substituting \(z = it\), we have
    \[
    \Phi(t) = f(i t) = \sum_{n=n_0}^\infty a_n e^{-2 \pi n t / h}.
    \]

    Consider the Mellin transform integral with complex parameter \(s\):
    \[
    I(s) := \int_0^\infty \Phi(t) t^{s} \frac{dt}{t}.
    \]

    For \(\mathrm{Re}(s)\) sufficiently large, interchange sum and integral (justified by absolute convergence):
    \[
    I(s) = \sum_{n=n_0}^\infty a_n \int_0^\infty e^{-2\pi n t / h} t^{s} \frac{dt}{t}.
    \]

    Evaluate the integral inside for each fixed \(n\). Change variables by setting \(u = \frac{2\pi n t}{h}\), so \(t = \frac{h u}{2 \pi n}\), and
    \[
    dt = \frac{h}{2 \pi n} du.
    \]
    Hence,
    \[
    \int_0^\infty e^{-2\pi n t / h} t^{s} \frac{dt}{t} = \int_0^\infty e^{-u} \left(\frac{h u}{2 \pi n}\right)^s \frac{du}{u} = \left(\frac{h}{2 \pi n}\right)^s \int_0^\infty e^{-u} u^{s} \frac{du}{u}.
    \]

    Recognizing the Gamma function, we get
    \[
    \int_0^\infty e^{-u} u^{s} \frac{du}{u} = \Gamma(s).
    \]

    Therefore,
    \[
    I(s) = \Gamma(s) \left(\frac{h}{2\pi}\right)^s \sum_{n=n_0}^\infty a_n n^{-s}.
    \]

    By definition, the completed \(L\)-function \(\Lambda(f,s)\) associated to \(f\) is proportional to this Mellin transform expression:
    \[
    \Lambda(f,s) = \Gamma(s) \left(\frac{h}{2\pi}\right)^s L(f,s),
    \]
    where
    \[
    L(f,s) := \sum_{n=n_0}^\infty \frac{a_n}{n^s}.
    \]

    This establishes the integral representation of \(\Lambda(f,s)\) as the Mellin transform of \(\Phi(t) = f(it)\), adapted to period \(h\).
\end{proof}


\subsection{$L$-function of a modular form}





\section{Automorphic forms}


\begin{definition}[Adelic Quotient] \label{definition:adelic_quotient_of_a_linear_algebraic_group_over_a_global_field}
Let $F$ be a \CrefAndHyperrefIfExist{definition:global_field}{global field}, and let $\A_F$ be the \CrefAndHyperrefIfExist{definition:adeles_and_ideles_of_a_global_field}{ring of adeles} of $F$. Let $G$ be a \CrefAndHyperrefIfExist{definition:linear_algebraic_group_over_a_scheme}{linear algebraic group} defined over $F$. 
The \hldef{adelic quotient of $G$ over $F$} is the quotient space
$$ G(F) \backslash G(\A_F).  $$
\CrefIfExists{definition:group_of_adelic_points_of_an_algebraic_group_over_a_global_field} where $G(F)$ is given the \CrefAndHyperrefIfExist{definition:discrete_topological_space_of_a_set}{discrete topology}.
\TODO{ What topology does $G(F)$ have?}
\end{definition}

\begin{definition}[Automorphy] \label{definition:automorphy_for_a_function_on_the_adelic_quotient_group_of_a_linear_algebraic_group_over_a_global_field}
    Let $F$ be a \CrefAndHyperrefIfExist{definition:global_field}{global field}, and let $\A_F$ be the \CrefAndHyperrefIfExist{definition:adeles_and_ideles_of_a_global_field}{ring of adeles of $F$}. Let $G$ be a \CrefAndHyperrefIfExist{definition:linear_algebraic_group_over_a_scheme}{linear algebraic group} defined over $F$.

    Let $f: G(\A_F) \to \C$ be a function.  
    We say $f$ satisfies \hldef{automorphy} if it is left $G(F)$-invariant, i.e.,
    $$
    f(\gamma g) = f(g) \quad \text{for all } \gamma \in G(F), \ g \in G(\A_F).
    $$
    Equivalently, $f$ is a function which descends to a well defined function $f:G(F)\backslash G(\A_F) \to \C$ on the \CrefAndHyperrefIfExist{definition:adelic_quotient_of_a_linear_algebraic_group_over_a_global_field}{adelic quotient}.
\end{definition}


\begin{notation}[Right-regular action of $G(\A_F)$] \label{definition:right_regular_action_of_the_adeles_of_a_linear_algebraic_group_over_a_global_field_on_the_space_of_complex_valued_functions_on_the_adelic_quotient}
Let $F$ be a \CrefAndHyperrefIfExist{definition:global_field}{global field}, and let $\A_F$ be the \CrefAndHyperrefIfExist{definition:adeles_and_ideles_of_a_global_field}{ring of adeles} of $F$. Let $G$ be a \CrefAndHyperrefIfExist{definition:linear_algebraic_group_over_a_scheme}{linear algebraic group} defined over $F$. 
    For \CrefAndHyperrefIfExist{definition:group_of_adelic_points_of_an_algebraic_group_over_a_global_field}{$g \in G(\A_F)$} and a function $f : G(F) \backslash G(\A_F) \to \C$, the \hldef{right-regular action/representation of $G(\A_F)$} is the representation
    $$\hlin{R: G(\bbA_F) \to \Aut(\Fun(G(F) \backslash G(\bbA_F), \bbC))}$$
    on the space of complex valued functions on $G(F) \backslash G(\bbA_F)$ given by 
    $$(R(g)f)(x) = f(xg)$$
    for $g \in G(\bbA_F)$, $f: G(F) \backslash G(\bbA_F) \to \bbC$, and $x \in G(F) \backslash G(\bbA_F)$.
\end{notation}

\begin{definition}[Smooth function] \label{definition:smooth_function_on_the_adelic_quotient_group_of_a_linear_algebraic_group_over_a_global_field}
    Let $F$ be a \CrefAndHyperrefIfExist{definition:global_field}{global field}, and let $\A_F$ be the \CrefAndHyperrefIfExist{definition:adeles_and_ideles_of_a_global_field}{ring of adeles of $F$}. Let $G$ be a \CrefAndHyperrefIfExist{definition:linear_algebraic_group_over_a_scheme}{linear algebraic group} defined over $F$. 

    Let $f: G(F) \backslash G(\A_F) \to \C$ be a function. We say $f$ is \hldef{smooth} if it is 
        \TODO{define $G(\bbA_F)$, $G(\bbA_F^{\mathrm{fin}})$, $G(F_\infty)$}
    \begin{enumerate}
        \item locally constant in the non-archimedean directions, i.e. for every point $g = (g_v)_v \in G(\bbA_F)$, there exists an open compact subgroup $U = \prod_v U_v \subset G(\bbA_F^{\mathrm{fin}})$\CrefIfExists{definition:group_of_adelic_points_of_an_algebraic_group_over_a_global_field} such that for all $u \in U$, we have 
        $$f(gu) = f(g),$$
        and
        \item $C^\infty$ in each archimedean direction, i.e. for every point $g = (g_v)_v \in G(\bbA_F)$, the map 
        $$G(F_\infty) \to \bbC, (h_v)_{v \mid \infty} \mapsto f(g \cdot(h_v)_{v \mid \infty})$$
        \CrefIfExists{definition:group_of_adelic_points_of_an_algebraic_group_over_a_global_field}
        is \CrefAndHyperrefIfExist{definition:C_k_morphism_between_C_k_manifolds}{$C^\infty$} as a \CrefAndHyperrefIfExist{definition:C_k_manifold}{smooth manifold}; note that $G(F_\infty) = \prod_{v \mid \infty} G(F_v)$ is a product of real and complex Lie groups.
        \TODO{define real and omcplex lie groups}
    \end{enumerate}
\end{definition}

\begin{definition}[$K$-finiteness] \label{definition:K_finite_function_on_the_adelic_quotient_group_of_a_linear_algebraic_group_over_a_global_field}
    Let $F$ be a \CrefAndHyperrefIfExist{definition:global_field}{global field}, and let $\A_F$ be the \CrefAndHyperrefIfExist{definition:adeles_and_ideles_of_a_global_field}{ring of adeles of $F$}. Let $G$ be a \CrefAndHyperrefIfExist{definition:linear_algebraic_group_over_a_scheme}{linear algebraic group} defined over $F$. 

    Let $K \subseteq G(\A_F)$ be a maximal compact subgroup. 
    A \CrefAndHyperrefIfExist{definition:smooth_function_on_the_adelic_quotient_group_of_a_linear_algebraic_group_over_a_global_field}{smooth function} $f: G(F) \backslash G(\A_F) \to \C$ is called \hldef{$K$-finite} if the vector space 
    $$ \langle R(k)f : \; k \in K \rangle $$
    \CrefIfExists{definition:right_regular_action_of_the_adeles_of_a_linear_algebraic_group_over_a_global_field_on_the_space_of_complex_valued_functions_on_the_adelic_quotient} spanned by the $K$-translates of $f$ under the \CrefAndHyperrefIfExist{definition:right_regular_action_of_the_adeles_of_a_linear_algebraic_group_over_a_global_field_on_the_space_of_complex_valued_functions_on_the_adelic_quotient}{right-regular action} is finite-dimensional.
\end{definition}

\begin{definition}[Moderate growth] \label{definition:moderate_growth_for_a_function_on_the_adelic_quotient_group_of_a_linear_algebraic_group_over_a_global_field}
    Let $F$ be a \CrefAndHyperrefIfExist{definition:global_field}{global field}, and let $\A_F$ be the \CrefAndHyperrefIfExist{definition:adeles_and_ideles_of_a_global_field}{ring of adeles of $F$}. Let $G$ be a \CrefAndHyperrefIfExist{definition:linear_algebraic_group_over_a_scheme}{linear algebraic group} defined over $F$. 
    Let $\|\cdot\|$ be a fixed height function (or norm-like function) on $G(\A_F)$ compatible with $F$ and $G$ (e.g. the height function of \Cref{definition:height_function_on_the_adelic_points_of_a_linear_algebraic_group_over_a_global_field_with_respect_to_an_embedding}).

    A \CrefAndHyperrefIfExist{definition:smooth_function_on_the_adelic_quotient_group_of_a_linear_algebraic_group_over_a_global_field}{smooth function} $f: G(F)\backslash G(\A_F) \to \C$ is said to have \hldef{moderate growth} if there exist constants $C > 0$ and $N \geq 0$ such that
    $$
    |f(g)| \leq C \cdot \|g\|^N \quad \text{for all } g \in G(\A_F),
    $$
\end{definition}

\begin{definition}[Automorphic form] \label{definition:automorphic_form_for_a_linear_algebraic_group_over_a_global_field}
    Let $F$ be a \CrefAndHyperrefIfExist{definition:global_field}{global field}, and let $\A_F$ be the \CrefAndHyperrefIfExist{definition:adeles_and_ideles_of_a_global_field}{ring of adeles of $F$}. Let $G$ be a \CrefAndHyperrefIfExist{definition:linear_algebraic_group_over_a_scheme}{linear algebraic group} defined over $F$. 

    A function $f: G(F) \backslash G(\A_F) \to \C$ is called an \hldef{automorphic form} if it satisfies all of the following conditions:
    \begin{enumerate}
        \item $f$ is \CrefAndHyperrefIfExist{definition:smooth_function_on_the_adelic_quotient_group_of_a_linear_algebraic_group_over_a_global_field}{smooth},
        \item $f$ is \CrefAndHyperrefIfExist{definition:K_finite_function_on_the_adelic_quotient_group_of_a_linear_algebraic_group_over_a_global_field}{$K$-finite} for some maximal compact subgroup $K \subseteq G(\A_F)$,
        \item $f$ has \CrefAndHyperrefIfExist{definition:moderate_growth_for_a_function_on_the_adelic_quotient_group_of_a_linear_algebraic_group_over_a_global_field}{moderate growth},
        % \item $f$ satisfies \CrefAndHyperrefIfExist{automorphy_for_a_function_on_the_adelic_quotient_group_of_a_linear_algebraic_group_over_a_global_field}{automorphy}, i.e. is left $G(F)$-invariant.
    \end{enumerate}
\end{definition}

\TODO{read the following AI generated statements and make them precise}
\TODO{Formulate how modular forms are automorphic forms}
\begin{theorem}[Basic Properties of Automorphic Forms]
    Let $F$ be a global field, $\A_F$ its ring of adeles, and $G$ a linear algebraic group defined over $F$. Let $K \subseteq G(\A_F)$ be a maximal compact subgroup. Suppose 
    $$
    f : G(F) \backslash G(\A_F) \to \C
    $$
    is an automorphic form, i.e., $f$ is smooth, $K$-finite, and of moderate growth. Then:
    \begin{enumerate}
        \item $f$ is left-invariant under $G(F)$ (automorphy condition).
        \item $f$ generates a $(\mathfrak{g}, K)$-module under the right regular representation of $G(\A_F)$, where $\mathfrak{g}$ is the Lie algebra of $G(F_\infty)$.
        \item The space of automorphic forms is preserved by Hecke operators arising from elements in the spherical Hecke algebra associated with $G(\A_F^{\infty})$.
    \end{enumerate}
\end{theorem}

\begin{proposition}[Equivalent Characterizations of Automorphic Forms]
Under standard assumptions on $G$, the following conditions on a function $f: G(F) \backslash G(\A_F) \to \C$ are equivalent:
\begin{enumerate}
    \item $f$ is smooth, $K$-finite, of moderate growth, and left $G(F)$-invariant.
    \item $f$ corresponds to a vector in a smooth admissible representation of $G(\A_F)$ realized as a subrepresentation of the space of $L^2$ automorphic forms.
\end{enumerate}
\end{proposition}

\begin{corollary}[Automorphic Forms as Sections]
Automorphic forms can be identified with smooth sections of certain automorphic vector bundles over the quotient space $G(F)\backslash G(\A_F)/K$ satisfying growth conditions.
\end{corollary}

\begin{remark}[Intuition]
Automorphic forms generalize classical modular forms by encoding arithmetic and representation-theoretic data through functions on adelic quotients. The conditions of smoothness, $K$-finiteness, and moderate growth control analytic and algebraic properties needed for harmonic analysis and spectral theory on these quotients.
\end{remark}


\appendix

\section{Miscellaneous definitions}

\begin{definition}[Topological groups]  \label{definition:topological_group}
    \TODO{Product topology}
    A \hldef{topological group} is a \CrefAndHyperrefIfExist{definition:group}{group} $(G,\cdot)$ together with a \CrefAndHyperrefIfExist{definition:topological_space}{topology} $\mathcal{T}$ on $G$ such that the maps
    \begin{align*}
    \mu : G \times G &\to G, & (g,h) &\mapsto g \cdot h, \\
    \iota : G &\to G, & g &\mapsto g^{-1},
    \end{align*}
    are \CrefAndHyperrefIfExist{definition:continuous_map_between_open_subsets_of_euclidean_spaces}{continuous} with respect to the product topology on $G \times G$ and the topology $\mathcal{T}$ on $G$.
\end{definition}



\section{Ad\`eles and id\'eles of global fields }

\begin{definition} \label{definition:global_field}
A \hldef{global field} is a field $K$ that is either:
\begin{itemize}
    \item a finite extension of the field of rational numbers $\mathbb{Q}$ (i.e., a \hl{number field}), or
    \item a finite extension of the field of rational functions $\mathbb{F}_q(t)$ in one variable over a finite field $\mathbb{F}_q$ (i.e., a \hl{global function field}).
\end{itemize}
\end{definition}



\begin{definition} \label{definition:completion_of_a_global_field_at_a_place}
    Let $K$ be a \hyperrefIfExists{definition:global_field}{global field}\CrefIfExists{definition:global_field} and let $v$ be a \hyperrefIfExists{definition:place_of_a_global_field}{place}\CrefIfExists{definition:place_of_a_global_field} of $K$. Write $|\cdot|_v$ for an absolute value representing $v$. 
    The \hldef{completion of $K$ at $v$}, often denoted \hl{$K_v$}, is the \hyperrefIfExists{definition:completion_of_an_extended_metric_space}{completion of $K$ with respect to} the \hyperrefIfExists{definition:metric_induced_by_an_absolute_value_on_a_field}{metric induced by $|\cdot|_v$}\CrefIfExists{definition:metric_induced_by_an_absolute_value_on_a_field}.

    % \hyperrefIfExists{definition:extended_metric_induced_by_an_extended_norm_on_a_vector_space_over_a_field_with_absolute_value}{metric induced by $|\cdot|_v$} as a \hyperrefIfExists{definition:extended_norm_on_a_vector_space_over_a_field_with_absolute_value}{norm}\CrefIfExists{definition:extended_norm_on_a_vector_space_over_a_field_with_absolute_value} on the $1$-dimensional $K$-vector space $K$ with absolute value $|\cdot|_v$. 
\end{definition}


\begin{definition}\label{definition:restricted_product_of_a_family_of_topological_spaces_with_respect_to_subspaces}
Let $\{X_i\}_{i \in I}$ be a family of topological spaces indexed by a set $I$. For each $i \in I$, let $K_i \subseteq X_i$ be a topological subspace.

The \hldef{restricted product topology} on the \hyperrefIfExists{definition:restricted_product_of_a_family_of_sets_with_respect_to_a_family_of_distinguished_subsets}{restricted product}
$$ \prod_{i \in I}' X_i := \left\{ (x_i)_{i \in I} \in \prod_{i \in I} X_i \;\middle|\; x_i \in K_i \text{ for all but finitely many } i \in I \right\}, $$
with respect to the subsets $\{K_i\}_{i \in I}$, is the coarsest topology such that:
\begin{itemize}
    \item The natural inclusion maps $X_j \to \prod_{i \in I}' X_i$, defined by $x_j \mapsto (y_i)$ where $y_j = x_j$ and $y_i = k_i$ (a fixed element in $K_i$) for all $i \neq j$, are continuous for all $j \in I$.
    \item The subspace topology on the product $\prod_{i \in F} X_i$ for any finite subset $F \subseteq I$ (where coordinates outside $F$ are fixed in $K_i$) coincides with the product topology on finitely many factors.
\end{itemize}
Equivalently, the restricted product topology is generated by the base consisting of sets of the form
$$ \prod_{i \in F} U_i \times \prod_{i \notin F} K_i, $$
where $F \subseteq I$ is finite, $U_i$ are open sets in $X_i$, and outside $F$ the coordinates lie in $K_i$.
\end{definition}




\begin{definition} \label{definition:adeles_and_ideles_of_a_global_field}
Let $K$ be a \hyperrefIfExists{definition:global_field}{global field}. Write $M_K$ for the set of all \hyperrefIfExists{definition:place_of_a_global_field}{places}\CrefIfExists{definition:place_of_a_global_field} of $K$ and write $M_K^\infty$ for the set of \hyperrefIfExists{definition:place_of_a_global_field}{archimedean places} of $K$. Let $S \subseteq M_K$ be some subset of places of $K$ (typically, $S$ is a finite set). For each $v \in M_K$, write $\calO_v$ for the \CrefAndHyperrefIfExist{definition:ring_of_integers_of_a_global_or_local_ring}{ring of integers} in \CrefAndHyperrefIfExist{definition:completion_of_a_global_field_at_a_place}{the completion $K_v$}\CrefIfExists{theorem:completion_of_a_global_field_at_a_place_is_a_local_field}
\begin{itemize}
    \item The \hldef{adèle ring of $K$}, denoted \hl{$\mathbb{A}_K$}, is the \hyperrefIfExists{definition:restricted_product_of_a_family_of_topological_spaces_with_respect_to_subspaces}{restricted direct product} of the $K_v$ (over all places $v$ of $K$), with respect to the $\mathcal{O}_v$ at \hyperrefIfExists{definition:archimedean_absolute_value_on_a_field}{non-archimedean} $v$:
    $$ \hlin{\mathbb{A}_K = \left\{ (x_v)_v \in \prod_{v \in M_K} K_v \;\middle|\; x_v \in \mathcal{O}_v \text{ for all but finitely many non\text{-}archimedean $v$} \right\}.}  $$

    \item The \hldef{idèle group of $K$}, commonly denoted \hl{$\mathbb{A}_K^\times$} or \hl{$\bbI_K$}, is the group of invertible elements of $\mathbb{A}_K$: 
    $$ \hlin{\bbI_K = \mathbb{A}_K^\times = \left\{ (x_v)_v \in \prod_{v \in M_K} K_v^\times \;\middle|\; x_v \in \mathcal{O}_v^\times \text{ for all but finitely many non\text{-}archimedean $v$} \right\},} $$
    where $\mathcal{O}_v^\times$ denotes the group of units of $\mathcal{O}_v$ for non-archimedean $v$.

    \item The \hldef{adèle ring outside $S$ of $K$}, commonly denoted \hl{$\mathbb{A}_K^S$} or \hl{$\mathbb{A}_{K,S}$}, is the restricted product of the completions $K_v$ over all places $v \in M_K \setminus S$, with respect to the rings of integers $\mathcal{O}_v$ at non-archimedean places:
    $$\hlin{\mathbb{A}_{K,S} = \mathbb{A}_K^S = \left\{ (x_v)_v \in \prod_{v \in M_K \setminus S} K_v \;\middle|\; x_v \in \mathcal{O}_v \text{ for all but finitely many non-archimedean } v \right\}.}$$
    \item The \hldef{idèle group outside $S$ of $K$}, commonly denoted \hl{$(\mathbb{A}_K^\times)^S$}, \hl{$(\mathbb{A}_{K,S}^\times)$}, \hl{$\bbI_K^S$}, or \hl{$\bbI_{K,S}$} is the group of invertible elements of $\mathbb{A}_K^S$:
    $$\hlin{(\mathbb{A}_K^\times)^S = \left\{ (x_v)_v \in \prod_{v \in M_K \setminus S} K_v^\times \;\middle|\; x_v \in \mathcal{O}_v^\times \text{ for all but finitely many non-archimedean } v \right\}.}$$

    \item The \hldef{ring of finite adèles of $K$}, commonly denoted \hl{$\mathbb{A}_{K, \mathrm{fin}}$}, \hl{$\mathbb{A}_{K}^{\mathrm{fin}}$}, \hl{$\mathbb{A}_{K, \mathrm{f}}$}, \hl{$\mathbb{A}_{K}^{\mathrm{f}}$}, is the adèle ring outside $S = M_K^\infty$, the set of archimedean places of $K$:
    $$\hlin{\mathbb{A}_{K, \mathrm{fin}} := \mathbb{A}_K^{M_K^\infty} = \left\{ (x_v)_v \in \prod_{v \notin M_K^\infty} K_v \;\middle|\; x_v \in \mathcal{O}_v \text{ for all but finitely many non-archimedean } v \right\}.} $$

    \item The \hldef{finite idèle group of $K$}, commonly denoted \hl{$\mathbb{A}_{K, \mathrm{fin}}^\times$}, \hl{$\mathbb{I}_{K, \mathrm{fin}}$}, \hl{$\mathbb{I}_{K}^{\mathrm{fin}}$}, \hl{$\mathbb{I}_{K, \mathrm{f}}$}, \hl{$\mathbb{I}_{K}^{\mathrm{f}}$} etc.  is the group of units of the ring of finite adèles:
    $$\hlin{\mathbb{A}_{K, \mathrm{fin}}^\times := (\mathbb{A}_K^\times)^{M_K^\infty} = \left\{ (x_v)_v \in \prod_{v \notin M_K^\infty} K_v^\times \;\middle|\; x_v \in \mathcal{O}_v^\times \text{ for all but finitely many non-archimedean } v \right\}.}$$

\end{itemize}
All of these are equipped with the \hyperrefIfExists{definition:restricted_product_of_a_family_of_topological_spaces_with_respect_to_subspaces}{restricted product topology} induced by the topologies of the \hyperrefIfExists{definition:local_field}{local fields}\CrefIfExists{definition:local_field} $K_v$ and the subspace topologies thereof.
\end{definition}



\begin{definition}[Adelic points of an algebraic group] \label{definition:group_of_adelic_points_of_an_algebraic_group_over_a_global_field}
    Let $K$ be a \CrefAndHyperrefIfExist{definition:global_field}{global field} with ring of integers $\calO_K$, and let $G$ be a \CrefAndHyperrefIfExist{definition:linear_algebraic_group_over_a_scheme}{linear algebraic group} defined over $K$.  
    \begin{itemize}
        \item The \hldef{group of adelic points of $G$} is 
        the group \hl{$G(\mathbb{A}_K)$} defined by
        $$G(\mathbb{A}_K) := \prod\nolimits_v' G(K_v),$$
        \CrefIfExists{definition:completion_of_a_global_field_at_a_place}
        where the restricted product is taken with respect to the compact open subgroups $G(\mathcal{O}_v)$ for almost all non-archimedean $v$.
        \item For a set $S$ (usually finite) of places of $K$, the \hl{group of adelic points of $G$ outside $S$} is
        the group \hl{$G(\mathbb{A}_{K,S})$} defined by
        $$G(\mathbb{A}_{K,S}) := \prod\nolimits_{v \notin S}' G(K_v).$$
        \item The \hl{group of finite adelic points} is
        the group \hl{$G(\mathbb{A}_{K,{\mathrm{fin}}}) = G(\mathbb{A}_K^{\mathrm{fin}})$} defined by 
        $$G(\mathbb{A}_K^{\mathrm{fin}}) := \prod\nolimits_{v \ \mathrm{non\text{-}archimedean}}' G(K_v).$$
        \item The \hl{group of archimedean adelic points} is
        the group \hl{$G(K_\infty)$} defined by
        $$G(K_\infty) := \prod\nolimits_{v \ \mathrm{archimedean}} G(K_v).$$
    \end{itemize}
    Thus one has a natural factorization
    $$G(\mathbb{A}_K) \;\cong\; G(\mathbb{A}_K^{\mathrm{fin}}) \times G(K_\infty).$$
    The groups $G(\bbA_K)$, $G(\bbA_{K,S})$, and $G(\bbA_{K}^{\mathrm{fin}})$ are all \CrefAndHyperrefIfExist{definition:topological_group}{topological groups} under the \CrefAndHyperrefIfExist{definition:restricted_product_of_a_family_of_topological_spaces_with_respect_to_subspaces}{restricted product topology} induced by the topologies of the local fields $K_v$ and the subspace topologies of $\calO_v$ and become topological groups.  The group $G(K_\infty)$ is a topological group under the direct product topology.
\end{definition}




\begin{definition}[Height function on \(G(\mathbb{A}_F)\)] \label{definition:height_function_on_the_adelic_points_of_a_linear_algebraic_group_over_a_global_field_with_respect_to_an_embedding}
    Let $F$ be a \CrefAndHyperrefIfExist{definition:global_field}{global field}, and let $\A_F$ be the \CrefAndHyperrefIfExist{definition:adeles_and_ideles_of_a_global_field}{ring of adeles of $F$}. Let $G$ be a \CrefAndHyperrefIfExist{definition:linear_algebraic_group_over_a_scheme}{linear algebraic group} defined over $F$ embedded as an algebraic subgroup in \(\mathrm{GL}_n\).

    For each place \(v\) of \(F\), let \(\|\cdot\|_v\) be a norm on \(G(F_v) \subseteq \mathrm{GL}_n(F_v)\) defined as follows:
    \[
    \|g_v\|_v := \max_{1 \leq i,j \leq n} |(g_v)_{ij}|_v,
    \]
    \TODO{define the usual absolute value on $F_v$}
    where \(|\cdot|_v\) is the usual absolute value or valuation on \(F_v\).

    Then, for \(g = (g_v)_{v} \in G(\mathbb{A}_F)\), define the adelic height function by the product
    \[
    \|g\| := \prod_{v} \|g_v\|_v,
    \]
    where the product converges due to \(\|g_v\|_v = 1\) for all but finitely many non-archimedean places.
\end{definition}



\begin{definition}[Linear algebraic group over a scheme] \label{definition:linear_algebraic_group_over_a_scheme}
Let \(S\) be a base scheme.  
A \hldef{linear algebraic group over $S$} is an \CrefAndHyperrefIfExist{definition:algebraic_group_scheme_over_a_scheme}{affine group scheme} \(G\) over \(S\) that is finitely presented and smooth over \(S\),  
and such that for some integer \(n \geq 1\), there exists a closed immersion of \(S\)-group schemes
$$ G \hookrightarrow \operatorname{GL}_{n, S}.$$
\CrefIfExists{definition:general_linear_group_over_a_scheme}
\end{definition}
