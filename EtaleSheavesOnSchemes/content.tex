
\section{Definitions}

\subsection{\'Etale morphisms of schemes}

% \import{../_notations/}{notation_Sch_S_as_the_category_of_schemes_over_a_scheme.tex}
\begin{notation} \label{notation:Sch_S_as_the_category_of_schemes_over_a_scheme}
Let $S$ be a scheme. Let \hl{$\Sch/S$} denote the category of \CrefAndHyperrefIfExist{definition:scheme_over_a_scheme}{schemes over $S$}.
\end{notation}


\begin{definition}[Locally of finite presentation morphism of schemes] \label{definition:locally_of_finite_presentation_finite_presentation_morphism_of_schemes}
Let $f : X \to Y$ be a morphism of schemes. 
\begin{enumerate}
    \item We say that $f$ is \hldef{locally of finite presentation} if for every \CrefAndHyperrefIfExist{definition:affine_open_subscheme_of_a_scheme}{affine open subset} \CrefAndHyperrefIfExist{definition:affine_morphism_of_schemes}{$\mathrm{Spec}(B) \subseteq Y$}, and every affine open subset $\mathrm{Spec}(A) \subseteq f^{-1}(\mathrm{Spec}(B))$\CrefIfExists{definition:preimage_of_an_open_subset_of_a_scheme_under_a_morphism_of_schemes}, the induced ring homomorphism $B \to A$ presents $A$ as a $B$-algebra of \CrefAndHyperrefIfExist{definition:finitely_presented_algebra_over_a_not_necessarily_commutative_ring}{finite presentation}; that is, $A$ is isomorphic to a quotient of a polynomial ring in finitely many variables over $B$ by a finitely generated ideal:
    $$A \cong B[x_1, \ldots, x_n] / (f_1, \ldots, f_m),$$
    for some finite $n,m$.

    \item A morphism of schemes $f : X \to Y$ is of \hldef{finite presentation} if it is locally of finite presentation and \CrefAndHyperrefIfExist{definition:quasi_compact_morphism_of_schemes}{quasi-compact} and \CrefAndHyperrefIfExist{definition:quasi_separated_morphism_of_schemes}{quasi-separated}; in particular, $f$ can be covered by finitely many affine opens satisfying the finite presentation condition above.
\end{enumerate}
\end{definition}
\begin{definition}[Flat module over a ring] \label{definition:flat_left_right_module_over_a_ring}
    Let $R$ be a \CrefAndHyperrefIfExist{definition:ring}{(not necessarily commutative) ring}. 
    \begin{enumerate}
        \item Let $M$ be a left $R$-module. The module $M$ is said to be \hldef{flat (with respect to the left $R$-module structure)} if the functor
        \[
        -\otimes_R M : \mathrm{Mod}_R \to \mathbf{Ab}
        \]
        \CrefIfExists{definition:tensor_product_of_bimodules_of_rings} from the category of right $R$-modules to abelian groups is exact; that is, for every exact sequence of right $R$-modules
        \[
        0 \to N' \to N \to N'' \to 0,
        \]
        the induced sequence
        \[
        0 \to N' \otimes_R M \to N \otimes_R M \to N'' \otimes_R M \to 0
        \]
        is exact.


        \TODO{tor}
        Equivalently, $M$ is flat if $\mathrm{Tor}_1^R(-,M) = 0$.

        \item Let $M$ be a right $R$-module. The module $M$ is said to be \hldef{flat (with respect to the right $R$-module structure)} if the functor

        \[
        M \otimes_R - : {}_R \mathrm{Mod} \to \mathbf{Ab}
        \]
        from the category of left $R$-modules to abelian groups is exact; that is, for every exact sequence of right $R$-modules
        \[
        0 \to N' \to N \to N'' \to 0,
        \]
        the induced sequence
        \[
        0 \to M \otimes_R N'  \to M \otimes_R N \to M \otimes_R N'' \to 0
        \]
        is exact.

    \end{enumerate}
    \TextIfExists{definition:flat_object_in_an_abelian_category_with_respect_to_a_right_exact_monoidal_product_functor}{Equivalently in either case, $M$ is flat if it is \CrefAndHyperrefIfExist{definition:flat_object_in_an_abelian_category_with_respect_to_a_right_exact_monoidal_product_functor}{flat with respect to the biadditive functor $$- \otimes_R -: \mathbf{Mod}_R \times {}_R \mathbf{Mod} \to \Ab.$$}}
\end{definition}
\begin{definition}[Flat morphism of schemes] \label{definition:flat_morphism_of_schemes}
    Let $f : X \to Y$ be a \CrefAndHyperrefIfExist{definition:morphism_of_schemes}{morphism of schemes}.
    
    \begin{enumerate}
        \item Let $x \in X$ be a point and let $y = f(x)$. We say that $f$ is \hldef{flat at $x$} if the \CrefAndHyperrefIfExist{definition:induced_ring_homomorphism_of_stalks_for_a_morphism_of_ringed_spaces}{induced ring homomorphism} on \CrefAndHyperrefIfExist{definition:local_ring}{local rings}
        \[
        \mathcal{O}_{Y,y} \to \mathcal{O}_{X,x}
        \]
        makes \CrefAndHyperrefIfExist{definition:stalk_of_a_presheaf_on_a_topological_space_at_a_point}{$\mathcal{O}_{X,x}$} into a \CrefAndHyperrefIfExist{definition:flat_left_right_module_over_a_ring}{flat} $\mathcal{O}_{Y,y}$-module. 
        \item We say $f$ is \hldef{flat} if it is flat at every point $x \in X$.
        \item $f$ is \hldef{faithfully flat} if it is flat and \CrefAndHyperrefIfExist{definition:injective_surjective_bijective_map_of_sets}{surjective}.
    \end{enumerate}
\end{definition}
\begin{definition}[Unramified morphism of schemes] \label{definition:unramified_morphism_of_schemes}
    \TODO{ sheaf of relative differentials}
    A \CrefAndHyperrefIfExist{definition:morphism_of_schemes}{morphism of schemes} $f : X \to Y$ is \hldef{unramified} if it is \CrefAndHyperrefIfExist{definition:finite_type_morphism_of_schemes}{locally of finite type} and the sheaf of relative differentials $\Omega_{X/Y}$ is zero. Equivalently:
    \begin{itemize}
        \item For every $x \in X$, the induced ring map on stalks $\mathcal{O}_{Y,f(x)} \to \mathcal{O}_{X,x}$ is of \CrefAndHyperrefIfExist{definition:finitely_generated_algebra_over_a_not_necessarily_commutative_ring}{finite type},
        \item and the module of Kähler differentials $\Omega_{\mathcal{O}_{X,x}/ \mathcal{O}_{Y,f(x)}}$ is $0$.
    \end{itemize}
\end{definition}
\begin{definition} \label{definition:etale_morphism_of_schemes}
A \CrefAndHyperrefIfExist{definition:morphism_of_schemes}{morphism of schemes} $f : X \to Y$ is called \hldef{\'etale} if it satisfies the following conditions:
\TODO{sheaf of relative differentials}
\begin{itemize}
  \item $f$ is \CrefAndHyperrefIfExist{definition:locally_of_finite_presentation_finite_presentation_morphism_of_schemes}{locally of finite presentation},
  \item $f$ is \CrefAndHyperrefIfExist{definition:flat_morphism_of_schemes}{flat},
  \item $f$ is \CrefAndHyperrefIfExist{definition:unramified_morphism_of_schemes}{unramified}, i.e., the sheaf of relative differentials $\Omega_{X/Y}$ equals $0$.
\end{itemize}
\TODO{relative dimension}
Equivalently, a morphism of schemes is \'etale if and only if it is \CrefAndHyperrefIfExist{definition:smooth_morphism_of_schemes}{smooth} of relative dimension $0$.
A \CrefAndHyperrefIfExist{definition:finite_morphism_of_schemes}{finite} \'etale morphism is synonymously called a \hldef{finite \'etale cover}.
\end{definition}

\begin{definition} \label{definition:automorphism_group_of_a_scheme_over_a_scheme}
Let $S$ be a scheme and let $f : X \to S$ be a morphism of schemes over $S$. The \hldef{automorphism group of $X$ over $S$}, denoted \hl{$\operatorname{Aut}(X/S) = \operatorname{Aut}_S(X)$}, is the group of all \CrefAndHyperrefIfExist{definition:scheme_over_a_scheme}{$S$-scheme} automorphisms of $X$, i.e., all isomorphisms of schemes $\varphi : X \to X$ such that the diagram
$$
\begin{tikzcd}
X \arrow[rd, "f"] \arrow[rr, "\varphi", "\sim"'] & & X \arrow[ld, "f"'] \\
 & S & 
\end{tikzcd}
$$
commutes. The group operation is composition of morphisms.
\end{definition}



\begin{definition} \label{definition:finite_morphism_of_schemes}
Let $f : X \to Y$ be an \CrefAndHyperrefIfExist{definition:affine_morphism_of_schemes}{affine morphism of schemes}. We say that $f$ is a \hldef{finite morphism} if for every \CrefAndHyperrefIfExist{definition:affine_open_subscheme_of_a_scheme}{affine open} $V = \operatorname{Spec} B \subseteq Y$ with $U = f^{-1}(V) = \operatorname{Spec} A$, the ring $A$ is a \CrefAndHyperrefIfExist{definition:finite_algebra_over_a_ring}{finite $B$-algebra}.
\end{definition}


\begin{definition} \label{definition:galois_morphism_of_schemes}
Let $S$ be a scheme. Let $f: X \to Y$ over $S$ be a \CrefAndHyperrefIfExist{definition:finite_morphism_of_schemes}{finite morphism}. It is called \hldef{Galois} (or synonymously a \hldef{Galois cover/Galois covering}) if there exists a finite group $G$ acting on $X$ over $Y$ such that
\begin{itemize}
  \item $f$ is identified with the quotient map $X \to X/G$. 
  \item the group action realizes $Y$ as the categorical quotient under the group action.
\end{itemize}
If $f$ is Galois, then the finite group $G$ is isomorphic to the \hyperrefIfExists{definition:automorphism_group_of_a_scheme_over_a_scheme}{automorphism group $\Aut_Y(X)$}. By a \hldef{Galois $G$-cover/morphism}, we mean a Galois morphism whose automorphism group is (isomorphic to) $G$, often equipped with a fixed isomorphism $\phi: G \to \Aut_Y(X)$. 

% If $f$ is Galois and \hyperrefIfExists{definition:etale_morphism_of_schemes}{\'etale}, then we say that $f$ is a \hldef{finite Galois covering} or that $X$ is a \hldef{finite Galois cover of $Y$}.

In case $f$ is an \'etale morphism, then $f$ is equivalently Galois if there exists a finite group $G$ acting on $X$ over $Y$ such that 
\begin{itemize}
    \item $f$ is identified with the quotient map $X \to X/G$. 
    \item $G$ acts simply transitively on geometric fibres.
\end{itemize}
% Equivalently, $f$ is a Galois covering with Galois group $G$.
\end{definition}




\section{The \'etale fundamental group of a scheme at a geometric point}

% \begin{notation}
% Let $S$ be a connected, locally noetherian scheme. Let $\Sch/S$ denote the category of schemes over $S$ with morphisms over $S$. For an $S$-scheme $X$, denote by $\pi_1^{et}(X, \bar{x})$ the \emph{\'etale fundamental group} of $X$ based at a geometric point $\bar{x} : \Spec(\Omega) \to X$, where $\Omega$ is a separably closed field.
% \end{notation}



\begin{definition} \label{definition:geometric_point_of_a_scheme}
Let $S$ be a scheme. A \hldef{geometric point of $S$} is a morphism $\bar{s} : \Spec(\Omega) \to S$ where $\Omega$ is an algebraically closed field. 
\end{definition}

\begin{definition} \label{definition:etale_fundamental_group_of_a_connected_scheme}
    \TODO{TODO: define profinite topology}
Let $S$ be a connected scheme. The \hldef{\'etale fundamental group of $S$ at a geometric point $\bars : \Spec(\Omega) \to S$} is the profinite group
$$\hlin{\pioneet(S, \bars) := \varprojlim_{X \to S} \mathrm{Aut}_S(X),}$$
\CrefIfExists{definition:automorphism_group_of_a_scheme_over_a_scheme} where the inverse limit is taken over all finite \hyperrefIfExists{definition:etale_morphism_of_schemes}{\'etale} covers $X \to S$ pointed above $\bars$. Equivalently, one may take the limit over all finite \'etale \hyperrefIfExists{definition:galois_morphism_of_schemes}{Galois covers} $X \to S$ pointed above $\bars$. The \'etale fundamental group is equipped with the profinite topology.
\end{definition}

\begin{theorem} \label{theorem:finite_etale_covers_of_a_connected_scheme_correspond_to_finite_sets_with_an_action_of_the_etale_fundamental_group}
Let $S$ be a connected scheme with a chosen \hyperrefIfExists{definition:geometric_point_of_a_scheme}{geometric point} $\bar{s}$. 
\begin{enumerate}
    \item The functor
    $$ X \mapsto \mathrm{Hom}_S(\bars, X).$$
    is an equivalence of categories between the category of \hyperrefIfExists{definition:etale_morphism_of_schemes}{finite \'etale covers} of $S$ and the category of finite sets (equipped with the discrete topology) with a continuous (left) action of the \hyperrefIfExists{definition:etale_fundamental_group_of_a_connected_scheme}{\'etale fundamental group $\pi_1^{\mathrm{et}}(S, \bars)$}\CrefIfExists{definition:etale_fundamental_group_of_a_connected_scheme}.
    We also note that the discrete set $\mathrm{Hom}_S(\bars, X)$ is identifiable with the fiber $X \times_S \bars$.

    \item There is a natural bijection between isomorphism classes of finite étale \hyperrefIfExists{definition:galois_morphism_of_schemes}{Galois covers}\CrefIfExists{definition:galois_morphism_of_schemes} $f : X \to S$ with Galois group isomorphic to $G$ and continuous group homomorphisms
    \begin{align}
    \varphi : \pi_1^{\mathrm{et}}(S, \bars) \to G \label{eq:group_homomorphism_from_etale_fundamental_group_of_a_connected_scheme_to_a_group}
    \end{align}
    up to conjugation in $G$. More explicitly, two homomorphism $\varphi, \varphi'$ correspond to equivalent finite \'etale Galois $G$-covers of $S$ if and only if there exists some $g \in G$ such that 
    $$\varphi'(\gamma) = g\varphi(\gamma) g^{-1}$$
    for all $\gamma \in \pioneet(S, \bars)$.

    Concretely, given a finite \'etale Galois cover $f: X \to S$, the fiber $f^{-1}(\bars) \cong X \times_S \bars$ has an action of $\pioneet(S, \bars)$. Since $G$ acts simply transitively on this fiber, there is an induced group homomorphism \eqref{eq:group_homomorphism_from_etale_fundamental_group_of_a_connected_scheme_to_a_group}; the conjugacy class of this group homomorphism corresponds to $f$.

    Furthemore, the covering scheme $X$ is connected if and only if the homomorphism $\varphi$ is surjective.
\end{enumerate}
\end{theorem}


\section{Sheaves on the small \'etale site of a scheme}

We define presheaves generally.

\begin{definition}[Presheaf on a category] \label{definition:presheaf_on_a_category}
    Let $C$ and $\mathcal{A}$ be \hyperrefIfExists{definition:category}{(large) categories}\CrefIfExists{definition:category}. 
    \begin{enumerate}
        \item A \hldef{presheaf $\mathcal{F}$ on $C$ with values in $\mathcal{A}$} is a functor
        \[
        \mathcal{F}: C^{\mathrm{op}} \to \mathcal{A}.
        \]
        In other words, a presheaf $\calF$ on $C$ with values in $\calA$ is simply a \CrefAndHyperrefIfExist{definition:functor_between_categories}{contravariant functor} from $C$ to $\calA$. 
        Explicitly, for every object $U$ in $C$, one has an object $\mathcal{F}(U)$ in $\mathcal{A}$ (called the \hldef{$U$-valued sections/sections evaluated at $U$ of $\calF$}\TextIfExists{definition:sections_of_a_presheaf_on_a_category_valued_in_a_category}{, cf. \Cref{definition:sections_of_a_presheaf_on_a_category_valued_in_a_category}}), and for every morphism $f: V \to U$ in $C$, one has a morphism (called the \hldef{restriction map})
        \[
        \mathcal{F}(f): \mathcal{F}(U) \to \mathcal{F}(V)
        \]
        in $\mathcal{A}$, such that for all composable morphisms $W \xrightarrow{g} V \xrightarrow{f} U$ in $C$, the following diagram in $\mathcal{A}$ commutes:
        \[
        \begin{tikzcd}
        \mathcal{F}(U) \arrow[r, "\mathcal{F}(f)"] \arrow[rr, bend left, "\mathcal{F}(f \circ g)"] & \mathcal{F}(V) \arrow[r, "\mathcal{F}(g)"] & \mathcal{F}(W)
        \end{tikzcd}
        \]
        That is,
        \[
        \mathcal{F}(g) \circ \mathcal{F}(f) = \mathcal{F}(f \circ g),
        \]
        and for every object $U$ in $C$, $\mathcal{F}(\mathrm{id}_U) = \mathrm{id}_{\mathcal{F}(U)}$.


        \item 
        Let $\mathcal{F},\mathcal{G}: C^{\mathrm{op}} \to \mathcal{A}$ be two presheaves on $C$ with values in $\mathcal{A}$. A \hldef{morphism of presheaves}
        \[
        \varphi: \mathcal{F} \to \mathcal{G}
        \]
        is a \hyperrefIfExists{definition:natural_transformation_between_functors_between_categories}{natural transformation of functors}\CrefIfExists{definition:natural_transformation_between_functors_between_categories}: for each object $U$ of $C$, one has a morphism
        \[
        \varphi_U: \mathcal{F}(U) \to \mathcal{G}(U)
        \]
        in $\mathcal{A}$, such that for every morphism $f: V \to U$ in $C$, the diagram
        \[
        \begin{tikzcd}
        \mathcal{F}(U) \arrow[r, "\mathcal{F}(f)"] \arrow[d, "\varphi_U"'] & \mathcal{F}(V) \arrow[d, "\varphi_V"] \\
        \mathcal{G}(U) \arrow[r, "\mathcal{G}(f)"'] & \mathcal{G}(V)
        \end{tikzcd}
        \]
        commutes, i.e.,
        \[
        \varphi_V \circ \mathcal{F}(f) = \mathcal{G}(f) \circ \varphi_U
        \]
        for all objects and morphisms in $C$.

        \item Given a \hyperrefIfExists{definition:grothendieck_universe}{universe}\CrefIfExists{definition:grothendieck_universe} $U$, a \hldef{$U$-presheaf on $\calC$} typically refers to a presheaf of $U$-sets on $C$.

        \item The \hldef{presheaf category/category of $\calA$-valued presheaves on $\calC$} is the (large) category whose objects are the presheaves on $C$ with values in $\calA$ and whose morphisms are the presheaf morphisms. Common notations for the presheaf category include, but are not limited to: \hl{$\calA^{\calC^{\op}}$}, \hl{$\PreShv(\calC, \calA)$}, \hl{$[\calC^{\op}, \calA]$}. If the value category $\calA$ is clear from context, then notations such as \hl{$\PreShv(\calC)$} are also common. \TextIfExists{definition:diagram_in_a_category_indexed_by_a_small_category}{Note that the presheaf category $\PreShv(\calC, \calA)$ is equivalent to the \CrefAndHyperrefIfExist{definition:diagram_in_a_category_indexed_by_a_small_category}{category of functors} $\calC^{\op} \to \calA$ and hence notations for the functor categories are applicable as notations for presheaf categories.}

    \end{enumerate}
\end{definition}


We can speak of sheaves on a general site.

We will be interested in sheaves of sets/abelian groups/$R$-modules on the \CrefAndHyperrefIfExist{definition:small_etale_site_of_a_scheme}{small \'etale site $X_{\et}$} of any scheme as $X_{\et}$ is \CrefAndHyperrefIfExist{definition:essentially_small_category}{essentially small}.

\begin{definition}[Sheaf on a site] \label{definition:sheaf_on_a_site}

% \TODO{There might be some need to say that $\calA$ is a category for which sheaves on the site ``can be defined''}
% \TODO{go through statements using the notion of sheaves and make sure that the value categories have small products and that the categories have small generating families.}

Let $(\calC, J)$ be a \CrefAndHyperrefIfExist{definition:grothendieck_topology_on_a_category_site_covering_sieve_topologically_generating_family}{site}. Let $\calA$ be a \CrefAndHyperrefIfExist{definition:category}{(large) category}.
\begin{enumerate}
    \item A \CrefAndHyperrefIfExist{definition:presheaf_on_a_category}{presheaf} $\calF: \calC^{\op} \to \calA$\CrefIfExists{definition:opposite_category_of_a_category} is called a \hldef{sheaf on the site $(\calC, J)$ valued in $\calA$} if, for every object $U$ of $\calC$ and every \CrefAndHyperrefIfExist{definition:grothendieck_topology_on_a_category_site_covering_sieve_topologically_generating_family}{covering sieve} $S \in J(U)$, the \CrefAndHyperrefIfExist{definition:limit_and_colimit_of_a_diagram_in_a_category}{limit}
    $$\varprojlim_{(V \to U) \in (\calD_S)^{\op}} \calF|_{\calD_S}(V),$$
    exists and the canonical natural morphism
    $$\calF(U) \to \varprojlim_{(V \to U) \in (\calD_S)^{\op}} \calF|_{\calD_S}(V)$$
    is an isomorphism. Here, $\calD_S \hookrightarrow \calC/U$\CrefIfExists{definition:category_of_objects_over_under_a_fixed_object_in_a_category} is the full \CrefAndHyperrefIfExist{definition:downward_upward_closed_subcategory_of_a_category}{downward-closed subcategory} such that $\operatorname{Ob}(\calD_S) = \{(f: V \to U): f \in S(V)\}$,

    In particular, when we are working with a \CrefAndHyperref{definition:basis_and_grothendieck_pretopology_for_a_grothendieck_topology_on_a_category}{Grothendieck pretopology} $K$ on a category $\calC$, we may speak of sheaves on the site whose Grothendieck topology is the \CrefAndHyperref{definition:grothendieck_topology_generated_by_a_pretopology}{one generated by} $K$.

    \item Given sheaves $\calF, \calG: \calC^{\op} \to \calA$ on the site $(\calC, J)$, a \hldef{morphism between the sheaves} is a \CrefAndHyperrefIfExist{definition:presheaf_on_a_category}{morphism} between $\calF$ and $\calG$ as presheaves.


    \item Let $U$ be a \hyperrefIfExists{definition:grothendieck_universe}{universe}\CrefIfExists{definition:grothendieck_universe}. A \hldef{$U$-sheaf} typically refers to a $U$-presheaf that is a sheaf for a $U$-site. In other words, a $U$-sheaf is a sheaf on a site whose underlying category is \hyperrefIfExists{definition:locally_small_category}{$U$-locally small}\CrefIfExists{definition:locally_small_category} and which has a $U$-small topologically generating family such that the sheaf is valued in $U$-sets.

    \item The \hldef{sheaf category/category of $\calA$-valued sheaves on $\calC$} is the (large) category defined as the full subcategory of $\PreShv(\calC, \calA)$ whose objects are the sheaves on $\calC$ with values in $\calA$. Common notations for the sheaf category include \hl{$\Shv(\calC, \calA)$}, \hl{$\Shv(\calC, J, \calA)$}, \hl{$\Sh(\calC, \calA)$}, \hl{$\Sh(\calC, J, \calA)$}. If the value category $\calA$ is clear from context, then notations such as \hl{$\Shv(\calC)$}, \hl{$\Shv(\calC, J)$}, \hl{$\Sh(\calC)$}, \hl{$\Sh(\calC, J)$} are also common.

\end{enumerate}

% Let $(\calC, J)$ be a \CrefAndHyperrefIfExist{definition:grothendieck_topology_on_a_category_site_covering_sieve_topologically_generating_family}{site} with a small \CrefAndHyperrefIfExist{definition:grothendieck_topology_on_a_category_site_covering_sieve_topologically_generating_family}{topological generating family} (or a $U$-small topologically generating family if a \CrefAndHyperrefIfExist{definition:grothendieck_universe}{universe} $U$ is available) and let $\mathcal{A}$ be a \CrefAndHyperrefIfExist{definition:category}{(large) category} that has all \CrefAndHyperrefIfExist{definition:locally_small_category}{small} \CrefAndHyperrefIfExist{definition:product_and_coproduct_of_objects_in_a_category}{products} (Some common examples of categories that have small products and thus play the role of $\calA$ here include $\mathcal{A} = \text{Set}$, $\text{Ab}$, $R\mathbf{-mod}$ for a fixed ring $R$, $\text{rings}$). 
% \begin{enumerate}

%     \item For any object $U$ of $\calC$ and every covering $\{U_i \to U\}_{i \in I}$ in $J$, note that there are morphisms $U_i \times_U U_j \to U_i$ for every $i,j \in I$. 
%     % Consider the subcategory of $C$ consisting of the objects $U_i$ and $U_i \times_U U_j$, together with these morphisms.
%     Given any presheaf $\calF: C^{\op} \to \calA$, there is a \CrefAndHyperrefIfExist{definition:diagram_in_a_category_indexed_by_a_small_category}{diagram} in $\calA$ consisting of objects $\calF(U_i)$ and $\calF(U_i \times_U U_j)$ and morphisms $\calF(U_i) \to \calF(U_i \times_U U_j)$. The presheaf $\calF$ is called a \hldef{sheaf on the site $(\calC, J)$ valued in $\calA$} if, for every object $U$ of $\calC$ and every covering $\{U_i \to U\}_{i \in I}$ in $J$, the sections object $\calF(U)$ is the \CrefAndHyperrefIfExist{definition:limit_and_colimit_of_a_diagram_in_a_category}{limit} of the aforementioned diagram:
    
%     % A \hyperrefIfExists{definition:presheaf_on_a_category}{presheaf}\CrefIfExists{definition:presheaf_on_a_category} $\mathcal{F}: C^{\mathrm{op}} \to \mathcal{A}$ is a \hldef{sheaf on the site $(\calC,J)$ valued in $\calA$} if, for every object $U$ of $\calC$ and every covering $\{U_i \to U\}_{i \in I}$ in $J$, the sections object $\calF(U)$ is the \CrefAndHyperrefIfExist{definition:limit_and_colimit_of_a_diagram_in_a_category}{limit} of the sections objects $\calF(U_i)$:
%     % $$\calF(U) \cong \varprojlim_{}$$
    
%     % following sequence is an \CrefAndHyperrefIfExist{definition:equalizer_and_coequalizer_of_morphisms_in_a_category}{equalizer} in $\mathcal{A}$:
%     % \[
%     % \mathcal{F}(U) \to \prod_{i} \mathcal{F}(U_i) \rightrightarrows \prod_{i, j} \mathcal{F}(U_i \times_U U_j)
%     % \]
%     % where the first map sends $s$ to $(\mathcal{F}(U_i \to U)(s))_i$ and the arrows to $(\mathcal{F}(U_i \times_U U_j \to U_i)(s_i))_{i,j}$ and $(\mathcal{F}(U_i \times_U U_j \to U_j)(s_j))_{i,j}$, respectively.

%     % \item A \hyperrefIfExists{definition:presheaf_on_a_category}{presheaf}\CrefIfExists{definition:presheaf_on_a_category} $\mathcal{F}: C^{\mathrm{op}} \to \mathcal{A}$ is a \hldef{sheaf on the site $(\calC,J)$ valued in $\calA$} if, for every object $U$ of $\calC$ and every covering $\{U_i \to U\}_{i \in I}$ in $J$, the following sequence is an \CrefAndHyperrefIfExist{definition:equalizer_and_coequalizer_of_morphisms_in_a_category}{equalizer} in $\mathcal{A}$:
%     % \[
%     % \mathcal{F}(U) \to \prod_{i} \mathcal{F}(U_i) \rightrightarrows \prod_{i, j} \mathcal{F}(U_i \times_U U_j)
%     % \]
%     % where the first map sends $s$ to $(\mathcal{F}(U_i \to U)(s))_i$ and the arrows to $(\mathcal{F}(U_i \times_U U_j \to U_i)(s_i))_{i,j}$ and $(\mathcal{F}(U_i \times_U U_j \to U_j)(s_j))_{i,j}$, respectively.

%     \item A \hldef{morphism of sheaves} $\calF: \calC^{\op} \to \calA$ is a \hyperrefIfExists{definition:presheaf_on_a_category}{morphism as presheaves}\CrefIfExists{definition:presheaf_on_a_category}. 


%     \item Let $U$ be a \hyperrefIfExists{definition:grothendieck_universe}{universe}\CrefIfExists{definition:grothendieck_universe}. A \hldef{$U$-sheaf} typically refers to a $U$-presheaf that is a sheaf for a $U$-site. In other words, a $U$-sheaf is a sheaf on a site whose underlying category is \hyperrefIfExists{definition:locally_small_category}{$U$-locally small}\CrefIfExists{definition:locally_small_category} and which has a $U$-small topologically generating family such that the sheaf is valued in $U$-sets.

%     \item The \hldef{sheaf category/category of $\calA$-valued sheaves on $\calC$} is the (large) category defined as the full subcategory of $\PreShv(\calC, \calA)$ whose objects are the sheaves on $C$ with values in $\calA$. Common notations for the sheaf category include \hl{$\Shv(\calC, \calA)$}, \hl{$\Shv(\calC, J, \calA)$}, \hl{$\Sh(\calC, \calA)$}, \hl{$\Sh(\calC, J, \calA)$}. If the value category $\calA$ is clear from context, then notations such as \hl{$\Shv(\calC)$}, \hl{$\Shv(\calC, J)$}, \hl{$\Sh(\calC)$}, \hl{$\Sh(\calC, J)$} are also common.

% \end{enumerate}
\end{definition}



\subsection{The small and big \'Etale sites of a scheme}
\begin{definition}[e.g. see {\cite[II $\S$1]{milne_ec}}]\label{definition:big_site_on_the_category_of_schemes_over_a_scheme_and_small_site}
    Let $S$ be a \CrefAndHyperrefIfExist{definition:scheme}{scheme}. Let $E$ be some class of morphisms in \CrefAndHyperrefIfExist{definition:scheme_over_a_scheme}{$\Sch/S$} satisfying the following:
    \begin{itemize}
        \item All isomorphisms are in $E$.
        \item $E$ is closed under compositions.
        \item Any \CrefAndHyperrefIfExist{definition:base_change_of_a_morphism_in_a_category_by_a_morphism}{base change} of a morphism in $E$ is in $E$.
    \end{itemize}
    \begin{enumerate}
        \item Let $C/S$ be some full subcategory of $\Sch/S$ that is closed under fiber products such that for any $X \to S$ in $C/S$ and any $E$-morphism $U \to X$, the composite $U \to S$ is in $C/S$. An \hldef{$E$-covering of an object $X$ of $C/S$} is a family $(U_i \xrightarrow{g_i} X)_{i \in I}$ of $E$-morphisms such that $Y = \bigcup_i g_i(X_i)$. The class of all such coverings of all such objects is the \hldef{$E$-topology on $C/S$}; it is a \CrefAndHyperrefIfExist{definition:basis_and_grothendieck_pretopology_for_a_grothendieck_topology_on_a_category}{Grothendieck pretopology}. The category $C/S$ equipped with the $E$-topology (more precisely, the \CrefAndHyperrefIfExist{definition:grothendieck_topology_on_a_category_site_covering_sieve_topologically_generating_family}{Grothendieck topology} \CrefAndHyperrefIfExist{definition:grothendieck_topology_generated_by_a_pretopology}{generated by} the $E$-topology) is the \hldef{$E$-site over $S$}.

        \item Assuming that all morphisms in $E$ are locally of finite-type, the \hldef{big $E$-site on $S$} is the \CrefAndHyperrefIfExist{definition:grothendieck_topology_on_a_category_site_covering_sieve_topologically_generating_family}{site} whose underlying category is the category of \CrefAndHyperrefIfExist{definition:finite_type_morphism_of_schemes}{locally of finite type schemes} over $S$ and whose \CrefAndHyperrefIfExist{definition:grothendieck_topology_on_a_category_site_covering_sieve_topologically_generating_family}{Grothendieck topology} is the one \CrefAndHyperrefIfExist{definition:grothendieck_topology_generated_by_a_pretopology}{generated by} the \CrefAndHyperrefIfExist{definition:basis_and_grothendieck_pretopology_for_a_grothendieck_topology_on_a_category}{pretopology} whose coverings are the $E$-coverings. 

        \item The \hldef{small $E$-site on $S$} is the site whose underlying category is the full subcategory of $\Sch/S$ of $S$-schemes whose structure morphisms are $E$-morphisms and whose coverings are $E$-coverings.
    \end{enumerate}
\end{definition}


\begin{definition}[Big étale site of a scheme] \label{definition:big_etale_site_of_a_scheme}
Let $S$ be a fixed scheme. The \hldef{big étale site on $S$}, denoted by \hl{$(\Sch/S)_{\et}$}, is defined as the following \CrefAndHyperrefIfExist{definition:grothendieck_topology_on_a_category_site_covering_sieve_topologically_generating_family}{site}:
\begin{itemize}
    \item The underlying category is the category \hyperrefIfExists{notation:Sch_S_as_the_category_of_schemes_over_a_scheme}{$\Sch/S$} of all schemes over $S$\CrefIfExists{notation:Sch_S_as_the_category_of_schemes_over_a_scheme}. That is, objects are morphisms of schemes $X \to S$, and morphisms are $S$-morphisms between such $X$.
    \item The \CrefAndHyperrefIfExist{definition:grothendieck_topology_on_a_category_site_covering_sieve_topologically_generating_family}{Grothendieck topology} is the one \CrefAndHyperrefIfExist{definition:grothendieck_topology_generated_by_a_pretopology}{generated by} the \CrefAndHyperrefIfExist{definition:basis_and_grothendieck_pretopology_for_a_grothendieck_topology_on_a_category}{pretopology} whose covering families are families $\{f_i : X_i \to X\}_{i \in I}$ in $\Sch/S$ such that each $f_i$ is an \CrefAndHyperrefIfExist{definition:etale_morphism_of_schemes}{\'etale morphism} and the family is jointly surjective on the underlying topological spaces. Such a cover is called an \hldef{\'etale cover of $X$}.
\end{itemize}
\TextIfExists{definition:big_site_on_the_category_of_schemes_over_a_scheme_and_small_site}{Equivalently, the big \'etale site on $S$ is the \CrefAndHyperrefIfExist{definition:big_site_on_the_category_of_schemes_over_a_scheme_and_small_site}{big site for \'etale morphisms on $S$}.}
\end{definition}

\begin{definition}[Small étale site of a scheme] \label{definition:small_etale_site_of_a_scheme}
Let $X$ be a fixed scheme. The \hldef{small étale site on $X$}, commonly denoted by notations including \hl{$X_{\et}$}, \hl{$X_{\mathrm{\acute{e}tale}}$}, or \hl{$\mathrm{Et}_{/X}$}, is defined as the following \hyperrefIfExists{definition:grothendieck_topology_on_a_category_site_covering_sieve_topologically_generating_family}{site}\CrefIfExists{definition:grothendieck_topology_on_a_category_site_covering_sieve_topologically_generating_family}:
\begin{itemize}
    \item The underlying category is the full subcategory of the \hyperrefIfExists{definition:big_etale_site_of_a_scheme}{big étale site $(\Sch/X)_{\et}$}\CrefIfExists{definition:big_etale_site_of_a_scheme} whose objects are schemes $U$ equipped with an étale morphism $U \to X$.
    \item The Grothendieck topology is the one \CrefAndHyperrefIfExist{definition:grothendieck_topology_generated_by_a_pretopology}{generated by} the \CrefAndHyperrefIfExist{definition:basis_and_grothendieck_pretopology_for_a_grothendieck_topology_on_a_category}{pretopology} whose covering families are families $\{g_j : U_j \to U\}_{j \in J}$ of morphisms such that each $g_j$ is an \CrefAndHyperrefIfExist{definition:etale_morphism_of_schemes}{\'etale} and the family is jointly surjective on the underlying topological spaces.
\end{itemize}
\TextIfExists{definition:big_site_on_the_category_of_schemes_over_a_scheme_and_small_site}{Equivalently, the small \'etale site on $X$ is the \CrefAndHyperrefIfExist{definition:big_site_on_the_category_of_schemes_over_a_scheme_and_small_site}{small site for \'etale morphisms on $X$}.}
\TODO{state this as a fact}
$X_{\et}$ is an \CrefAndHyperrefIfExist{definition:essentially_small_category}{essentially small category}.
\end{definition}


\subsection{Other common sites of schemes}

\begin{definition} \label{definition:grothendieck_topology_generated_by_a_pretopology}
Let $\mathcal{C}$ be a category equipped with a \CrefAndHyperrefIfExist{definition:basis_and_grothendieck_pretopology_for_a_grothendieck_topology_on_a_category}{Grothendieck pretopology} $K$. The \hldef{Grothendieck topology generated by $K$}, denoted \hl{$J_K$}, is the smallest \CrefAndHyperrefIfExist{definition:grothendieck_topology_on_a_category_site_covering_sieve_topologically_generating_family}{Grothendieck topology} on $\mathcal{C}$ such that every family in $K(U)$ is a covering family for $J_K$.

Explicitly, a \CrefAndHyperrefIfExist{definition:sieve_on_an_object_in_a_category}{sieve} $S$ on an object $U$ belongs to $J_K(U)$ if and only if there exists a \CrefAndHyperrefIfExist{definition:basis_and_grothendieck_pretopology_for_a_grothendieck_topology_on_a_category}{covering family} $\{U_i \to U\}_{i \in I} \in K(U)$ such that for every $i \in I$, the morphism $U_i \to U$ belongs to $S$.

The condition that $S$ contains the family $\{U_i \to U\}$ is equivalent to saying that the sieve generated by this family is a sub-sieve of $S$.
\end{definition}
\begin{definition} \label{definition:basis_and_grothendieck_pretopology_for_a_grothendieck_topology_on_a_category}
Let $\mathcal{C}$ be a \CrefAndHyperrefIfExist{definition:category}{category}.
A \hldef{basis for a Grothendieck topology} (also called a \hldef{Grothendieck pretopology} or simply a \hldef{pretopology}) on $\mathcal{C}$ is a collection of families $K(U)$ of morphisms for each object $U \in \mathcal{C}$, called \hldef{coverings} or \hldef{covering families}, satisfying the following axioms:
\begin{enumerate}
    \item \textbf{(Identity)} For every isomorphism $U' \to U$, the singleton family $\{U' \to U\}$ is in $K(U)$.
    \item \textbf{(Base Change)} If $\{U_i \to U\}_{i \in I}$ is a covering family in $K(U)$ and $V \to U$ is any morphism in $\mathcal{C}$, then the \CrefAndHyperrefIfExist{definition:cartesian_product_of_two_objects_in_a_category_over_an_object}{fiber products} $U_i \times_U V$ exist, and the family of projections $\{U_i \times_U V \to V\}_{i \in I}$ is in $K(V)$.
    \item \textbf{(Composition)} If $\{U_i \to U\}_{i \in I}$ is in $K(U)$ and for each $i \in I$, $\{V_{ij} \to U_i\}_{j \in J_i}$ is in $K(U_i)$, then the composite family $\{V_{ij} \to U_i \to U\}_{i \in I, j \in J_i}$ is in $K(U)$.
\end{enumerate}

\end{definition}
\begin{definition}[Big Zariski site] \label{definition:big_zariski_site_of_a_scheme}
    Let $S$ be a scheme. The \hldef{big Zariski site of $S$}, denoted by \hl{$(\mathrm{Sch}/S)_{\mathrm{Zar}}$}, is the \CrefAndHyperrefIfExist{definition:grothendieck_topology_on_a_category_site_covering_sieve_topologically_generating_family}{site} whose underlying category is the cateogry of $S$-schemes and whose Grothendieck topology is the one \CrefAndHyperrefIfExist{definition:grothendieck_topology_generated_by_a_pretopology}{generated by} the \CrefAndHyperrefIfExist{definition:basis_and_grothendieck_pretopology_for_a_grothendieck_topology_on_a_category}{pretopology} whose coverings are families of morphisms
    $$ \{ f_i : U_i \to U \}_{i \in I} $$
    such that each $f_i$ is an \CrefAndHyperrefIfExist{definition:open_and_closed_immersions_of_schemes}{open immersion} and the images $\{ f_i(U_i) \}_{i \in I}$ form an open cover of $U$. 
    Such a covering is called a \hldef{Zariski covering of $U$}.
    
    Equivalently, the big Zariski site of $S$ is the \CrefAndHyperrefIfExist{definition:big_site_on_the_category_of_schemes_over_a_scheme_and_small_site}{big site on the category of schemes over $S$} for the class of open immersions.
\end{definition}

\begin{definition}[Small Zariski site] \label{definition:small_zariski_site_of_a_schem}
    Let $X$ be a scheme. The \hldef{small Zariski site of $X$}, denoted by \hl{$X_{\mathrm{Zar}}$}, is the \CrefAndHyperrefIfExist{definition:grothendieck_topology_on_a_category_site_covering_sieve_topologically_generating_family}{site} 
    
    whose underlying category consists of open subschemes $U \subseteq X$, with inclusions as morphisms,
    and whose Grothendieck topology is the one \CrefAndHyperrefIfExist{definition:grothendieck_topology_generated_by_a_pretopology}{generated by} the \CrefAndHyperrefIfExist{definition:basis_and_grothendieck_pretopology_for_a_grothendieck_topology_on_a_category}{pretopology} whose coverings are families of \CrefAndHyperrefIfExist{definition:open_and_closed_immersions_of_schemes}{open immersions}
    $$ \{ U_i \to U \}_{i \in I} $$
    such that the $U_i$ form an open cover of $U$ in the usual topological sense. 

    Equivalently, the small Zariski site of $X$ is the \CrefAndHyperrefIfExist{definition:big_site_on_the_category_of_schemes_over_a_scheme_and_small_site}{small site on the category of schemes over $X$} for the class of open immersions. 

    Also equivalently, the small Zariski site of $X$ is the \CrefAndHyperrefIfExist{definition:site_of_opens_on_a_topological_space}{Site of opens of $X$} as a topological space.
\end{definition}



\begin{definition}[Faithfully flat morphism of schemes] \label{definition:faithfully_flat_morphism_of_schemes}
    Let $f : X \to Y$ be a morphism of schemes.

    The morphism $f$ is \hldef{faithfully flat} if it is \hldef{flat} and \hldef{surjective} on the underlying topological spaces.

    More precisely:
    \begin{itemize}
        \item $f$ is \hldef{flat}, meaning for every $x \in X$ with $y = f(x)$, the local ring homomorphism
        \[
        \mathcal{O}_{Y,y} \to \mathcal{O}_{X,x}
        \]
        makes $\mathcal{O}_{X,x}$ a flat $\mathcal{O}_{Y,y}$-module.
        \item $f$ is \hldef{surjective} at the level of topological spaces, i.e., the continuous map $f: |X| \to |Y|$ is surjective.
    \end{itemize}

    Equivalently, $f$ is faithfully flat if the functor $f^* : \mathrm{QCoh}(Y) \to \mathrm{QCoh}(X)$ on quasi-coherent sheaves is both exact and faithful.
\end{definition}

\begin{definition}[big fppf site] \label{definition:big_fppf_site_over_a_scheme}
    Let $S$ be a scheme. The \hldef{big fppf site of $S$}, denoted by $(\mathrm{Sch}/S)_{\mathrm{fppf}}$, is the following \CrefAndHyperrefIfExist{definition:grothendieck_topology_on_a_category_site_covering_sieve_topologically_generating_family}{site}:
    \begin{itemize}
        \item The underlying category is the category \hyperrefIfExists{notation:Sch_S_as_the_category_of_schemes_over_a_scheme}{$\Sch/S$} of all schemes over $S$.

        \item The Grothendieck topology is the one \CrefAndHyperrefIfExist{definition:grothendieck_topology_generated_by_a_pretopology}{generated by} the \CrefAndHyperrefIfExist{definition:basis_and_grothendieck_pretopology_for_a_grothendieck_topology_on_a_category}{pretopology} whose coverings are families of morphisms $\{f_i: X_i \to X\}_{i \in I}$ in $\Sch/S$ is a covering family in $(\mathrm{Sch}/S)_{\mathrm{fppf}}$ if each $f_i$ is a \CrefAndHyperrefIfExist{definition:flat_morphism_of_schemes}{flat} morphism of \CrefAndHyperrefIfExist{definition:finitely_presented_algebra_over_a_not_necessarily_commutative_ring}{locally of finite presentation} such that $\bigcup_i f_i(X_i) = f(X)$. Such a cover is called an \hldef{fppf cover of $X$}.

    \end{itemize}
    Equivalently, the big fppf site of $S$ is the \CrefAndHyperrefIfExist{definition:big_site_on_the_category_of_schemes_over_a_scheme_and_small_site}{big site on $S$} for the class of \CrefAndHyperrefIfExist{definition:flat_morphism_of_schemes}{flat} and \CrefAndHyperrefIfExist{definition:locally_of_finite_presentation_finite_presentation_morphism_of_schemes}{locally of finite presentation} morphisms. 

    fppf stands for \emph{fid\`element plate de pr\'esentation finie}, which translates to faithfully flat and of finite presentation; note however, that the morphisms in an fppf cover are flat and locally of finite presentation, not \CrefAndHyperrefIfExist{definition:faithfully_flat_morphism_of_schemes}{faithfully flat} and of \CrefAndHyperrefIfExist{definition:finitely_presented_algebra_over_a_not_necessarily_commutative_ring}{finite presentation}.
\end{definition}

\begin{definition}[small fppf site] \label{definition:small_fppf_site_of_a_scheme}
    Let $X$ be a fixed scheme. The \hldef{small fppf site on $X$}, commonly denoted by notations including \hl{$X_{fppf}$}, or \hl{$\mathrm{fppf}_{/X}$}, is defined as the following \CrefAndHyperrefIfExist{definition:grothendieck_topology_on_a_category_site_covering_sieve_topologically_generating_family}{site}:
    \begin{itemize}
        \item The underlying category is the full subcategory of the \CrefAndHyperrefIfExist{definition:big_fppf_site_over_a_scheme}{big étale site $(\Sch/X)_{\mathrm{fppf}}$} whose objects are schemes $U$ equipped with a \TODO{}
        
        \TODO{}
        \item The Grothendieck topology is the one \CrefAndHyperrefIfExist{definition:grothendieck_topology_generated_by_a_pretopology}{generated by} the \CrefAndHyperrefIfExist{definition:basis_and_grothendieck_pretopology_for_a_grothendieck_topology_on_a_category}{pretopology} whose coverings are families of morphisms $\{g_j : U_j \to U\}_{j \in J}$ in this category is a covering family in $X_{\et}$ if each $g_j$ is \hyperrefIfExists{definition:etale_morphism_of_schemes}{\'etale}\CrefIfExists{definition:etale_morphism_of_schemes} and the family is jointly surjective on the underlying topological spaces.
    \end{itemize}
    \TextIfExists{definition:big_site_on_the_category_of_schemes_over_a_scheme_and_small_site}{Equivalently, the small fppfsite on $X$ is the \CrefAndHyperrefIfExist{definition:big_site_on_the_category_of_schemes_over_a_scheme_and_small_site}{small site for morphisms that are flat and locally of finite presentation on $X$}.}
    \TODO{state this as a fact}
    $X_{fppf}$ is an \CrefAndHyperrefIfExist{definition:essentially_small_category}{essentially small category}.
\end{definition}

\begin{definition}[big fpqc site] \label{definition:big_fpqc_site_of_a_scheme}
    \TODO{}
Let $S$ be a scheme. The \hldef{big fpqc site of $S$}, denoted by \hl{$(\mathrm{Sch}/S)_{\mathrm{fpqc}}$}, is following \CrefAndHyperrefIfExist{definition:grothendieck_topology_on_a_category_site_covering_sieve_topologically_generating_family}{site}:
\begin{enumerate}
    \item The underlying category is the category \hyperrefIfExists{notation:Sch_S_as_the_category_of_schemes_over_a_scheme}{$\Sch/S$} of all schemes over $S$.

    \item A family of morphisms $\{f_i: X_i \to X\}_{i \in I}$ in $\Sch/S$ is a covering family in $(\mathrm{Sch}/S)_{\mathrm{fpqc}}$ if 
    \begin{itemize}
        \item each $f_i$ is a \CrefAndHyperrefIfExist{definition:flat_morphism_of_schemes}{flat} morphism 
        \item for each affine open $U \subseteq X$, there exists a finite subset $J \subseteq I$ and affineo pen subsets $U_j \subseteq X_j$ for each $j \in J$ such that $U = \bigcup_{j \in J} f_j(U_j)$. 
        % exists a finite set $K$, a map $i: K \to I$, and affine opens $U_{i(k)} \subseteq T_{i(k)}$ such that $U = \bigcup_{k \in K} \varphi_{i(k)}(U_{i(k)})$
        % of \CrefAndHyperrefIfExist{definition:finitely_presented_algebra_over_a_not_necessarily_commutative_ring}{locally of finite presentation} such that $\bigcup_i f_i(X_i) = f(X)$. Such a cover is called an \hldef{fppf cover of $X$}.
    \end{itemize}
    Such a cover is called an \hldef{fpqc cover of $X$}.
\end{enumerate}
\end{definition}

\begin{definition}[Small fpqc site] \label{definition:small_fpqc_site_of_a_schem}
    \TODO{}
    Let $X$ be a scheme. The \hldef{small fpqc site of $X$}, denoted by \hl{$X_{\mathrm{fpqc}}$},
    is the site whose underlying category consists of schemes $U$ equipped with a morphism
    $$
    U \longrightarrow X,
    $$
    such that the morphism is \hldef{flat} and \hldef{quasi-compact} and whose Grothendieck topology is the one \CrefAndHyperrefIfExist{definition:grothendieck_topology_generated_by_a_pretopology}{generated by} the \CrefAndHyperrefIfExist{definition:basis_and_grothendieck_pretopology_for_a_grothendieck_topology_on_a_category}{pretopology} whose coverings are families of morphisms
    $$
    \{f_i : U_i \to U\}_{i \in I}
    $$
    over $X$ such that

    \begin{enumerate}
        \item Each $f_i$ is \hldef{faithfully flat}.
        \item Each $f_i$ is \hldef{quasi-compact}.
        \item The images of the family jointly cover $U$; that is,
        $$
        \bigcup_{i \in I} f_i(U_i) = U.
        $$
        \item For every affine open $V \subseteq U$, there exists a finite subset $J \subseteq I$ and affine opens $V_j \subseteq U_j$, for $j \in J$, such that
        $$
        V = \bigcup_{j \in J} f_j(V_j).
        $$
    \end{enumerate}

    This topology is called the \hldef{fpqc topology} (faithfully flat and quasi-compact).
\end{definition}


\begin{definition} \label{definition:elementary_distinguished_square_in_the_category_of_somoth_schemes_over_a_scheme}
    [See {\cite[Definition 2.1]{voevodsky_A1}}, {\cite[Section 3 Definition 1.3]{morel_voevodsky_1999}}]
    Let $S$ be a \CrefAndHyperrefIfExist{definition:scheme}{scheme}. An \hl{elementary distinguished square in the category $\Sm/S$} of \CrefAndHyperrefIfExist{definition:smooth_morphism_of_schemes}{smooth schemes} over $S$ is a square of the form
    \begin{equation}  \label{center:elementary_distinguished_square_of_nisnevich_topology}
    \begin{tikzcd} 
    p^{-1}(U) \ar[r] \ar[d, "p"] & V \ar[d, "p"] \\ U \ar[r,"j"] & X
    \end{tikzcd}
    \end{equation}
    \TODO{open embedding, reduced subscheme, support}
    such that $p$ is an \CrefAndHyperrefIfExist{definition:etale_morphism_of_schemes}{\'etale morphism}, $j$ is an open embedding, and $p^{-1}(X-U) \to X-U$ is an isomorphism (where $X-U$ is the maximal reduced subscheme with support in the closed subset $X-U$).
\end{definition}
\begin{proposition}[{\cite[Section 3 Proposition 1.1]{morel_voevodsky_1999}}] \label{proposition:nisnevich_pretopology_on_the_category_of_smooth_schemes_over_a_scheme}
    Let $S$ be a \CrefAndHyperrefIfExist{definition:locally_noetherian_and_noetherian_scheme}{Noetherian} scheme of finite \CrefAndHyperrefIfExist{definition:dimension_of_a_scheme}{dimension}. Let $X$ be a scheme of \CrefAndHyperrefIfExist{definition:finite_type_morphism_of_schemes}{finite type} over $S$ and let $\{U_i \to X\}$ be a finite family of \CrefAndHyperrefIfExist{definition:etale_morphism_of_schemes}{\'etale morphisms} in \CrefAndHyperrefIfExist{definition:scheme_over_a_scheme}{$\Sch/S$}. The following conditions are equivalent:
    \TODO{residue field}
    \begin{enumerate}
        \item For any point $x$ of $X$ there is an $i$ and a point $u$ of $U_i$ over $x$ such that the corresponding morphism of residue fields is an isomorphism which maps to $x$ with the same residue field.
        \item For any point $x \in X$, the morphism
        $$\coprod_i (U_i \times_X \Spec \scrO_{X,x}^h) \to \Spec \scrO_{X,x}^h$$
        of $S$-schemes admits a section.
    \end{enumerate}

    Moroever, the collection of families of \'etale morphisms $\{U_i \to X\}$ in $\Sm/S$ satisfying the equivalent conditions above forms a pretopology on \CrefAndHyperrefIfExist{definition:smooth_morphism_of_schemes}{$\Sm/S$}.
\end{proposition}
\begin{definition} \label{definition:nisnevich_topology_on_a_noetherian_scheme_of_finite_dimension}
    [{See \cite[Section 3 Definition 1.2]{morel_voevodsky_1999}}]
    Let $S$ be a \CrefAndHyperrefIfExist{definition:locally_noetherian_and_noetherian_scheme}{Noetherian} scheme of finite \CrefAndHyperrefIfExist{definition:dimension_of_a_scheme}{dimension}. 
    The \CrefAndHyperrefIfExist{definition:grothendieck_topology_on_a_category_site_covering_sieve_topologically_generating_family}{Grothendieck topology} generated by the \CrefAndHyperrefIfExist{definition:basis_and_grothendieck_pretopology_for_a_grothendieck_topology_on_a_category}{pretopology} of \Cref{proposition:nisnevich_pretopology_on_the_category_of_smooth_schemes_over_a_scheme} is called the \hldef{Nisnevich topology on $\Sm/S$}. The site whose underlying category is $\Sm/S$ and whose Grothendieck topology is the Nisnevich topology is called the \hldef{(big) Nisnevich site of $S$} and is denoted by notations such as \hl{$(\mathrm{Sm}/S)_{\mathrm{Nis}}$}, \hl{$(\mathbf{Sm}/S)_{\mathrm{Nis}}$}, etc. A covering family in the Nisnevich site of $S$ is called a \hldef{Nisnevich covering}.

    \TODO{establish that a nisnevich covering is a family of etale moprhisms such that there is an isomorphism of residue fields at every point}
\end{definition}

% \begin{definition}[Nisnevich covering]
%     Let $X$ be a scheme. A family of étale morphisms
%     $$
%     \{ f_i: U_i \to U \}
%     $$
%     in the category of schemes over $X$ is called a \hldef{Nisnevich covering} if for every point $x \in U$ there exists an index $i$ and a point $u \in U_i$ such that 
%     $$
%     f_i(u) = x
%     $$
%     and the induced extension of residue fields
%     $$
%     k(x) \to k(u)
%     $$
%     is an isomorphism.
% \end{definition}

\begin{definition}[Small Nisnevich site] \label{definition:small_nisnevich_site_of_a_schem}
    Let $X$ be a scheme.
    The \hldef{small Nisnevich site of $X$}, denoted by
    $$ X_{\mathrm{Nis}}, $$
    is the site whose underlying category consists of schemes $U$ equipped with an \CrefAndHyperrefIfExist{definition:etale_morphism_of_schemes}{étale morphism}
    $$ U \to X, $$
    and whose Grothendieck topology is the one \CrefAndHyperrefIfExist{definition:grothendieck_topology_generated_by_a_pretopology}{generated by} the \CrefAndHyperrefIfExist{definition:basis_and_grothendieck_pretopology_for_a_grothendieck_topology_on_a_category}{Grothendieck pretopology} whose coverings are given by \CrefAndHyperrefIfExist{definition:nisnevich_topology_on_a_noetherian_scheme_of_finite_dimension}{Nisnevich coverings}.
\end{definition}




\begin{definition}[crystalline site]
    \TODO{}
Let $S$ be a scheme and $p$ a prime number. 
Let $\mathcal{C}=\mathrm{CRIS}(S/\mathbf{Z}_p)$ denote the category whose objects are triples $(U,T,\delta)$ where 
\begin{itemize}
    \item $U$ is an open subscheme of $S$,
    \item $T$ is a scheme equipped with a divided power structure $\delta$ on an ideal $\mathcal{I} \subseteq \mathcal{O}_T$,
    \item together with a closed immersion $U \hookrightarrow T$ compatible with $\delta$ (called a PD-thickening).
\end{itemize}
The \hldef{crystalline site} $(S/\mathbf{Z}_p)_{\mathrm{cris}}$ is this category equipped with the \CrefAndHyperrefIfExist{definition:grothendieck_topology_on_a_category_site_covering_sieve_topologically_generating_family}{Grothendieck topology} \CrefAndHyperrefIfExist{definition:grothendieck_topology_generated_by_a_pretopology}{generated by} families of morphisms $\{(U_i,T_i,\delta_i) \to (U,T,\delta)\}$ such that the underlying morphisms of the $T_i \to T$ form a Zariski open cover of $T$.
\end{definition}


\begin{theorem}[Hierarchy of Grothendieck topologies on $\mathrm{Sch}/S$] \label{theorem:hierarchy_of_common_grothendieck_topologies_on_Sch_S}
    Let $S$ be a scheme and consider the following \CrefAndHyperrefIfExist{definition:grothendieck_topology_on_a_category_site_covering_sieve_topologically_generating_family}{Grothendieck topologies} on the category \CrefAndHyperrefIfExist{definition:scheme_over_a_scheme}{$\mathrm{Sch}/S$ of schemes over $S$}:
    \[
    \mathrm{Zar} \quad \text{(Zariski)}, \quad \mathrm{Nis} \quad \text{(Nisnevich)}, \quad \mathrm{\acute{E}t} \quad \text{(étale)}, \quad \mathrm{fppf}, \quad \mathrm{fpqc}.
    \]
    (\Cref{definition:big_zariski_site_of_a_scheme}, \Cref{definition:nisnevich_topology_on_a_noetherian_scheme_of_finite_dimension}, \Cref{definition:big_etale_site_of_a_scheme}, \Cref{definition:big_fppf_site_over_a_scheme}, \Cref{definition:big_fpqc_site_of_a_scheme})

    Then these topologies satisfy the chain of refinements (fineness) relations
    \[
    \mathrm{Zar} \; \prec \; \mathrm{Nis} \; \prec \; \mathrm{\acute{E}t} \; \prec \; \mathrm{fppf} \; \prec \; \mathrm{fpqc},
    \]
    meaning that each topology is strictly \CrefAndHyperrefIfExist{definition:coarser_and_finer_grothendieck_topologies}{finer} than the previous one, i.e., every covering in a coarser topology is a covering in any finer topology.

    More explicitly, for every object $X \in \mathrm{Sch}/S$,
    $$
    \mathrm{Cov}_{\mathrm{Zar}}(X) \subsetneq \mathrm{Cov}_{\mathrm{Nis}}(X) \subsetneq \mathrm{Cov}_{\mathrm{\acute{E}t}}(X) \subsetneq \mathrm{Cov}_{\mathrm{fppf}}(X) \subsetneq \mathrm{Cov}_{\mathrm{fpqc}}(X).
    $$
\end{theorem}

\begin{theorem}[Essential smallness of small sites] \label{theorem:common_small_sites_of_a_scheme_are_essentially_small}
Let $X$ be a scheme. Consider the following sites associated to $X$:
\begin{itemize}
    \item \CrefAndHyperrefIfExist{definition:small_zariski_site_of_a_schem}{small Zariski site $X_{\mathrm{Zar}}$}, 
    \item \CrefAndHyperrefIfExist{definition:small_nisnevich_site_of_a_schem}{small Nisnevich site $X_{\mathrm{Nis}}$},
    \item \CrefAndHyperrefIfExist{definition:small_etale_site_of_a_scheme}{small étale site $X_{\mathrm{\acute{E}t}}$}, 
    \item \CrefAndHyperrefIfExist{definition:small_fppf_site_of_a_scheme}{small fppf site $X_{\mathrm{fppf}}$}.
    \item \CrefAndHyperrefIfExist{definition:small_fpqc_site_of_a_schem}{small fpqc site $X_{\mathrm{fpqc}}$}.
\end{itemize}
These sites are \CrefAndHyperrefIfExist{definition:essentially_small_category}{essentially small}.
% Then the underlying category forming each of these small sites of $X$ is \hldef{essentially small}; that is, it is equivalent to a category whose objects form a set (rather than a proper class).

% In particular, this essential smallness follows from imposing suitable universe or size conditions by restricting to affine schemes of finite presentation over $X$ or by other standard set-theoretic techniques.

% Hence, these small sites can be used as Grothendieck sites in the usual sense without foundational size issues.
\end{theorem}







\subsection{Sheafication of a presheaf on a site}

\begin{definition} \label{definition:sheafification_functor_on_a_site}
    Let $\calC$ be a \CrefAndHyperrefIfExist{definition:grothendieck_topology_on_a_category_site_covering_sieve_topologically_generating_family}{site} and let $\calA$ be a \CrefAndHyperrefIfExist{definition:category}{(large) category}.

    Assuming that the \CrefAndHyperrefIfExist{definition:presheaf_on_a_category}{presheaf} category $\PreShv(\calC, \calA)$ (and hence the \CrefAndHyperrefIfExist{definition:sheaf_on_a_site}{sheaf} category $\Shv(\calC, \calA)$) is \CrefAndHyperrefIfExist{definition:locally_small_category}{locally small} (or $U$-locally small if a \CrefAndHyperrefIfExist{definition:grothendieck_universe}{Grothendieck universe} $U$ is available), a \hldef{sheafification functor} refers to a functor
    $$a: \PreShv(\calC, \calA) \to \Shv(\calC, \calA) $$
    that is \CrefAndHyperrefIfExist{definition:adjoint_functors_between_categories_unit_counit_of_adjoint_functors}{left adjoint} to the inclusion functor 
    $$i:\Shv(\calC, \calA) \hookrightarrow \PreShv(\calC, \calA)  .$$
    If such a sheafification functor exists, then it is unique up to unique natural isomorphism. Given a presheaf $P$, the sheafification $a(P)$ is also sometimes called the \hldef{sheaf associated to $P$}.
    \TextIfExists{theorem:sheafification_of_a_presheaf_of_sets_on_a_small_enough_site}{See \Cref{theorem:sheafification_of_a_presheaf_of_sets_on_a_small_enough_site} for common conditions under which sheafification exists.} 
\end{definition}

% See Also
%theorem:sheafification_of_a_presheaf_of_sets_on_a_small_enough_site
\begin{theorem}{cf. {\cite[Expos\'e II, Th\'eor\`eme 3.4]{SGA4_I}}} \label{theorem:sheafification_of_a_presheaf_of_sets_on_a_small_enough_site}
    \begin{enumerate}
        \item Let $U$ be a universe. Let $\calC$ be a \hyperrefIfExists{definition:grothendieck_topology_on_a_category_site_covering_sieve_topologically_generating_family}{$U$-site}\CrefIfExists{definition:grothendieck_topology_on_a_category_site_covering_sieve_topologically_generating_family}. A \CrefAndHyperrefIfExist{definition:sheafification_of_a_presheaf_on_a_topological_space_valued_in_a_category_admitting_direct_colimits}{sheafification functor}
        $$a: \Shv(\calC, \USets) \to \PreShv(\calC, \USets).$$
        exists. 
        % The inclusion functor 
        % $$i: \PreShv(\calC, \USets) \hookrightarrow \Shv(\calC, \USets)$$
        % has a \hyperrefIfExists{definition:adjoint_functors_between_categories_unit_counit_of_adjoint_functors}{left adjoint functor}\CrefIfExists{definition:adjoint_functors_between_categories_unit_counit_of_adjoint_functors}

        \item Let $\calC$ be a site whose underlying category is \CrefAndHyperrefIfExist{definition:locally_small_category}{locally small} and which has a \CrefAndHyperrefIfExist{definition:grothendieck_topology_on_a_category_site_covering_sieve_topologically_generating_family}{topologically generating family} that is a set (rather than a proper class). A sheafification functor 
        $$a: \Shv(\calC, \Sets) \to \PreShv(\calC, \Sets)$$
        exists.

        \item (see e.g. {\cite[3]{nlab:sheafification}}) Let $(\calC, J)$ be a \CrefAndHyperrefIfExist{definition:grothendieck_topology_on_a_category_site_covering_sieve_topologically_generating_family}{site} on an \CrefAndHyperrefIfExist{definition:essentially_small_category}{essentially small category} $\calC$. Suppose that the category $\calA$ is \CrefAndHyperrefIfExist{definition:complete_and_cocomplete_category}{complete, cocomplete}, that small \CrefAndHyperrefIfExist{definition:projective_and_inductive_limits_in_categories}{filtered colimits} in $\calA$ are exact, and that $\calA$ satisfies the IPC-property. A \CrefAndHyperrefIfExist{definition:sheafification_functor_on_a_site}{sheafification functor} 
        $$a: \PreShv(\calC, \calA) \to \Shv(\calC, \calA) $$
        exists.
        \TODO{IPC-property, exactess in this context.}

        \TODO{state as a fact that these categories are complete, cocomplete, with small filtered colimits that are exact}
        This is true for instance of $\calA = \mathbf{Set}, \mathbf{Grp}$, $k-\mathbf{Alg}$ for a field $k$, or $\mathbf{Mod}_R$ for a \CrefAndHyperrefIfExist{definition:ring}{(not necessarily commutative unital) ring $R$}.
    \end{enumerate}
\end{theorem}
\begin{remark}
    If the presheaf is valued in nice ``algebraic category'', e.g. groups, abelian groups, rings, modules over a ring, etc., then the sheafification is again valued in that category. \TODO{Make this more precise.}
\end{remark}

\begin{corollary}  \label{corollary:sheafification_functor_exists_for_nice_enough_categories_of_sheaves_on_common_small_sites_of_schemes}
    Let $X$ be a scheme. For the categories $\calA = \mathbf{Set}, \mathbf{Grp}$, $k-\mathbf{Alg}$ for a field $k$, or $\mathbf{Mod}_R$ for a \CrefAndHyperrefIfExist{definition:ring}{(not necessarily commutative unital) ring $R$}, and for the sites $\calC = X_{\mathrm{Zar}}, X_{\mathrm{Nis}}, X_{\mathrm{\acute{E}t}}, X_{\mathrm{fppf}}, X_{\mathrm{fpqc}}$,
    there exists a \CrefAndHyperrefIfExist{theorem:sheafification_of_a_presheaf_of_sets_on_a_small_enough_site}{sheafification functor}
    $$a: \PreShv(\calC, \calA) \to \Shv(\calC, \calA).$$
\end{corollary}
\begin{proof}
    This follows from \Cref{theorem:sheafification_of_a_presheaf_of_sets_on_a_small_enough_site} and \Cref{theorem:common_small_sites_of_a_scheme_are_essentially_small}.
\end{proof}

\subsection{Examples of sheaves on the \'etale site of a scheme}

Constant and locally constant sheaves are discussed not just for $X_{\et}$, but also for more general sites.


\begin{definition}[Constant sheaf on a site] \label{definition:constant_sheaf_on_a_site_with_sheafification}
    Let $\calC$ be a \hyperrefIfExists{definition:category}{(large) category}\CrefIfExists{definition:category}, let $\calA$ be a (large category), and let $A$ be an object of $\calA$. %a set (or more generally, an abelian group, ring, etc.).
    
    \begin{enumerate}
        \item The \hldef{constant presheaf on $\calC$ with value $A$} is the \hyperrefIfExists{definition:presheaf_on_a_category}{presheaf}\CrefIfExists{definition:presheaf_on_a_category} $P$ defined by
        \[
        P(U) = A
        \]
        for every object $U$ of $\calC$ such that every morphism $f: V \to U$ in $\calC$ induces the identity map $A = P(U)\to P(V) = A$. 

        \item Let $\calC$ be a \CrefAndHyperrefIfExist{definition:grothendieck_topology_on_a_category_site_covering_sieve_topologically_generating_family}{site} and assume that a \CrefAndHyperrefIfExist{definition:sheafification_functor_on_a_site}{sheafification functor} 
        $$a: \Shv(\calC, \calA) \to \PreShv(\calC, \calA)$$
        exists\TextIfExists{theorem:sheafification_of_a_presheaf_of_sets_on_a_small_enough_site}{~(e.g. see \Cref{theorem:sheafification_of_a_presheaf_of_sets_on_a_small_enough_site})}.
        The \hldef{constant sheaf on $\calC$ with value $A$}, or the \hldef{constant sheaf on $\calC$ associated to $A$} commonly denoted \hl{$\underline{A}$} or sometimes just \hl{$A$} by abuse of notation, is the \hyperrefIfExists{theorem:sheafification_of_a_presheaf_of_sets_on_a_small_enough_site}{sheaf associated to}\CrefIfExists{theorem:sheafification_of_a_presheaf_of_sets_on_a_small_enough_site} the constant presheaf $P$ with value $A$ above.

        \item Let $\calC$ be a site. Let $\calO$ be a sheaf of (not-necessarily commutative) rings on $\calC$. Assume that the \CrefAndHyperrefIfExist{definition:sections_of_a_presheaf_on_a_category_valued_in_a_category}{global sections ring $\Gamma(\calO)$} exists. A \hldef{constant $\calO$-module} is an \CrefAndHyperrefIfExist{definition:module_over_a_sheaf_of_rings_on_a_site}{$\calO$-module} $\calF$ which is isomorphic as a sheaf to the constant sheaf on $\calC$ with value $M$ where $M$ is a module of the ring $\Gamma(\calO)$. Note that sheafification functors exist for presheaves/sheaves valued in $\Ab$ (\Cref{theorem:sheafification_of_a_presheaf_of_sets_on_a_small_enough_site}).

        In case that $\calO$ is the constant sheaf associated to $A$ for some (not-necessarily commutative) ring $A$, a constant $\calO$-module is simply called a \hldef{constant $A$-module}.
    \end{enumerate}
\end{definition}

\begin{definition}[Locally constant sheaf on a site] \label{definition:locally_constant_sheaf_on_a_site_with_sheafification}

    Let $(\calC, J)$ be a \CrefAndHyperrefIfExist{definition:grothendieck_topology_on_a_category_site_covering_sieve_topologically_generating_family}{site}. 
\begin{enumerate}
    \item 

    Let $\calA$ be a (large) category, and let $A$ be an object of $\calA$. %a set (or more generally, an abelian group, ring, etc.).
    \TODO{If such a sheafification functor exist, does a sheafification functor exist when restricted to an object $U$?}
    Assume that a \CrefAndHyperrefIfExist{definition:sheafification_functor_on_a_site}{sheafification functor} 
    $$a: \PreShv(\calC, J, \calA) \to \Shv(\calC,J, \calA)$$
    \CrefIfExists{definition:sheaf_on_a_site}
    \CrefIfExists{definition:presheaf_on_a_category}
    exists\TextIfExists{theorem:sheafification_of_a_presheaf_of_sets_on_a_small_enough_site}{~(e.g. see \Cref{theorem:sheafification_of_a_presheaf_of_sets_on_a_small_enough_site})}. Let $U$ be an object of $\calC$.

    A sheaf $\mathcal{F}$ on $C$ with values in $\calA$ is said to be a \hldef{locally constant sheaf on $U$ with value $A$} if there exists a \CrefAndHyperrefIfExist{definition:grothendieck_topology_on_a_category_site_covering_sieve_topologically_generating_family}{covering sieve} $\{ U_i \to U \}$ of every object $U$ in $C$ such that for each $i$, the \CrefAndHyperrefIfExist{definition:restriction_of_a_sheaf_on_a_site_to_an_object_of_the_underlying_category_of_the_site}{restriction $\mathcal{F}|_{U_i}$} is isomorphic to the \CrefAndHyperrefIfExist{definition:constant_sheaf_on_a_site_with_sheafification}{constant sheaf} $\underline{A}$ on the \CrefAndHyperrefIfExist{definition:site_induced_by_a_site_on_an_over_category}{slice site $\calC_{/{U_i}}$}. 

    If $\calC$ has a final object, then we may say that $\calF$ is a \hldef{locally constant sheaf with value $A$} if it is a locally constants sheaf on the final object with value $A$.
    
    %constant sheaf $\underline{A}|_{U_i}$.

    % In other words,
    % \[
    % \forall U \in C,\, \exists\, \text{cover} \{ U_i \to U \} :\, \forall i,\, \mathcal{F}|_{U_i} \cong \underline{A}|_{U_i}.
    % \]

    \item Let $\calO$ be a sheaf of (not-necessarily commutative) rings on $\calC$. 
    % Assume that the \CrefAndHyperrefIfExist{definition:sections_of_a_presheaf_on_a_category_valued_in_a_category}{global sections ring $\Gamma(\calO)$} exists. 
    A \hldef{locally constant $\calO$-module} is a \CrefAndHyperrefIfExist{definition:module_over_a_sheaf_of_rings_on_a_site}{$\calO$-module} $\calF$ such that there exists a \CrefAndHyperrefIfExist{definition:grothendieck_topology_on_a_category_site_covering_sieve_topologically_generating_family}{covering sieve} $\{U_i \to U\}$ for every object $U$ in $C$ such that for each $i$, the \CrefAndHyperrefIfExist{definition:restriction_of_a_sheaf_on_a_site_to_an_object_of_the_underlying_category_of_the_site}{restriction} $\calF|_{U_i}$ is isomorphic, as an $\calO|_{U_i}$-module, to a \CrefAndHyperrefIfExist{definition:constant_sheaf_on_a_site_with_sheafification}{constant $\calO|_{U_i}$-module}\footnote{The \CrefAndHyperrefIfExist{definition:sections_of_a_presheaf_on_a_category_valued_in_a_category}{global sections rings $\Gamma(\calO|_{U_i})$} of the sheaves $\calO|_{U_i}$ on \CrefAndHyperrefIfExist{definition:site_induced_by_a_site_on_an_over_category}{the slice sites $\calC_{/U_i}$} exist because each $\calC_{/U_i}$ has a \CrefAndHyperrefIfExist{definition:initial_final_zero_objects_of_a_category}{final object}(\Cref{lemma:slice_category_has_final_object}), so we may speak of constant $\calO|_{U_i}$-modules.}.

    We additionally say that $\calF$ is 
    \begin{enumerate}
        \item \hldef{locally free of rank $r$ over $\calO$} if there exists a covering $\{U_i \to U\}$ such that $\calF|_{U_i}$ is isomorphic as a $\calO|_{U_i}$-module to $(\calO|_{U_i})^{\oplus r}$ for each $i$.

        \item \hldef{free of rank $r$ over $\calO$} if $\calF \cong \calO^{\oplus r}$ as $\calO$-modules.

        \item \hldef{of finite type} if there exists a covering $\{U_i \to U\}$ such that $\calF|_{U_i}$ generated by finitely many sections over $U_i$ as an $\calO|_{U_i}$-module. In other words, there is an epimorphism 
        $$(\calO|_{U_i})^{\oplus n_i} \to \calF|_{U_i}$$
        of $\calO|_{U_i}$-modules for each $i$.
    \end{enumerate}

    In case that $\calO$ is the \CrefAndHyperrefIfExist{definition:constant_sheaf_on_a_site_with_sheafification}{constant sheaf on $\calC$ associated to $A$} for some (not-necessarily commutative) ring $A$, a locally constant $\calO$-module is simply called a \hldef{locally constant $A$-module}.

    

    % \item Let $\Lambda$ be a commutative ring. A locally constant sheaf $\calF$ of $\Lambda$-modules is said to be 
    % \begin{enumerate}
    %     \item \hldef{locally free of rank $r$ over $\Lambda$} if it is valued in $\Lambda^{\oplus r}$.
    %     \item \hldef{of finite type} if it is locally valued in some finitely generated $\Lambda$-modules. 
    % \end{enumerate}
\end{enumerate}
\end{definition}


% \begin{definition}[Locally Constant Sheaf on a General Site]
% Let \((\mathcal{C}, J)\) be a site and let \(\Lambda\) be a commutative ring. A sheaf \(\mathcal{F}\) of \(\Lambda\)-modules on \(\mathcal{C}\) is said to be \hldef{locally constant} if for every object \(U \in \mathcal{C}\) there exists a covering \(\{U_i \to U\}\) in the topology \(J\) such that for each \(i\), the restricted sheaf \(\mathcal{F}|_{U_i}\) is isomorphic to a constant sheaf on \(U_i\) associated to a \(\Lambda\)-module \(M_i\).

% Moreover, \(\mathcal{F}\) is said to be of \hldef{finite rank over $\Lambda$} if for each such \(U_i\), the \(\Lambda\)-module \(M_i\) is a finitely generated free \(\Lambda\)-module.
% \end{definition}


\begin{definition} \label{definition:representable_functor_on_a_category_enriched_in_a_monoidal_category}
    Let $C$ be a \CrefAndHyperrefIfExist{definition:category_enriched_in_a_monoidal_category}{category enriched in a monidal category} $\mathcal{V}$. Given an object $X$ of $C$, the \hldef{functor of points} \hl{$h_X$} is the \CrefAndHyperrefIfExist{definition:functor_between_categories}{functor}/\CrefAndHyperrefIfExist{definition:presheaf_on_a_category}{presheaf} $C^{\op} \to \mathcal{V}$ given by $T \mapsto \Hom_C(T, X)$. A functor $C^{\op} \to \mathcal{V}$ (or equivalently, a presheaf on $C$ valued in $\mathcal{V}$) is said to be \hldef{representable} if it is \CrefAndHyperrefIfExist{definition:natural_transformation_between_functors_between_categories}{naturally isomorphic} to some functor $h_X$ of points for an object $X$ of $C$.

    Dually, a functor $C \to \calV$ is called \hldef{co-representable} if it is naturally isomorphic to a functor $T \mapsto \Hom_C(X, T)$ for an object $X$ in $C$. 

    For instance, we may speak of these notions when $\calV$ is the monoidal category $\Sets$, i.e. $C$ is a \CrefAndHyperrefIfExist{definition:locally_small_category}{locally small category}.
\end{definition}

\begin{proposition}[Representable functors are étale sheaves] \label{proposition:representable_functors_are_etale_sheaves}
    \TODO{This is probably more generally true for fppf, fpqc sites, etc.}
    Let $X$ be a \CrefAndHyperrefIfExist{definition:scheme}{scheme} and consider the \CrefAndHyperrefIfExist{definition:small_etale_site_of_a_scheme}{étale site $X_{\et}$}.

    If a functor
    $$ F : (X_{\et})^{\mathrm{op}} \to \mathbf{Sets} $$
    is \CrefAndHyperrefIfExist{definition:representable_functor_on_a_category_enriched_in_a_monoidal_category}{representable} by an $X$-scheme $Y$, i.e.,
    $$ F(U) \cong \operatorname{Hom}_X(U, Y) $$
    for all étale $U \to X$, then the presheaf $F$ is an \CrefAndHyperrefIfExist{definition:small_etale_site_of_a_scheme}{étale} \CrefAndHyperrefIfExist{definition:sheaf_on_a_site}{sheaf}.
\end{proposition}


\begin{definition} \label{definition:additive_sheaf_on_the_small_etale_site_of_a_scheme}
    \TODO{regular function on a scheme}
For a \CrefAndHyperrefIfExist{definition:scheme}{scheme} $X$, the \hldef{additive sheaf} \hldef{$\mathbb{G}_a$} on $X_{\text{ét}}$ is defined by
$$ \mathbb{G}_a(U) := \mathcal{O}_U(U), $$
the ring of regular functions on $U$ for each étale $U \to X$.  It may also be called the \hldef{additive group scheme}, on account of the fact that the sheaf $\mathbb{G}_a$ is \CrefAndHyperrefIfExist{definition:representable_functor_on_a_category_enriched_in_a_monoidal_category}{representable by} the scheme $\mathbb{G}_a = \bbA_X^1$\CrefIfExists{definition:additive_group_scheme_over_a_scheme}. \TODO{affine line over a scheme} In particular, it is indeed a sheaf (\Cref{proposition:representable_functors_are_etale_sheaves}).
\end{definition}

\begin{definition}
For a scheme $X$, the \hldef{multiplicative sheaf} \hldef{$\mathbb{G}_m$} on $X_{\text{ét}}$ is defined by
$$ \mathbb{G}_m(U) := \mathcal{O}_U(U)^\times, $$
the group of invertible regular functions on $U$ for each étale $U \to X$. It may also be called the \hldef{multiplicative group scheme}, on account of the fact that the sheaf $\mathbb{G}_m$ is \CrefAndHyperrefIfExist{definition:representable_functor_on_a_category_enriched_in_a_monoidal_category}{representable by} the scheme $\mathbb{G}_m = \bbA_X^1 \setminus \{0\}$. \TODO{affine line over a scheme} In particular, it is indeed a sheaf (\Cref{proposition:representable_functors_are_etale_sheaves}).
\end{definition}


\begin{definition}[The sheaf of $n$-th roots of unity \hl{$\mu_n$}] \label{definition:sheaf_of_nth_roots_of_unity_on_the_small_etale_site_of_a_scheme}
    \TODO{invertible integer on a scheme}
    \TODO{kernel for sheaves}
Let $n$ be a positive integer invertible on the \CrefAndHyperrefIfExist{definition:scheme}{scheme} $X$. The \CrefAndHyperrefIfExist{definition:sheaf_on_a_site}{sheaf} \hldef{$\mu_n$} on \CrefAndHyperrefIfExist{definition:small_etale_site_of_a_scheme}{$X_{\text{ét}}$} is defined as the kernel of the multiplication-by-$n$ map on $\mathbb{G}_m$:
$$ \mu_n := \ker\big( \mathbb{G}_m \xrightarrow{(\cdot)^n} \mathbb{G}_m \big). $$
In other words,
$$ \mu_n(U) = \{ f \in \mathcal{O}_U(U)^\times \mid f^n = 1 \} $$
for each \CrefAndHyperrefIfExist{definition:etale_morphism_of_schemes}{étale} $U \to X$.
\end{definition}

We can use the sheaves $\mu_{\ell^n}$ for a prime $\ell$ to construct $\ell$-adic sheaves \CrefAndHyperrefIfExist{definition:ell_adic_tate_twist_sheaves_on_the_small_etale_site_of_a_scheme}{$\bbZ_\ell(d)$} called \emph{Tate twists}; we note that \CrefAndHyperrefIfExist{definition:lambda_adic_sheaf_on_the_small_etale_site_of_a_scheme_for_an_ideal_of_a_commutative_ring}{$\ell$-adic sheaves} are not actual sheaves on $X_{\et}$, but rather a certain type of inverse system of sheaves on $X_{\et}$. 




\subsection{Stalk of a (pre)sheaf on the small \'etale site of a scheme}

As with Zariski (pre)sheaves (or more generally locally ringed spaces), one can talk about stalks at points --- for the \'etale site the stalks are not just as closed points, but rather at \emph{geometric} points.


\begin{definition}[Stalk of an étale sheaf at a geometric point] \label{definition:stalk_of_a_presheaf_on_the_small_etale_site_of_a_schem_at_a_geometric_point}
Let $X$ be a \CrefAndHyperrefIfExist{definition:scheme}{scheme}, $\bar{x} : \operatorname{Spec}(\Omega) \to X$ a \CrefAndHyperrefIfExist{definition:geometric_point_of_a_scheme}{geometric point}, and let $\mathcal{F}$ be a \CrefAndHyperrefIfExist{definition:presheaf_on_a_category}{presheaf} of sets (resp. abelian groups, etc.) on the \CrefAndHyperrefIfExist{definition:small_etale_site_of_a_scheme}{étale site $X_{\text{ét}}$}. 

The \hldef{stalk of $\mathcal{F}$ at $\bar{x}$} is defined as the \CrefAndHyperrefIfExist{definition:limit_and_colimit_of_a_diagram_in_a_category}{colimit}

$$ \mathcal{F}_{\bar{x}} := \varinjlim_{(U,u)} \mathcal{F}(U), $$
where the limit runs over the \CrefAndHyperrefIfExist{definition:filtered_cofiltered_category}{filtering category} of pairs $(U,u)$ consisting of an \CrefAndHyperrefIfExist{definition:etale_morphism_of_schemes}{étale morphism} $U \to X$ together with a lift $u : \operatorname{Spec}(\Omega) \to U$ of $\bar{x}$, i.e., such that the composed morphism $u: \operatorname{Spec}(\Omega) \to U \to X$ equals $\bar{x}$.

This construction makes $\mathcal{F}_{\bar{x}}$ a set (resp. an abelian group, etc.), intuitively representing the "germ" of sections of $\mathcal{F}$ near the geometric point $\bar{x}$.
\end{definition}

\Cref{theorem:stalks_of_etale_sheaves_at_geometric_points_are_the_values_of_the_sheaves_at_the_strict_henselization_of_the_local_rings_at_the_image_of_the_point} describes stalks of \'etale sheaves in terms of \CrefAndHyperrefIfExist{definition:strict_henselization_of_a_local_ring}{strict henselizations}. We define the notions of Henselian and strictly Henselian local rings.


\begin{definition}[Henselian ring] \label{defintion:henselian_ring}
Let $(R, \mathfrak{m})$ be a \CrefAndHyperrefIfExist{definition:local_ring}{local ring} with \CrefAndHyperrefIfExist{definition:prime_and_maximal_ideal_of_a_ring}{maximal ideal $\mathfrak{m}$}. The ring $R$ is called \hldef{henselian} if for every monic polynomial $f(x) \in R[x]$ and factorization
$$ \overline{f}(x) = \overline{g}(x) \cdot \overline{h}(x) $$
in $(R/\mathfrak{m})[x]$ into coprime monic polynomials $\overline{g}, \overline{h}$, there exist monic polynomials $g(x), h(x) \in R[x]$ lifting $\overline{g}, \overline{h}$ respectively such that
$$ f(x) = g(x) \cdot h(x).  $$
\end{definition}

\begin{theorem} \label{theorem:complete_local_rings_are_henselian}
    \TODO{complete local ring}
    All complete local rings are \CrefAndHyperrefIfExist{defintion:henselian_ring}{henselian}.
\end{theorem}

\begin{definition}[Strictly henselian ring] \label{definition:strictly_henselian_ring}
    \TODO{residue field, separably closed}
A \hldef{strictly henselian ring} is a \CrefAndHyperrefIfExist{defintion:henselian_ring}{henselian local ring} $(R, \mathfrak{m})$ whose residue field $k = R/\mathfrak{m}$ is separably closed.
\end{definition}

\begin{theorem} \label{theorem:complete_local_rings_with_separably_closed_residue_fields_are_strictly_henselian}
    \TODO{complete local ring}
    \TODO{separably closed}
    \TODO{residue field of local ring}
    All complete local rings with separably closed residue fields are \CrefAndHyperrefIfExist{defintion:strictly_henselian_ring}{strictly henselian}.
\end{theorem}



\begin{definition}[Henselization] \label{definition:henselization_of_a_local_ring}
Let $(R, \mathfrak{m})$ be a \CrefAndHyperrefIfExist{definition:local_ring}{local ring}. The \hldef{henselization of $R$} is the initial object among \CrefAndHyperrefIfExist{defintion:henselian_ring}{henselian local} $R$-algebras $(R^h, \mathfrak{m}^h)$ equipped with a \CrefAndHyperrefIfExist{definition:local_morphism_of_local_rings}{local homomorphism} $R \to R^h$ inducing an isomorphism on residue fields. Concretely, it is a henselian local ring $R^h$ together with a morphism $R \to R^h$ such that any local ring homomorphism from $R$ to a henselian local ring factors uniquely through $R^h$.
\end{definition}

\begin{definition}[Strict henselization] \label{definition:strict_henselization_of_a_local_ring}
    \TODO{separable closure}
    \TODO{residue field}
Let $(R, \mathfrak{m})$ be a \CrefAndHyperrefIfExist{definition:local_ring}{local ring} with residue field $k$. Fix a separable closure $k^{\text{sep}}$ of $k$. The \hldef{strict henselization} \hl{$R^{sh}$} of $R$ is the henselization of the local ring of the étale site of $\operatorname{Spec}(R)$ at the geometric point corresponding to $k^{\text{sep}}$. It is a strictly henselian local ring equipped with a local homomorphism $R \to R^{sh}$ inducing the embedding $k \subseteq k^{\text{sep}}$ on residue fields.
\end{definition}

\begin{theorem}[Examples of Henselizations and Strict Henselizations of Specific Local Rings] \label{theorem:examples_of_henselizations_and_strict_henselizations}
Let $p$ be a prime number
% and $K$ be a number field or a global function field with ring of integers $\mathcal{O}_K$. Consider the following examples of Henselizations and strict Henselizations:

\begin{enumerate}
    \item The ring of integers localized at $p$, $\mathbb{Z}_{(p)}$, has \CrefAndHyperrefIfExist{definition:henselization_of_a_local_ring}{Henselization} given by the $p$-adic integers $\mathbb{Z}_p$. More precisely, 
    $$
    \mathbb{Z}_p \cong \text{Henselization of } \mathbb{Z}_{(p)}
    $$
    and is a complete discrete valuation ring that is Henselian.

    \item The \CrefAndHyperrefIfExist{definition:strict_henselization_of_a_local_ring}{strict Henselization} of $\mathbb{Z}_{(p)}$ is the maximal unramified extension of $\mathbb{Z}_p$, obtained by adjoining all roots of unity of order prime to $p$. It is strictly Henselian and can be viewed as the ring of integers of the maximal unramified extension of the $p$-adic field $\mathbb{Q}_p$.
    
    % \item For a number field or global function field $K$ with ring of integers $\mathcal{O}_K$, let $\mathfrak{p}$ be a prime ideal of $\mathcal{O}_K$. The Henselization of the localization $(\mathcal{O}_K)_{\mathfrak{p}}$ is a Henselian local ring that algebraically approximates the completion $\widehat{(\mathcal{O}_K)_{\mathfrak{p}}}$, but need not be complete.
    
    % \item Its strict Henselization is obtained by extending scalars to a separable closure of the residue field at $\mathfrak{p}$ and then henselizing. This gives a strictly Henselian local ring associated to $(\mathcal{O}_K)_{\mathfrak{p}}$.
\end{enumerate}
% These examples illustrate the importance of Henselizations and strict Henselizations as algebraic substitutes of completions, capturing local algebraic properties crucial in arithmetic and algebraic geometry.
\end{theorem}
\begin{theorem}[Stalks of étale sheaves and strict henselizations] \label{theorem:stalks_of_etale_sheaves_at_geometric_points_are_the_values_of_the_sheaves_at_the_strict_henselization_of_the_local_rings_at_the_image_of_the_point}


Let $X$ be a \CrefAndHyperrefIfExist{definition:scheme}{scheme}, and let $\bar{x} : \operatorname{Spec}(\Omega) \to X$ be a \CrefAndHyperrefIfExist{definition:geometric_point_of_a_scheme}{geometric point}, where $\Omega$ is an algebraically closed field. Let $\mathcal{F}$ be a sheaf on the \CrefAndHyperrefIfExist{definition:small_etale_site_of_a_scheme}{étale site $X_{\text{ét}}$}. Then the \CrefAndHyperrefIfExist{definition:stalk_of_a_presheaf_on_the_small_etale_site_of_a_schem_at_a_geometric_point}{stalk of $\mathcal{F}$ at $\bar{x}$} is isomorphic to the value of the pullback sheaf on the strict henselization of the local ring of $X$ at the image of $\bar{x}$.

More precisely, if $(\mathcal{O}_{X,x})^{sh}$ denotes the \CrefAndHyperrefIfExist{definition:strict_henselization_of_a_local_ring}{strict henselization} of the local ring at the point $x \in X$ which is the image of $\bar{x}$, then

$$ \mathcal{F}_{\bar{x}} \cong \mathcal{F}((\mathcal{O}_{X,x})^{sh}), $$

where the right-hand side denotes the sections of $\mathcal{F}$ over $\operatorname{Spec}((\mathcal{O}_{X,x})^{sh})$ viewed as an object in the étale site.

Consequently, the notion of stalks at geometric points connects the local behavior of étale sheaves with the algebraic and topological properties of strictly henselian local rings.
\end{theorem}




\subsection{Constant and locally constant sheaves}


\section{Locally constant sheaves and local systems}


\begin{definition} \label{definition:local_system_of_Lambda_modules_on_a_scheme}
    \TODO{TODO: distinguish between the case where $\Lambda$ is finite and where $\Lambda$ is a limit of finite rings}
Let $X$ be a scheme and let $\Lambda$ be a commutative ring. A \hldef{local system of $\Lambda$-modules on $X$} is a \hyperrefIfExists{definition:locally_constant_sheaf_on_a_site_with_sheafification}{locally constant sheaf of finite free $\Lambda$-modules}\CrefIfExists{definition:locally_constant_sheaf_on_a_site_with_sheafification} on the \hyperrefIfExists{definition:small_etale_site_of_a_scheme}{(small) étale site $X_{\mathrm{\acute{e}t}}$}\CrefIfExists{definition:small_etale_site_of_a_scheme}.
\end{definition}

% See Also
% \begin{theorem}[See {\cite[Tags 0DV5, 0GIY]{stacks-project}}] \label{theorem:locally_constant_sheaves_on_X_etale_are_equivalent_to_representations_of_the_etale_fundamental_group} 
    \TODO{Work out a statement for $\Qell$-coefficients}
    Let $X$ be a connected scheme and let $\barx \in X$ be a \CrefAndHyperrefIfExist{definition:geometric_point_of_a_scheme}{geometric point}.
    \begin{enumerate}
        \item There is an equivalence of categories
        \TODO{define finite in this context}
        \[
        \left\{
        \begin{matrix}
        \text{finite locally constant} \\
        \text{sheaves of sets on } X_{\acute{e}tale}
        \end{matrix}
        \right\}
        \longleftrightarrow
        \left\{
        \begin{matrix}
        \text{finite } \pioneet(X, \overline{x})\text{-sets}
        \end{matrix}
        \right\}.
        \]
        \CrefIfExists{definition:locally_constant_sheaf_on_a_site_with_sheafification}\CrefIfExists{definition:small_etale_site_of_a_scheme}\CrefIfExists{definition:etale_fundamental_group_of_a_connected_scheme}

        \item There is an equivalence of categories
        \[
        \left\{
        \begin{matrix}
        \text{finite locally constant} \\
        \text{sheaves of abelian groups on } X_{\acute{e}tale}
        \end{matrix}
        \right\}
        \longleftrightarrow
        \left\{
        \begin{matrix}
        \text{finite } \pioneet(X, \overline{x})\text{-modules}
        \end{matrix}
        \right\}.
        \]

        \item For a finite ring $\Lambda$, there is an equivalence of categories
        \[
        \left\{
        \begin{matrix}
        \text{finite type, locally constant} \\
        \text{sheaves of } \Lambda\text{-modules on } X_{\acute{e}tale}
        \end{matrix}
        \right\}
        \longleftrightarrow
        \left\{
        \begin{matrix}
        \text{finite } \pioneet(X, \overline{x})\text{-modules endowed} \\
        \text{with commuting } \Lambda\text{-module structure}
        \end{matrix}
        \right\}.
        \]
        (\CrefIfExists{definition:locally_constant_sheaf_on_a_site_with_sheafification})

        \item Assume that $X$ is irreducible and geometrically unibranch. For a ring $\Lambda$, there is an equivalence of categories
        \[
        \left\{
        \begin{matrix}
        \text{finite type, locally constant} \\
        \text{sheaves of } \Lambda\text{-modules on } X_{\acute{e}tale}
        \end{matrix}
        \right\}
        \longleftrightarrow
        \left\{
        \begin{matrix}
        \text{finite } \Lambda\text{-modules } M \text{ endowed}
        \\
        \text{with a continuous } \pioneet(X, \overline{x})\text{-action}
        \end{matrix}
        \right\}.
        \]
    \end{enumerate}

\end{theorem}

\begin{theorem}[See {\cite[Tags 0DV5, 0GIY]{stacks-project}}] \label{theorem:locally_constant_sheaves_on_X_etale_are_equivalent_to_representations_of_the_etale_fundamental_group} 
    \TODO{Work out a statement for $\Qell$-coefficients}
    Let $X$ be a connected scheme and let $\barx \in X$ be a \CrefAndHyperrefIfExist{definition:geometric_point_of_a_scheme}{geometric point}.
    \begin{enumerate}
        \item There is an equivalence of categories
        \TODO{define finite in this context}
        \[
        \left\{
        \begin{matrix}
        \text{finite locally constant} \\
        \text{sheaves of sets on } X_{\acute{e}tale}
        \end{matrix}
        \right\}
        \longleftrightarrow
        \left\{
        \begin{matrix}
        \text{finite } \pioneet(X, \overline{x})\text{-sets}
        \end{matrix}
        \right\}.
        \]
        \CrefIfExists{definition:locally_constant_sheaf_on_a_site_with_sheafification}\CrefIfExists{definition:small_etale_site_of_a_scheme}\CrefIfExists{definition:etale_fundamental_group_of_a_connected_scheme}

        \item There is an equivalence of categories
        \[
        \left\{
        \begin{matrix}
        \text{finite locally constant} \\
        \text{sheaves of abelian groups on } X_{\acute{e}tale}
        \end{matrix}
        \right\}
        \longleftrightarrow
        \left\{
        \begin{matrix}
        \text{finite } \pioneet(X, \overline{x})\text{-modules}
        \end{matrix}
        \right\}.
        \]

        \item For a finite ring $\Lambda$, there is an equivalence of categories
        \[
        \left\{
        \begin{matrix}
        \text{finite type, locally constant} \\
        \text{sheaves of } \Lambda\text{-modules on } X_{\acute{e}tale}
        \end{matrix}
        \right\}
        \longleftrightarrow
        \left\{
        \begin{matrix}
        \text{finite } \pioneet(X, \overline{x})\text{-modules endowed} \\
        \text{with commuting } \Lambda\text{-module structure}
        \end{matrix}
        \right\}.
        \]
        (\CrefIfExists{definition:locally_constant_sheaf_on_a_site_with_sheafification})

        \item Assume that $X$ is irreducible and geometrically unibranch. For a ring $\Lambda$, there is an equivalence of categories
        \[
        \left\{
        \begin{matrix}
        \text{finite type, locally constant} \\
        \text{sheaves of } \Lambda\text{-modules on } X_{\acute{e}tale}
        \end{matrix}
        \right\}
        \longleftrightarrow
        \left\{
        \begin{matrix}
        \text{finite } \Lambda\text{-modules } M \text{ endowed}
        \\
        \text{with a continuous } \pioneet(X, \overline{x})\text{-action}
        \end{matrix}
        \right\}.
        \]
    \end{enumerate}

\end{theorem}

\section{Six functors}

\subsection{The functors on sheaves}

\subsubsection{Inverse and direct images of sheaves via continuous functors of general sites}
\begin{definition} \label{definition:continuous_functor_of_sites}
Let $(\calC,J)$ and $(\calD,K)$ be \CrefAndHyperrefIfExist{definition:grothendieck_topology_on_a_category_site_covering_sieve_topologically_generating_family}{sites}. 

A functor $u : \calC \to \calD$ is said to be a \hldef{continuous functor of sites} if, for every object $U \in \operatorname{Ob}(\calD)$ and every \CrefAndHyperrefIfExist{definition:grothendieck_topology_on_a_category_site_covering_sieve_topologically_generating_family}{covering sieve} $S \in K(U)$, the \CrefAndHyperrefIfExist{definition:pullback_sieve_of_an_object_in_a_category_via_a_morphism_to_the_object}{pullback sieve $u^*S$} belongs to $J(V)$ for all $V \in \calC$ with a morphism $u(V) \to U$ in $\calD$.

Equivalently, $u$ is continuous if for every \CrefAndHyperrefIfExist{definition:sheaf_on_a_site}{sheaf} of sets $F$ on $\calD$, the \CrefAndHyperrefIfExist{definition:presheaf_on_a_category}{presheaf} $\calC^{\op} \to \Sets, X \mapsto F(u(X))$ is a sheaf on $\calC$. 
\TODO{show these are equivalent}
\TODO{define morphism of sites and recheck ref's to this definition}

% A \hldef{morphism of sites} $f: (\calD, K) \to (\calC, J)$ 

% Synonymously, we call a continuous functor $u: C \to D$ a \hldef{morphism of sites}.
\end{definition}
\begin{definition} \label{definition:continuous_functor_between_sites_of_opens_on_topological_spaces_induced_by_continuous_map}
    Let $(X, \tau_X)$ and $(Y, \tau_Y)$ be \CrefAndHyperrefIfExist{definition:topological_space}{topological spaces}, and let $f : X \to Y$ be a \CrefAndHyperrefIfExist{definition:continuous_map_of_topological_spaces}{continuous map}.
    Let $\operatorname{Open}(X)$ and $\operatorname{Open}(Y)$ be their respective \CrefAndHyperrefIfExist{definition:category_of_opens_of_a_topological_space}{categories of open sets} with inclusion morphisms, equipped with the \CrefAndHyperrefIfExist{definition:site_of_opens_on_a_topological_space}{canonical} \CrefAndHyperrefIfExist{definition:grothendieck_topology_on_a_category_site_covering_sieve_topologically_generating_family}{Grothendieck topologies} given by open coverings.

    Define the functor 
    $$\hlin{f^{-1} : \operatorname{Open}(Y) \to \operatorname{Open}(X), \quad U \mapsto f^{-1}(U).}$$
    It is a \CrefAndHyperrefIfExist{definition:continuous_functor_of_sites}{continuous functor of sites} from $\operatorname{Open}(Y)$ to $\operatorname{Open}(Y)$ which induces a \CrefAndHyperrefIfExist{definition:morphism_of_sites}{site morphism} 
    $$f: (\operatorname{Open}(X), \text{can}) \to (\operatorname{Open}(Y), \text{can})$$.
\end{definition}
\begin{definition} \label{definition:continuous_functors_on_sites_on_schemes_induced_by_scheme_morphism}
        Let $f : X \to Y$ be a morphism of schemes, and consider one of the common Grothendieck topologies on schemes such as the Zariski, étale, Nisnevich, fppf, fpqc, or crystalline topology. Denote by $\mathbf{C}(X)$ and $\mathbf{C}(Y)$ the corresponding small \CrefAndHyperrefIfExist{definition:grothendieck_topology_on_a_category_site_covering_sieve_topologically_generating_family}{sites} of $X$ and $Y$ (i.e., categories of morphisms to $X$ and $Y$ respectively equipped with one of these topologies).

        Then the \CrefAndHyperrefIfExist{definition:base_change_of_a_morphism_in_a_category_by_a_morphism}{base change functor}
        \[
        f^{-1} : \mathbf{C}(Y) \to \mathbf{C}(X), \quad (V \to Y) \mapsto (V \times_Y X \to X)
        \]
        \CrefIfExists{definition:cartesian_product_of_two_objects_in_a_category_over_an_object} is a \CrefAndHyperrefIfExist{definition:continuous_functor_of_sites }{continuous functor}. It in fact induces a \CrefAndHyperrefIfExist{definition:morphism_of_sites}{morphism of sites} 
        $$f: (\mathbf{C}(X), \tau_{\mathbf{C}}) \to (\mathbf{C}(Y), \tau_{\mathbf{C}})$$
        where $\tau_{\mathbf{C}}$ denotes the chosen topology (Zariski, étale, Nisnevich, fppf, fpqc, crystalline).
\end{definition}

\begin{definition} \label{definition:direct_image_of_a_sheaf_on_a_site_under_a_continuous_functor_of_sites_or_a_site_morphism}
Let $(\calC,J)$ and $(\calD,K)$ be \CrefAndHyperrefIfExist{definition:grothendieck_topology_on_a_category_site_covering_sieve_topologically_generating_family}{sites}  with small \CrefAndHyperrefIfExist{definition:grothendieck_topology_on_a_category_site_covering_sieve_topologically_generating_family}{topological generating families} (or $U$-small topologically generating families if a \CrefAndHyperrefIfExist{definition:grothendieck_universe}{universe} $U$ is available), and let $u : \calC \to \calD$ be a \CrefAndHyperrefIfExist{definition:continuous_functor_of_sites}{continuous functor of sites}. 
%Let $\mathcal{A}$ be a (large) category which has all small (or $U$-small) \CrefAndHyperrefIfExist{definition:product_and_coproduct_of_objects_in_a_category}{products}.

For any \CrefAndHyperrefIfExist{definition:sheaf_on_a_site}{sheaf} 
\[
\mathcal{F} \in \operatorname{Sh}(\calD,K;\mathcal{A}),
\]
Define the \hldef{pushforward/direct image sheaf} \hl{$u^s \calF$} by 
$$\hlin{u^s \mathcal{F} := \mathcal{F} \circ u : \calC^{\mathrm{op}} \to \mathcal{A}.}$$
Because $u$ is continuous, $u^s\mathcal{F}$ is a sheaf on $(\calC,J)$ valued in $\mathcal{A}$. The assignment $\mathcal{F} \mapsto u^s\mathcal{F}$ defines the \hldef{direct image/pushforward functor}
$$\hlin{u^s : \operatorname{Sh}(\calD,K;\mathcal{A}) \to \operatorname{Sh}(\calC,J;\mathcal{A}).}$$

If $u$ is the functor underlying a \CrefAndHyperrefIfExist{definition:morphism_of_sites}{site morphism} $f: (\calD, K) \to (\calC, J)$, we may alternatively denote $u^s \calF$ by \hl{$f_* \calF$} and call it the \hldef{direct image/pushforward of $\calF$ under $f$}; the assignment $\calF \mapsto f_* \calF$ is then the \hldef{direct image/pushforward functor}.
$$\hlin{f_* : \operatorname{Sh}(\calD,K;\mathcal{A}) \to \operatorname{Sh}(\calC,J;\mathcal{A}).}$$

Note that while the continuous functor $u$ and the site morphism $f$ point in opposite directions, the definition $f_* := u^s$ ensures that $f_*$ corresponds to the standard geometric pushforward used in topology and algebraic geometry.



% We further note that $u^*$ is ``categorical'' notation whereas $f_*$ is ``geometric'' notation; loosely speaking, given a morphism $f: X \to Y$ of topological spaces or schemes, 
% \begin{itemize}
%     \item we may have a continuous functor $u: \mathbf{C}(Y) \to \mathbf{C}(X)$ where $\mathbf{C}(X), \mathbf{C}(Y)$ are appropriate sites induced by $X$ and $Y$ respectively,
%     \item $u$ may underlie a site morphism $f: \mathbf{C}(X) \to \mathbf{C}(Y)$ roughly given by pullbacks under the morphism $f: X \to Y$, and
%     \item given a sheaf $\calF$ on $\mathbf{C}(X)$, we may speak of its direct image $f_* \calF$ on $\mathbf{C}(Y)$.
% \end{itemize}
\end{definition}


% \begin{definition} \label{definition:inverse_image_of_a_sheaf_under_a_continuous_functor_of_sites_or_a_site_morphism}
% Let $(\calC,J)$ and $(\calD,K)$ be \CrefAndHyperrefIfExist{definition:grothendieck_topology_on_a_category_site_covering_sieve_topologically_generating_family}{sites}  with small \CrefAndHyperrefIfExist{definition:grothendieck_topology_on_a_category_site_covering_sieve_topologically_generating_family}{topological generating families} (or $U$-small topologically generating families if a \CrefAndHyperrefIfExist{definition:grothendieck_universe}{universe} $U$ is available), and let $u : \calC \to \calD$ be a \CrefAndHyperrefIfExist{definition:continuous_functor_of_sites}{continuous functor of sites}. Let $\mathcal{A}$ be a (large) category which has all small (or $U$-small) \CrefAndHyperrefIfExist{definition:product_and_coproduct_of_objects_in_a_category}{products}. For any \CrefAndHyperrefIfExist{definition:sheaf_on_a_site}{sheaf} 
% \[
% \mathcal{G} \in \operatorname{Sh}(\calC,J;\mathcal{A}),
% \]
% the \hldef{direct image/pushforward sheaf of $\mathcal{G}$ under $u$} is defined by
% $$\hlin{u_*\mathcal{G} : \calD^{\mathrm{op}} \to \mathcal{A}, \quad V \mapsto \varprojlim_{(u \downarrow V)^{\op}} \mathcal{G}(U),}$$
% where the \CrefAndHyperrefIfExist{definition:projective_and_inductive_limits_in_categories}{limit} is taken over the \CrefAndHyperrefIfExist{definition:opposite_category_of_a_category}{opposite} of the \CrefAndHyperrefIfExist{definition:comma_category_of_two_functors_to_a_category}{comma category $(u \downarrow V)$} of whose objects are pairs $(U, u(U) \to V)$ with $U \in \calC$ and $u(U) \to V$ a moprhism in $\calD$. 

% The assignment $\mathcal{G} \mapsto u_*\mathcal{G}$ defines the \hldef{direct image functor}
% $$\hlin{u_* : \operatorname{Sh}(\calC,J;\mathcal{A}) \to \operatorname{Sh}(\calD,K;\mathcal{A})}.$$

% If $u$ is the functor underlying a \CrefAndHyperrefIfExist{definition:morphism_of_sites}{site morphism} $f: (\calD, K) \to (\calC, J)$, we may alternatively denote $u_* \calG$ by \hl{$f^* \calG$} and call it the \hldef{inverse image/pullback of $\calG$ under $f$}; the assignment $\calG \mapsto f^* \calG$ is then the \hldef{inverse image/pullback functor}.
% $$\hlin{f^* : \operatorname{Sh}(\calC,J;\mathcal{A}) \to \operatorname{Sh}(D,K;\mathcal{A}).}$$

% We further note that $u_*$ is ``categorical'' notation whereas $f^*$ is ``geometric'' notation; loosely speaking, given a morphism $f: X \to Y$ of topological spaces or schemes, 
% \begin{itemize}
%     \item we may have a continuous functor $u: \mathbf{C}(Y) \to \mathbf{C}(X)$ where $\mathbf{C}(X), \mathbf{C}(Y)$ are appropriate sites induced by $X$ and $Y$ respectively,
%     \item $u$ may underlie a site morphism $f: \mathbf{C}(X) \to \mathbf{C}(Y)$ roughly given by pullbacks under the morphism $f: X \to Y$, and
%     \item given a sheaf $\calG$ on $\mathbf{C}(Y)$, we may speak of its direct image $f^* \calG$ on $\mathbf{C}(X)$.
% \end{itemize}

% \end{definition}

\begin{definition} \label{definition:inverse_image_of_a_sheaf_under_a_continuous_functor_of_sites_or_a_site_morphism}
    % \begin{definition} \label{definition:inverse_image_of_a_sheaf_on_a_site_under_a_continuous_functor_of_sites}
Let $(\calC,J)$ and $(\calD,K)$ be \CrefAndHyperrefIfExist{definition:grothendieck_topology_on_a_category_site_covering_sieve_topologically_generating_family}{sites} with small \CrefAndHyperrefIfExist{definition:grothendieck_topology_on_a_category_site_covering_sieve_topologically_generating_family}{topological generating families}, and let $u : \calC \to \calD$ be a \CrefAndHyperrefIfExist{definition:continuous_functor_of_sites}{continuous functor of sites}. Let $\mathcal{A}$ be a (large) category such that the \CrefAndHyperrefIfExist{definition:presheaf_on_a_category}{presheaf category} $\operatorname{PreSh}(\calD,K;\mathcal{A})$ has \CrefAndHyperrefIfExist{definition:sheafification_functor_on_a_site}{sheafification}.

% Let $\mathcal{A}$ be a category admitting all small colimits and finite limits. 
For any \CrefAndHyperrefIfExist{definition:sheaf_on_a_site}{sheaf} 
\[
\mathcal{G} \in \operatorname{Sh}(\calC,J;\mathcal{A}),
\]
the \hldef{inverse image/pullback sheaf of $\mathcal{G}$ under $u$} is defined, assuming that all colimits below exist, as:
$$\hlin{u_s \mathcal{G} : \calD^{\mathrm{op}} \to \mathcal{A}, \quad V \mapsto a \left( \varinjlim_{(V \downarrow u)} \mathcal{G}(U) \right),}$$
where $a$ is the \CrefAndHyperrefIfExist{definition:definition:sheafification_functor_on_a_site}{sheafification} functor of presheaves and the \CrefAndHyperrefIfExist{definition:limit_and_colimit_of_a_diagram_in_a_category}{colimit} is taken over the \CrefAndHyperrefIfExist{definition:comma_category_of_two_functors_to_a_category}{comma category $(V \downarrow u)$} of pairs $(U, V \to u(U))$ with $U \in \calC$.

The assignment $\mathcal{G} \mapsto u_s\mathcal{G}$ defines the \hldef{inverse image/pullback functor}
$$\hlin{u_s : \operatorname{Sh}(\calC,J;\mathcal{A}) \to \operatorname{Sh}(\calD,K;\mathcal{A})}.$$

If $u$ is the functor underlying a \CrefAndHyperrefIfExist{definition:morphism_of_sites}{site morphism} $f: (\calD, K) \to (\calC, J)$, we may alternatively denote $u_s \calG$ by \hl{$f^{*} \calG$} (or sometimes by \hl{$f^{-1} \calG$}) and call it the \hldef{inverse image/pullback of $\calG$ under $f$}.


Note that while the continuous functor $u$ and the site morphism $f$ point in opposite directions, the identification $f^* := u_s$ ensures that $f^*$ corresponds to the standard geometric pullback. In the case of topological spaces, this recovers the usual construction involving colimits over open neighborhoods to obtain stalks followed by sheafification.

\end{definition}
% \end{definition}

\begin{theorem}
    Let $(C,J)$ and $(D,K)$ be \CrefAndHyperrefIfExist{definition:grothendieck_topology_on_a_category_site_covering_sieve_topologically_generating_family}{sites}  whose underlying categories are \CrefAndHyperrefIfExist{definition:essentially_small_category}{essentially small} (or essentially $U$-small if a \CrefAndHyperrefIfExist{definition:grothendieck_universe}{universe} $U$ is available), and let $u : C \to D$ be a \CrefAndHyperrefIfExist{definition:continuous_functor_of_sites}{continuous functor of sites}. Let $\mathcal{A}$ be a \CrefAndHyperrefIfExist{definition:locally_small_category}{locally small} (or $U$-locally small) category which has all small (or $U$-small) \CrefAndHyperrefIfExist{definition:product_and_coproduct_of_objects_in_a_category}{products}. In particular, $\operatorname{Sh}(D,K;\mathcal{A})$ and $\operatorname{Sh}(C,J;\mathcal{A})$ are locally small\footnote{as a consequence of \Cref{lemma:category_of_presheaves_on_a_small_category_of_locally_small_value_is_locally_small}}.

    The \CrefAndHyperrefIfExist{definition:direct_image_of_a_sheaf_on_a_site_under_a_continuous_functor_of_sites_or_a_site_morphism}{inverse image} and \CrefAndHyperrefIfExist{definition:inverse_image_of_a_sheaf_under_a_continuous_functor_of_sites_or_a_site_morphism}{direct image} functors 
    $$u^* : \operatorname{Sh}(D,K;\mathcal{A}) \to \operatorname{Sh}(C,J;\mathcal{A})$$
    $$u_* : \operatorname{Sh}(C,J;\mathcal{A}) \to \operatorname{Sh}(D,K;\mathcal{A})$$
    are \CrefAndHyperrefIfExist{definition:adjoint_functors_between_categories_unit_counit_of_adjoint_functors}{adjoint}
    For any \CrefAndHyperrefIfExist{definition:sheaf_on_a_site}{sheaves}, 
    \[
    \calF \in\operatorname{Sh}(C,J;\mathcal{A}), \quad \mathcal{G} \in \operatorname{Sh}(D,K;\mathcal{A}),
    \]
    there is a natural isomorphism
    $$\Hom_{\operatorname{Sh}(C,J;\mathcal{A})}(u^*\calG, \calF) \cong \Hom_{\operatorname{Sh}(D,K;\mathcal{A})}(\calG, u_* \calF).$$

\end{theorem}


\begin{definition}[Pullback (inverse image) of a sheaf] \label{definition:inverse_image_of_a_sheaf_on_a_topological_space}
    Let $f : X \to Y$ be a \CrefAndHyperrefIfExist{definition:continuous_map_between_open_subsets_of_euclidean_spaces}{continuous map} between \CrefAndHyperrefIfExist{definition:topological_space}{topological spaces}. Let $\calD$ be a category  with a \CrefAndHyperrefIfExist{lemma:initial_or_final_object_in_a_category_that_is_also_in_a_full_subcategory_is_initial_or_final_in_the_subcategory}{terminal object}.
    
    \begin{enumerate}
        \item  Let $\mathcal{G}$ be a \CrefAndHyperrefIfExist{definition:presheaf_on_a_topological_space}{presheaf on $Y$ valued in a $\calD$}.  
        The \hldef{pullback} or \hldef{inverse image presheaf} 
        \hl{$f^{-1} \mathcal{G}$} on $X$ is defined as the presheaf 
        $$U \mapsto \varinjlim_{V \supseteq f(U)} \mathcal{G}(V)$$
        where $U$ ranges over open subsets of $X$ and the colimit is taken over all open subsets $V \subseteq Y$ containing $f(U)$.  
        \TextIfExists{definition:stalk_of_a_presheaf_on_a_topological_space_at_a_point}{This construction admits a natural isomorphism
        $$(f^{-1}\mathcal{G})_x \to \mathcal{G}_{f(x)}$$
        of \CrefAndHyperrefIfExist{definition:stalk_of_a_presheaf_on_a_topological_space_at_a_point}{stalks} for every $x \in X$.}  

        \item 
        If $\calG$ is a \CrefAndHyperrefIfExist{definition:sheaf_on_a_topological_space_valued_in_a_category_with_a_terminal_object}{sheaf} valued in $\calD$, then we can define the \hldef{pullback} or \hldef{inverse image sheaf} \hl{$f^* \calG$} on $X$ as the \CrefAndHyperrefIfExist{definition:sheafification_functor_on_a_site}{sheaf associated to the presheaf} $f^{-1} \calG$, assuming it exists.
        \TextIfExistsElse{definition:direct_image_of_a_sheaf_on_a_site_under_a_continuous_functor_of_sites_or_a_site_morphism}{

            Assuming that a \CrefAndHyperrefIfExist{definition:sheafification_functor_on_a_site}{sheafification functor} exists, one may equivalently define $f^* \calG$ via \Cref{definition:inverse_image_of_a_sheaf_under_a_continuous_functor_of_sites_or_a_site_morphism} --- More concretely, $f^* \calG$ is the following equivalent constructions:
            \begin{itemize}
                \item The \CrefAndHyperrefIfExist{definition:inverse_image_of_a_sheaf_under_a_continuous_functor_of_sites_or_a_site_morphism}{direct image} $(f^{-1})_s \calG$ of $\calG$ under the continuous functor $f^{-1}: \operatorname{Open} Y \to \operatorname{Open} X, \quad W \mapsto f^{-1}(W)$\CrefIfExists{definition:site_of_opens_on_a_topological_space}\CrefIfExists{definition:continuous_functor_between_sites_of_opens_on_topological_spaces_induced_by_continuous_map}. 

                \item The \CrefAndHyperrefIfExist{definition:inverse_image_of_a_sheaf_under_a_continuous_functor_of_sites_or_a_site_morphism}{inverse image} of $\calG$ under the \CrefAndHyperrefIfExist{definition:morphism_of_sites}{site morphism} $\operatorname{Open} X \to \operatorname{Open} Y$ whose underlying \CrefAndHyperrefIfExist{definition:continuous_functor_of_sites}{continuous functor} is $f^{-1}$

            \end{itemize}
            
        }{
        }

    \end{enumerate}
\end{definition}


\begin{definition}[Pushforward (direct image) of a sheaf] \label{definition:direct_image_of_a_sheaf_on_a_topological_space}
    Let $f : X \to Y$ be a \CrefAndHyperrefIfExist{definition:continuous_map_between_open_subsets_of_euclidean_spaces}{continuous map} between \CrefAndHyperrefIfExist{definition:topological_space}{topological spaces}, and let $\mathcal{F}$ be a \CrefAndHyperrefIfExist{definition:presheaf_on_a_topological_space}{presheaf on $X$ valued in a category $\mathcal{D}$} with a \CrefAndHyperrefIfExist{lemma:initial_or_final_object_in_a_category_that_is_also_in_a_full_subcategory_is_initial_or_final_in_the_subcategory}{terminal object}.  
    The \hldef{pushforward} or \hldef{direct image presheaf} 
    \hl{$f_* \mathcal{F}$} on $Y$ is the \CrefAndHyperrefIfExist{definition:presheaf_on_a_category}{presheaf valued in $\calD$ on $Y$} defined as follows: For every open set $V \subseteq Y$, the value of the pushforward is given by
    \[ 
    f_* \mathcal{F}(V) := \mathcal{F}(f^{-1}(V)). 
    \]
    For an inclusion of open sets $V' \subseteq V$ in $Y$, the restriction morphism 
    \[ 
    \operatorname{res}_{V, V'}^{f_* \mathcal{F}} : f_* \mathcal{F}(V) \to f_* \mathcal{F}(V') 
    \]
    is defined as the restriction morphism of $\mathcal{F}$ associated with the inclusion of preimages $f^{-1}(V') \subseteq f^{-1}(V)$ in $X$:
    \[ 
    \operatorname{res}_{f^{-1}(V), f^{-1}(V')}^{\mathcal{F}} : \mathcal{F}(f^{-1}(V)) \to \mathcal{F}(f^{-1}(V')). 
    \]
    
    % by
    % $$f_* \mathcal{F}(V) := \mathcal{F}(f^{-1}(V))$$
    % for every open set $V \subseteq Y$, with restriction maps induced from those of $\mathcal{F}$ via preimages.  

    If $\calF$ is a \CrefAndHyperrefIfExist{definition:sheaf_on_a_topological_space_valued_in_a_category_with_a_terminal_object}{sheaf}, then so is $f_* \calF$. \TextIfExists{definition:direct_image_of_a_sheaf_on_a_site_under_a_continuous_functor_of_sites_or_a_site_morphism}{In this case, it is equivalent to define $f_* \calF$ as the \CrefAndHyperrefIfExist{definition:direct_image_of_a_sheaf_on_a_site_under_a_continuous_functor_of_sites_or_a_site_morphism}{direct image} $(f^{-1})^s \calF$ of $\calF$ under the \CrefAndHyperrefIfExist{definition:continuous_functor_between_sites_of_opens_on_topological_spaces_induced_by_continuous_map}{continuous functor $f^{-1}: \operatorname{Open} Y \to \operatorname{Open} X$}\CrefIfExists{definition:continuous_functor_between_sites_of_opens_on_topological_spaces_induced_by_continuous_map}\CrefIfExists{definition:site_of_opens_on_a_topological_space}, which is also equivalent to the \CrefAndHyperrefIfExist{definition:direct_image_of_a_sheaf_on_a_site_under_a_continuous_functor_of_sites_or_a_site_morphism}{direct image} of $\calF$ under the \CrefAndHyperrefIfExist{definition:morphism_of_sites}{site morphism} $\operatorname{Open} X \to \operatorname{Open} Y$ whose underlying continuous functor is $f^{-1}$.}
    
\end{definition}


\begin{definition}  \label{definition:inverse_image_and_direct_image_of_sheaves_on_schemes_for_various_topologies}
    \TODO{think about if assumptions on the small site and $\calA$ are needed}
Let $f : X \to Y$ be a morphism of schemes, and let $\mathbf{C}(X)$ and $\mathbf{C}(Y)$ be \CrefAndHyperrefIfExist{definition:big_site_on_the_category_of_schemes_over_a_scheme_and_small_site}{small sites} associated to $X$ and $Y$ respectively, equipped with any common Grothendieck topologies such as Zariski, étale, Nisnevich, fppf, fpqc, or crystalline. Let $\calA$ be a (large) category.

\begin{enumerate}
    \item Given a sheaf $\calG \in \operatorname{Sh}(\mathbf{C}(Y); \mathcal{A})$, the \hldef{inverse image} \hl{$f^* \calG$} along $f$ is defined to be the \CrefAndHyperrefIfExist{definition:inverse_image_of_a_sheaf_under_a_continuous_functor_of_sites_or_a_site_morphism}{inverse image} of $\calG$ under the \CrefAndHyperrefIfExist{definition:continuous_functor_of_sites}{continuous functor} \CrefAndHyperrefIfExist{definition:continuous_functors_on_sites_on_schemes_induced_by_scheme_morphism}{$f^{-1} : \mathbf{C}(Y) \to \mathbf{C}(X)$} on sites. In particular, it is an object of $\operatorname{Sh}(\mathbf{C}(X); \mathcal{A})$ and $f^*$ yields a functor
    $$f^* : \operatorname{Sh}(\mathbf{C}(Y); \mathcal{A}) \to \operatorname{Sh}(\mathbf{C}(X); \mathcal{A}),$$

    \item Given a sheaf $\calF \in \operatorname{Sh}(\mathbf{C}(X); \mathcal{A})$, the \hldef{direct image} \hl{$f_* \calF$} along $f$ is defined to be the \CrefAndHyperrefIfExist{definition:direct_image_of_a_sheaf_on_a_site_under_a_continuous_functor_of_sites_or_a_site_morphism}{direct image} of $\calF$ under the \CrefAndHyperrefIfExist{definition:continuous_functor_of_sites}{continuous functor} \CrefAndHyperrefIfExist{definition:continuous_functors_on_sites_on_schemes_induced_by_scheme_morphism}{$f^{-1} : \mathbf{C}(Y) \to \mathbf{C}(X)$} on sites. In particular, it is an object of $\operatorname{Sh}(\mathbf{C}(Y); \mathcal{A})$ if it exists. If $f_* \calF$ exists for all sheaves $\calF \in \operatorname{Sh}(\mathbf{C}(X); \mathcal{A})$, then $f_*$ yields a functor
    $$f_* : \operatorname{Sh}(\mathbf{C}(X); \mathcal{A}) \to \operatorname{Sh}(\mathbf{C}(Y); \mathcal{A}).$$
\end{enumerate}
\end{definition}

\begin{definition} \label{definition:ringed_site}
    \TODO{there are places where sites and sheaves of rings on them are used, but it would be better to just have them be ringed sites.}

    A \hldef{ringed site} is a \CrefAndHyperrefIfExist{definition:grothendieck_topology_on_a_category_site_covering_sieve_topologically_generating_family}{site} $(\mathcal{C}, J)$ with a small \CrefAndHyperrefIfExist{definition:grothendieck_topology_on_a_category_site_covering_sieve_topologically_generating_family}{topological generating family} equipped with a \CrefAndHyperrefIfExist{definition:sheaf_on_a_site}{sheaf} of (not necessarily commutative) rings $\mathcal{O}$. If the Grothendieck topology $J$ is clear in context, one may even write that $(C, \calO)$ is a ringed site.

    A \hldef{morphism of ringed sites}
    $$ ((\mathcal{C},J),\mathcal{O}) \to ((\mathcal{C}',J'),\mathcal{O}') $$
    consists of a \CrefAndHyperrefIfExist{definition:morphism_of_sites}{morphism of sites} $f : (\mathcal{C},J) \to (\mathcal{C}',J')$ and a \CrefAndHyperrefIfExist{definition:sheaf_on_a_site}{morphism of sheaves} of rings $f^\# : \mathcal{O}' \to f_*\mathcal{O}$ \CrefIfExists{definition:inverse_image_of_a_sheaf_under_a_continuous_functor_of_sites_or_a_site_morphism}.
\end{definition}



\subsubsection{Extension by zero of sheaves via open immersions of schemes}

\begin{definition}[Extension by zero functor for open immersions of topological spaces] \label{definition:extension_by_zero_of_a_sheaf_on_a_topological_space_under_open_immersion}
Let $X$ be a topological space and let $j : U \hookrightarrow X$ be an open immersion, where $U$ is an open subset of $X$. Consider the category $\mathrm{Sh}(X)$ of sheaves of abelian groups on $X$ and $\mathrm{Sh}(U)$ on $U$. The \hldef{extension by zero functor}
$$\hlin{j_! : \mathrm{Sh}(U) \to \mathrm{Sh}(X)}$$
is defined as follows: for any sheaf $\mathcal{F}$ on $U$, the sheaf $j_! \mathcal{F}$ on $X$ is given by
\[
(j_! \mathcal{F})(V) = \{ s \in \mathcal{F}(V \cap U) \mid \text{$s$ extends by zero outside $U$} \}
\]
for each open subset $V \subseteq X$. Concretely, sections over $V$ are sections over $V \cap U$, and sections supported outside $U$ are identified with zero.
\end{definition}


\begin{definition}[Extension by zero functor for open immersions and the étale site] \label{definition:extension_by_zero_of_a_sheaf_on_the_small_etale_site_on_a_scheme_under_open_immersion}
Let $X$ be a \CrefAndHyperrefIfExist{definition:scheme}{scheme}, $j : U \hookrightarrow X$ an open immersion of schemes, and let $\mathit{Et}/X$ (resp. $\mathit{Et}/U$) denote the \CrefAndHyperrefIfExist{definition:small_etale_site_of_a_scheme}{small étale site} of $X$ (resp. $U$). The \hldef{extension by zero functor}
$$\hlin{j_! : \mathrm{Sh}(\mathit{Et}/U) \to \mathrm{Sh}(\mathit{Et}/X)}$$
is defined as follows: for any sheaf $\mathcal{F}$ of abelian groups on $\mathit{Et}/U$ and any object $V \to X$ of $\mathit{Et}/X$,
\begin{itemize}
    \item if $V \to X$ factors through $U$ (i.e., $V \times_X U \cong V$), then $(j_!\mathcal{F})(V) = \mathcal{F}(V \times_X U)$,
    \item otherwise, $(j_!\mathcal{F})(V) = 0$.
\end{itemize}
The assignment $j_!$ defines an exact functor called the extension by zero of sheaves via the open immersion $j$.
\end{definition}

\subsubsection{Hom and tensor product functors of sheaves}

\TODO{}

\subsubsection{Exactness properties on sheaves of abelian groups}


\begin{theorem}[Exactness of stalk functors on étale sheaves of abelian groups]
Let $X$ be a \CrefAndHyperrefIfExist{definition:scheme}{scheme}. For any \CrefAndHyperrefIfExist{definition:geometric_point_of_a_scheme}{geometric point} $\bar{x} : \operatorname{Spec}(\Omega) \to X$, the \CrefAndHyperrefIfExist{definition:stalk_of_a_presheaf_on_the_small_etale_site_of_a_schem_at_a_geometric_point}{stalk} functor
$$ \mathcal{F} \mapsto \mathcal{F}_{\bar{x}} $$
is an \CrefAndHyperrefIfExist{definition:exact_functor_between_abelian_categories}{exact functor} from the category of \CrefAndHyperrefIfExist{definition:sheaf_on_a_site}{sheaves} of abelian groups on \CrefAndHyperrefIfExist{definition:small_etale_site_of_a_scheme}{$X_{\text{ét}}$} (\Cref{proposition:examples_of_abelian_categories}) to the category of abelian groups. 

That is, for every short exact sequence of sheaves of abelian groups
$$ 0 \to \mathcal{F}' \to \mathcal{F} \to \mathcal{F}'' \to 0, $$
the induced sequence of stalks at $\bar{x}$
$$ 0 \to \mathcal{F}'_{\bar{x}} \to \mathcal{F}_{\bar{x}} \to \mathcal{F}''_{\bar{x}} \to 0 $$
is exact.
\end{theorem}

\begin{proposition}[Exactness of inverse image functors]
Let $f : Y \to X$ be a morphism of schemes. Then the \CrefAndHyperrefIfExist{definition:inverse_image_and_direct_image_of_sheaves_on_schemes_for_various_topologies}{inverse image functor}
$$ f^{-1} : \text{Sh}(X_{\text{ét}}, \text{Ab}) \to \text{Sh}(Y_{\text{ét}}, \text{Ab}) $$
on sheaves of abelian groups on the small étale sites is \CrefAndHyperrefIfExist{definition:exact_functor_between_abelian_categories}{exact}.
\end{proposition}


\begin{proposition}[Left exactness and exactness criteria for direct image functors] 
Let $f : Y \to X$ be a morphism of schemes. The \CrefAndHyperrefIfExist{definition:inverse_image_and_direct_image_of_sheaves_on_schemes_for_various_topologies}{direct image functor} $$ f_* : \text{Sh}(Y_{\text{ét}}, \text{Ab}) \to \text{Sh}(X_{\text{ét}}, \text{Ab}) $$ is \CrefAndHyperrefIfExist{definition:exact_functor_between_abelian_categories}{left exact} in general. 

\TODO{pin down precise conditions under which $f_*$ is exact.}
Moreover, if $f$ is an étale morphism, or more generally if $f$ is a morphism satisfying certain finiteness and flatness conditions (e.g., finite étale), then $f_*$ is exact.  In particular, if $f$ is an \CrefAndHyperrefIfExist{definition:open_and_closed_immersions_of_schemes}{open immersion} or \CrefAndHyperrefIfExist{definition:etale_morphism_of_schemes}{étale morphism}, then $f_*$ is exact. 
\end{proposition}



\subsection{The functors on derived categories of sheaves}

\TODO{Take the derived functors here and incorporate them in the definition of things for $D_c^b(X,R)$.}
\begin{definition}[Torsion sheaf of abelian groups] \label{definition:torsion_sheaf_of_abelian_groups_on_a_site}
    \TODO{must the site have a small topological generating family?}
Let $(\mathcal{C}, J)$ be a \CrefAndHyperrefIfExist{definition:grothendieck_topology_on_a_category_site_covering_sieve_topologically_generating_family}{site} with a small \CrefAndHyperrefIfExist{definition:grothendieck_topology_on_a_category_site_covering_sieve_topologically_generating_family}{topological generating family} (or a $U$-small topologically generating family if a \CrefAndHyperrefIfExist{definition:grothendieck_universe}{universe} $U$ is available). A \CrefAndHyperrefIfExist{definition:sheaf_cohomology_group_of_a_sheaf_of_modules_over_a_sheaf_of_rings_on_a_site}{sheaf} of abelian groups $\mathcal{F}$ on $(\mathcal{C}, J)$ is called a \hldef{torsion sheaf of abelian groups} if for every object $U$ in $\mathcal{C}$ and every section $s \in \mathcal{F}(U)$, there exists a nonzero integer $n$ such that $n s = 0$ in $\mathcal{F}(U)$.
\end{definition}

\begin{notation}
    Let $X/S$ be a scheme over a base scheme $S$, and choose some \CrefAndHyperrefIfExist{definition:big_site_on_the_category_of_schemes_over_a_scheme_and_small_site}{small site} $C(X)$ on $X$ associated to a big site on $(\Sch/S)$. It is standard to write the following:
    \begin{itemize}
        \item \hl{$D(X)$} for the \CrefAndHyperrefIfExist{definition:derived_category_of_an_abelian_category}{derived category} of the category of sheaves of abelian groups on $C(X)$.
        \item \hl{$D_{\tor}(X)$} or\hl{$D_{\mathrm{tors}}(X)$} for the full subcategory of $D(X)$ of complexes whose \CrefAndHyperrefIfExist{definition:homology_and_cohomology_objects_for_a_chain_complex_in_an_additive_category}{cohomology objects} are \CrefAndHyperrefIfExist{definition:torsion_sheaf_of_abelian_groups_on_a_site}{torsion sheaves} 
        \TODO{continue}
        \item \TODO{continue notation}
    \end{itemize}
\end{notation}

\begin{definition}{e.g. see {\cite[6.3]{fu_ect}}}
    Let $f: X \to Y$ be a morphism of schemes. \TODO{define $Rf_*: D^+(X) \to D^+(Y)$, $f^*$}
\end{definition}

\TODO{state adjunction}

\begin{definition}{e.g. see \TODO{}}
    Let $f: X \to Y$ be an $S$-compactifiable morphism of schemes. \TODO{define $Rf_!: D^+(X) \to D^+(Y)$, $f^!$}
\end{definition}

\TODO{state adjunction}

\begin{definition}{e.g. see \TODO{}}
    Let $X/S$ be a scheme over a base scheme. \TODO{define $Ext$, sheaf Ext, $Tor$} 
\end{definition}

\TODO{state adjunction}

\section{Cohomology of literal sheaves for the small \'etale site on a scheme}

In this section, we define the \'etale cohomology of a \emph{literal} \CrefAndHyperrefIfExist{definition:sheaf_on_a_site}{sheaf} of abelian groups; we emphasize that the sheaves here as \emph{literal} because \CrefAndHyperrefIfExist{definition:lambda_adic_sheaf_on_the_small_etale_site_of_a_scheme_for_an_ideal_of_a_commutative_ring}{$\lambda$-adic sheaves} as defined later are not actual sheaves but rather projective systems of them.

\subsection{Sheaf cohomology of a general module over a sheaf of rings on a site}

\begin{definition} \label{definition:sections_of_a_presheaf_on_a_category_valued_in_a_category}
Let $\mathcal{C}$ be a \CrefAndHyperrefIfExist{definition:category}{(large) category}, and let $\mathcal{D}$ be a \CrefAndHyperrefIfExist{definition:category}{(large) category}. Let $\mathcal{F} : \mathcal{C}^{op} \to \mathcal{D}$ be a \CrefAndHyperrefIfExist{definition:presheaf_on_a_category}{presheaf valued in $\mathcal{D}$}.

\begin{enumerate}
    \item For an object $U \in \mathcal{C}$, the \hldef{sections functor evaluated at $U$} is the functor
    $$\hlin{\Gamma(U, -) : \mathrm{PSh}(\mathcal{C}, \mathcal{D}) \to \mathcal{D}}$$
    defined by
    $$\Gamma(U, \mathcal{F}) := \mathcal{F}(U),$$
    i.e., the value of the presheaf $\mathcal{F}$ at the object $U$.

    \item The \hldef{global sections of $\calF$} is the object \hl{$\Gamma(\calF)$} of $\calD$ defined as the \CrefAndHyperrefIfExist{definition:limit_and_colimit_of_a_diagram_in_a_category}{limit}
    $$\Gamma(\calF) = \varprojlim_{U \in \calC^{\op}} \calF(U)$$
    assuming that such a limit exists, where the limit is taken over objects $U \in \calC$ and the restriction morphisms $\calF(V) \to \calF(U)$ in $\calD$ for  morphisms $U \to V$ in $\calC$. 
    
    If a \CrefAndHyperrefIfExist{definition:initial_final_zero_objects_of_a_category}{final object} $\ast \in \mathcal{C}$ exists, then $\Gamma(\calF)$ exists and coincides with $\Gamma(\ast, \calF) = \calF(\ast)$. The construction $\Gamma(\calF)$ is functorial; in particular, if $\Gamma(\calF)$ exists for all $\calF$ in $\mathrm{PSh}(\calC, \calD)$, e.g. if \CrefAndHyperrefIfExist{definition:limit_and_colimit_of_a_diagram_in_a_category}{limits of} diagrams in $\calD$ indexed by $\calC$ exist, then $\Gamma$ is a functor 
    $$\hlin{\Gamma : \mathrm{PSh}(\mathcal{C}, \mathcal{D}) \to \mathcal{D}}$$
    called the \hldef{global sections functor on $\mathrm{PSh}(\mathcal{C}, \mathcal{D})$}.
\end{enumerate}
\end{definition}


\begin{proposition} \label{proposition:sections_functors_on_presheaves_vlaued_in_an_abelian_category_are_left_exact}
Let $\calC$ be an \CrefAndHyperrefIfExist{definition:essentially_small_category}{essentially small category} and let $\mathcal{A}$ be an \CrefAndHyperrefIfExist{definition:abelian_category}{abelian category}. 
\begin{enumerate}
    \item Let $U \in \Ob(\calC)$ be some fixed object. The \CrefAndHyperrefIfExist{definition:sections_of_a_presheaf_on_a_category_valued_in_a_category}{sections functor}
    $$\Gamma(U, -) : \mathrm{PSh}(\mathcal{C}, \mathcal{A}) \to \mathcal{A}$$ 
    is \CrefAndHyperrefIfExist{definition:exact_functor_between_abelian_categories}{left exact}. \TODO{state that presheaves and sheaves valued in an abelian category form abelian categories}

    \item Assume that $\Gamma(\calF)$ exists for all $\calF$ in $\mathrm{PSh}(\calC, \calA)$ so that $\Gamma$ is a functor 
    $$\Gamma: \mathrm{PSh}(\mathcal{C}, \calA) \to \calA.$$
    The functor $\Gamma$ is left exact.

\end{enumerate}
\end{proposition}
\begin{proof}
    Recall that $\mathrm{PSh}(\mathcal{C}, \mathcal{D})$ is an abelian category (\Cref{proposition:examples_of_abelian_categories}) since $\calC$ is essentially small.  \TODO{Talk about how limits are left exact and }
\end{proof}




\begin{definition} \label{definition:module_over_a_sheaf_of_rings_on_a_site}

    \begin{enumerate}
        \item 
        Let $\mathcal{C}$ be a \CrefAndHyperrefIfExist{definition:grothendieck_topology_on_a_category_site_covering_sieve_topologically_generating_family}{site}, and let $\mathcal{A}$ and $\mathcal{B}$ be \CrefAndHyperrefIfExist{definition:sheaf_on_a_site}{sheaves} of (not necessarily commutative) \CrefAndHyperrefIfExist{definition:ring}{rings} on $\mathcal{C}$. 
        
        \begin{enumerate}
            \item 
            An \hldef{$(\mathcal{A}, \mathcal{B})$-bimodule} (or a \hldef{bimodule over $(\mathcal{A}, \mathcal{B})$}) is a \CrefAndHyperrefIfExist{definition:sheaf_on_a_site}{sheaf} $\mathcal{M}$ of abelian groups on $\mathcal{C}$ equipped with a left $\mathcal{A}$-module structure given by a \CrefAndHyperrefIfExist{definition:sheaf_on_a_site}{morphism of sheaves} of sets
            $$ \lambda: \mathcal{A} \times \mathcal{M} \longrightarrow \mathcal{M}, $$
            and a right $\mathcal{B}$-module structure given by a morphism of sheaves of sets
            $$ \rho: \mathcal{M} \times \mathcal{B} \longrightarrow \mathcal{M}, $$
            such that the actions are compatible. Specifically, for every object $U$ in $\mathcal{C}$, every section $m \in \mathcal{M}(U)$, every $a \in \mathcal{A}(U)$, and every $b \in \mathcal{B}(U)$, the equality
            $$ \lambda_U(a, \rho_U(m, b)) = \rho_U(\lambda_U(a, m), b) $$
            holds in $\mathcal{M}(U)$. In standard multiplicative notation where $\lambda(a,m)$ is denoted $a \cdot m$ and $\rho(m,b)$ is denoted $m \cdot b$, this condition is the associativity axiom
            $$ (a \cdot m) \cdot b = a \cdot (m \cdot b). $$

            In particular, for every object $U \in \calC$, the abelian group $\calM(U)$ has the structure of an \CrefAndHyperrefIfExist{definition:module_of_a_ring}{$\calA(U)-\calB(U)$-bimodule}.

            \item Let $\mathcal{M}$ and $\mathcal{N}$ be $(\mathcal{A}, \mathcal{B})$-bimodules. A \hldef{homomorphism of $(\mathcal{A}, \mathcal{B})$-bimodules} (or an \hldef{$(\mathcal{A}, \mathcal{B})$-linear morphism}) is a morphism of sheaves of abelian groups $f: \mathcal{M} \to \mathcal{N}$ such that for every object $U$ of $\mathcal{C}$, every section $m \in \mathcal{M}(U)$, every $a \in \mathcal{A}(U)$, and every $b \in \mathcal{B}(U)$, the following compatibility conditions hold:
            $$ f_U(a \cdot m) = a \cdot f_U(m) \quad \text{and} \quad f_U(m \cdot b) = f_U(m) \cdot b. $$


        \end{enumerate}

        \noindent We denote the category of $(\mathcal{A}, \mathcal{B})$-bimodules, with morphisms being morphisms of sheaves of abelian groups compatible with both the left $\mathcal{A}$-action and the right $\mathcal{B}$-action, by
        \hl{$ \mathcal{A}\text{-}\mathcal{B}\text{-}\mathsf{Mod} $}
        or sometimes by
        \hl{$ {}_{\mathcal{A}}\mathsf{Mod}_{\mathcal{B}} $}
        \TODO{talk about how bimodules can be identifies with left/right modules}

        \item 

        Let $(\mathcal{C}, J)$ be a \CrefAndHyperrefIfExist{definition:grothendieck_topology_on_a_category_site_covering_sieve_topologically_generating_family}{site}. Let $\mathcal{O}$ be a \CrefAndHyperrefIfExist{definition:sheaf_on_a_site}{sheaf of (not necessarily commutative) rings on $(\mathcal{C}, J)$}, i.e. $((\calC, J), \calO)$ is a \CrefAndHyperrefIfExist{definition:ringed_site}{ringed site}.  

        \begin{enumerate}
            \item An \hldef{(left/right/two-sided) $\mathcal{O}$-module} consists of the following data:
            \begin{itemize}
                \item A sheaf $\mathcal{F}$ of abelian groups on $(\mathcal{C}, J)$,
            \item for every object $U \in \mathcal{C}$, the structure of an (left/right/two-sided) $\mathcal{O}(U)$-module on $\mathcal{F}(U)$,
            \end{itemize}
            such that for every morphism $f: V \to U$ in $\mathcal{C}$, the restriction map 
            $$\rho_{U,V}: \mathcal{F}(U) \to \mathcal{F}(V)$$ 
            is $\mathcal{O}(U)$-linear when the $\mathcal{O}(U)$-action on $\mathcal{F}(V)$ is defined via the natural ring homomorphism 
            $$\mathcal{O}(U) \to \mathcal{O}(V)$$
            induced by $f$.


            \item Let $\mathcal{F}$ and $\mathcal{G}$ be \CrefAndHyperrefIfExist{definition:module_over_a_sheaf_of_rings_on_a_site}{$\mathcal{O}$-modules}.

            A \hldef{morphism of $\mathcal{O}$-modules} $\varphi: \mathcal{F} \to \mathcal{G}$ is a \CrefAndHyperrefIfExist{definition:sheaf_on_a_site}{morphism of sheaves} of abelian groups such that, for every object $U \in \mathcal{C}$, the component map
            $$\varphi_U : \mathcal{F}(U) \to \mathcal{G}(U)$$
            is $\mathcal{O}(U)$-linear, i.e. it satisfies
            $$\varphi_U(r \cdot s) = r \cdot \varphi_U(s) \quad \text{for all } r \in \mathcal{O}(U), \, s \in \mathcal{F}(U).$$

            The collection of all $\mathcal{O}$-modules together with their morphisms of $\mathcal{O}$-modules forms the \hldef{category of $\mathcal{O}$-modules}, denoted \hl{$\mathbf{Mod}(\mathcal{O})$}.

            \TextIfExists{definition:algebra_over_a_sheaf_of_rings_on_a_site}{See also \Cref{definition:algebra_over_a_sheaf_of_rings_on_a_site}.}
        \end{enumerate}

        \noindent In case that a \CrefAndHyperrefIfExist{definition:sheafification_functor_on_a_site}{sheafification functor} 
        $$\PreShv(\calC, \mathbf{Rings}) \to \Shv(\calC, \mathbf{Rings})$$ 
        exists, a left, right, two-sided $\calO$-module (and morphisms thereof) is equivalent to a $(\calO,\bbZ)$-bimodule, $(\bbZ,\calO)$-bimodule, and $(\calO, \calO)$-bimodule (and morphisms thereof) respectively, where $\bbZ$ is the \CrefAndHyperrefIfExist{definition:constant_sheaf_on_a_site_with_sheafification}{constant sheaf} of the integer ring $\bbZ$.

\end{enumerate}


\end{definition}


% See Also
% theorem:category_of_modules_over_a_sheaf_of_rings_on_a_site_on_an_essentially_small_category_has_enough_injectives



\begin{definition}[Tensor product of sheaves of bimodules] \label{definition:tensor_product_of_sheaves_of_bimodules_over_sheaves_of_rings_on_a_site}
Let $\mathcal{C}$ be a \CrefAndHyperrefIfExist{definition:grothendieck_topology_on_a_category_site_covering_sieve_topologically_generating_family}{site}. Let $\mathcal{R}, \mathcal{S}, \mathcal{T}$ be (not necessarily commutative) \CrefAndHyperrefIfExist{definition:sheaf_on_a_site}{sheaves} of \CrefAndHyperrefIfExist{definition:ring}{rings} on $\mathcal{C}$. Let $\mathcal{M}$ be an \CrefAndHyperrefIfExist{definition:module_over_a_sheaf_of_rings_on_a_site}{$(\mathcal{R}, \mathcal{S})$-bimodule}, and let $\mathcal{N}$ be an $(\mathcal{S}, \mathcal{T})$-bimodule.

\begin{enumerate}
    \item Define the \hldef{presheaf tensor product}
    \hl{$\mathcal{M} \otimes_{p,\mathcal{S}} \mathcal{N}$} to be the \CrefAndHyperrefIfExist{definition:presheaf_on_a_category}{presheaf} of abelian groups given by
    $$ U \longmapsto \mathcal{M}(U) \otimes_{\mathcal{S}(U)} \mathcal{N}(U).$$
    \CrefIfExists{definition:tensor_product_of_bimodules_of_rings} As a presheaf, it has the structure of a $\calR-\calT$-bimodule \TODO{bimodule for a presheaf}.

    \item Assume that a \CrefAndHyperrefIfExist{definition:sheafification_functor_on_a_site}{sheafification functor} 
    $$\Shv(\calC, \Sets) \to \PreShv(\calC, \Sets)$$
    exists.  The \hldef{tensor product of sheaves $\mathcal{M}$ and $\mathcal{N}$}, denoted \hl{$\mathcal{M} \otimes_{\mathcal{S}} \mathcal{N}$}, is defined to be the \CrefAndHyperrefIfExist{definition:sheafification}{sheafification} of the presheaf $\mathcal{M} \otimes_{p,\mathcal{S}} \mathcal{N}$. This tensor product becomes naturally an $(\mathcal{R}, \mathcal{T})$-bimodule.

\end{enumerate}

The left $\mathcal{R}$-action and right $\mathcal{T}$-action are induced by the presheaf actions defined section-wise by
\begin{align*}
r \cdot (m \otimes n) &= (r \cdot m) \otimes n, \\
(m \otimes n) \cdot t &= m \otimes (n \cdot t),
\end{align*}
for all $r \in \mathcal{R}(U)$, $t \in \mathcal{T}(U)$, $m \in \mathcal{M}(U)$, and $n \in \mathcal{N}(U)$. The sheafification functor preserves these actions, yielding the required bimodule structure on $\mathcal{M} \otimes_{\mathcal{S}} \mathcal{N}$.


Inductively, given sheaves of rings $\mathcal{R}_0, \ldots, \mathcal{R}_k$ and $(\mathcal{R}_{i-1}, \mathcal{R}_i)$-bimodules $\mathcal{M}_i$ for $i = 1, \ldots, k$, we may speak of the tensor product
$$ \hlin{\mathcal{M}_0 \otimes_{\mathcal{R}_1} \mathcal{M}_1 \otimes_{\mathcal{R}_2} \cdots \otimes_{\mathcal{R}_{k-1}} \mathcal{M}_k. }$$
Tensor products of sheaves are associative up to canonical isomorphism\TODO{}, allowing us to omit parentheses. This iterated tensor product carries a natural $(\mathcal{R}_0, \mathcal{R}_k)$-bimodule structure.

\TextIfExists{definition:n_ary_additive_functor_between_additive_categories}{In general, the assignment $(\mathcal{M}_0, \ldots, \mathcal{M}_k) \mapsto \mathcal{M}_0 \otimes_{\mathcal{R}_1} \cdots \otimes_{\mathcal{R}_{k-1}} \mathcal{M}_k$ defines a \CrefAndHyperrefIfExist{definition:n_ary_additive_functor_between_additive_categories}{$(k+1)$-ary additive functor}
$$ {}_{\mathcal{R}_0}\mathbf{Mod}_{\mathcal{R}_1} \times \cdots \times {}_{\mathcal{R}_{k-1}}\mathbf{Mod}_{\mathcal{R}_k} \to {}_{\mathcal{R}_0} \mathbf{Mod}_{\mathcal{R}_k} $$
()\Cref{proposition:examples_of_abelian_categories}).}

Given a sheaf of rings $\mathcal{R}$ and a two-sided $\mathcal{R}$-module $\mathcal{M}$, we may also speak of the \hldef{$n$-fold tensor product} \hl{$\mathcal{M}^{\otimes n} = \mathcal{M}^{\otimes_{\mathcal{R}} n}$}.
\end{definition}



\begin{definition} \label{definition:subsheaf_of_a_sheaf_on_a_site}
Let $(\mathcal{C}, J)$ be a \CrefAndHyperrefIfExist{definition:sheaf_on_a_site}{site}, and let $\mathcal{A}$ be a category in which \CrefAndHyperrefIfExist{definition:sheaf_on_a_site}{sheaves} on $(\mathcal{C}, J)$ can be defined.

For a sheaf $\mathcal{F} \in \operatorname{Sh}(\mathcal{C}, J; \mathcal{A})$, a \hldef{subsheaf $\mathcal{G}$ of $\mathcal{F}$} is a sheaf equipped with a \CrefAndHyperrefIfExist{definition:monomorphism_and_epimorphism_in_categories}{monomorphism} of sheaves
$$\iota : \mathcal{G} \hookrightarrow \mathcal{F}$$
in $\operatorname{Sh}(\mathcal{C}, J; \mathcal{A})$.

Concretely, for every object $U \in \mathcal{C}$, the morphism
\[ \iota(U) : \mathcal{G}(U) \to \mathcal{F}(U) \]
is a monomorphism in $\mathcal{A}$, and the collection $\{ \iota(U) \}_U$ is compatible with restriction morphisms so that $\mathcal{G}$ is a subsheaf in the usual sense.
\end{definition}


\begin{definition} \label{definition:sheaf_of_ideals_for_a_sheaf_of_rings_on_a_site}
Let $(\mathcal{C}, J)$ be a \CrefAndHyperrefIfExist{definition:grothendieck_topology_on_a_category_site_covering_sieve_topologically_generating_family}{site}, and let $\mathcal{O}$ be a \CrefAndHyperrefIfExist{definition:sheaf_on_a_site}{sheaf of (not necessarily comutative) rings} on $(\mathcal{C}, J)$ with values in a category $\mathcal{A}$ for which sheaves are defined.

A \hldef{sheaf of (left/right/two-sided) ideals} $\mathcal{I}$ of the sheaf of rings $\mathcal{O}$ is a \CrefAndHyperrefIfExist{definition:subsheaf_of_a_sheaf_on_a_site}{subsheaf of $\mathcal{O}$} in the category of sheaves of (left/right/two-sided) $\mathcal{O}$-modules, i.e., a sheaf of $\mathcal{O}$-modules $\mathcal{I}$ together with a \CrefAndHyperrefIfExist{definition:monomorphism_and_epimorphism_in_categories}{monomorphism} of sheaves of $\mathcal{O}$-modules $\iota : \mathcal{I} \hookrightarrow \mathcal{O}$
such that for every object $U$ in $\mathcal{C}$, the morphism $\iota(U) : \mathcal{I}(U) \to \mathcal{O}(U)$ identifies $\mathcal{I}(U)$ as an ideal of the ring $\mathcal{O}(U)$.

Equivalently, a sheaf of ideals $\calI$ is a \CrefAndHyperrefIfExist{definition:subsheaf_of_a_sheaf_on_a_site}{subsheaf} of modules of $\calO$ as a \CrefAndHyperrefIfExist{definition:module_over_a_sheaf_of_rings_on_a_site}{sheaf of modules} of $\calO$
\end{definition}



\begin{theorem}[e.g. see {\cite[Tag 01DU]{stacks-project}}] \label{theorem:category_of_modules_over_a_sheaf_of_rings_on_a_site_on_an_essentially_small_category_has_enough_injectives}
    For any \CrefAndHyperrefIfExist{definition:grothendieck_topology_on_a_category_site_covering_sieve_topologically_generating_family}{site} $(\calC, J)$ on an \CrefAndHyperrefIfExist{definition:essentially_small_category}{essentially small category} $\mathcal{C}$ and a \CrefAndHyperrefIfExist{definition:sheaf_on_a_site}{sheaf of rings} $\calO$ on $\calC$, the category $\mathbf{Mod}(\mathcal{O})$ of \CrefAndHyperrefIfExist{definition:module_over_a_sheaf_of_rings_on_a_site}{$\calO$-modules} is an abelian category that \CrefAndHyperrefIfExist{definition:has_enough_injectives_or_projectives_for_an_abelian_category}{has enough injectives}. In fact, there is a functorial injective embedding \TODO{ what does this mean?}
\end{theorem}

We can describe the sections of a sheaf over an object by the direct image from the object to the final object, if it exists.

\begin{theorem} \label{theorem:sections_of_a_sheaf_on_a_site_with_a_final_object_coincides_with_direct_image_from_object_to_final_object}
Let $(\mathcal{C}, J)$ be a \CrefAndHyperrefIfExist{definition:sheaf_on_a_site}{site} with \CrefAndHyperrefIfExist{definition:initial_final_zero_objects_of_a_category}{final object $x$}, and let $F$ be a \CrefAndHyperrefIfExist{definition:sheaf_on_a_site}{sheaf on $(\mathcal{C}, J)$}. Let $f : X \to x$ be any object $X$ over $x$ (regarded as a morphism in $\mathcal{C}$), and let $f_* F$ denote the \CrefAndHyperrefIfExist{definition:inverse_image_of_a_sheaf_under_a_continuous_functor_of_sites_or_a_site_morphism}{direct image (pushforward) of $F$ along $f$}. Then
\[ (f_* F)(x) = F(X) = \Gamma(X, F).  \]
\CrefIfExists{definition:sections_of_a_presheaf_on_a_category_valued_in_a_category}
% In particular, the global sections of $F$ over $X$ are equal to the pushforward of $F$ under $f$, evaluated at the final object.
\end{theorem}


\begin{corollary}
    \TODO{the sites}
    Let $f: X \to S$ be a scheme over a scheme $S$ and let $\mathbf{C}(X)$ and $\mathbf{C}(S)$ be small sites associated to $X$ and $S$ respectively, equipped with any common Grothendieck topologies such as Zariski, étale, Nisnevich, fppf, fpqc, or crystalline. Let $\calA$ be a (large) category and let \CrefAndHyperrefIfExist{definition:sheaf_on_a_site}{$\calF \in \operatorname{Sh}(\mathbf{C}(X), \calA)$}. We have a natural isomorphism
    $$\Gamma(\calF) \cong (f_* \calF)(S)$$ 
    \CrefIfExists{definition:inverse_image_and_direct_image_of_sheaves_on_schemes_for_various_topologies}
    \TODO{natural in both $\calF$ and $S$ perhaps?}
\end{corollary}


\begin{definition} \label{definition:sheaf_cohomology_group_of_a_sheaf_of_modules_over_a_sheaf_of_rings_on_a_site}

Let $(\mathcal{C}, J)$ be a \CrefAndHyperrefIfExist{definition:grothendieck_topology_on_a_category_site_covering_sieve_topologically_generating_family}{site} on a \CrefAndHyperrefIfExist{definition:locally_small_category}{locally small category} or a $U$-site for some \CrefAndHyperrefIfExist{definition:grothendieck_universe}{universe} $U$. Let $\calO$ be a \CrefAndHyperrefIfExist{definition:sheaf_on_a_site}{sheaf of rings} on $\calC$, so that $(\calC, J, \calO)$ is a \CrefAndHyperrefIfExist{definition:ringed_site}{ringed site}. Recall that the category \CrefAndHyperrefIfExist{definition:module_over_a_sheaf_of_rings_on_a_site}{$\mathbf{Mod}(\mathcal{O})$} of $\calO$-modules is abelian and \CrefAndHyperrefIfExist{definition:has_enough_injectives_or_projectives_for_an_abelian_category}{has enough injectives} (\Cref{theorem:category_of_modules_over_a_sheaf_of_rings_on_a_site_on_an_essentially_small_category_has_enough_injectives}).

Assume that \CrefAndHyperrefIfExist{definition:sections_of_a_presheaf_on_a_category_valued_in_a_category}{global sections objects $\Gamma(\calG)$} exist for all objects $\calG$ of $\mathrm{Sh}(\mathcal{C}, \mathbf{Ab})$\footnote{for example, this occurs when $\calC$ is \CrefAndHyperrefIfExist{definition:essentially_small_category}{essentially small}} so that $\Gamma$ is a functor
$$\Sh(\calC, \mathbf{Ab}) \to \mathbf{Ab},$$
which is a \CrefAndHyperrefIfExist{definition:exact_functor_between_abelian_categories}{left exact functor} (\Cref{proposition:sections_functors_on_presheaves_vlaued_in_an_abelian_category_are_left_exact}).  Note that $\Gamma$ restricts to a left exact functor 
$$\mathbf{Mod}(\mathcal{O}) \to \mathbf{Ab}.$$
If $\calC$ has a \CrefAndHyperrefIfExist{definition:initial_final_zero_objects_of_a_category}{final object} $\ast$ as well, then recall that $\Gamma(\calF) = \calF(\ast)$. 

Let $\calF$ be an object of $\mathbf{Mod}(\mathcal{O})$. 
\begin{enumerate}
    \item For each integer $n \geq 0$, the \hldef{$n$-th (abelian) (global) sheaf cohomology group of $\mathcal{F}$} is
    $$\hlin{H^n(\mathcal{C}, J; \mathcal{F}) := R^n \Gamma(\mathcal{F}),}$$
    where $R^n \Gamma$ is the $n$-th \CrefAndHyperrefIfExist{definition:left_right_derived_functors_of_a_right_left_exact_functor_between_abelian_categories_where_source_has_enough_projectives_injectives}{right derived functor} of the \CrefAndHyperrefIfExist{definition:sections_of_a_presheaf_on_a_category_valued_in_a_category}{global sections functor $\Gamma$}.

    In particular, each $H^n$ is a functor 
    $$H^n: \mathbf{Mod}(\mathcal{O})  \to \mathbf{Ab}.$$

    \item Given an object $U \in \calC$ and for each integer $n \geq 0$, the \hldef{$n$-th (abelian) sheaf cohomology group of $\mathcal{F}$ of sections at $U$} is
    $$\hlin{H^n(U, \mathcal{F}) := (R^n \Gamma(U,-))(\mathcal{F}),}$$
    where $R^n \Gamma(U,-)$ is the $n$-th \CrefAndHyperrefIfExist{definition:left_right_derived_functors_of_a_right_left_exact_functor_between_abelian_categories_where_source_has_enough_projectives_injectives}{right derived functor} of the \CrefAndHyperrefIfExist{definition:sections_of_a_presheaf_on_a_category_valued_in_a_category}{sections functor $\Gamma(U,-)$ evaluated at $U$}.

    In particular, $H^n(U,\calF)$ can be regarded as the $n$th global sheaf cohomology group of the \CrefAndHyperrefIfExist{definition:restriction_of_a_sheaf_on_a_site_to_an_object_of_the_underlying_category_of_the_site}{restriction $\calF|_U$} of $\calF$ to $U$.
    % to the 
    % \CrefAndHyperrefIfExist{definition:site_induced_by_a_site_on_an_over_category}{site induced by $(\calC, J)$} on the \CrefAndHyperrefIfExist{definition:category_of_objects_over_under_a_fixed_object_in_a_category}{over category $\calC_{/U}$}.

\end{enumerate}

% In the case that $\calA = R\mathbf{-mod}$, the category of (left/right/two-sided)modules over some fixed (not necessarily commutative) ring $R$, recall that $R\mathbf{-Mod}$ is complete (and cocomplete) \TODO{Is it precisely completeness that I need or cocompleteness? In other words, for the limits defining $\Gamma$, are those limits projective limits or colimtis?} \TODO{talk about how $R$-modules are complete and cocomplete}, so all global sections objects $\Gamma(\calG)$ exist. \TODO{continue talking about this context of modules}

% The \hldef{sheaf cohomology groups of $\mathcal{F}$} on the site $(\mathcal{C}, J)$ are defined as the \CrefAndHyperrefIfExist{definition:left_right_derived_functors_of_a_right_left_exact_functor_between_abelian_categories_where_source_has_enough_projectives_injectives}{right derived functors} of the global sections functor
% $$\Gamma : \mathrm{Sh}(\mathcal{C}, J) \to \mathbf{Ab}$$
% where $\mathrm{Sh}(\mathcal{C}, J)$ \CrefAndHyperrefIfExist{definition:sheaf_on_a_site}{denotes} the category of sheaves of abelian groups on $(\mathcal{C}, J)$, and $\ast$ denotes the \CrefAndHyperrefIfExist{definition:initial_final_zero_objects_of_a_category}{final object} in $\mathcal{C}$ if it exists.

% More precisely, 

% If $\mathcal{C}$ has no final object, $H^n(\mathcal{C}, J; \mathcal{F})$ is defined by choosing an injective resolution of $\mathcal{F}$ and taking cohomology of the resulting complex obtained by applying $\Gamma$.

% These groups measure the extent to which global sections fail to be exact, and generalize classical sheaf cohomology defined on topological spaces to arbitrary sites.
\end{definition}


\subsection{\'Etale cohomology of literal sheaves}

\begin{definition} \label{definition:etale_cohomology_of_a_sheaf_of_abelian_groups_on_the_small_etale_site_of_a_schem}
Let $X$ be a \CrefAndHyperrefIfExist{scheme}{scheme}, let $\calO$ be a \CrefAndHyperrefIfExist{definition:sheaf_on_a_site}{sheaf of rings} on \CrefAndHyperrefIfExist{definition:small_etale_site_of_a_scheme}{$X_{\et}$} and let $\mathcal{F}$ be an \CrefAndHyperrefIfExist{definition:module_over_a_sheaf_of_rings_on_a_site}{$\calO$-module}, which is sheaf of abelian groups on the \CrefAndHyperrefIfExist{definition:small_etale_site_of_a_scheme}{small étale site $X_{\mathrm{\acute{e}t}}$}. Note that the category of $\calO$-modules \CrefAndHyperrefIfExist{definition:has_enough_injectives_or_projectives_for_an_abelian_category}{has enough injectives} (\Cref{theorem:category_of_modules_over_a_sheaf_of_rings_on_a_site_on_an_essentially_small_category_has_enough_injectives})

The \hldef{étale cohomology groups of $X$ with coefficients in $\mathcal{F}$} are defined as the \CrefAndHyperrefIfExist{definition:left_right_derived_functors_of_a_right_left_exact_functor_between_abelian_categories_where_source_has_enough_projectives_injectives}{right derived functors} of the \CrefAndHyperrefIfExist{definition:sections_of_a_presheaf_on_a_category_valued_in_a_category}{global sections functor}, which is \CrefAndHyperrefIfExist{definition:exact_functor_between_abelian_categories}{left exact} (\Cref{proposition:sections_functors_on_presheaves_vlaued_in_an_abelian_category_are_left_exact}). Explicitly, they are the abelian groups
$$H^i_{\mathrm{\acute{e}t}}(X, \mathcal{F}) \coloneq H^i(X_{\et}, \calF) = R^i \Gamma(X, \calF), \quad i \geq 0.$$
\TextIfExists{definition:sheaf_cohomology_group_of_a_sheaf_of_modules_over_a_sheaf_of_rings_on_a_site}{Equivalently, $H^i_{\mathrm{\acute{e}t}}(X, \mathcal{F})$ is defined as the \CrefAndHyperrefIfExist{definition:sheaf_cohomology_group_of_a_sheaf_of_modules_over_a_sheaf_of_rings_on_a_site}{sheaf cohomology groups of the sheaf $\calF$ on the essentially small site $X_{\et}$}}
\end{definition}

\begin{definition} \label{definition:etale_cohomology_group_of_a_scheme_with_coefficients_in_an_abelian_group}
Let $X$ be a scheme. Let $A$ be an abelian group and write $\underline{A}$ for the \CrefAndHyperrefIfExist{definition:constant_sheaf_on_a_site_with_sheafification}{constant sheaf} on \CrefAndHyperrefIfExist{definition:small_etale_site_of_a_scheme}{$X_{\mathrm{\acute{e}t}}$} associated to $A$. In this case, the \hldef{étale cohomology groups of $X$ with coefficients in $A$} are defined as 
$$\hlin{H^i_{\mathrm{\acute{e}t}}(X, A) := H^i_{\mathrm{\acute{e}t}}(X, \underline{A}).}$$
(\Cref{definition:etale_cohomology_of_a_sheaf_of_abelian_groups_on_the_small_etale_site_of_a_schem}); here $\underline{A}$ is regarded as a module of the constant sheaf $\bbZ$ on $X_{\et}$. 
\end{definition}



\section{Derived categories of \texorpdfstring{$\lambda$}{lambad}-adic sheaves constructible through limits of categories}


\TODO{Give an outline of how $D_c^b(X, \Qellbar)$ is defined}
Classically, given a noetherian scheme $X$ and a prime $\ell$ invertible on $X$, the category $D_c^b(X, \Qellbar)$, which is intuitively treated as a cohomologically bounded derived category of constructible complexes of sheaves with $\Qellbar$ coefficients, is constructed via the following steps:

\TODO{Cite the statements of the constructions}
\begin{enumerate}
    \item 

    Write $E$ for a finite extension of $\Qell$, and write $R$ for the integral closure of $\bbZ_\ell$ in $E$. Note that $R$ is a complete discrete valuation ring. Write $\lambda$ for a uniformizer of $R$. 
    \item Construct $D_c^b(X,R)$ \TODO{Explain how $D_c^b(X,R)$ is constructed.}.
    \item Define $D_c^b(X,E) = D_c^b(X,R)[\lambda^{-1}]$, i.e. construct $D_c^b(X,E)$ as the localization of $D_c^b(X,R)$ by the maps $\lambda^m: \calF \to \calF$ in $D_c^b(X,R)$.
    \item Observe that for finite extensions $E'/E$, there are functors 
    $$D_c^b(X,E) \to D_c^b(X,E')$$
    \item Construct $D_c^b(X, \Qellbar)$ as the $2$-direct limit of the $D_c^b(X,E)$ as $E$ runs over the finite extensions of $\Qell$ in $\Qellbar$:
    $$D_c^b(X, \Qellbar) = \varinjlim_{\substack{\Qellbar \supset E \supseteq \Qell \\ E/\Qell \text{ fintie}}} D_c^b(X,E)$$
\end{enumerate}


\subsection{\texorpdfstring{$\ell$}{ell}-adic sheaves}

$\ell$-adic sheaves do not refer to literal sheaves on $X_{\et}$ with coefficients in $\ell$-adic rings such as $\bbZ_\ell$ or $\Qell$. In a purely formal sense, one could talk about sheaves on $X_{\et}$ with such coefficients, but the categories of such sheaves are too ill-behaved to be of arithmetic use. 

\subsection{\texorpdfstring{$\ell$}{ell}-adic cohomology}

$\ell$-adic cohomology does not refer to \CrefAndHyperref{definition:etale_cohomology_group_of_a_scheme_with_coefficients_in_an_abelian_group}{\'etale cohomology with coefficients} in the rings $\bbZ_\ell$ or $\bbQ_\ell$ . 

\TODO{define $\ell$-adic sheaves (say with $R$, $E$, and $\Qellbar$-coefficients)}
\TODO{define $\ell$-adic cohomology in $R$-coefficients, $E$-coefficients, and $\Qellbar$-coefficients}

\subsection{The categories of \texorpdfstring{$\lambda$}{lambda}-adic sheaves and A-R \texorpdfstring{$\lambda$}{lambda}-adic sheaves}

Historically, the theory of $\lambda$-adic sheaves (as opposed to ``derived categories''/categories of complexes of $\lambda$-adic sheaves) was first studied in SGA5 \cite[Expos\'e V]{SGA5}. Note that SGA5 \cite[Expos\'e III]{SGA5} did study derived categories of sheaves with \emph{finite} coefficients.

This subsection considers four abelian categories whose objects are inverse systems of sheaves:
\begin{enumerate}
    \item The category of inverse systems of $\lambda$-torsion sheaves (\Cref{definition:lambda_torsion_sheaf_of_modules_over_a_sheaf_of_rings_on_a_site_for_a_sheaf_of_ideals})
    \item The A-R category of inverse systems of $\lambda$-torsion sheaves (\Cref{definition:A_R_category_of_inverse_systems_of_lambda_torsion_sheaves_on_small_etale_site})
    \item The category of $\lambda$-adic sheaves (\Cref{definition:lambda_adic_sheaf_on_the_small_etale_site_of_a_scheme_for_an_ideal_of_a_commutative_ring})
    \item The category of A-R $\lambda$-adic sheaves (\Cref{definition:lambda_adic_sheaf_on_the_small_etale_site_of_a_scheme_for_an_ideal_of_a_commutative_ring})
\end{enumerate}

These categories are related via the following commutative diagram of functors:
\begin{center}
    \begin{tikzcd}[column sep=5em]
        (\text{inverse system of } \lambda \text{-torsion sheaves}) \ar[r, "\text{localization}"] & (\text{A-R category}) \\
        (\lambda\text{-adic sheaves}) \ar[u, hookrightarrow, "\text{full subcategory}"] \ar[r, "\cong"] & (\text{A-R } \lambda\text{-adic sheaves}) \ar[u, "\text{full subcategory}", hookrightarrow]
    \end{tikzcd}
\end{center}

\subsection{Constructible sheaves}



\begin{definition}[Constructible sheaf] \label{definition:constructible_sheaf_on_a_small_site_on_a_scheme_or_a_topological_space}
Let $X$ be either
\begin{enumerate}
    \item A \CrefAndHyperrefIfExist{definition:topological_space}{topological space} (made into a \CrefAndHyperrefIfExist{definition:grothendieck_topology_on_a_category_site_covering_sieve_topologically_generating_family}{site} with the \CrefAndHyperrefIfExist{definition:site_of_opens_on_a_topological_space}{natural Grothendieck topology}) or
    \TODO{This second case needs to be formulated more precisely; in particular, a notion of ``restriction'' is necessary, and there needs to be a more precise way to formulate that the site is a site associated to $X$.}
    \item A \CrefAndHyperrefIfExist{definition:scheme}{scheme} along with a \CrefAndHyperrefIfExist{definition:big_site_on_the_category_of_schemes_over_a_scheme_and_small_site}{small site $(\calC, J)$ on $X$}. Assume that the there is a \CrefAndHyperrefIfExist{definition:sheafification_functor_on_a_site}{sheafification functor} for the presheaves/sheaves valued in abelian groups so that \CrefAndHyperrefIfExist{definition:locally_constant_sheaf_on_a_site_with_sheafification}{locally constant sheaves} of abelian groups are defined. For example, it suffices for the underlying category of the site to be \CrefAndHyperrefIfExist{definition:essentially_small_category}{essentially small}(\Cref{theorem:sheafification_of_a_presheaf_of_sets_on_a_small_enough_site}).
\end{enumerate}

Let $\calO$ be a \CrefAndHyperrefIfExist{definition:sheaf_on_a_site}{sheaf of rings} on the chosen site $(\calC, J)$ associated to $X$. Let $\calF$ be a sheaf of \CrefAndHyperrefIfExist{definition:module_over_a_sheaf_of_rings_on_a_site}{$\calO$-modules} on $X$.

% Let $\mathcal{F}$ be a sheaf of abelian groups (or more generally, of modules over a fixed ring) on $X$.

The sheaf $\mathcal{F}$ is called \hldef{constructible} if there exists a finite partition of $X$ into \CrefAndHyperrefIfExist{definition:locally_closed_subset_of_a_topological_space}{locally closed subsets}
\[
X = \bigsqcup_{\alpha \in A} X_{\alpha}
\]
such that the restriction $\mathcal{F}|_{X_{\alpha}}$ is a \hyperrefIfExists{definition:locally_constant_sheaf_on_a_site_with_sheafification}{locally constant sheaf of $\calO|_{X_\alpha}$-modules of finite type}\CrefIfExists{definition:locally_constant_sheaf_on_a_site_with_sheafification} on the site on $X_{\alpha}$ induced by $(\calC, J)$ and the embedding $X_{\alpha} \hookrightarrow X$ \TODO{The site on $X_{\alpha}$ needs to be elaborated on.} for each $\alpha \in A$.
\end{definition}


\begin{definition}[Full subcategory] \label{definition:full_subcategory_of_a_category}
    Let $\mathcal{C}$ be a \CrefAndHyperrefIfExist{definition:category}{(large) category}. A \hldef{full subcategory} $\mathcal{D}$ of $\mathcal{C}$ is a \CrefAndHyperrefIfExist{definition:subcategory_of_a_category}{subcategory} such that for every pair of objects $X, Y \in \mathrm{Ob}(\mathcal{D})$, the morphism classes coincide:
    $$\mathrm{Hom}_{\mathcal{D}}(X,Y) = \mathrm{Hom}_{\mathcal{C}}(X,Y).$$
    In other words, a full subcategory includes all morphisms between its objects that exist in the ambient category $\mathcal{C}$.
\end{definition}


\begin{definition} \label{definition:derived_category_of_cohomologically_constructible_complexes_of_sheaves_of_modules_of_a_sheaf_of_rings_on_a_topological_space_or_scheme}

    Let $X$ be either
    \begin{enumerate}
        \item A \CrefAndHyperrefIfExist{definition:topological_space}{topological space} (made into a \CrefAndHyperrefIfExist{definition:grothendieck_topology_on_a_category_site_covering_sieve_topologically_generating_family}{site} with the \CrefAndHyperrefIfExist{definition:site_of_opens_on_a_topological_space}{natural Grothendieck topology}) or
        \item A \CrefAndHyperrefIfExist{definition:scheme}{scheme} along with a \CrefAndHyperrefIfExist{definition:big_site_on_the_category_of_schemes_over_a_scheme_and_small_site}{small site $(\calC, J)$ on $X$}. Assume that the there is a \CrefAndHyperrefIfExist{definition:sheafification_functor_on_a_site}{sheafification functor} for the presheaves/sheaves valued in abelian groups so that \CrefAndHyperrefIfExist{definition:locally_constant_sheaf_on_a_site_with_sheafification}{locally constant sheaves} of abelian groups are defined. For example, it suffices for the underlying category of the site to be \CrefAndHyperrefIfExist{definition:essentially_small_category}{essentially small}(\Cref{theorem:sheafification_of_a_presheaf_of_sets_on_a_small_enough_site}).
    \end{enumerate}
    Note that the sites are \CrefAndHyperrefIfExist{definition:essentially_small_category}{essentially small} (see \Cref{theorem:common_small_sites_of_a_scheme_are_essentially_small}).

    The category \hl{$D_c(X, \calO)$} is the \CrefAndHyperrefIfExist{definition:full_subcategory_of_a_category}{full subcategory} of the derived category \CrefAndHyperrefIfExist{notation:derived_category_of_category_of_modules_of_a_ringed_site}{$D(X, \calO)$} of objects $K^\bullet$ whose \CrefAndHyperrefIfExist{definition:homology_and_cohomology_objects_for_a_chain_complex_in_an_additive_category}{(co)homology objects} are \CrefAndHyperrefIfExist{definition:constructible_sheaf_on_a_small_site_on_a_scheme_or_a_topological_space}{constructible sheaves}. It is customary to speak of the subcategories \hl{$D^{?}(\calC, \calO)$} for $? \in \{+,-,b\}$ as usual.
\end{definition}



\subsection{\texorpdfstring{$\lambda$}{lambda}-adic sheaves}
\begin{definition}[$\lambda$-torsion sheaf on a scheme] \label{definition:lambda_torsion_sheaf_of_modules_over_a_sheaf_of_rings_on_a_site_for_a_sheaf_of_ideals}

Let $(\calC, J)$ be a \CrefAndHyperrefIfExist{definition:grothendieck_topology_on_a_category_site_covering_sieve_topologically_generating_family}{site}. Let $\calO$ be a \CrefAndHyperrefIfExist{definition:sheaf_on_a_site}{sheaf of commutative rings} on $(\calC, J)$. Let $\lambda \subseteq \calO$ be a \CrefAndHyperrefIfExist{definition:sheaf_of_ideals_for_a_sheaf_of_rings_on_a_site}{sheaf of ideals}.

A sheaf $\calF$ of \CrefAndHyperrefIfExist{definition:module_over_a_sheaf_of_rings_on_a_site}{$\calO$-modules} on $(\calC, J)$ is called a \hldef{$\lambda$-torsion sheaf} if 
$$\calF = \bigcup_{n \geq 1} \ker(\lambda^n: \calF \to \calF)$$
or equivalently if for every \CrefAndHyperrefIfExist{definition:sections_of_a_presheaf_on_a_category_valued_in_a_category}{section} $s \in \calF(U)$ for some object $U \in \Ob(\calC)$, there exists some $n \geq 1$ such that $\lambda^n \cdot s = 0$.

For instance, we may speak of $\lambda$-torsion sheaves by letting \CrefAndHyperrefIfExist{definition:constant_sheaf_on_a_site_with_sheafification}{$\calO = \underline{R}$} where $R$ is a commutative ring, $\lambda = \underline{I}$ where $I \subseteq R$ is an ideal, and $(\calC, J)$ be the \CrefAndHyperrefIfExist{definition:small_etale_site_of_a_scheme}{small \'etale site of a scheme}. 

% Let $R$ be a commutative ring, $\lambda\subseteq R$ an ideal, and $X$ a scheme. A sheaf $\mathcal{F}$ of $R$-modules on the \hyperrefIfExists{definition:small_etale_site_of_a_scheme}{(small) étale site $X_{\text{ét}}$}\CrefIfExists{definition:small_etale_site_of_a_scheme} is called a \hldef{$\lambda$-torsion sheaf} if for every \hyperrefIfExists{definition:etale_morphism_of_schemes}{étale morphism}\CrefIfExists{definition:etale_morphism_of_schemes} $U\to X$, every section $s\in \mathcal{F}(U)$ is annihilated by some power of $\lambda$, i.e.
% \[
% \forall s\in \mathcal{F}(U),\ \exists n\geq 1,\ \lambda^n \cdot s = 0,
% \]
% or equivalently, if 
% $$\calF = \bigcup_{n \geq 1} \ker(\lambda^n: \calF \to \calF).$$
\end{definition}


% The \CrefAndHyperrefIfExist{definition:mittag_leffler_condition_for_an_inverse_system_of_objects_in_a_category_with_images}{Mittag-Leffler condition} and the Artin-Rees\TODO{}
\CrefAndHyperrefIfExist{definition:lambda_adic_sheaf_on_the_small_etale_site_of_a_scheme_for_an_ideal_of_a_commutative_ring}{$\lambda$-adic sheaves} are specific inverse systems of torsion sheaves; imposing the \CrefAndHyperrefIfExist{definition:mittag_leffler_condition_for_an_inverse_system_of_objects_in_a_category_with_images}{Mittag-Leffler condition} ensures the existence of the limit as a sheaf and the stronger \CrefAndHyperrefIfExist{definition:artin_rees_mittag_leffler_condition_on_a_tower_in_a_category_with_images}{Artin-Rees-Mittag-Leffler condition} yields an abelian category. 

\begin{definition} \label{definition:tower_of_objects_in_a_category}
Let $\mathcal{C}$ be a \CrefAndHyperrefIfExist{definition:category}{category} and let $I$ be the \CrefAndHyperrefIfExist{definition:partially_ordered_set}{poset} of non-negative integers $\mathbb{N}$ with the standard ordering $\geq$ (i.e. $I$ has arrows $\cdots \rightarrow 2 \rightarrow 1 \rightarrow 0$).
A \hldef{tower in $\mathcal{C}$} is a \CrefAndHyperrefIfExist{definition:functor_between_categories}{functor} $X: I \rightarrow \mathcal{C}$. Explicitly, a tower consists of a sequence of objects $\{A_i\}_{i \in \mathbb{N}}$ and morphisms $f_i: A_i \rightarrow A_{i-1}$ for each $i \geq 1$:
$$ \cdots \xrightarrow{f_3} A_2 \xrightarrow{f_2} A_1 \xrightarrow{f_1} A_0.$$
Moreover a \hldef{morphism of towers} is a morphism of towers as objects of the \CrefAndHyperrefIfExist{definition:diagram_in_a_category_indexed_by_a_small_category}{functor category} $\calC^I$.
% Unless otherwise specified, the category of towers 
\end{definition}
\begin{definition} \label{definition:shifted_tower_of_objects_in_a_category}
    Let $\mathcal{C}$ be a \CrefAndHyperrefIfExist{definition:category}{category} and let $I$ be the \CrefAndHyperrefIfExist{definition:partially_ordered_set}{poset} of non-negative integers $\mathbb{N}$ with the standard ordering $\geq$ (i.e. $I$ has arrows $\cdots \rightarrow 2 \rightarrow 1 \rightarrow 0$) (or more generally let $I$ be the poset of integers $\bbZ$ with similar ordering).

    Given a \CrefAndHyperrefIfExist{definition:tower_of_objects_in_a_category}{tower} $A = (A_i, u_i)_{i \in I}$ where $u_i: A_i \to A_{i-1}$ (for $i \geq 1$ if $\bbN$ is the collection of objects of $I$)  and $r \geq 0$, the \hldef{shifted tower} \hl{$A[r]$} is the tower $A[r] = (A[r]_i, u[r]_i)$ given by $A[r]_i = A_{r+i}$ and whose transition maps are given by $u[r]_i: A[r]_i = A{r+i} \xrightarrow{u_{r+i}} A_{r+i-1} =  A[r]_{i-1}$.

    Note that there is a \CrefAndHyperrefIfExist{definition:tower_of_objects_in_a_category}{morphism of towers} $A[r] \to A$ where $A[r]_i \to A_i$ is given by the transition morphism $A_{r+i} \to A_i$.
\end{definition}
\begin{definition}[cf. {\cite[Definition 3.5.6]{weibel}}] \label{definition:mittag_leffler_condition_for_an_inverse_system_of_objects_in_a_category_with_images}
Let $I$ be a \CrefAndHyperrefIfExist{definition:filtered_cofiltered_category}{directed set} and let $\{A_i, \phi_{ji}\}_{i \in I}$ be an \CrefAndHyperrefIfExist{definition:system_in_a_category_indexed_by_a_directed_poset}{inverse system} of objects in a category $\mathcal{C}$ where \CrefAndHyperrefIfExist{definition:image_coimage_of_a_morphism_in_a_category}{images} are well-defined (such as the category of sets, abelian groups, or modules).

The system is said to satisfy the \hldef{Mittag-Leffler condition} if for every index $i \in I$, there exists an index $j \geq i$ such that for all $k \geq j$, the image of the transition map $\phi_{ki}: A_k \to A_i$ is equal to the image of $\phi_{ji}: A_j \to A_i$. 

For a fixed $i$, let $I_{k,i} = \text{im}(\phi_{ki}) \subseteq A_i$ for all $k \geq i$. The condition states that the decreasing family of subobjects 
$$ A_i \supseteq I_{i,i} \supseteq I_{i+1,i} \supseteq I_{i+2,i} \supseteq \cdots $$
becomes stationary.

\end{definition}
\begin{definition}[cf. {\cite[Definition 3.5.6]{weibel}}] \label{definition:trivial_mittag_leffler_condition_for_an_inverse_system_of_objects_in_a_pointed_category}
    Let $I$ be a \CrefAndHyperrefIfExist{definition:filtered_cofiltered_category}{directed set} and let $\{A_i, \phi_{ji}\}_{i \in I}$ be an \CrefAndHyperrefIfExist{definition:system_in_a_category_indexed_by_a_directed_poset}{inverse system} of objects in a \CrefAndHyperrefIfExist{definition:pointed_category}{pointed category} $\mathcal{C}$. 
    % where \CrefAndHyperrefIfExist{definition:image_coimage_of_a_morphism_in_a_category}{images} are well-defined (such as the category of sets, abelian groups, or modules).

    The system is said to satisfy the \hldef{trivial Mittag-Leffler condition} if for every index $i \in I$, there exists an index $j > i$ such that the map $A_j \to A_i$ is a \CrefAndHyperrefIfExist{definition:zero_morphism_in_a_pointed_category}{zero morphism}. 
\end{definition}
\begin{definition}  \label{definition:null_system_in_a_pointed_category_with_images}
    Let $I$ be the \CrefAndHyperrefIfExist{definition:partially_ordered_set}{poset} of non-negative integers $\mathbb{N}$ with the standard ordering $\geq$ (i.e. $I$ has arrows $\cdots \rightarrow 2 \rightarrow 1 \rightarrow 0$) and let $\{A_i, \phi_{ji}\}_{i \in I}$ be an \CrefAndHyperrefIfExist{definition:system_in_a_category_indexed_by_a_directed_poset}{inverse system} of objects (so a \CrefAndHyperrefIfExist{definition:tower_of_objects_in_a_category}{tower}) in a \CrefAndHyperrefIfExist{definition:pointed_category}{pointed category} $\mathcal{C}$ where \CrefAndHyperrefIfExist{definition:image_coimage_of_a_morphism_in_a_category}{images} are well-defined (such as the category of sets, abelian groups, or modules).

    The system is said to be a \hldef{null system} if there exists $r \geq 0$ such that for every $i \in I$, the transition map $\phi_{(i+r)i}: A_{i+r} \to A_i$ is the \CrefAndHyperrefIfExist{definition:zero_morphism_in_a_pointed_category}{zero morphism}.

    \TextIfExists{definition:shifted_tower_of_objects_in_a_category}{
    Equivalently, the tower is a null system if there exists an integer $r \geq 0$ such that the \CrefAndHyperrefIfExist{definition:tower_of_objects_in_a_category}{morphism of towers} $A[r] \to A$ (\Cref{definition:shifted_tower_of_objects_in_a_category}) is the \CrefAndHyperrefIfExist{definition:zero_morphism_in_a_pointed_category}{zero morphism}.
        \TODO{The category of towers $\mathcal{C}^I$ is pointed as $\mathcal{C}$ is pointed; the zero morphism is defined component-wise.}
    }

    % \TextIfExists{definition:shifted_tower_of_objects_in_a_category}{
    %     Equivalently, the tower is a null system if for any integer $n$, there exists an integer $r \geq 0$ such that $A[r] \to A$ (\Cref{definition:shifted_tower_of_objects_in_a_category}) is the \CrefAndHyperrefIfExist{definition:zero_morphism_in_a_pointed_category}{zero morphism}.
    %     \TODO{comment on how the category of towers is pointed}
    %     % $$\operatorname{im}(A[r] \to A) = \operatorname{im}(A[t] \to A)$$
    %     % \Cref{definition:shifted_tower_of_objects_in_a_category} for all $t \geq r$. 
    % }

\end{definition}
\input{../_definitions/definition_artin_rees_mittag_leffler_condition_on_a_tower_in_a_category_with_images.tex}

% \begin{definition}[Artin–Rees–Mittag–Leffler condition]
% Let $R$ be a commutative ring, $\lambda \subseteq R$ an ideal, and let $\{\mathcal{F}_n\}_{n\geq 1}$ be an \hyperrefIfExists{definition:system_in_a_category_indexed_by_a_directed_poset}{inverse system}\CrefIfExists{definition:system_in_a_category_indexed_by_a_directed_poset} of \hyperrefIfExists{definition:lambda_torsion_sheaf_of_modules_over_a_sheaf_of_rings_on_a_site_for_a_sheaf_of_ideals}{$\lambda$-torsion sheaves}\CrefIfExists{definition:lambda_torsion_sheaf_of_modules_over_a_sheaf_of_rings_on_a_site_for_a_sheaf_of_ideals} of $R$-modules on $X_{\text{ét}}$. This system satisfies the \hldef{Artin–Rees–Mittag–Leffler (A–R–M–L) condition} if for every $n\geq 1$, there exists $m\geq n$ such that for all $m'\geq m$,
% \[
% \mathrm{Im}\big(\mathcal{F}_{m'} \to \mathcal{F}_n\big) = \mathrm{Im}\big( \lambda^{m'-m}\cdot \mathcal{F}_m \to \mathcal{F}_n \big).
% \]
% \end{definition}

% \begin{definition}[Null system of $\lambda$-torsion sheaves]
% Let $\{\mathcal{F}_n\}_{n\geq 1}$ be an \hyperrefIfExists{definition:system_in_a_category_indexed_by_a_directed_poset}{inverse system}\CrefIfExists{definition:system_in_a_category_indexed_by_a_directed_poset} of \hyperrefIfExists{definition:lambda_torsion_sheaf_of_modules_over_a_sheaf_of_rings_on_a_site_for_a_sheaf_of_ideals}{$\lambda$-torsion sheaves}\CrefIfExists{definition:lambda_torsion_sheaf_of_modules_over_a_sheaf_of_rings_on_a_site_for_a_sheaf_of_ideals} of $R$-modules on $X_{\text{ét}}$. The system is called a \hldef{null system} if for every $n\geq 1$, there exists $m\geq n$ such that the transition morphism $\mathcal{F}_m \to \mathcal{F}_n$ is the zero morphism.
% \end{definition}

\begin{definition}[A–R category of inverse systems of $\lambda$-torsion sheaves] \label{definition:A_R_category_of_inverse_systems_of_lambda_torsion_sheaves_on_small_etale_site}
Let $X$ be a scheme. Let 
$$\calF = (\calF_n, u_n)_{n \in \bbZ}, \quad u_n: \calF_n \to \calF_{n-1}$$
be an \hyperrefIfExists{definition:system_in_a_category_indexed_by_a_directed_poset}{inverse system}\CrefIfExists{definition:system_in_a_category_indexed_by_a_directed_poset} of sheaves of abelian groups on the \hyperrefIfExists{definition:small_etale_site_of_a_scheme}{(small) étale site $X_{\text{ét}}$}. Let $R$ be a commutative ring and let $\lambda\subseteq R$ be an ideal. 

The \hldef{A–R category of inverse systems of $\lambda$-torsion sheaves on $X_{\text{ét}}$}, denoted \hl{$\mathrm{AR}(\lambda\text{-tors}(X))$}, is the localization of the multiplicative system consisting of the family of morphisms of the form $\calF[r] \to \calF$. In other words, the objects of the category are inverse systems of \hyperrefIfExists{definition:lambda_torsion_sheaf_of_modules_over_a_sheaf_of_rings_on_a_site_for_a_sheaf_of_ideals}{$\lambda$-torsion sheaves}\CrefIfExists{definition:lambda_torsion_sheaf_of_modules_over_a_sheaf_of_rings_on_a_site_for_a_sheaf_of_ideals} and for any objects $\calF$ and $\calG$, we have 
$$\Hom_{\mathrm{AR}(\lambda\text{-tors}(X))} = \varinjlim_{r \geq 0} \Hom(\calF[r], \calG).$$
\TODO{morphism of inverse systems; more generally diagrams}
In particular, a morphism $\calF \to \calG$ in $\mathrm{AR}(\lambda\text{-tors}(X))$ is represented by a morphsm $\calF[r] \to \calG$ of inverse systems for some $r \geq 0$, and the kernel and cokernel of $\calF[r] \to \calG$ are the kernel and cokernel of $\calF \to \calG$ in $\mathrm{AR}(\lambda\text{-tors}(X))$ 

It is worth noting that $\mathrm{AR}(\lambda\text{-tors}(X))$ is an \hyperrefIfExists{definition:abelian_category}{abelian category}\CrefIfExists{definition:abelian_category} and that $\calF \in \mathrm{AR}(\lambda\text{-tors}(X))$ is zero if and only if it is a \hyperrefIfExists{definition:artin_rees_mittag_leffler_condition_on_an_inverse_system_of_sheaves_of_abelian_groups_on_a_scheme_for_the_small_etale_site}{null system}\CrefIfExists{definition:artin_rees_mittag_leffler_condition_on_an_inverse_system_of_sheaves_of_abelian_groups_on_a_scheme_for_the_small_etale_site}
\end{definition}


\begin{definition}[$\lambda$-adic sheaf on a scheme] \label{definition:lambda_adic_sheaf_on_the_small_etale_site_of_a_scheme_for_an_ideal_of_a_commutative_ring}

Let $R$ be a commutative ring and let $\lambda\subseteq R$ be an ideal. Let $X$ be a scheme. Let 
$$\calF = (\calF_n, u_n)_{n \in \bbZ}, \quad u_n: \calF_n \to \calF_{n-1}$$
be an \hyperrefIfExists{definition:system_in_a_category_indexed_by_a_directed_poset}{inverse system}\CrefIfExists{definition:system_in_a_category_indexed_by_a_directed_poset} of \hyperrefIfExists{definition:lambda_torsion_sheaf_of_modules_over_a_sheaf_of_rings_on_a_site_for_a_sheaf_of_ideals}{$\lambda$-torsion sheaves}\CrefIfExists{definition:lambda_torsion_sheaf_of_modules_over_a_sheaf_of_rings_on_a_site_for_a_sheaf_of_ideals} on the \hyperrefIfExists{definition:small_etale_site_of_a_scheme}{(small) étale site $X_{\text{ét}}$}. 

\begin{enumerate}
    \item The inverse system is called a \hldef{$\lambda$-adic sheaf on $X$} if 
    \begin{itemize}
        \item $\calF_n$ is \CrefAndHyperrefIfExist{definition:constructible_sheaf_on_a_small_site_on_a_scheme_or_a_topological_space}{constructible} for all $n$,
        \item $\mathcal{F}_n$ is annihilated by $\lambda^n$ for all $n$,
        \item there are isomorphisms $\mathcal{F}_n \cong \mathcal{F}_{n+1}/\lambda^n\mathcal{F}_{n+1}$ compatible with transition maps.
    \end{itemize}

    \item A $\lambda$-adic sheaf is called \hldef{lisse} if each $\calF_n$ is \hyperrefIfExists{definition:locally_constant_sheaf_on_a_site_with_sheafification}{locally constant}\CrefIfExists{definition:locally_constant_sheaf_on_a_site_with_sheafification}.

    \item A $\lambda$-adic sheaf is called an \hldef{A–R $\lambda$-adic sheaf} if it satisfies the \hyperrefIfExists{definition:artin_rees_mittag_leffler_condition_on_an_inverse_system_of_sheaves_of_abelian_groups_on_a_scheme_for_the_small_etale_site}{Artin–Rees–Mittag–Leffler condition}\CrefIfExists{definition:artin_rees_mittag_leffler_condition_on_an_inverse_system_of_sheaves_of_abelian_groups_on_a_scheme_for_the_small_etale_site}.
\end{enumerate}
Under nice enough circumstances, the categories of $\lambda$-adic sheaves and $A-R$ $\lambda$-adic sheaves are equivalent (\Cref{theorem:categories_of_lambda_adic_and_A_R_lambda_adic_sheaves_over_the_ring_of_integers_of_a_finite_extension_of_Q_ell_are_equivalent}).
\end{definition}


Here is a basic example of a lambda adic sheaf:

\begin{definition}[$\ell$-adic sheaf $\mathbb{Z}_\ell$ on the small étale site] \label{definition:ell_adic_sheaf_Z_ell_on_small_etale_site_of_a_scheme}
Let $X$ be a scheme and let $\ell$ be a prime invertible on $X$.

The \hldef{$\ell$-adic sheaf} \hldef{$\mathbb{Z}_\ell$} on the \CrefAndHyperrefIfExist{definition:small_etale_site_of_a_scheme}{small étale site $X_{\text{ét}}$} is defined as the \CrefAndHyperrefIfExist{definition:system_in_a_category_indexed_by_a_directed_poset}{inverse system}
$$ \hlin{ \mathbb{Z}_\ell := \varprojlim_k \mathbb{Z}/\ell^k \mathbb{Z}, } $$
where each $\mathbb{Z}/\ell^k \mathbb{Z}$ is the \CrefAndHyperrefIfExist{definition:constant_sheaf_on_a_site_with_sheafification}{constant sheaf} of abelian groups on $X_{\text{ét}}$, and the transition maps are the canonical projections.

These inverse limits define objects in the category of \hldef{$\ell$-adic sheaves}, rather than single sheaves in the classical étale topology.
\end{definition}


\begin{definition}[The $\ell$-adic Tate twists $\mathbb{Z}_\ell(1)$ and $\mathbb{Z}_\ell(d)$] \label{definition:ell_adic_tate_twist_sheaves_on_the_small_etale_site_of_a_scheme}
Let $\ell$ be a prime invertible on $X$, and set
$$\hlin{\mathbb{Z}_\ell(1) := \varprojlim_{k} \mu_{\ell^k},}$$
the \CrefAndHyperrefIfExist{definition:limit_and_colimit_of_a_diagram_in_a_category}{inverse limit (projective system)} of \CrefAndHyperrefIfExist{definition:sheaf_of_nth_roots_of_unity_on_the_small_etale_site_of_a_scheme}{sheaves of $\ell^k$-th roots of unity} on \CrefAndHyperrefIfExist{definition:small_etale_site_of_a_scheme}{$X_{\text{ét}}$}, equipped with its natural $\mathbb{Z}_\ell$-module structure.

For an integer $d \geq 1$, define the \hldef{$d$-th Tate twist}
$$\hlin{\mathbb{Z}_\ell(d) := \underbrace{\mathbb{Z}_\ell(1) \otimes_{\mathbb{Z}_\ell} \cdots \otimes_{\mathbb{Z}_\ell} \mathbb{Z}_\ell(1)}_{d \text{ times}}.}$$

For $d=0$ set $\mathbb{Z}_\ell(0) := \mathbb{Z}_\ell$ \CrefIfExists{definition:ell_adic_sheaf_Z_ell_on_small_etale_site_of_a_scheme}.
\end{definition}

\begin{lemma}
    Let $(\calC, J)$ be a \CrefAndHyperrefIfExist{definition:grothendieck_topology_on_a_category_site_covering_sieve_topologically_generating_family}{site}. Let $\calO$ be a \CrefAndHyperrefIfExist{definition:sheaf_on_a_site}{sheaf of commutative rings} on $(\calC, J)$. Let $\lambda \subseteq \calO$ be a \CrefAndHyperrefIfExist{definition:sheaf_of_ideals_for_a_sheaf_of_rings_on_a_site}{sheaf of ideals}. Let $\{\calG_n\}_{n \geq 1}$ be an inverse system of constructible \TODO{generalize the notion of constructible sheaves to a sheaf of $\calO$-modules}

    Let $R$ be a commutative ring and let $\lambda \subseteq R$ be an ideal. \TODO{Try to generalize stacks 03UN as an exercise}
\end{lemma}

\begin{theorem}{See e.g. {\cite[10.1]{fu_ect}}} \label{theorem:categories_of_lambda_adic_and_A_R_lambda_adic_sheaves_over_the_ring_of_integers_of_a_finite_extension_of_Q_ell_are_equivalent}
    Let $X$ be a noetherian scheme. Let $\ell$ be a prime number that is invertible on $X$. Let $R$ be the integral closure of $\bbZ_\ell$ in a finite extension $E$ of $\Qell$. Note that $R$ is a complete discrete valuation ring and let $\lambda$ be the maximal ideal of $R$. The category of \hyperrefIfExists{definition:lambda_adic_sheaf_on_the_small_etale_site_of_a_scheme_for_an_ideal_of_a_commutative_ring}{$\lambda$-adic sheaves}\CrefIfExists{definition:lambda_adic_sheaf_on_the_small_etale_site_of_a_scheme_for_an_ideal_of_a_commutative_ring} is equivalent to the category of \hyperrefIfExists{definition:lambda_adic_sheaf_on_the_small_etale_site_of_a_scheme_for_an_ideal_of_a_commutative_ring}{A-R $\lambda$-adic sheaves}\CrefIfExists{definition:lambda_adic_sheaf_on_the_small_etale_site_of_a_scheme_for_an_ideal_of_a_commutative_ring}.
\end{theorem}



\subsection{The category \texorpdfstring{$D_c^b(X,R)$}{Dcb(X,R)} of integral coefficients}

For a scheme $X$, a prime $\ell$ invertible on $X$, and the ring of integers $R$ in a finite extension $E$ of $\Qell$, Deligne {\cite[I.I.2]{deligne_lcwii}} defined ``derived'' categories \hyperrefIfExists{definition:derived_category_of_bounded_constructible_complexes_of_adic_sheaves_with_integral_coefficients_on_a_noetherian_scheme}{$D_c^b(X, R)$}\CrefIfExists{definition:derived_category_of_bounded_constructible_complexes_of_adic_sheaves_with_integral_coefficients_on_a_noetherian_scheme}, $D_c^b(X, E)$, and $D_c^b(X, \Qellbar)$, intuitively of ``integral'', ``rational'', and $\Qellbar$ coefficients respectively. These categories are not derived categories themselves but constructed $\lambda$-adically through the genuine derived categories $D_c^b(X,R/\lambda^n)$ of finite coefficients. loc.~cit.~ also established that the six functor formalisms on the categories $D_c^b(X, R/\lambda^n)$ induce a six functor formalism on $D_c^b(X,R)$ under suitable finiteness conditions, e.g. $X$ is of finite type over a regular base $S$ of dimension $\leq 1$, and that $D_c^b(X,R)$ is a triangulated category under other finiteness conditions, e.g. $X$ is of finite type over a finite field or an algebraically closed field.

\begin{definition}[see {\cite[I.I.2]{deligne_lcwii}}, cf. {\cite[Between Propositions 10.1.16 and 10.1.17]{fu_ect}}, for a discussion in the case that $R$ is the ring of integers in a finite extension of $\Qell$ and $\lambda$ is the maximal ideal of $R$] \label{definition:derived_category_of_bounded_constructible_complexes_of_adic_sheaves_with_integral_coefficients_on_a_noetherian_scheme}

    \TODO{tensor product}
    Let $R$ be a \CrefAndHyperrefIfExist{definition:commutative_ring}{commutative ring} and let $\lambda\subseteq R$ be an \CrefAndHyperrefIfExist{definition:ideal_of_a_ring}{ideal} such that $R/\lambda^n$ is of finite cardinality for all $n \geq 0$. Let $X$ be a scheme, and consider sheaves on some site on $X$ whose underlying category is a (not necessarily full) subcategory of the category of $X$-schemes (e.g. the \CrefAndHyperrefIfExist{definition:small_zariski_site_of_a_schem}{small Zariski site}, the \CrefAndHyperrefIfExist{definition:small_nisnevich_site_of_a_schem}{small Nisnevich site}, the \CrefAndHyperrefIfExist{definition:small_etale_site_of_a_scheme}{small \'etale site}, the \CrefAndHyperrefIfExist{definition:small_fpqc_site_of_a_schem }{small fpqc site}, or the \CrefAndHyperrefIfExist{definition:small_fppf_site_of_a_scheme}{small fppf site} on $X$).  
    \TODO{It might be necessary to but more conditions on $\lambda$, particularly that $R/\lambda$ is finite and has exponent that is invertible on $X$}

    \CrefIfExists{definition:derived_category_of_an_abelian_category}

    \begin{enumerate}
        \item Let \hldef{$D^-(X,R)$} be the following category: 
        \begin{itemize}
            \item Objects are families $K = (K_n, u_n)_{n \geq 0}$ where $K_n \in \Ob D^-(X, R/\lambda^{n+1})$\CrefIfExists{notation:notations_for_homotopy_and_derived_categories_of_sheaves_of_modules_on_a_ringed_site}\footnote{Note that \cite[I.I.2]{deligne_lcwii} uses slightly different numbering conventions, letting $K_n$ be an object of $D^-(X, R/\lambda^n)$ instead.} and $u_n$ are isomorphisms
            $$u_n: K_{n+1} \Lotimes_{R / \lambda^{n+2}} R/\lambda^{n+1} \cong K_n$$
            \CrefIfExists{definition:derived_tensor_product_on_bounded_above_derived_categories_of_abelian_categories_for_a_biadditive_right_exact_functor}\CrefIfExists{definition:tensor_product_of_sheaves_of_bimodules_over_sheaves_of_rings_on_a_site}
            in $D^-(X, R/\lambda^{n+1})$.

            \item Morphisms $f: K = (K_n, u_n)_{n \geq 0} \to K' = (K'_n, u'_n)_{n \geq 0}$ are families $(f_n)$ of morphisms $f_n: K_n \to K_n$ in $D(X, R/\lambda^{n+1})$ such that 
            $$f_n u_n = u_n' (f_{n+1} \Lotimes_{R/\lambda^{n+2}} \id_{R/\lambda^{n+1}}) .$$
        \end{itemize}
        $K_n$ is also often denoted by \hl{$K \Lotimes R/\lambda^{n+1}$}.

        \item Let \hldef{$D_c^b(X,R)$} be the full subcategory of $D^-(X,R)$\CrefIfExists{notation:notations_for_homotopy_and_derived_categories_of_sheaves_of_modules_on_a_ringed_site} whose objects $K = (K_n,u_n)_{n \geq 0}$ satisfy $K_0 \in \Ob D_c^b(X, R/(\lambda))$\CrefIfExists{definition:derived_category_of_cohomologically_constructible_complexes_of_sheaves_of_modules_of_a_sheaf_of_rings_on_a_topological_space_or_scheme}.
    \end{enumerate}
\end{definition}




\begin{proposition}[See, e.g. {\cite[Proposition 10.1.16]{fu_ect}}]
    Let $X$ be a noetherian scheme. Let $\ell$ be a prime number that is invertible on $X$. Let $R$ be the integral closure of $\bbZ_\ell$ in a finite extension $E$ of $\Qell$. Note that $R$ is a complete discrete valuation ring and let $\lambda$ be the maximal ideal of $R$.

    For $K = (K_n, u_n)_{n \geq 0}$ in \CrefAndHyperrefIfExist{definition:derived_category_of_bounded_constructible_complexes_of_adic_sheaves_with_integral_coefficients_on_a_noetherian_scheme}{$D_c^b(X,R)$}, we have that $K_n \in D_{\mathrm{ctf}}^b(X, R/(\lambda^{n+1}))$\TODO{ctf} for all $n$, and the inverse systems $(H^i(K_n)_{n \in \bbZ})$ are \hyperrefIfExists{definition:lambda_adic_sheaf_on_the_small_etale_site_of_a_scheme_for_an_ideal_of_a_commutative_ring}{A-R $\lambda$-adic}\CrefIfExists{definition:lambda_adic_sheaf_on_the_small_etale_site_of_a_scheme_for_an_ideal_of_a_commutative_ring}.
\end{proposition}

\begin{remark}
    Let $X$ be a noetherian scheme. Let $\ell$ be a prime number that is invertible on $X$. Let $R$ be the integral closure of $\bbZ_\ell$ in a finite extension $E$ of $\Qell$.
    If $\calF = (\calF_n)$ is a torsion free $\lambda$-adic sheaf \TODO{define torsion free}, then $\calF$ defines an objects in $\Dbc(X,R)$. 
\end{remark}

\begin{definition}
    \TODO{cite these things}
    Let $X$ be a \CrefAndHyperrefIfExist{definition:locally_noetherian_and_noetherian_scheme}{noetherian} scheme. Let $\ell$ be a prime number that is invertible on $X$. Let $R$ be the \CrefAndHyperrefIfExist{definition:integral_element_over_a_ring}{integral closure} of $\bbZ_\ell$ in a finite extension $E$ of $\Qell$.

    \TODO{Define the six functors on $D_c^b(X,R/lambda^n)$}
    Let $f: X \to Y$ be a morphism between noetherian-schemes over a base scheme $S$. Consider \CrefAndHyperrefIfExist{definition:sheaf_on_a_site}{sheaves} on the \CrefAndHyperrefIfExist{definition:small_etale_site_of_a_scheme}{small \'etale sites} on $X$ and $Y$. Take objects $K=(K_n)$ and $L=(L_n)$ of \CrefAndHyperrefIfExist{definition:derived_category_of_bounded_constructible_complexes_of_adic_sheaves_with_integral_coefficients_on_a_noetherian_scheme}{$D_c^b(X,R)$} and $M = (M_n)$ of $D_c^b(Y,R)$. Define the following:
    \begin{enumerate}
        \item $$\hlin{f^* K = (f^* K_n)_n}$$
        \CrefIfExists{definition:inverse_image_and_direct_image_of_sheaves_on_schemes_for_various_topologies}\CrefIfExists{definition:direct_image_of_a_sheaf_on_a_site_under_a_continuous_functor_of_sites_or_a_site_morphism}

        \TODO{compactifiable morphism}
        \item If $f$ is an $S$-compactifiable morphism over some base scheme $S$, 
        $$\hlin{Rf_! K = (Rf_! K_n)}$$
        \TODO{}
        \item If $f$ is an $S$-compactifiable morphism where $S$ is a noetherian \CrefAndHyperrefIfExist{definition:regular_at_a_point_for_a_scheme_regular_scheme_nonsingular_scheme}{regular scheme} of \CrefAndHyperrefIfExist{definition:dimension_of_a_scheme}{dimension} $\leq 1$ and $X$ and $Y$ are $S$-schemes of \CrefAndHyperrefIfExist{definition:finite_type_morphism_of_schemes}{finite type},

        \hlalign{
        \begin{align*}
            Rf_* K &= (Rf_* K_n)_n \\
            % Rf_! K &= (Rf_! K_n)_n \quad (\text{assuming that } f \text{ is compactifiable over } S)\\
            Rf^! K &= (Rf^! K_n)_n \quad (\text{assuming that } f \text{ is compactifiable over } S)\\
            K \Lotimes_R L &= (K_n \Lotimes_{R/\lambda^{n+1}} L_n) \\
            R\mathscr{H}om(K,L) &= (R\mathscr{H}om)(K_n,L_n).
        \end{align*}
        }
        \CrefIfExists{definition:inverse_image_of_a_sheaf_under_a_continuous_functor_of_sites_or_a_site_morphism}
        \TODO{exceptional inverse image}
        \CrefIfExists{definition:derived_tensor_product_on_bounded_above_derived_categories_of_abelian_categories_for_a_biadditive_right_exact_functor}\TODO{derived sheaf hom}
    \end{enumerate}
    % \CrefIfExists{}


    % Let $R$ be a commutative ring and let $\lambda\subseteq R$ be an ideal. Let $S$ be a scheme. Let $f: X \to Y$ be a morphism between $S$-schemes. 

    % \TODO{It might be necessary to but more conditions on $\lambda$, particularly that $R/\lambda$ is finite and has exponent that is invertible on $X$}

    % Given objects $K=(K_n)$ and $L=(L_n)$ of $D_c^b(X,R)$\CrefIfExists{definition:derived_category_of_bounded_constructible_complexes_of_adic_sheaves_with_integral_coefficients_on_a_noetherian_scheme} and $M = (M_n)$ of $D_c^b(Y,R)$, define
    % \hlalign{
    % \begin{align*}
    %     Rf_* K &= (Rf_* K_n)_n \\
    %     Rf_! K &= (Rf_! K_n)_n \quad (\text{assuming that } f \text{ is compactifiable over } S)\\
    %     Rf^! K &= (Rf^! K_n)_n \quad (\text{assuming that } f \text{ is compactifiable over } S)\\
    %     K \Lotimes_R L &= (K_n \Lotimes_{R/\lambda^{n+1}} L_n) \\
    %     R\mathscr{H}om(K,L) &= (R\mathscr{H}om)(K_n,L_n).
    % \end{align*}
    % }
    When they are defined, $Rf_* K$ and $Rf_! K$ are objects in $D_c^b(Y,R)$ and $f^* M, Rf^! M, K \otimes_R^L L$ and $R\mathscr{H}om(K,L)$ are objects in $D_c^b(X,R)$.
    $Rf^! K$, $K \Lotimes_R L$, and $R\mathscr{H}om(K,L)$ may alternatively be denoted by notations such as \hl{$f^! K$}, \hl{$K \otimes_R L$}, and \hl{$\mathrm{RHom}(K,L)$} respectively.
\end{definition}

\begin{proposition}
    Let $f: X \to Y$ be a morphism between noetherian schemes. Let $\ell$ be a prime number that is invertible on $X$ and $Y$. Let $R$ be the integral closure of $\bbZ_\ell$ in a finite extension $E$ of $\Qell$. Note that $R$ is a complete discrete valuation ring and let $\lambda$ be the maximal ideal of $R$.
    \begin{enumerate}
        \item For any $K = (K_n) \in D_c^b(Y,R)$, we have $f^*k = (f^*K_n) \in D_c^b(X,R)$.
    \end{enumerate}
\end{proposition}


\begin{definition} \label{definition:zero_divisor_of_a_ring}
    Let $(R,+,\cdot)$ be a \CrefAndHyperrefIfExist{definition:ring}{not-necessarily commutative ring}.
    \begin{enumerate}
        \item An element $a \in R$ is a \hldef{left zero-divisor} if there exists a nonzero $x \in R$ such that $ax = 0$. Otherwise, $a$ is called \hldef{left regular} or \hldef{left cancellable}.
        \item An element $a \in R$ is a \hldef{right zero-divisor} if there exists a nonzero $x \in R$ such that $xa = 0$. Otherwise, $a$ is called \hldef{right regular} or \hldef{right cancellable}.

        \item An element $a \in R$ is a \hldef{zero-divisor} if it is a left zero-divisor or a right zero-divsor.
        \item An element $a \in R$ is a \hldef{two-sided zero-divisor} if it is both a left zero-divisor and a right zero-divsor.

        \item An element $a \in R$ is \hldef{regular}, \hldef{cancellable}, or a \hldef{non-zero-divisor} if it is both left and right regular.
    \end{enumerate}

    A zero-divisor of any kind that is not itself $0$ is said to be a \hldef{nonzero zero divisor} or a \hldef{nontrivial zero divisor} of its kind. 
    
    A non-zero ring with no nontrivial zero divisors is called a \hldef{domain}. A domain that it also a \CrefAndHyperref{definition:commutative_ring}{commutative ring} is also called an \hldef{integral domain}.
\end{definition}


\begin{definition}[cf. {\cite[Between Corollary 10.1.21 and Lemma 10.1.22]{fu_ect}}] \label{definition:derived_category_of_cohomologically_bounded_constructible_sheaves_with_rational_adic_coefficients_on_a_noetherian_scheme}

    Let $R$ be a \CrefAndHyperrefIfExist{definition:domain}{integral domain} with fraction field $E$, and let $\lambda\subseteq R$ be an \CrefAndHyperrefIfExist{definition:ideal_of_a_ring}{ideal} such that $R/\lambda^n$ is of finite cardinality for all $n \geq 0$. Let $X$ be a scheme, and consider sheaves on some site on $X$ whose underlying category is a (not necessarily full) subcategory of the category of $X$-schemes (e.g. the \CrefAndHyperrefIfExist{definition:small_zariski_site_of_a_schem}{small Zariski site}, the \CrefAndHyperrefIfExist{definition:small_nisnevich_site_of_a_schem}{small Nisnevich site}, the \CrefAndHyperrefIfExist{definition:small_etale_site_of_a_scheme}{small \'etale site}, the \CrefAndHyperrefIfExist{definition:small_fpqc_site_of_a_schem }{small fpqc site}, or the \CrefAndHyperrefIfExist{definition:small_fppf_site_of_a_scheme}{small fppf site} on $X$).  
    \TODO{It might be necessary to but more conditions on $\lambda$, particularly that $R/\lambda$ is finite and has exponent that is invertible on $X$}

    \begin{enumerate}
        \item On the category \CrefAndHyperref{definition:derived_category_of_bounded_constructible_complexes_of_adic_sheaves_with_integral_coefficients_on_a_noetherian_scheme}{$\Dbc(X,R)$}, the morphisms defined by multiplications by $\lambda^m$ for $m \geq 0$ form a \CrefAndHyperrefIfExist{definition:multiplicative_system_of_morphisms_in_a_category}{multiplicative system} \TODO{why}. Define the category \hl{$\Dbc(X,E)$} as the \CrefAndHyperrefIfExist{definition:localization_of_a_category_by_a_multiplicative_system}{localization} of $\Dbc(X,R)$ by this system. In particular, the objects of $\Dbc(X,E)$ are those of $\Dbc(X,R)$ and 
        $$\Hom_{\Dbc(X,R)}(K,L) \otimes_R E \cong \Hom_{\Dbc(X,E)}(K,L)$$
        for all objects $K,L$ of $\Dbc(X,R)$.

        \item Let $K/E$ be some \CrefAndHyperrefIfExist{definition:algebraic_element_minimal_polynomial_of_an_algebraic_element_algebraic_field_extension_transcendental_element_field_extension}{algebraic extension}. The category \hl{$\Dbc(X,K)$} is defined as the $2$-direct limit of the categories $\Dbc(X,L)$ where $L$ run over the finite extensions of $K$:
        $$\Dbc(X,K) = \varinjlim_{L/K \text{ finite extension}} \Dbc(X,E).$$
        In other words, each object of $\Dbc(X,K)$ is represented by some object of $\Dbc(X,L)$ for some $L/K$ and given two objects $M_1 \in \Ob (\Dbc(X, L_1))$ and $M_2 \in \Ob(\Dbc(X,L_2))$, we have
        $$\Hom_{\Dbc(X,K)}(M_1, M_2) \cong \varinjlim_{L/K \text{ finite extension, } L_1,L_2 \subseteq L} \Hom(M_1 \otimes_{L_1} L, M_2 \otimes_{L_2} L).$$
        \TODO{define base change}

    \end{enumerate}


\end{definition}


\begin{definition}[Algebraic Element, Algebraic Extension] \label{definition:algebraic_element_minimal_polynomial_of_an_algebraic_element_algebraic_field_extension_transcendental_element_field_extension}
Let $L/K$ be a \CrefAndHyperrefIfExist{definition:extension_of_a_field}{field extension} and let $x \in L$.  
\begin{itemize}
    \item If there exists a nonzero polynomial $f(t) \in K[t]$ such that $f(x) = 0$, then $x$ is called an \hldef{algebraic element over $K$}. There exists a unique such monic irreducible polynomial $f(t)$, which is called the \hldef{minimal polynomial of $x$ over $K$}.
    \item Otherwise, $x$ is called a \hldef{transcendental element over $K$}.  
\end{itemize}
If every $x \in L$ is algebraic over $K$, then $L/K$ is called an \hldef{algebraic extension}.
\end{definition}


\begin{definition} \label{definition:derived_category_of_cohomologically_bounded_constructible_sheaves_with_rational_adic_coefficients_on_a_noetherian_scheme}
    Let $R$ be a \CrefAndHyperrefIfExist{definition:domain}{integral domain} with fraction field $E$, and let $\lambda\subseteq R$ be an \CrefAndHyperrefIfExist{definition:ideal_of_a_ring}{ideal} such that $R/\lambda^n$ is of finite cardinality for all $n \geq 0$. Let $K/E$ be some \CrefAndHyperrefIfExist{definition:algebraic_element_minimal_polynomial_of_an_algebraic_element_algebraic_field_extension_transcendental_element_field_extension}{algebraic extension}. Let $X$ be a scheme, and consider sheaves on some site on $X$ whose underlying category is a (not necessarily full) subcategory of the category of $X$-schemes (e.g. the \CrefAndHyperrefIfExist{definition:small_zariski_site_of_a_schem}{small Zariski site}, the \CrefAndHyperrefIfExist{definition:small_nisnevich_site_of_a_schem}{small Nisnevich site}, the \CrefAndHyperrefIfExist{definition:small_etale_site_of_a_scheme}{small \'etale site}, the \CrefAndHyperrefIfExist{definition:small_fpqc_site_of_a_schem }{small fpqc site}, or the \CrefAndHyperrefIfExist{definition:small_fppf_site_of_a_scheme}{small fppf site} on $X$). 
    \TODO{It might be necessary to but more conditions on $\lambda$, particularly that $R/\lambda$ is finite and has exponent that is invertible on $X$}

    \TODO{probably need to talk about base changes of ell-adic sheaves, $\Qellbar$-sheaves, etc.}
    \TODO{$\Dbc(X, \Qellbar)$}
\end{definition}

\section{Theorems}

\begin{corollary}[Finiteness of cohomology]
    \TODO{read}
Let $X$ be a variety of finite type over a separably closed field $k$, and let $\ell \neq \operatorname{char}(k)$. Then each $H^i_{\text{ét}}(X, \mathbb{Q}_\ell)$ is a finite-dimensional $\mathbb{Q}_\ell$-vector space, and vanishes for $i > 2 \dim(X)$.

\end{corollary}


\subsection{Base change theorem}

\begin{proposition}[Smooth base change]
    \TODO{read}
Let $f : X \to S$ be a smooth, proper morphism of schemes, and let $\ell$ be a prime invertible on $S$. Then the formation of étale cohomology commutes with base change: for any cartesian square

\[
\begin{aligned}
Y &\longrightarrow X \\
\downarrow & \;\;\;\;\;\;\; \downarrow f \\
T &\longrightarrow S
\end{aligned}
\]

one has natural isomorphisms

$$ H^i_{\text{ét}}(Y_{\overline{t}}, \mathbb{Q}_\ell) \;\cong\; H^i_{\text{ét}}(X_{\overline{s}}, \mathbb{Q}_\ell), $$

where $\overline{t}$ lies over $t \in T$, and $\overline{s}$ its image in $S$.

\end{proposition}


\begin{theorem}[Proper base change theorem]
Let $f : X \to Y$ be a proper morphism of schemes, and let $\ell$ be a prime invertible on $Y$. Then for any sheaf of $\mathbb{Z}_\ell$-modules $\mathcal{F}$ on $X_{\text{ét}}$, the higher direct image sheaves $R^if_*\mathcal{F}$ on $Y_{\text{ét}}$ satisfy
$$
(R^i f_* \mathcal{F})_y \cong H^i_{\text{ét}}(X_y, \mathcal{F}|_{X_y})
$$
for each geometric point $y$ of $Y$, where $X_y$ is the fiber over $y$.
\end{theorem}

\begin{theorem}[Smooth and proper base change compatibility]
Under suitable smooth and proper morphisms, the formation of étale cohomology commutes with base change on the target scheme. More precisely, given a cartesian square
\[
\begin{aligned}
X' &\to X \\
\downarrow & \;\;\;\downarrow \\
Y' &\to Y
\end{aligned}
\]
with $f:X \to Y$ smooth and proper, and a prime $\ell$ invertible on $Y$, the natural base change morphism induces isomorphisms in étale cohomology
$$
H^i_{\text{ét}}(X_y, \mathbb{Q}_\ell) \cong H^i_{\text{ét}}(X'_{y'}, \mathbb{Q}_\ell)
$$
for geometric points $y$ of $Y$ and $y'$ over $y$ in $Y'$.
\end{theorem}


\subsection{Poincar\'e duality}


\begin{theorem}[Generalized Poincaré Duality in Étale Cohomology]
    \TODO{read this}
Let $k$ be a separably closed field, and let $X$ be a smooth, separated scheme of finite type over $k$, of pure dimension $d$. Let $\ell$ be a prime different from $\operatorname{char}(k)$. 

Let $\Lambda$ be one of the following coefficient rings: 
\begin{itemize}
  \item a finite extension of $\mathbb{Z}_\ell$, 
  \item the ring of integers $\mathcal{O}_E$ of a finite extension $E$ of $\mathbb{Q}_\ell$, 
  \item $E$ itself, or 
  \item an algebraic closure $\overline{\mathbb{Q}}_\ell$.
\end{itemize}

Let $\mathcal{F}^\bullet$ be a bounded constructible complex of $\Lambda$-modules on the étale site $X_{\et}$.

Then there is a canonical perfect pairing in the derived category of $\Lambda$-modules:
$$
R\Gamma_{\et}(X, \mathcal{F}^\bullet) \;\otimes_{\Lambda}^{\mathbf{L}}\; R\Gamma_c^{\et}(X, R\mathcal{H}om_{\Lambda}(\mathcal{F}^\bullet, \Lambda(d)[2d])) \;\longrightarrow\; \Lambda[-2d],
$$
where $R\Gamma_c^{\et}$ denotes étale cohomology with compact support, and 
$$
\Lambda(d) := \underbrace{\Lambda(1) \otimes_{\Lambda} \cdots \otimes_{\Lambda} \Lambda(1)}_{d \text{ times}}
$$
is the $d$-th Tate twist.

In particular, this induces a perfect duality between finite generated $\Lambda$-modules
$$
H^i_{\et}(X, \mathcal{F}^\bullet) \cong \operatorname{Hom}_{\Lambda}(H_c^{2d - i}(X, R\mathcal{H}om_{\Lambda}(\mathcal{F}^\bullet, \Lambda(d)[2d])), \Lambda).
$$

\textbf{Remarks:}
\begin{itemize}
    \item For projective $X$, $R\Gamma_c^{\et}$ can be replaced by the usual cohomology $R\Gamma_{\et}$.
    \item The duality is a manifestation of Verdier duality and requires the formalism of the six operations and dualizing complexes in étale cohomology.
    \item This generalizes classical Poincaré duality by allowing more general coefficient complexes and non-projective varieties with compact support.
\end{itemize}
\end{theorem}

\subsection{Lefschetz trace formula}

\begin{theorem}[Lefschetz trace formula in étale cohomology]
    \TODO{read}
Let $X$ be a separated scheme of finite type over a finite field $\mathbb{F}_q$, and let $\ell$ be a prime number not dividing $q$. Then the number of $\mathbb{F}_q$-rational points of $X$ is given by the Lefschetz trace formula:
$$
|X(\mathbb{F}_q)| = \sum_{i=0}^{2 \dim(X)} (-1)^i \operatorname{Tr}(\operatorname{Frob}_q^* \mid H^i_{\text{ét}}(X_{\overline{\mathbb{F}}_q}, \mathbb{Q}_\ell)),
$$
where $\operatorname{Frob}_q$ is the geometric Frobenius acting on the $\ell$-adic étale cohomology groups of $X$.

\end{theorem}

\begin{theorem}[Grothendieck–Lefschetz trace formula in terms of traces]
Let $X_0$ be a separated scheme of finite type over a finite field $\mathbb{F}_q$. Denote by $\overline{X} = X_0 \times_{\mathbb{F}_q} \overline{\mathbb{F}}_q$ the base change to an algebraic closure, and let $K_0 \in D_c^b(X_0, \overline{\mathbb{Q}}_\ell)$ be a bounded constructible complex of étale sheaves.

Then the number of fixed points of the $n$-th iterate of the geometric Frobenius on $X_0$, weighted by the trace of the induced endomorphism on stalks of $K_0$, satisfies the formula
$$
\sum_{x \in X_0(\mathbb{F}_{q^n})} \operatorname{tr}\left( ( \mathrm{Frob}_{q^n} )_x \mid K_{0,\overline{x}} \right) = \sum_{i \in \mathbb{Z}} (-1)^i \operatorname{tr}\left( \mathrm{Frob}_{q^n} \mid H_c^i(\overline{X}, K) \right),
$$
where $K$ denotes the pullback of $K_0$ to $\overline{X}$, and $( \mathrm{Frob}_{q^n} )_x$ is the induced action on the stalk of $K_0$ at a geometric point over $x$.

In particular, this expresses the weighted count of $\mathbb{F}_{q^n}$-rational points on $X_0$ in terms of traces of Frobenius on étale cohomology with compact support, encapsulating the Grothendieck–Lefschetz trace formula in terms of traces rather than directly using determinant or Frobenius polynomials.
\end{theorem}


\subsection{Weil conjectures}

\begin{theorem}[Weil Conjectures via Étale Cohomology]
Let $X$ be a smooth projective variety of pure dimension $d$ defined over a finite field $\mathbb{F}_q$. Let $\ell$ be a prime number different from the characteristic of $\mathbb{F}_q$. Let $F$ denote the geometric Frobenius endomorphism acting on the étale cohomology groups $H^i_{\mathrm{\acute{e}t}}(\overline{X}, \mathbb{Q}_\ell)$, where $\overline{X} = X \times_{\mathbb{F}_q} \overline{\mathbb{F}}_q$.

Then the Weil conjectures state the following:

\begin{enumerate}
    \item (Rationality) The zeta function
    $$
    Z(X, t) = \exp\left( \sum_{n=1}^\infty \frac{\# X(\mathbb{F}_{q^n})}{n} t^n \right)
    $$
    is a rational function which can be expressed as
    $$
    Z(X,t) = \prod_{i=0}^{2d} P_i(t)^{(-1)^{i+1}},
    $$
    where each $P_i(t) = \det(1 - t F \mid H^i_{\mathrm{\acute{e}t}}(\overline{X}, \mathbb{Q}_\ell))$ is a polynomial with coefficients in $\mathbb{Q}_\ell$.
    
    \item (Functional Equation) The zeta function satisfies a functional equation relating $Z(X,t)$ and $Z(X, q^{-d} t^{-1})$, induced by Poincaré duality on cohomology and the action of $F$.
    
    \item (Betti Numbers) Each polynomial $P_i(t)$ has degree equal to the $i$-th Betti number of $X$ (i.e., $\dim_{\mathbb{Q}_\ell} H^i_{\mathrm{\acute{e}t}}(\overline{X}, \mathbb{Q}_\ell)$).
    
    \item (Riemann Hypothesis) The eigenvalues of the geometric Frobenius $F$ acting on $H^i_{\mathrm{\acute{e}t}}(\overline{X}, \mathbb{Q}_\ell)$ are algebraic numbers, all of whose complex absolute values equal $q^{i/2}$.
\end{enumerate}

This theorem links the counting of points over finite fields to the action of Frobenius on étale cohomology and provides a cohomological interpretation and proof of the Weil conjectures.
\end{theorem}


Here is another, more general, formulation of the rationality statement of the Weil conjectures:


\begin{theorem}[Grothendieck, see {\cite[Rapport 3 Th\'eor\`eme 3.1]{SGA4_5}}; see also {\cite[Theorem 10.5.1]{fu_ect}} for a statement and cf. {\cite[Theorem I.1.1]{kiehl_weissauer_wcps}}] \label{theorem:grothendieck_L_function_of_a_complex_is_a_determinant_of_the_total_compactly_supported_cohomology_of_the_complex} 
Let \(X_0\) be a scheme of finite type over the finite field \(\mathbb{F}_q\), and let \(K_0 \in D^b_c(X_0, \overline{\mathbb{Q}}_\ell)\). Write \(X = X_0 \times_{\mathbb{F}_q} \overline{\mathbb{F}}_q\) for the base change, and \(K\) for the the pullback of \(K_0\) to \(X\).

We have 
\[
L(X_0, K_0, t) = \prod_{i \in \mathbb{Z}} \det\left( 1 - t \cdot \mathrm{Frob}_q \mid H_c^i(X, K) \right)^{(-1)^{i+1}}.
\]
\end{theorem}


\section{Ekedahl's formalism}

\TODO{discuss what \cite{ekedahl_af} does} We will discuss Ekedahl's formalism on the \CrefAndHyperrefIfExist{definition:small_etale_site_of_a_scheme}{small \'etale site} of finite type and seperated schemes over regular bases of dimension $0$ or $1$ 

\begin{remark}
    Ekedahl's definitions and notations \cite{ekedahl_af} are developed in terms of a ringed topos $(S,R)$. We will develop definitions and notations of his theory in terms of a \CrefAndHyperrefIfExist{definition:ringed_site}{ringed site}. 
\end{remark}

\begin{definition}[Categories of Inverse and Direct Sequences of objects in a category]
\label{definition:category_of_inverse_direct_sequences_of_objects_in_a_category}
    Let $\mathcal{C}$ be a \CrefAndHyperrefIfExist{definition:category}{category}. 
    \begin{enumerate}
        \item  The \hldef{category of inverse sequences} or \hldef{Pro-category indexed by $\mathbb{N}$}, denoted by \hl{$\mathrm{Pro}_{\mathbb{N}}(\mathcal{C})$}, is defined as follows:
        \begin{itemize}
            \item Objects are sequences $(M_n, p_n)_{n \geq 1}$ where for each $n \geq 1$, $M_n \in \mathrm{Ob}(\mathcal{C})$ and $p_n : M_{n+1} \to M_n$ is a morphism in $\mathcal{C}$.
            \item A morphism $f : (M_n, p_n) \to (N_n, q_n)$ consists of a collection of morphisms $\{f_n : M_n \to N_n\}_{n \geq 1}$ in $\mathcal{C}$ satisfying the compatibility condition
            $$
            \forall n \geq 1, \quad q_n \circ f_{n+1} = f_n \circ p_n.
            $$
        \end{itemize}
        Composition and identities are defined componentwise. 


        \item The \hldef{category of direct sequences} or \hldef{Ind-category indexed by $\mathbb{N}$}, denoted by \hl{$\mathrm{Ind}_{\mathbb{N}}(\mathcal{C})$}, is defined as follows:
        \begin{itemize}
            \item Objects are sequences $(M_n, i_n)_{n \geq 1}$ where for each $n \geq 1$, $M_n \in \mathrm{Ob}(\mathcal{C})$ and $i_n : M_n \to M_{n+1}$ is a morphism in $\mathcal{C}$.
            \item A morphism $f : (M_n, i_n) \to (N_n, j_n)$ consists of a collection of morphisms $\{f_n : M_n \to N_n\}_{n \geq 1}$ in $\mathcal{C}$ satisfying the compatibility condition
            $$
            \forall n \geq 1, \quad j_n \circ f_n = f_{n+1} \circ i_n.
            $$
        \end{itemize}
        Composition and identities are defined componentwise.
    \end{enumerate}

    Equivalently, $\mathrm{Pro}_{\mathbb{N}}(\mathcal{C})$ and $\mathrm{Ind}_{\bbN}(\calC)$ may be regarded as the functor categories $\operatorname{Fun}(\mathbb{N}^{\op}, \calC)$ and $\operatorname{Fun}(\mathbb{N}, \calC)$ respectively, where $\bbN$ is \CrefAndHyperrefIfExist{lemma:posets_correspond_to_small_filtered_thin_categories}{regarded} as a \CrefAndHyperrefIfExist{definition:partially_ordered_set}{poset} category with objects $1,2,\ldots$ with a unique morphism $n \to m$ if and only if $n \leq m$.
\end{definition}

\begin{context} \label{context:ekedahl_af_ringed_topos_category_of_inverse_sequences}
    Let $(\calC, J, R)$ be a \CrefAndHyperrefIfExist{definition:ringed_site}{ringed site} and let $m \subseteq R$ be a two-sided ideal. Write $S$ for the \CrefAndHyperrefIfExist{definition:topos}{topos} of $(\calC, J)$. Let $\mathrm{Pro}_\bbN(S)$ be the \CrefAndHyperrefIfExist{definition:category_of_inverse_direct_sequences_of_objects_in_a_category}{category of inverse sequences of $S$}; this is equivalent to the \CrefAndHyperrefIfExist{definition:diagram_in_a_category_indexed_by_a_small_category}{functor category $[\bbN^{\op}, S]$} and is a \CrefAndHyperrefIfExist{theorem:functor_category_of_a_grothendieck_topos_indexed_by_a_small_category_is_a_grothendieck_topos}{topos} itself. Regard it as a ringed topos equipped with \CrefAndHyperref{definition:ringed_site_structure_on_category_of_inverse_sequences_of_a_site}{$R_\bullet$}.

\end{context}


\begin{context} \label{context:ekedahl_af_topos_moprhism_from_Pro_N_S_to_S}
    Assume \Cref{context:ekedahl_af_ringed_topos_category_of_inverse_sequences}. Write $\pi: \mathrm{Pro}_\bbN(S) \to S$ for the topos morphism \TODO{topos morphism} given by $\pi_* M_\bullet = \varprojlim M_\bullet$ and $\pi^* M = \{M, \id\}_{n \geq 1}$, i.e. the system consisting of the objects $M$ at each level and the identity maps between them. 
\end{context}

\begin{definition} \label{definition:ringed_site_structure_on_category_of_inverse_sequences_of_a_site}
    % Let $(\calC, J, R)$ be a \CrefAndHyperrefIfExist{definition:ringed_site}{ringed site} and let $m \subseteq R$ be a two-sided ideal. Write $S$ for the \CrefAndHyperrefIfExist{definition:topos}{topos} of $(\calC, J)$. Let $\mathrm{Pro}_\bbN(S)$ be the \CrefAndHyperrefIfExist{definition:category_of_inverse_direct_sequences_of_objects_in_a_category}{category of inverse sequences of $S$}; this is equivalent to the \CrefAndHyperrefIfExist{definition:diagram_in_a_category_indexed_by_a_small_category}{functor category $[\bbN^{\op}, S]$} and is a \CrefAndHyperrefIfExist{theorem:functor_category_of_a_grothendieck_topos_indexed_by_a_small_category_is_a_grothendieck_topos}{topos} itself.
    % Write $\pi: \mathrm{Pro}_\bbN(S) \to S$ for the topos morphism \TODO{topos morphism} given by $\pi_* M_\bullet = \varprojlim M_\bullet$ and $\pi^* M = \{M, \id\}_{n \geq 1}$, i.e. the system consisting of the objects $M$ at each level and the identity maps between them. 
    Assume \Cref{context:ekedahl_af_topos_moprhism_from_Pro_N_S_to_S}.
    There is a ring object, which we might sometimes denote by \hl{$R_\bullet$}, given by the projective system
    $$\cdots \to R/m^{n+1} \to R/m^n \to  \cdots.$$
    When $\mathrm{Pro}_\bbN(S)$ is equipped with $R_\bullet$, $\pi$ is a morphism of ringed topoi. \TODO{morphism of ringed topoi}. In particular, for an object $M$ of the ringed topos $(S, R)$, its pullback $\pi^* M$ is the inverse system $\{M \otimes_R R_n\}_{n \geq 1}$. 
\end{definition}

\begin{definition} \label{definition:essentially_zero_inverse_sequence_of_sheaves_of_abelian_groups_on_a_site}
%    Let $(\calC, J, R)$ be a \CrefAndHyperrefIfExist{definition:ringed_site}{ringed site} and let $m \subseteq R$ be a two-sided ideal. Write $S$ for the \CrefAndHyperrefIfExist{definition:topos}{topos} of $(\calC, J)$. Let $\mathrm{Pro}_\bbN(S)$ be the \CrefAndHyperrefIfExist{definition:category_of_inverse_direct_sequences_of_objects_in_a_category}{category of inverse sequences of $S$}; this is equivalent to the \CrefAndHyperrefIfExist{definition:diagram_in_a_category_indexed_by_a_small_category}{functor category $[\bbN^{\op}, S]$} and is a \CrefAndHyperrefIfExist{theorem:functor_category_of_a_grothendieck_topos_indexed_by_a_small_category_is_a_grothendieck_topos}{topos} itself. Regard it as a ringed topos equipped with \CrefAndHyperref{definition:ringed_site_structure_on_category_of_inverse_sequences_of_a_site}{$R_\bullet$}.
    Assume \Cref{context:ekedahl_af_ringed_topos_category_of_inverse_sequences}

   \begin{enumerate}
    \item We say that an abelian group object $M_\bullet$ of $\mathrm{Pro}_\bbN(S)$ is \hldef{essentially zero} if there is a covering of the \CrefAndHyperrefIfExist{definition:initial_final_zero_objects_of_a_category}{final object} of $S$ such that, when $M_\bullet$ is restricted to each element of the covering, there is for all $n$ an $m \geq n$ such that the map $M_m \to M_n$ in $M_\bullet$ is zero. 

    \item Let $M^\bullet$ be a \CrefAndHyperrefIfExist{definition:chain_complex_of_objects_in_an_additive_category}{complex} of abelian group objects of $\mathrm{Pro}_{\bbN}(S)$. We say that $M^\bullet$ is \hldef{essentially zero} if \CrefAndHyperrefIfExist{definition:homology_and_cohomology_objects_for_a_chain_complex_in_an_additive_category}{$H^i(M^\bullet)$} is essentially zero for all $i$.

    The category of essentially zero complexes form a \CrefAndHyperrefIfExist{definition:triangulated_category}{triangulated subcategory} of the category of complexes of abelian group objects of $\mathrm{Pro}_{\bbN}(S)$.

    One may say that $M^\bullet$ is \hldef{essentially bounded}, \hldef{essentially bounded from below}, or \hldef{essentially bounded from above}, if $H^m(M^\bullet)$ is essentially zero for all $m$ with $|m| \gg 0$, $-m \gg 0$, and $m \gg 0$ respectively. 

    The \CrefAndHyperrefIfExist{definition:derived_category_of_an_abelian_category}{(derived) categories} of essentially bounded, essentially bounded from below, and essentially bounded from above complexes of $R$-module objects of $\mathrm{Pro}_{\bbN}(S)$ may be denoted by \hl{$D^{eb}(\mathrm{Pro}_{\bbN}(S), R_\bullet)$}, \hl{$D^{e+}(\mathrm{Pro}_{\bbN}(S), R_\bullet)$}, and \hl{$D^{e-}(\mathrm{Pro}_{\bbN}(S), R_\bullet)$} respectively\footnote{In \cite{ekedahl_af}, these are denoted by $D^{eb}(S^\bbN-R_\bullet)$, etc.}. 
    
   \end{enumerate}
\end{definition}

\begin{notation} \label{notation:derived_category_of_category_of_modules_of_a_ringed_site}
    Let $(S,R)$ be a \CrefAndHyperrefIfExist{definition:ringed_site}{ringed site}. Write \hl{$D(S, R)$} for the \CrefAndHyperrefIfExist{definition:derived_category_of_an_abelian_category}{derived category} of the category of \CrefAndHyperrefIfExist{definition:module_over_a_sheaf_of_rings_on_a_site}{$R$-modules}. Accordingly, \hl{$D^+(S, R)$}, \hl{$D^-(S,R)$}, and \hl{$D^b(A)$} may be used to denote the full subcategories of \CrefAndHyperrefIfExist{definition:bounded_complexes_on_an_additive_category_and_homologically_bounded_objects_on_an_abelian_category}{cohomologically bounded below, cohomologically bounded above, and cohomologically bounded complexes} respectively, cf. \Cref{definition:derived_category_of_an_abelian_category}
    % $D(\mathrm{Pro}_\bbN(S), R_\bullet)$
\end{notation}


\begin{definition}[{\cite[Section 1]{ekedahl_af}}]  \label{definition:ekedahls_condition_A_and_B}
    Assume \Cref{context:ekedahl_af_topos_moprhism_from_Pro_N_S_to_S}.
%    Let $(\calC, J, R)$ be a \CrefAndHyperrefIfExist{definition:ringed_site}{ringed site} and let $m \subseteq R$ be a two-sided ideal. Write $S$ for the \CrefAndHyperrefIfExist{definition:topos}{topos} of $(\calC, J)$. Let $\mathrm{Pro}_\bbN(S)$ be the \CrefAndHyperrefIfExist{definition:category_of_inverse_direct_sequences_of_objects_in_a_category}{category of inverse sequences of $S$}; this is equivalent to the \CrefAndHyperrefIfExist{definition:diagram_in_a_category_indexed_by_a_small_category}{functor category $[\bbN^{\op}, S]$} and is a \CrefAndHyperrefIfExist{theorem:functor_category_of_a_grothendieck_topos_indexed_by_a_small_category_is_a_grothendieck_topos}{topos} itself.
%    Write $\pi: \mathrm{Pro}_\bbN(S) \to S$ be the topos morphism \TODO{topos morphism} given by $\pi_* M_\bullet = \varprojlim M_\bullet$ and $\pi^* M = \{M, \id\}_{n \geq 1}$, i.e. the system consisting of the objects $M$ at each level and the identity maps between them. Regard it as a ringed topos equipped with \CrefAndHyperref{definition:ringed_site_structure_on_category_of_inverse_sequences_of_a_site}{$R_\bullet$}.
   \begin{enumerate}
    \item We may say that $(\calC, J, R)$ or $(S,R)$ \hldef{satisfies Ekedahl's condition $A$} if the following holds: there exists a class $S^{\text{gen}}$ of \CrefAndHyperrefIfExist{definition:generator_of_a_category}{generators of $S$} and an integer $N$ such that for all $T \in S^{\text{gen}}$, all $R/m$-\CrefAndHyperrefIfExist{definition:module_over_a_sheaf_of_rings_on_a_site}{modules} $M$, and $i > N$, we have $H^i(T,M) = 0$.

    \item We may say that $(\calC, J, R)$ or $(S,R)$ \hldef{satisfies Ekedahl's condition $B$} if the following hold: 
    \begin{itemize}
        \item there is an integer $N$ such that for every $n \geq 1$ and locally on $(\calC, J)$, there is a \CrefAndHyperrefIfExist{definition:left_right_resolution_of_a_class_of_objects_in_an_abelian_category}{left resolution} $F_n^\bullet \to R/m^n$ of $R/m^n$ by a complex of (right) $R$-modules of finitely generated free objects such that $\pi^* F_n^\bullet \to \pi^*(R/m^n)$ is a resolution modulo essentially zero systems \TODO{essentially zero system}, where $\pi: (\mathrm{Pro}_{\bbN}(S), R_\bullet) \to (S,R)$ is considered as a morphism of ringed topoi. 

        \item $m^n/m^{n+1}$ is, locally on $(\calC, J)$, of finite Tor-dimension over $R/m$. \TODO{finite Tor dimension}
    \end{itemize}
    
   \end{enumerate}  
\end{definition}

\begin{notation}[{\cite[After Lemma 1.1]{ekedahl_af}}] \label{notation:ekedahl_af_tau_M_either_pro_object_of_truncations_or_M_depending_on_whether_ekedahls_condition_A_holds}
    Assume \Cref{context:ekedahl_af_ringed_topos_category_of_inverse_sequences}
%    Let $(\calC, J, R)$ be a \CrefAndHyperrefIfExist{definition:ringed_site}{ringed site} and let $m \subseteq R$ be a two-sided ideal. Write $S$ for the \CrefAndHyperrefIfExist{definition:topos}{topos} of $(\calC, J)$. Let $\mathrm{Pro}_\bbN(S)$ be the \CrefAndHyperrefIfExist{definition:category_of_inverse_direct_sequences_of_objects_in_a_category}{category of inverse sequences of $S$}; this is equivalent to the \CrefAndHyperrefIfExist{definition:diagram_in_a_category_indexed_by_a_small_category}{functor category $[\bbN^{\op}, S]$} and is a \CrefAndHyperrefIfExist{theorem:functor_category_of_a_grothendieck_topos_indexed_by_a_small_category_is_a_grothendieck_topos}{topos} itself. Regard it as a ringed topos equipped with \CrefAndHyperref{definition:ringed_site_structure_on_category_of_inverse_sequences_of_a_site}{$R_\bullet$}.

   Given $M \in D^{e+}(\mathrm{Pro}_\bbN(S), \bbZ_\bullet)$, \cite{ekedahl_af} lets \hl{$\tau(M)$} denote the following:
   \begin{itemize}
    \item If $(\calC, J, R)$ satisifes \CrefAndHyperrefIfExist{definition:ekedahls_condition_A_and_B}{Ekedahl's condition $A$}, then $\tau(M) = M$. 
    \item Otherwise, $\tau(M)$ is the pro-object
    $$(\cdots \to \tau_{i - 1} M \to \tau_{\geq i} M \to \tau_{i +1} M \to \cdots)$$
    (\Cref{definition:canonical_truncation_of_chain_complexes_of_objects_in_an_abelian_category}).
   \end{itemize}
   \TODO{Talk about $R\pi_* M$}
\end{notation}


\begin{definition}[{\cite[Definition 2.1]{ekedahl_af}}] \label{definition:ekedahl_af_R_complex_negligible_normalised_objects_of_the_derived_category_of_modules_over_ring_of_category_of_inverse_sequences_essentially_an_isomorphism}
    Assume \Cref{context:ekedahl_af_topos_moprhism_from_Pro_N_S_to_S}
%    Let $(\calC, J, R)$ be a \CrefAndHyperrefIfExist{definition:ringed_site}{ringed site} and let $m \subseteq R$ be a two-sided ideal. Write $S$ for the \CrefAndHyperrefIfExist{definition:topos}{topos} of $(\calC, J)$. Let $\mathrm{Pro}_\bbN(S)$ be the \CrefAndHyperrefIfExist{definition:category_of_inverse_direct_sequences_of_objects_in_a_category}{category of inverse sequences of $S$}; this is equivalent to the \CrefAndHyperrefIfExist{definition:diagram_in_a_category_indexed_by_a_small_category}{functor category $[\bbN^{\op}, S]$} and is a \CrefAndHyperrefIfExist{theorem:functor_category_of_a_grothendieck_topos_indexed_by_a_small_category_is_a_grothendieck_topos}{topos} itself.
%    Write $\pi: \mathrm{Pro}_\bbN(S) \to S$ for the topos morphism \TODO{topos morphism} given by $\pi_* M_\bullet = \varprojlim M_\bullet$ and $\pi^* M = \{M, \id\}_{n \geq 1}$, i.e. the system consisting of the objects $M$ at each level and the identity maps between them. Regard it as a ringed topos equipped with \CrefAndHyperref{definition:ringed_site_structure_on_category_of_inverse_sequences_of_a_site}{$R_\bullet$}.

    Let $M,N$ be objects \CrefAndHyperrefIfExist{notation:derived_category_of_category_of_modules_of_a_ringed_site}{$D(\mathrm{Pro}_\bbN(S), R_\bullet)$}.
    \begin{enumerate}
        \item We may say that $M$ is an \hldef{$R$-complex} if $\pi^*(R_1) \otimes_{R_\bullet}^L M$ \TODO{derived tensor product} is essentially constant \TODO{essentially constant}

        \item $M$ is \hldef{negligible} if $\pi^*(R_1) \otimes_{R_\bullet}^L M$ is \CrefAndHyperrefIfExist{definition:essentially_zero_inverse_sequence_of_sheaves_of_abelian_groups_on_a_site}{essentially zero}.


        \item $M$ is \hldef{normalised} if $\tau(\widehat{M}) \to \tau(M)$ (\Cref{notation:ekedahl_af_tau_M_either_pro_object_of_truncations_or_M_depending_on_whether_ekedahls_condition_A_holds})\TODO{$\widehat{M}$; what is the morphism $\tau(\widehat{M}) \to \tau(M)$} is a \CrefAndHyperrefIfExist{definition:quasi_isomorphism_of_chain_complexes_of_objects_in_an_abelian_category}{quasi-isomorphism}. 

        \item A morphism $M \to N$ is \hldef{essentially an isomorphism} if it has a negligible \CrefAndHyperrefIfExist{definition:mapping_cone_of_a_map_of_chain_cochain_complexes}{mapping cone}.
    \end{enumerate}
\end{definition}

\begin{context} \label{context:ekedahl_af_superscript_for_ekedahls_categories}
    Assume \Cref{context:ekedahl_af_topos_moprhism_from_Pro_N_S_to_S}; in particular, \Cref{context:ekedahl_af_ringed_topos_category_of_inverse_sequences} is also assumed. Assume that \CrefAndHyperref{definition:ekedahls_condition_A_and_B}{Ekedahl's condition $A$ or $B$ holds}.
    Let 
    $$
    * = \begin{cases} - &\text{if Ekedahl's condition A holds} \\ e+ &\text{if Ekedahl's condition B holds} \\ (\text{blank}) &\text{if both Ekedahl's condition A and condition B hold}. \end{cases}
    $$
    In particular, we will consider the category \CrefAndHyperrefIfExist{notation:derived_category_of_category_of_modules_of_a_ringed_site}{$D^-(\mathrm{Pro}_\bbN, R_\bullet)$}, \CrefAndHyperrefIfExist{definition:essentially_zero_inverse_sequence_of_sheaves_of_abelian_groups_on_a_site}{$D^{e+}(\mathrm{Pro}_\bbN, R_\bullet)$}, or \CrefAndHyperrefIfExist{notation:derived_category_of_category_of_modules_of_a_ringed_site}{$D(\mathrm{Pro}_\bbN, R_\bullet)$}, depending on which combination of A and B holds. 
\end{context}

\begin{proposition}[{\cite[Proposition 2.2]{ekedahl_af}}]
    Assume \Cref{context:ekedahl_af_superscript_for_ekedahls_categories}.
    \begin{enumerate}
        \item $M$ is an \CrefAndHyperrefIfExist{definition:ekedahl_af_R_complex_negligible_normalised_objects_of_the_derived_category_of_modules_over_ring_of_category_of_inverse_sequences_essentially_an_isomorphism}{$R$-complex} if and only if $\tau(\widehat{M}) \to \tau(M)$ (\Cref{notation:ekedahl_af_tau_M_either_pro_object_of_truncations_or_M_depending_on_whether_ekedahls_condition_A_holds}) \TODO{$\widehat{M}$} is \CrefAndHyperrefIfExist{definition:ekedahl_af_R_complex_negligible_normalised_objects_of_the_derived_category_of_modules_over_ring_of_category_of_inverse_sequences_essentially_an_isomorphism}{essentially an isomorphism}.

        \item $M$ is \CrefAndHyperrefIfExist{definition:ekedahl_af_R_complex_negligible_normalised_objects_of_the_derived_category_of_modules_over_ring_of_category_of_inverse_sequences_essentially_an_isomorphism}{normalised} if and only if $R_n \otimes_{R_{n+1}}^L i_{n+1}^* M \to i_n^* M$ is an isomorphism for all $n$. \TODO{tensor product, $i_n^*$}

        \item Let $?$ be $-$, $+$, or blank if Ekedahl's condition A, condition B, or conditions A and B hold respectively. Let \CrefAndHyperrefIfExist{notation:derived_category_of_category_of_modules_of_a_ringed_site}{$M' \in D^?(S,R)$}.  The object $L\pi^* M'$ \TODO{$L\pi^*$} belongs to $D^?(\mathrm{Pro}_\bbN, R_\bullet)$. In particular, $\widehat{M}$ is normalised. 
    \end{enumerate}
\end{proposition}


\begin{definition}[cf. {\cite[Definition 2.5]{ekedahl_af}}] \label{definition:ekedahls_adic_category_of_a_ringed_site}
    \TODO{I don't think this is quite the right definition of Ekedahl's category; somehow, the conditoin that the transition maps induce isomorphisms need to be incorporated.}
    Assume \Cref{context:ekedahl_af_topos_moprhism_from_Pro_N_S_to_S}.

    Denote by \hl{$D_{\mathrm{Ek}}(S, R)$} or \hl{$D_{\mathrm{Ek}}(\calC, R)$} the category whose objects are \CrefAndHyperrefIfExist{definition:ekedahl_af_R_complex_negligible_normalised_objects_of_the_derived_category_of_modules_over_ring_of_category_of_inverse_sequences_essentially_an_isomorphism}{$R$-complexes} and morphisms are the morphisms \CrefAndHyperrefIfExist{notation:ekedahl_af_tau_M_either_pro_object_of_truncations_or_M_depending_on_whether_ekedahls_condition_A_holds}{$\tau(M) \to \tau(N)$} with \CrefAndHyperrefIfExist{definition:essentially_zero_inverse_sequence_of_sheaves_of_abelian_groups_on_a_site}{essential isomorphisms} inverted. For $* \in \{b,+,-\}$, Denote by \hl{$D_{\mathrm{Ek}}^*(S,R)$} or \hl{$D_{\mathrm{Ek}}^*(\calC,R)$} the full subcategory whose objects are the complexes which are \CrefAndHyperrefIfExist{definition:essentially_zero_inverse_sequence_of_sheaves_of_abelian_groups_on_a_site}{essentially bounded, essentially bounded from below, and essentially bounded from above} respectively. 

    It is appropriate to call these categories \hldef{``Ekedahl's adic categories''}. By abuse of language in conflict with the terminology of \Cref{definition:ekedahl_af_R_complex_negligible_normalised_objects_of_the_derived_category_of_modules_over_ring_of_category_of_inverse_sequences_essentially_an_isomorphism}, \cite{ekedahl_af} calls the objects of $D_{\mathrm{Ek}}(S,R)$ \hldef{$R$-complexes}.
\end{definition}

\begin{theorem}[{\cite[Proposition 2.7]{ekedahl_af}}]
    Assume \Cref{context:ekedahl_af_topos_moprhism_from_Pro_N_S_to_S}. Let $* \in \{b,+,-\}$. 
    \begin{enumerate}
        \item $R_1 \otimes_R^L (-): D_{\mathrm{Ek}}(S,R) \to D(S,R_1)$ \TODO{tensor product} (\Cref{context:ekedahl_af_ringed_topos_category_of_inverse_sequences}, \Cref{definition:ekedahls_adic_category_of_a_ringed_site}, \Cref{notation:derived_category_of_category_of_modules_of_a_ringed_site}) is conservative. \TODO{conservative functor}

        \item Letting $D_{\mathrm{norm}}^*(\mathrm{Pro}_\bbN(S), R_\bullet) \subset D^*(\mathrm{Pro}_\bbN(S), R_\bullet)$ (\Cref{definition:category_of_inverse_direct_sequences_of_objects_in_a_category}, \Cref{notation:derived_category_of_category_of_modules_of_a_ringed_site}) be the subcategory of \CrefAndHyperrefIfExist{definition:ekedahl_af_R_complex_negligible_normalised_objects_of_the_derived_category_of_modules_over_ring_of_category_of_inverse_sequences_essentially_an_isomorphism}{normalised $R$-complexes}. The canonical functor
        $$D_{\mathrm{norm}}^*(\mathrm{Pro}_\bbN(S), R_\bullet) \to D_{\mathrm{Ek}}^*(S-R)$$
        \TODO{What is this canonical functor}
        and the functor
        $$(\widehat{-}): D_{\mathrm{Ek}}^*(S-R) \to D_{\mathrm{norm}}^*(\mathrm{Pro}_\bbN(S), R_\bullet)$$
        are inverse \CrefAndHyperrefIfExist{definition:equivalence_of_categories}{equivalences of categories}.
    \end{enumerate}
\end{theorem}

\begin{theorem}[{\cite[Theorem 6.3]{ekedahl_af}}]
    \TODO{regular ring, scheme}
     Let $S$ be a regular scheme of \CrefAndHyperrefIfExist{definition:dimension_of_a_scheme}{dimension} at most $1$. Let $R$ be a \CrefAndHyperrefIfExist{definition:commutative_ring}{commutative}, \CrefAndHyperrefIfExist{definition:local_ring}{local}, regular ring with maximal ideal $m$ such that the residue field $R_1 = R/m$ is of positive characteristic invertible in $\calO_s$. Write $R$ also for the  \CrefAndHyperrefIfExist{definition:constant_sheaf_on_a_site_with_sheafification}{constant sheaf} of $R$ on \CrefAndHyperrefIfExist{definition:small_etale_site_of_a_scheme}{$S_{\et}$}. Let $c$ be the full subcategory of the category of sheaves of \CrefAndHyperrefIfExist{definition:module_over_a_sheaf_of_rings_on_a_site}{$R_1$-modules} whose objects are \CrefAndHyperrefIfExist{definition:constructible_sheaf_on_a_small_site_on_a_scheme_or_a_topological_space}{constructible}.

     Let $X,Y$ be $S$-schemes
     \begin{enumerate}
        \item The category \TODO{Ekedahl's constructible, bounded category}$D_c^b(X_{\et}, R)$ and the functor
        $$R_1 \otimes_R^L (-): D_c^b(X_{\et}, R) \to D_c^b(X_{\et}, R_1)$$
        is a conservative triangulated functor \TODO{conservative triangulated functor}

        \item $D_c^b(X_{\et}, R)$ has a $t$-structure \TODO{} whose heart is equivalent to the category of $m$-adic constructible sheaves and for which every object of $D_c^b(X_{\et}, R)$ has finite amplitude, i.e. $D_c^b(X_{\et}, R) = \bigcup_{i \in \bbZ} D_c^b(X_{\et}, R)^{\geq i} = \bigcup_{i \in \bbZ} D_c^b(X_{\et}, R)^{\leq i}$. 

        \item \TODO{tensor and RHom}
        \item \TODO{pushforward, pullback}
        \item \TODO{perverse sheaves}
     \end{enumerate}
\end{theorem}


\section{Pro-\'etale topology}

\begin{definition}[Diagonal morphism of a morphism of schemes] \label{definition:diagonal_morphism_of_a_morphism_of_schemes}
    Let $f : X \to Y$ be a \CrefAndHyperrefIfExist{definition:morphism_of_schemes}{morphism of schemes}.

    The \hldef{diagonal morphism associated to $f$} is the morphism
    \hlin{$$\Delta_f : X \to X \times_Y X$$}
    \CrefIfExists{definition:cartesian_product_of_two_objects_in_a_category_over_an_object} which is the \CrefAndHyperrefIfExist{definition:diagonal_morphism_of_an_object_over_an_object_in_a_category}{diagonal morphism} associated to the morphism $f$ in the category of schemes.

    In other words, $\Delta_f$ is defined as the unique morphism induced by the universal property of the \CrefAndHyperrefIfExist{definition:cartesian_product_of_two_objects_in_a_category_over_an_object}{fiber product} making the following diagram commute:
    \[
    \begin{tikzcd}
    X \arrow[rd, "\Delta_f"] \ar[rrd, bend left, "\id_X"], \ar[ddr, bend right, "\id_X"] & & \\
    & X \times_Y X \arrow[d, "p_1"] \arrow[r, "p_2"] & X \arrow[d, "f"] \\
    & X \arrow[r, "f"] & Y
    \end{tikzcd}
    % \begin{tikzcd}
    %     X \arrow[rd, "\Delta_f"] & & 
    %   & X \times_Y X \arrow[d, "{p_1}"] \arrow[r,  "{p_2}"] \ar[r] & X \ar[d, "f"] \\
    % & X \ar[r, "f"] & Y
    % \end{tikzcd}
    \]
    where $p_1$ and $p_2$ are the natural projections from the fiber product.

    In other words, $\Delta_f$ is given by the pair of identity morphisms $(\mathrm{id}_X, \mathrm{id}_X)$ over $Y$:
    \[
    \Delta_f := (\mathrm{id}_X, \mathrm{id}_X) : X \to X \times_Y X.
    \]
\end{definition}

\begin{definition}[Weakly étale morphism of schemes, {\cite[Definition 2.2]{bhatt_scholze_pfgs}}] \label{definition:weakly_etale_morphism_of_schemes}
        A \CrefAndHyperrefIfExist{definition:morphism_of_schemes}{morphism $f: X \to Y$ of schemes} is \hldef{weakly étale} if it and its \CrefAndHyperrefIfExist{definition:diagonal_morphism_of_a_morphism_of_schemes}{diagonal} are both \CrefAndHyperrefIfExist{definition:flat_morphism_of_schemes}{flat}.
\end{definition}




\begin{definition}[Pro-\'etale site of a scheme, {\cite[Definition 1.2]{bhatt_scholze_pfgs}}] \label{definition:pro_etale_site_of_a_scheme}
    Let $X$ be a \CrefAndHyperrefIfExist{definition:scheme}{scheme}. The \hldef{pro-\'etale site} \hl{$X_{\proet}$} is the site whose underlying category is the category of \CrefAndHyperrefIfExist{definition:weakly_etale_morphism_of_schemes}{weakly \'etale} $X$-schemes and whose covers are covers in the \CrefAndHyperrefIfExist{definition:big_fpqc_site_of_a_scheme}{fpqc topology}, i.e. a family $\{Y_i \to Y\}$ of maps on $X_{\proet}$ is a covering family if any open affine in $Y$ is mapped onto by an open affine in $\coprod_i Y_i$. 
\end{definition}

\begin{theorem}
    \TODO{the categories}
    Let $\ell$ be a prime number. Let $E$ be an algebraic extension of $\bbQ_\ell$. The triangulated category $D_{\mathrm{cons}}(X_{\proet}, E)$ is equivalent to the triangulated category traditionally called $D_c^b(X,E)$
\end{theorem}


\appendix

\subsection{Categorical definitions}

\begin{definition}[Category] \label{definition:category}
    A 
    \defin{category}{category}{
        name={Category},
        description={A nice enough collection of objects and morphisms (\Cref{definition:category})},
    }
    \hldef{category} $\mathcal{C}$ consists of the following data:
    \begin{itemize}
        \item A class of \defin{objects}{object_of_a_category}{
            name={Object of a category},
            description={\Cref{definition:category}},
        }
        denoted \notat{\operatorname{Ob}(\mathcal{C})}{class_of_objects_of_a_category}{
            name={$\operatorname{Ob}(\mathcal{C})$},
            description={Class of objects of a category $\calC$ \Cref{definition:category}},
            sort={Ob},
        }.
        % \hl{$\operatorname{Ob}(\mathcal{C})$}.
        \item For each pair of objects $X, Y \in \operatorname{Ob}(\mathcal{C})$, a class
        \notatin{\operatorname{Hom}_{\mathcal{C}}(X,Y)}{class_of_morphisms_between_two_objects_of_a_category}
        {
            name={$\operatorname{Hom}_{\mathcal{C}}(X,Y)$},
            description={Class of morphisms between objects $X$ and $Y$ of the category $\calC$ (\Cref{definition:category})},
            sort={Hom},
        }
        % $$\hlin{\operatorname{Hom}_{\mathcal{C}}(X,Y)}$$
        of \defin{morphisms}{morphism_between_objects_of_a_category}{
            name={Morphism between objects of a category},
            description={(\Cref{definition:category})},
        }
        (also called 
        \defin{arrows}{arrow_between_objects_of_a_category}{
            name={Arrow between objects of a category},
            description={Synonym for morphism (\Cref{definition:category})},
        }
        or
        \defin{homs}{hom_between_objects_of_a_category}{
            name={Hom between objects of a category},
            description={Synonym for morphism (\Cref{definition:category})},
        }). If the category $\calC$ is clear, then this \hldef{hom-class} is also denoted by \hl{$\operatorname{Hom}(X,Y)$}. It may also be denoted by \hl{$\operatorname{hom}_{\mathcal{C}}(X,Y)$} or \hl{$\operatorname{hom}(X,Y)$}, especially to distinguish from other types of hom's (e.g. \hyperrefIfExists{definition:internal_hom_object_in_a_category}{internal hom's})
        \item For each triple of objects $X,Y,Z$, a composition law
        $$ \circ : \operatorname{Hom}_{\mathcal{C}}(Y,Z) \times \operatorname{Hom}_{\mathcal{C}}(X,Y) \to \operatorname{Hom}_{\mathcal{C}}(X,Z), $$
        denoted \hl{$(g,f) \mapsto g \circ f$}.
        \item For each object $X$, an \hldef{identity morphism}
        $$\hlin{\operatorname{id}_X \in \operatorname{Hom}_{\mathcal{C}}(X,X).}$$
    \end{itemize}
    These data satisfy the following axioms:
    \begin{itemize}
        \item (Associativity) For all morphisms $f \in \operatorname{Hom}_{\mathcal{C}}(X,Y)$, $g \in \operatorname{Hom}_{\mathcal{C}}(Y,Z)$, and $h \in \operatorname{Hom}_{\mathcal{C}}(Z,W)$, 
        $$
        h \circ (g \circ f) = (h \circ g) \circ f.
        $$
        \item (Identity) For all $f \in \operatorname{Hom}_{\mathcal{C}}(X,Y)$,
        $$
        \operatorname{id}_Y \circ f = f = f \circ \operatorname{id}_X.
        $$
    \end{itemize}
    One often writes \hl{$X \in \calC$} synonymously with $X \in \Ob(\calC)$, i.e. to denote that $X$ is an object of of $\calC$. 

    We may call a category as above an \hldef{ordinary category} to distinguish this notion from the notions of \hyperrefIfExists{definition:category_enriched_in_a_monoidal_category}{\emph{categories enriched in monoidal categories}} or higher/$n$-categories.
    \TODO{TODO: define $n$-categories}

    A category as defined above may be called called a \hldef{large category} or a \hldef{class category} to emphasize that the hom-classes may be proper classes rather than sets (note, however, that the possibility that hom-classes are sets is not excluded for large categories). Accordingly, a \hldef{category} may often refer to a \hyperrefIfExists{definition:locally_small_category}{locally small category}\CrefIfExists{definition:locally_small_category}, which is a category whose hom-classes are all sets.
\end{definition}

% Later on, we refer to the \gls{category} again.

\begin{definition}[Grothendieck Universe] \label{definition:grothendieck_universe}
    Let $U$ be a set. We say $U$ is a \hldef{Grothendieck universe} (or just a \hldef{universe}) if the following conditions hold:
    \begin{enumerate}
        \item If $x \in U$ and $y \in x$, then $y \in U$ (transitivity).
        \item If $x,y \in U$, then $\{x,y\} \in U$ (closed under pair formation).
        \item If $x \in U$, then the power set $\mathcal{P}(x) \in U$.
        \item If $I \in U$ and $(x_\alpha)_{\alpha \in I}$ is a family with each $x_\alpha \in U$, then $\bigcup_{\alpha \in I} x_\alpha \in U$.
    \end{enumerate}
    A set $X$ is called \hldef{$U$-small} or a \hldef{$U$-set} if $X \in U$.
\end{definition}


\begin{definition}[Coarser and finer Grothendieck topologies] \label{definition:coarser_and_finer_grothendieck_topologies}
    Let $\mathcal{C}$ be a \CrefAndHyperrefIfExist{definition:category}{category} and let $\tau, \tau'$ be two \CrefAndHyperrefIfExist{theorem:grothendieck_L_function_of_a_complex_is_a_determinant_of_the_total_compactly_supported_cohomology_of_the_complex}{Grothendieck topologies} on $\mathcal{C}$.

    We say that \hldef{$\tau$ is finer than $\tau'$} (or equivalently, \hldef{$\tau'$ is coarser than $\tau$}) if for every object $U \in \mathcal{C}$ every $\tau$-covering of $U$ is also a $\tau'$-covering of $U$.

    If $\tau$ and $\tau'$ are not comparable by inclusion, i.e., neither is finer or coarser than the other, they are said to be \hldef{incomparable}.
\end{definition}



\begin{definition}[Locally small category] \label{definition:locally_small_category}
A \hyperrefIfExists{definition:category}{(large) category}\CrefIfExists{definition:category} $\mathcal{C}$ is called a \hldef{locally small category} if for every pair of objects $X, Y \in \operatorname{Ob}(\mathcal{C})$, the collection $\operatorname{Hom}_{\mathcal{C}}(X,Y)$ of morphisms between them is a (\CrefAndHyperrefIfExist{definition:small_set}{small}) \emph{set} (as opposed to a proper class). In other words, each hom-class is a set and may even be called a \hldef{hom-set}.

In some contexts, a locally small category may simply be called a \hldef{category}, especially when genuinely large categories are not considered.

A category $\mathcal{C}$ is called a \hldef{small category} if it is a locally small category and the class $\operatorname{Ob}(\mathcal{C})$ of objects is a set.

\TextIfExists{definition:grothendieck_universe}{
Given a \hyperrefIfExists{definition:grothendieck_universe}{universe}\CrefIfExists{definition:grothendieck_universe} $U$, we can define the notion of a \hldef{$U$-locally small category} and of a \hldef{$U$-small category} similarly. More explicitly, 
\begin{enumerate}
    \item a $U$-locally small category is a category such that for every pair of objects $X, Y \in \operatorname{Ob}(\mathcal{C})$, the collection $\operatorname{Hom}_{\mathcal{C}}(X,Y)$ of morphisms between them is a $U$-set.
    \item a $U$-small category is a category such that $\operatorname{Ob}(\mathcal{C})$ is a $U$-set and for every pair of objects $X, Y \in \operatorname{Ob}(\mathcal{C})$, the collection $\operatorname{Hom}_{\mathcal{C}}(X,Y)$ of morphisms between them is a $U$-set; in particular the collection of all objects and morhpisms in a $U$-small category is a $U$-set.
\end{enumerate}
}
\end{definition}

\begin{remark}
    Many ``concrete'' categories considered in ``classical mathematics'' or outside of more ``abstract'' category theory tend to be locally small. For example, the categories of sets, groups, $R$-modules, vector spaces, topological spaces, schemes, manifolds, sheaves on ``small enough'' sites are all locally small.
\end{remark}

\begin{definition} \label{definition:equivalence_of_categories}
An \hldef{equivalence of categories} between two \CrefAndHyperrefIfExist{definition:category}{(large) categories} $\mathcal{C}$ and $\mathcal{D}$ consists of a pair of \CrefAndHyperrefIfExist{definition:functor_between_categories}{functors}
$$F : \mathcal{C} \to \mathcal{D} \quad \text{and} \quad G : \mathcal{D} \to \mathcal{C}$$
together with \CrefAndHyperrefIfExist{definition:natural_transformation_between_functors_between_categories}{natural isomorphisms}
$$\eta : \mathrm{Id}_{\mathcal{C}} \xrightarrow{\sim} G \circ F \quad \text{and} \quad \epsilon : F \circ G \xrightarrow{\sim} \mathrm{Id}_{\mathcal{D}}.$$
\CrefIfExists{definition:identity_functor_on_a_category} Such functors $F$ and $G$ may be called \hldef{(natural) inverses of each other}.

When $\calC$ and $\calD$ are \CrefAndHyperrefIfExist{definition:locally_small_category}{locally small categories}, $F$ is an equivalence of categories if and only if $F$ is \CrefAndHyperrefIfExist{definition:full_and_faithful_functor_between_locally_small_categories}{fully faithful} and \CrefAndHyperrefIfExist{definition:essentially_surjective_functor_between_categories}{essentially surjective}
\end{definition}

\begin{lemma} \label{lemma:category_of_presheaves_on_a_small_category_of_locally_small_value_is_locally_small}
    Let $\calC$ be a \hyperrefIfExists{definition:locally_small_category}{small category}\CrefIfExists{definition:locally_small_category} (resp. $U$-small category where $U$ is some \hyperrefIfExists{definition:grothendieck_universe}{universe}\CrefIfExists{definition:grothendieck_universe}) and let $\calA$ be a \CrefAndHyperrefIfExist{definition:locally_small_category}{locally small} category (resp. $U$-locally small category). The \hyperrefIfExists{definition:presheaf_on_a_category}{presheaf category $\PreShv(\calC, \calA)$}\CrefIfExists{definition:presheaf_on_a_category} is locally small (resp. $U$-locally small).
\end{lemma}
\begin{proof}
    A morphism $\calF \to \calG$ in $\PreShv(\calC, \calA)$ is a \hyperrefIfExists{definition:natural_transformation_between_functors_between_categories}{natural transformation}\CrefIfExists{definition:natural_transformation_between_functors_between_categories} of the functors $\calF, \calG: \calC^{\op} \to \calA$. Such a natural transformation is encoded by a family $(\eta_C)_C$ of morphisms (satisfying certain conditions) $\eta_C: \calF(C) \to \calG(C)$ in $\calA$ over objects $C$ of $\calC^{\op}$. The product $\prod_{C \in \Ob \calC^{\op}} \Hom_{\calA}(\calF(C), \calG(C))$ is a product of ($U$-small) sets indexed by a ($U$-small) set, and the collection of natural transformations is a subset of this set. Therefore, $\Hom_{\PreShv(\calC, \calA)}(\calF, \calG)$ is a ($U$-small) set.  
\end{proof}


\begin{definition} \label{definition:adjoint_functors_between_categories_unit_counit_of_adjoint_functors}
Let $\mathcal{C}$ and $\mathcal{D}$ be \CrefAndHyperrefIfExist{definition:category}{categories}. Let $F : \mathcal{C} \to \mathcal{D}$ and $G : \mathcal{D} \to \mathcal{C}$ be functors. 

An \hldef{adjunction between $F$ and $G$} consists of two \CrefAndHyperrefIfExist{definition:natural_transformation_between_functors_between_categories}{natural transformations}: $\eta : \mathrm{Id}_{\mathcal{C}} \implies GF$ (the \hldef{unit}), and  $\varepsilon : FG \implies \mathrm{Id}_{\mathcal{D}}$ (the \hldef{counit})

These must satisfy the triangle identities: For every object $X \in \mathcal{C}$ 
and $Y \in \mathcal{D}$, 
$$\varepsilon_{FX} \circ F(\eta_X) = \text{id}_{FX}$$
$$G(\varepsilon_Y) \circ \eta_{GY} = \text{id}_{GY}.$$
In diagrammatic form, the triangle identities assert that the following are commutative diagrams:
\begin{center}
\begin{tikzcd}
F(X) \arrow[r, "F(\eta_X)"] \arrow[rd, "\text{id}_{F(X)}"'] & FGF(X) \arrow[d, "\varepsilon_{F(X)}"] \\
& F(X)
\end{tikzcd}
\begin{tikzcd}
G(Y) \arrow[r, "\eta_{G(Y)}"] \arrow[rd, "\text{id}_{G(Y)}"'] & GFG(Y) \arrow[d, "G(\varepsilon_Y)"] \\
& G(Y)
\end{tikzcd}
\end{center}

We say that $F$ is a \hldef{left adjoint to $G$} and $G$ is a \hldef{right adjoint to $F$} (written \hl{$F \dashv G$}). 

% for every object $A$ in $\mathcal{C}$ and $B$ in $\mathcal{D}$ there is a \CrefAndHyperrefIfExist{definition:natural_transformation_between_functors_between_categories}{natural isomorphism}
% \begin{align*}
% \operatorname{Hom}_{\mathcal{D}}(F(A), B) \cong \operatorname{Hom}_{\mathcal{C}}(A, G(B))
% \end{align*}
% that is natural in both $A$ and $B$.


In the case that $\mathcal{C}$ and $\mathcal{D}$ are \CrefAndHyperrefIfExist{definition:locally_small_category}{locally small} categories (or $U$-locally small categories if a \CrefAndHyperrefIfExist{definition:grothendieck_universe}{universe} $U$ is available), we have an adjunction $F \dashv G$ if and only if for every object $X$ in $\mathcal{C}$ and $Y$ in $\mathcal{D}$ there is a \CrefAndHyperrefIfExist{definition:natural_transformation_between_functors_between_categories}{natural isomorphism}
\begin{align*}
\operatorname{Hom}_{\mathcal{D}}(F(X), Y) \cong \operatorname{Hom}_{\mathcal{C}}(X, G(Y))
\end{align*}
that is natural in both $X$ and $Y$. In this case, the \hldef{unit of the adjunction} is the natural transformation $\eta : \mathrm{Id}_{\mathcal{C}} \Rightarrow G F$ such that, 
\begin{enumerate}
    \item for every $X \in \calC$, the morphism $\eta_X: X \to GF(X)$ (each called a \hldef{unit morphism}) in $\calC$ is obtained as the image of $\id_{F(X)}$ via the adjoint isomorphism
    $$\Hom_\calD(F(X), F(X)) \cong \Hom_\calC(X, GF(X)). $$

    \item for every $Y \in \calD$, the morphism $\epsilon_Y: FG(Y) \to Y$ (each called a \hldef{counit morphism}) in $\calD$ is obtained as the image of $\id_{G(Y)}$ via the adjoint isomorphism 
    $$\Hom_\calC(G(Y), G(Y)) \cong \Hom_\calD(FG(Y), Y).$$

\end{enumerate}


% Let $F : \mathcal{C} \to \mathcal{D}$ and $G : \mathcal{D} \to \mathcal{C}$ be functors. 
% $F$ is a \hldef{left adjoint to $G$} and $G$ is a \hldef{right adjoint to $F$} (written \hl{$F \dashv G$}) if for every object $A$ in $\mathcal{C}$ and $B$ in $\mathcal{D}$ there is a \CrefAndHyperrefIfExist{definition:natural_transformation_between_functors_between_categories}{natural isomorphism}
% \begin{align*}
% \operatorname{Hom}_{\mathcal{D}}(F(A), B) \cong \operatorname{Hom}_{\mathcal{C}}(A, G(B))
% \end{align*}
% that is natural in both $A$ and $B$.
\end{definition}
\begin{definition} \label{definition:site_of_opens_on_a_topological_space}
    Let $(X, \tau_X)$ be a topological space. The \hldef{small site associated to $X$} or \hldef{the site of open covers of $X$} or \hldef{the canonical site on $\operatorname{Open} X$} is the \CrefAndHyperrefIfExist{definition:category_of_opens_of_a_topological_space}{category $\operatorname{Open}(X)$ of open subsets} of $X$ with inclusion morphisms, equipped with the canonical \CrefAndHyperrefIfExist{definition:grothendieck_topology_on_a_category_site_covering_sieve_topologically_generating_family}{Grothendieck topology} \CrefAndHyperrefIfExist{definition:grothendieck_topology_generated_by_a_pretopology}{generated by} the \CrefAndHyperrefIfExist{definition:basis_and_grothendieck_pretopology_for_a_grothendieck_topology_on_a_category}{Grothendieck pretopology} whose covering families $\{U_i \to U\}_{i \in I}$, for $U \in \operatorname{Open}(X)$ are families of morphisms in $\operatorname{Open}(X)$ such that $\bigcup_{i \in I} U_i = U$. In other words, $\{U_i \to U\}_{i \in I}$ is a covering for the pretopology if it is an \CrefAndHyperrefIfExist{definition:open_covering_of_a_topological_space}{open coverings}.
\end{definition}

\begin{definition} \label{definition:initial_final_zero_objects_of_a_category}
Let $\mathcal{C}$ be a \CrefAndHyperrefIfExist{definition:category}{(large) category}.

\begin{enumerate}
    \item An object $I \in \mathcal{C}$ is called an \hldef{initial object} if for every object $X \in \mathcal{C}$ there exists a unique morphism
    $$I \to X.$$
    Equivalently, an initial object is a \CrefAndHyperrefIfExist{definition:limit_and_colimit_of_a_diagram_in_a_category}{limit} of the empty \CrefAndHyperrefIfExist{definition:diagram_in_a_category_indexed_by_a_small_category}{diagram}, if such a limit exists.

    \item An object $F \in \mathcal{C}$ is called a \hldef{final object} (or \hldef{terminal object}) if for every object $X \in \mathcal{C}$ there exists a unique morphism
    $$X \to F.$$
    Equivalently, a final object is a \CrefAndHyperrefIfExist{definition:limit_and_colimit_of_a_diagram_in_a_category}{colimit} of the empty \CrefAndHyperrefIfExist{definition:diagram_in_a_category_indexed_by_a_small_category}{diagram}, if such a colimit exists.

    \item An object $Z \in \mathcal{C}$ is called a \hldef{zero object} if $Z$ is both initial and final in $\mathcal{C}$. In particular, for every object $X \in \mathcal{C}$ there exist unique morphisms
    $$Z \to X \quad \text{and} \quad X \to Z.$$
\end{enumerate}
In particular, if initial/final/zero objects exist in a cateogry, then they are unique up to unique isomorphism.
\end{definition}


\begin{definition}[Filtered category] \label{definition:filtered_cofiltered_category}
    \begin{enumerate}
        \item 
        A \hldef{filtered category} is a (nonempty, large) category $\mathcal{I}$ satisfying the following conditions:

        \begin{itemize}
            \item For every finite collection of objects $i_1, i_2, \ldots, i_n$ in $\mathcal{I}$, there exists an object $j$ and morphisms
            \[
            \phi_k: i_k \to j, \quad \text{for each } k=1, \ldots, n.
            \]

            \item For every pair of morphisms $f,g: i \to j$ in $\mathcal{I}$, there exists an object $k$ and a morphism 
            \[
            h: j \to k
            \] 
            that satisfies 
            \[
            h \circ f = h \circ g.
            \]
        \end{itemize}

        \begin{figure}[h]
            \centering
            \begin{minipage}{0.45\textwidth}
                \centering
                % Diagram 1: "Joint Limit" (Upper Bound)
                \begin{tikzcd}[row sep=large, column sep=large]
                    i_1 \arrow[dr, "\phi_1", dashed] & \\
                    & j \\
                    i_2 \arrow[ur, "\phi_2"', dashed] & 
                \end{tikzcd}
                \caption*{Condition 1: Upper Bound}
            \end{minipage}
            \hfill
            \begin{minipage}{0.45\textwidth}
                \centering
                \begin{tikzcd}[row sep=large, column sep=large]
                    i \arrow[r, "f", shift left] \arrow[r, "g"', shift right] & 
                    j \arrow[r, "h", dashed] & 
                    k
                \end{tikzcd}
                \caption*{Condition 2: Coequalizing map}
            \end{minipage}
        \end{figure}
        In other words, $\mathcal{I}$ is nonempty, any finite diagram of objects admits a \CrefAndHyperrefIfExist{definition:limit_and_colimit_of_a_diagram_in_a_category}{cocone}, and any pair of parallel morphisms become equal after post-composition with an appropriate morphism.

    \item Dually, a \hldef{Cofiltered category} is a category whose \hyperrefIfExists{definition:opposite_category_of_a_category}{opposite category}\CrefIfExists{definition:opposite_category_of_a_category} is filtered. More explicitly, A cofiltered category is a (nonempty, large) category $\mathcal{I}$ satisfying the following conditions:

    \begin{itemize}
        \item For every finite collection of objects $i_1, i_2, \ldots, i_n$ in $\mathcal{I}$, there exists an object $j$ and morphisms
        \[
        \phi_k: j \to i_k, \quad \text{for each } k=1, \ldots, n.
        \]

        \item For every pair of morphisms $f,g: j \to i$ in $\mathcal{I}$, there exists an object $k$ and a morphism 
        \[
        h: k \to j
        \] 
        that satisfies
        \[
        f \circ h = g \circ h.
        \]
    \end{itemize}

    \begin{figure}[h]
        \centering
        \begin{minipage}{0.45\textwidth}
            \centering
            % Diagram 1: "Joint Limit" (Lower Bound)
            \begin{tikzcd}[row sep=large, column sep=large]
                i_1 & \\
                & j \arrow[ul, "\phi_1", dashed] \arrow[dl, "\phi_2"', dashed] \\
                i_2 & 
            \end{tikzcd}
            \caption*{Condition 1: Lower Bound}
        \end{minipage}
        \hfill
        \begin{minipage}{0.45\textwidth}
            \centering
            \begin{tikzcd}[row sep=large, column sep=large]
                k \arrow[r, "h", dashed] &
                j \arrow[r, "f", shift left] \arrow[r, "g"', shift right] & 
                i
            \end{tikzcd}
            \caption*{Condition 2: Equalizing map}
        \end{minipage}
    \end{figure}

    In other words, $\mathcal{I}$ is nonempty, any finite diagram of objects admits a cone, and any pair of parallel morphisms become equal after pre-composition with an appropriate morphism.

    \end{enumerate}

    
\end{definition}

\begin{definition}[Cones, limits and colimits in a category] \label{definition:limit_and_colimit_of_a_diagram_in_a_category}
Let $\mathcal{C}$ be a \CrefAndHyperrefIfExist{definition:category}{(large) category}, let $I$ be a (large) category, and let $D: I \to \mathcal{C}$ be a \CrefAndHyperrefIfExist{definition:diagram_in_a_category_indexed_by_a_small_category}{diagram}\CrefIfExists{definition:diagram_in_a_category_indexed_by_a_small_category}.

\begin{enumerate}
    \item A \hldef{cone to the diagram $D$} is an object $L \in \mathcal{C}$ together with a family of morphisms
    \[
    \{\pi_i: L \to D(i)\}_{i \in I}
    \]
    such that for every morphism $f: i \to j$ in $I$, the diagram
    \begin{center}
    \begin{tikzcd}[row sep=large, column sep=large]
        & L \arrow[dl, "\pi_i"'] \arrow[dr, "\pi_j"] & \\
        D(i) \arrow[rr, "D(f)"] & & D(j)
    \end{tikzcd}
    \end{center}
    commutes, i.e.  $D(f) \circ \pi_i = \pi_j$.
    


    \item A cone $(L, \{\pi_i\})$ is called a \hldef{limit of $D$} if it satisfies the following ``universal property'':
    for any cone $(C, \{ f_i \})$ over $D$, there exists a \textit{unique} morphism $u: C \to L$ such that
    \[
    \pi_i \circ u = f_i \quad \text{for all } i \in I.
    \]
    Visually, the following diagrams commute every morphism $f: i \to j$ in $I$:
    \begin{center}
    \begin{tikzcd}[row sep=large, column sep=large]
        & C \arrow[d, "\exists ! u", dashed] \arrow[ddl, "f_i"', bend right=20] \arrow[ddr, "f_j", bend left=20] & \\
        & L \arrow[dl, "\pi_i"] \arrow[dr, "\pi_j"'] & \\
        D(i) \arrow[rr, "D(f)"] & & D(j).
    \end{tikzcd}
    \end{center}
    If such a cone exists, then the object $L$ is necessarily unique up to unique isomorphism by the universal property. In this case, $L$ is denoted by \hl{$\lim_{i \in I} D$} or \hl{$\lim D$}.



    
    \item A \hldef{cocone from the diagram $D$} is an object $C \in \mathcal{C}$ together with a family of morphisms
    \[
    \{\iota_i: D(i) \to C\}_{i \in I}
    \]
    such that for every morphism $f: i \to j$ in $I$, the diagram
    \begin{center}
    \begin{tikzcd}[row sep=large, column sep=large]
        D(i) \arrow[rr, "D(f)"] \arrow[dr, "\iota_i"'] & & D(j) \arrow[dl, "\iota_j"] \\
        & C & 
    \end{tikzcd}
    \end{center}
    commutes, i.e. $\iota_j \circ D(f) = \iota_i$.

    \item A cocone $(L, \{\iota_i\})$ is called a \hldef{colimit of $D$} if it satisfies the following ``universal property'':
    for any cocone $(C, \{ g_i \})$ under $D$, there exists a \textit{unique} morphism $u: L \to C$ such that
    \[
    u \circ \iota_i = g_i \quad \text{for all } i \in I.
    \]
    Visually, the following diagrams commute every morphism $f: i \to j$ in $I$:
    \begin{center}
    \begin{tikzcd}[row sep=large, column sep=large]
        D(i) \arrow[rr, "D(f)"] \arrow[dr, "\iota_i"] \arrow[ddr, "g_i"', bend right=20] & & D(j) \arrow[dl, "\iota_j"'] \arrow[ddl, "g_j", bend left=20] \\
        & L \arrow[d, "\exists ! u", dashed] & \\
        & C &. 
    \end{tikzcd}
    \end{center}
    If such a cocone exists, then the object $L$ is necessarily unique up to unique isomorphism by the universal property. In this case, $L$ is denoted by \hl{$\colim_{i \in I} D$} or \hl{$\colim D$}.

\end{enumerate}

A limit/colimit is called \hldef{finite} (resp. \hldef{small}) if the diagram category $I$ is finite (resp. small).

Some authors use the terms \hldef{projective limit} or \hldef{inverse limit} to refer to what is defined here as a limit, Similarly, the terms \hldef{inductive limit} or \hldef{direct limit} are sometimes used to mean a colimit. However, these phrases can have more specific meanings to other authors: a \emph{projective} or \emph{inverse limit} may refer to a limit over a diagram indexed by a \hyperrefIfExists{definition:partially_ordered_set}{codirected poset}\CrefIfExists{definition:partially_ordered_set}. Likewise, an \emph{inductive} or \emph{direct limit} may refer to a colimit over a \hyperrefIfExists{definition:partially_ordered_set}{directed poset}\CrefIfExists{definition:partially_ordered_set}\TextIfExists{definition:projective_and_inductive_limits_in_categories}{ (see \Cref{definition:projective_and_inductive_limits_in_categories})}.

Thus, while the terms are sometimes used interchangeably with ``limit'' and ``colimit,'' they may also emphasize particular indexing shapes and directions, distinguishing them from general limits and colimits taken over arbitrary small categories.
\end{definition}

\begin{definition}[Special cases of limits] \label{definition:projective_and_inductive_limits_in_categories}
Let $\mathcal{C}$ be a (large) category. Let $I$ be a (large) category. Let $I \to \mathcal{C}$ be a diagram/system. 
\begin{itemize}
    \item Suppose that the system is a \hyperrefIfExists{definition:system_in_a_category_indexed_by_a_directed_poset}{cofiltered system}\CrefIfExists{definition:system_in_a_category_indexed_by_a_directed_poset}, i.e. $I$ is a cofiltered category. A \hyperrefIfExists{definition:limit_and_colimit_of_a_diagram_in_a_category}{limit}\CrefIfExists{definition:limit_and_colimit_of_a_diagram_in_a_category} of this diagram is often denoted by 
    $$\hlin{ \varprojlim_{i\in I} D(i) }$$
    and may be called a \hldef{cofiltered (inverse/projective) limit}. In case that the system is more specifically an \hyperrefIfExists{definition:system_in_a_category_indexed_by_a_directed_poset}{inverse/projective system}\CrefIfExists{definition:system_in_a_category_indexed_by_a_directed_poset}, i.e. $I$ is a cofiltered poset, the preferred term for such a limit is \emph{inverse/projective limit}.

    \item Suppose that the system is a filtered system, i.e. $I$ is a filtered category. A colimit of this diagram is often denoted by 
    $$\hlin{ \varinjlim_{i\in I} D(i) }$$
    and may be called a \hldef{filtered colimit} or a \hldef{direct/inductive/injective limit}. In case that the system is more specifically a direct/inductive system, i.e. $I$ is a filtered poset, the preferred term for such a limit is \emph{direct/inductive limit}.

\end{itemize}
\end{definition}


\begin{definition}[Ind-category] \label{definition:ind_pro_category_of_a_locally_small_category}
Let $\mathcal{C}$ be a \CrefAndHyperrefIfExist{definition:locally_small_category}{locally small category}.

\begin{enumerate}
    \item  The \hldef{Ind-category of $\mathcal{C}$}, denoted \hl{$\mathrm{Ind}(\mathcal{C})$}, is defined as follows:
    \begin{itemize}
        \item Objects of $\mathrm{Ind}(\mathcal{C})$ are formal \CrefAndHyperrefIfExist{definition:projective_and_inductive_limits_in_categories}{filtered colimits} of objects in $\mathcal{C}$. More precisely, an object is given by a \CrefAndHyperrefIfExist{definition:filtered_cofiltered_category}{filtered} small category $I$ and a functor 
        $$ X : I \to \mathcal{C}.  $$
        \item Morphisms between objects $X : I \to \mathcal{C}$ and $Y : J \to \mathcal{C}$ are defined by
        $$ \mathrm{Hom}_{\mathrm{Ind}(\mathcal{C})}(X,Y) \;:=\; \varprojlim_{i \in I} \varinjlim_{j \in J} \mathrm{Hom}_{\mathcal{C}}(X_i, Y_j), $$
        \CrefIfExists{definition:projective_and_inductive_limits_in_categories}
        where $X_i$ and $Y_j$ denote the images of $i \in I$ and $j \in J$ under $X$ and $Y$, respectively.
    \end{itemize}
    The composition of morphisms is induced naturally from composition in $\mathcal{C}$.  
    Hence, $\mathrm{Ind}(\mathcal{C})$ is the completion of $\mathcal{C}$ under filtered colimits. Objects of $\mathrm{Ind}(\mathcal{C})$ are called \hldef{Ind-objects of $\calC$}.
    
    \item 
    The \hldef{Pro-category of $\mathcal{C}$}, denoted \hl{$\mathrm{Pro}(\mathcal{C})$}, is defined as follows:
    \begin{itemize}
        \item Objects of \(\mathrm{Pro}(\mathcal{C})\) are formal \CrefAndHyperrefIfExist{definition:projective_and_inductive_limits_in_categories}{cofiltered limits} of objects in \(\mathcal{C}\). More precisely, an object is given by a \CrefAndHyperrefIfExist{definition:cofiltered_cofiltered_category}{cofiltered} small category \(I\) and a functor
        \[
        X : I \to \mathcal{C}.
        \]
        \item Morphisms between objects \(X : I \to \mathcal{C}\) and \(Y : J \to \mathcal{C}\) are defined by
        \[
        \mathrm{Hom}_{\mathrm{Pro}(\mathcal{C})}(X,Y) := \varinjlim_{j \in J} \varprojlim_{i \in I} \mathrm{Hom}_{\mathcal{C}}(X_i, Y_j),
        \]
        where \(X_i\) and \(Y_j\) denote the images of \(i \in I\) and \(j \in J\) under \(X\) and \(Y\), respectively.
    \end{itemize}
    The composition of morphisms is induced naturally from composition in \(\mathcal{C}\).

    Hence, \(\mathrm{Pro}(\mathcal{C})\) is the completion of \(\mathcal{C}\) under cofiltered limits. Objects of $\mathrm{Pro}(\mathcal{C})$ are called \hldef{Pro-objects of $\calC$}.


\end{enumerate}

    Since $\Sets$ has all limits and colimits \TODO{} and hence has all projective and inductive limits and since $\calC$ is locally small, $\mathrm{Ind}(\calC)$ and $\mathrm{Pro}(\calC)$ are locally small.

\end{definition}


\begin{definition}[Product in a category] \label{definition:product_and_coproduct_of_objects_in_a_category}
Let $\mathcal{C}$ be a category and let $\{X_i\}_{i \in I}$ be a family of objects in $\mathcal{C}$ indexed by a class $I$. 
\begin{enumerate}
    \item A \hldef{product of the family $\{X_i\}$} is an object $P$ of $\mathcal{C}$ together with a ``universal'' family of morphisms
    $$\pi_i : P \to X_i, \quad \text{for each } i \in I. $$
    More precisely, for any object $Y$ and any family of morphisms $\{f_i : Y \to X_i\}_{i \in I}$, there exists a unique morphism
    $$f : Y \to P$$
    making the following diagram commute for all $i \in I$, i.e. $\pi_i \circ f = f_i$:
    \begin{center}
    \begin{tikzcd}[row sep=large, column sep=large]
        Y \arrow[d, "\exists ! f", dashed] \arrow[dr, "f_i"] & \\
        \prod X_i \arrow[r, "\pi_i"'] & X_i
    \end{tikzcd}
    \end{center}
    Such a product is often denoted by \hl{$\prod_{i \in I} X_i$}. If $\prod_{i \in I} X_i$ exists in $\calC$, then it is unique up to unique isomorphism by the universal property described above.
    
    Equivalently, the product $\prod_{i \in I} X_i$ is the \CrefAndHyperrefIfExist{definition:limit_and_colimit_of_a_diagram_in_a_category}{limit} of the \CrefAndHyperrefIfExist{definition:diagram_in_a_category_indexed_by_a_small_category}{diagram} $I \to \calC, i \mapsto X_i$, where $I$ is made into a category whose objects are the members of $I$ and whose morphisms are just the identity morphisms.


    \item A \hldef{coproduct} (or synonymously \hldef{direct sum}) of the family $\{X_i\}$ is an object $C$ of $\mathcal{C}$ together with a ``universal'' family of morphisms
    $$\iota_i : X_i \to C, \quad \text{for each } i \in I.$$
    More precisely, for any object $Y$ and any family of morphisms $\{g_i : X_i \to Y\}_{i \in I}$, there exists a unique morphism
    $$g : C \to Y$$
    making the following diagram commute for all $i \in I$, i.e. $g \circ \iota_i = g_i$:
    \begin{center}
    \begin{tikzcd}[row sep=large, column sep=large]
        X_i \arrow[r, "\iota_i"] \arrow[dr, "g_i"'] & \coprod X_i \arrow[d, "\exists ! g", dashed] \\
        & Y
    \end{tikzcd}
    \end{center}
    Such a coproduct is often denoted by \hl{$\coprod_{i \in I} X_i$} or \hl{$\oplus_{i \in I} X_i$}. If $\coprod_{i \in I} X_i$ exists in $\calC$, then it is unique up to unique isomorphism by the universal property described above.

    Equivalently, the coproduct $\coprod_{i \in I} X_i$ is the \CrefAndHyperrefIfExist{definition:limit_and_colimit_of_a_diagram_in_a_category}{colimit} of the \CrefAndHyperrefIfExist{definition:diagram_in_a_category_indexed_by_a_small_category}{diagram} $I \to \calC, i \mapsto X_i$, where $I$ is made into a category whose objects are the members of $I$ and whose morphisms are just the identity morphisms.
\end{enumerate}
\end{definition}
\begin{definition}[Equalizer in a category] \label{definition:equalizer_and_coequalizer_of_morphisms_in_a_category}
Let $\mathcal{C}$ be a \CrefAndHyperrefIfExist{definition:category}{(large) category} and let $f, g : X \to Y$ be morphisms in $\mathcal{C}$. 
\begin{enumerate}
    \item An \hldef{equalizer of $f$ and $g$} is an object $E$ together with a morphism
    $$e : E \to X$$
    such that
    $$f \circ e = g \circ e$$
    and for any object $Z$ with morphism $z : Z \to X$ satisfying
    $$f \circ z = g \circ z,$$
    there exists a unique morphism $u : Z \to E$ making the diagram commute:
    $$e \circ u = z.$$

    \begin{center}
        \begin{tikzcd}[column sep=large, row sep=large]
        Z \arrow[d, dashed, "\exists! u"] \arrow[dr, "z"] & & \\
        E \arrow[r, "e"] & X \arrow[r, shift left, "f"] \arrow[r, shift right, "g"'] & Y
        \end{tikzcd}
    \end{center}
    If such an equalizer of $f$ and $g$ exists, then we say that the following \hldef{equalizer diagram is exact}:
    \begin{center}
    \begin{tikzcd}[column sep=large, row sep=large]
    E \arrow[r, "e"] & X \arrow[r, shift left, "f"] \arrow[r, shift right, "g"'] & Y
    \end{tikzcd}
    \end{center}

    \item A \hldef{coequalizer of $f$ and $g$} is an object $Q$ together with a morphism
    $$q : Y \to Q$$
    such that
    $$q \circ f = q \circ g$$
    and for any object $Z$ with morphism $w : Y \to Z$ satisfying
    $$w \circ f = w \circ g,$$
    there exists a unique morphism $v : Q \to Z$ making the diagram commute:
    $$v \circ q = w.$$

    \begin{center}
        \begin{tikzcd}[column sep=large]
        X \arrow[r, shift left, "f"] \arrow[r, shift right, "g"'] & Y \arrow[r, "q"] \arrow[dr, "w"'] & Q \arrow[d, dashed, "\exists! v"] \\
        & & Z
        \end{tikzcd}
    \end{center}
    If such a coequalizer of $f$ and $g$ exists, then we say that the following \hldef{coequalizer diagram is exact}:
    \begin{center}
    \begin{tikzcd}[column sep=large, row sep=large]
        X \arrow[r, shift left, "f"] \arrow[r, shift right, "g"'] & Y \arrow[r, "q"] & Q 
    \end{tikzcd}
    \end{center}



\end{enumerate}
\end{definition}

\begin{definition}[Systems in a category] \label{definition:system_in_a_category_indexed_by_a_directed_poset}
Let $\mathcal{C}$ be a (large) category. Let $I$ be a (large) category. 
\begin{enumerate}
   \item A \hyperrefIfExists{definition:diagram_in_a_category_indexed_by_a_small_category}{diagram/system}\CrefIfExists{definition:diagram_in_a_category_indexed_by_a_small_category} $I \to C$ is called \hldef{filtered} (resp. \hldef{cofiltered}) if $I$ is a \CrefAndHyperrefIfExist{definition:filtered_cofiltered_category}{filtered} (resp. \CrefAndHyperrefIfExist{definition:filtered_cofiltered_category}{cofiltered}) category.

    \item A diagram/system $I \to C$ is called \hldef{directed} (resp. \hldef{codirected} if $I$ is small and thing, i.e. is regardable/comes from\CrefIfExists{lemma:posets_correspond_to_small_filtered_thin_categories} a \hyperrefIfExists{definition:partially_ordered_set}{directed (resp. codirected) partially ordered set}\CrefIfExists{definition:partially_ordered_set}. A \hldef{direct system} or \hldef{inductive system} is synonymous for a directed system and a \hldef{inverse system} or \hldef{projective system} is synonymous for a codirected system.
\end{enumerate}
One might also speak of a \hldef{filtered direct/inductive system} synonymously for a filtered system to emphasize that the indexing caetgory is a general filtered category, rather than a directed poset.
\end{definition}

\begin{definition} \label{definition:essentially_small_category}
A category $\mathcal{C}$ is called \hldef{essentially small} if it is \CrefAndHyperrefIfExist{definition:equivalence_of_categories}{equivalent} to a \CrefAndHyperrefIfExist{definition:locally_small_category}{small category}, i.e., there exists a small category $\mathcal{D}$ and an equivalence of categories
$$F : \mathcal{D} \to \mathcal{C}.$$
Note that an essentially small category is necessarily \CrefAndHyperrefIfExist{definition:locally_small_category}{locally small}.
\end{definition}

\begin{definition}[Monomorphism and Epimorphism in Categories] \label{definition:monomorphism_and_epimorphism_in_categories}
Let $\mathcal{C}$ be a \CrefAndHyperrefIfExist{definition:category}{category}. For objects $A, B \in \mathcal{C}$, let $f: A \to B$ be a morphism in $\mathcal{C}$.  
\begin{itemize}
    \item The morphism $f$ is called a \hldef{monomorphism} (or a \hldef{monic morphism}) if for every object $X$ and every pair of morphisms $g_1, g_2 : X \to A$, the equality $f \circ g_1 = f \circ g_2$ implies $g_1 = g_2$.  
    \item The morphism $f$ is called an \hldef{epimorphism} (or an \hldef{epic morphism}) if for every object $Y$ and every pair of morphisms $h_1, h_2: B \to Y$, the equality $h_1 \circ f = h_2 \circ f$ implies $h_1 = h_2$.  
\end{itemize}
\end{definition}


\begin{definition}[Partially ordered set] \label{definition:partially_ordered_set}
    \begin{enumerate}
        \item 
        A \hldef{partially ordered set} (or \hldef{poset}), or \hldef{ordered set} is a pair $(P, \leq)$ where $P$ is a set and 
        \[
        \leq : P \times P \to \{\text{true}, \text{false}\}
        \]
        is a binary relation on $P$ satisfying the following axioms for all $a,b,c \in P$:
        \begin{itemize}
            \item \hldef{Reflexivity:} $a \leq a$,
            \item \hldef{Antisymmetry:} if $a \leq b$ and $b \leq a$, then $a = b$,
            \item \hldef{Transitivity:} if $a \leq b$ and $b \leq c$, then $a \leq c$.
        \end{itemize}
        The relation $\leq$ is called an \hldef{order} or a \hldef{partial order}

        \item A partially ordered set $(P, \leq)$ is called a \hldef{directed partially ordered set} if for every pair $a,b \in P$, there exists $c \in P$ such that
        \[
        a \leq c \quad \text{and} \quad b \leq c.
        \]

        \item A partially ordered set $(P, \leq)$ is called a \hldef{codirected partially ordered set} (or \hldef{downward directed poset}) if for every pair $a,b \in P$, there exists $d \in P$ such that
        \[
        d \leq a \quad \text{and} \quad d \leq b.
        \]
        \end{enumerate}
\end{definition}

\begin{lemma} \label{lemma:posets_correspond_to_small_filtered_thin_categories}
    Let $(P, \leq)$ be a nonempty \hyperrefIfExists{definition:partially_ordered_set}{poset}\CrefIfExists{definition:partially_ordered_set}. 
    \begin{enumerate}
        \item Regarding $P$ as a category whose objects are the elements of $P$ and such that there is a unique arrow $a \to b$ if and only if $a \leq b$, the category is filtered. 
        \item Every nonempty \hyperrefIfExists{definition:locally_small_category}{small}\CrefIfExists{definition:locally_small_category} \hyperrefIfExists{definition:thin_category}{thin}\CrefIfExists{definition:thin_category} \hyperrefIfExists{definition:filtered_cofiltered_category}{filtered category}\CrefIfExists{definition:filtered_cofiltered_category} corresponds to a poset in this way.
        \item Moreover, the poset $P$ is \hyperrefIfExists{definition:partially_ordered_set}{directed}\CrefIfExists{definition:partially_ordered_set} if and only if the category is filtered. The poset $P$ is \hyperrefIfExists{definition:partially_ordered_set}{codirected}\CrefIfExists{definition:partially_ordered_set} if and only if the category is cofiltered.
    \end{enumerate}
\end{lemma}

\begin{definition}[Generator of a category] \label{definition:generator_of_a_category}
Let \(\mathcal{C}\) be a \CrefAndHyperrefIfExist{definition:category}{category}. 
\begin{enumerate}
    \item  An object \(G \in \mathcal{C}\) is called a \hldef{generator} (or \hldef{separator}) if for every pair of distinct morphisms \(f, g : X \to Y\) in \(\mathcal{C}\), there exists a morphism \(h : G \to X\) such that
    \[
    f \circ h \neq g \circ h.
    \]
    In case that $\calC$ is \CrefAndHyperrefIfExist{definition:locally_small_category}{locally small}, this is equivalent to the condition that the \CrefAndHyperrefIfExist{definition:representable_functor_on_a_category_enriched_in_a_monoidal_category}{representable functor}
    \[
    \mathrm{Hom}_{\mathcal{C}}(G, -) : \mathcal{C} \to \mathbf{Set}
    \]
    is \CrefAndHyperrefIfExist{definition:full_and_faithful_functor_between_locally_small_categories}{faithful}, 
    %
    % In other words, for every pair of distinct morphisms \(f, g : X \to Y\) in \(\mathcal{C}\), there exists a morphism \(h : G \to X\) such that
    % \[
    % f \circ h \neq g \circ h.
    % \]
    %
    which in turn is equivalent to the condition that for every object \(X \in \mathcal{C}\), there exists an epimorphism
    \[
    \bigoplus_{i \in I} G \twoheadrightarrow X
    \]
    for some indexing set \(I\), where \(\bigoplus\) denotes the \CrefAndHyperrefIfExist{definition:product_and_coproduct_of_objects_in_a_category}{coproduct} in \(\mathcal{C}\).

    \item A family \(\{G_i\}_{i \in I}\) is called a \hldef{generating family} if for every pair of distinct morphisms \(f, g : X \to Y\) in \(\mathcal{C}\), there exists some index \(i \in I\) and a morphism \(h : G_i \to X\) such that
    \[
    f \circ h \neq g \circ h.
    \]
    In case $\calC$ is locally small, this is equivalent to the condition that the collection of representable functors
    \[
    \{\mathrm{Hom}_{\mathcal{C}}(G_i, -) : \mathcal{C} \to \mathbf{Set}\}_{i \in I}
    \]
    is jointly faithful, which in turn is equivalent to the condition that for every object \(X \in \mathcal{C}\), there exists a family of objects \(\{G_i\}_{i \in J}\) from the generating set indexed by some set \(J\), and an epimorphism
    \[
    \bigoplus_{i \in J} G_i \twoheadrightarrow X.
    \]

\end{enumerate}
\end{definition}
% \begin{definition}[Topos] \label{definition:topos}
%     There are a multitude of notions of topos. Here are some that we consider; more notions may be added later.
%     \begin{enumerate}
%         \item A \hldef{(sheaf/Grothendieck) topos} is a \CrefAndHyperrefIfExist{definition:category}{category} \CrefAndHyperrefIfExist{definition:equivalence_of_categories}{equivalent} to the category of \CrefAndHyperrefIfExist{definition:sheaf_on_a_site}{sheaves} of sets on some \CrefAndHyperrefIfExist{definition:grothendieck_topology_on_a_category_site_covering_sieve_topologically_generating_family}{site}. That is, there exists a site $(C, J)$ such that the category is equivalent to $\operatorname{Sh}(C, J)$, the category of sheaves of sets on $(C, J)$.
%         \item Let $U$ be a universe. A \hldef{$U$-(sheaf )topos} is a category equivalent to the category of \hyperrefIfExists{definition:sheaf_on_a_site}{$U$-sheaves}\CrefIfExists{definition:sheaf_on_a_site} (valued in $U$-sets) \cite[Expos\'e IV D\'efinition 1.1]{SGA4_I}

%         \item An \hldef{elementary topos} is a cateogry which has all finite \CrefAndHyperrefIfExist{definition:limit_and_colimit_of_a_diagram_in_a_category}{limits}, is cartesian closed, and has a subobject classifier \TODO{cartesian closed, subobject classifier}
%     \end{enumerate}
% \end{definition}

\begin{definition}[Topos] \label{definition:topos}
    There are multiple notions of a topos depending on the context (geometric vs. logical).
    \begin{enumerate}
        \item A \hldef{Grothendieck topos} (or \hldef{sheaf topos}) is a \CrefAndHyperrefIfExist{definition:category}{category} \CrefAndHyperrefIfExist{definition:equivalence_of_categories}{equivalent} to the category of \CrefAndHyperrefIfExist{definition:sheaf_on_a_site}{sheaves} of sets on a \hldef{small} \CrefAndHyperrefIfExist{definition:grothendieck_topology_on_a_category_site_covering_sieve_topologically_generating_family}{site}. That is, there exists a small site $(\mathcal{C}, J)$ such that the category is equivalent to $\operatorname{Sh}(\mathcal{C}, J)$.
        
        \item Let $\mathscr{U}$ be a \hyperrefIfExists{definition:grothendieck_universe}{universe}\CrefIfExists{definition:grothendieck_universe}. A \hldef{$\mathscr{U}$-topos} is a category equivalent to the category of sheaves of sets on a $\mathscr{U}$-small site $(\mathcal{C}, J)$, where the sheaves take values in the category of $\mathscr{U}$-sets ($\mathbf{Set}_{\mathscr{U}}$). \cite[Expos\'e IV D\'efinition 1.1]{SGA4_I}

        \item An \hldef{elementary topos} is a category which has all finite \CrefAndHyperrefIfExist{definition:limit_and_colimit_of_a_diagram_in_a_category}{limits}, is \CrefAndHyperrefIfExist{definition:cartesian_closed_category}{cartesian closed}, and has a \CrefAndHyperrefIfExist{definition:subobject_classifier_in_a_category_with_a_final_object}{subobject classifier}.
    \end{enumerate}
    \textit{Remark:} Every Grothendieck topos is an elementary topos, but the converse is not true (e.g., the category of finite sets is an elementary topos but not a Grothendieck topos).
\end{definition}


% {\cite[Expos\'e IV D\'efinition 1.1]{SGA4_I}}
% Let $\scrU$ be a fixed universe. A \hldef{$\scrU$-topos}, or simply \hldef{topos} if there is no confusion, $E$ is a category that is equivalent to the category $\Shv(T)$ of sheaves of sets on a fixed site $T$ in $\scrU$.
\begin{theorem}[Functor Category of a Grothendieck Topos is a Topos]
    \label{theorem:functor_category_of_a_grothendieck_topos_indexed_by_a_small_category_is_a_grothendieck_topos}
    Let $I$ be a \CrefAndHyperrefIfExist{definition:locally_small_category}{small category}, and let $\mathcal{S}$ be a \CrefAndHyperrefIfExist{definition:topos}{Grothendieck topos}. Then the \CrefAndHyperrefIfExist{definition:diagram_in_a_category_indexed_by_a_small_category}{functor category $\mathrm{Fun}(I, \mathcal{S})$} is a Grothendieck topos.

    % More precisely, $\mathrm{Fun}(I, \mathcal{S})$ has all small limits and colimits, exponentials, a subobject classifier, and a generator, and it is equivalent to the category of sheaves of sets on the product site associated to the site defining $\mathcal{S}$ and $I$.
\end{theorem}

\begin{definition}[Diagram in a category and category of diagrams] \label{definition:diagram_in_a_category_indexed_by_a_small_category}
Let $\mathcal{C}$ be a \hyperrefIfExists{definition:category}{(large) category}\CrefIfExists{definition:category}, and let $I$ be a \CrefAndHyperrefIfExist{definition:category}{(large) category}. 
    \begin{enumerate}
        \item 
        A \hldef{diagram of shape $I$ in $\mathcal{C}$} is a \hyperrefIfExists{definition:functor_between_categories}{functor}\CrefIfExists{definition:functor_between_categories} $D: I \to \mathcal{C}$.
        We often denote such a diagram by the family \hl{$\{ D(i) \}_{i \in \mathrm{Ob}(I)}$} with transition maps given by the functorial image of morphisms in $I$. 
        
        A diagram is also synonymously called a \hldef{system}. Moreover, the category $I$ is called the \hldef{index category} or the \hldef{indexing category of the diagram $D$}.

        \item Given two diagrams $D,E: I \to \mathcal{C}$, a \hldef{morphism of diagrams} is a simply a \hyperrefIfExists{definition:natural_transformation_between_functors_between_categories}{natural transformation}\CrefIfExists{definition:natural_transformation_between_functors_between_categories} $D \Rightarrow E$ of the functors $D$ and $E$. 

        \item The \hldef{category of $I$-shaped diagrams in $\mathcal{C}$} or simply \hldef{diagram category (of $I$-shaped diagrams in $\calC$)}, often denoted \hl{$\mathcal{C}^I$}, \hl{$[I, \calC]$}, or \hl{$\operatorname{Fun}(I, \calC)$},
        is the (large) category whose objects are functors $I \to \mathcal{C}$ (that is, diagrams of shape $I$ in $\mathcal{C}$) and whose morphisms are \CrefAndHyperrefIfExist{definition:natural_transformation_between_functors_between_categories}{natural transformations} between such functors. The category $\calC^I$ is also called the \hldef{functor category of functors $I \to \calC$}. \TextIfExists{definition:presheaf_on_a_category}{Equivalently, the functor category $\calC^I$ is the category \CrefAndHyperrefIfExist{definition:presheaf_on_a_category}{$\PreShv(I^{\op}, \calC)$ of presheaves} on $I^{\op}$ with values in $\calC$ and hence notations for presheaf categories are applicable as notations for functor categories.}

        If $\calC$ is \hyperrefIfExists{definition:locally_small_category}{locally small}\CrefIfExists{definition:locally_small_category} and $I$ is small, then $\calC^I$ is locally small by Lemma \ref{lemma:category_of_presheaves_on_a_small_category_of_locally_small_value_is_locally_small}.
    \end{enumerate}
\end{definition}



\subsubsection{Abelian categories}
\begin{definition}[Abelian category] \label{definition:abelian_category}
Let $\mathcal{A}$ be a category. The category $\mathcal{A}$ is an \hldef{abelian category} if:
\begin{itemize}
    \item $\mathcal{A}$ is an \CrefAndHyperrefIfExist{definition:additive_category_preadditive_category}{additive category}.

    \item Every morphism $f: A \to B$ has a \CrefAndHyperrefIfExist{definition:kernel_and_cokernel_of_a_morphism_in_a_category}{kernel $\ker(f)$ and a cokernel $\operatorname{coker}(f)$}.

    \item For every morphism $f: A \to B$, the canonical morphism $\operatorname{coim}(f) \to \operatorname{im}(f)$ is an isomorphism, where
    $$
    \operatorname{coim}(f) = \operatorname{coker}(\ker(f) \to A),\quad \operatorname{im}(f) = \ker(B \to \operatorname{coker}(f)).
    $$
    \TODO{I think I need to re-check this defintion}
    \TODO{coimage}
\end{itemize}

\TextIfExists{definition:pre_abelian_category}{In particular, every abelian category is \Cref{definition:pre_abelian_category}{pre-abelian}}.

It is also worth considering Grothendieck's additional axioms for abelian categories\CrefIfExists{definition:grothendiecks_additional_axioms_for_abelian_categories}.

\end{definition}

\begin{proposition} \label{proposition:examples_of_abelian_categories}
The following are examples of \CrefAndHyperrefIfExist{definition:abelian_category}{abelian categories}:

\begin{enumerate}
    \item The category of $R$-$S$ bimodules where $R$,$S$ are \CrefAndHyperrefIfExist{definition:ring}{(not necessarily commutative) rings} (\Cref{theorem:the_category_of_R_S_bimodules_is_a_grothendieck_abelian_category_and_AB4_star}).

    \item The category $\mathbf{Ab}$ of abelian groups and group homomorphisms is abelian.

    \item The category $\text{Vect}_k$ of vector spaces over a field $k$ and $k$-linear maps is abelian.

    \item More generally, if $R$ is a \CrefAndHyperrefIfExist{definition:noetherian_ring}{noetherian ring}, then the category of \CrefAndHyperrefIfExist{definition:finitely_generated_modules_over_rings}{finitely generated} $R$-modules is abelian.

    \item For a \CrefAndHyperrefIfExist{definition:ringed_space}{ringed space} $(X, \mathcal{O}_X)$, the category of \CrefAndHyperrefIfExist{definition:module_over_a_sheaf_of_rings_on_a_site}{$\mathcal{O}_X$-modules} is abelian.
    \TODO{a quasi-coherent sheaf on a locally ringed space}
    \item If $X$ is a \CrefAndHyperrefIfExist{definition:scheme}{scheme} (or more generally a \CrefAndHyperrefIfExist{definition:locally_ringed_space_on_a_topological_space}{locally ringed space}), the category of \CrefAndHyperrefIfExist{definition:quasi_coherent_sheaf_on_a_general_scheme}{quasi-coherent sheaves on $X$} is abelian.
    \item For any \CrefAndHyperrefIfExist{definition:essentially_small_category}{essentially small category} $\mathcal{C}$ and any abelian category $\mathcal{A}$, the \CrefAndHyperrefIfExist{definition:diagram_in_a_category_indexed_by_a_small_category}{functor category $[\mathcal{C}, \mathcal{A}]$} and the category $\PreShv(\calC, \calA)$ of \CrefAndHyperrefIfExist{definition:presheaf_on_a_category}{presheaves} are abelian.
    \TODO{apparently, the essentially smallness condition is removable, provided that the sheafification functor exists. However, the essentially small assumption is needed to show that the category of sheaves of $O$-modules is a Grothendieck abelian caetgory. Verify all this. Moreover, when working with a big site of a scheme, one typically fixes a unvierse or work relative to a cardinal cutoff to treat it as essentially small}

    \item For any \CrefAndHyperrefIfExist{definition:grothendieck_topology_on_a_category_site_covering_sieve_topologically_generating_family}{site} $(\calC, J)$ on an \CrefAndHyperrefIfExist{definition:essentially_small_category}{essentially small category} $\mathcal{C}$ and any abelian category $\mathcal{A}$, the category $\Shv(\calC, \calA)$ of \CrefAndHyperrefIfExist{definition:sheaf_on_a_site}{sheaves} is abelian.

    \item For any \CrefAndHyperrefIfExist{definition:grothendieck_topology_on_a_category_site_covering_sieve_topologically_generating_family}{site} $(\calC, J)$ on an \CrefAndHyperrefIfExist{definition:essentially_small_category}{essentially small category} $\mathcal{C}$ and a \CrefAndHyperrefIfExist{definition:sheaf_on_a_site}{sheaf of rings} $\calO$ on $\calC$, the category $\mathbf{Mod}(\mathcal{O})$ of \CrefAndHyperrefIfExist{definition:module_over_a_sheaf_of_rings_on_a_site}{$\calO$-modules} is an abelian category.

\end{enumerate}
\end{proposition}
\begin{definition} \label{definition:exact_functor_between_abelian_categories}
    Let $F: \mathcal{A} \to \mathcal{B}$ be an \hyperrefIfExists{definition:additive_functor_between_additive_categories}{additive functor}\CrefIfExists{definition:additive_functor_between_additive_categories} between \hyperrefIfExists{definition:abelian_category}{abelian categories}\CrefIfExists{definition:abelian_category}.
    \begin{enumerate}

        \item $F$ is called \hldef{left exact} if it preserves all \CrefAndHyperrefIfExist{definition:limit_and_colimit_of_a_diagram_in_a_category}{finite limits}, or equivalently it preserves \CrefAndHyperrefIfExist{definition:kernel_and_cokernel_of_a_morphism_in_a_category}{kernels} and any finite limit diagrams. Equivalently, for every left exact sequence in $\mathcal{A}$
        \[
        0 \to A' \xrightarrow{f} A \xrightarrow{g} A''
        \]
        the sequence
        \[
        0 \to F(A') \xrightarrow{F(f)} F(A) \xrightarrow{F(g)} F(A'')
        \]
        is exact at $F(A')$ and $F(A)$ (i.e., $F$ preserves \CrefAndHyperrefIfExist{definition:monomorphism_and_epimorphism_in_categories}{monomorphisms} and exactness at the first two terms).

        \item Dually, $F$ is called \hldef{right exact} if it preserves all \CrefAndHyperrefIfExist{definition:limit_and_colimit_of_a_diagram_in_a_category}{finite colimits}, or equivalently it preserves \CrefAndHyperrefIfExist{definition:kernel_and_cokernel_of_a_morphism_in_a_category}{cokernels} and any finite colimit diagrams. Equivalently, for every right exact sequence in $\mathcal{A}$
        \[
        A' \xrightarrow{f} A \xrightarrow{g} A'' \to 0,
        \]
        the sequence
        \[
        F(A') \xrightarrow{F(f)} F(A) \xrightarrow{F(g)} F(A'') \to 0
        \]
        is exact at $F(A)$ and $F(A'')$ (i.e., $F$ preserves \CrefAndHyperrefIfExist{definition:monomorphism_and_epimorphism_in_categories}{epimorphisms} and exactness at the last two terms).

        \item $F$ is called \hldef{exact} if it is both left and right exact.
    \end{enumerate}

    \TextIfExists{definition:left_right_exact_functor_between_categories}{
        The additive functor $F$ is left/right exact if and only if it is \CrefAndHyperref{definition:left_right_exact_functor_between_categories}{left/right exact} in the more general sense, i.e. if it preserves all \CrefAndHyperrefIfExist{definition:small_and_finite_limits_and_colimits_in_a_category}{finite} \CrefAndHyperrefIfExist{definition:limit_and_colimit_of_a_diagram_in_a_category}{limits/colimits}

    }
\end{definition}

\begin{definition} \label{definition:has_enough_injectives_or_projectives_for_an_abelian_category}
Let $\mathcal{A}$ be an \CrefAndHyperrefIfExist{definition:abelian_category}{abelian category}.
\begin{enumerate}
    \item $\mathcal{A}$ is said to \hldef{have enough injectives} if for every object $A$ in $\calA$, there is an \CrefAndHyperrefIfExist{definition:monomorphism_and_epimorphism_in_categories}{monomorphism} $A \to I$ with $I$ an \CrefAndHyperrefIfExist{definition:injective_and_projective_objects_in_a_category}{injective object} of $\calA$. \TextIfExistsElse{definition:has_enough_objects_of_a_class_on_the_left_right_for_an_abelian_category}{Equivalently, $\calA$ has enough injectives if it has enough objects of the class of injectives on the right (\Cref{definition:has_enough_objects_of_a_class_on_the_left_right_for_an_abelian_category})}

    \item $\mathcal{A}$ is said to \hldef{have enough projectives} if for every object $A$ in $\calA$, there is a \CrefAndHyperrefIfExist{definition:monomorphism_and_epimorphism_in_categories}{epimorphism} $P \to A$ with $P$ a \CrefAndHyperrefIfExist{definition:injective_and_projective_objects_in_a_category}{projective object} of $\calA$. \TextIfExistsElse{definition:has_enough_objects_of_a_class_on_the_left_right_for_an_abelian_category}{Equivalently, $\calA$ has enough projectives if it has enough objects of the class of projectives on the left (\Cref{definition:has_enough_objects_of_a_class_on_the_left_right_for_an_abelian_category})}

\end{enumerate}
\end{definition}

% See Also
% definition:left_right_derived_functors_of_a_right_left_exact_functor_between_abelian_categories_where_source_has_enough_projectives_injectives 

\begin{lemma} [cf. {\cite[Lemma 2.2.5, Lemma 2.3.6]{weibel}}] \label{lemma:an_object_of_abelian_category_with_enough_objects_of_a_class_on_the_right_left_has_right_left_resolution_by_the_class}
Let $\mathcal{A}$ be an \CrefAndHyperrefIfExist{definition:abelian_category}{abelian category} and let $\calX$ be a class of objects in $\calA$.

\begin{enumerate}
    \item If $\mathcal{A}$ \CrefAndHyperrefIfExist{definition:has_enough_objects_of_a_class_on_the_left_right_for_an_abelian_category}{has enough objects of class $\calX$ on the right}, then for every object $A \in \mathcal{A}$ there exists an \CrefAndHyperrefIfExist{definition:left_right_resolution_of_a_class_of_objects_in_an_abelian_category}{$\calX$-right resolution of $A$}.

    \item If $\mathcal{A}$ \CrefAndHyperrefIfExist{definition:has_enough_objects_of_a_class_on_the_left_right_for_an_abelian_category}{has enough objects of class $\calX$ on the left}, then for every object $A \in \mathcal{A}$ there exists an \CrefAndHyperrefIfExist{definition:left_right_resolution_of_a_class_of_objects_in_an_abelian_category}{$\calX$-left resolution of $A$}.

\end{enumerate}

Note that this is a special case of \Cref{proposition:abelian_category_with_enough_objets_of_a_class_on_the_right_left_has_resolutions_of_complexes} obtained by letting the complex $M^\bullet$ be the complex such that
$$M^i = \begin{cases} A &\text{if } i = 0 \\ 0 &\text{otherwise} \end{cases}.$$


In particular,
\begin{itemize}
    \item If $\mathcal{A}$ \CrefAndHyperrefIfExist{definition:has_enough_injectives_or_projectives_for_an_abelian_category}{has enough injective objects}, then for every object $A \in \mathcal{A}$ there exists an \CrefAndHyperrefIfExist{definition:left_right_resolution_of_a_class_of_objects_in_an_abelian_category}{injective resolution of $A$}.
    \item If $\mathcal{A}$ \CrefAndHyperrefIfExist{definition:has_enough_injectives_or_projectives_for_an_abelian_category}{has enough projective objects}, then for every object $A \in \mathcal{A}$ there exists a \CrefAndHyperrefIfExist{definition:left_right_resolution_of_a_class_of_objects_in_an_abelian_category}{projective resolution of $A$}.
    \item If $F: \calA \to \calB$ is a \CrefAndHyperrefIfExist{definition:exact_functor_between_abelian_categories}{left (resp. right) exact functor} between abelian categories and $\calA$ has enough $F$-acyclic objects on the right (resp. left), then for every object $A \in \calA$, there exists an \CrefAndHyperrefIfExist{definition:F_acyclic_resolution_for_a_right_left_exact_functor_between_abelian_categories}{right (resp. left) $F$-acyclic resolution} of $A$.
\end{itemize}

\end{lemma}
\begin{proof}
    \begin{enumerate}
        \item Let $A \in \calA$ be an object. Since $\calA$ has enough objects of class $\calX$ of the right, there is an object $X_0$ of $\calX$ and a monomorphism $\varepsilon_0: A \to X_0$. Let \CrefAndHyperrefIfExist{definition:kernel_and_cokernel_of_a_morphism_in_a_category}{$A_0 = \operatorname{coker} \varepsilon_0$}. Inductively, given an object $A_{n-1}$ of $\calA$, choose an object $X_n$ of $\calX$ and a monomorphism $\varepsilon_{n}: A_{n-1} \hookrightarrow X_n$. Let $A_n = \operatorname{coker} \varepsilon_n$. In particular, there is a surjection $X_n \twoheadrightarrow A_n$. Let $d_n$ be the composition
        $$X_{n-1} \twoheadrightarrow A_{n-1} \xrightarrow{\varepsilon_n} X_n.$$
        The chain complex
        $$0 \to A \xrightarrow{\varepsilon_0} X_0 \xrightarrow{d_0} X_1 \xrightarrow{d_1} \cdots$$
        is thus an $\calX$-right resolution of $A$.

        \item This is simply the dual statement of the next statement.

    \end{enumerate} 
\end{proof}

\begin{definition} \label{definition:left_right_derived_functors_of_a_right_left_exact_functor_between_abelian_categories_where_source_has_enough_projectives_injectives}
    \TODO{I think that the definition of derived categories might be doable for more general kinds of resolutions? Perhaps it is that if I have a right exact functor $F$, then $L^i F$ can be computed with resolutions of $F$-acyclic objects? \CrefIfExists{definition:F_acyclic_object_for_a_left_or_right_functor_between_abelian_categories}}
    \TODO{Apparently, left/right derived functors may be defined for functors that are additive and preserve finite coproducts, and not necessarily right/left exact; the exactness condition ensures that the zeroth derived functor agrees with $F$.}
Let $\mathcal{A}$ and $\mathcal{B}$ be \CrefAndHyperrefIfExist{definition:abelian_category}{abelian categories}, and let 
$$F: \mathcal{A} \to \mathcal{B}$$ 
be an \CrefAndHyperrefIfExist{definition:additive_functor_between_additive_categories}{additive functor}.

\begin{enumerate}
    \item Suppose that the functor $F$ is \CrefAndHyperrefIfExist{definition:exact_functor_between_abelian_categories}{right exact} and suppose that $A \in \calA$ is an object for which a \CrefAndHyperrefIfExist{definition:left_right_resolution_of_a_class_of_objects_in_an_abelian_category}{projective resolution}
    $$\cdots \to P_2 \to P_1 \to P_0 \to A \to 0$$
    exists in $\calA$. We define the \hldef{left derived object} \hl{$L_n F A \in \calB$} by applying $F$ to obtain a complex
    $$\cdots \to F(P_2) \to F(P_1) \to F(P_0) \to 0$$
    and letting $L_n F(A)$ be the \CrefAndHyperrefIfExist{definition:homology_and_cohomology_objects_for_a_chain_complex_in_an_additive_category}{$n$-th homology object} of this complex in $\mathcal{B}$:
    $$L_n F(A) := H_n(F(P_\bullet)).$$
    The object $L_n F(A)$ is independent of the choice of projective resolution up to natural isomorphism (\Cref{proposition:left_right_derived_objects_for_a_right_left_exact_functor_between_abelian_categories_are_well_defined}). 

    By convention, set $L_n F = 0$ for $n < 0$.

    The \hldef{higher left derived objects} refer to the object $L_n F(A)$ for $n > 0$. 

    \item  Suppose that the functor $F$ is \CrefAndHyperrefIfExist{definition:exact_functor_between_abelian_categories}{right exact} and that $\calA$ \CrefAndHyperrefIfExist{definition:has_enough_injectives_or_projectives_for_an_abelian_category}{has enough projectives}. The \hldef{left derived functors} refer to the family of functors
    $$\hlin{L_n F : \mathcal{A} \to \mathcal{B}, \quad A \mapsto L_n F(A).}$$
    The \hldef{higher left derived functors} refer to the functors $L_n F$ for $n > 0$. 

    \item Suppose that the functor $F$ is \CrefAndHyperrefIfExist{definition:exact_functor_between_abelian_categories}{right exact} and suppose that $A \in \calA$ is an object for which a \CrefAndHyperrefIfExist{definition:left_right_resolution_of_a_class_of_objects_in_an_abelian_category}{injective resolution}
    $$0 \to A \to I^0 \to I^1 \to I^2 \to \cdots$$
    exists in $\calA$. We define the \hldef{right derived object} \hl{$R_n F A \in \calB$}, also often denoted by \hl{$R^n FA$}, by applying $F$ to obtain a complex
    $$0 \to F(I^0) \to F(I^1) \to F(I^2) \to \cdots.$$
    and letting $R_n F(A)$ be the \CrefAndHyperrefIfExist{definition:homology_and_cohomology_objects_for_a_chain_complex_in_an_additive_category}{$n$-th cohomology object} of this complex in $\mathcal{B}$:
    $$R_n F(A) := H^n(F(I_\bullet)).$$
    The object $R_n F(A)$ is independent of the choice of injective resolution up to natural isomorphism (\Cref{proposition:left_right_derived_objects_for_a_right_left_exact_functor_between_abelian_categories_are_well_defined}). 

    By convention, set $R_n F = 0$ for $n < 0$.

    The \hldef{higher right derived objects} refer to the object $R_n F(A)$ for $n > 0$. 

    \item  Suppose that the functor $F$ is \CrefAndHyperrefIfExist{definition:exact_functor_between_abelian_categories}{right exact} and that $\calA$ \CrefAndHyperrefIfExist{definition:has_enough_injectives_or_projectives_for_an_abelian_category}{has enough injectives}. The \hldef{right derived functors} refer to the family of functors
    $$\hlin{R_n F : \mathcal{A} \to \mathcal{B}, \quad A \mapsto R_n F(A).}$$
    The right derived functors are also often denoted by \hl{$R^n F$}.
    The \hldef{higher right derived functors} refer to the functors $R_n F$ for $n > 0$. 

    
    % If the functor $F$ is right exact and $\calA$ \CrefAndHyperrefIfExist{definition:has_enough_injectives_or_projectives_for_an_abelian_category}{has enough projectives}, then its \hldef{left derived functors} are a family of functors
    % $$\hlin{L_n F : \mathcal{A} \to \mathcal{B}, \quad n \geq 0,}$$
    % which are defined for each object $A$ in $\mathcal{A}$ by choosing (\Cref{lemma:an_object_of_abelian_category_with_enough_objects_of_a_class_on_the_right_left_has_right_left_resolution_by_the_class}) a \CrefAndHyperrefIfExist{definition:left_right_resolution_of_a_class_of_objects_in_an_abelian_category}{projective resolution}
    % $$\cdots \to P_2 \to P_1 \to P_0 \to A \to 0$$
    % in $\mathcal{A}$ and applying $F$ to obtain a complex
    % $$\cdots \to F(P_2) \to F(P_1) \to F(P_0) \to 0.$$
    % Then $L_n F(A)$ is defined to be the \CrefAndHyperrefIfExist{definition:homology_and_cohomology_objects_for_a_chain_complex_in_an_additive_category}{$n$-th homology object} of this complex in $\mathcal{B}$:
    % $$L_n F(A) := H_n(F(P_\bullet)).$$
    % The functors $L_n F$ are independent of the choice of projective resolution up to natural isomorphism. 

    % By convention, set $L_n F = 0$ for $n < 0$.

    % \item If the functor $F$ is \CrefAndHyperrefIfExist{definition:exact_functor_between_abelian_categories}{left exact} and $\calA$ \CrefAndHyperrefIfExist{definition:has_enough_injectives_or_projectives_for_an_abelian_category}{has enough injectives}, then its \hldef{right derived functors} are a family of functors
    % $$\hlin{R^n F : \mathcal{A} \to \mathcal{B}, \quad n \geq 0,}$$
    % which are defined for each object $A$ in $\mathcal{A}$ by choosing (\Cref{lemma:an_object_of_abelian_category_with_enough_objects_of_a_class_on_the_right_left_has_right_left_resolution_by_the_class}) an \CrefAndHyperrefIfExist{definition:left_right_resolution_of_a_class_of_objects_in_an_abelian_category}{injective resolution}
    % $$0 \to A \to I^0 \to I^1 \to I^2 \to \cdots$$
    % in $\mathcal{A}$ and applying $F$ to obtain a complex
    % $$0 \to F(I^0) \to F(I^1) \to F(I^2) \to \cdots.$$
    % Then $R^n F(A)$ is defined to be the \CrefAndHyperrefIfExist{definition:homology_and_cohomology_objects_for_a_chain_complex_in_an_additive_category}{$n$-th cohomology object} of this complex in $\mathcal{B}$:
    % $$R^n F(A) := H^n(F(I^\bullet)).$$
    % The functors $R^n F$ are independent of the choice of injective resolution up to natural isomorphism.

    % By convention, set $R^n F = 0$ for $n < 0$.
\end{enumerate}
\end{definition}

\begin{proposition}[cf.{\cite[Lemma 2.4.1]{weibel}}] \label{proposition:left_right_derived_objects_for_a_right_left_exact_functor_between_abelian_categories_are_well_defined}
    Let $F: \calA \to \calB$ be an \CrefAndHyperrefIfExist{definition:additive_functor_between_additive_categories}{additive functor} between \CrefAndHyperrefIfExist{definition:abelian_category}{abelian categories}. Let $A$ be an object of $\calA$. 
    \begin{enumerate}
        \item Suppose that $F$ is \CrefAndHyperrefIfExist{definition:exact_functor_between_abelian_categories}{right exact}, and suppose that a \CrefAndHyperrefIfExist{definition:left_right_resolution_of_a_class_of_objects_in_an_abelian_category}{projective resolution}
        $$\cdots \to P_2 \to P_1 \to P_0 \to A \to 0$$
        of $A$ exists in $\calA$. Let 
        $$\cdots \to Q_2 \to Q_1 \to Q_0 \to A \to 0$$
        be any projective resolution of $A$ in $\calA$. For all $n$, there are natural isomorphisms
        $$H_n(F(P_\bullet)) \cong H_n(F(Q_\bullet)).$$
        In other words, the \CrefAndHyperrefIfExist{definition:left_right_derived_functors_of_a_right_left_exact_functor_between_abelian_categories_where_source_has_enough_projectives_injectives}{left derived objects $L_n F(A)$} is well defined.

        \item Suppose that $F$ is \CrefAndHyperrefIfExist{definition:exact_functor_between_abelian_categories}{left exact}, and suppose that a \CrefAndHyperrefIfExist{definition:left_right_resolution_of_a_class_of_objects_in_an_abelian_category}{injective resolution}
        $$0 \to A \to I^0 \to I^1 \to I^2 \to \cdots$$
        of $A$ exists in $\calA$. Let 
        $$0 \to A \to Q^0 \to Q^1 \to Q^2 \to \cdots$$
        be any injective resolution of $A$ in $\calA$. For all $n$, there are natural isomorphisms
        $$H_n(F(I^\bullet)) \cong H_n(F(Q^\bullet)).$$
        In other words, the \CrefAndHyperrefIfExist{definition:left_right_derived_functors_of_a_right_left_exact_functor_between_abelian_categories_where_source_has_enough_projectives_injectives}{right derived objects $R_n F(A)$} is well defined.
    \end{enumerate}
\end{proposition}

\begin{proof}
    \begin{enumerate}
        \item By \Cref{lemma:projective_injective_complex_with_map_to_from_object_with_left_right_resolution_lifts_uniquely_up_to_chain_homotopy}, there is a lift $f: P_\bullet \to Q_\bullet$ of the identity map $A \to A$ unique up to chain homotopy. There are then induced natural maps $H_n(F(f)): H_n(F(P_\bullet)) \to H_n(F(Q_\bullet))$. There is also a lift $f': Q_\bullet \to P_\bullet$ of the identity map $A \to A$ unique up to chain homotopy, and this also induces natural maps $H_n(F(f')): H_n(F(Q_\bullet)) \to H_n(F(P_\bullet))$. The chain maps $f$ and $f'$ are in fact \CrefAndHyperrefIfExist{definition:chain_homotopy_between_chain_maps_between_complexes}{chain homotopy inverses} because \Cref{lemma:projective_injective_complex_with_map_to_from_object_with_left_right_resolution_lifts_uniquely_up_to_chain_homotopy} also implies that any lifts $P_\bullet \to P_\bullet$ and $Q_\bullet \to Q_\bullet$ of the identity map $A \to A$ are chain homotopic to the identity chain maps. Therefore, $H_n(F(f))$ and $H_n(F(f'))$ are inverses of each other as morphisms in $\calB$.  \TODO{prove basic facts about the fucntoriality of homology/cohomology of chain complexes}

        \item This is dual to the previous part.
    \end{enumerate}
\end{proof}
\begin{lemma}[cf. {\cite[Porism 2.2.7]{weibel}}] \label{lemma:projective_injective_complex_with_map_to_from_object_with_left_right_resolution_lifts_uniquely_up_to_chain_homotopy}
    Let $\calA$ be an \CrefAndHyperrefIfExist{definition:abelian_category}{abelian category}.
    \begin{enumerate}
        \item Let
        $$\cdots \to P_2 \to P_1 \to P_0 \to M \to 0$$
        be a \CrefAndHyperrefIfExist{definition:chain_complex_of_objects_in_an_additive_category}{chain complex} with $P_i$ \CrefAndHyperrefIfExist{definition:injective_and_projective_objects_in_a_category}{projective}. For every \CrefAndHyperrefIfExist{definition:left_right_resolution_of_a_class_of_objects_in_an_abelian_category}{left resolution} $Q_\bullet \to N$ of an object $N$, every map $M \to N$ lifts to a \CrefAndHyperrefIfExist{definition:chain_complex_of_objects_in_an_additive_category}{complex map} $P_\bullet \to Q_\bullet$ unique up to \CrefAndHyperrefIfExist{definition:chain_homotopy_between_chain_maps_between_complexes}{chain homotopy}.

        \item Let
        $$0 \to M \to I^0 \to I^1 \to I^2 \to \cdots$$
        be a \CrefAndHyperrefIfExist{definition:chain_complex_of_objects_in_an_additive_category}{(co)chain complex} with $I^i$ \CrefAndHyperrefIfExist{definition:injective_and_projective_objects_in_a_category}{injective}. For every \CrefAndHyperrefIfExist{definition:left_right_resolution_of_a_class_of_objects_in_an_abelian_category}{right resolution} $N \to Q^\bullet$ of an object $N$, every map $N \to M$ lifts to a \CrefAndHyperrefIfExist{definition:chain_complex_of_objects_in_an_additive_category}{complex map} $Q^\bullet \to I^\bullet$ unique up to \CrefAndHyperrefIfExist{definition:chain_homotopy_between_chain_maps_between_complexes}{chain homotopy}.
    \end{enumerate}
\end{lemma}

\begin{proof}
    \begin{enumerate}
        \item The map $P_0 \to M \to N$ lifts to a map $P_0 \to Q_0$ because $P_0$ is projective and $Q_0 \to N$ is an epimorphism. 
        Inductively suppose that there are morphisms $P_i \to Q_i$ for $0 \leq i \leq n$, where $n \geq 0$ that make 
        \begin{center}
        \begin{tikzcd}
            P_n \ar[r] \ar[d] & P_{n-1} \ar[r] \ar[d] & \cdots \ar[r] & P_0 \ar[r] \ar[d] & M \ar[r] \ar[d] & 0 \\
            Q_n \ar[r] & Q_{n-1} \ar[r] & \cdots \ar[r] & Q_0 \ar[r] & N \ar[r] & 0 \\
        \end{tikzcd}
        \end{center}
        into a commuting diagram are established. The morphism $Q_{n} \to Q_{n-1}$ (where we let $Q_{-1} = N$ and $P_{-1} = M$ here in case that $n = 0$) acts as $0$ when restricted to \CrefAndHyperrefIfExist{definition:image_coimage_of_a_morphism_in_a_category}{$\mathfrak{I} \coloneq \operatorname{im} (P_{n+1} \to P_n \to Q_n)$} because the composition 
        $$P_{n+1} \to P_n \to Q_n \to Q_{n-1}$$
        equals the composition 
        $$P_{n+1} \to P_n \to P_{n-1} \to Q_{n-1}.$$
        In other words, $\mathfrak{I}$ is a \CrefAndHyperrefIfExist{definition:subobject_of_an_object_of_an_additive_category}{subobject} of \CrefAndHyperrefIfExist{definition:kernel_and_cokernel_of_a_morphism_in_a_category}{$\ker(Q_n \to Q_{n-1})$}, which is isomorphic to $\operatorname{im}(Q_{n+1} \to Q_n)$ by the acyclicity of the sequence of the $Q_i$'s. Therefore, we have a map $P_{n+1} \twoheadrightarrow \mathfrak{I}\hookrightarrow \operatorname{im}(Q_{n+1} \to Q_n)$ along with an epimorphism $Q_{n+1} \twoheadrightarrow \operatorname{im}(Q_{n+1} \to Q_n)$. Since $P_{n+1}$ is projective, the former map lifts to a map $P_{n+1} \to Q_{n+1}$ in a way that is compatible with the latter, i.e. the following commutes:
        \begin{center}
        \begin{tikzcd}
            P_{n+1} \ar[rd] \ar[d,dotted] & \\
            Q_{n+1} \ar[r] & \operatorname{im}(Q_{n+1} \to Q_n).
        \end{tikzcd}
        \end{center}
        By induction, this shows thta $M \to N$ lifts to a morphism $P_\bullet \to Q_\bullet$ of complexes.

        We show that the morphism of complexes is unique up to chain homotopy, i.e. if $f_1, f_2: P_\bullet \to Q_\bullet$ are two morphisms of complexes, then $h \coloneq f_1 - f_2$ is null homotopic. We construct a \CrefAndHyperrefIfExist{definition:chain_homotopy_between_chain_maps_between_complexes}{chain contraction} $\{s_n: P_n \to Q_{n+1}\}$ of $h$ by induction on $n$. If $n < 0$, then set $s_n = 0$. If $n = 0$, note that the composition $P_0 \xrightarrow{h_0} Q_0 \to N$ equals the composition $P_0 \to M \xrightarrow{0} N$, so $\operatorname{im}(h_0)$ is a subobject of $\ker(Q_0 \to N) \cong \operatorname{im}(Q_1 \to Q_0)$. The projectivitiy of $P_0$ thus yields a lift $s_0: P_0 \to Q_1$ such that $h_0$ equals the composition $P_0 \xrightarrow{s_0} X_1 \xrightarrow{d} Q_0$:
        \begin{center}
        \begin{tikzcd}
           &  P_0 \ar[dl, "s_0", dotted] \ar[d, "h_0"] \\
           Q_1 \ar[r, "d"] & Q_0 
        \end{tikzcd}
        \end{center}
        Note moreover that $h_0 = ds_0 + s_{-1} d$ because $s_{-1} = 0$. Inductively suppose that we have maps $s_i$ for $i \leq n$  such that $h_n = d s_{n} + s_{n-1} d$ or equivalently that $ds_{n} = h_n - s_{n-1} d$. Consider the map $h_{n+1} - s_{n} d: P_{n+1} \to Q_{n+1}$. Compute
        $$d(h_{n+1} - s_{n} d) = dh_{n+1} - d s_{n} d = dh_{n+1} - (h_n - s_{n-1}d)d = (dh_{n+1} - h_n d) + s_{n-1} d d = 0$$
        Therefore, $\operatorname{im} (h_{n+1} - s_n d)$ is a subobject of $\ker(Q_{n+1} \to Q_{n}) \cong \operatorname{im}(Q_{n+2} \to Q_{n+1})$, which is in turn a quotient of $Q_{n+2}$. Since $P_{n+1}$ is projective, there is a morphism $s_{n+1}: P_{n+1} \to Q_{n+2}$ such that $d s_{n+1} = h_{n+1} - s_{n} d$. 
        \begin{center}
        \begin{tikzcd}
           &  P_{n+1} \ar[dl, dotted, "s_{n+1}"] \ar[d, "h_n - s_{n-1} d = ds_{n}"] \\
           Q_{n+2} \ar[r, "d"]  &\operatorname{im}(Q_{n+2} \to Q_{n+1}) \cong \ker(Q_{n+1} \to Q_{n})
        \end{tikzcd}
        \end{center}
        The $s_n$ thus form a chain contraction as needed.

        \item This is simply dual to the previous part.
    \end{enumerate}
\end{proof}


\subsection{Homological algebra}
% \begin{definition}[Chain complex in an additive category] \label{definition:chain_complex_of_objects_in_an_additive_category}
% Let $\mathcal{A}$ be an \hyperrefIfExists{definition:additive_category_preadditive_category}{preadditive category} and let $I$ be a totally ordered set (typically $\mathbb{Z}$, but $I \subseteq \mathbb{Z}$ is also allowed). 
% \begin{enumerate}
%     \item A \hldef{chain complex} $(K^\bullet, d^\bullet)$ in $\mathcal{A}$ indexed by $I$ consists of:
%     \begin{itemize}
%         \item Objects $\{ K^i \}_{i \in I}$ in $\mathcal{A}$, called the \hldef{terms in degree $i$},
%         \item Morphisms $d^i: K^i \to K^{i+1}$ in $\mathcal{A}$, called the \hldef{differentials in degree $i$},
%     \end{itemize}
%     such that for every $i \in I$, $d^{i+1} \circ d^i = 0$. That is,
%     $$ K^\bullet: \cdots \xrightarrow{d^{i-2}} K^{i-1} \xrightarrow{d^{i-1}} K^i \xrightarrow{d^i} K^{i+1} \xrightarrow{d^{i+1}} \cdots $$
%     with $d^{i+1}d^i = 0$ for all $i$. We might typically use notation such as \hl{$K^\bullet = (K^i, d^i)_{i \in I}$} to denote a chain complex in $\mathcal{A}$.

%     A cochain complex can be defined similarly/dually.

%     \item Let $K^\bullet = (K^i, d_K^i)$ and $L^\bullet = (L^i, d_L^i)$ be \CrefAndHyperrefIfExist{definition:chain_complex_of_objects_in_an_additive_category}{chain complexes} in $\mathcal{A}$ indexed by the same set $I$. 
%     A \hldef{morphism of chain complexes} (or \hldef{chain map})
%     $$ f^\bullet: K^\bullet \to L^\bullet $$
%     consists of morphisms $f^i: K^i \to L^i$ for all $i \in I$, such that for every $i \in I$,
%     $$ d_L^i \circ f^i = f^{i+1} \circ d_K^i, $$
%     i.e., the following diagram commutes for all $i$:

%     $$ \begin{array}{ccc} K^i & \xrightarrow{d_K^i} & K^{i+1} \\ \downarrow{f^i} && \downarrow{f^{i+1}} \\ L^i & \xrightarrow{d_L^i} & L^{i+1} \end{array}.$$

% \end{enumerate}

% There is then a category, often denoted by \hl{$\mathrm{Ch}(\mathcal{A})$} or \hl{$\mathbf{Ch}(\mathcal{A})$}, whose objects are chain complexes in $\calA$ and whose morphisms are morphisms of chain complexes. In particular, we may denote by 
% $$\hlin{\operatorname{Hom}(K^\bullet, L^\bullet)=  \operatorname{Hom}_{\mathrm{Ch}(\mathcal{A})}(K^\bullet, L^\bullet)}$$
% the set of chain maps $K^\bullet \to L^\bullet$; it is in fact an abelian group.

% A \hldef{morphism of cochain complexes} is defined similarly, and we similarly denote by \hl{$\mathrm{Ch}(\mathcal{A})$} or \hl{$\mathbf{Ch}(\mathcal{A})$} the caetgory of cochain complexes in $\calA$. 


% \TextIfExists{definition:dg_category_over_a_ring}{
% If $k$ is a \CrefAndHyperrefIfExist{definition:commutative_ring}{commutative ring} such that $\Hom_\calA(X,Y)$ is \CrefAndHyperrefIfExist{definition:category_enriched_in_a_monoidal_category}{enriched in} the category of \CrefAndHyperrefIfExist{definition:module_of_a_ring}{$k$-modules}, then $\mathrm{Ch}(\calA)$ \CrefAndHyperref{definition:category_of_chain_complexes_of_objects_in_an_additive_category_as_a_dg_category}{can be equipped with} the structure of a \CrefAndHyperrefIfExist{definition:dg_category_over_a_ring}{dg-category over $k$}.
% }


% \end{definition}

\begin{definition}[Chain complex in a preadditive category] \label{definition:chain_complex_of_objects_in_an_additive_category}
Let $\mathcal{A}$ be a \hyperrefIfExists{definition:additive_category_preadditive_category}{preadditive category} and let $I$ be a totally ordered set (typically $\mathbb{Z}$, but $I \subseteq \mathbb{Z}$ is also allowed). 
\begin{enumerate}
    \item A \hldef{chain complex} $(K_\bullet, d_\bullet)$ in $\mathcal{A}$ indexed by $I$ is the homological convention for sequences with decreasing degrees. It consists of:
    \begin{itemize}
        \item Objects $\{ K_i \}_{i \in I}$ in $\mathcal{A}$, called the \hldef{terms in degree $i$},
        \item Morphisms $d_i: K_i \to K_{i-1}$ in $\mathcal{A}$, called the \hldef{boundary maps} or \hldef{differentials in degree $i$},
    \end{itemize}
    such that for every $i \in I$, $d_{i-1} \circ d_i = 0$. That is,
    $$ K_\bullet: \cdots \xrightarrow{d_{i+1}} K_i \xrightarrow{d_i} K_{i-1} \xrightarrow{d_{i-1}} K_{i-2} \xrightarrow{} \cdots $$
    with $d_{i-1}d_i = 0$ for all $i$. We typically use the notation \hl{$K_\bullet = (K_i, d_i)_{i \in I}$}.



    \item Dually, a \hldef{cochain complex} $(K^\bullet, d^\bullet)$ in $\mathcal{A}$ follows the \hldef{cohomological convention} with increasing degrees. It consists of objects $\{ K^i \}_{i \in I}$ and \hldef{coboundary maps} $d^i: K^i \to K^{i+1}$ such that $d^{i+1} \circ d^i = 0$:
    $$ K^\bullet: \cdots \xrightarrow{d^{i-1}} K^i \xrightarrow{d^i} K^{i+1} \xrightarrow{d^{i+1}} K^{i+2} \xrightarrow{} \cdots $$
    We typically use the notation \hl{$K^\bullet = (K^i, d^i)_{i \in I}$}.

    \item Let $K_\bullet = (K_i, d_i^K)$ and $L_\bullet = (L_i, d_i^L)$ be \CrefAndHyperrefIfExist{definition:chain_complex_of_objects_in_an_additive_category}{chain complexes} in $\mathcal{A}$ indexed by the same set $I$. A \hldef{morphism of chain complexes} (or \hldef{chain map})
    $$ f_\bullet: K_\bullet \to L_\bullet $$
    consists of morphisms $f_i: K_i \to L_i$ for all $i \in I$, such that for every $i \in I$, the following diagram commutes:
    $$ \begin{array}{ccc} K_i & \xrightarrow{d_i^K} & K_{i-1} \\ \downarrow{f_i} && \downarrow{f_{i-1}} \\ L_i & \xrightarrow{d_i^L} & L_{i-1} \end{array} $$
    i.e., $d_i^L \circ f_i = f_{i-1} \circ d_i^K$. 



    A \hldef{morphism of cochain complexes} $f^\bullet: K^\bullet \to L^\bullet$ is defined similarly, satisfying the commutativity condition $d_L^i \circ f^i = f^{i+1} \circ d_K^i$.
\end{enumerate}

The collection of these objects and morphisms forms a category. Notation for these categories is as follows:
\begin{itemize}
    \item \hl{$\mathrm{Ch}(\mathcal{A})$} or \hl{$\mathbf{Ch}(\mathcal{A})$} is often used as a general term.
    \item To be explicit about the indexing convention, one uses \hl{$\mathrm{Ch}_\bullet(\mathcal{A})$} for chain complexes and \hl{$\mathrm{Ch}^\bullet(\mathcal{A})$} (or sometimes $\mathrm{CoCh}(\mathcal{A})$) for cochain complexes.
    \item The set of chain maps between two complexes is denoted by $\hlin{\operatorname{Hom}_{\mathrm{Ch}(\mathcal{A})}(K_\bullet, L_\bullet)}$; it is an abelian group under pointwise addition $(f+g)_i = f_i + g_i$.
\end{itemize}

\TextIfExists{definition:dg_category_over_a_ring}{
If $k$ is a \CrefAndHyperrefIfExist{definition:commutative_ring}{commutative ring} such that $\Hom_\calA(X,Y)$ is \CrefAndHyperrefIfExist{definition:category_enriched_in_a_monoidal_category}{enriched in} the category of \CrefAndHyperrefIfExist{definition:module_of_a_ring}{$k$-modules}, then $\mathrm{Ch}(\calA)$ \CrefAndHyperref{definition:category_of_chain_complexes_of_objects_in_an_additive_category_as_a_dg_category}{can be equipped with} the structure of a \CrefAndHyperrefIfExist{definition:dg_category_over_a_ring}{dg-category over $k$}.
}
\end{definition}

% 
\begin{remark} \label{remark:cohomological_vs_homological_conventions}
    The convention used to define chain complexes in \Cref{definition:chain_complex_of_objects_in_an_additive_category} is a \emph{cohomological one} --- note that indices are written as superscripts and increase when ``following the arrows''. Such a chain complex may also be referred to as a \hldef{cochain complex} or a \hldef{cohomological chain complex} to emphasize an adoption of a cohomological convention. 

    The dual convention would be a \emph{homological one}, in which indices are written as subscripts and decrease when ``following the arrow''. As such, one may speak of a \hldef{(homological) chain complex} $(K_\bullet, d_\bullet)$ indexed by $I$ as consisting of:

    \begin{itemize}
    \item Objects $\{ K_i \}_{i \in I}$ in $\mathcal{A}$, called the \hldef{terms in degree $i$},
    \item Morphisms $d_i: K_i \to K_{i-1}$ in $\mathcal{A}$, called the \hldef{differentials in degree $i$},
    \end{itemize}
    such that for every $i \in I$, $d_{i-1} \circ d_i = 0$. That is,
    $$ 
    K_\bullet: \cdots \xrightarrow{d_{i+1}} K_i \xrightarrow{d_i} K_{i-1} \xrightarrow{d_{i-1}} K_{i-2} \xrightarrow{d_{i-2}} \cdots
    $$
    with $d_{i-1} d_i = 0$ for all $i$. We might typically use notation such as \hl{$K_\bullet = (K_i, d_i)_{i \in I}$} to denote a chain complex in $\mathcal{A}$.

    The differences between the conventions persist --- for example, cohomological objects are usually written with superscript indicees whereas homological objects are usually written with subscript indicees.
\end{remark}
%\begin{convention} \label{convention:homological_algebra_is_discussed_in_cohomological_terms}
    When discussing homological algebra in abstract terms, we may often adopt the homological convention in some discussions and the cohomological convention in others \CrefIfExists{remark:cohomological_vs_homological_conventions}.
    %; for instance, indices are written as superscripts and increase along the direction of the arrows in chain complexes. 
\end{convention}

% \begin{definition}[Morphisms of chain complexes] \label{definition:chain_complex_of_objects_in_an_additive_category}
Let $\mathcal{A}$ be an \CrefAndHyperrefIfExist{definition:additive_category}{additive category}, and let $K^\bullet = (K^i, d_K^i)$ and $L^\bullet = (L^i, d_L^i)$ be \CrefAndHyperrefIfExist{definition:chain_complex_of_objects_in_an_additive_category}{chain complexes} in $\mathcal{A}$ indexed by the same set $I$. 
A \hldef{morphism of chain complexes} (or \hldef{chain map})
$$ f^\bullet: K^\bullet \to L^\bullet $$
consists of morphisms $f^i: K^i \to L^i$ for all $i \in I$, such that for every $i \in I$,
$$ d_L^i \circ f^i = f^{i+1} \circ d_K^i, $$
i.e., the following diagram commutes for all $i$:

$$ \begin{array}{ccc} K^i & \xrightarrow{d_K^i} & K^{i+1} \\ \downarrow{f^i} && \downarrow{f^{i+1}} \\ L^i & \xrightarrow{d_L^i} & L^{i+1} \end{array}.$$

There is then a category, often denoted by \hl{$\mathrm{Ch}(\mathcal{A})$} or \hl{$\mathbf{Ch}(\mathcal{A})$}, whose objects are chain complexes in $\calA$ and whose morphisms are morphisms of chain complexes. In particular, we may denote by 
$$\hlin{\operatorname{Hom}(K^\bullet, L^\bullet)=  \operatorname{Hom}_{\mathrm{Ch}(\mathcal{A})}(K^\bullet, L^\bullet)}$$
the set of chain maps $K^\bullet \to L^\bullet$; it is in fact an abelian group.

A \hldef{morphism of cochain complexes} is defined similarly, and we similarly denote by \hl{$\mathrm{Ch}(\mathcal{A})$} or \hl{$\mathbf{Ch}(\mathcal{A})$} the caetgory of cochain complexes in $\calA$. 
\end{definition}

% See Also
% 
\begin{proposition} \label{proposition:category_of_chain_complexes_in_an_additive_category_is_additive}
Let $\mathcal{A}$ be an \hyperrefIfExists{definition:additive_category}{additive category}. 
\begin{enumerate}
    \item The category \hyperrefIfExists{definition:chain_complex_of_objects_in_an_additive_category}{$\mathrm{Ch}(\calA)$} of chain complexes is itself and additive category.

    \item If $\calA$ is an \hyperrefIfExists{definition:abelian_category}{abelian category}, then $\mathrm{Ch}(\calA)$ is an abelian category.

    \item If $\calA$ is an \hyperrefIfExists{definition:abelian_category}{abelian category} satisfying Grothendieck's axiom \CrefAndHyperrefIfExist{definition:grothendiecks_additional_axioms_for_abelian_categories}{AB$n$ (resp. AB$n^*$)} for $n \in \{3,4,5,6\}$, then $\mathrm{Ch}(\calA)$ also satisfies AB$n$ (resp. AB$n^*$). If $\calA$ is a \CrefAndHyperrefIfExist{definition:grothendiecks_additional_axioms_for_abelian_categories}{Grothendieck abelian category}, then so is $\mathrm{Ch}(\calA)$
\end{enumerate}
\end{proposition}
\begin{proof}
    Combine \Cref{proposition:category_of_chain_complexes_of_objects_in_a_preadditive_category_is_equivalent_to_the_category_of_additive_functors_from_the_walking_chain_complex_category} and \Cref{lemma:additive_functor_category_from_small_preadditive_categories_preserves}.
\end{proof}

% \begin{corollary}
% Let $\calB$ be a \CrefAndHyperrefIfExist{definition:additive_category}{preadditive category}.
% \begin{enumerate}
%     \item The \CrefAndHyperrefIfExist{definition:chain_complex_of_objects_in_an_additive_category}{(co)chain complex category} $\text{Ch}(\calB)$ is preadditive. If $\calB$ is additionally \CrefAndHyperrefIfExist{definition:additive_category}{additive}/\CrefAndHyperrefIfExist{definition:abelian_category}{abelian}, then so is $\text{Ch}(\calB)$.

%     \item If $\calB$ is an abelian category with property \CrefAndHyperrefIfExist{definition:grothendiecks_additional_axioms_for_abelian_categories}{$ABn$ for $n = 3,4,5,6$ or $ABn^*$ for $n = 3,4,5$}, then $\text{Add}(\calA, \calB)$ possesses the same property.
% \end{enumerate}

% \end{corollary}

\begin{definition}[Boundedness conditions on chain complexes] \label{definition:bounded_complexes_on_an_additive_category_and_homologically_bounded_objects_on_an_abelian_category}
Let $\mathcal{A}$ be an additive category.

\noindent\textbf{Cohomological convention:} Let $K^\bullet = (K^i, d^i)_{i \in \mathbb{Z}}$ be a cohomologically indexed chain complex in $\mathcal{A}$.
\begin{itemize}
    \item $K^\bullet$ is \hldef{bounded above} if there exists $n \in \mathbb{Z}$ such that $K^i = 0$ for all $i > n$.
    \item $K^\bullet$ is \hldef{bounded below} if there exists $m \in \mathbb{Z}$ such that $K^i = 0$ for all $i < m$.
    \item $K^\bullet$ is \hldef{bounded} if it both bounded above and below, or equivalently if $K^i = 0$ for all but finitely many $i \in \mathbb{Z}$.
    \item Assuming that $\calA$ is an \hyperrefIfExists{definition:abelian_category}{abelian category}, 
    \begin{itemize}
        \item $K^\bullet$ is \hldef{cohomologically bounded above} if there exists $n \in \mathbb{Z}$ such that $H^i(K^\bullet) = 0$ for all $i > n$.
        \item $K^\bullet$ is \hldef{cohomologically bounded below} if there exists $m \in \mathbb{Z}$ such that $H^i(K^\bullet) = 0$ for all $i < m$.
        \item $K^\bullet$ is \hldef{cohomologically bounded} if it is both cohomologically bounded above and below, or equivalently if the \hyperrefIfExists{definition:homology_and_cohomology_objects_for_a_chain_complex_in_an_additive_category}{cohomology objects} $H^i(K^\bullet)$ vanish for all but finitely many $i \in \mathbb{Z}$.
    \end{itemize}
\end{itemize}

\noindent\textbf{Homological convention:} Let $K_\bullet = (K_i, d_i)_{i \in \mathbb{Z}}$ be a homologically indexed chain complex in $\mathcal{A}$.
\begin{itemize}
    \item $K_\bullet$ is \hldef{bounded above} if there exists $n \in \mathbb{Z}$ such that $K_i = 0$ for all $i > n$.
    \item $K_\bullet$ is \hldef{bounded below} if there exists $m \in \mathbb{Z}$ such that $K_i = 0$ for all $i < m$.

    % \item $K^\bullet$ is \hldef{bounded} if it both bounded above and below, equivalently if $K^i = 0$ for all but finitely many $i \in \mathbb{Z}$.
    \item $K_\bullet$ is \hldef{bounded} if it both bounded above and below, or equivalently if $K_i = 0$ for all but finitely many $i \in \mathbb{Z}$.
    \item Assuming that $\calA$ is an \hyperrefIfExists{definition:abelian_category}{abelian category},
    \begin{itemize}
        \item $K_\bullet$ is \hldef{homologically bounded above} if there exists $n \in \mathbb{Z}$ such that $H_i(K_\bullet) = 0$ for all $i > n$.
        \item $K_\bullet$ is \hldef{homologically bounded below} if there exists $m \in \mathbb{Z}$ such that $H_i(K_\bullet) = 0$ for all $i < m$.

        \item $K_\bullet$ is \hldef{homologically bounded} if it is both homologically bounded above and below, or equivalently if the \hyperrefIfExists{definition:homology_and_cohomology_objects_for_a_chain_complex_in_an_additive_category}{homology objects} $H_i(K_\bullet)$ vanish for all but finitely many $i \in \mathbb{Z}$.
    \end{itemize}
\end{itemize}
\end{definition}

% \begin{definition}[Boundedness conditions on chain complexes] \label{definition:bounded_complexes_on_an_additive_category_and_homologically_bounded_objects_on_an_abelian_category}
% Let $\mathcal{A}$ be an additive category.
% \noindent\textbf{Cohomological convention:} Let $K^\bullet = (K^i, d^i)_{i \in \mathbb{Z}}$ be a cohomologically indexed chain complex in $\mathcal{A}$.
% \begin{itemize}
%     \item $K^\bullet$ is \hldef{bounded} if $K^i = 0$ for all but finitely many $i \in \mathbb{Z}$.
%     \item $K^\bullet$ is \hldef{bounded above} if there exists $n \in \mathbb{Z}$ such that $K^i = 0$ for all $i > n$.
%     \item $K^\bullet$ is \hldef{bounded below} if there exists $m \in \mathbb{Z}$ such that $K^i = 0$ for all $i < m$.
%     \item Assuming that $\calA$ is an \hyperrefIfExists{definition:abelian_category}{abelian category}, $K^\bullet$ is \hldef{cohomologically bounded} if the \hyperrefIfExists{definition:homology_and_cohomology_objects_for_a_chain_complex_in_an_additive_category}{cohomology objects} $H^i(K^\bullet)$ vanish for all but finitely many $i \in \mathbb{Z}$.
% \end{itemize}
% \noindent\textbf{Homological convention:} Let $K_\bullet = (K_i, d_i)_{i \in \mathbb{Z}}$ be a homologically indexed chain complex in $\mathcal{A}$.
% \begin{itemize}
%     \item $K_\bullet$ is \hldef{bounded} if $K_i = 0$ for all but finitely many $i \in \mathbb{Z}$.
%     \item $K_\bullet$ is \hldef{bounded above} if there exists $n \in \mathbb{Z}$ such that $K_i = 0$ for all $i < n$.
%     \item $K_\bullet$ is \hldef{bounded below} if there exists $m \in \mathbb{Z}$ such that $K_i = 0$ for all $i > m$.
%     \item Assuming that $\calA$ is an \hyperrefIfExists{definition:abelian_category}{abelian category}, $K_\bullet$ is \hldef{homologically bounded} if the \hyperrefIfExists{definition:homology_and_cohomology_objects_for_a_chain_complex_in_an_additive_category}{homology objects} $H_i(K_\bullet)$ vanish for all but finitely many $i \in \mathbb{Z}$.
% \end{itemize}
% \end{definition}


% \begin{definition}[Chain complexes and their (co)homology objects] \label{definition:homology_and_cohomology_objects_for_a_chain_complex_in_an_additive_category}
%     Let $\mathcal{A}$ be an \hyperrefIfExists{definition:abelian_category}{abelian category}.
    
%     \begin{itemize}
%         \item For a cohomologically indexed chain complex $K^\bullet$, its \hldef{cohomology object in degree $i$} is defined by
%         $$ \hlin{H^i(K^\bullet) := \ker(d^i) / \operatorname{im}(d^{i-1})} $$
%         where the kernel and image are taken in $\mathcal{A}$.

%         \item For a homologically indexed chain complex $K_\bullet$, its \hldef{homology object in degree $i$} is defined by
%         $$ \hlin{H_i(K_\bullet) := \ker(d_i) / \operatorname{im}(d_{i+1})} $$
%         where the kernel and image are taken in $\mathcal{A}$.
%     \end{itemize}
% \end{definition}

\begin{definition}[Chain complexes and their (co)homology objects] \label{definition:homology_and_cohomology_objects_for_a_chain_complex_in_an_additive_category}
    Let $\mathcal{A}$ be an \hyperrefIfExists{definition:abelian_category}{abelian category}.
    
    \begin{itemize}
        \item For a \CrefAndHyperrefIfExist{definition:cochain_complex}{cochain complex} $K^\bullet$, its \hldef{cohomology object in degree $i$} is defined as the quotient of the object of $i$-cocycles by the object of $i$-coboundaries:
        $$ \hlin{H^i(K^\bullet) := Z^i(K) / B^i(K) = \ker(d^i) / \operatorname{im}(d^{i-1}).} $$

        \item For a \CrefAndHyperrefIfExist{definition:chain_complex}{chain complex} $K_\bullet$, its \hldef{homology object in degree $i$} is defined as the quotient of the object of $i$-cycles by the object of $i$-boundaries:
        $$ \hlin{H_i(K_\bullet) := Z_i(K) / B_i(K). = \ker(d_i) / \operatorname{im}(d_{i+1}).} $$
    \end{itemize}
\end{definition}



\begin{definition}[Quasi-isomorphism] \label{definition:quasi_isomorphism_of_chain_complexes_of_objects_in_an_abelian_category}
    Let $\mathcal{A}$ be an \hyperrefIfExists{definition:abelian_category}{abelian category}\CrefIfExists{definition:abelian_category}, and let
    \[
    f_\bullet: (C_\bullet, d_\bullet^C) \to (D_\bullet, d_\bullet^D)
    \]
    be a \hyperrefIfExists{definition:chain_complex_of_objects_in_an_additive_category}{chain map between complexes}\CrefIfExists{definition:chain_complex_of_objects_in_an_additive_category} in $\mathcal{A}$.

    The morphism $f_\bullet$ is called a \hldef{quasi-isomorphism} if it induces isomorphisms on all cohomology objects, i.e., for every integer $n$, the induced morphism on \hyperrefIfExists{definition:homology_and_cohomology_objects_for_a_chain_complex_in_an_additive_category}{homology}\CrefIfExists{definition:homology_and_cohomology_objects_for_a_chain_complex_in_an_additive_category} (or cohomology, depending on the convention)
    \[
    H^n(f_\bullet): H^n(C_\bullet) \to H^n(D_\bullet)
    \]
    is an isomorphism in $\mathcal{A}$.

    Note that all of these notions are applicable to the cohomological convention as well\CrefIfExists{remark:cohomological_vs_homological_conventions}.
\end{definition}

\begin{definition} \label{definition:mapping_cone_of_a_map_of_chain_cochain_complexes}
    \begin{enumerate}
        \item Let $f : (C_\bullet, d^C_\bullet) \to (D_\bullet, d^D_\bullet)$ be a \CrefAndHyperrefIfExist{definition:chain_complex_of_objects_in_an_additive_category}{morphism of chain complexes} in an \CrefAndHyperrefIfExist{definition:additive_category_preadditive_category}{additive category $\mathcal{A}$}.

        The \hldef{mapping cone of $f$}, denoted \hl{$\operatorname{Cone}(f)$}, is the chain complex defined by:
        \begin{itemize}
        \item Objects: For each $n$, 
        $$\operatorname{Cone}(f)_n = D_n \oplus C_{n-1}.$$
        \item Differential: For each $n$, define
        $$d^{\operatorname{Cone}(f)}_n : \operatorname{Cone}(f)_n \to \operatorname{Cone}(f)_{n-1}$$
        by the matrix morphism
        $$d^{\operatorname{Cone}(f)}_n = \begin{pmatrix} d^D_n & f_{n-1} \\ 0 & -d^C_{n-1} \end{pmatrix} : D_n \oplus C_{n-1} \to D_{n-1} \oplus C_{n-2}.$$
        \end{itemize}

        This construction defines a chain complex, i.e., $d^{\operatorname{Cone}(f)}_{n-1} \circ d^{\operatorname{Cone}(f)}_n = 0$.

        \item Dually, let $g : (C^\bullet, d_C^\bullet) \to (D^\bullet, d_D^\bullet)$ be a \CrefAndHyperrefIfExist{definition:chain_complex_of_objects_in_an_additive_category}{morphism of cochain complexes} in $\mathcal{A}$. 

        The \hldef{mapping cone of $g$}, denoted \hl{$\operatorname{Cone}(g)$}, is the \CrefAndHyperrefIfExist{definition:chain_complex_of_objects_in_an_additive_category}{cochain complex} defined by:
        \begin{itemize}
        \item Objects: For each $n$, 
        $$\operatorname{Cone}(g)^n = D^n \oplus C^{n+1}.$$
        \item Differential: For each $n$, define
        $$d_{\operatorname{Cone}(g)}^n : \operatorname{Cone}(g)^n \to \operatorname{Cone}(g)^{n+1}$$
        by the matrix morphism
        $$d_{\operatorname{Cone}(g)}^n = \begin{pmatrix} d_D^n & g^{n+1} \\ 0 & -d_C^{n+1} \end{pmatrix} : D^n \oplus C^{n+1} \to D^{n+1} \oplus C^{n+2}.$$
        \end{itemize}

        This construction defines a cochain complex, i.e., $d_{\operatorname{Cone}(g)}^{n+1} \circ d_{\operatorname{Cone}(g)}^n = 0$.
            \end{enumerate}
\end{definition}


\begin{definition} \label{definition:canonical_truncation_of_chain_complexes_of_objects_in_an_abelian_category}

    Let $\calA$ be an \CrefAndHyperrefIfExist{definition:abelian_category}{abelian category}.

    \begin{enumerate}
        \item  
    The \hldef{canonical truncations} of a 
    \CrefAndHyperrefIfExist{definition:chain_complex_of_objects_in_an_additive_category}{chain complex} 
    $A_\bullet$ in 
    \CrefAndHyperrefIfExist{definition:chain_complex_of_objects_in_an_additive_category}{$\mathbf{Ch}(\mathcal{A})$} 
    are defined by

    \hlalign{
    \begin{align*}
    (\tau_{\ge n} A)_i &=
    \begin{cases}
        A_i, & i > n, \\
        \ker(d_n : A_n \to A_{n-1}), & i = n, \\
        0, & i < n,
    \end{cases}
    &
    (\tau_{\le n} A)_i &=
    \begin{cases}
        0, & i > n, \\
        \mathrm{coker}(d_{n+1} : A_{n+1} \to A_n), & i = n, \\
        A_i, & i < n.
    \end{cases}
    \end{align*}
    }

    The differentials are the restrictions and/or quotient maps induced from $A_\bullet$. 
    In particular,
    $$
    H_i(\tau_{\ge n} A_\bullet) = 
    \begin{cases}
        H_i(A_\bullet), & i \ge n, \\
        0, & i < n,
    \end{cases}
    \quad\text{and}\quad
    H_i(\tau_{\le n} A_\bullet) = 
    \begin{cases}
        H_i(A_\bullet), & i \le n, \\
        0, & i > n.
    \end{cases}
    $$
    The assignments $A_\bullet \mapsto \tau_{\ge n} A_\bullet$ and $A_\bullet \mapsto \tau_{\le n} A_\bullet$ extend to endofunctors

    \hlalign{
    \begin{align*}
    \tau_{\ge n},\, \tau_{\le n} : \mathbf{Ch}(\mathcal{A}) \to \mathbf{Ch}(\mathcal{A}),
    \end{align*}
    }

    called the \hldef{truncation functors}. They are natural in both $A_\bullet$ and $n$, and fit into canonical morphisms of complexes
    $$
    \tau_{\ge n} A_\bullet \longrightarrow A_\bullet \longrightarrow \tau_{\le n} A_\bullet.
    $$

    \item 
    \noindent
    Similarly, let $A^\bullet$ be a 
    \CrefAndHyperrefIfExist{definition:cochain_complex_of_objects_in_an_additive_category}{cochain complex} 
    in $\mathbf{Ch}(\mathcal{A})$, i.e.
    $$
    \cdots \xrightarrow{d^{n-2}} A^{n-1} \xrightarrow{d^{n-1}} A^n \xrightarrow{d^n} A^{n+1} \xrightarrow{d^{n+1}} \cdots
    $$
    with $d^{n+1} \circ d^n = 0$ for all $n \in \mathbb{Z}$. 
    The \hldef{canonical truncations of $A^\bullet$} are defined by

    \hlalign{
    \begin{align*}
    (\tau_{\le n} A)^i &=
    \begin{cases}
        A^i, & i < n, \\
        \ker(d^n : A^n \to A^{n+1}), & i = n, \\
        0, & i > n,
    \end{cases}
    &
    (\tau_{\ge n} A)^i &=
    \begin{cases}
        0, & i < n, \\
        \mathrm{coker}(d^{n-1} : A^{n-1} \to A^n), & i = n, \\
        A^i, & i > n.
    \end{cases}
    \end{align*}
    }

    The differentials are the restrictions or quotient maps induced by those of $A^\bullet$. 
    These truncations satisfy
    $$
    H^i(\tau_{\le n} A^\bullet) = 
    \begin{cases}
        H^i(A^\bullet), & i \le n, \\
        0, & i > n,
    \end{cases}
    \quad\text{and}\quad
    H^i(\tau_{\ge n} A^\bullet) = 
    \begin{cases}
        0, & i < n, \\
        H^i(A^\bullet), & i \ge n.
    \end{cases}
    $$

    They also extend to endofunctors

    \hlalign{
    \begin{align*}
    \tau_{\le n},\, \tau_{\ge n} : \mathbf{Ch}(\mathcal{A}) \to \mathbf{Ch}(\mathcal{A}),
    \end{align*}
    }

    natural in both $A^\bullet$ and $n$, fitting into canonical morphisms of cochain complexes
    $$
    \tau_{\le n} A^\bullet \longrightarrow A^\bullet \longrightarrow \tau_{\ge n} A^\bullet.
    $$

    \end{enumerate}

%     Let $\calA$ be an \CrefAndHyperrefIfExist{definition:abelian_category}{abelian category}.
% The \hldef{canonical truncations} of a \CrefAndHyperrefIfExist{definition:chain_complex_of_objects_in_an_additive_category}{chain complex} $A_\bullet$ in \CrefAndHyperrefIfExist{definition:chain_complex_of_objects_in_an_additive_category}{$\mathbf{Ch}(\mathcal{A})$} are defined by

% \hlalign{
% \begin{align*}
% (\tau_{\ge n} A)_i &=
%   \begin{cases}
%     A_i, & i > n, \\
%     \ker(d_n : A_n \to A_{n-1}), & i = n, \\
%     0, & i < n,
%   \end{cases}
% &
% (\tau_{\le n} A)_i &=
%   \begin{cases}
%     0, & i > n, \\
%     \mathrm{coker}(d_{n+1} : A_{n+1} \to A_n), & i = n, \\
%     A_i, & i < n.
%   \end{cases}
% \end{align*}
% }

% The differentials are the restrictions and/or quotient maps induced from $A_\bullet$. 
% In particular,
% $$
% H_i(\tau_{\ge n} A_\bullet) = 
%   \begin{cases}
%     H_i(A_\bullet), & i \ge n, \\
%     0, & i < n,
%   \end{cases}
% \quad\text{and}\quad
% H_i(\tau_{\le n} A_\bullet) = 
%   \begin{cases}
%     H_i(A_\bullet), & i \le n, \\
%     0, & i > n.
%   \end{cases}
% $$

% The assignments $A_\bullet \mapsto \tau_{\ge n} A_\bullet$ and $A_\bullet \mapsto \tau_{\le n} A_\bullet$ extend to endofunctors

% \hlalign{
% \begin{align*}
% \tau_{\ge n},\, \tau_{\le n} : \mathbf{Ch}(\mathcal{A}) \to \mathbf{Ch}(\mathcal{A}),
% \end{align*}
% }

% called the \hldef{truncation functors}. They are natural in both $A_\bullet$ and $n$, and fit into canonical morphisms of complexes
% $$
% \tau_{\ge n} A_\bullet \longrightarrow A_\bullet \longrightarrow \tau_{\le n} A_\bullet.
% $$
\end{definition}

\begin{definition} \label{definition:left_right_resolution_of_a_class_of_objects_in_an_abelian_category}
Let $\mathcal{A}$ be an \CrefAndHyperrefIfExist{definition:abelian_category}{abelian category} and let $\mathcal{X}$ be a class of objects in $\mathcal{A}$. Let $M$ be an object of $\calA$.

\begin{enumerate}
    \item A \hldef{right resolution of $M$} is a \CrefAndHyperrefIfExist{definition:chain_complex_of_objects_in_an_additive_category}{cochain complex} $I^\bullet$ with $I^i = 0$ for $i < 0$ and a map $M \to I^0$ such that the augmented complex
    $$0 \to M \to I^0 \to I^1 \to I^2 \to \cdots$$
    is \CrefAndHyperrefIfExist{definition:acyclic_complex_of_objects_in_an_abelian_category}{exact}.

    % \item An \hldef{injective resolution of $M$} is a right resolution $I^\bullet$ for which the objects $I^i$ are all \CrefAndHyperrefIfExist{definition:injective_and_projective_objects_in_a_category}{injective}.

    \item A \hldef{left resolution of $M$} is a \CrefAndHyperrefIfExist{definition:chain_complex_of_objects_in_an_additive_category}{chain complex} $P_\bullet$ with $P_i = 0$ for $i < 0$ and a map $P_0 \to M$ such that the augmented complex
    $$\cdots P_2 \to P_1 \to P_0 \to M \to 0$$
    is \CrefAndHyperrefIfExist{definition:acyclic_complex_of_objects_in_an_abelian_category}{exact}.
    
    \item An \hldef{$\mathcal{X}$-left resolution} of an object $M \in \mathcal{A}$ a \CrefAndHyperrefIfExist{definition:left_right_resolution_of_a_class_of_objects_in_an_abelian_category}{left resolution} by objects of $X$, i.e. an exact complex
    $$ \cdots \to X_2 \to X_1 \to X_0 \to M \to 0 $$
    with each $X_i \in \mathcal{X}$.

    \item An \hldef{$\mathcal{X}$-right resolution} of an object $M \in \mathcal{A}$ a \CrefAndHyperrefIfExist{definition:left_right_resolution_of_a_class_of_objects_in_an_abelian_category}{right resolution} by objects of $X$, i.e. an exact complex
    $$ 0 \to M \to X^0 \to X^1 \to X^2 \to \cdots $$
    with each $X_i \in \mathcal{X}$.

    \item A \hldef{projective resolution of $M$} is a left resolution $P^\bullet$ for which the objects $P^i$ are all \CrefAndHyperrefIfExist{definition:injective_and_projective_objects_in_a_category}{projective}.

    \item An \hldef{injective resolution of $M$} is a right resolution $I^\bullet$ for which the objects $I^i$ are all \CrefAndHyperrefIfExist{definition:injective_and_projective_objects_in_a_category}{injective}.
\end{enumerate}

\end{definition}

\subsection{Schemes}


\begin{definition}[Scheme] \label{definition:scheme}
    A \hldef{scheme} is a \CrefAndHyperrefIfExist{definition:locally_ringed_space_on_a_topological_space}{locally ringed space} $(X, \mathcal{O}_X)$ that admits an open cover $\{U_i\}_{i \in I}$ such that each $(U_i, \mathcal{O}_X|_{U_i})$ is \CrefAndHyperrefIfExist{definition:morphism_of_locally_ringed_spaces}{isomorphic (as a locally ringed space)} to an \CrefAndHyperrefIfExist{definition:affine_scheme}{affine scheme $(\mathrm{Spec}(A_i), \mathcal{O}_{\mathrm{Spec}(A_i)})$} for some \CrefAndHyperrefIfExist{ring}{commutative ring} $A_i$.  
    In other words, a scheme is a locally ringed space locally isomorphic to affine schemes.

    
\end{definition}


\begin{definition}[Morphism of schemes] \label{definition:morphism_of_schemes}
    Let $(X, \mathcal{O}_X)$ and $(Y, \mathcal{O}_Y)$ be \CrefAndHyperrefIfExist{definition:scheme}{schemes}.  
    A \hldef{morphism of schemes} is a \CrefAndHyperrefIfExist{definition:morphism_of_locally_ringed_spaces}{morphism as locally ringed spaces}.

    In particular, there is a \CrefAndHyperrefIfExist{definition:category}{category}, often denoted by \hl{$\mathrm{Sch}$}, \hl{$\mathbf{Sch}$} etc., whose objects are schemes and whose morphisms are morphisms of schemes.
    
\end{definition}


\begin{definition}[Scheme over a scheme] \label{definition:scheme_over_a_scheme}
    Let $(S, \mathcal{O}_S)$ be a scheme. A \hldef{scheme over $S$} (or an \hldef{$S$-scheme}) is a scheme $(X, \mathcal{O}_X)$ together with a morphism of schemes
    $$\pi: (X, \mathcal{O}_X) \to (S, \mathcal{O}_S).$$
    This morphism $\pi$ is called the \hldef{structure morphism of the scheme $X$ over $S$}.  

    If $S = \mathrm{Spec}(R)$ is an affine scheme for a commutative ring $R$, then an $S$-scheme is synonymously called an \hldef{$R$-scheme} or a \hldef{scheme over $R$}. 

    Let $(S, \mathcal{O}_S)$ be a scheme, and let $(X, \mathcal{O}_X)$ and $(Y, \mathcal{O}_Y)$ be schemes over $S$ with structure morphisms
    $$\pi_X: X \to S, \quad \pi_Y: Y \to S.$$
    A \hldef{morphism of $S$-schemes} (or synonymously a \hldef{$S$-scheme morphism}) is a \CrefAndHyperrefIfExist{definition:morphism_of_schemes}{morphism of schemes}
    $$(f, f^\#): (X, \mathcal{O}_X) \to (Y, \mathcal{O}_Y)$$
    such that the following diagram commutes:
    $$
    \begin{array}{ccc}
    X & \xrightarrow{f} & Y \\
    {\scriptstyle \pi_X} \downarrow & & \downarrow {\scriptstyle \pi_Y} \\
    S & = & S
    \end{array}
    $$
    In other words,
    $$\pi_Y \circ f = \pi_X.$$

    Given a fixed scheme $S$, there is a category, often denoted by \hl{$\mathrm{Sch}_S$}, \hl{$\mathrm{Sch}_{/S}$}, \hl{$\mathrm{Sch}/S$}, \hl{$\mathbf{Sch}_S$}, \hl{$\mathbf{Sch}_{/S}$}, \hl{$\mathbf{Sch}/S$} etc. whose objects are schemes $T$ over $S$ and whose morphisms $T_1 \to T_2$ are morphisms of schemes over $S$. If $S = \Spec R$ for some commutative ring $R$, then we may instead write \hl{$\mathrm{Sch}_R$} to denote $\mathrm{Sch}_{\Spec R}$, etc. It is noteworthy that $\mathrm{Sch}_\bbZ$ coincides with the category \CrefAndHyperrefIfExist{definition:morphism_of_schemes}{$\mathrm{Sch}$} of all schemes. In other words, a $\bbZ$-scheme can be identified simply with a scheme. 

    \TextIfExists{definition:category_of_objects_over_under_a_fixed_object_in_a_category}{Equivalently, the category $\mathrm{Sch}_{/S}$ is the category of schemes over $S$ in the sense of \Cref{definition:category_of_objects_over_under_a_fixed_object_in_a_category}.}

\end{definition}


\begin{definition} \label{definition:affine_morphism_of_schemes}
Let $f : X \to Y$ be a \CrefAndHyperrefIfExist{definition:morphism_of_schemes}{morphism of schemes}. We say that $f$ is an \hldef{affine morphism} if for every \CrefAndHyperrefIfExist{definition:affine_open_subscheme_of_a_scheme}{affine open} $V = \operatorname{Spec} B \subseteq Y$, the \CrefAndHyperrefIfExist{definition:preimage_of_an_open_subset_of_a_scheme_under_a_morphism_of_schemes}{preimage $U = f^{-1}(V)$} is an \CrefAndHyperrefIfExist{definition:affine_scheme}{affine scheme}.
\end{definition}


\begin{definition} \label{definition:finite_type_morphism_of_schemes}
Let $f : X \to Y$ be a \CrefAndHyperrefIfExist{definition:morphism_of_schemes}{morphism of schemes}. We say that $f$ is a \hldef{finite type morphism} if for every \CrefAndHyperrefIfExist{definition:affine_open_subscheme_of_a_scheme}{affine open} $V = \operatorname{Spec} B \subseteq Y$ with $U = f^{-1}(V)$ affine, say $U = \operatorname{Spec} A$, the ring $A$ is a \CrefAndHyperrefIfExist{definition:finitely_generated_algebra_over_a_not_necessarily_commutative_ring}{finitely generated $B$-algebra}.

When $X$ is equipped with a finite type morphism $f: X \to Y$, we say that $X$ is a \hldef{finite type scheme over $Y$} or a \hldef{finite type $Y$-scheme} or a \hldef{$Y$-scheme of finite type} \CrefIfExists{definition:scheme_over_a_scheme}, etc.
\end{definition}

\begin{definition}[Finitely presented algebra over a ring] \label{definition:finitely_presented_algebra_over_a_not_necessarily_commutative_ring}
    Let $R$ be a \CrefAndHyperrefIfExist{definition:ring}{(not necessarily commutative) ring}. An \CrefAndHyperrefIfExist{definition:algebra_of_a_ring}{$R$-algebra} $A$ is said to be \hldef{finitely presented} if there exists an integer $n \geq 0$ and a surjective $R$-algebra homomorphism
    $$\varphi: R\langle x_1, \ldots, x_n \rangle \twoheadrightarrow A$$
    where $R\langle x_1, \ldots, x_n \rangle$ is the \CrefAndHyperrefIfExist{definition:free_associative_algebra_over_a_ring}{free $R$-algebra on $n$ generators}, such that the kernel $\ker(\varphi)$ is a finitely generated \CrefAndHyperrefIfExist{definition:ideal_of_a_ring}{two-sided ideal} of $R\langle x_1, \ldots, x_n \rangle$.

    In other words, $A$ admits a presentation as
    $$A \cong R\langle x_1, \ldots, x_n \rangle / I,$$
    where $I$ is a \CrefAndHyperrefIfExist{definition:ideal_generated_by_a_subset_of_a_ring}{finitely generated} two-sided ideal.

    If $R$ and $A$ are commutative rings, this recovers the usual definition of a finitely presented commutative $R$-algebra by replacing $R\langle x_1, \ldots, x_n \rangle$ with the polynomial ring $R[x_1, \ldots, x_n]$ and $I$ a finitely generated ideal.
\end{definition}

\begin{definition} \label{definition:additive_group_scheme_over_a_scheme}
    \TODO{TODO: define a (commutative) group scheme}
    \TODO{TODO: define the multiplicative group scheme}
Let $S$ be a scheme. The \hldef{additive group scheme over $S$} is the group scheme \hl{$\bbG_{a} = \bbG_{a,S}$} over $S$ defined by
$$ \bbG_{a,S} = \Spec \mathcal{O}_S[T] $$
with group structure morphisms:
\begin{itemize}
  \item \textbf{Comultiplication (Addition):} The morphism $\Delta: \bbG_{a,S} \to \bbG_{a,S} \times_S \bbG_{a,S}$ corresponding to the ring homomorphism $\mathcal{O}_S[T] \to \mathcal{O}_S[T] \otimes_{\mathcal{O}_S} \mathcal{O}_S[T]$ given by $T \mapsto T \otimes 1 + 1 \otimes T$.
  \item \textbf{Counit (Identity):} The morphism $\varepsilon: \bbG_{a,S} \to S$ corresponding to $T \mapsto 0$.
  \item \textbf{Coinverse (Inversion):} The morphism $\iota: \bbG_{a,S} \to \bbG_{a,S}$ corresponding to $T \mapsto -T$.
\end{itemize}
Thus, $(\bbG_{a,S}, \Delta, \varepsilon, \iota)$ is a commutative group scheme over $S$. Note that we may speak of the \hldef{additive group scheme over a ring $R$} as the additive group scheme over $\Spec R$. 
\end{definition}


\begin{definition}[Dimension of a Scheme] \label{definition:dimension_of_a_scheme}
Let $X$ be a scheme with underlying topological space $|X|$.

    \TODO{krull dimension}
\begin{itemize}
    \item The \hldef{dimension at a point $x \in |X|$}, denoted $\dim_x(X)$, is the Krull dimension of the \CrefAndHyperrefIfExist{definition:locally_ringed_space_on_a_topological_space}{local ring} \CrefAndHyperrefIfExist{definition:stalk_of_a_presheaf_on_a_topological_space_at_a_point}{$\mathcal{O}_{X,x}$}. This is the supremum of the lengths $n$ of chains of prime ideals
    $$ \mathfrak{p}_0 \subsetneq \mathfrak{p}_1 \subsetneq \cdots \subsetneq \mathfrak{p}_n \subseteq \mathcal{O}_{X,x}.  $$

    \item The \hldef{dimension of the scheme $X$} is defined as
    $$ \hlin{\dim(X) := \sup_{x \in |X|} \dim_x(X). }$$
    Equivalently, it is the supremum of the lengths of chains of distinct irreducible closed subsets of $|X|$ ordered by inclusion.
\end{itemize}
\end{definition}


\section{Presheaves and sheaves}


\begin{definition}[{\cite[Expos\'e I D\'efinition 4.1]{SGA4_I}}] \label{definition:sieve_on_an_object_in_a_category}
Let $C$ be a \CrefAndHyperrefIfExist{definition:category}{(large) category}. 

\begin{enumerate}
    \item A \hldef{sieve $S$ on the category $C$} is a \CrefAndHyperrefIfExist{definition:full_subcategory_of_a_category}{full subcategory} $D$ of $C$ such that for any object $U$ of $C$ there exists an object $V$ of \TODO{correctly parse the definiton}
    \item A \hldef{sieve $S$ on an object $U \in \operatorname{Ob}(C)$} is a collection of morphisms in $C$ with codomain $U$ that is closed under precomposition by any compatible morphism in $C$. In other words, $S$ is a sieve if for every $f : V \to U$ in $S$ and morphism $g : W \to V$ in $C$, the composition $f \circ g : W \to U$ is also in $S$. 

    Given a morphism $f: V \to U$ in a sieve $S$, we also say that \hldef{$f$ factors through $U$}.
\end{enumerate}
\end{definition}

\begin{definition} \label{definition:pullback_sieve_of_an_object_in_a_category_via_a_morphism_to_the_object}
Let $C$ be a category, let $U \in \operatorname{Ob}(C)$, and let $S$ be a \CrefAndHyperrefIfExist{definition:sieve_on_an_object_in_a_category}{sieve on $U$}.
For a morphism $f : V \to U$ in $C$, the \hldef{pullback sieve} \hl{$f^*S$} (or \hldef{basechange sieve} \hl{$S \times_U V$}) on $V$ is defined by
\[
f^*S = \{ g : W \to V \mid f \circ g \in S \}.
\]
In other words, $f^*S$ consists of all morphisms into $V$ whose composite with $f$ belongs to the sieve $S$ on $U$.
\end{definition}



% \begin{definition}[Generated sieve] \label{definition:sieve_on_an_object_of_a_category_generated_by_a_family_of_morphisms}
%     Let $\mathcal{C}$ be a (large) category, $X \in \mathcal{C}$ an object, and $S$ a \CrefAndHyperrefIfExist{definition:sieve_on_an_object_in_a_category}{sieve} on $X$. A sieve $S$ is said to be \hldef{generated} by a family of morphisms $\{f_i : U_i \to X\}_{i \in I}$ if $S$ is the smallest sieve on $X$ containing all the morphisms $f_i$, i.e., $S$ consists precisely of all morphisms $g : Y \to X$ such that $g$ factors through some $f_i$.
% \end{definition}

\begin{definition} \label{definition:sieve_on_an_object_of_a_category_generated_by_a_family_of_morphisms}
Let $\mathcal{C}$ be a \CrefAndHyperrefIfExist{definition:category}{category} and $U \in \mathcal{C}$ an object. Let $\mathcal{S} = \{f_i: U_i \to U\}_{i \in I}$ be a family of morphisms with codomain $U$. 

The \hldef{sieve generated by $\mathcal{S}$}, denoted \hl{$(\mathcal{S})$} or \hl{$\langle \mathcal{S} \rangle$}, is the smallest \CrefAndHyperrefIfExist{definition:sieve_on_an_object_in_a_category}{sieve on $U$} containing all the morphisms in $\mathcal{S}$.

Explicitly, a morphism $h: V \to U$ belongs to the generated sieve if and only if $h$ factors through some morphism in $\mathcal{S}$. That is, there exists an index $i \in I$ and a morphism $g: V \to U_i$ such that
$$ h = f_i \circ g. $$
\end{definition}


% \begin{definition}[Grothendieck topology] \label{definition:grothendieck_topology_on_a_category_site_covering_sieve_topologically_generating_family}
%     Let $\mathscr{U}$ be a \hyperrefIfExists{definition:grothendieck_universe}{universe}\CrefIfExists{definition:grothendieck_universe} and let $\calC$ be a \hyperrefIfExists{definition:locally_small_category}{locally small category}\CrefIfExists{definition:locally_small_category}.

%     \begin{enumerate}
%         \item \textbf{(Grothendieck Topology via Sieves)}
%         A \hldef{Grothendieck topology} $J$ on $\calC$ is an assignment to each object $U \in \calC$ of a collection $J(U)$ of \CrefAndHyperrefIfExist{definition:sieve_on_an_object_in_a_category}{sieves} on $U$, called \hldef{covering sieves}, satisfying:
%         \begin{enumerate}
%             \item (Maximality) The maximal \CrefAndHyperrefIfExist{definition:sieve_on_an_object_in_a_category}{sieve} $\{ f : V \to U \mid V \in \calC \}$ is in $J(U)$.
%             \item (Stability) If $S \in J(U)$ and $f : V \to U$ is any morphism, then the \CrefAndHyperrefIfExist{definition:pullback_sieve_of_an_object_in_a_category_via_a_morphism_to_the_object}{pullback sieve} $f^{*}S$ is in $J(V)$.
%             \item (Transitivity/Local Character) If $S$ is a sieve on $U$ and there exists a covering sieve $R \in J(U)$ such that for every morphism $f : V \to U$ in $R$, the pullback sieve $f^{*}S$ is in $J(V)$, then $S \in J(U)$.
%         \end{enumerate}

%         % \item \textbf{(Grothendieck Pretopology / Basis)}
%         % If $\calC$ admits fiber products, one can define a topology via \hldef{covering families}. A \hldef{Grothendieck pretopology} (or basis) is a collection $K(U)$ of families $\{U_i \to U\}_{i \in I}$ for each object $U$, satisfying:
%         % \begin{itemize}
%         %     \item (Isomorphism) $\{U' \xrightarrow{\sim} U\} \in K(U)$ for any isomorphism.
%         %     \item (Stability) If $\{U_i \to U\} \in K(U)$ and $V \to U$ is a morphism, then $\{U_i \times_U V \to V\} \in K(V)$.
%         %     \item (Composition) If $\{U_i \to U\} \in K(U)$ and for each $i$, $\{V_{ij} \to U_i\} \in K(U_i)$, then the composite family $\{V_{ij} \to U\} \in K(U)$.
%         % \end{itemize}
%         % Every pretopology generates a unique Grothendieck topology $J$, where $S \in J(U)$ iff $S$ contains a covering family from the pretopology.

%         \item A \hldef{site} is a pair $(\calC, J)$ consisting of a category $\calC$ and a Grothendieck topology $J$.

%         \item A family of objects $\mathcal{G} = \{G_\alpha\}$ in a site $(\calC, J)$ is called a \hldef{topologically generating family} if for every object $X \in \calC$, there exists a covering sieve $S \in J(X)$ \CrefAndHyperrefIfExist{definition:sieve_on_an_object_of_a_category_generated_by_a_family_of_morphisms}{generated by} morphisms with domains in $\mathcal{G}$. Equivalently, every object $X$ admits a cover $\{U_i \to X\}$ where each $U_i \in \mathcal{G}$.

%         \item A \hldef{$\mathscr{U}$-site} is a site whose underlying category is $\mathscr{U}$-locally small and which admits a $\mathscr{U}$-small topologically generating family.
%     \end{enumerate}
% \end{definition}

\begin{definition}[Grothendieck topology] \label{definition:grothendieck_topology_on_a_category_site_covering_sieve_topologically_generating_family}
    Let $\mathscr{U}$ be a \hyperrefIfExists{definition:grothendieck_universe}{universe}\CrefIfExists{definition:grothendieck_universe}.
    \begin{enumerate}
        % \item Let $C$ be a \hyperrefIfExists{definition:locally_small_category}{locally small category}\CrefIfExists{definition:locally_small_category}. A \hldef{Grothendieck topology on $C$} assigns to each object $U$ of $C$ a collection of families of morphisms $\{U_i \to U\}_{i \in I}$, called \hldef{coverings of $U$}, satisfying:
        % \begin{itemize}
        %     \item (Isomorphism) If $f: V \to U$ is an isomorphism in $C$, then $\{f: V \to U\}$ is a covering of $U$.
        %     \item (Stability under base change) If $\{U_i \to U\}_{i \in I}$ is a covering of $U$ and $V \to U$ is any morphism, then the family $\{ U_i \times_U V \to V \}_{i \in I}$ is a covering of $V$.
        %     \item (Transitivity) If $\{U_i \to U\}_{i \in I}$ is a covering of $U$ and for each $i$, $\{V_{ij} \to U_i\}_{j \in J_i}$ is a covering of $U_i$, then the family $\{ V_{ij} \to U \}_{i \in I,\, j \in J_i}$ is a covering of $U$.
        % \end{itemize}

        \item (See \cite[Expos\'e II, D\'efinition 1.1]{SGA4_I}) Let $\calC$ be a \CrefAndHyperrefIfExist{definition:category}{category}. A \hldef{Grothendieck topology on $\calC$} assigns to each object $U$ of $\calC$ a collection \hl{$J(U)$} of \CrefAndHyperrefIfExist{definition:sieve_on_an_object_in_a_category}{sieves} $\{U_i \to U\}_{i \in I}$, each called a \hldef{covering sieve of $U$}, satisfying:
        \begin{enumerate}
            \item (Stability under ``base change''): If $S \in J(U)$ is a covering sieve of an object $U$, and $f: V \to U$ is any morphism in $\calC$, then the \CrefAndHyperrefIfExist{definition:pullback_sieve_of_an_object_in_a_category_via_a_morphism_to_the_object}{pullback sieve} $f^* S$ is a covering sieve of $U$.
            % \item (Local character condition) If $F$ is a sieve on $U$ such that the sieve $\bigcup_...$ \TODO{}
            \item (Local character condition) If $S$ is a sieve on $U$, and if there exists a covering sieve $R \in J(U)$ such that for all $f: V \to U$ in $R$ the \CrefAndHyperrefIfExist{definition:pullback_sieve_of_an_object_in_a_category_via_a_morphism_to_the_object}{pullback sieve} $f^* S$ is in $J(V)$, then $S \in J(U)$. 
            
            \item The \CrefAndHyperrefIfExist{definition:maximal_sieve_on_an_object_in_a_category}{maximal sieve} is a covering sieve.
        \end{enumerate}


        % Equivalently, a Grothendieck topology $J$ on a category $C$ is an assignment of a collection $J(U)$ of \CrefAndHyperrefIfExist{definition:sieve_on_an_object_in_a_category}{sieves} on each object $U \in \operatorname{Ob}(C)$ such that:
        % \begin{enumerate}
        %     \item the maximal \CrefAndHyperrefIfExist{definition:sieve_on_an_object_in_a_category}{sieve} $\{ f : V \to U \mid f \in \operatorname{Mor}(C) \}$ belongs to $J(U)$,
        %     \item if $S \in J(U)$ and $f : V \to U$, then the \CrefAndHyperrefIfExist{definition:pullback_sieve_of_an_object_in_a_category_via_a_morphism_to_the_object}{pullback sieve $f^{*}S$} on $V$ belongs to $J(V)$,
        %     \item if $S$ is a sieve on $U$, and if there exists $R \in J(U)$ such that for all $f : V \to U$ in $R$ the \CrefAndHyperrefIfExist{definition:pullback_sieve_of_an_object_in_a_category_via_a_morphism_to_the_object}{pullback sieve $f^{*}S$} is in $J(V)$, then $S \in J(U)$.
        % \end{enumerate}

        Some will refer to a Grothendieck topology as simply a \hldef{topology}, not to be confused with the related, but less general, notion of a \CrefAndHyperrefIfExist{definition:topological_space}{topology on a set}.


        \item (See \cite[Expos\'e II, 1.1.5]{SGA4_I}) A \hldef{site} is a category $\calC$ equipped with a Grothendieck topology.

        When we are working with a \CrefAndHyperref{definition:basis_and_grothendieck_pretopology_for_a_grothendieck_topology_on_a_category}{Grothendieck pretopology} $K$ on a category $\calC$, we may regard $\calC$ as a site by equipping it with the \CrefAndHyperref{definition:grothendieck_topology_generated_by_a_pretopology}{Grothendieck topology generated by} $K$. 

        \item (See \cite[Expos\'e II, D\'efinition 1.2]{SGA4_I}) Let $(\calC, J)$ be a site. A family of morphisms $(U_i \to U)_{i \in I}$ is called a \hldef{covering family of $U$ (with respect to the site/topology)} or a \hldef{cover of $U$ (with respect to the site/topology)} if the \CrefAndHyperrefIfExist{definition:sieve_on_an_object_of_a_category_generated_by_a_family_of_morphisms}{sieve generated by} the family is a covering sieve of $U$. 

        \item (See \cite[Expos\'e II, D\'efinition 3.0.1]{SGA4_I}) Let $(\calC, J)$ be a \CrefAndHyperrefIfExist{definition:grothendieck_topology_on_a_category_site_covering_sieve_topologically_generating_family}{site}, where $J$ is a Grothendieck topology on $\calC$.

        A family $G$ of objects $\calC$ is called a \hldef{topologically generating family of the site/topology} or a \hldef{generating family/collection of the site/topology} if for every object $X \in \calC$, there is a covering family $\{X_\alpha \to X\}_{\alpha \in A}$ of $X$ such that every $X_\alpha$ is a member of $G$.  

        Equivalently, the Grothendieck topology $J$ is the smallest Grothendieck topology containing all covers of the $U_i$. Also equivalently, for any $S \in J(X)$, the sieve $S$ contains a covering family $\{V_i \to X\}$ such that each morphism $V_i \to X$ factors through some member of $G$. \TODO{Verify that these claimed equivalences are indeed equivalences}
        
        % A family of objects $\{U_i\}_{i \in I}$ in $\calC$ is called a \hldef{topologically generating family} if for every object $X \in \calC$ and every covering sieve $S \in J(X)$, the sieve $S$ is \CrefAndHyperrefIfExist{definition:sieve_on_an_object_of_a_category_generated_by_a_family_of_morphisms}{generated by} pullbacks of covering families from the family $\{U_i\}$.

        % More precisely, this means that for any $S \in J(X)$, the sieve $S$ contains a covering family $\{V_j \to X\}$ such that each morphism $V_j \to X$ factors through some $U_i$, and the covering families of the $U_i$ generate the topology $J$. 
        % Equivalently, the Grothendieck topology $J$ is the smallest Grothendieck topology containing all coverings of the $U_i$.

        % When one speaks of a \hldef{generating family/collection} of a site, one usually refers to the above notion of a topologically generating family.

        \item (See \cite[Expos\'e II, D\'efinition 3.0.2]{SGA4_I}) A \hldef {$\mathscr{U}$-site} is a site whose underlying category $\calC$ is \hyperrefIfExists{definition:locally_small_category}{$\mathscr{U}$-locally small}\CrefIfExists{definition:locally_small_category} and which has a $\mathscr{U}$-small topologically generating family. A $\mathscr{U}$-site is called \hldef{$\mathscr{U}$-small} if its underlying category is $\mathscr{U}$-small. Similarly, a \hldef{small site} is a site whose underlying category is a set and a \hldef{locally small site} is a site whose underlying category is \CrefAndHyperrefIfExist{definition:locally_small_category}{locally small}.
    \end{enumerate}
\end{definition}




\begin{definition}[Category of objects over a fixed object] \label{definition:category_of_objects_over_under_a_fixed_object_in_a_category}
Let $\mathcal{C}$ be a \hyperrefIfExists{definition:category}{category}\CrefIfExists{definition:category} and let $X \in \operatorname{Ob}(\mathcal{C})$ be a fixed object.
\begin{enumerate}
    \item 
        The \hldef{category of objects over $X$} (or synonymously the \hldef{slice category of $X$ in $\calC$} or the \hldef{over category of $X$ in $\calC$}), commonly denoted \hl{$\mathcal{C}/X$}, \hl{$\mathcal{C}_{/X}$}, or \hl{$(\mathcal{C} \downarrow X)$} is the category defined as follows:
        \begin{itemize}
            \item An object of $\mathcal{C}/X$ is a morphism $f \colon A \to X$ in $\mathcal{C}$, where $A \in \operatorname{Ob}(\mathcal{C})$.
            \item A morphism from $f \colon A \to X$ to $g \colon B \to X$ in $\mathcal{C}/X$ is a morphism $h \colon A \to B$ in $\mathcal{C}$ such that the following diagram commutes:
            $$
            \begin{aligned}
            \xymatrix{
            A \ar[dr]_f \ar[r]^h & B \ar[d]^g \\
            & X
            }
            \end{aligned}
            $$
            i.e. such that $g \circ h = f$.
            \item The identity morphisms and composition in $\mathcal{C}/X$ are inherited from $\mathcal{C}$.
        \end{itemize}

    \item 
    The \hldef{category of objects under $X$} (or synonymously the \hldef{coslice category of $X$ in $\calC$} or the \hldef{under category of $X$ in $\calC$}), commonly denoted \hl{$X/\mathcal{C}$}, \hl{$X \backslash \calC$}, \hl{$\mathcal{C}_{X/}$}, or \hl{$(X \downarrow \calC)$}, is the category defined as follows:
    \begin{itemize}
        \item An object of $X/\mathcal{C}$ is a morphism $f \colon X \to A$ in $\mathcal{C}$, where $A \in \operatorname{Ob}(\mathcal{C})$.
        \item A morphism from $f \colon X \to A$ to $g \colon X \to B$ in $X/\mathcal{C}$ is a morphism $h \colon A \to B$ in $\mathcal{C}$ such that the following diagram commutes:
        $$
        \begin{aligned}
        \xymatrix{
        X \ar[dr]^g \ar[r]^f & A \ar[d]^h \\
        & B
        }
        \end{aligned}
        $$
        i.e. such that $h \circ f = g$.
        \item The identity morphisms and composition in $X/\mathcal{C}$ are inherited from $\mathcal{C}$.
    \end{itemize}

\end{enumerate} 
\TextIfExists{definition:comma_category_of_two_functors_to_a_category}{Both notions are special cases of \CrefAndHyperrefIfExist{definition:comma_category_of_two_functors_to_a_category}{comma categories}.}
\end{definition}


\begin{lemma} \label{lemma:slice_category_has_final_object}
    Let $\mathcal{C}$ be a \hyperrefIfExists{definition:category}{category}\CrefIfExists{definition:category} and let $X \in \operatorname{Ob}(\mathcal{C})$ be a fixed object. The \CrefAndHyperrefIfExist{definition:category_of_objects_over_under_a_fixed_object_in_a_category}{slice category $\calC / X$} has $X$ as its \CrefAndHyperrefIfExist{definition:initial_final_zero_objects_of_a_category}{final object}.
\end{lemma}
\begin{proof}
    This is clear.
\end{proof}

\begin{definition}[Slice site] \label{definition:site_induced_by_a_site_on_an_over_category}
Let $(\mathcal{C}, \tau)$ be a \CrefAndHyperrefIfExist{definition:grothendieck_topology_on_a_category_site_covering_sieve_topologically_generating_family}{site}, where $\tau$ is a Grothendieck topology on the (\CrefAndHyperrefIfExist{definition:locally_small_category}{locally small or $U$-locally small}, if a \CrefAndHyperrefIfExist{definition:grothendieck_universe}{universe} $U$ is available) category $\mathcal{C}$. For a fixed object $X$ in $\mathcal{C}$, the \hldef{slice site} (or the \hldef{over site}, the \hldef{site on the slice category $\mathcal{C}_{/X}$}, the \hldef{site induced on the over category $\mathcal{C}_{/X}$}, the \hldef{localization of the site $\calC$ at the object $X$}, etc.) $(\mathcal{C}_{/X}, \tau_{/X})$ is the site whose underlying category is the \CrefAndHyperrefIfExist{definition:category_of_objects_over_under_a_fixed_object_in_a_category}{slice category $\mathcal{C}_{/X}$}, and whose Grothendieck topology \hl{$\tau_{/X}$} (also denoted by notations such as \hl{$\tau|_{X}$} or \hl{$\tau/X$}) is defined by declaring a family of morphisms $\{f_i : Y_i \to Y\}$ in $\mathcal{C}_{/X}$ to be a covering if and only if the family $\{f_i : Y_i \to Y\}$ is a covering in $(\mathcal{C}, \tau)$.

% The forgetful functor 
% $$\hlin{j_X: \calC/X \to \calC}$$
% is \CrefAndHyperrefIfExist{definition:continuous_cocontinuous_functor_between_categories}{cocontinuous and continuous} 

\end{definition}




\section{Derived categories}


\import{../_excerpts}{excerpts_derived_categories.tex}

\begin{proposition} \label{proposition:examples_of_abelian_categories}
The following are examples of \CrefAndHyperrefIfExist{definition:abelian_category}{abelian categories}:

\begin{enumerate}
    \item The category of $R$-$S$ bimodules where $R$,$S$ are \CrefAndHyperrefIfExist{definition:ring}{(not necessarily commutative) rings} (\Cref{theorem:the_category_of_R_S_bimodules_is_a_grothendieck_abelian_category_and_AB4_star}).

    \item The category $\mathbf{Ab}$ of abelian groups and group homomorphisms is abelian.

    \item The category $\text{Vect}_k$ of vector spaces over a field $k$ and $k$-linear maps is abelian.

    \item More generally, if $R$ is a \CrefAndHyperrefIfExist{definition:noetherian_ring}{noetherian ring}, then the category of \CrefAndHyperrefIfExist{definition:finitely_generated_modules_over_rings}{finitely generated} $R$-modules is abelian.

    \item For a \CrefAndHyperrefIfExist{definition:ringed_space}{ringed space} $(X, \mathcal{O}_X)$, the category of \CrefAndHyperrefIfExist{definition:module_over_a_sheaf_of_rings_on_a_site}{$\mathcal{O}_X$-modules} is abelian.
    \TODO{a quasi-coherent sheaf on a locally ringed space}
    \item If $X$ is a \CrefAndHyperrefIfExist{definition:scheme}{scheme} (or more generally a \CrefAndHyperrefIfExist{definition:locally_ringed_space_on_a_topological_space}{locally ringed space}), the category of \CrefAndHyperrefIfExist{definition:quasi_coherent_sheaf_on_a_general_scheme}{quasi-coherent sheaves on $X$} is abelian.
    \item For any \CrefAndHyperrefIfExist{definition:essentially_small_category}{essentially small category} $\mathcal{C}$ and any abelian category $\mathcal{A}$, the \CrefAndHyperrefIfExist{definition:diagram_in_a_category_indexed_by_a_small_category}{functor category $[\mathcal{C}, \mathcal{A}]$} and the category $\PreShv(\calC, \calA)$ of \CrefAndHyperrefIfExist{definition:presheaf_on_a_category}{presheaves} are abelian.
    \TODO{apparently, the essentially smallness condition is removable, provided that the sheafification functor exists. However, the essentially small assumption is needed to show that the category of sheaves of $O$-modules is a Grothendieck abelian caetgory. Verify all this. Moreover, when working with a big site of a scheme, one typically fixes a unvierse or work relative to a cardinal cutoff to treat it as essentially small}

    \item For any \CrefAndHyperrefIfExist{definition:grothendieck_topology_on_a_category_site_covering_sieve_topologically_generating_family}{site} $(\calC, J)$ on an \CrefAndHyperrefIfExist{definition:essentially_small_category}{essentially small category} $\mathcal{C}$ and any abelian category $\mathcal{A}$, the category $\Shv(\calC, \calA)$ of \CrefAndHyperrefIfExist{definition:sheaf_on_a_site}{sheaves} is abelian.

    \item For any \CrefAndHyperrefIfExist{definition:grothendieck_topology_on_a_category_site_covering_sieve_topologically_generating_family}{site} $(\calC, J)$ on an \CrefAndHyperrefIfExist{definition:essentially_small_category}{essentially small category} $\mathcal{C}$ and a \CrefAndHyperrefIfExist{definition:sheaf_on_a_site}{sheaf of rings} $\calO$ on $\calC$, the category $\mathbf{Mod}(\mathcal{O})$ of \CrefAndHyperrefIfExist{definition:module_over_a_sheaf_of_rings_on_a_site}{$\calO$-modules} is an abelian category.

\end{enumerate}
\end{proposition}

\begin{notation} \label{notation:notations_for_homotopy_and_derived_categories_of_sheaves_of_modules_on_a_ringed_site}
    Let $((\calC,J), \calO)$ be a \CrefAndHyperrefIfExist{definition:ringed_site}{ringed site} where $(\calC, J)$ is essentially small. 
    \TODO{The essentially smallness hypothesis might not be strictly needed to have that the category of sheaves of O-modules is abelian.}
    
    \begin{enumerate}
        \item The \CrefAndHyperrefIfExist{definition:homotopy_category_of_chain_complexes_of_an_additive_category}{homotopy category} of the category of \CrefAndHyperrefIfExist{definition:module_over_a_sheaf_of_rings_on_a_site}{$\calO$-modules}, which \CrefAndHyperrefIfExist{proposition:examples_of_abelian_categories}{is} abelian, is often denoted by notations such as \hl{$K(\calC, \calO)$} or \hl{$K(\calO)$}. All of the usual superscripts apply --- we may thus speak of \hl{$K^{?,??}(\calC, \calO)$} or \hl{$K^{?,??}(\calO)$} for $? \in \{+,-,b\}$ and $?? \in \{+,-,b,(\text{blank})\}$. 
        
        If $(\calC, J)$ is a site on some ``space'', e.g. a scheme or topological space, $X$, then it is also usual to denote the homotopy category by \hl{$K(X, \calO)$}. We may also speak of \hl{$K^{?,??}(X, \calO)$} for $? \in \{+,-,b\}$ and $?? \in \{+,-,b,(\text{blank})\}$. 


        \item Similarly, it is customary to denote the \CrefAndHyperrefIfExist{definition:derived_category_of_an_abelian_category}{derived category} of the category of \CrefAndHyperrefIfExist{definition:module_over_a_sheaf_of_rings_on_a_site}{$\calO$-modules} of the category of $\calO$-modules by notations such as \hl{$D(\calC,\calO)$} or \hl{$D(\calC,\calO)$}. All of the usual superscripts apply --- we may thus speak of \hl{$D^{?}(\calC, \calO)$} or \hl{$D^{?}(\calO)$} for $? \in \{+,-,b\}$. 


        If $(\calC, J)$ is a site on some ``space'', e.g. a scheme or topological space, $X$, then it is also usual to denote the homotopy category by \hl{$D(X, \calO)$}. We may also speak of \hl{$D^{?}(X, \calO)$} for $? \in \{+,-,b\}$. 
    \end{enumerate}

\end{notation}

\begin{lemma} \label{lemma:applying_biadd_func_to_res_is_defined_up_to_quasi_iso_and_agrees_with_the_tot_complex_of_the_double_complex_of_the_biadd_func_on_the_two_res}
    Let $F: \calA \times \calB \to \calC$ be a \CrefAndHyperrefIfExist{definition:n_ary_additive_functor_between_additive_categories}{biadditive functor} of \CrefAndHyperrefIfExist{definition:abelian_category}{abelian categories}. Assume that (small) filtered colimits which exist in $\calC$ are exact (e.g. which holds if $\calC$ satisfies \CrefAndHyperrefIfExist{definition:grothendiecks_additional_axioms_for_abelian_categories}{Ab5}).

    Let $A \in \calA$ and $B \in \calB$ be objects. 
    \begin{enumerate}
        \item Suppose that \CrefAndHyperrefIfExist{definition:left_right_resolution_of_a_class_of_objects_in_an_abelian_category}{left resolutions} $P_{A,\bullet} \to A$ and $P_{B,\bullet} \to B$ exist such that $P_{A,i}$ and $P_{B,i}$ are \CrefAndHyperrefIfExist{definition:flat_object_in_an_abelian_category_with_respect_to_a_right_exact_monoidal_product_functor}{flat} with respect to $F$ on the left and right respectively, i.e. $F(P_{A,i}, -): \calB \to \calC$ and $F(-, P_{B,i}): \calA \to \calC$ are exact for all $i$. 

        The complexes $F(P_{A,\bullet}, B)$ and $F(A, P_{B,\bullet})$ are \CrefAndHyperrefIfExist{definition:quasi_isomorphism_of_chain_complexes_of_objects_in_an_abelian_category}{quasi-isomorphic} to the complex $\Tot(F(P_{A,\bullet}, P_{B,\bullet}))$\CrefIfExists{definition:total_complexes_of_a_double_complex_of_objects_in_an_additive_category}\CrefIfExists{definition:double_complex_associated_to_biadditive_functor_and_chain_complexes}.

        \item Suppose that \CrefAndHyperrefIfExist{definition:left_right_resolution_of_a_class_of_objects_in_an_abelian_category}{right resolutions} $A \to I^{A,\bullet}$ and $B \to I^{B,\bullet}$ exist such that $I^{A,i}$ and $I^{B,i}$ are \CrefAndHyperrefIfExist{definition:flat_object_in_an_abelian_category_with_respect_to_a_right_exact_monoidal_product_functor}{flat} with respect to $F$ on the left and right respectively, $F(I^{A,i}, -): \calB \to \calC$ and $F(-, I^{B,i}): \calA \to \calC$ are exact for all $i$. 

        The complexes $F(I^{A,\bullet}, B)$ and $F(A, I^{B,\bullet})$ are \CrefAndHyperrefIfExist{definition:quasi_isomorphism_of_chain_complexes_of_objects_in_an_abelian_category}{quasi-isomorphic} to the complex $\Tot(F(I^{A,\bullet}, I^{B,\bullet}))$\CrefIfExists{definition:total_complexes_of_a_double_complex_of_objects_in_an_additive_category}\CrefIfExists{definition:double_complex_associated_to_biadditive_functor_and_chain_complexes}.
    \end{enumerate}
\end{lemma}

\begin{proof}
    We prove 1. The other part is the dual statement.

    Choose resolutions $P_{A,\bullet} \xrightarrow{\varepsilon} A$ and $P_{B,\bullet} \xrightarrow{\eta} B$ such that $F(P_{A,i}, -): \calB \to \calC$ and $F(-, P_{B,i}): \calA \to \calC$ are exact for all $i$. 
    Identifying $A$ and $B$ with complexes concentrated in degree $0$, we can \CrefAndHyperrefIfExist{definition:double_complex_associated_to_biadditive_functor_and_chain_complexes}{form} the three \CrefAndHyperrefIfExist{definition:double_complex_of_objects_in_an_additive_category}{double complexes} $F(P_{A,\bullet}, P_{B,\bullet})$, $F(A, P_{B,\bullet})$, and $F(P_{A,\bullet} , B)$. Note that the augmentation morphisms $\varepsilon$ and $\eta$ induce morphisms $P_{A,\bullet} \otimes P_{B,\bullet} \to A \otimes P_{B,\bullet}, P_{A,\bullet} \otimes B$.

    Let $C$ be the double complex of objects in $\calC$ obtained from $F(P_{A,\bullet}, P_{B,\bullet})$ by adding $F(A, P_{B,\bullet}[-1])$ in the column $p = -1$. One can show that the translate $\Tot(C)[1]$ is the \CrefAndHyperrefIfExist{definition:mapping_cone_of_a_map_of_chain_cochain_complexes}{mapping cone} of the map 
    $$\Tot(F(P_{A,\bullet}, P_{B,\bullet})) \xrightarrow{\varepsilon \otimes \id} \Tot(F(A, P_{B,\bullet})) = F(A, P_{B,\bullet}).$$
    Moreover, since each $F(-, P_{B,i})$ is an exact functor, every row of $C$ is exact, so $\Tot(C)$ is exact by \Cref{lemma:acyclic_assembly_lemma_for_bounded_double_complexes_with_exact_rows_or_columns}. Therefore, $F(\varepsilon, \id)$ is a quasi-isomorphism and hence 
    $$H_*(\Tot(F(P_{A,\bullet}, P_{B,\bullet}))) \xrightarrow{H_*(F(\varepsilon, P_{B,\bullet}))} H_*(F(A, P_{B,\bullet}))$$
    is a natural isomorphism. 

    By symmetry, there is a natural isomorphism $H_*(\Tot(F(P_{A,\bullet} P_{B,\bullet}))) \xrightarrow{} H_*(F(P_{A,\bullet}, B))$.
\end{proof}

\begin{proposition} \label{proposition:total_derived_functors_for_biadditive_functors_on_abelian_categories_are_well_defined_if_both_source_categories_have_enough_projectives_injectives}
    Let $\mathcal{A}, \calB, \calC$ be \CrefAndHyperrefIfExist{definition:abelian_category}{abelian categories}, and let $F: \mathcal{A} \times \mathcal{B} \to \mathcal{C}$ be a \CrefAndHyperrefIfExist{definition:n_ary_additive_functor_between_additive_categories}{biadditive functor}. 
    \begin{enumerate}
        \item Suppose that $\calA$ and $\calB$ both \CrefAndHyperrefIfExist{definition:has_enough_injectives_or_projectives_for_an_abelian_category}{have enough projectives}. Given objects $M \in D^-(\calA)$ and $N \in D^-(\calA)$\CrefIfExists{definition:derived_category_of_an_abelian_category}, the objects $LF(M,N)$ obtained as $(LF(M,-))(N)$ and $(LF(-,N))(M)$ are naturally isomorphic. Thus, the two definitions of $LF(M,N)$ in \Cref{definition:left_and_right_total_derived_functors_of_biadditive_functors_of_abelian_categories} are in agreement.
        \item Dually, suppose that $\calA$ and $\calB$ both \CrefAndHyperrefIfExist{definition:has_enough_injectives_or_projectives_for_an_abelian_category}{have enough injectives}. Given objects $M \in D^+(\calA)$ and $N \in D^+(\calA)$\CrefIfExists{definition:derived_category_of_an_abelian_category}, the objects $RF(M,N)$ obtained as $(RF(M,-))(N)$ and $(RF(-,N))(M)$ are naturally isomorphic. Thus, the two definitions of $RF(M,N)$ in \Cref{definition:left_and_right_total_derived_functors_of_biadditive_functors_of_abelian_categories} are in agreement.
    \end{enumerate}
\end{proposition}
\begin{proof}
    We prove 1. The other part is dual. By \Cref{theorem:derived_categories_can_be_identified_with_homotopy_categories_of_injectives_or_projectives}, note that $D^-(\calA)$ and $D^-(\calB)$ are respectively equivalent to the categories \CrefAndHyperrefIfExist{definition:category_of_bounded_complexes_of_injectives_projectives}{$K^-(\calP_\calA)$ and $K^-(\calP_\calB)$} of cohomologically bounded above complexes of projectives in $\calA$ and $\calB$ respectively. Let 
    $$P_\bullet \to M$$
    and 
    $$Q_\bullet \to N$$
    be projective resolutions.
    By \Cref{corollary:additive_functor_bewteen_abelian_categories_has_a_right_or_left_total_derived_functor_if_the_source_abelian_category_has_enough_injectives_or_projectives}, 
    $$(LF(M,-)(N) \cong q(K(F(M,-)))(N)$$
    $$(LF(-,N)(M) \cong q(K(F(-,N)))(M)$$
    The former is represented by the complex $F(M,Q_\bullet)$ and the latter is represented by the complex $F(P_\bullet, N)$. These are quasi-isomorphic by \Cref{lemma:applying_biadd_func_to_res_is_defined_up_to_quasi_iso_and_agrees_with_the_tot_complex_of_the_double_complex_of_the_biadd_func_on_the_two_res}.
\end{proof}


\begin{definition} \label{definition:left_and_right_total_derived_functors_of_biadditive_functors_of_abelian_categories}
    Let $\mathcal{A}, \calB, \calC$ be \CrefAndHyperrefIfExist{definition:abelian_category}{abelian categories}, and let $F: \mathcal{A} \times \mathcal{B} \to \mathcal{C}$ be a \CrefAndHyperrefIfExist{definition:n_ary_additive_functor_between_additive_categories}{biadditive functor}. 

    There are notions of derived functors that we may consider depending on whether $\calA$ or $\calB$  has enough projectives or injectives or \CrefAndHyperrefIfExist{definition:flat_object_in_an_abelian_category_with_respect_to_a_right_exact_monoidal_product_functor}{flats} with respect to $F$. \TODO{define for flats}

    \TODO{I really should be letting $M$ and $N$ be complxes, not just objects.}

    \begin{enumerate}
    \item         
    \end{enumerate}


    \begin{enumerate}
        \item 
        \begin{enumerate}
            \item Let $M \in \calA$. Assume that $\calB$ \CrefAndHyperrefIfExist{definition:has_enough_injectives_or_projectives_for_an_abelian_category}{has enough projectives}. By \Cref{corollary:additive_functor_bewteen_abelian_categories_has_a_right_or_left_total_derived_functor_if_the_source_abelian_category_has_enough_injectives_or_projectives}, the functor $F(M, -): \calB \to \calC$ has a \CrefAndHyperrefIfExist{definition:total_derived_functor_of_an_exact_functor_between_homotopy_categories_of_abelian_categories}{left derived functor} $L(F(M,-)): D^-(\calB) \to D(\calC)$\CrefIfExists{definition:derived_category_of_an_abelian_category}, which we may write as \hl{$LF(M,-)$}.

            \item Symmetrically, let $N \in \calB$. Assume that $\calA$ \CrefAndHyperrefIfExist{definition:has_enough_injectives_or_projectives_for_an_abelian_category}{has enough projectives}. By \Cref{corollary:additive_functor_bewteen_abelian_categories_has_a_right_or_left_total_derived_functor_if_the_source_abelian_category_has_enough_injectives_or_projectives}, the functor $F(-, N): \calA \to \calC$ has a \CrefAndHyperrefIfExist{definition:total_derived_functor_of_an_exact_functor_between_homotopy_categories_of_abelian_categories}{left derived functor} $L(F(-, N)): D^-(\calA) \to D(\calC)$\CrefIfExists{definition:derived_category_of_an_abelian_category}, which we may write as \hl{$LF(-,N)$}.
        \end{enumerate}
        Assuming that $\calA$ and $\calB$ have enough projectives, the notations $LF(M,N)$ above are in agreement (\Cref{proposition:total_derived_functors_for_biadditive_functors_on_abelian_categories_are_well_defined_if_both_source_categories_have_enough_projectives_injectives})


        \item
        \begin{enumerate}
            \item Let $M \in \calA$. Assume that $\calB$ \CrefAndHyperrefIfExist{definition:has_enough_injectives_or_projectives_for_an_abelian_category}{has enough injectives}. By \Cref{corollary:additive_functor_bewteen_abelian_categories_has_a_right_or_left_total_derived_functor_if_the_source_abelian_category_has_enough_injectives_or_projectives}, the functor $F(M, -): \calB \to \calC$ has a \CrefAndHyperrefIfExist{definition:total_derived_functor_of_an_exact_functor_between_homotopy_categories_of_abelian_categories}{right derived functor} $R(F(M,-)): D^+(\calB) \to D(\calC)$\CrefIfExists{definition:derived_category_of_an_abelian_category}, which we may write as \hl{$RF(M,-)$}.

            \item Symmetrically, let $N \in \calB$. Assume that $\calA$ \CrefAndHyperrefIfExist{definition:has_enough_injectives_or_projectives_for_an_abelian_category}{has enough injectives}. By \Cref{corollary:additive_functor_bewteen_abelian_categories_has_a_right_or_left_total_derived_functor_if_the_source_abelian_category_has_enough_injectives_or_projectives}, the functor $F(-, N): \calA \to \calC$ has a \CrefAndHyperrefIfExist{definition:total_derived_functor_of_an_exact_functor_between_homotopy_categories_of_abelian_categories}{right derived functor} $R(F(-, N)): D^+(\calA) \to D(\calC)$\CrefIfExists{definition:derived_category_of_an_abelian_category}, which we may write as \hl{$RF(-,N)$}.
        \end{enumerate}
        Assuming that $\calA$ and $\calB$ have enough injectives, the notations $RF(M,N)$ above are in agreement (\Cref{proposition:total_derived_functors_for_biadditive_functors_on_abelian_categories_are_well_defined_if_both_source_categories_have_enough_projectives_injectives})

        \item Let $M \in \calA$. Assume that $\calB$ \CrefAndHyperrefIfExist{definition:has_enough_flat_objects_for_an_abelian_category_with_respect_to_a_right_exact_bifunctor}{has enough flats} with respect to $\otimes$. Define \hl{$LF(M, -): D^-(\calB) \to D(\calC)$} as follows:
    \end{enumerate}
    See \Cref{definition:derived_tensor_product_on_bounded_above_derived_categories_of_abelian_categories_for_a_biadditive_right_exact_functor} for notation used in the case that $F$ is written as a tensor product.
\end{definition}


\begin{definition} \label{definition:derived_tensor_product_on_bounded_above_derived_categories_of_abelian_categories_for_a_biadditive_right_exact_functor}
    Let $\mathcal{A}, \calB, \calC$ be \CrefAndHyperrefIfExist{definition:abelian_category}{abelian categories}, and let $\otimes : \mathcal{A} \times \mathcal{B} \to \mathcal{C}$ be a \CrefAndHyperrefIfExist{definition:n_ary_additive_functor_between_additive_categories}{biadditive functor} written as tensor product. 

    \TODO{need a definition with enough flats}
    \begin{enumerate}
        \item Let $M \in \calA$. Assume that $\calB$ \CrefAndHyperrefIfExist{definition:has_enough_injectives_or_projectives_for_an_abelian_category}{has enough projectives}. We write \hl{$M \otimes^L -$} for \CrefAndHyperref{definition:left_and_right_total_derived_functors_of_biadditive_functors_of_abelian_categories}{$LF(M, -): D^-(\calB) \to D(\calC)$} in the case that $F = M \otimes -: \calB \to \calC$.
        
        \item Symmetrically, let $N \in \calB$. Assume that $\calA$ \CrefAndHyperrefIfExist{definition:has_enough_injectives_or_projectives_for_an_abelian_category}{has enough projectives}. We write \hl{$- \otimes^L N$} for \CrefAndHyperref{definition:left_and_right_total_derived_functors_of_biadditive_functors_of_abelian_categories}{$LF(-, N): D^+(\calA) \to D(\calC)$} in the case that $F = - \otimes N: \calA \to \calC$.

        \TODO{show that projectives vs. flats yield the same thing}
        \TODO{show that flats in each variable yield the same thing}
        \item Let $M \in \calA$. Assume that $\calB$ \CrefAndHyperrefIfExist{definition:has_enough_flat_objects_for_an_abelian_category_with_respect_to_a_right_exact_bifunctor}{has enough flats}. We alternatively define \hl{$M \otimes^L -: D^-(\calB) \to D(\calC)$} as follows --- given an object $N \in D^-(\calB)$, say that $Q^\bullet$ is a complex of flat objects in $\calB$ representing $N$ \TODO{show that such a thing exists}, and let $M \otimes^L N$ be the object of $D(\calC)$ represented by the complex $M \otimes Q^\bullet$.  \TODO{show that this is well defined, i.e. does not depend on the choice of flat resolution}

        \item Symmetrically, let $N \in \calB$. Assume that $\calA$ \CrefAndHyperrefIfExist{definition:has_enough_flat_objects_for_an_abelian_category_with_respect_to_a_right_exact_bifunctor}{has enough flats}. We alternatively define \hl{$- \otimes^L N: D^-(\calA) \to D(\calC)$} as follows --- given an object $M \in D^-(\calA)$, say that $P^\bullet$ is a complex of flat objects in $\calA$ representing $M$ \TODO{show that such a thing exists}, and let $M \otimes^L N$ be the object of $D(\calC)$ represented by the complex $P^\bullet \otimes N$.  \TODO{show that this is well defined, i.e. does not depend on the choice of flat resolution}

    \end{enumerate}
    The first two notions agree assuming that $\calA$ and $\calB$ have enough projectives. \TODO{comment on the next two notions agreeing}
\end{definition}


\section{Miscellaneous definitions}


\begin{definition} \label{definition:commutative_ring}
   A \hldef{commutative (unital) ring} is a \CrefAndHyperrefIfExist{definition:ring}{ring} $(R, +, \cdot)$ such that $\cdot$ is a \CrefAndHyperrefIfExist{definition:commutative_binary_operation}{commutative operation}, i.e. $a \cdot b = b \cdot a$. 

   For many writers (e.g. ``commutative'' algebraists or number theorists), a \hldef{ring} refers to a commutative ring as above.
\end{definition}


\begin{definition} \label{definition:ideal_of_a_ring}
Let $R$ be a (not necessarily commutative, possibly nonunital) \CrefAndHyperrefIfExist{definition:ring}{ring}.  
A \hldef{left ideal of $R$} is a subset $I \subseteq R$ such that
\begin{itemize}
    \item $(I,+)$ is an additive \CrefAndHyperrefIfExist{definition:subgroup_of_a_group}{subgroup} of $(R,+)$,
    \item $RI \subseteq I$, i.e., for all $r \in R$ and $x \in I$, one has $rx \in I$.
\end{itemize}
Similarly, a \hldef{right ideal of $R$} is a subset $I \subseteq R$ such that
\begin{itemize}
    \item $(I,+)$ is an additive subgroup of $(R,+)$,
    \item $IR \subseteq I$, i.e., for all $r \in R$ and $x \in I$, one has $xr \in I$.
\end{itemize}
A \hldef{two-sided ideal} (or simply an \hldef{ideal}) of $R$ is a subset $I \subseteq R$ which is both a left ideal and a right ideal of $R$. We denote by \hl{$I \unlhd R$} the relation expressing that $I$ is a two-sided ideal of $R$.

\TextIfExists{definition:module_of_a_ring}{Equivalently, an left/right/two-sided ideal of $R$ is a \CrefAndHyperrefIfExist{definition:submodule_of_a_module_over_a_ring}{submodule} of $R$ as an \CrefAndHyperrefIfExist{definition:module_of_a_ring}{$R$-module}.}

A left/right/two-sided ideal is said to be \hldef{proper} if it is strictly contained in $R$.

Note that every left or right ideal of a commutative ring is a two-sided ideal.
\end{definition}


\begin{definition} \label{definition:local_ring}
Let $R$ be a \CrefAndHyperrefIfExist{definition:ring}{ring} with unity, not necessarily commutative.  
The ring $R$ is called a \hldef{local ring} if it has a unique \CrefAndHyperrefIfExist{definition:prime_and_maximal_ideal_of_a_ring}{maximal left ideal}. In this case, $R$ also has a unique maximal right ideal, and these coincide with the \CrefAndHyperrefIfExist{definition:jacobson_radical_of_a_ring}{Jacobson radical $J(R)$ of $R$}.  
The unique maximal left (and right) ideal of a local ring $R$ may sometimes be denoted by \hl{$\mathfrak{m}_R$}.
\end{definition}


\begin{definition} \label{definition:prime_and_maximal_ideal_of_a_ring}
Let $R$ be a \CrefAndHyperrefIfExist{definition:ring}{(not necessarily commutative) ring}. A \CrefAndHyperrefIfExist{definition:ideal_of_a_ring}{proper two-sided ideal $P \unlhd R$} is called a \hldef{prime ideal} if the following equivalent conditions holds:
\begin{enumerate}
    \item If $I,J$ are left ideals and \CrefAndHyperrefIfExist{definition:product_of_ideals_of_a_ring}{$IJ \subset P$}, then $I \subset P$ or $J \subset P$.
    \item If $I,J$ are right ideals and $IJ \subset P$, then $I \subset P$ or $J \subset P$.
    \item If $I,J$ are two-sided idaels and $IJ \subset P$, then $I \subset P$ or $J \subset P$.
    \item If $x,y \in R$ with $xRy \subset \mathfrak{p}$, then $x \in \mathfrak{p}$ or $y \in \mathfrak{p}$.
\end{enumerate}

A proper left/right/two-sided ideal $M \subsetneq R$ is called \hldef{maximal} if there exists no other left/right/two-sided ideal $J \unlhd R$ such that $M \subsetneq J \subsetneq R$. Equivalently, 
\begin{itemize}
    \item a left/right ideal $M$ of $R$ is maximal if and only if the \CrefAndHyperrefIfExist{definition:quotient_module_of_a_module_by_a_module_of_a_ring}{quotient module $R/M$} is a \CrefAndHyperrefIfExist{definition:simple_module_of_a_ring}{simple} left/right $R$-module.
    \item a two-sided ideal $M$ of $R$ is maximal if and only if the \CrefAndHyperrefIfExist{definition:quotient_ring_of_a_ring_by_a_two_sided_ideal}{quotient ring $R/M$} is a \CrefAndHyperrefIfExist{definition:simple_ring}{simple ring}. 
\end{itemize}
\end{definition}

\begin{definition} \label{definition:local_morphism_of_local_rings}
Let $(R,\mathfrak{m}_R)$ and $(S,\mathfrak{m}_S)$ be \CrefAndHyperrefIfExist{definition:local_ring}{local rings}, not necessarily commutative.  
A \CrefAndHyperrefIfExist{definition:ring_homomorphism}{ring homomorphism} $\varphi : R \to S$ is called a \hldef{local morphism} (or \hldef{local homomorphism}) if $\varphi(\mathfrak{m}_R) \subseteq \mathfrak{m}_S$.
\end{definition}




\begin{definition}[Topology] \label{definition:topological_space}
Let $X$ be a set. A \hldef{topology on $X$} is a collection $\mathcal{T}$ of subsets of $X$ such that:
\begin{enumerate}
    \item $\emptyset \in \mathcal{T}$ and $X \in \mathcal{T}$,
    \item For any collection $\{ U_i \}_{i \in I} \subseteq \mathcal{T}$ (with $I$ arbitrary), the union $\bigcup_{i \in I} U_i \in \mathcal{T}$,
    \item For any finite collection $\{ U_1, \ldots, U_n \} \subseteq \mathcal{T}$, the intersection $U_1 \cap \cdots \cap U_n \in \mathcal{T}$.
\end{enumerate}
If $\mathcal{T}$ is a topology on $X$, the pair $(X, \mathcal{T})$ is called a \hldef{topological space}. Members of $\mathcal{T}$ are called \hldef{open sets}. 

A subset $C \subseteq X$ is \hldef{closed} if its complement $X \setminus C$ is an open set in $\mathcal{T}$

One very often refers to $X$ as a topological spcae, omitting the notation of the topology $\mathcal{T}$. 

The collection of all topologies on a set $X$ may be denoted by notations such as \hl{$\mathrm{Top}(X)$}, \hl{$\mathbf{Top}(X)$}, or \hl{$\mathsf{Top}(X)$}.
\end{definition}






\begin{definition} \label{definition:locally_closed_subset_of_a_topological_space}
    Let $X$ be a \CrefAndHyperrefIfExist{definition:topological_space}{topological space}. A subset $Z \subseteq X$ is called a \hldef{locally closed subset} if $Z$ can be written as the intersection $U \cap C$, where $U$ is an open subset of $X$ and $C$ is a closed subset of $X$. Equivalently, $Z$ is a locally closed subset if it is an open subset of its closure $\overline{Z}$ endowed with the \CrefAndHyperrefIfExist{definition:subspace_of_a_topological_space}{subspace topology}.
\end{definition}


\begin{definition}\label{definition:ring}
    A \hldef{ring} is a triple $(R, +, \cdot)$ where 
    \begin{enumerate}
        \item $(R,+)$ is a \CrefAndHyperrefIfExist{definition:group}{commutative group}, and
        \item $(R, \cdot)$ is a \CrefAndHyperrefIfExist{definition:monoid}{monoid}. 
        \item $\cdot$ is distributive over $+$, i.e. for all $a,b,c \in R$, we have
        $$a \cdot (b+c) = a \cdot b + a \cdot c \quad \text{and} \quad (a+b) \cdot c = a \cdot c + b \cdot c.$$
    \end{enumerate}

    Equivalently, a ring is a triple $(R,+,\cdot)$ where $+,\cdot: R \times R \to R$ are binary operations satisfying
    \begin{enumerate}
        \item $(a+b)+c = a+(b+c)$ and $(ab)c = a(bc)$ for all $a,b,c \in R$
        \item There exists an element \hl{$0 \in R$} such that $a+0 = a = 0 + a$ for all $a \in R$.
        \item For every $a \in R$, there exists an element \hl{$-a \in R$} such that $a+(-a) = 0 = (-a) + a$ for all $a \in R$.
        \item There exists an element \hl{$1 \in R$} such that $a \cdot 1 = a = 1 \cdot a$ for all $a \in R$.
        \item For all $a,b,c \in R$, we have
        $$a \cdot (b+c) = a \cdot b + a \cdot c \quad \text{and} \quad (a+b) \cdot c = a \cdot c + b \cdot c.$$
    \end{enumerate} 

    The operation $+$ is often called \hldef{addition} and the operation $\cdot$ is often called \hldef{multiplication}. Accordingly, the identity element $0$ of $+$ is often called the \hldef{additive identity} and the identity element $1$ of $\cdot$ is often called the \hldef{multiplicative identity}.

    % If $\cdot$ is additionally a \CrefAndHyperrefIfExist{definition:commutative_binary_operation}{commutative operation}, i.e. $a \cdot b = b \cdot a$ for all $a,b \in R$, then we call the ring \hldef{commutative}.  


\end{definition}
\begin{remark}
    Some writers might not require a ring to have a multiplicative identity element, i.e. would define a ring so that $(R,+)$ is a commutative group, $(R, \cdot)$ is a semigroup, and $\cdot$ is distributive over $+$. Such writers would call the notion of ring in \Cref{definition:ring} a \hldef{unitary ring} to emphasize the existence of the multiplicative identity $1$. 
\end{remark}


\begin{definition}[Constant sheaf on a site] \label{definition:constant_sheaf_on_a_site_with_sheafification}
    Let $\calC$ be a \hyperrefIfExists{definition:category}{(large) category}\CrefIfExists{definition:category}, let $\calA$ be a (large category), and let $A$ be an object of $\calA$. %a set (or more generally, an abelian group, ring, etc.).
    
    \begin{enumerate}
        \item The \hldef{constant presheaf on $\calC$ with value $A$} is the \hyperrefIfExists{definition:presheaf_on_a_category}{presheaf}\CrefIfExists{definition:presheaf_on_a_category} $P$ defined by
        \[
        P(U) = A
        \]
        for every object $U$ of $\calC$ such that every morphism $f: V \to U$ in $\calC$ induces the identity map $A = P(U)\to P(V) = A$. 

        \item Let $\calC$ be a \CrefAndHyperrefIfExist{definition:grothendieck_topology_on_a_category_site_covering_sieve_topologically_generating_family}{site} and assume that a \CrefAndHyperrefIfExist{definition:sheafification_functor_on_a_site}{sheafification functor} 
        $$a: \Shv(\calC, \calA) \to \PreShv(\calC, \calA)$$
        exists\TextIfExists{theorem:sheafification_of_a_presheaf_of_sets_on_a_small_enough_site}{~(e.g. see \Cref{theorem:sheafification_of_a_presheaf_of_sets_on_a_small_enough_site})}.
        The \hldef{constant sheaf on $\calC$ with value $A$}, or the \hldef{constant sheaf on $\calC$ associated to $A$} commonly denoted \hl{$\underline{A}$} or sometimes just \hl{$A$} by abuse of notation, is the \hyperrefIfExists{theorem:sheafification_of_a_presheaf_of_sets_on_a_small_enough_site}{sheaf associated to}\CrefIfExists{theorem:sheafification_of_a_presheaf_of_sets_on_a_small_enough_site} the constant presheaf $P$ with value $A$ above.

        \item Let $\calC$ be a site. Let $\calO$ be a sheaf of (not-necessarily commutative) rings on $\calC$. Assume that the \CrefAndHyperrefIfExist{definition:sections_of_a_presheaf_on_a_category_valued_in_a_category}{global sections ring $\Gamma(\calO)$} exists. A \hldef{constant $\calO$-module} is an \CrefAndHyperrefIfExist{definition:module_over_a_sheaf_of_rings_on_a_site}{$\calO$-module} $\calF$ which is isomorphic as a sheaf to the constant sheaf on $\calC$ with value $M$ where $M$ is a module of the ring $\Gamma(\calO)$. Note that sheafification functors exist for presheaves/sheaves valued in $\Ab$ (\Cref{theorem:sheafification_of_a_presheaf_of_sets_on_a_small_enough_site}).

        In case that $\calO$ is the constant sheaf associated to $A$ for some (not-necessarily commutative) ring $A$, a constant $\calO$-module is simply called a \hldef{constant $A$-module}.
    \end{enumerate}
\end{definition}



\begin{definition} \label{definition:module_of_a_ring}
Let $R$ be a \CrefAndHyperrefIfExist{definition:ring}{not-necessarily commutative ring}. 
\begin{enumerate}
    \item A \hldef{left $R$-module} is an abelian group $(M,+)$ together with an operation $R \times M \to M$, denoted $(r,m) \mapsto rm$, such that for all $r,s \in R$ and $m,n \in M$:
    \begin{itemize}
        \item $r(m+n) = rm + rn$,
        \item $(r+s)m = rm + sm$,
        \item $(rs)m = r(sm)$,
        \item $1_R m = m$ where $1_R$ is the multiplicative identity of $R$.
    \end{itemize}

    \item A \hldef{right $R$-module} is defined similarly as an abelian group $(M,+)$ with an operation $M \times R \to M$, denoted $(m,r) \mapsto mr$, such that for all $r,s \in R$ and $m,n \in M$:
    \begin{itemize}
        \item $(m+n)r = mr + nr$,
        \item $m(r+s) = mr + ms$,
        \item $m(rs) = (mr)s$,
        \item $m 1_R = m$.
    \end{itemize}

    \item Let $R$ and $S$ be  (not necessarily commutative) \CrefAndHyperrefIfExist{definition:ring}{rings}.

    An \hldef{$R$-$S$-bimodule} (or an \hldef{$R$-$S$-module} or an $(R,S)$-module, etc.)is an \CrefAndHyperrefIfExist{definition:group}{abelian group} $(M,+)$ equipped with
    \begin{enumerate}
        \item a left action of $R$:
        $$\hlin{R \times M \to M, \quad (r,m) \mapsto r \cdot m},$$
        making $M$ a \CrefAndHyperrefIfExist{definition:module_of_a_ring}{left $R$-module},
        \item a right action of $S$:
        $$\hlin{M \times S \to M, \quad (m,s) \mapsto m \cdot s},$$
        making $M$ a right $S$-module,
    \end{enumerate}
    such that the left and right actions commute; that is, for all $r \in R$, $s \in S$, and $m \in M$,
    $$ r \cdot (m \cdot s) = (r \cdot m) \cdot s.  $$

    \item A \hldef{two-sided $R$-module} (or \hldef{$R$-bimodule}) is an $R$-$R$-bimodule.
    
    % an abelian group $(M,+)$ which is simultaneously a left $R$-module and a right $R$-module, such that $(rm)s = r(ms)$ for all $r,s \in R$, $m \in M$. Equivalently, a two-sided $R$-module is an \hldef{$R$-$R$-bimodule}\CrefIfExists{definition:module_of_a_ring}


\end{enumerate}
If $R$ is a \CrefAndHyperrefIfExist{definition:commutative_ring}{commutative ring}, then a left/right $R$-module can automatically be regarded as a two-sided $R$-module. As such, we simply talk about \hldef{$R$-modules} in this case. 

Any abelian group is equivalent to a two-sided $\bbZ$-module. Moreover, any left $R$-module is equivalent to an \CrefAndHyperrefIfExist{definition:module_of_a_ring}{$R-\bbZ$-bimodule} and any right $R$-module is equivalent to an \CrefAndHyperrefIfExist{definition:module_of_a_ring}{$\bbZ-R$-bimodule}. Given a left/right/two-sided $R$-module, its \hldef{natural bimodule structure} will refer to its structure as a $R$-$\bbZ$/$\bbZ$-$R$/$R$-$R$ bimodule. In this way, many definitions associated with the notions of left/right/two-sided $R$-modules can be defined as special cases for definitions for $R$-$S$-bimodules.
\end{definition}

\begin{definition}[Tensor product of bimodules] \label{definition:tensor_product_of_bimodules_of_rings}
Let $R,S,T$ be \CrefAndHyperrefIfExist{definition:ring}{(not necessarily commutative) rings}, let $M$ be an \CrefAndHyperrefIfExist{definition:module_of_a_ring}{$R$-$S$ bimodule}, and let $N$ be an $S$-$T$ bimodule. In the \CrefAndHyperrefIfExist{definition:free_abelian_group_generated_by_a_set}{free abelian group} $\bbZ[M \times N]$ generated by the \CrefAndHyperrefIfExist{definition:product_of_sets}{Cartesian product $M \times N$}, let $U$ be the subgroup generated by elements of the form
\TODO{subgroup generated}
\begin{align*}
&(m+m',n) - (m,n) - (m',n),\\
&(m,n+n') - (m,n) - (m,n'),\\
&(m \cdot s, n) - (m, s \cdot n),
\end{align*}
for all $m,m' \in M$, $n,n' \in N$, and $s \in S$. The \hldef{tensor product of $M$ and $N$ over $S$} is the \CrefAndHyperrefIfExist{definition:quotient_of_a_group_by_a_normal_subgroup}{quotient} abelian group
$$M \otimes_S N := \mathbb{Z}[M \times N] / U.$$
The image of an element of the form $(m,n) \in M \times N$ in $M \otimes_S N$ is denoted \hl{$m \otimes n$} and called a \hldef{pure tensor}. In general, the elements of $M \otimes_S N$ are finite sums 
$$\sum_{i=1}^n m_i \otimes n_i \quad m_i \in M, n_i \in N$$
of pure tensors. Thus, the pure tensors satisfy the following relations:
\begin{align*}
    (m + m') \otimes n &= m \otimes n + m' \otimes n \\ 
    m \otimes (n + n') &= m \otimes n + m \otimes n' \\
    (m \cdot s) \otimes n &= m \otimes (s \cdot n)
\end{align*}

This tensor product becomes naturally an $R$-$T$ bimodule with left action and right action defined by
\begin{align*}
r \cdot (m \otimes n) &= (r \cdot m) \otimes n, \\
(m \otimes n) \cdot t &= m \otimes (n \cdot t),
\end{align*}
for all $r \in R$, $t \in T$, $m \in M$, and $n \in N$.

Inductively, given rings $R_0,\ldots,R_k$ and $R_{i-1}-R_i$-bimodules $M_i$ for $i = 1,\ldots,k$, we may speak of the tensor product
$$M_0 \otimes_{R_1} M_1 \otimes_{R_2} \cdots \otimes_{R_{k-1}} M_k;$$
tensor products are associative\TODO{}, so parentheses are not strictly needed to notate them. Its \hldef{pure tensors} are elements of the form $m_0 \otimes m_1 \otimes \cdots \otimes m_k$ for $m_i \in M_i$, and its general elements are finite sums
$$\sum_{j=1}^n m_{0j} \otimes m_{1j} \otimes \cdots m_{kj} \quad m_{ij} \in M_i.$$
of pure tensors. It also has a natural $R_0-R_k$-bimodule structure.

\TextIfExists{definition:n_ary_additive_functor_between_additive_categories}{In general, $(M_0,\ldots,M_k) \mapsto M_0 \otimes_{R_1} M_1 \otimes_{R_2} \cdots \otimes_{R_{k-1}} M_k$ defines a \CrefAndHyperrefIfExist{definition:n_ary_additive_functor_between_additive_categories}{$(k+1)$-ary additive functor}
$${}_{R_0}\mathbf{Mod}_{R_1} \times \cdots \times {}_{R_{k-1}}\mathbf{Mod}_{R_k} \to {}_{R_0} \mathbf{Mod}_{R_k}$$
(\Cref{theorem:the_category_of_R_S_bimodules_is_a_grothendieck_abelian_category_and_AB4_star}).}


Given a ring $R$ and a two-sided $R$-module $M$, we may also speak of the \hldef{$n$-fold tensor product} \hl{$M^{\otimes n} = M^{\otimes_R n}$}

\end{definition}
\begin{definition}[n-ary Additive Functor] \label{definition:n_ary_additive_functor_between_additive_categories}
Let $I$ be a finite set with $|I| = n$. Let $\{\mathcal{A}_i\}_{i\in I}$ be \CrefAndHyperrefIfExist{definition:additive_category_preadditive_category}{additive categories} and let $\mathcal{B}$ be an additive category. An \hldef{n-ary additive functor} (or \hldef{multilinear functor})
$$F : \prod_{i\in I}\mathcal{A}_i \to \mathcal{B}$$
\CrefIfExists{definition:product_category_of_a_family_of_categories} is a functor such that for each fixed collection of all but one variable, the resulting functor in the remaining variable is \CrefAndHyperrefIfExist{definition:additive_functor_between_additive_categories}{additive}. Equivalently, for every $j\in I$ and objects $(A_i)_{i\in I}$ and morphisms $f_1,f_2:A_j\to A'_j$ in $\mathcal{A}_j$, we have
\begin{align*}
&F(A_1,\dots,A_{j-1}, f_1+f_2, A_{j+1},\dots,A_n)
\\
=& F(A_1,\dots,A_{j-1}, f_1, A_{j+1},\dots,A_n)  \\
& + F(A_1,\dots,A_{j-1}, f_2, A_{j+1},\dots,A_n),
\end{align*}
and $F$ preserves zero morphisms componentwise:
$$F(A_1,\dots,0_{A_j,A'_j},\dots,A_n) = 0_{F(A_1,\dots),F(A'_1,\dots)}.$$
A bifunctor that satisfies this property for $n=2$ is simply called a \hldef{biadditive functor}.
\end{definition}

\begin{definition} \label{definition:open_and_closed_immersions_of_schemes}
Let $f: (X, \mathcal{O}_X) \to (Y, \mathcal{O}_Y)$ be a \CrefAndHyperrefIfExist{definition:morphism_of_schemes}{morphism of schemes}.
\begin{itemize}
    \item The morphism $f$ is called an \hldef{open immersion} if the underlying map of topological spaces induces a \CrefAndHyperrefIfExist{definition:homeomorphism_of_topological_spaces}{homeomorphism} from $X$ onto an open subset $V \subseteq Y$, and the induced \CrefAndHyperrefIfExist{definition:sheaf_on_a_site}{map of sheaves} \CrefAndHyperrefIfExist{definition:morphism_of_locally_ringed_spaces}{$f^\sharp|_V: \mathcal{O}_Y|_V \to f_* \mathcal{O}_X$} is an isomorphism of sheaves of rings on $V$.

    \TODO{surjective map of sheaves of sets}
    \item The morphism $f$ is called a \hldef{closed immersion} if the underlying map of topological spaces induces a homeomorphism from $X$ onto a closed subset $Z \subseteq Y$, and the induced map of sheaves $f^\sharp: \mathcal{O}_Y \to f_* \mathcal{O}_X$ is surjective.
\end{itemize}
\end{definition}
\begin{definition} \label{definition:cartesian_product_of_two_objects_in_a_category_over_an_object}
    Let $\mathcal{C}$ be a \CrefAndHyperrefIfExist{definition:category}{category}, let $Z$ be an object, and let $X, Y$ be objects of $\mathcal{C}$ \CrefAndHyperrefIfExist{definition:category_of_objects_over_under_a_fixed_object_in_a_category}{over} $Z$, i.e. morphisms $X \to Z$ and $Y \to Z$ are fixed. A \hldef{cartesian product of $X$ and $Y$ over $Z$ in $\mathcal{C}$} (or \hldef{fiber product} or \hldef{pullback diagram}) is an object, often denoted by \hl{$X \times_Z Y$}, with \hldef{projection morphisms} $X \times_Z Y \to X$ and $X \times_Z Y \to Y$ that are universal. 
    More precisely, for any object $T$ of $\mathcal{C}$ and morphisms $f_X : T \to X$, $f_Y : T \to Y$, there exists a unique morphism $u : T \to X \times_Z Y$ such that the following diagram commutes:
        \begin{center}
        \begin{tikzcd}
            T \ar[rd, dotted, "u" ] \ar[rrd, "f_X", bend left] \ar[ddr, "f_Y", bend right] & & \\
            & X \times_Z Y \ar[r] \ar[d] &  \ar[d] X \\
            & Y \ar[r] & Z
        \end{tikzcd}
        \end{center}
        Equivalently, $X \times_Z Y$ is the \CrefAndHyperrefIfExist{definition:limit_and_colimit_of_a_diagram_in_a_category}{limit} of the \CrefAndHyperrefIfExist{definition:diagram_in_a_category_indexed_by_a_small_category}{diagram}
        \begin{center}
            \begin{tikzcd}
            & X \ar[d] \\
            Y \ar[r] & Z
            \end{tikzcd}
        \end{center}
        in $\calC$. 

        The commutative diagram 
        \begin{center}
        \begin{tikzcd}
        X \times_Z Y \ar[r] \ar[d] & X \ar[d] \\
        Y \ar[r] & Z
        \end{tikzcd} 
        \end{center}
        may be referred to as a \hldef{cartesian square}.

\end{definition}
\begin{definition}[Locally Noetherian Scheme and Noetherian Scheme] \label{definition:locally_noetherian_and_noetherian_scheme}
Let $X$ be a \CrefAndHyperrefIfExist{definition:scheme}{scheme}.

\begin{itemize}
    \item $X$ is called \hldef{locally Noetherian} if it admits an open cover $\{U_i\}$ such that for each $i$, the ring $\mathcal{O}_X(U_i)$ of regular functions on $U_i$ is a \CrefAndHyperrefIfExist{definition:noetherian_ring}{Noetherian ring}. Equivalently, $X$ is locally Noetherian if it is covered by open affine subschemes $\Spec A_i$ with each $A_i$ a Noetherian ring.

    \item $X$ is called \hldef{Noetherian} if it is locally Noetherian and \CrefAndHyperrefIfExist{definition:quasi_compact_scheme}{quasi-compact}, i.e., $X$ can be covered by finitely many affine opens $\Spec A_i$ where each $A_i$ is Noetherian.
\end{itemize}
\end{definition}

\begin{definition}[Integral element over a ring] \label{definition:integral_element_over_a_ring}
    Let $R$ be a \CrefAndHyperrefIfExist{definition:commutative_ring}{commutative ring with unity}. 
    \begin{enumerate}
        \item Let $A$ be an \CrefAndHyperrefIfExist{definition:algebra_of_a_ring}{$R$-algebra}. An element $a \in A$ is called \hldef{integral over $R$} if there exists a monic polynomial
        $$p(x) = x^n + r_{n-1} x^{n-1} + \cdots + r_1 x + r_0$$
        with coefficients $r_i \in R$ such that
        $$p(a) = a^n + r_{n-1} a^{n-1} + \cdots + r_1 a + r_0 = 0 \quad \text{in } A.$$

        \item Let $A$ be an \CrefAndHyperrefIfExist{definition:extension_of_a_ring}{extension ring} of $R$. The ring extension $A/R$ is called an \hldef{integral extension} if every element of $A$ is integral over $R$.

        \item Let $A$ be an \CrefAndHyperrefIfExist{definition:extension_of_a_ring}{extension ring} of $R$. The \hldef{integral closure of $R$ in $A$}, sometimes denoted by $\widetilde{A}$, is the subring 
        $$\widetilde{A} = \{a \in A: a \text{ is integral over } R\}.$$
        We say that $R$ is integrally closed in $A$ if $\widetilde{A}$ coincides with $A$ (considered as a \CrefAndHyperrefIfExist{definition:subring_of_a_ring}{subring of $R$}).

        \item Let $R$ be an integral domain with field of fractions $K = \mathrm{Frac}(R)$. We say that $R$ is \hldef{integrally closed} if it is integrally closed as a subring of $K$.
        
        % in $K$ if every element of $K$ that is integral over $R$ actually lies in $R$.
    \end{enumerate}
\end{definition}

\begin{definition} \label{definition:regular_at_a_point_for_a_scheme_regular_scheme_nonsingular_scheme}
Let $X$ be a \CrefAndHyperrefIfExist{definition:locally_noetherian_and_noetherian_scheme}{locally Noetherian scheme}.
\begin{itemize}
    \item The scheme $X$ is \hldef{regular at a point $x \in X$} (or \hldef{nonsingular at a point $x \in X$}) if the \CrefAndHyperrefIfExist{definition:locally_ringed_space_on_a_topological_space}{local ring} \CrefAndHyperrefIfExist{definition:stalk_of_a_presheaf_on_a_topological_space_at_a_point}{$\mathcal{O}_{X,x}$} is a \CrefAndHyperrefIfExist{definition:embedding_dimension_of_a_noetherian_local_ring_and_a_regular_local_ring}{regular local ring}. Otherwise, $X$ is said to be \hldef{singular at $x$}, or synonymously at $x$ is a \hldef{singularity of $X$}. 

    \item The scheme $X$ is called a \hldef{regular scheme} (or \hldef{nonsingular scheme}) if it is regular at every point $x \in X$. Otherwise, $X$ is said to be \hldef{singular}.
\end{itemize}
\end{definition}

\begin{definition} \label{definition:finite_type_morphism_of_schemes}
Let $f : X \to Y$ be a \CrefAndHyperrefIfExist{definition:morphism_of_schemes}{morphism of schemes}. We say that $f$ is a \hldef{finite type morphism} if for every \CrefAndHyperrefIfExist{definition:affine_open_subscheme_of_a_scheme}{affine open} $V = \operatorname{Spec} B \subseteq Y$ with $U = f^{-1}(V)$ affine, say $U = \operatorname{Spec} A$, the ring $A$ is a \CrefAndHyperrefIfExist{definition:finitely_generated_algebra_over_a_not_necessarily_commutative_ring}{finitely generated $B$-algebra}.

When $X$ is equipped with a finite type morphism $f: X \to Y$, we say that $X$ is a \hldef{finite type scheme over $Y$} or a \hldef{finite type $Y$-scheme} or a \hldef{$Y$-scheme of finite type} \CrefIfExists{definition:scheme_over_a_scheme}, etc.
\end{definition}

\begin{definition}[Dimension of a Scheme] \label{definition:dimension_of_a_scheme}
Let $X$ be a scheme with underlying topological space $|X|$.

    \TODO{krull dimension}
\begin{itemize}
    \item The \hldef{dimension at a point $x \in |X|$}, denoted $\dim_x(X)$, is the Krull dimension of the \CrefAndHyperrefIfExist{definition:locally_ringed_space_on_a_topological_space}{local ring} \CrefAndHyperrefIfExist{definition:stalk_of_a_presheaf_on_a_topological_space_at_a_point}{$\mathcal{O}_{X,x}$}. This is the supremum of the lengths $n$ of chains of prime ideals
    $$ \mathfrak{p}_0 \subsetneq \mathfrak{p}_1 \subsetneq \cdots \subsetneq \mathfrak{p}_n \subseteq \mathcal{O}_{X,x}.  $$

    \item The \hldef{dimension of the scheme $X$} is defined as
    $$ \hlin{\dim(X) := \sup_{x \in |X|} \dim_x(X). }$$
    Equivalently, it is the supremum of the lengths of chains of distinct irreducible closed subsets of $|X|$ ordered by inclusion.
\end{itemize}
\end{definition}

\begin{definition} \label{definition:local_ring}
Let $R$ be a \CrefAndHyperrefIfExist{definition:ring}{ring} with unity, not necessarily commutative.  
The ring $R$ is called a \hldef{local ring} if it has a unique \CrefAndHyperrefIfExist{definition:prime_and_maximal_ideal_of_a_ring}{maximal left ideal}. In this case, $R$ also has a unique maximal right ideal, and these coincide with the \CrefAndHyperrefIfExist{definition:jacobson_radical_of_a_ring}{Jacobson radical $J(R)$ of $R$}.  
The unique maximal left (and right) ideal of a local ring $R$ may sometimes be denoted by \hl{$\mathfrak{m}_R$}.
\end{definition}


\begin{definition}[Presheaf on a topological space] \label{definition:presheaf_on_a_topological_space}
    Let $X$ be a \CrefAndHyperrefIfExist{definition:topological_space}{topological space}. Let $\calD$ be a category.
    
    A \hldef{presheaf (of objects of $\calD$/valued in $\calD$) on $X$} is a rule $\mathcal{F}$ that assigns:
    \begin{itemize}
        \item to each open set $U \subseteq X$, an object $\mathcal{F}(U) \in \Ob \calD$, called the \hldef{sections of $\mathcal{F}$ over $U$},
        \item to each inclusion of open sets $V \subseteq U$, a morphism
        $$\rho^U_V: \mathcal{F}(U) \to \mathcal{F}(V), \quad s \mapsto s|_V,$$
    \end{itemize}
        in the category $\calD$ called the \hldef{restriction map} such that the following conditions hold:
    \begin{itemize}
        \item (Identity) For every open set $U \subseteq X$, the restriction map $\rho^U_U$ is the identity on $\mathcal{F}(U)$.
        \item (Transitivity) For inclusions $W \subseteq V \subseteq U$ of open sets, one has
        $$\rho^U_W = \rho^V_W \circ \rho^U_V.$$
    \end{itemize}
    For instance, we may speak of a \hldef{presheaf of sets/groups/rings/etc. on the topological space $X$}.


    Equivalently, a presheaf on $X$ (of objects in a category $\calD$) is a \CrefAndHyperrefIfExist{definition:functor_between_categories}{functor}
    $$\mathbf{Open}(X)^{\op} \to \calD$$
    from the opposite of the category \CrefAndHyperrefIfExist{definition:category_of_opens_of_a_topological_space}{$\mathbf{Open}(X)$} of open subsets of $X$\TextIfExists{definition:presheaf_on_a_category}{ (see also \Cref{definition:presheaf_on_a_category})}.

    \TextIfExists{definition:presheaf_on_a_category}{Equivalently, a presheaf on $X$ is a presheaf on the category $\mathbf{Open}(X)$ in the sense of \Cref{definition:presheaf_on_a_category}}.


    The sections object $\calF(U)$ is also denoted by \hl{$\Gamma(U, \calF)$}\TextIfExists{definition:sections_of_a_presheaf_on_a_category_valued_in_a_category}{ (see \Cref{definition:sections_of_a_presheaf_on_a_category_valued_in_a_category})}.
    Moreover, the object $\calF(X) = \Gamma(X, \calF)$ is called the \hldef{global sections object of $\calF$}. \TextIfExists{definition:sections_of_a_presheaf_on_a_category_valued_in_a_category}{This agrees with the notion of global sections as discussed in \Cref{definition:sections_of_a_presheaf_on_a_category_valued_in_a_category}.}

\end{definition}


\begin{definition}[Stalk of a sheaf] \label{definition:stalk_of_a_presheaf_on_a_topological_space_at_a_point}
    Let $X$ be a topological space, and let $\mathcal{D}$ be a \CrefAndHyperrefIfExist{definition:category}{category}
    % admitting \CrefAndHyperrefIfExist{definition:projective_and_inductive_limits_in_categories}{direct colimits} (e.g. the category of sets, groups, abelian groups, modules over rings, or vector spaces over fields).
    Let $\mathcal{F}$ be a \CrefAndHyperrefIfExist{definition:presheaf_on_a_topological_space}{presheaf on $X$ valued in $\calD$}.  
    For a point $x \in X$, the \hldef{stalk of $\mathcal{F}$} at $x$, denoted \hl{$\mathcal{F}_x$}, is defined as the \CrefAndHyperrefIfExist{definition:projective_and_inductive_limits_in_categories}{direct limit}
    $$\mathcal{F}_x := \varinjlim_{x \in U} \mathcal{F}(U),$$
    where the limit ranges over all open neighborhoods $U$ of $x$ in $X$ ordered by inclusion, assuming that such a direct limit exists. 

    If $\calD$ is some kind of category of sets (e.g. $\calD = \Sets$, $\mathbf{Grps}$, $\mathbf{Rings}$), then an element of $\mathcal{F}_x$ is called a \hldef{germ of a section at $x$}. Concretely, a germ at $x$ is given by a pair $(U,s)$ with $U$ an open neighborhood of $x$ and $s \in \mathcal{F}(U)$, modulo the equivalence relation: $(U,s) \sim (V,t)$ if there exists an open neighborhood $W \subseteq U \cap V$ of $x$ such that $s|_W = t|_W$.

    If $\mathcal{F}$ is a sheaf of groups, rings, or modules, then each stalk $\mathcal{F}_x$ inherits the corresponding algebraic structure.
\end{definition}
