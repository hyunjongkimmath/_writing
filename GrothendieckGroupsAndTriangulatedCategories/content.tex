
\section{Grothendieck groups of general triangulated categories}

\cite[Expos\'e VIII, 2]{SGA5} and \cite[Expos\'e I, 6.3 and Expos\'e IV, 1.1]{SGA6} each define additive functions on and Grothendieck groups of triangulated categories.

\begin{definition}
    Let $D$ be a triangulated category, and let $G$ be an abelian group. A function $\Ob{D} \to G$ is said to be \hldef{additive} if for all $X,Y,Z \in \Ob{D}$ such that there is a distinguished triangle $X \to Y \to Z \to X[1]$, we have 
    $$f(Y) = f(X) + f(Z).$$
    Write \hl{$\Add(D,F)$} for the abelian group of additive functions $\Ob{D} \to G$. 
\end{definition}

\begin{claim}
    The functor $\Add(D, \cdot): \Ab \to \Ab$ is representable by an abelian group.  
\end{claim}

\begin{definition}
    Let $D$ be a triangulated category. The \hldef{Grothendieck group} \hl{$k(D)$} is the abelian category representing the functor $\Add(D, \cdot): \Ab \to \Ab$. It may be constructed as the quotient of the free abelian group generated by $\Ob(D)$ by the relations $Y = X + Z$ whenever there is a distinguished triangle $X \to Y \to Z \to X[1]$. 
    There is a universal additive function \hl{$\operatorname{cl}_D: \Ob(D) \to k(D)$}. When there is no confusion, we may also denote it by \hl{$\operatorname{cl}$}. Given an object $X \in \Ob(D)$, denote by \hl{$[X]$} the elemtn $\operatorname{cl}(X)$.

\end{definition}

\begin{lemma}[{\cite[Expos\'e VIII 2]{SGA5}}]
    Let $D$ be a triangulated category, let $G$ be an abelian group, and let $f: D \to G$ be an additive function. We have the following:
    \begin{enumerate}
        \item $f(0) = 0$.
        \item $f(X[n]) = (-1)^f(X)$.
        \item $f(X) = f(Y)$ if $X$ and $Y$ are isomorphic.
        \item $f(X \oplus Y) = f(X) + f(Y)$.
    \end{enumerate}
\end{lemma}
\begin{proof}
These hold because the following triangles are distinguished:
\begin{align*}
    X &\xrightarrow{\id_X} X \to 0 \to X[1] \\
    X &\to 0 \to X[1] \to X[1] \\
    X &\xrightarrow{\cong} Y \to 0 \to X[1] \\
    X &\oplus Y \to X \to Y[1] \to (X \oplus Y)[1].
\end{align*}
\end{proof}

\begin{definition} \label{definition:exact_functor_between_triangulated_categories}
    A functor $T: D_1 \to D_2$ between triangulated categories is called \hldef{exact} if it is additive, is  translation preserving, and transforms distinguished triangles to distinguished triangles. Oftentimes, a \hldef{morphism between triangulated categories} refers to an exact functor between triangulated categories.
\end{definition}

\begin{proposition}[{\cite[Expos\'e VIII, 3]{SGA5}}]
    Let $D_1$ and $D_2$ be two triangulated categories and let $T: D_1 \to D_2$ be an exact functor. There is a unique group homomorphism \hl{$k(T): k(D_1) \to k(D_2)$} between the Grothendieck groups of $D_1$ and $D_2$ such that 
    \begin{center}
    \begin{tikzcd}
    \Ob D_1 \ar[r, "T"] \ar[d, "\operatorname{cl}_{C_1}"] & \Ob D_2 \ar[d, "\operatorname{cl}_{C_2}"] \\
    k(C_1) \ar[r, "k(T)"] & k(C_2) 
    \end{tikzcd}
    \end{center}
    is commutative. Explicitly, $k(T)$ can be defined by $[X] \mapsto [TX]$ and extended linearly.
\end{proposition}
\begin{proof}
    All that needs to be proven is that the proposed definition for $k(T)$ is well defined. In other words, we need to show that $k(T)$ sends relations on $k(C_1)$ to relations on $k(C_2)$. Suppose that there is a distinguished triangle $X \to Y \to Z \to X[1]$ in $D_1$ or equvialently that 
    $$[Y] = [X] + [Z]$$
    in $k(C_1)$. Since $T$ is assumed to be exact, there is a distinguished triangle $TX \to TY \to TZ \to TX[1]$ in $D_2$ and hence 
    $$[TY] = [TX] + [TZ]$$
    in $k(C_2)$.
\end{proof}

\section{Grothendieck groups of derived categories of abelian categories}

\begin{notation}
Given an additive category $A$, write \hl{$K^b(A)$} for the bounded chain homotopy category of $A$; it is a triangulated category. Given an abelian category $A$, write \hl{$D^b(A)$} for the bounded derived category of $A$; it is also a triangulated category. 

\end{notation}

\begin{lemma}[{\cite[Expos\'e IV, Lemme 1.4]{SGA6}}]
Let $A$ be an additive (resp. abelian) category. Let $f$ be an additive function on $\Ob K^b(A)$ (resp. $\Ob D^b(A)$). For any $E \in \Ob K^b(A)$ (resp. $\in \Ob D^b(A)$), we have 
\begin{align*}
f(E) &= \sum_i (-1)^i f(E^i) \\
f(E) &= \sum_i (-1)^i f(H^i(E)).
\end{align*}
\end{lemma}



