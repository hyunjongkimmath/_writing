
\section{Basic category theory}

\begin{definition}[Category] \label{definition:category}
    A 
    \defin{category}{category}{
        name={Category},
        description={A nice enough collection of objects and morphisms (\Cref{definition:category})},
    }
    \hldef{category} $\mathcal{C}$ consists of the following data:
    \begin{itemize}
        \item A class of \defin{objects}{object_of_a_category}{
            name={Object of a category},
            description={\Cref{definition:category}},
        }
        denoted \notat{\operatorname{Ob}(\mathcal{C})}{class_of_objects_of_a_category}{
            name={$\operatorname{Ob}(\mathcal{C})$},
            description={Class of objects of a category $\calC$ \Cref{definition:category}},
            sort={Ob},
        }.
        % \hl{$\operatorname{Ob}(\mathcal{C})$}.
        \item For each pair of objects $X, Y \in \operatorname{Ob}(\mathcal{C})$, a class
        \notatin{\operatorname{Hom}_{\mathcal{C}}(X,Y)}{class_of_morphisms_between_two_objects_of_a_category}
        {
            name={$\operatorname{Hom}_{\mathcal{C}}(X,Y)$},
            description={Class of morphisms between objects $X$ and $Y$ of the category $\calC$ (\Cref{definition:category})},
            sort={Hom},
        }
        % $$\hlin{\operatorname{Hom}_{\mathcal{C}}(X,Y)}$$
        of \defin{morphisms}{morphism_between_objects_of_a_category}{
            name={Morphism between objects of a category},
            description={(\Cref{definition:category})},
        }
        (also called 
        \defin{arrows}{arrow_between_objects_of_a_category}{
            name={Arrow between objects of a category},
            description={Synonym for morphism (\Cref{definition:category})},
        }
        or
        \defin{homs}{hom_between_objects_of_a_category}{
            name={Hom between objects of a category},
            description={Synonym for morphism (\Cref{definition:category})},
        }). If the category $\calC$ is clear, then this \hldef{hom-class} is also denoted by \hl{$\operatorname{Hom}(X,Y)$}. It may also be denoted by \hl{$\operatorname{hom}_{\mathcal{C}}(X,Y)$} or \hl{$\operatorname{hom}(X,Y)$}, especially to distinguish from other types of hom's (e.g. \hyperrefIfExists{definition:internal_hom_object_in_a_category}{internal hom's})
        \item For each triple of objects $X,Y,Z$, a composition law
        $$ \circ : \operatorname{Hom}_{\mathcal{C}}(Y,Z) \times \operatorname{Hom}_{\mathcal{C}}(X,Y) \to \operatorname{Hom}_{\mathcal{C}}(X,Z), $$
        denoted \hl{$(g,f) \mapsto g \circ f$}.
        \item For each object $X$, an \hldef{identity morphism}
        $$\hlin{\operatorname{id}_X \in \operatorname{Hom}_{\mathcal{C}}(X,X).}$$
    \end{itemize}
    These data satisfy the following axioms:
    \begin{itemize}
        \item (Associativity) For all morphisms $f \in \operatorname{Hom}_{\mathcal{C}}(X,Y)$, $g \in \operatorname{Hom}_{\mathcal{C}}(Y,Z)$, and $h \in \operatorname{Hom}_{\mathcal{C}}(Z,W)$, 
        $$
        h \circ (g \circ f) = (h \circ g) \circ f.
        $$
        \item (Identity) For all $f \in \operatorname{Hom}_{\mathcal{C}}(X,Y)$,
        $$
        \operatorname{id}_Y \circ f = f = f \circ \operatorname{id}_X.
        $$
    \end{itemize}
    One often writes \hl{$X \in \calC$} synonymously with $X \in \Ob(\calC)$, i.e. to denote that $X$ is an object of of $\calC$. 

    We may call a category as above an \hldef{ordinary category} to distinguish this notion from the notions of \hyperrefIfExists{definition:category_enriched_in_a_monoidal_category}{\emph{categories enriched in monoidal categories}} or higher/$n$-categories.
    \TODO{TODO: define $n$-categories}

    A category as defined above may be called called a \hldef{large category} or a \hldef{class category} to emphasize that the hom-classes may be proper classes rather than sets (note, however, that the possibility that hom-classes are sets is not excluded for large categories). Accordingly, a \hldef{category} may often refer to a \hyperrefIfExists{definition:locally_small_category}{locally small category}\CrefIfExists{definition:locally_small_category}, which is a category whose hom-classes are all sets.
\end{definition}

% Later on, we refer to the \gls{category} again.

\begin{definition} \label{definition:examples_of_common_categories}
We introduce the following common examples of categories, together with some frequently used notational variants:

\begin{itemize}
    \item The \hldef{category of sets}\CrefIfExists{definition:category_of_sets}, denoted \hl{$\mathbf{Set}$}, also written \hl{$\mathrm{Set}$}, \hl{$\mathbf{Sets}$}, or \hl{$(\mathbf{Set})$}, has objects all sets and morphisms all set-theoretic functions.
    \item The \hldef{category of groups}\CrefIfExists{definition:group_homomorphism}, denoted \hl{$\mathbf{Grp}$}, also written \hl{$\mathrm{Grp}$} or \hl{$\mathbf{Groups}$}, has objects all \CrefAndHyperrefIfExist{definition:group}{groups} and morphisms all \CrefAndHyperrefIfExist{definition:group_homomorphism}{group homomorphisms}.
    \item The \hldef{category of abelian groups}, denoted \hl{$\mathbf{Ab}$}, also written \hl{$\mathrm{Ab}$} or \hl{$\mathbf{AbGrp}$}, has objects all \CrefAndHyperrefIfExist{definition:group_homomorphism}{abelian groups} and morphisms all \CrefAndHyperrefIfExist{definition:group_homomorphism}{group homomorphisms}.
    \item Given a \CrefAndHyperrefIfExist{definition:ring}{not-necessarily commutative ring with unity $R$}, the \hldef{category of (left/right/two-sided)$R$-modules}, denoted \hl{$\mathbf{Mod}_R$}, also written \hl{$R\text{-Mod}$}, has objects all (left/right/two-sided)$R$-modules and morphisms the \CrefAndHyperrefIfExist{definition:homomorphism_of_modules_over_a_ring}{$R$-module homomorphisms}.
    \item Given a field $k$, the \hldef{category of vector spaces over $k$}, denoted \hl{$\mathbf{Vect}_k$}, also written \hl{$k\text{-Vect}$} or \hl{$\mathrm{Vect}(k)$}, has objects all $k$-vector spaces and morphisms all $k$-linear maps.
    \item The \hldef{category of topological spaces}, denoted \hl{$\mathbf{Top}$}, also written \hl{$\mathrm{Top}$} or \hl{$\mathbf{TopSpaces}$}, has objects all topological spaces and morphisms all continuous maps.
    \item The \hldef{category of rings}, denoted \hl{$\mathbf{Ring}$}, also written \hl{$\mathrm{Ring}$}, has objects all rings with unity and morphisms all unital ring homomorphisms.
    \item The \hldef{category of commutative rings}, denoted \hl{$\mathbf{CRing}$}, also written \hl{$\mathrm{CRing}$} or \hl{$\mathbf{CommRing}$}, has objects all commutative rings with unity and morphisms all unital ring homomorphisms.
    \item The \hldef{category of fields}, denoted \hl{$\mathbf{Field}$}, also written \hl{$\mathrm{Field}$} or \hl{$\mathbf{Fields}$}, has objects all fields and morphisms all field homomorphisms.
    \item The \hldef{category of monoids}, denoted \hl{$\mathbf{Mon}$}, also written \hl{$\mathrm{Mon}$} or \hl{$\mathbf{Monoids}$}, has objects all monoids and morphisms all monoid homomorphisms.
    \item The \hldef{category of semigroups}, denoted \hl{$\mathbf{SemiGrp}$}, also written \hl{$\mathrm{SGrp}$} or \hl{$\mathbf{Semigroups}$}, has objects all semigroups and morphisms all semigroup homomorphisms.
    \item Given a topological space $X$, the \hldef{category of sheaves of sets on $X$}, denoted \hl{$\mathbf{Sh}(X)$}, also written \hl{$\mathrm{Sh}(X)$}, has objects all sheaves of sets on $X$ and morphisms all morphisms of sheaves.
    \item The \hldef{category of small categories}, denoted \hl{$\mathbf{Cat}$}, also written \hl{$\mathrm{Cat}$} or \hl{$\mathbf{Categories}$}, has objects all \CrefAndHyperrefIfExist{definition:locally_small_category}{small categories} and morphisms all \CrefAndHyperrefIfExist{definition:functor_between_categories}{functors} between them.
    % \item Given a category $\mathcal{C}$, its \hldef{opposite category}, denoted \hl{$\mathcal{C}^{\mathrm{op}}$}, also written \hl{$\mathcal{C}^{op}$}, has objects the same as $\mathcal{C}$ but with morphisms reversed.
    \item Given categories $\mathcal{C}$ and $\mathcal{D}$, the \hldef{functor category}\CrefIfExists{definition:diagram_in_a_category_indexed_by_a_small_category}, denoted \hl{$[\mathcal{C},\mathcal{D}]$}, also written \hl{$\mathrm{Fun}(\mathcal{C},\mathcal{D})$}, has objects all functors from $\mathcal{C}$ to $\mathcal{D}$ and morphisms all natural transformations between them.
\end{itemize}
\end{definition}
\begin{definition} \label{definition:category_of_sets}
    The category of sets is the \CrefAndHyperrefIfExist{definition:locally_small_category}{(locally small)} \CrefAndHyperrefIfExist{definition:category}{category} 
    \begin{itemize}
        \item whose objects are \CrefAndHyperrefIfExist{definition:zermelo_fraenkel_set_theory}{sets}, and 
        \item whose morphisms $X \to Y$ are \CrefAndHyperrefIfExist{definition:function_of_sets}{set functions} $X \to Y$. 
    \end{itemize}

    The category of sets is often denoted by notations such as \hl{$\mathrm{Set}$}, \hl{$\mathbf{Set}$}, \hl{$\mathrm{Sets}$}, \hl{$\mathbf{Sets}$}, \hl{$(\mathrm{Set})$}, \hl{$(\mathbf{Set})$}, \hl{$(\mathrm{Sets})$}, \hl{$(\mathbf{Sets})$}.
\end{definition}
\begin{definition} \label{definition:category_of_groups_of_abelian_groups}
    \begin{enumerate}
        \item The \hldef{category of groups} is the \CrefAndHyperrefIfExist{definition:locally_small_category}{locally small} \CrefAndHyperrefIfExist{definition:category}{category} whose objects are \CrefAndHyperrefIfExist{definition:group}{groups} and whose morphisms are \CrefAndHyperrefIfExist{definition:group_homomorphism}{group homomorphisms}. It is often denoted by notations such as \hl{$\mathbf{Grp}$}.

        \item The \hldef{category of abelian groups} is the \CrefAndHyperrefIfExist{definition:locally_small_category}{locally small} \CrefAndHyperrefIfExist{definition:category}{category} whose objects are \CrefAndHyperrefIfExist{definition:group}{abelian groups} and whose morphisms are \CrefAndHyperrefIfExist{definition:group_homomorphism}{group homomorphisms}. It is often denoted by notations such as \hl{$\mathbf{Ab}$}.
    \end{enumerate}
\end{definition}
\begin{definition} \label{definition:category_of_rings}
    \begin{enumerate}
        \item The \hldef{category of rings} is the \CrefAndHyperrefIfExist{definition:locally_small_category}{locally small} \CrefAndHyperrefIfExist{definition:category}{category} whose objects are \CrefAndHyperrefIfExist{definition:ring}{rings} $R$ and whose morphisms $R \to S$ are \CrefAndHyperrefIfExist{definition:ring_homomorphism}{ring homomorphisms}.
        The category of rings over $R$ is often denoted by notations such as \hl{$\mathbf{Ring}$}.

        \item The \hldef{category of commutative rings} is the \CrefAndHyperrefIfExist{definition:full_subcategory_of_a_category}{full subcategory} of $\mathbf{Ring}$ consisting of the \CrefAndHyperrefIfExist{definition:commutative_ring}{commutative rings}. It is denoted by notations such as \hl{$\mathbf{CommRing}$} or \hl{$\mathbf{CRing}$}. 
    \end{enumerate}

\end{definition}
\begin{definition} \label{definition:category_of_topological_spaces}
    The \hldef{category of topological spaces} is the \CrefAndHyperrefIfExist{definition:locally_small_category}{(locally small)} \CrefAndHyperrefIfExist{definition:category}{category}
    \begin{itemize}
        \item whose objects are \CrefAndHyperrefIfExist{definition:topological_space}{topological spaces}, and
        \item whose morphisms are \CrefAndHyperrefIfExist{definition:continuous_map_of_topological_spaces}{continuous maps}.
    \end{itemize}
    The category of topological spaces is often denoted by notations such as \hl{$\mathrm{Top}$}, \hl{$\mathbf{Top}$}, etc. 
\end{definition}

\begin{definition}[Category of opens of a topological space] \label{definition:category_of_opens_of_a_topological_space}
    Let $X$ be a \CrefAndHyperrefIfExist{definition:topological_space}{topological space}. The \hldef{category of opens of $X$}, sometimes denoted \hl{$\mathbf{Open}(X)$} (or \hl{$\mathrm{Open}(X)$} or \hl{$\mathrm{Ouv}(X)$} (for the French word ``ouvert'', meaning open), etc.), is the \CrefAndHyperrefIfExist{definition:locally_small_category}{small} \CrefAndHyperrefIfExist{definition:category}{category} defined as follows:
    \begin{itemize}
        \item The objects are the open subsets $U \subseteq X$.
        \item For two open sets $U, V \subseteq X$, the morphism set is
        $$\mathrm{Hom}_{\mathbf{Open}(X)}(U,V) =
        \begin{cases}
        \{\iota_{U,V}\}, & \text{if } U \subseteq V, \\
        \varnothing, & \text{otherwise},
        \end{cases}$$
        where $\iota_{U,V}$ denotes the inclusion morphism $U \hookrightarrow V$.
        \item Composition of morphisms is given by composition of set-theoretic inclusions, i.e.
        $$\iota_{V,W} \circ \iota_{U,V} = \iota_{U,W} \quad \text{whenever } U \subseteq V \subseteq W.$$
        \item The identity morphism on an object $U$ is the inclusion $\iota_{U,U} = \mathrm{id}_U$.
    \end{itemize}
\end{definition}




\begin{definition}[Isomorphism in a category] \label{definition:isomorphism_in_a_category}
Let $\mathcal{C}$ be a \CrefAndHyperrefIfExist{definition:category}{(large) category}, and let $x,y \in \mathrm{Ob}(\mathcal{C})$.  
A morphism $f \in \mathcal{C}(x,y)$ is called an \hldef{isomorphism} if there exists a morphism $g \in \mathcal{C}(y,x)$ such that
$$ g \circ f = 1_x \qquad \text{and} \qquad f \circ g = 1_y.  $$
In this case, $g$ is called the \hldef{inverse of $f$}, and $x$ and $y$ are said to be \hldef{isomorphic objects} in $\mathcal{C}$. It is standard to write \hl{$x \cong y$} if there exists an isomorphism $f : x \to y$.

In practice, isomorphisms in specific categories may be defined in different, yet equivalent, ways.
\end{definition}





\begin{definition}[Locally small category] \label{definition:locally_small_category}
A \hyperrefIfExists{definition:category}{(large) category}\CrefIfExists{definition:category} $\mathcal{C}$ is called a \hldef{locally small category} if for every pair of objects $X, Y \in \operatorname{Ob}(\mathcal{C})$, the collection $\operatorname{Hom}_{\mathcal{C}}(X,Y)$ of morphisms between them is a (\CrefAndHyperrefIfExist{definition:small_set}{small}) \emph{set} (as opposed to a proper class). In other words, each hom-class is a set and may even be called a \hldef{hom-set}.

In some contexts, a locally small category may simply be called a \hldef{category}, especially when genuinely large categories are not considered.

A category $\mathcal{C}$ is called a \hldef{small category} if it is a locally small category and the class $\operatorname{Ob}(\mathcal{C})$ of objects is a set.

\TextIfExists{definition:grothendieck_universe}{
Given a \hyperrefIfExists{definition:grothendieck_universe}{universe}\CrefIfExists{definition:grothendieck_universe} $U$, we can define the notion of a \hldef{$U$-locally small category} and of a \hldef{$U$-small category} similarly. More explicitly, 
\begin{enumerate}
    \item a $U$-locally small category is a category such that for every pair of objects $X, Y \in \operatorname{Ob}(\mathcal{C})$, the collection $\operatorname{Hom}_{\mathcal{C}}(X,Y)$ of morphisms between them is a $U$-set.
    \item a $U$-small category is a category such that $\operatorname{Ob}(\mathcal{C})$ is a $U$-set and for every pair of objects $X, Y \in \operatorname{Ob}(\mathcal{C})$, the collection $\operatorname{Hom}_{\mathcal{C}}(X,Y)$ of morphisms between them is a $U$-set; in particular the collection of all objects and morhpisms in a $U$-small category is a $U$-set.
\end{enumerate}
}
\end{definition}

\begin{remark}
    Many ``concrete'' categories considered in ``classical mathematics'' or outside of more ``abstract'' category theory tend to be locally small. For example, the categories of sets, groups, $R$-modules, vector spaces, topological spaces, schemes, manifolds, sheaves on ``small enough'' sites are all locally small.
\end{remark}

\begin{convention}
    In this document, care should be taken to make it clear whether a category is locally small or large. However, currently such care has not been fully taken, so most statements that just speak of ``categories'' may not be applicable to large categories.
\end{convention}

\begin{definition}[Opposite category] \label{definition:opposite_category_of_a_category}

    Let $\mathcal{C}$ be a \hyperrefIfExists{definition:category}{(large) category}\CrefIfExists{definition:category}. The \hldef{opposite category} of $\mathcal{C}$, denoted \hl{$\mathcal{C}^{\mathrm{op}}$}, is defined as follows:
    \begin{itemize}
        \item The objects of $\mathcal{C}^{\mathrm{op}}$ are the same as those of $\mathcal{C}$.
        \item For any pair of objects $X,Y \in \mathcal{C}$, the morphisms from $X$ to $Y$ in $\mathcal{C}^{\mathrm{op}}$ are given by the morphisms from $Y$ to $X$ in $\mathcal{C}$:
        \[
        \mathrm{Hom}_{\mathcal{C}^{\mathrm{op}}}(X,Y) := \mathrm{Hom}_{\mathcal{C}}(Y,X).
        \]
        \item Composition in $\mathcal{C}^{\mathrm{op}}$ is defined by reversing the order of composition in $\mathcal{C}$. That is, for morphisms $f \in \mathrm{Hom}_{\mathcal{C}^{\mathrm{op}}}(X,Y)$ and $g \in \mathrm{Hom}_{\mathcal{C}^{\mathrm{op}}}(Y,Z)$, their composition is
        \[
        g \circ_{\mathcal{C}^{\mathrm{op}}} f := f \circ_{\mathcal{C}} g.
        \]
    \end{itemize}
    Intuitively, the category $\mathcal{C}^{\mathrm{op}}$ thus "reverses" the direction of all morphisms in $\mathcal{C}$.

\end{definition}



\subsection{Functors between categories}

\begin{definition} \label{definition:functor_between_categories}
Let $\mathcal{C}$ and $\mathcal{D}$ be \CrefAndHyperrefIfExist{definition:category}{(large) categories}. 
\begin{enumerate}
  \item A \hldef{functor $F: \calC \to \calD$ (from $\mathcal{C}$ to $\mathcal{D}$)} consists of :
  \begin{itemize}
    \item For each object $X$ in $\mathcal{C}$, an object $F(X)$ in $\mathcal{D}$.
    \item For each morphism $f: X \to Y$ in $\mathcal{C}$, a morphism $F(f): F(X) \to F(Y)$ in $\mathcal{D}$,
  \end{itemize}
  such that:
  \begin{align*}
    F(\mathrm{id}_X) &= \mathrm{id}_{F(X)} \quad \text{for all objects } X \text{ in } \mathcal{C}, \\
    F(g \circ f) &= F(g) \circ F(f) \quad \text{for all } X,Y,Z \in \Ob(\calC) \text{ and all } f: X \to Y, g: Y \to Z \text{ in } \mathcal{C}.
  \end{align*}

  Functors as defined above are also referred to as \hldef{covariant functors} to distinguish them from contravariant functors

  \item A \hldef{contravariant functor from $\calC$ to $\calD$} refers to a covariant functor $F:\calC^{\op} \to \calD$. Equivalently, such a functor consists of 
  \begin{itemize}
    \item For each object $X$ in $\mathcal{C}$, an object $F(X)$ in $\mathcal{D}$.
    \item For each morphism $f: X \to Y$ in $\mathcal{C}$, a morphism $F(f): F(Y) \to F(X)$ in $\mathcal{D}$,
  \end{itemize}
  such that:
  \begin{align*}
    F(\mathrm{id}_X) &= \mathrm{id}_{F(X)} \quad \text{for all objects } X \text{ in } \mathcal{C}, \\
    F(g \circ f) &= F(f) \circ F(g) \quad \text{for all } X,Y,Z \in \Ob(\calC) \text{ and all } f: X \to Y, g: Y \to Z \text{ in } \mathcal{C}.
  \end{align*}
  \TextIfExists{definition:presheaf_on_a_category}{A synonym for a ``contravariant functor from $\calC$ to $\calD$'' is a ``\CrefAndHyperrefIfExist{definition:presheaf_on_a_category}{presheaf on $\calC$ with values in $\calD$}''.}
  
\end{enumerate}
Note that declarations such as ``Let $F: \calC^{\op} \to \calD$ be a contravariant functor'' can be common; such declarations usually mean ``Let $F$ be a contravariant functor from $\calC$ to $\calD$'' as opposed to ``Let $F$ be a contravariant functor from $\calC^{\op}$ to $\calD$''. further note that a contravariant functor from $\calC$ to $\calD$ is equivalent to a covariant functor from $\calC^{\op}$ to $\calD$.
\end{definition}



\begin{convention} \label{convention:functors_by_default_are_contravariant}
    By a ``functor'', without the qualifying adjectives of \CrefAndHyperrefIfExist{definition:functor_between_categories}{``covariant''} or \CrefAndHyperrefIfExist{definition:functor_between_categories}{``contravariant''}, we will most usually mean a covariant functor.
\end{convention}





\subsubsection{Natural transformations between functors}

\begin{definition} \label{definition:natural_transformation_between_functors_between_categories}
Let $\mathcal{C}$ and $\mathcal{D}$ be \CrefAndHyperrefIfExist{definition:category}{(large) categories}. 
Let $F, G : \mathcal{C} \to \mathcal{D}$ be \CrefAndHyperrefIfExist{definition:functor_between_categories}{functors}.

A \hldef{natural transformation $\eta$ between $F$ and $G$} is a family of morphisms $\eta_X: F(X) \to G(X)$ in $\mathcal{D}$, one for each object $X$ in $\mathcal{C}$, such that for every morphism $f: X \to Y$ in $\mathcal{C}$,
\begin{align*}
G(f) \circ \eta_X = \eta_Y \circ F(f)
\end{align*}
in $\mathcal{D}$. In other words, the following diagram commutes:
\begin{center}
\begin{tikzcd}
    F(X) \arrow[r, "F(f)"] \arrow[d, "\eta_X"']
    & F(Y) \arrow[d, "\eta_Y"] \\
    G(X) \arrow[r, "G(f)"']
    & G(Y)
\end{tikzcd}
\end{center}

We write such a natural transformation by \hl{$\eta: F \Rightarrow G$}.

If $\eta_X$ is an \CrefAndHyperrefIfExist{definition:isomorphism_in_a_category}{isomorphism} for all objects $X$ of $\calC$, then $\eta$ is said to be a \hldef{natural isomorphism}.
\end{definition}



\begin{definition} \label{definition:full_and_faithful_functor_between_locally_small_categories}

Let $\mathcal{C}$ and $\mathcal{D}$ be \CrefAndHyperrefIfExist{definition:category}{(large)) categories}. Let $F : \mathcal{C} \to \mathcal{D}$ be a \CrefAndHyperrefIfExist{definition:functor_between_categories}{functor}. 
\begin{enumerate}
    \item $F$ is called \hldef{full} if for every pair of objects $x,y \in \mathrm{Ob}(\mathcal{C})$, the induced rule/assignment/class function
    $$ F_{x,y} : \Hom_\mathcal{C}(x,y) \to \Hom_\mathcal{D}(F(x), F(y)) $$
    on Hom-collections is ``surjective'', i.e. for all morphisms $g:F(x) \to F(y)$, there exists some morphism $f: x \to y$ such that $F(f) = g$. 

    \item $F$ is called \hldef{faithful} if for every pair of objects $x,y \in \mathrm{Ob}(\mathcal{C})$, 
    the induced class function (assignment)
    $$ F_{x,y} : \mathrm{Hom}_\mathcal{C}(x,y) \to \mathrm{Hom}_\mathcal{D}(F(x), F(y)) $$
    on Hom-collections is ``injective'', i.e., for any morphisms $f_1, f_2 \in \mathrm{Hom}_\mathcal{C}(x,y)$, 
    if $F(f_1) = F(f_2)$ in $\mathrm{Hom}_\mathcal{D}(F(x), F(y))$, then $f_1 = f_2$.

    \item $F$ is called \hldef{fully faithful} if it is both full and faithful.
\end{enumerate}

\end{definition}



\begin{definition}[Image of a functor] \label{definition:image_of_a_functor_between_categories}
Let $F : \mathcal{C} \to \mathcal{D}$ be a \CrefAndHyperrefIfExist{definition:functor_between_categories}{functor} between \CrefAndHyperrefIfExist{definition:category}{(large) categories}.  
The \hldef{image of $F$} is the subcategory \hl{$\mathrm{Im}(F)$} of $\mathcal{D}$ defined by:
\begin{itemize}
    \item $\mathrm{Ob}(\mathrm{Im}(F)) = \{\, F(c) \mid c \in \mathrm{Ob}(\mathcal{C}) \,\}$,  
    the collection of objects of $\mathcal{D}$ that arise as images of objects of $\mathcal{C}$.
    \item For $F(x),F(y) \in \mathrm{Ob}(\mathrm{Im}(F))$, the hom-set is
    $$ \mathrm{Im}(F)(F(x),F(y)) = \{\, F(f) \mid f \in \mathcal{C}(x,y) \,\}.  $$
    \item Composition and identities are inherited from $\mathcal{D}$ and are well-defined by functoriality of $F$.
\end{itemize}
Thus $\mathrm{Im}(F)$ is the smallest (not necessarily \CrefAndHyperrefIfExist{definition:full_subcategory_of_a_category}{full}) subcategory of $\mathcal{D}$ containing all $F(c)$ for $c \in \mathrm{Ob}(\mathcal{C})$ and all $F(f)$ for $f$ a morphism of $\mathcal{C}$.
\end{definition}



\begin{definition}[Essential image of a functor] \label{definition:essential_image_of_a_functor_between_categories}
Let $F : \mathcal{C} \to \mathcal{D}$ be a functor between \CrefAndHyperrefIfExist{definition:category}{(large) categories}.  
The \hldef{essential image of $F$} is the \CrefAndHyperrefIfExist{definition:full_subcategory_of_a_category}{full subcategory} of $\mathcal{D}$ whose objects are those $d \in \mathrm{Ob}(\mathcal{D})$ for which there exists an object $c \in \mathrm{Ob}(\mathcal{C})$ such that
$$ F(c) \cong d.  $$
\CrefIfExists{definition:isomorphism_in_a_category} Equivalently, the essential image is given by
$$\hlin{\mathrm{EssIm}(F) = \{\, d \in \mathrm{Ob}(\mathcal{D}) \mid \exists c \in \mathrm{Ob}(\mathcal{C}), \, F(c) \cong d \,\},}$$
endowed with all morphisms $\mathcal{D}(d,d')$ between such objects.

\TextIfExists{definition:replete_subcategory_of_a_category}{Equivalently, the essential image of $F$ is the smallest \CrefAndHyperrefIfExist{definition:replete_subcategory_of_a_category}{replete} \CrefAndHyperrefIfExist{definition:full_subcategory_of_a_category}{full subcategory} of $\calD$ containing the \CrefAndHyperrefIfExist{definition:image_of_a_functor_between_categories}{image} of $F$}
\end{definition}


\begin{definition} \label{definition:essentially_surjective_functor_between_categories}
Let $F : \mathcal{C} \to \mathcal{D}$ be a functor between \CrefAndHyperrefIfExist{definition:category}{(large) categories}. It is said to be essentially surjective if its \CrefAndHyperrefIfExist{definition:essential_image_of_a_functor_between_categories}{essential image} coincides with $\calD$.
\end{definition}




\begin{definition} \label{definition:equivalence_of_categories}
An \hldef{equivalence of categories} between two \CrefAndHyperrefIfExist{definition:category}{(large) categories} $\mathcal{C}$ and $\mathcal{D}$ consists of a pair of \CrefAndHyperrefIfExist{definition:functor_between_categories}{functors}
$$F : \mathcal{C} \to \mathcal{D} \quad \text{and} \quad G : \mathcal{D} \to \mathcal{C}$$
together with \CrefAndHyperrefIfExist{definition:natural_transformation_between_functors_between_categories}{natural isomorphisms}
$$\eta : \mathrm{Id}_{\mathcal{C}} \xrightarrow{\sim} G \circ F \quad \text{and} \quad \epsilon : F \circ G \xrightarrow{\sim} \mathrm{Id}_{\mathcal{D}}.$$
\CrefIfExists{definition:identity_functor_on_a_category} Such functors $F$ and $G$ may be called \hldef{(natural) inverses of each other}.

When $\calC$ and $\calD$ are \CrefAndHyperrefIfExist{definition:locally_small_category}{locally small categories}, $F$ is an equivalence of categories if and only if $F$ is \CrefAndHyperrefIfExist{definition:full_and_faithful_functor_between_locally_small_categories}{fully faithful} and \CrefAndHyperrefIfExist{definition:essentially_surjective_functor_between_categories}{essentially surjective}
\end{definition}

\begin{definition} \label{definition:essentially_small_category}
A category $\mathcal{C}$ is called \hldef{essentially small} if it is \CrefAndHyperrefIfExist{definition:equivalence_of_categories}{equivalent} to a \CrefAndHyperrefIfExist{definition:locally_small_category}{small category}, i.e., there exists a small category $\mathcal{D}$ and an equivalence of categories
$$F : \mathcal{D} \to \mathcal{C}.$$
Note that an essentially small category is necessarily \CrefAndHyperrefIfExist{definition:locally_small_category}{locally small}.
\end{definition}

\subsubsection{Yoneda's lemma}
\begin{theorem}[Yoneda Lemma] \label{theorem:yoneda_lemma_on_a_locally_small_category}
Let $\mathcal{C}$ be a \CrefAndHyperrefIfExist{definition:locally_small_category}{locally small category}. Let $A$ be an object of $\mathcal{C}$, and let $F: \mathcal{C} \to \mathbf{Set}$ be a \CrefAndHyperrefIfExist{definition:functor_between_categories}{covariant functor} to the \CrefAndHyperrefIfExist{definition:category_of_sets}{category of sets}. Let $h^A: \mathcal{C} \to \mathbf{Set}$ denote the covariant \CrefAndHyperrefIfExist{definition:representable_functor_on_a_locally_small_category}{representable functor} defined by $h^A(X) = \operatorname{Hom}_{\mathcal{C}}(A, X)$.

There exists a bijection
$$y_{A, F} : \operatorname{Nat}(h^A, F) \xrightarrow{\cong} F(A)$$
between the set of \CrefAndHyperrefIfExist{definition:natural_transformation_between_functors_between_categories}{natural transformations} from $h^A$ to $F$ and the set $F(A)$. This bijection is given by the mapping
$$\alpha \mapsto \alpha_A(\operatorname{id}_A),$$
where $\alpha: h^A \to F$ is a natural transformation, $\alpha_A: h^A(A) \to F(A)$ is its component at $A$, and $\operatorname{id}_A \in h^A(A) = \operatorname{Hom}_{\mathcal{C}}(A, A)$ is the identity morphism.

Furthermore, this isomorphism is natural in both $A$ and $F$. Explicitly:
\begin{enumerate}
    \item For any morphism $f: A \to B$ in $\mathcal{C}$, the following diagram commutes:
    \begin{center}
    \begin{tikzcd}[row sep=large, column sep=large]
        \operatorname{Nat}(h^B, F) \arrow[r, "y_{B,F}"] \arrow[d, "-\circ h^f"'] & F(B) \arrow[d, "F(f)"] \\
        \operatorname{Nat}(h^A, F) \arrow[r, "y_{A,F}"] & F(A)
    \end{tikzcd}
    \end{center}
    where $h^f: h^B \to h^A$ is the natural transformation induced by pre-composition with $f$.

    \item For any natural transformation $\eta: F \to G$, the following diagram commutes:
    \begin{center}
    \begin{tikzcd}[row sep=large, column sep=large]
        \operatorname{Nat}(h^A, F) \arrow[r, "y_{A,F}"] \arrow[d, "\eta \circ -"'] & F(A) \arrow[d, "\eta_A"] \\
        \operatorname{Nat}(h^A, G) \arrow[r, "y_{A,G}"] & G(A)
    \end{tikzcd}
    \end{center}
\end{enumerate}
\end{theorem}
\begin{corollary}[Yoneda Embedding] \label{corollary:yoneda_embedding_on_a_locally_small_category}
Let $\mathcal{C}$ be a \CrefAndHyperrefIfExist{definition:locally_small_category}{locally small category}. The functor
$$h^\bullet: \mathcal{C}^{\operatorname{op}} \to \mathbf{Set}^{\mathcal{C}}$$
\CrefIfExists{definition:opposite_category_of_a_category} \CrefIfExists{definition:diagram_in_a_category_indexed_by_a_small_category}
defined on objects by $A \mapsto h^A = \operatorname{Hom}_{\mathcal{C}}(A, -)$\CrefIfExists{definition:representable_functor_on_a_locally_small_category} and on morphisms by $f \mapsto h^f = (-\circ f)$ is \CrefAndHyperrefIfExist{definition:full_and_faithful_functor_between_locally_small_categories}{fully faithful}. That is, for any objects $A, B$ in $\mathcal{C}$, the map
$$\operatorname{Hom}_{\mathcal{C}}(A, B) \to \operatorname{Nat}(h^B, h^A)$$
given by sending a morphism $f: A \to B$ to the \CrefAndHyperrefIfExist{definition:natural_transformation_between_functors_between_categories}{natural transformation} $h^f: h^B \to h^A$ (pre-composition by $f$) is a bijection.

Consequently, $\mathcal{C}^{\operatorname{op}}$ embeds as a \CrefAndHyperrefIfExist{definition:full_subcategory_of_a_category}{full subcategory} of the functor category $\mathbf{Set}^{\mathcal{C}}$.
\end{corollary}
\begin{theorem}[Contravariant Yoneda Lemma] \label{theorem:contravariant_yoneda_lemma_on_a_locally_small_category}
Let $\mathcal{C}$ be a \CrefAndHyperrefIfExist{definition:locally_small_category}{locally small category}. Let $A$ be an object of $\mathcal{C}$, and let $G: \mathcal{C}^{\operatorname{op}} \to \mathbf{Set}$ be a \CrefAndHyperrefIfExist{definition:functor_between_categories}{contravariant functor} (i.e. a \CrefAndHyperrefIfExist{definition:presheaf_on_a_category}{presheaf}). Let $h_A: \mathcal{C}^{\operatorname{op}} \to \mathbf{Set}$ denote the contravariant representable functor defined by $h_A(X) = \operatorname{Hom}_{\mathcal{C}}(X, A)$.

There exists a bijection natural in $A$ and $G$:
$$\operatorname{Nat}(h_A, G) \cong G(A)$$
given by $\alpha \mapsto \alpha_A(\operatorname{id}_A)$.
\end{theorem}


\subsubsection{Adjoint functors between categories}

\begin{definition} \label{definition:adjoint_functors_between_categories_unit_counit_of_adjoint_functors}
Let $\mathcal{C}$ and $\mathcal{D}$ be \CrefAndHyperrefIfExist{definition:category}{categories}. Let $F : \mathcal{C} \to \mathcal{D}$ and $G : \mathcal{D} \to \mathcal{C}$ be functors. 

An \hldef{adjunction between $F$ and $G$} consists of two \CrefAndHyperrefIfExist{definition:natural_transformation_between_functors_between_categories}{natural transformations}: $\eta : \mathrm{Id}_{\mathcal{C}} \implies GF$ (the \hldef{unit}), and  $\varepsilon : FG \implies \mathrm{Id}_{\mathcal{D}}$ (the \hldef{counit})

These must satisfy the triangle identities: For every object $X \in \mathcal{C}$ 
and $Y \in \mathcal{D}$, 
$$\varepsilon_{FX} \circ F(\eta_X) = \text{id}_{FX}$$
$$G(\varepsilon_Y) \circ \eta_{GY} = \text{id}_{GY}.$$
In diagrammatic form, the triangle identities assert that the following are commutative diagrams:
\begin{center}
\begin{tikzcd}
F(X) \arrow[r, "F(\eta_X)"] \arrow[rd, "\text{id}_{F(X)}"'] & FGF(X) \arrow[d, "\varepsilon_{F(X)}"] \\
& F(X)
\end{tikzcd}
\begin{tikzcd}
G(Y) \arrow[r, "\eta_{G(Y)}"] \arrow[rd, "\text{id}_{G(Y)}"'] & GFG(Y) \arrow[d, "G(\varepsilon_Y)"] \\
& G(Y)
\end{tikzcd}
\end{center}

We say that $F$ is a \hldef{left adjoint to $G$} and $G$ is a \hldef{right adjoint to $F$} (written \hl{$F \dashv G$}). 

% for every object $A$ in $\mathcal{C}$ and $B$ in $\mathcal{D}$ there is a \CrefAndHyperrefIfExist{definition:natural_transformation_between_functors_between_categories}{natural isomorphism}
% \begin{align*}
% \operatorname{Hom}_{\mathcal{D}}(F(A), B) \cong \operatorname{Hom}_{\mathcal{C}}(A, G(B))
% \end{align*}
% that is natural in both $A$ and $B$.


In the case that $\mathcal{C}$ and $\mathcal{D}$ are \CrefAndHyperrefIfExist{definition:locally_small_category}{locally small} categories (or $U$-locally small categories if a \CrefAndHyperrefIfExist{definition:grothendieck_universe}{universe} $U$ is available), we have an adjunction $F \dashv G$ if and only if for every object $X$ in $\mathcal{C}$ and $Y$ in $\mathcal{D}$ there is a \CrefAndHyperrefIfExist{definition:natural_transformation_between_functors_between_categories}{natural isomorphism}
\begin{align*}
\operatorname{Hom}_{\mathcal{D}}(F(X), Y) \cong \operatorname{Hom}_{\mathcal{C}}(X, G(Y))
\end{align*}
that is natural in both $X$ and $Y$. In this case, the \hldef{unit of the adjunction} is the natural transformation $\eta : \mathrm{Id}_{\mathcal{C}} \Rightarrow G F$ such that, 
\begin{enumerate}
    \item for every $X \in \calC$, the morphism $\eta_X: X \to GF(X)$ (each called a \hldef{unit morphism}) in $\calC$ is obtained as the image of $\id_{F(X)}$ via the adjoint isomorphism
    $$\Hom_\calD(F(X), F(X)) \cong \Hom_\calC(X, GF(X)). $$

    \item for every $Y \in \calD$, the morphism $\epsilon_Y: FG(Y) \to Y$ (each called a \hldef{counit morphism}) in $\calD$ is obtained as the image of $\id_{G(Y)}$ via the adjoint isomorphism 
    $$\Hom_\calC(G(Y), G(Y)) \cong \Hom_\calD(FG(Y), Y).$$

\end{enumerate}


% Let $F : \mathcal{C} \to \mathcal{D}$ and $G : \mathcal{D} \to \mathcal{C}$ be functors. 
% $F$ is a \hldef{left adjoint to $G$} and $G$ is a \hldef{right adjoint to $F$} (written \hl{$F \dashv G$}) if for every object $A$ in $\mathcal{C}$ and $B$ in $\mathcal{D}$ there is a \CrefAndHyperrefIfExist{definition:natural_transformation_between_functors_between_categories}{natural isomorphism}
% \begin{align*}
% \operatorname{Hom}_{\mathcal{D}}(F(A), B) \cong \operatorname{Hom}_{\mathcal{C}}(A, G(B))
% \end{align*}
% that is natural in both $A$ and $B$.
\end{definition}


\begin{proposition}[Factorization through the counit and the unit] \label{proposition:factorization_of_morphisms_through_counit_and_unit_morphisms_for_adjunctions_between_categories}
Let $\mathcal{C}$ and $\mathcal{D}$ be categories, and let 
$$
L : \mathcal{C} \to \mathcal{D}, 
\qquad 
R : \mathcal{D} \to \mathcal{C}
$$ 
be \CrefAndHyperrefIfExist{definition:adjoint_functors_between_categories_unit_counit_of_adjoint_functors}{adjoint functors} with $L \dashv R$. Denote by 
$$ 
\eta : \mathrm{Id}_{\mathcal{C}} \Rightarrow R \circ L, 
\qquad 
\varepsilon : L \circ R \Rightarrow \mathrm{Id}_{\mathcal{D}} 
$$
the \CrefAndHyperrefIfExist{definition:adjoint_functors_between_categories_unit_counit_of_adjoint_functors}{unit and counit of the adjunction}.

\medskip

(1) For every $X' \in \mathrm{Ob}(\mathcal{C})$ and $X \in \mathrm{Ob}(\mathcal{D})$, a morphism 
$$ 
f : L(X') \to X \quad \text{in } \mathcal{D} 
$$
factors uniquely as
$$ 
f = \varepsilon_X \circ L(g), 
$$
for some morphism $g : X' \to R(X)$ in $\mathcal{C}$.
Equivalently, the assignment $g \mapsto \varepsilon_X \circ L(g)$ gives a bijection
$$ 
\Hom_\mathcal{C}(X', R(X)) \;\cong\; \Hom_\mathcal{D}(L(X'), X), 
$$
which is precisely the adjunction isomorphism.

\medskip

(2) Dually, for every $X \in \mathrm{Ob}(\mathcal{D})$ and $X' \in \mathrm{Ob}(\mathcal{C})$, a morphism 
$$ 
h : X \to R(X') \quad \text{in } \mathcal{C}
$$
factors uniquely as
$$ 
h = R(f) \circ \eta_X, 
$$
for some morphism $f : L(X) \to X'$ in $\mathcal{D}$.
Equivalently, the assignment $f \mapsto R(f) \circ \eta_X$ gives a bijection
$$ 
\Hom_\mathcal{D}(L(X), X') \;\cong\; \Hom_\mathcal{C}(X, R(X')), 
$$
again exhibiting the adjunction isomorphism.
\end{proposition}




\subsection{Diagrams, systems, and limits in categories}

\import{../_excerpts}{excerpts_diagrams_systems_and_limits_in_categories.tex}

\begin{definition} \label{definition:left_right_exact_functor_between_categories}
Let $\mathcal{C}$ and $\mathcal{D}$ be categories.
\begin{enumerate}
    \item Assume $\mathcal{C}$ admits all finite \CrefAndHyperrefIfExist{definition:limit_and_colimit_of_a_diagram_in_a_category}{limits}. A functor $F: \mathcal{C} \to \mathcal{D}$ is called \hldef{left exact} if it preserves all finite limits. Explicitly, for every finite diagram $D: \mathcal{J} \to \mathcal{C}$, the canonical map
    $$ F(\varprojlim D) \xrightarrow{\cong} \varprojlim (F \circ D) $$
    is an isomorphism.
    
    \item Assume $\mathcal{C}$ admits all finite \CrefAndHyperrefIfExist{definition:limit_and_colimit_of_a_diagram_in_a_category}{colimits}. A functor $F: \mathcal{C} \to \mathcal{D}$ is called \hldef{right exact} if it preserves all finite colimits. Explicitly, for every finite diagram $D: \mathcal{J} \to \mathcal{C}$, the canonical map
    $$ \varinjlim (F \circ D) \xrightarrow{\cong} F(\varinjlim D) $$
    is an isomorphism.
    
    \item If $\mathcal{C}$ admits both finite limits and finite colimits, a functor $F$ is called \hldef{exact} if it is both left exact and right exact.
\end{enumerate}
\end{definition}

\begin{lemma} \label{lemma:exactness_conditions_for_functors_of_topoi}
Let $\mathcal{E}$ and $\mathcal{F}$ be \CrefAndHyperrefIfExist{definition:topos}{topoi}. 
\begin{enumerate}
    \item A functor $F: \mathcal{E} \to \mathcal{F}$ is \CrefAndHyperrefIfExist{definition:left_right_exact_functor_between_categories}{left exact} if and only if it preserves \CrefAndHyperrefIfExist{definition:initial_final_zero_objects_of_a_category}{terminal objects} and \CrefAndHyperrefIfExist{definition:cartesian_product_of_two_objects_in_a_category_over_an_object}{pullbacks}.
    \item A functor $F: \mathcal{E} \to \mathcal{F}$ is \CrefAndHyperrefIfExist{definition:left_right_exact_functor_between_categories}{right exact} if and only if it preserves \CrefAndHyperrefIfExist{lemma:initial_or_final_object_in_a_category_that_is_also_in_a_full_subcategory_is_initial_or_final_in_the_subcategory}{initial objects} and \CrefAndHyperrefIfExist{definition:cocartesian_sum_of_two_objects_in_a_category_under_an_object}{pushouts}.
\end{enumerate}
\end{lemma}


\begin{definition} \label{definition:cartesian_product_of_two_objects_in_a_category_over_an_object}
    Let $\mathcal{C}$ be a \CrefAndHyperrefIfExist{definition:category}{category}, let $Z$ be an object, and let $X, Y$ be objects of $\mathcal{C}$ \CrefAndHyperrefIfExist{definition:category_of_objects_over_under_a_fixed_object_in_a_category}{over} $Z$, i.e. morphisms $X \to Z$ and $Y \to Z$ are fixed. A \hldef{cartesian product of $X$ and $Y$ over $Z$ in $\mathcal{C}$} (or \hldef{fiber product} or \hldef{pullback diagram}) is an object, often denoted by \hl{$X \times_Z Y$}, with \hldef{projection morphisms} $X \times_Z Y \to X$ and $X \times_Z Y \to Y$ that are universal. 
    More precisely, for any object $T$ of $\mathcal{C}$ and morphisms $f_X : T \to X$, $f_Y : T \to Y$, there exists a unique morphism $u : T \to X \times_Z Y$ such that the following diagram commutes:
        \begin{center}
        \begin{tikzcd}
            T \ar[rd, dotted, "u" ] \ar[rrd, "f_X", bend left] \ar[ddr, "f_Y", bend right] & & \\
            & X \times_Z Y \ar[r] \ar[d] &  \ar[d] X \\
            & Y \ar[r] & Z
        \end{tikzcd}
        \end{center}
        Equivalently, $X \times_Z Y$ is the \CrefAndHyperrefIfExist{definition:limit_and_colimit_of_a_diagram_in_a_category}{limit} of the \CrefAndHyperrefIfExist{definition:diagram_in_a_category_indexed_by_a_small_category}{diagram}
        \begin{center}
            \begin{tikzcd}
            & X \ar[d] \\
            Y \ar[r] & Z
            \end{tikzcd}
        \end{center}
        in $\calC$. 

        The commutative diagram 
        \begin{center}
        \begin{tikzcd}
        X \times_Z Y \ar[r] \ar[d] & X \ar[d] \\
        Y \ar[r] & Z
        \end{tikzcd} 
        \end{center}
        may be referred to as a \hldef{cartesian square}.

\end{definition}
\begin{definition} \label{definition:exponential_object_in_a_category_evaluation_morphism_transpose_currying_of_a_morphism_on_a_product_object}
Let $\mathcal{C}$ be a \CrefAndHyperrefIfExist{definition:category}{category} which admits finite \CrefAndHyperrefIfExist{definition:product_and_coproduct_of_objects_in_a_category}{products}.  Let $Y, Z$ be objects of $\mathcal{C}$.
\begin{enumerate}
    \item
    An \hldef{exponential object from $Y$ to $Z$} is an object \hl{$Z^Y$} (also denoted \hl{$[Y, Z]$} or \hl{$\underline{\hom}(Y, Z)$}) equipped with a morphism
    $$ \hlin{\text{ev}: Z^Y \times Y \to Z} $$
    called the \hldef{evaluation morphism}, satisfying the following universal property:

    \item 
    For any object $X$ in $\mathcal{C}$ and any morphism $f: X \times Y \to Z$, there exists a unique morphism \hl{$\tilde{f}: X \to Z^Y$} (often called the \hldef{transpose} or \hldef{currying of $f$}) such that the following diagram commutes:
    \begin{center}
    \begin{tikzcd}
        X \times Y \ar[r, "\tilde{f} \times \operatorname{id}_Y"] \ar[rd, "f"'] & Z^Y \times Y \ar[d, "\text{ev}"] \\
        & Z
    \end{tikzcd}
    \end{center}
    Equivalently, the functor $- \times Y : \mathcal{C} \to \mathcal{C}$ has a \CrefAndHyperrefIfExist{definition:adjoint_functors_between_categories_unit_counit_of_adjoint_functors}{right adjoint}, denoted $(-)^Y : \mathcal{C} \to \mathcal{C}$.
\end{enumerate}

\end{definition}
\begin{definition} \label{definition:cartesian_closed_category}
Let $\mathcal{C}$ be a \CrefAndHyperrefIfExist{definition:category}{category}. $\mathcal{C}$ is called \hldef{cartesian closed} if it satisfies the following three conditions:
\begin{enumerate}
    \item It has a \CrefAndHyperrefIfExist{definition:initial_final_zero_objects_of_a_category}{terminal object} (empty product), often denoted \hl{$1$}.
    \item It has all binary \CrefAndHyperrefIfExist{definition:product_and_coproduct_of_objects_in_a_category}{products} (and thus all finite products). That is, for any objects $A, B \in \mathcal{C}$, the product \hl{$A \times B$} exists.
    \item It has \CrefAndHyperrefIfExist{definition:exponential_object_in_a_category}{exponential objects} for all object $Y \to Z$.
\end{enumerate}
\end{definition}
\begin{definition} \label{definition:subobject_classifier_in_a_category_with_a_final_object}
Let $\mathcal{C}$ be a category with a \CrefAndHyperrefIfExist{lemma:initial_or_final_object_in_a_category_that_is_also_in_a_full_subcategory_is_initial_or_final_in_the_subcategory}{terminal object} $1$.
A \hldef{subobject classifier} is an object \hl{$\Omega$} together with a morphism \hl{$\text{true}: 1 \to \Omega$} satisfying the following universal property:

For any \CrefAndHyperrefIfExist{definition:monomorphism_and_epimorphism_in_categories}{monomorphism} \hl{$m: U \hookrightarrow X$} in $\mathcal{C}$, there exists a unique morphism \hl{$\chi_m: X \to \Omega$} (called the \hldef{characteristic morphism of $m$}) such that the following diagram is a \CrefAndHyperrefIfExist{definition:cartesian_product_of_two_objects_in_a_category_over_an_object}{pullback square}:
\begin{center}
\begin{tikzcd}
    U \ar[r, "!"] \ar[d, hook, "m"] & 1 \ar[d, "\text{true}"] \\
    X \ar[r, "\chi_m"'] & \Omega
\end{tikzcd}
\end{center}
where $U \to 1$ is the unique morphism to the terminal object. Intuitively, $\Omega$ acts as a "truth value object" where $\chi_m$ assigns "true" to the "points" of $X$ that lie in the subobject $U$.
\end{definition}
\begin{remark}
In the category of sets ($\mathbf{Set}$), the subobject classifier is the set $\Omega = \{0, 1\}$ (often denoted $2$), and "true" is the function picking out $1$. The characteristic morphism $\chi_m$ corresponds to the characteristic function of the subset $U \subseteq X$.
\end{remark}
\begin{definition} \label{definition:cocartesian_sum_of_two_objects_in_a_category_under_an_object}
    Let $\mathcal{C}$ be a \CrefAndHyperrefIfExist{definition:category}{category}, let $Z$ be an object, and let $X, Y$ be objects of $\mathcal{C}$ \CrefAndHyperrefIfExist{definition:category_of_objects_over_under_a_fixed_object_in_a_category}{under} $Z$, i.e. morphisms $Z \to X$ and $Z \to Y$ are fixed. A \hldef{cocartesian sum of $X$ and $Y$ under $Z$ in $\mathcal{C}$} (or \hldef{fiber coproduct} or \hldef{pushout diagram}) is an object, often denoted by \hl{$X \amalg_Z Y$}, with \hldef{inclusion morphisms} $X \to X \amalg_Z Y$ and $Y \to X \amalg_Z Y$ that are universal. 
    More precisely, for any object $T$ of $\mathcal{C}$ and morphisms $j_X : X \to T$, $j_Y : Y \to T$, there exists a unique morphism $v : X \amalg_Z Y \to T$ such that the following diagram commutes:
    \begin{center}
    \begin{tikzcd}
        Z \ar[r] \ar[d] 
        & X \ar[d] \ar[rdd, "j_X", bend left=20] 
        & \\
        Y \ar[r] \ar[rrd, "j_Y"', bend right=20] 
        & X \amalg_Z Y \ar[rd, dotted, "v"] 
        & \\
        & 
        & T
    \end{tikzcd}
    \end{center}

        Equivalently, $X \amalg_Z Y$ is the \CrefAndHyperrefIfExist{definition:limit_and_colimit_of_a_diagram_in_a_category}{colimit} of the \CrefAndHyperrefIfExist{definition:diagram_in_a_category_indexed_by_a_small_category}{diagram}
        \begin{center}
            \begin{tikzcd}
            X & Z \ar[l] \ar[r] & Y
            \end{tikzcd}
        \end{center}
        in $\calC$. 

        The commutative diagram 
        \begin{center}
        \begin{tikzcd}
        Z \ar[r] \ar[d] & X \ar[d] \\
        Y \ar[r] & X \amalg_Z Y
        \end{tikzcd} 
        \end{center}
        may be referred to as a \hldef{cocartesian square}.

\end{definition}


\Cref{theorem:right_left_adjoints_commute_with_limits_colimits} states the right adjoint functors commute with limits and left adjoint functors commute with colimits.

\begin{definition} \label{definition:comparison_morphism_of_limits_and_colimits_under_functors}
    Let $\mathcal{C}$ and $\mathcal{D}$ be \CrefAndHyperrefIfExist{definition:category}{(large) categories}, $F: \mathcal{C} \to \mathcal{D}$ a \CrefAndHyperrefIfExist{definition:functor_between_categories}{functor}, and $D: \mathcal{J} \to \mathcal{C}$ a \CrefAndHyperrefIfExist{definition:diagram_in_a_category_indexed_by_a_small_category}{diagram} indexed by a small category $\mathcal{J}$.

    \begin{enumerate}
        \item Assume that the \CrefAndHyperrefIfExist{definition:limit_and_colimit_of_a_diagram_in_a_category}{colimit} of $D$ exists in $\mathcal{C}$, denoted by $\operatorname{colim}_{\mathcal{J}} D$, with universal \CrefAndHyperrefIfExist{definition:limit_and_colimit_of_a_diagram_in_a_category}{cocone} $\eta: D \Rightarrow \Delta_{\operatorname{colim} D}$. The functor $F$ maps this to a cocone $F(\eta): F \circ D \Rightarrow \Delta_{F(\operatorname{colim} D)}$ in $\mathcal{D}$.
        
        If the colimit of the composite diagram $F \circ D$ exists in $\mathcal{D}$, denoted by $\operatorname{colim}_{\mathcal{J}} (F \circ D)$, then by the universal property of the colimit, there exists a unique morphism 
        $$\phi: \operatorname{colim}_{j \in \mathcal{J}} F(D(j)) \longrightarrow F\left(\operatorname{colim}_{j \in \mathcal{J}} D(j)\right)$$
        mediating between the cocone of the composite colimit and the image cocone $F(\eta)$. This unique morphism is called the \hldef{canonical colimit comparison morphism}.
        
        \item Dually, assume that the limit of $D$ exists in $\mathcal{C}$, denoted by $\lim_{\mathcal{J}} D$, with universal cone $\varepsilon: \Delta_{\lim D} \Rightarrow D$. The functor $F$ maps this to a cone $F(\varepsilon): \Delta_{F(\lim D)} \Rightarrow F \circ D$ in $\mathcal{D}$.
        
        If the limit of the composite diagram $F \circ D$ exists in $\mathcal{D}$, denoted by $\lim_{\mathcal{J}} (F \circ D)$, then by the universal property of the limit, there exists a unique morphism 
        $$\psi: F\left(\lim_{j \in \mathcal{J}} D(j)\right) \longrightarrow \lim_{j \in \mathcal{J}} F(D(j))$$
        mediating between the image cone $F(\varepsilon)$ and the cone of the composite limit. This unique morphism is called the \hldef{canonical limit comparison morphism}:
    \end{enumerate}
\end{definition}

\begin{definition} \label{definition:functor_preserving_limit_colimit_of_a_diagram}
    Let $F: \mathcal{C} \to \mathcal{D}$ be a \CrefAndHyperrefIfExist{definition:functor_between_categories}{functor} between \CrefAndHyperrefIfExist{definition:category}{(large) categories} $\mathcal{C}$ and $\mathcal{D}$.
    
    Let $D: \mathcal{J} \to \mathcal{C}$ be a diagram in $\mathcal{C}$ such that the limit $\lim_{\mathcal{J}} D$ exists in $\mathcal{C}$, with limiting cone $\lambda: \Delta_{\lim D} \Rightarrow D$. The functor $F$ \hldef{preserves the limit of $D$} if the image cone
    $$
    F(\lambda): \Delta_{F(\lim D)} \cong F \circ \Delta_{\lim D} \Rightarrow F \circ D
    $$
    exhibits $F(\lim_{\mathcal{J}} D)$ as a limit of the composite diagram $F \circ D$ in $\mathcal{D}$. Explicitly, the \CrefAndHyperrefIfExist{definition:comparison_morphism_of_limits_and_colimits_under_functors}{canonical comparison morphism}
    $$F(\lim_{\mathcal{J}} D) \xrightarrow{\cong} \lim_{\mathcal{J}} (F \circ D)$$
    must be an isomorphism.
    
    Dually, let $D: \mathcal{J} \to \mathcal{C}$ be a diagram such that the colimit $\operatorname{colim}_{\mathcal{J}} D$ exists in $\mathcal{C}$. The functor $F$ \hldef{preserves the colimit of $D$} if the \CrefAndHyperrefIfExist{definition:comparison_morphism_of_limits_and_colimits_under_functors}{canonical comparison morphism}
    $$\operatorname{colim}_{\mathcal{J}} (F \circ D) \xrightarrow{\cong} F(\operatorname{colim}_{\mathcal{J}} D)$$
    is an isomorphism.
\end{definition}

\begin{definition} \label{definition:continuous_cocontinuous_functor_between_categories}
    Let $F: \mathcal{C} \to \mathcal{D}$ be a \CrefAndHyperrefIfExist{definition:functor_between_categories}{functor} between \CrefAndHyperrefIfExist{definition:category}{(large) categories} $\mathcal{C}$ and $\mathcal{D}$.
    \begin{itemize}
        \item The functor $F$ is called \hldef{continuous} if it preserves all small limits that exist in $\mathcal{C}$. That is, for every small category $\mathcal{J}$ and every diagram $D: \mathcal{J} \to \mathcal{C}$ having a limit in $\mathcal{C}$, $F$ \CrefAndHyperrefIfExist{definition:functor_preserving_limit_colimit_of_a_diagram}{preserves the limit} of $D$.
        \item The functor $F$ is called \hldef{cocontinuous} if it preserves all small colimits that exist in $\mathcal{C}$. That is, for every small category $\mathcal{J}$ and every diagram $D: \mathcal{J} \to \mathcal{C}$ having a colimit in $\mathcal{C}$, $F$ \CrefAndHyperrefIfExist{definition:functor_preserving_limit_colimit_of_a_diagram}{preserves the colimit} of $D$.
    \end{itemize}
\end{definition}

\begin{theorem}[Continuity of Adjoint Functors] \label{theorem:right_left_adjoints_commute_with_limits_colimits}
    Let $\mathcal{C}$ and $\mathcal{D}$ be \CrefAndHyperrefIfExist{definition:category}{(large) categories}, and let $F: \mathcal{C} \to \mathcal{D}$ and $G: \mathcal{D} \to \mathcal{C}$ be functors forming an \CrefAndHyperrefIfExist{definition:adjoint_functors_between_categories_unit_counit_of_adjoint_functors}{adjunction} $F \dashv G$, i.e. $F$ is the left adjoint and $G$ is the right adjoint. Let $\mathcal{J}$ be a small index category.

    \begin{enumerate}
        \item \textbf{Right Adjoints Preserve Limits:}
        Let $D: \mathcal{J} \to \mathcal{D}$ be a \CrefAndHyperrefIfExist{definition:diagram_in_a_category_indexed_by_a_small_category}{diagram} such that the \CrefAndHyperrefIfExist{definition:limit_and_colimit_of_a_diagram_in_a_category}{limit} $\lim_{\mathcal{J}} D$ exists in $\mathcal{D}$. Then the limit of the composite diagram $G \circ D: \mathcal{J} \to \mathcal{C}$ exists in $\mathcal{C}$, and $G$ \CrefAndHyperrefIfExist{definition:functor_preserving_limit_colimit_of_a_diagram}{preserves the limit}, i.e. the \CrefAndHyperrefIfExist{definition:comparison_morphism_of_limits_and_colimits_under_functors}{comparison morphism} induced by the universal property of the limit,
        $$
        \vartheta: G\left(\lim_{j \in \mathcal{J}} D(j)\right) \xrightarrow{\cong} \lim_{j \in \mathcal{J}} G(D(j)),
        $$
        is an isomorphism.

        \item \textbf{Left Adjoints Preserve Colimits:}
        Let $D: \mathcal{J} \to \mathcal{C}$ be a diagram such that the colimit $\operatorname{colim}_{\mathcal{J}} D$ exists in $\mathcal{C}$. Then the colimit of the composite diagram $F \circ D: \mathcal{J} \to \mathcal{D}$ exists in $\mathcal{D}$, and $F$ \CrefAndHyperrefIfExist{definition:functor_preserving_limit_colimit_of_a_diagram}{preserves the colimits}, i.e. the \CrefAndHyperrefIfExist{definition:comparison_morphism_of_limits_and_colimits_under_functors}{comparison morphism} induced by the universal property of the colimit,
        $$
        \varphi: \operatorname{colim}_{j \in \mathcal{J}} F(D(j)) \xrightarrow{\cong} F\left(\operatorname{colim}_{j \in \mathcal{J}} D(j)\right),
        $$
        is an isomorphism.
    \end{enumerate}
\end{theorem}

\Cref{theorem:colimits_commute_with_colimits_and_limits_commute_with_limits} states the colimits commutes with colimits and limits commute with limits.

\begin{theorem}[Fubini Theorem for Colimits and Limtis] \label{theorem:colimits_commute_with_colimits_and_limits_commute_with_limits}
    Let $\mathcal{C}$ be a category and let $I$ and $J$ be small categories. Let $D: I \times J \to \mathcal{C}$ be a bifunctor. 
    
    \begin{enumerate}
        \item Assume that the following \CrefAndHyperrefIfExist{definition:limit_and_colimit_of_a_diagram_in_a_category}{colimits} exist in $\calC$:
        \begin{itemize}
            \item for every object $i \in I$, $\varinjlim_{j \in J} D(i, j)$
            \item $\varinjlim_{i \in I} \left( \varinjlim_{j \in J} D(i, j) \right)$
            \item for every object $j \in J$, $\varinjlim_{i \in I} D(i, j)$
            \item $\varinjlim_{j \in J} \left( \varinjlim_{i \in I} D(i, j) \right)$.
        \end{itemize}
        Then there is a canonical isomorphism:
        $$
        \varinjlim_{i \in I} \left( \varinjlim_{j \in J} D(i, j) \right) 
        \cong \varinjlim_{(i, j) \in I \times J} D(i, j) 
        \cong \varinjlim_{j \in J} \left( \varinjlim_{i \in I} D(i, j) \right).
        $$

                \item Assume that the following \CrefAndHyperrefIfExist{definition:limit_and_colimit_of_a_diagram_in_a_category}{limits} exist in $\calC$:
        \begin{itemize}
            \item for every object $i \in I$, $\varprojlim_{j \in J} D(i, j)$
            \item $\varprojlim_{i \in I} \left( \varprojlim_{j \in J} D(i, j) \right)$
            \item for every object $j \in J$, $\varprojlim_{i \in I} D(i, j)$
            \item $\varprojlim_{j \in J} \left( \varprojlim_{i \in I} D(i, j) \right)$.
        \end{itemize}
        Then there is a canonical isomorphism:
        $$
        \varprojlim_{i \in I} \left( \varprojlim_{j \in J} D(i, j) \right) 
        \cong \varprojlim_{(i, j) \in I \times J} D(i, j) 
        \cong \varprojlim_{j \in J} \left( \varprojlim_{i \in I} D(i, j) \right).
        $$

    \end{enumerate}
\end{theorem}
\begin{remark}
    These results assert that the colimit functor $\operatorname{colim}: \mathcal{C}^{I \times J} \to \mathcal{C}$ (assuming $\mathcal{C}$ is cocomplete) is isomorphic to the composition of partial colimit functors $\operatorname{colim}_I \circ \operatorname{colim}_J$ and $\operatorname{colim}_J \circ \operatorname{colim}_I$.
\end{remark}


At times, we may want to consider the limit/colimit of a system of functors. Such limits/colimits can be computed ``pointwise'', whenever the pointwise limits/colimits exist.

\begin{proposition}[Pointwise Computation of Limits and Colimits in Functor Categories] \label{proposition:limits_and_colimits_in_functor_categories_may_be_computed_pointwise}
    Let $\mathcal{C}$ and $\mathcal{D}$ be categories. Let $\operatorname{Fun}(\mathcal{C}, \mathcal{D})$ denote the category of functors from $\mathcal{C}$ to $\mathcal{D}$ (also denoted $\mathcal{D}^{\mathcal{C}}$). Let $\mathcal{J}$ be a small category and let $F: \mathcal{J} \to \operatorname{Fun}(\mathcal{C}, \mathcal{D})$ be a diagram of functors, denoted by $j \mapsto F_j$.
    
    \begin{enumerate}
        \item \textbf{Limits are computed pointwise:}
        Suppose that for every object $C \in \mathcal{C}$, the limit of the diagram $j \mapsto F_j(C)$ exists in $\mathcal{D}$. Then the limit of the diagram $F$ exists in $\operatorname{Fun}(\mathcal{C}, \mathcal{D})$ and is computed pointwise. That is, there is an isomorphism in $\operatorname{Fun}(\mathcal{C}, \mathcal{D})$:
        \[
        \left( \lim_{j \in \mathcal{J}} F_j \right)(C) \cong \lim_{j \in \mathcal{J}} (F_j(C)).
        \]
        The action of this limit functor on a morphism $f: C \to C'$ in $\mathcal{C}$ is the unique morphism induced by the family $\{ F_j(f) \}_{j \in \mathcal{J}}$ via the universal property of limits in $\mathcal{D}$.
        
        \item \textbf{Colimits are computed pointwise:}
        Suppose that for every object $C \in \mathcal{C}$, the colimit of the diagram $j \mapsto F_j(C)$ exists in $\mathcal{D}$. Then the colimit of the diagram $F$ exists in $\operatorname{Fun}(\mathcal{C}, \mathcal{D})$ and is computed pointwise. That is, there is an isomorphism in $\operatorname{Fun}(\mathcal{C}, \mathcal{D})$:
        \[
        \left( \operatorname{colim}_{j \in \mathcal{J}} F_j \right)(C) \cong \operatorname{colim}_{j \in \mathcal{J}} (F_j(C)).
        \]
        The action of this colimit functor on a morphism $f: C \to C'$ in $\mathcal{C}$ is the unique morphism induced by the family $\{ F_j(f) \}_{j \in \mathcal{J}}$ via the universal property of colimits in $\mathcal{D}$.
    \end{enumerate}
\end{proposition}



\begin{definition} \label{definition:diagonal_functor_from_a_category_to_a_diagram_category_of_the_category}
    Let $\mathcal{C}$ and $\mathcal{J}$ be \CrefAndHyperrefIfExist{definition:category}{categories}.
    The \hldef{diagonal functor}
    $$\hlin{\Delta: \mathcal{C} \to \mathcal{C}^{\mathcal{J}}}$$
    \CrefIfExists{definition:diagram_in_a_category_indexed_by_a_small_category}
    \TODO{constant functor}
    is the functor sending an object $X \in \mathcal{C}$ to the constant functor $\Delta(X): \mathcal{J} \to \mathcal{C}$, which maps every object in $\mathcal{J}$ to $X$ and every morphism in $\mathcal{J}$ to the identity morphism $1_X$.
    For a morphism $f: X \to Y$ in $\mathcal{C}$, $\Delta(f)$ is the \CrefAndHyperrefIfExist{definition:natural_transformation_between_functors_between_categories}{natural transformation} whose component at every $j \in \mathcal{J}$ is $f$.
\end{definition}

\begin{theorem} \label{theorem:limit_and_colimit_are_left_right_adjoint_to_diagonal_functor_for_locally_small_base_and_small_index}
    Let $\mathcal{C}$ be a \CrefAndHyperrefIfExist{definition:locally_small_category}{locally small} category and $\mathcal{J}$ be a \CrefAndHyperrefIfExist{definition:locally_small_category}{small index category}.
    
    \begin{enumerate}
        \item If $\mathcal{C}$ admits all \CrefAndHyperrefIfExist{definition:limit_and_colimit_of_a_diagram_in_a_category}{colimits} of \CrefAndHyperrefIfExist{definition:diagram_in_a_category_indexed_by_a_small_category}{shape} $\mathcal{J}$, then the \CrefAndHyperrefIfExist{theorem:limits_and_colimits_as_functors_from_functor_category_to_value_category}{colimit functor}
        $$\operatorname{colim}: \mathcal{C}^{\mathcal{J}} \to \mathcal{C}$$
        is \CrefAndHyperrefIfExist{definition:adjoint_functors_between_categories_unit_counit_of_adjoint_functors}{left adjoint} to the \CrefAndHyperrefIfExist{definition:diagonal_functor}{diagonal functor} $\Delta$. That is, for any functor $F: \mathcal{J} \to \mathcal{C}$ and any object $X \in \mathcal{C}$, there is a natural bijection:
        $$\operatorname{Hom}_{\mathcal{C}}(\operatorname{colim} F, X) \cong \operatorname{Hom}_{\mathcal{C}^{\mathcal{J}}}(F, \Delta(X)).$$
        
        \item If $\mathcal{C}$ admits all limits of shape $\mathcal{J}$, then the \CrefAndHyperrefIfExist{theorem:limits_and_colimits_as_functors_from_functor_category_to_value_category}{limit functor}
        $$\lim: \mathcal{C}^{\mathcal{J}} \to \mathcal{C}$$
        is \CrefAndHyperrefIfExist{definition:adjoint_functors_between_categories_unit_counit_of_adjoint_functors}{right adjoint} to the \CrefAndHyperrefIfExist{definition:diagonal_functor_from_a_category_to_a_diagram_category_of_the_category}{diagonal functor} $\Delta$. That is, for any object $X \in \mathcal{C}$ and any functor $F: \mathcal{J} \to \mathcal{C}$, there is a natural bijection:
        $$\operatorname{Hom}_{\mathcal{C}}(X, \lim F) \cong \operatorname{Hom}_{\mathcal{C}^{\mathcal{J}}}(\Delta(X), F).$$
    \end{enumerate}
    These adjunctions characterize limits and colimits via their universal properties.
\end{theorem}



\subsubsection{Some specific kinds of limits}

\begin{definition}[Product in a category] \label{definition:product_and_coproduct_of_objects_in_a_category}
Let $\mathcal{C}$ be a category and let $\{X_i\}_{i \in I}$ be a family of objects in $\mathcal{C}$ indexed by a class $I$. 
\begin{enumerate}
    \item A \hldef{product of the family $\{X_i\}$} is an object $P$ of $\mathcal{C}$ together with a ``universal'' family of morphisms
    $$\pi_i : P \to X_i, \quad \text{for each } i \in I. $$
    More precisely, for any object $Y$ and any family of morphisms $\{f_i : Y \to X_i\}_{i \in I}$, there exists a unique morphism
    $$f : Y \to P$$
    making the following diagram commute for all $i \in I$, i.e. $\pi_i \circ f = f_i$:
    \begin{center}
    \begin{tikzcd}[row sep=large, column sep=large]
        Y \arrow[d, "\exists ! f", dashed] \arrow[dr, "f_i"] & \\
        \prod X_i \arrow[r, "\pi_i"'] & X_i
    \end{tikzcd}
    \end{center}
    Such a product is often denoted by \hl{$\prod_{i \in I} X_i$}. If $\prod_{i \in I} X_i$ exists in $\calC$, then it is unique up to unique isomorphism by the universal property described above.
    
    Equivalently, the product $\prod_{i \in I} X_i$ is the \CrefAndHyperrefIfExist{definition:limit_and_colimit_of_a_diagram_in_a_category}{limit} of the \CrefAndHyperrefIfExist{definition:diagram_in_a_category_indexed_by_a_small_category}{diagram} $I \to \calC, i \mapsto X_i$, where $I$ is made into a category whose objects are the members of $I$ and whose morphisms are just the identity morphisms.


    \item A \hldef{coproduct} (or synonymously \hldef{direct sum}) of the family $\{X_i\}$ is an object $C$ of $\mathcal{C}$ together with a ``universal'' family of morphisms
    $$\iota_i : X_i \to C, \quad \text{for each } i \in I.$$
    More precisely, for any object $Y$ and any family of morphisms $\{g_i : X_i \to Y\}_{i \in I}$, there exists a unique morphism
    $$g : C \to Y$$
    making the following diagram commute for all $i \in I$, i.e. $g \circ \iota_i = g_i$:
    \begin{center}
    \begin{tikzcd}[row sep=large, column sep=large]
        X_i \arrow[r, "\iota_i"] \arrow[dr, "g_i"'] & \coprod X_i \arrow[d, "\exists ! g", dashed] \\
        & Y
    \end{tikzcd}
    \end{center}
    Such a coproduct is often denoted by \hl{$\coprod_{i \in I} X_i$} or \hl{$\oplus_{i \in I} X_i$}. If $\coprod_{i \in I} X_i$ exists in $\calC$, then it is unique up to unique isomorphism by the universal property described above.

    Equivalently, the coproduct $\coprod_{i \in I} X_i$ is the \CrefAndHyperrefIfExist{definition:limit_and_colimit_of_a_diagram_in_a_category}{colimit} of the \CrefAndHyperrefIfExist{definition:diagram_in_a_category_indexed_by_a_small_category}{diagram} $I \to \calC, i \mapsto X_i$, where $I$ is made into a category whose objects are the members of $I$ and whose morphisms are just the identity morphisms.
\end{enumerate}
\end{definition}


\begin{definition}[Equalizer in a category] \label{definition:equalizer_and_coequalizer_of_morphisms_in_a_category}
Let $\mathcal{C}$ be a \CrefAndHyperrefIfExist{definition:category}{(large) category} and let $f, g : X \to Y$ be morphisms in $\mathcal{C}$. 
\begin{enumerate}
    \item An \hldef{equalizer of $f$ and $g$} is an object $E$ together with a morphism
    $$e : E \to X$$
    such that
    $$f \circ e = g \circ e$$
    and for any object $Z$ with morphism $z : Z \to X$ satisfying
    $$f \circ z = g \circ z,$$
    there exists a unique morphism $u : Z \to E$ making the diagram commute:
    $$e \circ u = z.$$

    \begin{center}
        \begin{tikzcd}[column sep=large, row sep=large]
        Z \arrow[d, dashed, "\exists! u"] \arrow[dr, "z"] & & \\
        E \arrow[r, "e"] & X \arrow[r, shift left, "f"] \arrow[r, shift right, "g"'] & Y
        \end{tikzcd}
    \end{center}
    If such an equalizer of $f$ and $g$ exists, then we say that the following \hldef{equalizer diagram is exact}:
    \begin{center}
    \begin{tikzcd}[column sep=large, row sep=large]
    E \arrow[r, "e"] & X \arrow[r, shift left, "f"] \arrow[r, shift right, "g"'] & Y
    \end{tikzcd}
    \end{center}

    \item A \hldef{coequalizer of $f$ and $g$} is an object $Q$ together with a morphism
    $$q : Y \to Q$$
    such that
    $$q \circ f = q \circ g$$
    and for any object $Z$ with morphism $w : Y \to Z$ satisfying
    $$w \circ f = w \circ g,$$
    there exists a unique morphism $v : Q \to Z$ making the diagram commute:
    $$v \circ q = w.$$

    \begin{center}
        \begin{tikzcd}[column sep=large]
        X \arrow[r, shift left, "f"] \arrow[r, shift right, "g"'] & Y \arrow[r, "q"] \arrow[dr, "w"'] & Q \arrow[d, dashed, "\exists! v"] \\
        & & Z
        \end{tikzcd}
    \end{center}
    If such a coequalizer of $f$ and $g$ exists, then we say that the following \hldef{coequalizer diagram is exact}:
    \begin{center}
    \begin{tikzcd}[column sep=large, row sep=large]
        X \arrow[r, shift left, "f"] \arrow[r, shift right, "g"'] & Y \arrow[r, "q"] & Q 
    \end{tikzcd}
    \end{center}



\end{enumerate}
\end{definition}

\begin{definition} \label{definition:cartesian_product_of_two_objects_in_a_category_over_an_object}
    Let $\mathcal{C}$ be a \CrefAndHyperrefIfExist{definition:category}{category}, let $Z$ be an object, and let $X, Y$ be objects of $\mathcal{C}$ \CrefAndHyperrefIfExist{definition:category_of_objects_over_under_a_fixed_object_in_a_category}{over} $Z$, i.e. morphisms $X \to Z$ and $Y \to Z$ are fixed. A \hldef{cartesian product of $X$ and $Y$ over $Z$ in $\mathcal{C}$} (or \hldef{fiber product} or \hldef{pullback diagram}) is an object, often denoted by \hl{$X \times_Z Y$}, with \hldef{projection morphisms} $X \times_Z Y \to X$ and $X \times_Z Y \to Y$ that are universal. 
    More precisely, for any object $T$ of $\mathcal{C}$ and morphisms $f_X : T \to X$, $f_Y : T \to Y$, there exists a unique morphism $u : T \to X \times_Z Y$ such that the following diagram commutes:
        \begin{center}
        \begin{tikzcd}
            T \ar[rd, dotted, "u" ] \ar[rrd, "f_X", bend left] \ar[ddr, "f_Y", bend right] & & \\
            & X \times_Z Y \ar[r] \ar[d] &  \ar[d] X \\
            & Y \ar[r] & Z
        \end{tikzcd}
        \end{center}
        Equivalently, $X \times_Z Y$ is the \CrefAndHyperrefIfExist{definition:limit_and_colimit_of_a_diagram_in_a_category}{limit} of the \CrefAndHyperrefIfExist{definition:diagram_in_a_category_indexed_by_a_small_category}{diagram}
        \begin{center}
            \begin{tikzcd}
            & X \ar[d] \\
            Y \ar[r] & Z
            \end{tikzcd}
        \end{center}
        in $\calC$. 

        The commutative diagram 
        \begin{center}
        \begin{tikzcd}
        X \times_Z Y \ar[r] \ar[d] & X \ar[d] \\
        Y \ar[r] & Z
        \end{tikzcd} 
        \end{center}
        may be referred to as a \hldef{cartesian square}.

\end{definition}

\begin{definition} \label{definition:diagonal_morphism_of_an_object_over_an_object_in_a_category}
Let $\mathcal{C}$ be a category
\begin{enumerate}
    \item Let $X$ be an object in $\calC$. If the product $X \times X$ exists, then the \hldef{diagonal morphism of $X$} is the morphism
    $$\hlin{\Delta_X: X \to X \times X}$$
    induced by the universal property of $X \times X$ for the identity morphisms $\id_X: X \to X$ from $X$ to each of the first and second factors of $X$ in $X \times X$.
    In other words, $\Delta_f$ is the unique morphism such that the following equalities hold:
    $$ p_1 \circ \Delta_X = \operatorname{id}_X,\qquad p_2 \circ \Delta_X = \operatorname{id}_X $$
    where $p_1, p_2 : X \times X \to X$ are the two canonical projections.
    \item Let $f:X \to Z$ be a morphism in $\mathcal{C}$. If the \CrefAndHyperrefIfExist{definition:cartesian_product_of_two_objects_in_a_category_over_an_object}{cartesian product} $X \times_Z X$ exists, then the \hldef{diagonal morphism of $f$} is the morphism
    $$ \hlin{\Delta_f = \Delta_X : X \to X \times_Z X} $$
    induced by the universal property of $X \times_Z X$ for the identity morphisms $\id_X: X \to X$ from $X$ to each of the first and second factors of $X$ in $X \times_Z X$. In other words, $\Delta_f$ is the unique morphism such that the following equalities hold:
    $$ p_1 \circ \Delta_f = \operatorname{id}_X,\qquad p_2 \circ \Delta_f = \operatorname{id}_X $$
    where $p_1, p_2 : X \times X \to X$ are the two canonical projections.
\end{enumerate}
\end{definition}

\begin{definition} \label{definition:pushout_of_objects_under_an_object_in_a_category}
    Let $\mathcal{C}$ be a \CrefAndHyperrefIfExist{definition:category}{category}, let $Z$ be an object, and let $X, Y$ be objects of $\mathcal{C}$ \CrefAndHyperrefIfExist{definition:category_of_objects_under_a_fixed_object_in_a_category}{under} \TODO{under category} $Z$, i.e. morphisms $f : Z \to X$ and $g : Z \to Y$ are fixed. A \hldef{pushout of $X$ and $Y$ under $Z$ in $\mathcal{C}$} is an object, often denoted by \hl{$X \amalg_Z Y$}, with \hldef{inclusion morphisms} $i_X : X \to X \amalg_Z Y$ and $i_Y : Y \to X \amalg_Z Y$ that are universal. 
    More precisely, the morphisms satisfy $i_X \circ f = i_Y \circ g$, and for any object $Q$ of $\mathcal{C}$ with morphisms $u_X : X \to Q$, $u_Y : Y \to Q$ satisfying $u_X \circ f = u_Y \circ g$, there exists a unique morphism $w : X \amalg_Z Y \to Q$ such that the following diagram commutes:
        \begin{center}
        \begin{tikzcd}
            Z \ar[r, "f"] \ar[d, "g"'] & X \ar[d, "i_X"] \ar[ddr, "u_X", bend left] & \\
            Y \ar[r, "i_Y"'] \ar[drr, "u_Y"', bend right] & X \amalg_Z Y \ar[dr, dotted, "w"] & \\
            & & Q
        \end{tikzcd}
        \end{center}
        
    Equivalently, $X \amalg_Z Y$ is the \CrefAndHyperrefIfExist{definition:colimit_of_a_diagram_in_a_category}{colimit} of the \CrefAndHyperrefIfExist{definition:diagram_in_a_category_indexed_by_a_small_category}{diagram}
        \begin{center}
            \begin{tikzcd}
            & X \\
            Z \ar[ru, "f"] \ar[rd, "g"'] & \\
            & Y
            \end{tikzcd}
        \end{center}
    in $\mathcal{C}$. 

    The commutative diagram 
    \begin{center}
    \begin{tikzcd}
    Z \ar[r, "f"] \ar[d, "g"'] & X \ar[d, "i_X"] \\
    Y \ar[r, "i_Y"'] & X \amalg_Z Y
    \end{tikzcd} 
    \end{center}
    may be referred to as a \hldef{cocartesian square}.

    Note that the notion of cocartesian squares is dual ot the notion of \CrefAndHyperrefIfExist{definition:cartesian_product_of_two_objects_in_a_category_over_an_object}{cartesian squares}.
\end{definition}



\subsection{Miscellaneous categorical constructions and definitions}

\subsubsection{Subcategory of a category}

\begin{definition}[Subcategory] \label{definition:subcategory_of_a_category}
    Let $\mathcal{C}$ be a \CrefAndHyperrefIfExist{definition:category}{(large) category}. A \hldef{subcategory} $\mathcal{D}$ of $\mathcal{C}$ consists of:
    \begin{itemize}
        \item a subclass of objects $\mathrm{Ob}(\mathcal{D}) \subseteq \mathrm{Ob}(\mathcal{C})$,
        \item for each pair of objects $X, Y \in \mathrm{Ob}(\mathcal{D})$, a subclass of morphisms
        $$\mathrm{Hom}_{\mathcal{D}}(X,Y) \subseteq \mathrm{Hom}_{\mathcal{C}}(X,Y),$$
    \end{itemize}
    such that
    \begin{itemize}
        \item for every object $X \in \mathrm{Ob}(\mathcal{D})$, the identity morphism $\mathrm{id}_X$ of $X$ in $\mathcal{C}$ lies in $\mathrm{Hom}_{\mathcal{D}}(X,X)$,
        \item the composition of morphisms in $\mathcal{D}$ is inherited from $\mathcal{C}$ and is closed in $\mathcal{D}$: for morphisms $f \in \mathrm{Hom}_{\mathcal{D}}(X,Y)$ and $g \in \mathrm{Hom}_{\mathcal{D}}(Y,Z)$, their composition $g \circ f \in \mathrm{Hom}_{\mathcal{D}}(X,Z)$.
    \end{itemize}
\end{definition}



\begin{definition}[Full subcategory] \label{definition:full_subcategory_of_a_category}
    Let $\mathcal{C}$ be a \CrefAndHyperrefIfExist{definition:category}{(large) category}. A \hldef{full subcategory} $\mathcal{D}$ of $\mathcal{C}$ is a \CrefAndHyperrefIfExist{definition:subcategory_of_a_category}{subcategory} such that for every pair of objects $X, Y \in \mathrm{Ob}(\mathcal{D})$, the morphism classes coincide:
    $$\mathrm{Hom}_{\mathcal{D}}(X,Y) = \mathrm{Hom}_{\mathcal{C}}(X,Y).$$
    In other words, a full subcategory includes all morphisms between its objects that exist in the ambient category $\mathcal{C}$.
\end{definition}


\begin{definition}[Replete subcategory] \label{definition:replete_subcategory_of_a_category}
Let $\mathcal{C}$ be a \CrefAndHyperrefIfExist{definition:category}{(large) category}.  
A \CrefAndHyperrefIfExist{definition:full_subcategory_of_a_category}{full subcategory} $\mathcal{D} \subseteq \mathcal{C}$ is called \hldef{replete} if it satisfies the following property:
\begin{itemize}
    \item For all $x \in \mathrm{Ob}(\mathcal{C})$ and $y \in \mathrm{Ob}(\mathcal{D})$, if $x \cong y$ in $\mathcal{C}$, then $x \in \mathrm{Ob}(\mathcal{D})$ as well.
\end{itemize}
Equivalently, a replete subcategory of $\mathcal{C}$ is a full subcategory closed under isomorphisms in $\mathcal{C}$.
\end{definition}



\begin{definition}[Category of objects over a fixed object] \label{definition:category_of_objects_over_under_a_fixed_object_in_a_category}
Let $\mathcal{C}$ be a \hyperrefIfExists{definition:category}{category}\CrefIfExists{definition:category} and let $X \in \operatorname{Ob}(\mathcal{C})$ be a fixed object.
\begin{enumerate}
    \item 
        The \hldef{category of objects over $X$} (or synonymously the \hldef{slice category of $X$ in $\calC$} or the \hldef{over category of $X$ in $\calC$}), commonly denoted \hl{$\mathcal{C}/X$}, \hl{$\mathcal{C}_{/X}$}, or \hl{$(\mathcal{C} \downarrow X)$} is the category defined as follows:
        \begin{itemize}
            \item An object of $\mathcal{C}/X$ is a morphism $f \colon A \to X$ in $\mathcal{C}$, where $A \in \operatorname{Ob}(\mathcal{C})$.
            \item A morphism from $f \colon A \to X$ to $g \colon B \to X$ in $\mathcal{C}/X$ is a morphism $h \colon A \to B$ in $\mathcal{C}$ such that the following diagram commutes:
            $$
            \begin{aligned}
            \xymatrix{
            A \ar[dr]_f \ar[r]^h & B \ar[d]^g \\
            & X
            }
            \end{aligned}
            $$
            i.e. such that $g \circ h = f$.
            \item The identity morphisms and composition in $\mathcal{C}/X$ are inherited from $\mathcal{C}$.
        \end{itemize}

    \item 
    The \hldef{category of objects under $X$} (or synonymously the \hldef{coslice category of $X$ in $\calC$} or the \hldef{under category of $X$ in $\calC$}), commonly denoted \hl{$X/\mathcal{C}$}, \hl{$X \backslash \calC$}, \hl{$\mathcal{C}_{X/}$}, or \hl{$(X \downarrow \calC)$}, is the category defined as follows:
    \begin{itemize}
        \item An object of $X/\mathcal{C}$ is a morphism $f \colon X \to A$ in $\mathcal{C}$, where $A \in \operatorname{Ob}(\mathcal{C})$.
        \item A morphism from $f \colon X \to A$ to $g \colon X \to B$ in $X/\mathcal{C}$ is a morphism $h \colon A \to B$ in $\mathcal{C}$ such that the following diagram commutes:
        $$
        \begin{aligned}
        \xymatrix{
        X \ar[dr]^g \ar[r]^f & A \ar[d]^h \\
        & B
        }
        \end{aligned}
        $$
        i.e. such that $h \circ f = g$.
        \item The identity morphisms and composition in $X/\mathcal{C}$ are inherited from $\mathcal{C}$.
    \end{itemize}

\end{enumerate} 
\TextIfExists{definition:comma_category_of_two_functors_to_a_category}{Both notions are special cases of \CrefAndHyperrefIfExist{definition:comma_category_of_two_functors_to_a_category}{comma categories}.}
\end{definition}

\begin{lemma} \label{lemma:slice_category_has_final_object}
    Let $\mathcal{C}$ be a \hyperrefIfExists{definition:category}{category}\CrefIfExists{definition:category} and let $X \in \operatorname{Ob}(\mathcal{C})$ be a fixed object. The \CrefAndHyperrefIfExist{definition:category_of_objects_over_under_a_fixed_object_in_a_category}{slice category $\calC / X$} has $X$ as its \CrefAndHyperrefIfExist{definition:initial_final_zero_objects_of_a_category}{final object}.
\end{lemma}
\begin{proof}
    This is clear.
\end{proof}


\begin{definition}[Reflective subcategory] \label{definition:reflective_subcategory_of_a_category}
Let $\mathcal{C}$ be a \CrefAndHyperrefIfExist{definition:category}{(large) category} and let $\mathcal{D} \subseteq \mathcal{C}$ be a \CrefAndHyperrefIfExist{definition:full_subcategory_of_a_category}{full subcategory}.  
The subcategory $\mathcal{D}$ is called \hldef{reflective} if the inclusion functor
$$ I : \mathcal{D} \hookrightarrow \mathcal{C} $$
admits a left adjoint
$$ L : \mathcal{C} \to \mathcal{D}.  $$
In this case, $L$ is called the \hldef{reflector}, and for each $c \in \mathrm{Ob}(\mathcal{C})$, the unit morphism of the adjunction
$$ \eta_c : c \to I(L(c)) $$
is called the \hldef{reflection arrow of $c$ into $\mathcal{D}$}.
\end{definition}






\TODO{describe what monomorphisms and epimorphisms are like in additive or abelian categories}

\subsubsection{Types of morphisms in a category}


\begin{definition}[Monomorphism and Epimorphism in Categories] \label{definition:monomorphism_and_epimorphism_in_categories}
Let $\mathcal{C}$ be a \CrefAndHyperrefIfExist{definition:category}{category}. For objects $A, B \in \mathcal{C}$, let $f: A \to B$ be a morphism in $\mathcal{C}$.  
\begin{itemize}
    \item The morphism $f$ is called a \hldef{monomorphism} (or a \hldef{monic morphism}) if for every object $X$ and every pair of morphisms $g_1, g_2 : X \to A$, the equality $f \circ g_1 = f \circ g_2$ implies $g_1 = g_2$.  
    \item The morphism $f$ is called an \hldef{epimorphism} (or an \hldef{epic morphism}) if for every object $Y$ and every pair of morphisms $h_1, h_2: B \to Y$, the equality $h_1 \circ f = h_2 \circ f$ implies $h_1 = h_2$.  
\end{itemize}
\end{definition}



\begin{definition} \label{definition:kernel_and_cokernel_of_a_morphism_in_a_category}
Let $\mathcal{C}$ be a \CrefAndHyperrefIfExist{definition:category}{(large)} \CrefAndHyperrefIfExist{definition:pointed_category}{pointed category}, i.e. a category with a \CrefAndHyperrefIfExist{definition:initial_final_zero_objects_of_a_category}{zero object} $0$. Let $X,Y \in \mathrm{Ob}(\mathcal{C})$ be an object and let $f: X \to Y$ be a morphism. 

\begin{enumerate}
    \item A morphism $i: K \to X$ is called the \hldef{kernel of $f$} if:
    \begin{enumerate}
        \item $f \circ i = 0$, where $0$ is the \CrefAndHyperrefIfExist{definition:zero_morphism_in_a_pointed_category}{zero morphism} $K \to Y$,
        \item for any morphism $g: Z \to X$ such that $f \circ g = 0$, there exists a unique morphism $u: Z \to K$ such that $g = i \circ u$.
    \end{enumerate}
    The kernel, if it exists, is unique up to unique \CrefAndHyperrefIfExist{definition:isomorphism_in_a_category}{isomorphism}. \hl{$\ker(f)$} denotes the object $K$ determined (up to isomorphism) by a kernel of $f$.

    \TextIfExists{definition:equalizer_and_coequalizer_of_morphisms_in_a_category}{
        Equivalently, $\ker(f)$ is the \CrefAndHyperref{definition:equalizer_and_coequalizer_of_morphisms_in_a_category}{equalizer} of $f$ and the $0$ morphism $X \to Y$.
    }

    \item a morphism $p: Y \to Q$ is called the \hldef{cokernel of $f$} if:
    \begin{enumerate}
        \item $p \circ f = 0$, where $0$ is the \CrefAndHyperrefIfExist{definition:initial_final_zero_objects_of_a_category}{zero morphism} $X \to Q$,
        \item for any morphism $g: Y \to Z$ such that $g \circ f = 0$, there exists a unique morphism $v: Q \to Z$ such that $g = v \circ p$.
    \end{enumerate}
    The cokernel, if it exists, is unique up to unique isomorphism. \hl{$\operatorname{coker}(f)$} denotes the object $Q$ determined (up to isomorphism) by a cokernel of $f$.

    \TextIfExists{definition:equalizer_and_coequalizer_of_morphisms_in_a_category}{
        Equivalently, $\coker(f)$ is the \CrefAndHyperref{definition:equalizer_and_coequalizer_of_morphisms_in_a_category}{coequalizer} of $f$ and the $0$ morphism $X \to Y$.
    }

\end{enumerate}

\end{definition}


\subsubsection{Essential image of a functor between categories}




\subsubsection{Types of categories}


\begin{definition}[Complete and Cocomplete Category] \label{definition:complete_and_cocomplete_category}
Let $\mathcal{C}$ be a \CrefAndHyperrefIfExist{definition:category}{category}.  
\begin{itemize}
    \item The category $\mathcal{C}$ is called \hldef{complete} (resp. \hldef{finitely complete}) if all \CrefAndHyperrefIfExist{definition:small_and_finite_limits_and_colimits_in_a_category}{small limits} (resp. finite limits) exist in $\mathcal{C}$; that is, for every small diagram $D : J \to \mathcal{C}$ (with $J$ a small category), the limit $\lim D$ exists and is an object of $\mathcal{C}$.
    \item The category $\mathcal{C}$ is called \hldef{cocomplete} (resp. \hldef{finitely cocomplete}) if all \CrefAndHyperrefIfExist{definition:small_and_finite_limits_and_colimits_in_a_category}{small colimits} (resp. finite colimits) exist in $\mathcal{C}$; that is, for every small diagram $D : J \to \mathcal{C}$, the colimit $\mathrm{colim}\ D$ exists and is an object of $\mathcal{C}$.
\end{itemize}
\end{definition}



\subsubsection{Categories constructed from other categories}

\begin{definition}[Ind-category] \label{definition:ind_pro_category_of_a_locally_small_category}
Let $\mathcal{C}$ be a \CrefAndHyperrefIfExist{definition:locally_small_category}{locally small category}.

\begin{enumerate}
    \item  The \hldef{Ind-category of $\mathcal{C}$}, denoted \hl{$\mathrm{Ind}(\mathcal{C})$}, is defined as follows:
    \begin{itemize}
        \item Objects of $\mathrm{Ind}(\mathcal{C})$ are formal \CrefAndHyperrefIfExist{definition:projective_and_inductive_limits_in_categories}{filtered colimits} of objects in $\mathcal{C}$. More precisely, an object is given by a \CrefAndHyperrefIfExist{definition:filtered_cofiltered_category}{filtered} small category $I$ and a functor 
        $$ X : I \to \mathcal{C}.  $$
        \item Morphisms between objects $X : I \to \mathcal{C}$ and $Y : J \to \mathcal{C}$ are defined by
        $$ \mathrm{Hom}_{\mathrm{Ind}(\mathcal{C})}(X,Y) \;:=\; \varprojlim_{i \in I} \varinjlim_{j \in J} \mathrm{Hom}_{\mathcal{C}}(X_i, Y_j), $$
        \CrefIfExists{definition:projective_and_inductive_limits_in_categories}
        where $X_i$ and $Y_j$ denote the images of $i \in I$ and $j \in J$ under $X$ and $Y$, respectively.
    \end{itemize}
    The composition of morphisms is induced naturally from composition in $\mathcal{C}$.  
    Hence, $\mathrm{Ind}(\mathcal{C})$ is the completion of $\mathcal{C}$ under filtered colimits. Objects of $\mathrm{Ind}(\mathcal{C})$ are called \hldef{Ind-objects of $\calC$}.
    
    \item 
    The \hldef{Pro-category of $\mathcal{C}$}, denoted \hl{$\mathrm{Pro}(\mathcal{C})$}, is defined as follows:
    \begin{itemize}
        \item Objects of \(\mathrm{Pro}(\mathcal{C})\) are formal \CrefAndHyperrefIfExist{definition:projective_and_inductive_limits_in_categories}{cofiltered limits} of objects in \(\mathcal{C}\). More precisely, an object is given by a \CrefAndHyperrefIfExist{definition:cofiltered_cofiltered_category}{cofiltered} small category \(I\) and a functor
        \[
        X : I \to \mathcal{C}.
        \]
        \item Morphisms between objects \(X : I \to \mathcal{C}\) and \(Y : J \to \mathcal{C}\) are defined by
        \[
        \mathrm{Hom}_{\mathrm{Pro}(\mathcal{C})}(X,Y) := \varinjlim_{j \in J} \varprojlim_{i \in I} \mathrm{Hom}_{\mathcal{C}}(X_i, Y_j),
        \]
        where \(X_i\) and \(Y_j\) denote the images of \(i \in I\) and \(j \in J\) under \(X\) and \(Y\), respectively.
    \end{itemize}
    The composition of morphisms is induced naturally from composition in \(\mathcal{C}\).

    Hence, \(\mathrm{Pro}(\mathcal{C})\) is the completion of \(\mathcal{C}\) under cofiltered limits. Objects of $\mathrm{Pro}(\mathcal{C})$ are called \hldef{Pro-objects of $\calC$}.


\end{enumerate}

    Since $\Sets$ has all limits and colimits \TODO{} and hence has all projective and inductive limits and since $\calC$ is locally small, $\mathrm{Ind}(\calC)$ and $\mathrm{Pro}(\calC)$ are locally small.

\end{definition}

\begin{definition}[Product Category of a Family of Categories] \label{definition:product_category_of_a_family_of_categories}

    Let $\{\mathcal{C}_i\}_{i \in I}$ be a family of \CrefAndHyperrefIfExist{definition:category}{(large) categories} indexed by a class $I$. The \hldef{product category of the family}, denoted
        $$\hlin{\prod_{i\in I} \mathcal{C}_i},$$
        is the very large category \TODO{define very large categories} defined as follows:
        \begin{itemize}
            \item The class of objects is
            $$\mathrm{Ob}\Big(\prod_{i\in I} \mathcal{C}_i\Big) = \prod_{i\in I} \mathrm{Ob}(\mathcal{C}_i),$$
            i.e., an object is a family $(A_i)_{i\in I}$ with $A_i \in \mathrm{Ob}(\mathcal{C}_i)$.

            \item For two objects $(A_i)_i$ and $(B_i)_i$, the morphism class is
            $$\mathrm{Hom}_{\prod_{i\in I} \mathcal{C}_i}((A_i)_i,(B_i)_i) = \prod_{i\in I} \mathrm{Hom}_{\mathcal{C}_i}(A_i,B_i).$$
            In other words, a morphism $(f_i)_i : (A_i)_i \to (B_i)_i$ consists of morphisms $f_i: A_i \to B_i$ in each $\mathcal{C}_i$.

            \item For morphisms $(f_i)_i : (A_i)_i \to (B_i)_i$ and $(g_i)_i : (B_i)_i \to (C_i)_i$, composition is defined componentwise:
            $$(g_i)_i \circ (f_i)_i = (g_i \circ_i f_i)_i.$$

            \item For each object $(A_i)_i$, the identity morphism is given by the family $$(\mathrm{id}_{A_i})_i.$$
        \end{itemize}
        If $I$ is a set, then $\prod_{i \in I} \calC_i$ is a large category. If $I$ is a set and if each $\calC_i$ is \CrefAndHyperrefIfExist{definition:locally_small_category}{locally small}, then $\prod_{i \in I} \calC_i$ is locally small. 

        In case that $I$ is finite, the notation of \hl{$\times$} may be used for product categories, e.g. \hl{$\calC_i \times \calC_j$} denotes the product of two categories $\calC_i \times \calC_j$.

        \TODO{ordinal, $U_\alpha$}
        If $\alpha$ is an ordinal such that $\calC_i$ and $I$ are $U_\alpha$-large (i.e. they live in $U_{\alpha+1}$), then $\prod_{i \in I} \calC_i$ is $U_{\alpha+1}$-large.

\end{definition}

\begin{definition}[Comma Category] \label{definition:comma_category_of_two_functors_to_a_category}
Let $\mathcal{A}$, $\mathcal{B}$, and $\mathcal{C}$ be categories, and let $S: \mathcal{A} \to \mathcal{C}$ and $T: \mathcal{B} \to \mathcal{C}$ be functors. 

The \hldef{comma category} \hl{$(S \downarrow T)$} (also denoted \hl{$(S, T)$} or \hl{$S/T$}) is defined as follows:
\begin{itemize}
    \item \textbf{Objects:} Triples $(A, B, h)$, where $A$ is an object of $\mathcal{A}$, $B$ is an object of $\mathcal{B}$, and $h: S(A) \to T(B)$ is a morphism in $\mathcal{C}$.
    \item \textbf{Morphisms:} A morphism from $(A, B, h)$ to $(A', B', h')$ consists of a pair $(f, g)$, where $f: A \to A'$ is a morphism in $\mathcal{A}$ and $g: B \to B'$ is a morphism in $\mathcal{B}$, such that the following diagram in $\mathcal{C}$ commutes:
    $$
    \begin{tikzcd}
        S(A) \arrow[r, "h"] \arrow[d, "S(f)"'] & T(B) \arrow[d, "T(g)"] \\
        S(A') \arrow[r, "h'"] & T(B')
    \end{tikzcd}
    $$
    That is, $T(g) \circ h = h' \circ S(f)$.
    \item \textbf{Composition:} Composition is defined component-wise: $(f', g') \circ (f, g) = (f' \circ f, g' \circ g)$.
\end{itemize}

The general definition of the comma category subsumes several common categorical constructions:
\begin{enumerate}
    \item \textbf{Slice Category (Over-category):} If $\mathcal{A} = \mathcal{C}$, $S = \text{Id}_{\mathcal{C}}$, $\mathcal{B} = \mathbf{1}$ (the terminal category with one object $*$), and $T: \mathbf{1} \to \mathcal{C}$ is the functor selecting an object $X \in \mathcal{C}$ (i.e., $T(*) = X$), then $(S \downarrow T)$ is the \CrefAndHyperrefIfExist{definition:category_of_objects_over_under_a_fixed_object_in_a_category}{slice category $\mathcal{C}/X$}.
    % , often denoted $(\mathcal{C} \downarrow X)$. Its objects are morphisms $f: A \to X$.
    
    \item \textbf{Coslice Category (Under-category):} If $\mathcal{A} = \mathbf{1}$, $S(*) = X$, $\mathcal{B} = \mathcal{C}$, and $T = \text{Id}_{\mathcal{C}}$, then $(S \downarrow T)$ is the \CrefAndHyperrefIfExist{definition:category_of_objects_over_under_a_fixed_object_in_a_category}{coslice category $X/\mathcal{C}$}.
    % , often denoted $(X \downarrow \mathcal{C})$. Its objects are morphisms $f: X \to B$.
    
    \item \textbf{Relative Slice Category:} If $S = u: \mathcal{D} \to \mathcal{C}$ is a functor and $T: \mathbf{1} \to \mathcal{C}$ picks out an object $U \in \mathcal{C}$, then $(S \downarrow T)$ is the \CrefAndHyperrefIfExist{definition:relative_slice_category_of_objects_over_an_object_relative_to_a_functor}{relative slice category}. 
    % Its objects are pairs $(V, \varphi)$ with $V \in \mathcal{D}$ and $\varphi: u(V) \to U$ in $\mathcal{C}$.
    
    \item \textbf{Arrow Category:} If $\mathcal{A} = \mathcal{B} = \mathcal{C}$ and $S = T = \text{Id}_{\mathcal{C}}$, then $(S \downarrow T)$ is the \CrefAndHyperrefIfExist{definition:arrow_category_of_a_category}{arrow category $\mathcal{C}^{\to}$} of $\mathcal{C}$. Its objects are all morphisms in $\mathcal{C}$.
\end{enumerate}
\end{definition}

\begin{definition}[Arrow Category] \label{definition:arrow_category_of_a_category}
Let $\mathcal{C}$ be a \CrefAndHyperrefIfExist{definition:category}{category}.
The \hldef{arrow category of $\mathcal{C}$}, denoted \hl{$\mathcal{C}^{\to}$} (or sometimes \hl{$\text{Arr}(\mathcal{C})$} or \hl{$\mathcal{C}^{\mathbf{2}}$}), is defined as follows:
\begin{itemize}
    \item \textbf{Objects:} Morphisms $f: A \to B$ in $\mathcal{C}$. (We may view such an object as a triple $(A, B, f)$).
    \item \textbf{Morphisms:} A morphism from an object $f: A \to B$ to an object $f': A' \to B'$ is a pair of morphisms $(h_0, h_1)$ in $\mathcal{C}$, where $h_0: A \to A'$ and $h_1: B \to B'$, such that the following square commutes:
    $$
    \begin{tikzcd}
        A \arrow[r, "f"] \arrow[d, "h_0"'] & B \arrow[d, "h_1"] \\
        A' \arrow[r, "f'"] & B'
    \end{tikzcd}
    $$
    That is, $h_1 \circ f = f' \circ h_0$.
    \item \textbf{Composition:} Composition is defined component-wise: $(k_0, k_1) \circ (h_0, h_1) = (k_0 \circ h_0, k_1 \circ h_1)$.
\end{itemize}
    \TextIfExists{definition:comma_category_of_two_functors_to_a_category}{Arrow categories are special cases of \CrefAndHyperrefIfExist{definition:comma_category_of_two_functors_to_a_category}{comma categories}.}
\end{definition}





\subsubsection{Types of functors}


\begin{definition}[Reflecting a type of morphism] \label{definition:reflects_a_type_of_morphism_for_a_functor_between_categories}
Let $F : \mathcal{C} \to \mathcal{D}$ be a \CrefAndHyperrefIfExist{definition:functor_between_categories}{functor between (large) categories}, and let $\mathcal{P}$ be a property of morphisms (or more generally a property of sequences or families of morphisms) that is stable under \CrefAndHyperrefIfExist{definition:isomorphism_in_a_category}{isomorphism} (e.g. \CrefAndHyperrefIfExist{definition:monomorphism_and_epimorphism_in_categories}{monomorphism, epimorphism}, isomorphism, etc.). We say that $F$ \hldef{reflects $\mathcal{P}$-morphisms} if for every morphism $f : x \to y$ in $\mathcal{C}$, whenever $F(f)$ has property $\mathcal{P}$ in $\mathcal{D}$, it follows that $f$ has property $\mathcal{P}$ in $\mathcal{C}$.
\end{definition}


\begin{proposition}[Reflection of morphism properties by full or faithful functors] \label{proposition:reflection_of_monomorphism_and_epimorphisms_by_faithful_functors}
Let $F : \mathcal{C} \to \mathcal{D}$ be a \CrefAndHyperrefIfExist{definition:functor_between_categories}{functor} between \CrefAndHyperrefIfExist{definition:locally_small_category}{locally small categories}.

(1) If $F$ is \CrefAndHyperrefIfExist{definition:full_and_faithful_functor_between_locally_small_categories}{faithful}, then $F$ \CrefAndHyperrefIfExist{definition:reflects_a_type_of_morphism_for_a_functor_between_categories}{reflects} \CrefAndHyperrefIfExist{definition:monomorphism_and_epimorphism_in_categories}{monomorphisms and epimorphisms}.  
That is, if $f : x \to y$ in $\mathcal{C}$ is such that $F(f)$ is a monomorphism (resp. epimorphism) in $\mathcal{D}$, then $f$ is a monomorphism (resp. epimorphism) in $\mathcal{C}$.

\TODO{define split monos and epis}
(2) If $F$ is fully faithful, then $F$ reflects isomorphisms, split monomorphisms, and split epimorphisms.  
That is, if $f : x \to y$ in $\mathcal{C}$ is such that $F(f)$ is an isomorphism in $\mathcal{D}$, then $f$ is an isomorphism in $\mathcal{C}$.

\end{proposition}


\begin{definition}[n-ary (Multivariable) Functor] \label{definition:n_ary_functor}
Let $I$ be a finite set with $|I| = n$, and let $\{\mathcal{C}_i\}_{i \in I}$ be \CrefAndHyperrefIfExist{definition:category}{(large) categories}, together with another category $\mathcal{D}$. An \hldef{n-ary functor} (also called a \hldef{multivariable functor}, a \hldef{multivariate functor}, or a \hldef{multifunctor} ) from the categories $\{\mathcal{C}_i\}_{i\in I}$ to $\mathcal{D}$ is a \CrefAndHyperrefIfExist{definition:functor_between_categories}{functor}
$$F : \prod_{i \in I} \mathcal{C}_i \to \mathcal{D}.$$ 
\CrefIfExists{definition:product_category_of_a_family_of_categories} That is, $F$ assigns:
\begin{itemize}
    \item to each object $((A_i)_{i \in I})$ in $\prod_{i \in I} \mathcal{C}_i$, an object $F((A_i)_{i \in I})$ in $\mathcal{D}$,
    \item to each morphism $((f_i)_{i \in I}) : (A_i)_i \to (B_i)_i$, a morphism $F((f_i)_i) : F((A_i)_i) \to F((B_i)_i)$ in $\mathcal{D}$,
\end{itemize}
so that $F$ preserves identities and composition componentwise. 
For instance, a \hldef{bifunctor} is an $n$-ary functor when $n = 2$, a \hldef{ternary functor/trifunctor} is an $n$-ary functor when $n = 3$, etc.

\end{definition}

\subsubsection{Algebraic objects in a category}

\begin{definition} \label{definition:semigroup_object_in_a_category}
    Let $\mathcal{C}$ be a \CrefAndHyperrefIfExist{definition:category}{(large) category}. 
    A \hldef{semigroup object in $\mathcal{C}$} is an object $A \in \mathcal{C}$ such that the \CrefAndHyperrefIfExist{definition:product_and_coproduct_of_objects_in_a_category}{product} $A \times A$ exists in $\calC$ together with a morphism
    $$ \mu : A \times A \to A, $$

    called the \hldef{multiplication morphism} such that the associativity diagram
    \begin{center}
    \begin{tikzcd}
    A \times A \times A
    \arrow[r, "\mu \times \mathrm{id}_A"]
    \arrow[d, "\mathrm{id}_A \times \mu"']
    & A \times A
    \arrow[d, "\mu"]
    \\
    A \times A
    \arrow[r, "\mu"']
    & A
    \end{tikzcd}
    \end{center}
    commutes.

    The semigroup object structure $(A,\mu,\eta, \iota)$ is said to be \hldef{abelian} or \hldef{commutative} if the morphisms $\mu: A \times A \to A$ and $\mu \circ \tau_{A,A}: A \times A \to A$ coincide, where $\tau_{A,A}: A \times A \to A \times A$ is the symmetry morphism swapping the two factors.
\end{definition}

\begin{definition} \label{definition:monoid_object_in_a_category_with_a_final_object}
    Let $\mathcal{C}$ be a \CrefAndHyperrefIfExist{definition:category}{(large) category} with a \CrefAndHyperrefIfExist{definition:initial_final_zero_objects_of_a_category}{final object}. 
    A \hldef{monoid object in $\mathcal{C}$} is a \CrefAndHyperrefIfExist{definition:semigroup_object_in_a_category}{semigroup  object} $(A,\mu)$ together with a \hldef{unit morphism} 
    $$
    \eta : 1 \to A
    $$
    such that the \CrefAndHyperrefIfExist{definition:product_and_coproduct_of_objects_in_a_category}{products} $1 \times A$ and $A \times 1$ exist and the unit diagrams
    $$
    \begin{tikzcd}
    1 \times A
    \arrow[rr, "\eta \times \mathrm{id}_A"]
    \arrow[dr, "\mathrm{pr}_2"']
    && A \times A
    \arrow[dl, "\mu"]
    \\
    & A &
    \end{tikzcd}
    $$

    $$
    \begin{tikzcd}
    A \times 1
    \arrow[rr, "\mathrm{id}_A \times \eta"]
    \arrow[dr, "\mathrm{pr}_1"']
    && A \times A
    \arrow[dl, "\mu"]
    \\
    & A &
    \end{tikzcd}
    $$

    commute.
\end{definition}


\begin{definition} \label{definition:group_object_in_a_category_with_a_final_object}
    Let $\mathcal{C}$ be a \CrefAndHyperrefIfExist{definition:category}{(large) category} with a \CrefAndHyperrefIfExist{definition:initial_final_zero_objects_of_a_category}{final object}. 
    A \hldef{group object in $\mathcal{C}$} is a \CrefAndHyperrefIfExist{definition:monoid_object_in_a_category_with_a_final_object}{monoid object} $(A,\mu, \eta)$ together with a \hldef{inverse morphism} 
    $$
    \iota : A \to A
    $$
    such that the diagrams
    $$
    \begin{tikzcd}
    A
    \arrow[r, "\Delta"]
    \arrow[dr, "\eta \circ {!}_A"']
    & A \times A
    \arrow[d, "\mu \circ (\mathrm{id}_A \times \iota)"]
    \\
    & A
    \end{tikzcd}
    $$

    $$
    \begin{tikzcd}
    A
    \arrow[r, "\Delta"]
    \arrow[dr, "\eta \circ {!}_A"']
    & A \times A
    \arrow[d, "\mu \circ (\iota \times \mathrm{id}_A)"]
    \\
    & A
    \end{tikzcd}
    $$

    commute, where $\Delta : A \to A \times A$ is the diagonal and ${!}_A : A \to 1$ is the unique morphism.


\end{definition}



\begin{definition} \label{definition:ring_object_in_a_category_with_a_terminal_object}
Let $\mathcal{C}$ be a \CrefAndHyperrefIfExist{definition:category}{(large) category} with a \CrefAndHyperrefIfExist{definition:initial_final_zero_objects_of_a_category}{terminal object} $1$.

A \hldef{ring object in $\mathcal{C}$} is an object $R \in \mathcal{C}$ equipped with:
  \begin{itemize}
    \item an \CrefAndHyperrefIfExist{definition:semigroup_object_in_a_category}{abelian} \CrefAndHyperrefIfExist{definition:group_object_in_a_category_with_a_final_object}{group object} structure $(R,+,0,-)$ (written additively), i.e.\ morphisms
    $$
    + : R \times R \to R, \quad 0 : 1 \to R, \quad - : R \to R,
    $$
    making $(R,+,0,-)$ a group object;
    \item a \CrefAndHyperrefIfExist{definition:monoid_object_in_a_category_with_a_final_object}{monoid object} structure $(R,\cdot,1_R)$ (written multiplicatively), i.e.\ morphisms
    $$
    \cdot : R \times R \to R, \quad 1_R : 1 \to R,
    $$
    making $(R,\cdot,1_R)$ a monoid object;
  \end{itemize}
  such that the usual distributivity and absorption axioms hold, expressed by the commutativity of the diagrams corresponding to
  $$
  a \cdot (b + c) = a \cdot b + a \cdot c,
  \qquad
  (a + b) \cdot c = a \cdot c + b \cdot c,
  $$
  and $0 \cdot a = a \cdot 0 = 0$, for all $a,b,c$ (interpreted as morphisms in $\mathcal{C}$).
\end{definition}

\begin{definition}[Module object over a ring object in a category] \label{definition:module_object_over_a_ring_object_in_a_category_with_a_final_object}
Let $\mathcal{C}$ be a \CrefAndHyperrefIfExist{definition:category}{(large) category} with a \CrefAndHyperrefIfExist{definition:initial_final_zero_objects_of_a_category}{terminal object} $1$. Let $R$ be a \CrefAndHyperrefIfExist{definition:ring_object_in_a_category_with_a_terminal_object}{ring object} in $\mathcal{C}$.
An \hldef{$R$-module object in $\mathcal{C}$} is an object $M \in \mathcal{C}$ together with:
\begin{itemize}
  \item an \CrefAndHyperrefIfExist{definition:semigroup_object_in_a_category}{abelian} \CrefAndHyperrefIfExist{definition:group_object_in_a_category_with_a_final_object}{group object} structure $(M,+_M,0_M,-_M)$ on $M$;
  \item an action morphism
  $$
  \alpha : R \times M \to M,
  $$
\end{itemize}
such that the usual module axioms hold, expressed by commutative diagrams corresponding to
$$
r \cdot (m_1 +_M m_2)
=
r \cdot m_1 +_M r \cdot m_2,
\qquad
(r_1 + r_2) \cdot m
=
r_1 \cdot m +_M r_2 \cdot m,
$$
$$
(r_1 \cdot r_2) \cdot m
=
r_1 \cdot (r_2 \cdot m),
\qquad
1_R \cdot m = m,
\qquad
0_R \cdot m = 0_M = r \cdot 0_M,
$$
for all $r,r_1,r_2 \in R$ and $m,m_1,m_2 \in M$, interpreted as morphisms and equalities in $\mathcal{C}$.
\end{definition}

\begin{definition} \label{definition:groupoid_object_in_a_category}
Let $\mathcal{C}$ be a \CrefAndHyperrefIfExist{definition:category}{(large) category}.

A \hldef{groupoid object in $\mathcal{C}$} consists of two objects $X_0$ (the "object of objects") and $X_1$ (the "object of morphisms"), together with five structure morphisms:
\begin{itemize}
    \item \hldef{Source} and \hldef{Target}: $s, t: X_1 \to X_0$, such that the \CrefAndHyperrefIfExist{definition:cartesian_product_of_two_objects_in_a_category_over_an_object}{fiber product} $X_1 \times_{s,X_0,t} X_1$ of the morphisms $s$ and $t$ exists in $\calC$,
    \item \hldef{Identity}: $e: X_0 \to X_1$,
    \item \hldef{Composition}: $m: X_1 \times_{s, X_0, t} X_1 \to X_1$,
    \item \hldef{Inverse}: $i: X_1 \to X_1$,
\end{itemize}
such that the following conditions hold (expressing the axioms of a category where every morphism is invertible):
\begin{enumerate}
    \item \textbf{Source/Target identities}:
    $$ s \circ e = \operatorname{id}_{X_0}, \quad t \circ e = \operatorname{id}_{X_0} $$
    $$ s \circ m = s \circ \pi_2, \quad t \circ m = t \circ \pi_1 $$
    \item \textbf{Associativity}: The following diagram of composition commutes:
    $$ m \circ (m \times \operatorname{id}_{X_1}) = m \circ (\operatorname{id}_{X_1} \times m) $$
    \item \textbf{Unitality}:
    $$ m \circ (e \circ s, \operatorname{id}_{X_1}) = \operatorname{id}_{X_1}, \quad m \circ (\operatorname{id}_{X_1}, e \circ t) = \operatorname{id}_{X_1} $$
    \item \textbf{Invertibility}:
    $$ m \circ (i, \operatorname{id}_{X_1}) = e \circ s, \quad m \circ (\operatorname{id}_{X_1}, i) = e \circ t $$
\end{enumerate}
\end{definition}

\TODO{}
\begin{definition}[Semigroup, monoid and group objects in an $\infty$-category]
Let $\mathcal{C}$ be an $\infty$-category with finite products.
\begin{enumerate}
  \item A \hldef{semigroup object} in $\mathcal{C}$ is a functor
  $$
  A : \mathrm{N}(\Delta_{\mathrm{inj}}^{\mathrm{op}}) \to \mathcal{C}
  $$
  from the nerve of the subcategory of $\Delta$ consisting of injective maps, satisfying the Segal condition:
  for each $[n] \in \Delta$ with $n \geq 2$, the canonical map
  $$
  A([n]) \to A([1]) \times \cdots \times A([1])
  $$
  (induced by the $n$ face maps $[1] \to [n]$) is an equivalence in $\mathcal{C}$.

  \item A \hldef{monoid object} in $\mathcal{C}$ is a functor
  $$
  A : \mathrm{N}(\Delta^{\mathrm{op}}) \to \mathcal{C}
  $$
  satisfying the Segal condition as above, together with a choice of unit encoded by the image of the unique degeneracy map $[0] \to [1]$, such that the usual associativity and unit diagrams hold up to coherent homotopy (i.e.\ $A$ is a unital Segal object in $\mathcal{C}$).

  \item A \hldef{group object} in $\mathcal{C}$ is a monoid object $A$ such that the induced monoid object $\pi_0(A)$ in the homotopy category $h\mathcal{C}$ is a group object (equivalently, the underlying $E_1$-space is grouplike).
\end{enumerate}
\end{definition}

\begin{definition}[Ring objects in an $\infty$-category]
Let $\mathcal{C}$ be an $\infty$-category with finite products.
\begin{enumerate}
  \item A \hldef{ring object} in $\mathcal{C}$ is an object $R \in \mathcal{C}$ equipped with:
  \begin{itemize}
    \item a group object structure $(R,+,0,-)$ in $\mathcal{C}$ (additive structure),
    \item a monoid object structure $(R,\cdot,1_R)$ in $\mathcal{C}$ (multiplicative structure),
  \end{itemize}
  such that the distributivity, absorption and unital axioms for a ring hold up to coherent homotopy, expressed by the homotopy-commutativity of the diagrams corresponding to
  $$
  a \cdot (b + c) \simeq a \cdot b + a \cdot c,
  \qquad
  (a + b) \cdot c \simeq a \cdot c + b \cdot c,
  $$
  and $0 \cdot a \simeq 0 \simeq a \cdot 0$, for all $a,b,c \in R$.
  \item A \hldef{commutative ring object} in $\mathcal{C}$ is a ring object whose multiplicative monoid structure is commutative up to coherent homotopy (i.e.\ a commutative monoid object in the sense above).
\end{enumerate}
\end{definition}

\begin{definition}[Module objects over a ring object in an $\infty$-category]
Let $\mathcal{C}$ be an $\infty$-category with finite products and let $R$ be a ring object in $\mathcal{C}$.
An \hldef{$R$-module object in $\mathcal{C}$} is an object $M \in \mathcal{C}$ together with
\begin{itemize}
  \item a group object structure $(M,+_M,0_M,-_M)$ in $\mathcal{C}$,
  \item an action morphism in $\mathcal{C}$
  $$
  \alpha : R \times M \to M,
  $$
\end{itemize}
such that the usual module axioms hold up to coherent homotopy, i.e.\ the diagrams expressing
$$
r \cdot (m_1 +_M m_2) \simeq r \cdot m_1 +_M r \cdot m_2,
\qquad
(r_1 + r_2) \cdot m \simeq r_1 \cdot m +_M r_2 \cdot m,
$$
$$
(r_1 \cdot r_2) \cdot m \simeq r_1 \cdot (r_2 \cdot m),
\qquad
1_R \cdot m \simeq m,
\qquad
0_R \cdot m \simeq 0_M \simeq r \cdot 0_M
$$
commute in the homotopy-coherent sense in $\mathcal{C}$.
\end{definition}



\section{Monoidal categories}
\TODO{Carefully check what statements need to assume that the categories are locally small and which do not}

\begin{definition} \label{definition:monoidal_category}
A \hldef{monoidal category} is a \CrefAndHyperrefIfExist{definition:category}{(large) category} $\mathcal{C}$ equipped with:
\begin{itemize}
    \item a \CrefAndHyperrefIfExist{definition:n_ary_functor}{bifunctor} $\otimes : \mathcal{C} \times \mathcal{C} \to \mathcal{C}$\CrefIfExists{definition:product_category_of_a_family_of_categories} (called the \hldef{tensor product});
    \item an object $\mathbb{I} \in \mathrm{Ob}(\mathcal{C})$ (often called the \hldef{unit object}); common notations for the unit object include \hl{$\mathbb{I}$} and \hl{$\mathds{1}$}.
    \item natural isomorphisms (\hldef{associator}) $\alpha_{X,Y,Z} : (X \otimes Y) \otimes Z \to X \otimes (Y \otimes Z)$ for all $X, Y, Z \in \mathcal{C}$;
    \item natural isomorphisms (\hldef{left and right unitors}) $\lambda_X : \mathbb{I} \otimes X \to X$, $\rho_X : X \otimes \mathbb{I} \to X$ for all $X \in \mathcal{C}$;
\end{itemize}
% \TODO{TODO: add the pentagon and triangle coherence diagrams}

% such that the pentagon and triangle coherence diagrams commute.

such that the following coherence diagrams commute:

\bigskip

\textbf{Pentagon coherence:} For all $W,X,Y,Z \in \mathcal{C}$, the diagram
\[
\begin{tikzcd}[column sep=huge]
((W \otimes X) \otimes Y) \otimes Z \arrow[r, "\alpha_{W\otimes X, Y, Z}"] \arrow[d, "\alpha_{W,X,Y} \otimes \mathrm{id}_Z"'] & (W \otimes X) \otimes (Y \otimes Z) \arrow[r, "\alpha_{W,X,Y \otimes Z}"] & W \otimes (X \otimes (Y \otimes Z)) \\
(W \otimes (X \otimes Y)) \otimes Z \arrow[rr, "\alpha_{W,X \otimes Y, Z}"'] & & W \otimes ((X \otimes Y) \otimes Z) \arrow[u, "\mathrm{id}_W \otimes \alpha_{X,Y,Z}"'].
\end{tikzcd}
\]

\bigskip

\textbf{Triangle coherence:} For all $X,Y \in \mathcal{C}$, the diagram
\[
\begin{tikzcd}[column sep=large]
(X \otimes \mathbb{I}) \otimes Y \arrow[r, "\alpha_{X, \mathbb{I}, Y}"] \arrow[dr, "\rho_X \otimes \mathrm{id}_Y"'] & X \otimes (\mathbb{I} \otimes Y) \arrow[d, "\mathrm{id}_X \otimes \lambda_Y"] \\
& X \otimes Y.
\end{tikzcd}
\]
\end{definition}


\begin{definition} \label{definition:internal_hom_object_in_a_category}
Let $(\mathcal{C}, \otimes, \mathbb{I})$ be a \hyperrefIfExists{definition:monoidal_category}{monoidal category}. Given objects $X, Y \in \mathrm{Ob}(\mathcal{C})$, an \hldef{internal hom object from $X$ to $Y$} is an object $\underline{\mathrm{Hom}}(X,Y) \in \mathrm{Ob}(\mathcal{C})$ together with a morphism
$$\hlin{\mathrm{ev}_{X,Y}: \underline{\mathrm{Hom}}(X,Y) \otimes X \to Y}$$

such that for every object $Z \in \mathcal{C}$, the assignment
\[
\hom_{\mathcal{C}}(Z, \underline{\mathrm{Hom}}(X,Y)) \to \hom_{\mathcal{C}}(Z \otimes X, Y), \quad f \mapsto \mathrm{ev}_{X,Y} \circ (f \otimes \mathrm{id}_X)
\]
is a natural isomorphism of sets.
\end{definition}



% \begin{definition} \label{definition:closed_category}
% A category $\mathcal{C}$ equipped with a bifunctor
% \TODO{TODO: define opposite category, product category}
% \[
% \hlin{\otimes : \mathcal{C}^{\mathrm{op}} \times \mathcal{C} \to \mathcal{C}}
% \]
% is called a \hldef{closed category} if for every pair of objects $X, Y$ in $\mathcal{C}$, the internal hom object $\underline{\mathrm{Hom}}(X,Y)$ exists.
% \end{definition}

% \begin{definition} \label{definition:closed_monoidal_category}
% A \hldef{closed monoidal category} is a \hyperrefIfExists{definition:monoidal_category}{monoidal category} $(\mathcal{C}, \otimes, \mathbb{I})$ such that $\mathcal{C}$ is a \hyperrefIfExists{definition:closed_category}{closed category} with respect to the tensor product $\otimes$, i.e., for every $X,Y \in \mathrm{Ob}(\mathcal{C})$, there exists an internal hom object
% \[
% \hlin{\underline{\mathrm{Hom}}(X,Y)}
% \]
% with evaluation map
% \[
% \hlin{\mathrm{ev}_{X,Y} : \underline{\mathrm{Hom}}(X,Y) \otimes X \to Y.}
% \]
% The isomorphisms 
% \[
% \hom_{\mathcal{C}}(Z, \underline{\mathrm{Hom}}(X,Y)) \cong \hom_{\mathcal{C}}(Z \otimes X, Y)
% \]
% natural in $Z$, $X$, and $Y$ must hold.
% \end{definition}

\begin{definition} \label{definition:closed_category}
A \hldef{closed category} is a category $\mathcal{C}$ equipped with the following data:
\begin{itemize}
    \item A functor 
    $$
    \hlin{\underline{\mathrm{Hom}} : \mathcal{C}^{\mathrm{op}} \times \mathcal{C} \to \mathcal{C}}
    $$
    called the \hldef{internal hom-functor}.
    \item A \hldef{left evaluation morphism} 
    $$
    \hlin{j_X : I \to \underline{\mathrm{Hom}}(X,X)}
    $$
    for each object $X \in \mathcal{C}$, where \hl{$I$} is a fixed object serving as a "unit" of the internal hom-structure.
    \item A family of natural \hldef{left composition morphisms}
    $$ \hlin{L^X_{Y,Z} : \underline{\mathrm{Hom}}(Y,Z) \otimes \underline{\mathrm{Hom}}(X,Y) \to \underline{\mathrm{Hom}}(X,Z)} $$
    \TODO{TODO: the axioms}
    satisfying appropriate associativity and unit axioms that generalize composition.
\end{itemize}
\end{definition}
\begin{definition} \label{definition:closed_monoidal_category}
A \hldef{closed monoidal category} is a \CrefAndHyperrefIfExist{definition:monoidal_category}{monoidal category} that is a \CrefAndHyperrefIfExist{definition:closed_category}{closed category} where the \CrefAndHyperrefIfExist{definition:internal_hom_object_in_a_category}{internal hom} is right adjoint to the monoidal tensor.

In other words, for all objects $Y, Z \in \mathrm{Ob}(\mathcal{C})$ there is a natural isomorphism 
$$ \hom_{\mathcal{C}}(Y \otimes X, Z) \cong \hom_{\mathcal{C}}(Y, \underline{\mathrm{Hom}}(X,Z)) $$
of Hom-sets natural in $Y$ and $Z$.
\end{definition}

\begin{definition} \label{definition:symmetric_monoidal_category}
A \hldef{symmetric monoidal category} is a \hyperrefIfExists{definition:monoidal_category}{monoidal category} $(\mathcal{C}, \otimes, \mathbb{I})$ together with a natural isomorphism (symmetry)
$$\hlin{\gamma_{X,Y}: X \otimes Y \xrightarrow{\cong} Y \otimes X}$$
for all $X, Y \in \mathcal{C}$, such that for all $X, Y, Z \in \mathcal{C}$ the following holds:
\begin{itemize}
    \item $\gamma_{Y,X} \circ \gamma_{X,Y} = \mathrm{id}_{X \otimes Y}$ (involutivity);
    % % \TODO{TODO: add the hexagon and symmetry coherence diagrams}
    % \item the hexagon and symmetry coherence diagrams commute.
        \item the \textbf{hexagon coherence diagrams} commute:
        \[
        \begin{tikzcd}[column sep=large]
        (X \otimes Y) \otimes Z \arrow[r, "\alpha_{X,Y,Z}"] \arrow[d, "\gamma_{X,Y} \otimes \mathrm{id}_Z"'] & X \otimes (Y \otimes Z) \arrow[r, "\gamma_{X, Y \otimes Z}"] & (Y \otimes Z) \otimes X \\
        (Y \otimes X) \otimes Z \arrow[rr, "\alpha_{Y,X,Z}"'] & & Y \otimes (X \otimes Z) \arrow[u, "\mathrm{id}_Y \otimes \gamma_{X,Z}"']
        \end{tikzcd}
        \]
        and the analogous hexagon with inverse braiding:
        \[
        \begin{tikzcd}[column sep=large]
        X \otimes (Y \otimes Z) \arrow[r, "\alpha^{-1}_{X,Y,Z}"] \arrow[d, "\mathrm{id}_X \otimes \gamma_{Y,Z}"'] & (X \otimes Y) \otimes Z \arrow[r, "\gamma_{X \otimes Y, Z}"] & Z \otimes (X \otimes Y) \\
        X \otimes (Z \otimes Y) \arrow[rr, "\alpha^{-1}_{X,Z,Y}"'] & & (X \otimes Z) \otimes Y \arrow[u, "\gamma_{X,Z} \otimes \mathrm{id}_Y"']
        \end{tikzcd}
        \]
    \item the \textbf{symmetry coherence diagram} commutes:
        \[
        \begin{tikzcd}
        X \otimes Y \arrow[r, "\gamma_{X,Y}"] \arrow[dr, swap, "\mathrm{id}_{X \otimes Y}"] & Y \otimes X \arrow[d, "\gamma_{Y,X}"] \\
        & X \otimes Y
        \end{tikzcd}
        \]
\end{itemize}
A \hldef{closed symmetric monoidal category} usually refers to a symmetric monoidal category that is \hyperrefIfExists{definition:closed_monoidal_category}{closed as a monoidal category}. 
\end{definition}

\subsubsection{Types of objects in a category}




\begin{definition} \label{definition:initial_final_zero_objects_of_a_category}
Let $\mathcal{C}$ be a \CrefAndHyperrefIfExist{definition:category}{(large) category}.

\begin{enumerate}
    \item An object $I \in \mathcal{C}$ is called an \hldef{initial object} if for every object $X \in \mathcal{C}$ there exists a unique morphism
    $$I \to X.$$
    Equivalently, an initial object is a \CrefAndHyperrefIfExist{definition:limit_and_colimit_of_a_diagram_in_a_category}{limit} of the empty \CrefAndHyperrefIfExist{definition:diagram_in_a_category_indexed_by_a_small_category}{diagram}, if such a limit exists.

    \item An object $F \in \mathcal{C}$ is called a \hldef{final object} (or \hldef{terminal object}) if for every object $X \in \mathcal{C}$ there exists a unique morphism
    $$X \to F.$$
    Equivalently, a final object is a \CrefAndHyperrefIfExist{definition:limit_and_colimit_of_a_diagram_in_a_category}{colimit} of the empty \CrefAndHyperrefIfExist{definition:diagram_in_a_category_indexed_by_a_small_category}{diagram}, if such a colimit exists.

    \item An object $Z \in \mathcal{C}$ is called a \hldef{zero object} if $Z$ is both initial and final in $\mathcal{C}$. In particular, for every object $X \in \mathcal{C}$ there exist unique morphisms
    $$Z \to X \quad \text{and} \quad X \to Z.$$
\end{enumerate}
In particular, if initial/final/zero objects exist in a cateogry, then they are unique up to unique isomorphism.
\end{definition}

\begin{lemma} \label{lemma:initial_or_final_object_in_a_category_that_is_also_in_a_full_subcategory_is_initial_or_final_in_the_subcategory}
    Let $\calC$ be a \CrefAndHyperrefIfExist{definition:full_subcategory_of_a_category}{full subcategory} of a \CrefAndHyperrefIfExist{definition:category}{(large) category} $\calD$. 
    Suppose that $\calD$ hsa an \CrefAndHyperrefIfExist{definition:initial_final_zero_objects_of_a_category}{initial object} $I$ (resp. a \CrefAndHyperrefIfExist{definition:initial_final_zero_objects_of_a_category}{final object} $F$) and that this object also belongs to $\calC$. The object is initial (resp. final) in $\calC$. 
\end{lemma}


\begin{definition} \label{definition:wedge_cowedge_between_functor_from_opposite_and_category_to_category_and_object_of_target}
Let $C$ and $D$ be \CrefAndHyperrefIfExist{definition:category}{categories}, let $S: C^{op} \times C \to D$ be a functor, and let $d$ be an object of $D$.

A \hldef{wedge from $d$ to $S$} is a family of morphisms $\omega = \{\omega_c\}_{c \in \operatorname{Ob}(C)}$ in $D$, denoted

\hlalign{
\begin{align*}
\omega: d \to S,
\end{align*}
}

where each component is a morphism $\omega_c: d \to S(c,c)$, satisfying the wedge condition: for every morphism $f: c \to c'$ in $C$, the diagram
$$ S(1_c, f) \circ \omega_c = S(f, 1_{c'}) \circ \omega_{c'} $$
commutes in $D$. (Explicitly, both composites are morphisms $d \to S(c, c')$).

Dually, a \hldef{cowedge from $S$ to $d$} is a family of morphisms $\beta = \{\beta_c\}_{c \in \operatorname{Ob}(C)}$ in $D$, denoted

\hlalign{
\begin{align*}
\beta: S \to d,
\end{align*}
}

where each component is a morphism $\beta_c: S(c,c) \to d$, satisfying the \hldef{cowedge condition}: for every morphism $f: c \to c'$ in $C$, the equality
$$ \beta_c \circ S(f, 1_c) = \beta_{c'} \circ S(1_{c'}, f) $$
holds in $D$ (as morphisms from $S(c', c)$ to $d$).
\end{definition}

\begin{definition} \label{definition:end_coend_of_functor_from_opposite_and_category_to_category}
Let $C$ and $D$ be \CrefAndHyperrefIfExist{definition:category}{categories} and let $S: C^{op} \times C \to D$ be a functor.

An \hldef{end of $S$} is a pair $(E, \omega)$, where $E$ is an object of $D$ and $\omega: E \to S$ is a \CrefAndHyperrefIfExist{definition:wedge_cowedge_between_functor_from_opposite_and_category_to_category_and_object_of_target}{wedge}, which is universal among all wedges to $S$. That is, for any object $d \in D$ and any wedge $\nu: d \to S$, there exists a unique morphism $h: d \to E$ in $D$ such that $\nu_c = \omega_c \circ h$ for all $c \in \operatorname{Ob}(C)$. The object $E$ is uniquely determined up to unique isomorphism and is denoted by
$$
\hlin{ \int_{c \in C} S(c,c) \quad \text{or} \quad \int_C S. }
$$

Dually, a \hldef{coend of $S$} is a pair $(K, \beta)$, where $K$ is an object of $D$ and $\beta: S \to K$ is a cowedge, which is universal among all cowedges from $S$. That is, for any object $d \in D$ and any cowedge $\gamma: S \to d$, there exists a unique morphism $g: K \to d$ in $D$ such that $\gamma_c = g \circ \beta_c$ for all $c \in \operatorname{Ob}(C)$. The object $K$ is denoted by
$$ \hlin{ \int^{c \in C} S(c,c) \quad \text{or} \quad \int^C S. } $$
\end{definition}


\section{Additive and abelian categories}

\TODO{Carefully check what statements need to assume that the categories are locally small and which do not}
\begin{definition}[Additive category] \label{definition:additive_category}
Let $\mathcal{A}$ be a \CrefAndHyperrefIfExist{definition:locally_small_category}{locally small category}. 
\begin{enumerate}
    \item $\calA$ is said to be a \hldef{preadditive category} if the following hold:
    \begin{itemize}
        \item For any two objects $A, B$ in $\mathcal{A}$, the set $\operatorname{Hom}_{\mathcal{A}}(A, B)$ is an \CrefAndHyperrefIfExist{definition:group}{abelian group}, and composition of morphisms is bilinear.
        \item There is a \CrefAndHyperrefIfExist{definition:initial_final_zero_objects_of_a_category}{zero object} $0$ in $\mathcal{A}$.
    \end{itemize}
    \TextIfExists{definition:category_enriched_in_a_monoidal_category}{Equvialently, a preadditive cateogry $\calA$ is a (necessarily locally small) category \CrefAndHyperrefIfExist{definition:category_enriched_in_a_monoidal_category}{enriched in} the \CrefAndHyperrefIfExist{definition:monoidal_category}{monoidal category} $\Ab$ that also possesses a zero object.}

    \item
    If $\calA$ is preadditive, then it is called \hldef{additive} if it additionally satisfies the following:
    \begin{itemize}
        \item For any two objects $A, B$ in $\mathcal{A}$, there exists a \CrefAndHyperrefIfExist{definition:product_and_coproduct_of_objects_in_a_category}{product object $A \times B$}, often written \hl{$A \oplus B$}, called the \hldef{direct sum of $A$ and $B$}. In fact, $A \oplus B$ is not only a product but also a \CrefAndHyperrefIfExist{definition:coproduct_of_modules_of_rings}{coproduct} of $A$ and $B$\CrefIfExists{lemma:finite_products_and_finite_coproducts_coincide_in_preadditive_categories}.
    \end{itemize}

    Given a finite collection $\{A_i\}_i$ of objects $A_i$ in an additive category $\calA$, we may more generally speak of the \hldef{direct sum} \hl{$\bigoplus_i A_i$}; it has canonical injections from and projections to each $A_i$.


\end{enumerate}
\end{definition}

\begin{definition}[Additive functor] \label{definition:additive_functor_between_additive_categories}

    \begin{enumerate}
        \item Let $\mathcal{A}$ and $\mathcal{B}$ be \hyperrefIfExists{definition:additive_category_preadditive_category}{pre-additive categories}. A functor
        $$ F: \mathcal{A} \to \mathcal{B} $$
        is an \hldef{additive functor} if for every pair of objects $A, A' \in \mathcal{A}$, the induced map
        $$ F_{A,A'}: \operatorname{Hom}_{\mathcal{A}}(A, A') \to \operatorname{Hom}_{\mathcal{B}}(F(A), F(A')) $$
        is a group homomorphism of abelian groups, or equvialently if it is \CrefAndHyperrefIfExist{definition:category_enriched_in_a_monoidal_category}{enriched over the category $\Ab$ of abelian groups}.
        
        \item Let $\mathcal{A}$ and $\mathcal{B}$ be \hyperrefIfExists{definition:additive_category_preadditive_category}{additive categories}. A functor
        $$ F: \mathcal{A} \to \mathcal{B} $$
        is an \hldef{additive functor} if it an additive functor of pre-additive categories and satisfies the following:
        \begin{itemize}
            \item $F$ sends the zero object $0_{\mathcal{A}}$ of $\mathcal{A}$ to the zero object $0_{\mathcal{B}}$ of $\mathcal{B}$, i.e.,
            $$ F(0_{\mathcal{A}}) = 0_{\mathcal{B}}.  $$
            \item $F$ preserves finite direct sums: For any finite family of objects $\{A_i\}_{i=1}^n$ in $\mathcal{A}$,
            $$ F\left(\bigoplus_{i=1}^n A_i\right) \cong \bigoplus_{i=1}^n F(A_i) $$
            via the canonical isomorphism induced by $F$ applied to the canonical injections and projections.
        \end{itemize}
        In other words, $F$ is a functor that is compatible with the additive structures on $\mathcal{A}$ and $\mathcal{B}$.
    \end{enumerate}
\end{definition}

\begin{definition}[Abelian category] \label{definition:abelian_category}
Let $\mathcal{A}$ be a category. The category $\mathcal{A}$ is an \hldef{abelian category} if:
\begin{itemize}
    \item $\mathcal{A}$ is an \CrefAndHyperrefIfExist{definition:additive_category_preadditive_category}{additive category}.

    \item Every morphism $f: A \to B$ has a \CrefAndHyperrefIfExist{definition:kernel_and_cokernel_of_a_morphism_in_a_category}{kernel $\ker(f)$ and a cokernel $\operatorname{coker}(f)$}.

    \item For every morphism $f: A \to B$, the canonical morphism $\operatorname{coim}(f) \to \operatorname{im}(f)$ is an isomorphism, where
    $$
    \operatorname{coim}(f) = \operatorname{coker}(\ker(f) \to A),\quad \operatorname{im}(f) = \ker(B \to \operatorname{coker}(f)).
    $$
    \TODO{I think I need to re-check this defintion}
    \TODO{coimage}
\end{itemize}

\TextIfExists{definition:pre_abelian_category}{In particular, every abelian category is \Cref{definition:pre_abelian_category}{pre-abelian}}.

It is also worth considering Grothendieck's additional axioms for abelian categories\CrefIfExists{definition:grothendiecks_additional_axioms_for_abelian_categories}.

\end{definition}





\begin{definition}[Grothendieck's axioms for abelian categories (Ab1--Ab5)] \label{definition:grothendiecks_additional_axioms_for_abelian_categories}
Let $\mathcal{A}$ be an \CrefAndHyperrefIfExist{definition:abelian_category}{abelian category}.

Grothendieck introduced the following hierarchy of additional axioms to express stronger completeness and exactness properties in $\mathcal{A}$ --- we note that Ab1, Ab2, and Ab2\textsuperscript{*} are already satisfied for any abelian category:

\begin{itemize}

  \item \hldef{Ab1}: Every morphism in $\calA$ has a \CrefAndHyperrefIfExist{definition:kernel_and_cokernel_of_a_morphism_in_a_category}{kernel and a cokernel}.
  \item \hldef{Ab2}: Every \CrefAndHyperrefIfExist{definition:monomorphism_and_epimorphism_in_categories}{monic} in $\calA$ is the kernel of its cokernel. 
  \item \hldef{Ab2\textsuperscript{*}}: Every epi in $\calA$ is the cokernel of its kernel. 

  \item \hldef{AB3}: The category $\mathcal{A}$ is \CrefAndHyperrefIfExist{definition:complete_and_cocomplete_category}{cocomplete}.
  \begin{itemize}
    \item Since $\mathcal{A}$ is abelian (and hence \CrefAndHyperrefIfExist{lemma:equalizer_coequalizer_in_an_additive_category_are_given_by_kernel_and_cokernel}{admits} \CrefAndHyperrefIfExist{definition:equalizer_and_coequalizer_of_morphisms_in_a_category}{equalizers} as \CrefAndHyperrefIfExist{definition:kernel_image_cokernel_coimage_of_a_module_homomorphism}{kernels}), this is equivalent to requiring that $\mathcal{A}$ has all small \CrefAndHyperrefIfExist{definition:product_and_coproduct_of_objects_in_a_category}{coproducts} (direct sums).
  \end{itemize}

  \item \hldef{AB4}: The category $\mathcal{A}$ satisfies AB3, and coproducts are \emph{exact}.
  \begin{itemize}
    \item That is, the coproduct of a family of short exact sequences is a short exact sequence. Explicitly, for any family of short exact sequences $0 \to A_i \to B_i \to C_i \to 0$ indexed by a set $I$, the sequence
    \[ 0 \to \bigoplus_{i \in I} A_i \to \bigoplus_{i \in I} B_i \to \bigoplus_{i \in I} C_i \to 0 \]
    is exact in $\mathcal{A}$.
  \end{itemize}

  \item \hldef{AB5}: The category $\mathcal{A}$ satisfies AB3, and \CrefAndHyperrefIfExist{definition:projective_and_inductive_limits_in_categories}{filtered colimits} are \emph{exact}.
  \begin{itemize}
    \item Equivalently, for any \CrefAndHyperrefIfExist{definition:filtered_cofiltered_category}{filtered} index category $J$ and any \CrefAndHyperrefIfExist{definition:system_in_a_category_indexed_by_a_directed_poset}{directed system} of short exact sequences $0 \to A_j \to B_j \to C_j \to 0$, the colimit sequence
    \[ 0 \to \varinjlim A_j \to \varinjlim B_j \to \varinjlim C_j \to 0 \]
    is exact.
    \item Note: AB5 implies AB4. An abelian category satisfying AB5 and having a \CrefAndHyperrefIfExist{definition:generator_of_a_category}{generator} is called a \hldef{Grothendieck category}.
  \end{itemize}
  
  \item \hldef{AB6}: The category $\mathcal{A}$ satisfies AB3, and for any object $X$ and any family of filtered subobjects $\{F_i\}_{i \in I}$ of $X$ (where each $F_i$ is a filter of subobjects), the intersection commutes with the limit:
  \[ \bigcap_{i \in I} (\varinjlim_{j \in F_i} U_{i,j}) = \varinjlim_{(j_i) \in \prod F_i} (\bigcap_{i \in I} U_{i, j_i}). \]
  (This axiom is less commonly cited but appears in Grothendieck's Tohoku paper).

  \item \hldef{AB3\textsuperscript{*}}: The category \(\mathcal{A}\) is \CrefAndHyperrefIfExist{definition:complete_and_cocomplete_category}{complete} (i.e., has all small products).

  \item \hldef{AB4\textsuperscript{*}}: The category \(\mathcal{A}\) satisfies AB3\textsuperscript{*} and products are exact.
  \begin{itemize}
    \item Note: This is rarely satisfied for module categories (e.g., it fails for Abelian groups), but it is satisfied for the category of sheaves on a space.
  \end{itemize}

  \item \hldef{AB5\textsuperscript{*}}: The category \(\mathcal{A}\) satisfies AB3\textsuperscript{*} and filtered limits (inverse limits) are exact.
\end{itemize}

\textbf{Notes:}
\begin{itemize}
  \item AB5 implies AB4, and AB4 implies AB3.
  \item AB5\textsuperscript{*} implies AB4\textsuperscript{*}, and AB4\textsuperscript{*} implies AB3\textsuperscript{*}.
\end{itemize}
\end{definition}


% \begin{definition}[Grothendieck's axioms AB3--AB6] \label{definition:grothendiecks_ab_axioms}
% Let $\mathcal{A}$ be an \CrefAndHyperrefIfExist{definition:abelian_category}{abelian category}. (Recall that the axioms AB1 and AB2 refer to the existence of kernels/cokernels and the isomorphism between coimage and image, which are part of the definition of an abelian category).

% Grothendieck introduced the following hierarchy of additional axioms to express stronger completeness and exactness properties in $\mathcal{A}$:

% \begin{itemize}
%   \item \hldef{AB3}: The category $\mathcal{A}$ is \CrefAndHyperrefIfExist{definition:complete_and_cocomplete_category}{cocomplete}.
%   \begin{itemize}
%     \item Since $\mathcal{A}$ is abelian (hence has finite colimits), this is equivalent to requiring that $\mathcal{A}$ has all small \CrefAndHyperrefIfExist{definition:product_and_coproduct_of_objects_in_a_category}{coproducts} (direct sums).
%   \end{itemize}

%   \item \hldef{AB4}: The category $\mathcal{A}$ satisfies AB3, and coproducts are \emph{exact}.
%   \begin{itemize}
%     \item That is, the coproduct of a family of short exact sequences is a short exact sequence. Explicitly, for any family of short exact sequences $0 \to A_i \to B_i \to C_i \to 0$ indexed by a set $I$, the sequence
%     \[ 0 \to \bigoplus_{i \in I} A_i \to \bigoplus_{i \in I} B_i \to \bigoplus_{i \in I} C_i \to 0 \]
%     is exact in $\mathcal{A}$.
%   \end{itemize}

%   \item \hldef{AB5}: The category $\mathcal{A}$ satisfies AB3, and \CrefAndHyperrefIfExist{definition:projective_and_inductive_limits_in_categories}{filtered colimits} are \emph{exact}.
%   \begin{itemize}
%     \item Equivalently, for any \CrefAndHyperrefIfExist{definition:filtered_cofiltered_category}{filtered} index category $J$ and any \CrefAndHyperrefIfExist{definition:system_in_a_category_indexed_by_a_directed_poset}{directed system} of short exact sequences $0 \to A_j \to B_j \to C_j \to 0$, the colimit sequence
%     \[ 0 \to \varinjlim A_j \to \varinjlim B_j \to \varinjlim C_j \to 0 \]
%     is exact.
%     \item Note: AB5 implies AB4. An abelian category satisfying AB5 and having a \CrefAndHyperrefIfExist{definition:generator_of_a_category}{generator} is called a \hldef{Grothendieck category}.
%   \end{itemize}
  
%   \item \hldef{AB6}: The category $\mathcal{A}$ satisfies AB3, and for any object $X$ and any family of filtered subobjects $\{F_i\}_{i \in I}$ of $X$ (where each $F_i$ is a filter of subobjects), the intersection commutes with the limit:
%   \[ \bigcap_{i \in I} (\varinjlim_{j \in F_i} U_{i,j}) = \varinjlim_{(j_i) \in \prod F_i} (\bigcap_{i \in I} U_{i, j_i}). \]
%   (This axiom is less commonly cited but appears in Grothendieck's Tohoku paper).

%   \item \hldef{AB3\textsuperscript{*}}: The category \(\mathcal{A}\) is \CrefAndHyperrefIfExist{definition:complete_and_cocomplete_category}{complete} (i.e., has all small products).

%   \item \hldef{AB4\textsuperscript{*}}: The category \(\mathcal{A}\) satisfies AB3\textsuperscript{*} and products are exact.
%   \begin{itemize}
%     \item Note: This is rarely satisfied for module categories (e.g., it fails for Abelian groups), but it is satisfied for the category of sheaves on a space.
%   \end{itemize}

%   \item \hldef{AB5\textsuperscript{*}}: The category \(\mathcal{A}\) satisfies AB3\textsuperscript{*} and filtered limits (inverse limits) are exact.
% \end{itemize}

% \textbf{Notes:}
% \begin{itemize}
%   \item AB5 implies AB4, and AB4 implies AB3.
%   \item AB5\textsuperscript{*} implies AB4\textsuperscript{*}, and AB4\textsuperscript{*} implies AB3\textsuperscript{*}.
%   \item The condition you originally listed as "Ab2" (disjoint/universal sums) characterizes \emph{extensive categories} or \emph{toposes}, not abelian categories. In an abelian category, the coproduct is a biproduct and is never "disjoint" in the sense of set theory (unless $0=1$).
% \end{itemize}
% \end{definition}



\begin{proposition} \label{proposition:examples_of_abelian_categories}
The following are examples of \CrefAndHyperrefIfExist{definition:abelian_category}{abelian categories}:

\begin{enumerate}
    \item The category of $R$-$S$ bimodules where $R$,$S$ are \CrefAndHyperrefIfExist{definition:ring}{(not necessarily commutative) rings} (\Cref{theorem:the_category_of_R_S_bimodules_is_a_grothendieck_abelian_category_and_AB4_star}).

    \item The category $\mathbf{Ab}$ of abelian groups and group homomorphisms is abelian.

    \item The category $\text{Vect}_k$ of vector spaces over a field $k$ and $k$-linear maps is abelian.

    \item More generally, if $R$ is a \CrefAndHyperrefIfExist{definition:noetherian_ring}{noetherian ring}, then the category of \CrefAndHyperrefIfExist{definition:finitely_generated_modules_over_rings}{finitely generated} $R$-modules is abelian.

    \item For a \CrefAndHyperrefIfExist{definition:ringed_space}{ringed space} $(X, \mathcal{O}_X)$, the category of \CrefAndHyperrefIfExist{definition:module_over_a_sheaf_of_rings_on_a_site}{$\mathcal{O}_X$-modules} is abelian.
    \TODO{a quasi-coherent sheaf on a locally ringed space}
    \item If $X$ is a \CrefAndHyperrefIfExist{definition:scheme}{scheme} (or more generally a \CrefAndHyperrefIfExist{definition:locally_ringed_space_on_a_topological_space}{locally ringed space}), the category of \CrefAndHyperrefIfExist{definition:quasi_coherent_sheaf_on_a_general_scheme}{quasi-coherent sheaves on $X$} is abelian.
    \item For any \CrefAndHyperrefIfExist{definition:essentially_small_category}{essentially small category} $\mathcal{C}$ and any abelian category $\mathcal{A}$, the \CrefAndHyperrefIfExist{definition:diagram_in_a_category_indexed_by_a_small_category}{functor category $[\mathcal{C}, \mathcal{A}]$} and the category $\PreShv(\calC, \calA)$ of \CrefAndHyperrefIfExist{definition:presheaf_on_a_category}{presheaves} are abelian.
    \TODO{apparently, the essentially smallness condition is removable, provided that the sheafification functor exists. However, the essentially small assumption is needed to show that the category of sheaves of $O$-modules is a Grothendieck abelian caetgory. Verify all this. Moreover, when working with a big site of a scheme, one typically fixes a unvierse or work relative to a cardinal cutoff to treat it as essentially small}

    \item For any \CrefAndHyperrefIfExist{definition:grothendieck_topology_on_a_category_site_covering_sieve_topologically_generating_family}{site} $(\calC, J)$ on an \CrefAndHyperrefIfExist{definition:essentially_small_category}{essentially small category} $\mathcal{C}$ and any abelian category $\mathcal{A}$, the category $\Shv(\calC, \calA)$ of \CrefAndHyperrefIfExist{definition:sheaf_on_a_site}{sheaves} is abelian.

    \item For any \CrefAndHyperrefIfExist{definition:grothendieck_topology_on_a_category_site_covering_sieve_topologically_generating_family}{site} $(\calC, J)$ on an \CrefAndHyperrefIfExist{definition:essentially_small_category}{essentially small category} $\mathcal{C}$ and a \CrefAndHyperrefIfExist{definition:sheaf_on_a_site}{sheaf of rings} $\calO$ on $\calC$, the category $\mathbf{Mod}(\mathcal{O})$ of \CrefAndHyperrefIfExist{definition:module_over_a_sheaf_of_rings_on_a_site}{$\calO$-modules} is an abelian category.

\end{enumerate}
\end{proposition}




\begin{definition} \label{definition:exact_functor_between_abelian_categories}
    Let $F: \mathcal{A} \to \mathcal{B}$ be an \hyperrefIfExists{definition:additive_functor_between_additive_categories}{additive functor}\CrefIfExists{definition:additive_functor_between_additive_categories} between \hyperrefIfExists{definition:abelian_category}{abelian categories}\CrefIfExists{definition:abelian_category}.
    \begin{enumerate}

        \item $F$ is called \hldef{left exact} if it preserves all \CrefAndHyperrefIfExist{definition:limit_and_colimit_of_a_diagram_in_a_category}{finite limits}, or equivalently it preserves \CrefAndHyperrefIfExist{definition:kernel_and_cokernel_of_a_morphism_in_a_category}{kernels} and any finite limit diagrams. Equivalently, for every left exact sequence in $\mathcal{A}$
        \[
        0 \to A' \xrightarrow{f} A \xrightarrow{g} A''
        \]
        the sequence
        \[
        0 \to F(A') \xrightarrow{F(f)} F(A) \xrightarrow{F(g)} F(A'')
        \]
        is exact at $F(A')$ and $F(A)$ (i.e., $F$ preserves \CrefAndHyperrefIfExist{definition:monomorphism_and_epimorphism_in_categories}{monomorphisms} and exactness at the first two terms).

        \item Dually, $F$ is called \hldef{right exact} if it preserves all \CrefAndHyperrefIfExist{definition:limit_and_colimit_of_a_diagram_in_a_category}{finite colimits}, or equivalently it preserves \CrefAndHyperrefIfExist{definition:kernel_and_cokernel_of_a_morphism_in_a_category}{cokernels} and any finite colimit diagrams. Equivalently, for every right exact sequence in $\mathcal{A}$
        \[
        A' \xrightarrow{f} A \xrightarrow{g} A'' \to 0,
        \]
        the sequence
        \[
        F(A') \xrightarrow{F(f)} F(A) \xrightarrow{F(g)} F(A'') \to 0
        \]
        is exact at $F(A)$ and $F(A'')$ (i.e., $F$ preserves \CrefAndHyperrefIfExist{definition:monomorphism_and_epimorphism_in_categories}{epimorphisms} and exactness at the last two terms).

        \item $F$ is called \hldef{exact} if it is both left and right exact.
    \end{enumerate}

    \TextIfExists{definition:left_right_exact_functor_between_categories}{
        The additive functor $F$ is left/right exact if and only if it is \CrefAndHyperref{definition:left_right_exact_functor_between_categories}{left/right exact} in the more general sense, i.e. if it preserves all \CrefAndHyperrefIfExist{definition:small_and_finite_limits_and_colimits_in_a_category}{finite} \CrefAndHyperrefIfExist{definition:limit_and_colimit_of_a_diagram_in_a_category}{limits/colimits}

    }
\end{definition}

\begin{lemma} \label{lemma:on_abelian_categories_exactness_coincides_with_exactness_as_categories}
    Let $F: \mathcal{A} \to \mathcal{B}$ be an \hyperrefIfExists{definition:additive_functor_between_additive_categories}{additive functor}\CrefIfExists{definition:additive_functor_between_additive_categories} between \hyperrefIfExists{definition:abelian_category}{abelian categories}\CrefIfExists{definition:abelian_category}. The functor $F$ is \CrefAndHyperrefIfExist{definition:exact_functor_between_abelian_categories}{left exact/right exact/exact} as an additive functor if and only if it is \CrefAndHyperrefIfExist{definition:left_right_exact_functor_between_categories}{left exact/right exact/exact} as a functor.
\end{lemma}

\subsection{Simple and semisimple objects in an additive category}


\begin{definition} \label{definition:subobject_of_an_object_of_an_additive_category}
Let $\mathcal{C}$ be an \CrefAndHyperrefIfExist{definition:additive_category}{additive category}. Let $X \in \Ob(\calC)$ be an object. 
A \hldef{subobject of $X$} refers to a \CrefAndHyperrefIfExist{definition:monomorphism_and_epimorphism_in_categories}{monomorphism} $i: Y \hookrightarrow X$ in $\mathcal{C}$. We regard two subobjects $(Y,i)$ and $(Y',i')$ of $X$ as isomorphic if there exists an isomorphism $f: Y \to Y'$ such that $i = i' \circ f$. One often leaves the monomorphism $i$ implicit, suprressing it from the notation.
\end{definition}


Unlike the notion of a subobject, the notion of a quotient object is more appropriate to speak of for an abelian category rather than a more general additive category.


\begin{definition} \label{definition:quotient_object_of_an_object_of_an_abelian_category_by_a_subobject}
Let $\mathcal{C}$ be an \CrefAndHyperrefIfExist{definition:abelian_category}{abelian category}. Let $X \in \Ob(\calC)$ be an object. Let $i: A \hookrightarrow X$ be a \CrefAndHyperrefIfExist{definition:subobject_of_an_object_of_an_additive_category}{subobject}. The \CrefAndHyperrefIfExist{definition:kernel_and_cokernel_of_a_morphism_in_a_category}{cokernel} $\pi: X \twoheadrightarrow X/A := \operatorname{coker}(i)$ is called the \hldef{quotient object of $X$ by $A$}. The object $X/A$ is determined up to canonical isomorphism.
\end{definition}


\begin{definition} \label{definition:subquotient_of_an_object_in_an_abelian_category}
Let $\mathcal{C}$ be an \CrefAndHyperrefIfExist{definition:abelian_category}{abelian category}. Let $X \in \mathrm{Ob}(\mathcal{C})$, and let $A \hookrightarrow B \hookrightarrow X$ be subobjects of $X$. The \CrefAndHyperrefIfExist{definition:quotient_object_of_an_object_of_an_abelian_category_by_a_subobject}{quotient object} $B/A := \operatorname{coker}(A \hookrightarrow B)$ is called a \hldef{subquotient object of $X$}. 
\end{definition}


\begin{definition} \label{definition:direct_summand_of_an_object_of_an_additive_category}
Let $\mathcal{C}$ be an \CrefAndHyperrefIfExist{definition:additive_category}{additive category} and $X \in \mathrm{Ob}(\mathcal{C})$. A \CrefAndHyperrefIfExist{definition:subcategory_of_a_category}{subobject} $i: Y \hookrightarrow X$ is called a \hldef{direct summand of $X$} if there exists a morphism $p: X \to Y$ such that $p \circ i = \mathrm{id}_Y$. In this case, $X$ is isomorphic to the \CrefAndHyperrefIfExist{definition:additive_category}{direct sum} $Y \oplus Z$ for some object $Z$, with $i$ the canonical inclusion.
\end{definition}


\begin{definition} \label{definition:simple_object_of_an_additive_category}
Let $\mathcal{C}$ be an \CrefAndHyperrefIfExist{definition:additive_category}{additive category}. An object $S \in \mathrm{Ob}(\mathcal{C})$ is called \hldef{simple} (or \hldef{irreducible}) if $S \neq 0$ and the only \CrefAndHyperrefIfExist{definition:subobject_of_an_object_of_an_additive_category}{subobjects} of $S$ are $0$ and $S$ itself (up to isomorphism of subobjects).
\end{definition}


\begin{definition} \label{definition:semisimple_object_of_an_additive_category}
Let $\mathcal{C}$ be an \CrefAndHyperrefIfExist{definition:additive_category_preadditive_category}{additive category}. An object $X \in \mathrm{Ob}(\mathcal{C})$ is called \hldef{semisimple} if it is isomorphic to a finite \CrefAndHyperrefIfExist{definition:additive_category_preadditive_category}{direct sum} of \CrefAndHyperrefIfExist{definition:simple_object_of_an_additive_category}{simple objects} in $\mathcal{C}$.
\end{definition}


\begin{definition} \label{definition:semisimple_additive_category}
An \CrefAndHyperrefIfExist{definition:additive_category_preadditive_category}{additive category} $\mathcal{C}$ is called a \hldef{semisimple category} if every object of $\mathcal{C}$ is \CrefAndHyperrefIfExist{definition:semisimple_object_of_an_additive_category}{semisimple}.
\end{definition}






\TODO{TODO: Give examples of monoidal categoriess}

\section{Categories enriched in monoidal categories}

\TODO{Carefully check what statements need to assume that the categories are locally small and which do not}

We might want to study categories whose hom-sets carry additional structure beyond that of a set (e.g. an abelian group, vector space); the theory of \CrefAndHyperrefIfExist{definition:category_enriched_in_a_monoidal_category}{categories enriched} in monoidal categories accommodates such desires. In fact, enriched categories can have hom-objects which are not sets.

\begin{definition}[Category enriched in a monoidal category] \label{definition:category_enriched_in_a_monoidal_category}
Let $(\mathcal{V}, \otimes, \mathbf{1})$ be a \CrefAndHyperrefIfExist{definition:monoidal_category}{monoidal category}. A \hldef{category enriched in $\mathcal{V}$} (or a \hldef{$\mathcal{V}$-enriched category} or a \hldef{$\mathcal{V}$-category}) $\mathcal{C}$ consists of the following data:
\begin{itemize}
    \item A class \hl{$\operatorname{Ob}(\mathcal{C})$} of \hldef{objects}. As with \hyperrefIfExists{definition:category}{regular categories}, we may write \hl{$X \in \operatorname{Ob}(\mathcal{C})$} or \hl{$X \in \calC$} to mean that $X$ is an object of $\calC$.  
    \item For each pair of objects $X, Y \in \operatorname{Ob}(\mathcal{C})$, an object \hl{$\underline{\operatorname{Hom}}_{\mathcal{C}}(X,Y) \in \operatorname{Ob}(\mathcal{V})$} of \hldef{morphisms}; it is an object of the monoidal category $\mathcal{V}$. It is also often denoted by notations such as \hl{$\calC(X,Y)$}, \hl{$\Hom(X,Y) = \Hom_\calC(X,Y)$}, or \hl{$\operatorname{Mor}(X,Y) = \operatorname{Mor}_{\calC}(X,Y)$}.
    \item For each triple $X,Y,Z \in \operatorname{Ob}(\mathcal{C})$, a \hldef{composition morphism} 
    $$\mu_{X,Y,Z} : \underline{\operatorname{Hom}}_{\mathcal{C}}(Y,Z) \otimes \underline{\operatorname{Hom}}_{\mathcal{C}}(X,Y) \to \underline{\operatorname{Hom}}_{\mathcal{C}}(X,Z).$$
    It is a morphism in $\mathcal{V}$.
    \item For each object $X$, a \hldef{unit morphism} \hl{$\eta_X : \mathbf{1} \to \underline{\operatorname{Hom}}_{\mathcal{C}}(X,X)$} in $\mathcal{V}$.
\end{itemize}
These data satisfy the following axioms:
\begin{itemize}
    \item (Associativity) For all $W,X,Y,Z \in \operatorname{Ob}(\mathcal{C})$, the following diagram in $\mathcal{V}$ commutes:
    $$
    \begin{tikzcd}[column sep=large,row sep=large]
    \bigl(\underline{\operatorname{Hom}}_{\mathcal{C}}(Z,W) \otimes \underline{\operatorname{Hom}}_{\mathcal{C}}(Y,Z)\bigr) \otimes \underline{\operatorname{Hom}}_{\mathcal{C}}(X,Y) \ar[r,"\alpha"] \ar[d,"\mu \otimes \mathrm{id}"]
    & \underline{\operatorname{Hom}}_{\mathcal{C}}(Z,W) \otimes \bigl(\underline{\operatorname{Hom}}_{\mathcal{C}}(Y,Z) \otimes \underline{\operatorname{Hom}}_{\mathcal{C}}(X,Y)\bigr) \ar[d,"\mathrm{id} \otimes \mu"] \\
    \underline{\operatorname{Hom}}_{\mathcal{C}}(Y,W) \otimes \underline{\operatorname{Hom}}_{\mathcal{C}}(X,Y) \ar[d,"\mu"] 
    & \underline{\operatorname{Hom}}_{\mathcal{C}}(Z,W) \otimes \underline{\operatorname{Hom}}_{\mathcal{C}}(X,Z) \ar[d,"\mu"] \\
    \underline{\operatorname{Hom}}_{\mathcal{C}}(X,W) \ar[r,equal] & \underline{\operatorname{Hom}}_{\mathcal{C}}(X,W)
    \end{tikzcd}
    $$
    where $\alpha$ is the associativity constraint in $\mathcal{V}$.
    \item (Unit) For all $X,Y \in \operatorname{Ob}(\mathcal{C})$, the following diagrams commute:
    \begin{center}
    \begin{tikzcd}[column sep=large]
    \mathbf{1} \otimes \underline{\operatorname{Hom}}_{\mathcal{C}}(X,Y) \ar[r,"\eta_Y \otimes \mathrm{id}"] \ar[dr,"\lambda"']
    & \underline{\operatorname{Hom}}_{\mathcal{C}}(Y,Y) \otimes \underline{\operatorname{Hom}}_{\mathcal{C}}(X,Y) \ar[d,"\mu"] \\
    & \underline{\operatorname{Hom}}_{\mathcal{C}}(X,Y)
    \end{tikzcd}
    \begin{tikzcd}[column sep=large]
    \underline{\operatorname{Hom}}_{\mathcal{C}}(X,Y) \otimes \mathbf{1} \ar[r,"\mathrm{id} \otimes \eta_X"] \ar[dr,"\rho"']
    & \underline{\operatorname{Hom}}_{\mathcal{C}}(X,Y) \otimes \underline{\operatorname{Hom}}_{\mathcal{C}}(X,X) \ar[d,"\mu"] \\
    & \underline{\operatorname{Hom}}_{\mathcal{C}}(X,Y)
    \end{tikzcd}
    \end{center}
    % $$
    % \quad\quad
    % $$
    where $\lambda$ and $\rho$ are the left and right unit constraints in $\mathcal{V}$.
\end{itemize}
\end{definition}

\begin{example}
    \begin{enumerate}
        \item A \hyperrefIfExists{definition:category_enriched_in_a_monoidal_category}{category enriched in} $\Sets$ is equivalent to a \hyperrefIfExists{definition:locally_small_category}{locally small category}. 
        \item A category enriched in $\Ab$ is equivalent to a \hyperrefIfExists{definition:additive_category}{pre-additive category}. 
        \TODO{TODO: define an $R$-linear category and a $k$-linear category}
        \item Let $R$ be a commutative ring. A category enriched in $R\mathrm{-Mod}$ is equivalent to a $R$-linear category. 
        \item Let $k$ be a field. A category enriched in $k\mathrm{-Vect}$ is equivalent to a $k$-linear category. 
    \end{enumerate}
\end{example}


\begin{definition} \label{definition:enriched_functor_between_categories_enriched_in_a_}
Let $\mathcal{V}$ be a \CrefAndHyperrefIfExist{definition:monoidal_category}{monoidal category} and let $\calC$ and $\calD$ be \CrefAndHyperrefIfExist{definition:category_enriched_in_a_monoidal_category}{categories enriched in $\calV$}. A \hldef{$\mathcal{V}$-functor between $\mathcal{C}$ and $\mathcal{D}$} or an \hldef{enriched functor between $\calC$ and $\calD$}, written
$$F : \mathcal{C} \to \mathcal{D},$$
consists of:
\begin{itemize}
    \item a function on objects $F : \mathrm{Ob}(\mathcal{C}) \to \mathrm{Ob}(\mathcal{D})$,
    \item for all $A,B \in \mathrm{Ob}(\mathcal{C})$, morphisms in $\mathcal{V}$,
        $$F_{A,B} : \mathcal{C}(A,B) \to \mathcal{D}(FA,FB),$$
\end{itemize}
such that the following diagrams in $\mathcal{V}$ commute:
\begin{align*}
\mathcal{C}(B,C) \otimes \mathcal{C}(A,B) &\xrightarrow{\circ_{\mathcal{C}}} \mathcal{C}(A,C) \\
\downarrow{F_{B,C} \otimes F_{A,B}} \quad & \quad \downarrow{F_{A,C}} \\
\mathcal{D}(FB,FC) \otimes \mathcal{D}(FA,FB) &\xrightarrow{\circ_{\mathcal{D}}} \mathcal{D}(FA,FC)
\end{align*}
and for all $A \in \mathrm{Ob}(\mathcal{C})$, the unit compatibility condition holds:
$$\iota_{FA} = F_{A,A} \circ \iota_A.$$
\end{definition}

\begin{definition} \label{definition:natural_transformation_between_enriched_functors_between_categories_enriched_in_a_monoidal_category}
Let $\mathcal{V}$ be a \CrefAndHyperrefIfExist{definition:monoidal_category}{monoidal category}, let $\mathcal{A}, \mathcal{B}$ be \CrefAndHyperrefIfExist{definition:category_enriched_in_a_monoidal_category}{$\mathcal{V}$-categories}, and let $F, G: \mathcal{A} \to \mathcal{B}$ be \CrefAndHyperrefIfExist{definition:enriched_functor_between_categories_enriched_in_a_}{$\mathcal{V}$-functors}.

A \hldef{$\mathcal{V}$-natural transformation} $\alpha: F \Rightarrow G$ consists of a family of morphisms in $\mathcal{V}_0$:
$$\hlin{ \alpha_x: I \to \mathcal{B}(Fx, Gx) }$$
indexed by objects $x \in \mathrm{Ob}(\mathcal{A})$, satisfying the $\mathcal{V}$-naturality axiom.

For every pair of objects $x, y \in \mathrm{Ob}(\mathcal{A})$, the following hexagonal diagram in $\mathcal{V}_0$ commutes (expressing the enriched analogue of $\alpha_y \circ F(f) = G(f) \circ \alpha_x$):

\begin{tikzcd}[column sep=huge]
    \mathcal{A}(x, y) \arrow[r, "F_{xy}"] \arrow[d, "G_{xy}"'] & \mathcal{B}(Fx, Fy) \arrow[r, "\lambda^{-1}", "\cong"'] & I \otimes \mathcal{B}(Fx, Fy) \arrow[d, "\alpha_y \otimes 1"] \\
    \mathcal{B}(Gx, Gy) \arrow[d, "\rho^{-1}"', "\cong"] & & \mathcal{B}(Fy, Gy) \otimes \mathcal{B}(Fx, Fy) \arrow[d, "M^\mathcal{B}_{Fx,Fy,Gy}"] \\
    \mathcal{B}(Gx, Gy) \otimes I \arrow[r, "1 \otimes \alpha_x"'] & \mathcal{B}(Gx, Gy) \otimes \mathcal{B}(Fx, Gx) \arrow[r, "M^\mathcal{B}_{Fx,Gx,Gy}"'] & \mathcal{B}(Fx, Gy)
\end{tikzcd}
Here, $\lambda: I \otimes X \cong X$ and $\rho: X \otimes I \cong X$ denote the left and right unit isomorphisms of the monoidal category $\mathcal{V}$, and $M^\mathcal{B}$ denotes the composition in $\mathcal{B}$.
\end{definition}


\begin{definition} \label{definition:underlying_ordinary_category_of_enriched_category}
Let $\mathcal{V} = (\mathcal{V}_0, \otimes, I, \alpha, \lambda, \rho)$ be a \CrefAndHyperrefIfExist{definition:monoidal_category}{monoidal category}, and let $\mathcal{A}$ be a \CrefAndHyperrefIfExist{definition:category_enriched_in_a_monoidal_category}{$\mathcal{V}$-category} with object set $\mathrm{Ob}(\mathcal{A})$ and hom-objects $\mathcal{A}(X, Y) \in \mathrm{Ob}(\mathcal{V}_0)$.

The \hldef{underlying ordinary category} of $\mathcal{A}$, denoted \hl{$\mathcal{A}_0$}, is the \CrefAndHyperrefIfExist{definition:category}{category} defined as follows:
\begin{enumerate}
    \item The objects of $\mathcal{A}_0$ are the objects of $\mathcal{A}$, i.e., $\mathrm{Ob}(\mathcal{A}_0) = \mathrm{Ob}(\mathcal{A})$.
    \item For any two objects $X, Y \in \mathrm{Ob}(\mathcal{A}_0)$, the set of morphisms is defined by
    $$
    \mathrm{Hom}_{\mathcal{A}_0}(X, Y) = \mathrm{Hom}_{\mathcal{V}_0}(I, \mathcal{A}(X, Y)),
    $$
    where $\mathrm{Hom}_{\mathcal{V}_0}$ denotes the hom-set in the ordinary category $\mathcal{V}_0$.
    \item The identity morphism $\mathrm{id}_X \in \mathrm{Hom}_{\mathcal{A}_0}(X, X)$ is the element corresponding to the identity morphism $j_X: I \to \mathcal{A}(X, X)$ in $\mathcal{V}$.
    \item The composition of morphisms $f \in \mathrm{Hom}_{\mathcal{A}_0}(X, Y)$ and $g \in \mathrm{Hom}_{\mathcal{A}_0}(Y, Z)$ (represented by morphisms $f: I \to \mathcal{A}(X, Y)$ and $g: I \to \mathcal{A}(Y, Z)$ in $\mathcal{V}_0$) is defined to be the composition $g \circ f$ in $\mathcal{V}_0$ given by the following diagram in $\mathcal{V}_0$:
    $$
    I \xrightarrow{\cong} I \otimes I \xrightarrow{g \otimes f} \mathcal{A}(Y, Z) \otimes \mathcal{A}(X, Y) \xrightarrow{M_{XYZ}} \mathcal{A}(X, Z),
    $$
    where $M_{XYZ}$ is the composition morphism of the enriched category $\mathcal{A}$.
\end{enumerate}
\end{definition}


\begin{definition} \label{definition:representable_functor_on_a_category_enriched_in_a_monoidal_category}
    Let $C$ be a \CrefAndHyperrefIfExist{definition:category_enriched_in_a_monoidal_category}{category enriched in a monidal category} $\mathcal{V}$. Given an object $X$ of $C$, the \hldef{functor of points} \hl{$h_X$} is the \CrefAndHyperrefIfExist{definition:functor_between_categories}{functor}/\CrefAndHyperrefIfExist{definition:presheaf_on_a_category}{presheaf} $C^{\op} \to \mathcal{V}$ given by $T \mapsto \Hom_C(T, X)$. A functor $C^{\op} \to \mathcal{V}$ (or equivalently, a presheaf on $C$ valued in $\mathcal{V}$) is said to be \hldef{representable} if it is \CrefAndHyperrefIfExist{definition:natural_transformation_between_functors_between_categories}{naturally isomorphic} to some functor $h_X$ of points for an object $X$ of $C$.

    Dually, a functor $C \to \calV$ is called \hldef{co-representable} if it is naturally isomorphic to a functor $T \mapsto \Hom_C(X, T)$ for an object $X$ in $C$. 

    For instance, we may speak of these notions when $\calV$ is the monoidal category $\Sets$, i.e. $C$ is a \CrefAndHyperrefIfExist{definition:locally_small_category}{locally small category}.
\end{definition}

\begin{theorem}[(Enriched) Yoneda lemma] \label{theorem:enriched_yoneda_lemma}
    \TODO{functor of enriched categories, natural transform}
Let \(C\) be a \CrefAndHyperrefIfExist{definition:category_enriched_in_a_monoidal_category}{$\mathcal{V}$-enriched category} where $\calV$ is a \CrefAndHyperrefIfExist{definition:monoidal_category}{monoidal category}, and let \(F: C^{\mathrm{op}} \to \mathcal{V}\) be a \(\mathcal{V}\)-functor. For any object \(X \in C\), there is a natural isomorphism:
\[ \mathrm{Nat}_{\mathcal{V}}(h_X, F) \cong F(X), \]
where \CrefAndHyperrefIfExist{definition:representable_functor_on_a_category_enriched_in_a_monoidal_category}{$h_X = \mathrm{Hom}_C(-, X)$ is the representable functor associated to \(X\)}, and \(\mathrm{Nat}_{\mathcal{V}}(h_X, F)\) is the \(\mathcal{V}\)-object of \(\mathcal{V}\)-natural transformations.

\TODO{end}
Alternatively, if \(\mathcal{V}\) is \CrefAndHyperrefIfExist{definition:closed_monoidal_category}{closed} and \CrefAndHyperrefIfExist{definition:locally_small_category}{locally small}, then
\[ \mathrm{Nat}_{\mathcal{V}}(h_X, F) \cong \int_{T \in C} [\mathrm{Hom}_C(T, X), F(T)], \]
where \([\_, \_]\) denotes the \CrefAndHyperrefIfExist{definition:internal_hom_object_in_a_category}{internal hom} in \(\mathcal{V}\) and \(\int\) is the end.

In particular, when $\calV = \Sets$ (so $C$ is simply a \CrefAndHyperrefIfExist{definition:locally_small_category}{locally small category}), the (classical) Yoneda lemma states the following: let \(F: \mathsf{C}^{\mathrm{op}} \to \mathbf{Sets}\) a functor, and let \(X\) be an object of \(\mathsf{C}\). Then there is a natural isomorphism
\[ \mathrm{Nat}(h_X, F) \cong F(X).  \]
\end{theorem}


\section{Sites, sheaves and topoi}

\begin{definition}[Presheaf on a category] \label{definition:presheaf_on_a_category}
    Let $C$ and $\mathcal{A}$ be \hyperrefIfExists{definition:category}{(large) categories}\CrefIfExists{definition:category}. 
    \begin{enumerate}
        \item A \hldef{presheaf $\mathcal{F}$ on $C$ with values in $\mathcal{A}$} is a functor
        \[
        \mathcal{F}: C^{\mathrm{op}} \to \mathcal{A}.
        \]
        In other words, a presheaf $\calF$ on $C$ with values in $\calA$ is simply a \CrefAndHyperrefIfExist{definition:functor_between_categories}{contravariant functor} from $C$ to $\calA$. 
        Explicitly, for every object $U$ in $C$, one has an object $\mathcal{F}(U)$ in $\mathcal{A}$ (called the \hldef{$U$-valued sections/sections evaluated at $U$ of $\calF$}\TextIfExists{definition:sections_of_a_presheaf_on_a_category_valued_in_a_category}{, cf. \Cref{definition:sections_of_a_presheaf_on_a_category_valued_in_a_category}}), and for every morphism $f: V \to U$ in $C$, one has a morphism (called the \hldef{restriction map})
        \[
        \mathcal{F}(f): \mathcal{F}(U) \to \mathcal{F}(V)
        \]
        in $\mathcal{A}$, such that for all composable morphisms $W \xrightarrow{g} V \xrightarrow{f} U$ in $C$, the following diagram in $\mathcal{A}$ commutes:
        \[
        \begin{tikzcd}
        \mathcal{F}(U) \arrow[r, "\mathcal{F}(f)"] \arrow[rr, bend left, "\mathcal{F}(f \circ g)"] & \mathcal{F}(V) \arrow[r, "\mathcal{F}(g)"] & \mathcal{F}(W)
        \end{tikzcd}
        \]
        That is,
        \[
        \mathcal{F}(g) \circ \mathcal{F}(f) = \mathcal{F}(f \circ g),
        \]
        and for every object $U$ in $C$, $\mathcal{F}(\mathrm{id}_U) = \mathrm{id}_{\mathcal{F}(U)}$.


        \item 
        Let $\mathcal{F},\mathcal{G}: C^{\mathrm{op}} \to \mathcal{A}$ be two presheaves on $C$ with values in $\mathcal{A}$. A \hldef{morphism of presheaves}
        \[
        \varphi: \mathcal{F} \to \mathcal{G}
        \]
        is a \hyperrefIfExists{definition:natural_transformation_between_functors_between_categories}{natural transformation of functors}\CrefIfExists{definition:natural_transformation_between_functors_between_categories}: for each object $U$ of $C$, one has a morphism
        \[
        \varphi_U: \mathcal{F}(U) \to \mathcal{G}(U)
        \]
        in $\mathcal{A}$, such that for every morphism $f: V \to U$ in $C$, the diagram
        \[
        \begin{tikzcd}
        \mathcal{F}(U) \arrow[r, "\mathcal{F}(f)"] \arrow[d, "\varphi_U"'] & \mathcal{F}(V) \arrow[d, "\varphi_V"] \\
        \mathcal{G}(U) \arrow[r, "\mathcal{G}(f)"'] & \mathcal{G}(V)
        \end{tikzcd}
        \]
        commutes, i.e.,
        \[
        \varphi_V \circ \mathcal{F}(f) = \mathcal{G}(f) \circ \varphi_U
        \]
        for all objects and morphisms in $C$.

        \item Given a \hyperrefIfExists{definition:grothendieck_universe}{universe}\CrefIfExists{definition:grothendieck_universe} $U$, a \hldef{$U$-presheaf on $\calC$} typically refers to a presheaf of $U$-sets on $C$.

        \item The \hldef{presheaf category/category of $\calA$-valued presheaves on $\calC$} is the (large) category whose objects are the presheaves on $C$ with values in $\calA$ and whose morphisms are the presheaf morphisms. Common notations for the presheaf category include, but are not limited to: \hl{$\calA^{\calC^{\op}}$}, \hl{$\PreShv(\calC, \calA)$}, \hl{$[\calC^{\op}, \calA]$}. If the value category $\calA$ is clear from context, then notations such as \hl{$\PreShv(\calC)$} are also common. \TextIfExists{definition:diagram_in_a_category_indexed_by_a_small_category}{Note that the presheaf category $\PreShv(\calC, \calA)$ is equivalent to the \CrefAndHyperrefIfExist{definition:diagram_in_a_category_indexed_by_a_small_category}{category of functors} $\calC^{\op} \to \calA$ and hence notations for the functor categories are applicable as notations for presheaf categories.}

    \end{enumerate}
\end{definition}

\begin{lemma} \label{lemma:category_of_presheaves_on_a_small_category_of_locally_small_value_is_locally_small}
    Let $\calC$ be a \hyperrefIfExists{definition:locally_small_category}{small category}\CrefIfExists{definition:locally_small_category} (resp. $U$-small category where $U$ is some \hyperrefIfExists{definition:grothendieck_universe}{universe}\CrefIfExists{definition:grothendieck_universe}) and let $\calA$ be a \CrefAndHyperrefIfExist{definition:locally_small_category}{locally small} category (resp. $U$-locally small category). The \hyperrefIfExists{definition:presheaf_on_a_category}{presheaf category $\PreShv(\calC, \calA)$}\CrefIfExists{definition:presheaf_on_a_category} is locally small (resp. $U$-locally small).
\end{lemma}
\begin{proof}
    A morphism $\calF \to \calG$ in $\PreShv(\calC, \calA)$ is a \hyperrefIfExists{definition:natural_transformation_between_functors_between_categories}{natural transformation}\CrefIfExists{definition:natural_transformation_between_functors_between_categories} of the functors $\calF, \calG: \calC^{\op} \to \calA$. Such a natural transformation is encoded by a family $(\eta_C)_C$ of morphisms (satisfying certain conditions) $\eta_C: \calF(C) \to \calG(C)$ in $\calA$ over objects $C$ of $\calC^{\op}$. The product $\prod_{C \in \Ob \calC^{\op}} \Hom_{\calA}(\calF(C), \calG(C))$ is a product of ($U$-small) sets indexed by a ($U$-small) set, and the collection of natural transformations is a subset of this set. Therefore, $\Hom_{\PreShv(\calC, \calA)}(\calF, \calG)$ is a ($U$-small) set.  
\end{proof}

\begin{remark}
    Even when $\calC$ is ($U$-)locally small and ($\calA$ is ($U$-)locally small), $\PreShv(\calC, \calA)$ may not be locally small.
\end{remark}
\begin{remark}
    In practice, one might treat $\calC$ as if it were a small category, even when it is technically not a small category, to treat $\PreShv(\calC, \calA)$ as if it were a locally small category. For example, in algebraic geometry, we might take $\calC$ to be the \hyperrefIfExists{definition:small_etale_site_of_a_scheme}{small \'etale site $X_{\et}$}\CrefIfExists{definition:small_etale_site_of_a_scheme} of some scheme $X$; this is not a small category. Instead, we might technically replace $X_{\et}$ with a small category, e.g. the set of \'etale maps $U \to X$ of the following form: $U$ is obtained by patching the schemes attached to quotients of rings of the form $A[T_1,T_2,\ldots]$ where $A = \Gamma(V, \calO_X)$ for some open affine $V \subset X$ and $\{T_1,T_2,\ldots\}$ is a fixed countable set of symbols. Such technicalities of replacing $\calC$ with a small category are often glossed over.
\end{remark}

\begin{lemma} \label{lemma:category_of_set_valued_presheaves_on_any_category_has_a_final_object}
    Let $\calC$ be a \CrefAndHyperrefIfExist{definition:category}{(large) category}.
    The category of set-valued \CrefAndHyperrefIfExist{definition:presheaf_on_a_category}{presheaves} on $\calC$ has a \CrefAndHyperrefIfExist{definition:initial_final_zero_objects_of_a_category}{final object}, namely the presheaf $\calC^{\op} \to \Sets$ which sends any object $T$ of $\calC$ to a singleton set $\{*\}$.  
\end{lemma}

% 
\begin{definition}[Small étale site of a scheme] \label{definition:small_etale_site_of_a_scheme}
Let $X$ be a fixed scheme. The \hldef{small étale site on $X$}, commonly denoted by notations including \hl{$X_{\et}$}, \hl{$X_{\mathrm{\acute{e}tale}}$}, or \hl{$\mathrm{Et}_{/X}$}, is defined as the following \hyperrefIfExists{definition:grothendieck_topology_on_a_category_site_covering_sieve_topologically_generating_family}{site}\CrefIfExists{definition:grothendieck_topology_on_a_category_site_covering_sieve_topologically_generating_family}:
\begin{itemize}
    \item The underlying category is the full subcategory of the \hyperrefIfExists{definition:big_etale_site_of_a_scheme}{big étale site $(\Sch/X)_{\et}$}\CrefIfExists{definition:big_etale_site_of_a_scheme} whose objects are schemes $U$ equipped with an étale morphism $U \to X$.
    \item The Grothendieck topology is the one \CrefAndHyperrefIfExist{definition:grothendieck_topology_generated_by_a_pretopology}{generated by} the \CrefAndHyperrefIfExist{definition:basis_and_grothendieck_pretopology_for_a_grothendieck_topology_on_a_category}{pretopology} whose covering families are families $\{g_j : U_j \to U\}_{j \in J}$ of morphisms such that each $g_j$ is an \CrefAndHyperrefIfExist{definition:etale_morphism_of_schemes}{\'etale} and the family is jointly surjective on the underlying topological spaces.
\end{itemize}
\TextIfExists{definition:big_site_on_the_category_of_schemes_over_a_scheme_and_small_site}{Equivalently, the small \'etale site on $X$ is the \CrefAndHyperrefIfExist{definition:big_site_on_the_category_of_schemes_over_a_scheme_and_small_site}{small site for \'etale morphisms on $X$}.}
\TODO{state this as a fact}
$X_{\et}$ is an \CrefAndHyperrefIfExist{definition:essentially_small_category}{essentially small category}.
\end{definition}




\subsection{Sieves, Grothendieck topologies, and sites}

\subsubsection{Sieves}
% \begin{definition} \label{definition:continuous_functor_of_sites}
Let $(\calC,J)$ and $(\calD,K)$ be \CrefAndHyperrefIfExist{definition:grothendieck_topology_on_a_category_site_covering_sieve_topologically_generating_family}{sites}. 

A functor $u : \calC \to \calD$ is said to be a \hldef{continuous functor of sites} if, for every object $U \in \operatorname{Ob}(\calD)$ and every \CrefAndHyperrefIfExist{definition:grothendieck_topology_on_a_category_site_covering_sieve_topologically_generating_family}{covering sieve} $S \in K(U)$, the \CrefAndHyperrefIfExist{definition:pullback_sieve_of_an_object_in_a_category_via_a_morphism_to_the_object}{pullback sieve $u^*S$} belongs to $J(V)$ for all $V \in \calC$ with a morphism $u(V) \to U$ in $\calD$.

Equivalently, $u$ is continuous if for every \CrefAndHyperrefIfExist{definition:sheaf_on_a_site}{sheaf} of sets $F$ on $\calD$, the \CrefAndHyperrefIfExist{definition:presheaf_on_a_category}{presheaf} $\calC^{\op} \to \Sets, X \mapsto F(u(X))$ is a sheaf on $\calC$. 
\TODO{show these are equivalent}
\TODO{define morphism of sites and recheck ref's to this definition}

% A \hldef{morphism of sites} $f: (\calD, K) \to (\calC, J)$ 

% Synonymously, we call a continuous functor $u: C \to D$ a \hldef{morphism of sites}.
\end{definition}

\begin{definition}[{\cite[Expos\'e I D\'efinition 4.1]{SGA4_I}}] \label{definition:sieve_on_an_object_in_a_category}
Let $C$ be a \CrefAndHyperrefIfExist{definition:category}{(large) category}. 

\begin{enumerate}
    \item A \hldef{sieve $S$ on the category $C$} is a \CrefAndHyperrefIfExist{definition:full_subcategory_of_a_category}{full subcategory} $D$ of $C$ such that for any object $U$ of $C$ there exists an object $V$ of \TODO{correctly parse the definiton}
    \item A \hldef{sieve $S$ on an object $U \in \operatorname{Ob}(C)$} is a collection of morphisms in $C$ with codomain $U$ that is closed under precomposition by any compatible morphism in $C$. In other words, $S$ is a sieve if for every $f : V \to U$ in $S$ and morphism $g : W \to V$ in $C$, the composition $f \circ g : W \to U$ is also in $S$. 

    Given a morphism $f: V \to U$ in a sieve $S$, we also say that \hldef{$f$ factors through $U$}.
\end{enumerate}
\end{definition}


\begin{definition}[Downward/upward closed subcategory] \label{definition:downward_upward_closed_subcategory_of_a_category}
Let $\mathcal{C}$ be a \CrefAndHyperrefIfExist{definition:category}{(large) category} and $\mathcal{D} \hookrightarrow \mathcal{C}$ a \CrefAndHyperrefIfExist{definition:full_subcategory_of_a_category}{full subcategory}. 
\begin{enumerate}
    \item We say that $\mathcal{D}$ is \hldef{downward closed} if for every object $d \in \mathrm{Ob}(\mathcal{D})$ and every morphism $f: c \to d$ in $\mathcal{C}$ (with $c \in \mathrm{Ob}(\mathcal{C})$), the domain object $c$ belongs to $\mathrm{Ob}(\mathcal{D})$ (and hence $f \in \mathcal{D}$ by fullness of the embedding).

    \item We say that $\mathcal{D}$ is \hldef{upward closed} if for every object $d \in \mathrm{Ob}(\mathcal{D})$ and every morphism $f: d \to c$ in $\mathcal{C}$ (with $c \in \mathrm{Ob}(\mathcal{C})$), the codomain object $c$ belongs to $\mathrm{Ob}(\mathcal{D})$ (and hence $f \in \mathcal{D}$ by fullness of the embedding).
\end{enumerate}
\end{definition}

\begin{proposition} \label{proposition:sieves_on_an_object_in_a_category_correspond_to_downward_closed_full_subcategories_of_the_slice_category_over_the_object}
    Let $\calC$ be a \CrefAndHyperrefIfExist{definition:category}{category}. Let $X$ be an object of $\calC$.
    The \CrefAndHyperrefIfExist{definition:sieve_on_an_object_in_a_category}{sieves} on $X$ are in bijection with the \CrefAndHyperrefIfExist{definition:downward_upward_closed_subcategory_of_a_category}{downward closed} full subcategories of the \CrefAndHyperrefIfExist{definition:category_of_objects_over_under_a_fixed_object_in_a_category}{slice category $\mathcal{C}/X$}.
\end{proposition}


\begin{definition} \label{definition:maximal_sieve_on_an_object_in_a_category}
    Let $C$ be a category and let $U$ be na object of $C$. 
    The \hldef{maximal sieve on $U$} is the collection of all morphisms with target $U$; clearly, it is a sieve on $U$.
\end{definition}
% \begin{definition}[Generated sieve] \label{definition:sieve_on_an_object_of_a_category_generated_by_a_family_of_morphisms}
%     Let $\mathcal{C}$ be a (large) category, $X \in \mathcal{C}$ an object, and $S$ a \CrefAndHyperrefIfExist{definition:sieve_on_an_object_in_a_category}{sieve} on $X$. A sieve $S$ is said to be \hldef{generated} by a family of morphisms $\{f_i : U_i \to X\}_{i \in I}$ if $S$ is the smallest sieve on $X$ containing all the morphisms $f_i$, i.e., $S$ consists precisely of all morphisms $g : Y \to X$ such that $g$ factors through some $f_i$.
% \end{definition}

\begin{definition} \label{definition:sieve_on_an_object_of_a_category_generated_by_a_family_of_morphisms}
Let $\mathcal{C}$ be a \CrefAndHyperrefIfExist{definition:category}{category} and $U \in \mathcal{C}$ an object. Let $\mathcal{S} = \{f_i: U_i \to U\}_{i \in I}$ be a family of morphisms with codomain $U$. 

The \hldef{sieve generated by $\mathcal{S}$}, denoted \hl{$(\mathcal{S})$} or \hl{$\langle \mathcal{S} \rangle$}, is the smallest \CrefAndHyperrefIfExist{definition:sieve_on_an_object_in_a_category}{sieve on $U$} containing all the morphisms in $\mathcal{S}$.

Explicitly, a morphism $h: V \to U$ belongs to the generated sieve if and only if $h$ factors through some morphism in $\mathcal{S}$. That is, there exists an index $i \in I$ and a morphism $g: V \to U_i$ such that
$$ h = f_i \circ g. $$
\end{definition}


\begin{definition} \label{definition:pullback_sieve_of_an_object_in_a_category_via_a_morphism_to_the_object}
Let $C$ be a category, let $U \in \operatorname{Ob}(C)$, and let $S$ be a \CrefAndHyperrefIfExist{definition:sieve_on_an_object_in_a_category}{sieve on $U$}.
For a morphism $f : V \to U$ in $C$, the \hldef{pullback sieve} \hl{$f^*S$} (or \hldef{basechange sieve} \hl{$S \times_U V$}) on $V$ is defined by
\[
f^*S = \{ g : W \to V \mid f \circ g \in S \}.
\]
In other words, $f^*S$ consists of all morphisms into $V$ whose composite with $f$ belongs to the sieve $S$ on $U$.
\end{definition}

\begin{lemma} \label{lemma:pullback_sieve_on_an_object_in_a_category_via_a_morphism_to_the_object_is_indeed_a_sieve}
    Let $C$ be a category, let $U \in \operatorname{Ob}(C)$, and let $S$ be a \CrefAndHyperrefIfExist{definition:sieve_on_an_object_in_a_category}{sieve on $U$}. Let $f: V \to U$ be a morphism in $C$. The \CrefAndHyperrefIfExist{definition:pullback_sieve_of_an_object_in_a_category_via_a_morphism_to_the_object}{pullback sieve} $f^* S$ is indeed a \CrefAndHyperrefIfExist{definition:sieve_on_an_object_in_a_category}{sieve} on $V$.
\end{lemma}
\begin{proof}
    Let $g: W \to V$ be a morphism in $f^* S$, and let $h: X \to W$ be a morphism in $C$. The composition $g \circ h: X \to V$ is also in $f^* S$ because $f \circ (g \circ h) = (f \circ g) \circ h$, the map $(f \circ g)$ is in $S$, and $S$ is a sieve.
\end{proof}


\subsubsection{Grothendieck pretopologies}

\begin{definition} \label{definition:basis_and_grothendieck_pretopology_for_a_grothendieck_topology_on_a_category}
Let $\mathcal{C}$ be a \CrefAndHyperrefIfExist{definition:category}{category}.
A \hldef{basis for a Grothendieck topology} (also called a \hldef{Grothendieck pretopology} or simply a \hldef{pretopology}) on $\mathcal{C}$ is a collection of families $K(U)$ of morphisms for each object $U \in \mathcal{C}$, called \hldef{coverings} or \hldef{covering families}, satisfying the following axioms:
\begin{enumerate}
    \item \textbf{(Identity)} For every isomorphism $U' \to U$, the singleton family $\{U' \to U\}$ is in $K(U)$.
    \item \textbf{(Base Change)} If $\{U_i \to U\}_{i \in I}$ is a covering family in $K(U)$ and $V \to U$ is any morphism in $\mathcal{C}$, then the \CrefAndHyperrefIfExist{definition:cartesian_product_of_two_objects_in_a_category_over_an_object}{fiber products} $U_i \times_U V$ exist, and the family of projections $\{U_i \times_U V \to V\}_{i \in I}$ is in $K(V)$.
    \item \textbf{(Composition)} If $\{U_i \to U\}_{i \in I}$ is in $K(U)$ and for each $i \in I$, $\{V_{ij} \to U_i\}_{j \in J_i}$ is in $K(U_i)$, then the composite family $\{V_{ij} \to U_i \to U\}_{i \in I, j \in J_i}$ is in $K(U)$.
\end{enumerate}

\end{definition}

\subsubsection{Coverages}



\subsubsection{Grothendieck topologies}

% \begin{definition}[Grothendieck topology] \label{definition:grothendieck_topology_on_a_category_site_covering_sieve_topologically_generating_family}
%     Let $\mathscr{U}$ be a \hyperrefIfExists{definition:grothendieck_universe}{universe}\CrefIfExists{definition:grothendieck_universe} and let $\calC$ be a \hyperrefIfExists{definition:locally_small_category}{locally small category}\CrefIfExists{definition:locally_small_category}.

%     \begin{enumerate}
%         \item \textbf{(Grothendieck Topology via Sieves)}
%         A \hldef{Grothendieck topology} $J$ on $\calC$ is an assignment to each object $U \in \calC$ of a collection $J(U)$ of \CrefAndHyperrefIfExist{definition:sieve_on_an_object_in_a_category}{sieves} on $U$, called \hldef{covering sieves}, satisfying:
%         \begin{enumerate}
%             \item (Maximality) The maximal \CrefAndHyperrefIfExist{definition:sieve_on_an_object_in_a_category}{sieve} $\{ f : V \to U \mid V \in \calC \}$ is in $J(U)$.
%             \item (Stability) If $S \in J(U)$ and $f : V \to U$ is any morphism, then the \CrefAndHyperrefIfExist{definition:pullback_sieve_of_an_object_in_a_category_via_a_morphism_to_the_object}{pullback sieve} $f^{*}S$ is in $J(V)$.
%             \item (Transitivity/Local Character) If $S$ is a sieve on $U$ and there exists a covering sieve $R \in J(U)$ such that for every morphism $f : V \to U$ in $R$, the pullback sieve $f^{*}S$ is in $J(V)$, then $S \in J(U)$.
%         \end{enumerate}

%         % \item \textbf{(Grothendieck Pretopology / Basis)}
%         % If $\calC$ admits fiber products, one can define a topology via \hldef{covering families}. A \hldef{Grothendieck pretopology} (or basis) is a collection $K(U)$ of families $\{U_i \to U\}_{i \in I}$ for each object $U$, satisfying:
%         % \begin{itemize}
%         %     \item (Isomorphism) $\{U' \xrightarrow{\sim} U\} \in K(U)$ for any isomorphism.
%         %     \item (Stability) If $\{U_i \to U\} \in K(U)$ and $V \to U$ is a morphism, then $\{U_i \times_U V \to V\} \in K(V)$.
%         %     \item (Composition) If $\{U_i \to U\} \in K(U)$ and for each $i$, $\{V_{ij} \to U_i\} \in K(U_i)$, then the composite family $\{V_{ij} \to U\} \in K(U)$.
%         % \end{itemize}
%         % Every pretopology generates a unique Grothendieck topology $J$, where $S \in J(U)$ iff $S$ contains a covering family from the pretopology.

%         \item A \hldef{site} is a pair $(\calC, J)$ consisting of a category $\calC$ and a Grothendieck topology $J$.

%         \item A family of objects $\mathcal{G} = \{G_\alpha\}$ in a site $(\calC, J)$ is called a \hldef{topologically generating family} if for every object $X \in \calC$, there exists a covering sieve $S \in J(X)$ \CrefAndHyperrefIfExist{definition:sieve_on_an_object_of_a_category_generated_by_a_family_of_morphisms}{generated by} morphisms with domains in $\mathcal{G}$. Equivalently, every object $X$ admits a cover $\{U_i \to X\}$ where each $U_i \in \mathcal{G}$.

%         \item A \hldef{$\mathscr{U}$-site} is a site whose underlying category is $\mathscr{U}$-locally small and which admits a $\mathscr{U}$-small topologically generating family.
%     \end{enumerate}
% \end{definition}

\begin{definition}[Grothendieck topology] \label{definition:grothendieck_topology_on_a_category_site_covering_sieve_topologically_generating_family}
    Let $\mathscr{U}$ be a \hyperrefIfExists{definition:grothendieck_universe}{universe}\CrefIfExists{definition:grothendieck_universe}.
    \begin{enumerate}
        % \item Let $C$ be a \hyperrefIfExists{definition:locally_small_category}{locally small category}\CrefIfExists{definition:locally_small_category}. A \hldef{Grothendieck topology on $C$} assigns to each object $U$ of $C$ a collection of families of morphisms $\{U_i \to U\}_{i \in I}$, called \hldef{coverings of $U$}, satisfying:
        % \begin{itemize}
        %     \item (Isomorphism) If $f: V \to U$ is an isomorphism in $C$, then $\{f: V \to U\}$ is a covering of $U$.
        %     \item (Stability under base change) If $\{U_i \to U\}_{i \in I}$ is a covering of $U$ and $V \to U$ is any morphism, then the family $\{ U_i \times_U V \to V \}_{i \in I}$ is a covering of $V$.
        %     \item (Transitivity) If $\{U_i \to U\}_{i \in I}$ is a covering of $U$ and for each $i$, $\{V_{ij} \to U_i\}_{j \in J_i}$ is a covering of $U_i$, then the family $\{ V_{ij} \to U \}_{i \in I,\, j \in J_i}$ is a covering of $U$.
        % \end{itemize}

        \item (See \cite[Expos\'e II, D\'efinition 1.1]{SGA4_I}) Let $\calC$ be a \CrefAndHyperrefIfExist{definition:category}{category}. A \hldef{Grothendieck topology on $\calC$} assigns to each object $U$ of $\calC$ a collection \hl{$J(U)$} of \CrefAndHyperrefIfExist{definition:sieve_on_an_object_in_a_category}{sieves} $\{U_i \to U\}_{i \in I}$, each called a \hldef{covering sieve of $U$}, satisfying:
        \begin{enumerate}
            \item (Stability under ``base change''): If $S \in J(U)$ is a covering sieve of an object $U$, and $f: V \to U$ is any morphism in $\calC$, then the \CrefAndHyperrefIfExist{definition:pullback_sieve_of_an_object_in_a_category_via_a_morphism_to_the_object}{pullback sieve} $f^* S$ is a covering sieve of $U$.
            % \item (Local character condition) If $F$ is a sieve on $U$ such that the sieve $\bigcup_...$ \TODO{}
            \item (Local character condition) If $S$ is a sieve on $U$, and if there exists a covering sieve $R \in J(U)$ such that for all $f: V \to U$ in $R$ the \CrefAndHyperrefIfExist{definition:pullback_sieve_of_an_object_in_a_category_via_a_morphism_to_the_object}{pullback sieve} $f^* S$ is in $J(V)$, then $S \in J(U)$. 
            
            \item The \CrefAndHyperrefIfExist{definition:maximal_sieve_on_an_object_in_a_category}{maximal sieve} is a covering sieve.
        \end{enumerate}


        % Equivalently, a Grothendieck topology $J$ on a category $C$ is an assignment of a collection $J(U)$ of \CrefAndHyperrefIfExist{definition:sieve_on_an_object_in_a_category}{sieves} on each object $U \in \operatorname{Ob}(C)$ such that:
        % \begin{enumerate}
        %     \item the maximal \CrefAndHyperrefIfExist{definition:sieve_on_an_object_in_a_category}{sieve} $\{ f : V \to U \mid f \in \operatorname{Mor}(C) \}$ belongs to $J(U)$,
        %     \item if $S \in J(U)$ and $f : V \to U$, then the \CrefAndHyperrefIfExist{definition:pullback_sieve_of_an_object_in_a_category_via_a_morphism_to_the_object}{pullback sieve $f^{*}S$} on $V$ belongs to $J(V)$,
        %     \item if $S$ is a sieve on $U$, and if there exists $R \in J(U)$ such that for all $f : V \to U$ in $R$ the \CrefAndHyperrefIfExist{definition:pullback_sieve_of_an_object_in_a_category_via_a_morphism_to_the_object}{pullback sieve $f^{*}S$} is in $J(V)$, then $S \in J(U)$.
        % \end{enumerate}

        Some will refer to a Grothendieck topology as simply a \hldef{topology}, not to be confused with the related, but less general, notion of a \CrefAndHyperrefIfExist{definition:topological_space}{topology on a set}.


        \item (See \cite[Expos\'e II, 1.1.5]{SGA4_I}) A \hldef{site} is a category $\calC$ equipped with a Grothendieck topology.

        When we are working with a \CrefAndHyperref{definition:basis_and_grothendieck_pretopology_for_a_grothendieck_topology_on_a_category}{Grothendieck pretopology} $K$ on a category $\calC$, we may regard $\calC$ as a site by equipping it with the \CrefAndHyperref{definition:grothendieck_topology_generated_by_a_pretopology}{Grothendieck topology generated by} $K$. 

        \item (See \cite[Expos\'e II, D\'efinition 1.2]{SGA4_I}) Let $(\calC, J)$ be a site. A family of morphisms $(U_i \to U)_{i \in I}$ is called a \hldef{covering family of $U$ (with respect to the site/topology)} or a \hldef{cover of $U$ (with respect to the site/topology)} if the \CrefAndHyperrefIfExist{definition:sieve_on_an_object_of_a_category_generated_by_a_family_of_morphisms}{sieve generated by} the family is a covering sieve of $U$. 

        \item (See \cite[Expos\'e II, D\'efinition 3.0.1]{SGA4_I}) Let $(\calC, J)$ be a \CrefAndHyperrefIfExist{definition:grothendieck_topology_on_a_category_site_covering_sieve_topologically_generating_family}{site}, where $J$ is a Grothendieck topology on $\calC$.

        A family $G$ of objects $\calC$ is called a \hldef{topologically generating family of the site/topology} or a \hldef{generating family/collection of the site/topology} if for every object $X \in \calC$, there is a covering family $\{X_\alpha \to X\}_{\alpha \in A}$ of $X$ such that every $X_\alpha$ is a member of $G$.  

        Equivalently, the Grothendieck topology $J$ is the smallest Grothendieck topology containing all covers of the $U_i$. Also equivalently, for any $S \in J(X)$, the sieve $S$ contains a covering family $\{V_i \to X\}$ such that each morphism $V_i \to X$ factors through some member of $G$. \TODO{Verify that these claimed equivalences are indeed equivalences}
        
        % A family of objects $\{U_i\}_{i \in I}$ in $\calC$ is called a \hldef{topologically generating family} if for every object $X \in \calC$ and every covering sieve $S \in J(X)$, the sieve $S$ is \CrefAndHyperrefIfExist{definition:sieve_on_an_object_of_a_category_generated_by_a_family_of_morphisms}{generated by} pullbacks of covering families from the family $\{U_i\}$.

        % More precisely, this means that for any $S \in J(X)$, the sieve $S$ contains a covering family $\{V_j \to X\}$ such that each morphism $V_j \to X$ factors through some $U_i$, and the covering families of the $U_i$ generate the topology $J$. 
        % Equivalently, the Grothendieck topology $J$ is the smallest Grothendieck topology containing all coverings of the $U_i$.

        % When one speaks of a \hldef{generating family/collection} of a site, one usually refers to the above notion of a topologically generating family.

        \item (See \cite[Expos\'e II, D\'efinition 3.0.2]{SGA4_I}) A \hldef {$\mathscr{U}$-site} is a site whose underlying category $\calC$ is \hyperrefIfExists{definition:locally_small_category}{$\mathscr{U}$-locally small}\CrefIfExists{definition:locally_small_category} and which has a $\mathscr{U}$-small topologically generating family. A $\mathscr{U}$-site is called \hldef{$\mathscr{U}$-small} if its underlying category is $\mathscr{U}$-small. Similarly, a \hldef{small site} is a site whose underlying category is a set and a \hldef{locally small site} is a site whose underlying category is \CrefAndHyperrefIfExist{definition:locally_small_category}{locally small}.
    \end{enumerate}
\end{definition}

\begin{definition} \label{definition:grothendieck_topology_generated_by_a_pretopology}
Let $\mathcal{C}$ be a category equipped with a \CrefAndHyperrefIfExist{definition:basis_and_grothendieck_pretopology_for_a_grothendieck_topology_on_a_category}{Grothendieck pretopology} $K$. The \hldef{Grothendieck topology generated by $K$}, denoted \hl{$J_K$}, is the smallest \CrefAndHyperrefIfExist{definition:grothendieck_topology_on_a_category_site_covering_sieve_topologically_generating_family}{Grothendieck topology} on $\mathcal{C}$ such that every family in $K(U)$ is a covering family for $J_K$.

Explicitly, a \CrefAndHyperrefIfExist{definition:sieve_on_an_object_in_a_category}{sieve} $S$ on an object $U$ belongs to $J_K(U)$ if and only if there exists a \CrefAndHyperrefIfExist{definition:basis_and_grothendieck_pretopology_for_a_grothendieck_topology_on_a_category}{covering family} $\{U_i \to U\}_{i \in I} \in K(U)$ such that for every $i \in I$, the morphism $U_i \to U$ belongs to $S$.

The condition that $S$ contains the family $\{U_i \to U\}$ is equivalent to saying that the sieve generated by this family is a sub-sieve of $S$.
\end{definition}

\begin{definition} \label{definition:coarser_finer_than_for_sites_on_a_category}
Let $(\mathcal{C}, J)$ and $(\mathcal{C}, J')$ be two \CrefAndHyperrefIfExist{definition:grothendieck_topology_on_a_category_site_covering_sieve_topologically_generating_family}{sites} on the same underlying category $\mathcal{C}$. 
We say that the topology $J$ is \hldef{finer than $J'$} (or equivalently, that \hldef{$J'$ is coarser than $J$}) if for every object $U$ in $\mathcal{C}$, there is an inclusion
$$ J'(U) \subseteq J(U) $$
That is, every \CrefAndHyperrefIfExist{definition:sieve_on_an_object_in_a_category}{sieve} on $U$ which is a covering sieve for $J'$ is also a covering sieve for $J$.
\end{definition}

\begin{definition}[cf. {\cite[Expos\'e III, 3.1]{SGA4_I}}] \label{definition:induced_topology_on_a_category_by_a_functor_to_a_site}
Let $\mathcal{C}, \mathcal{D}$ be \CrefAndHyperrefIfExist{definition:category}{categories}, let $u: \mathcal{C} \to \mathcal{D}$ be a \CrefAndHyperrefIfExist{definition:functor_between_categories}{functor}, and let $(\mathcal{D}, K)$ be a \CrefAndHyperrefIfExist{definition:grothendieck_topology_on_a_category_site_covering_sieve_topologically_generating_family}{site}. 
The \hldef{induced topology} \hl{$u^*K$} (also called the \hldef{pullback topology)} on $\mathcal{C}$ is the \CrefAndHyperrefIfExist{definition:coarser_finer_than_for_sites_on_a_category}{finest} \CrefAndHyperrefIfExist{definition:grothendieck_topology_on_a_category_site_covering_sieve_topologically_generating_family}{Grothendieck topology} making $u$ a \CrefAndHyperrefIfExist{definition:continuous_functor_of_sites}{continuous functor}. Equivalently, it is the Grothendieck topology given as follows: for every object $V \in \mathcal{C}$, a \CrefAndHyperrefIfExist{definition:sieve_on_an_object_in_a_category}{sieve} $S \in J_{u^*K}(V)$ is a covering sieve if and only if its image $u(S)$ under $u$ is a covering sieve in $K(u(V))$.
\end{definition}

\begin{proposition} \label{proposition:functor_from_induced_site_to_site_is_continuous_functor}
Let $\mathcal{C}, \mathcal{D}$ be \CrefAndHyperrefIfExist{definition:category}{categories}, let $u: \mathcal{C} \to \mathcal{D}$ be a \CrefAndHyperrefIfExist{definition:functor_between_categories}{functor}, and let $(\mathcal{D}, K)$ be a \CrefAndHyperrefIfExist{definition:grothendieck_topology_on_a_category_site_covering_sieve_topologically_generating_family}{site}.

The functor $u: (\mathcal{C}, u^*K) \to (\mathcal{D}, K)$ (\Cref{definition:induced_topology_on_a_category_by_a_functor_to_a_site}) is \CrefAndHyperrefIfExist{definition:continuous_functor_of_sites}{continuous}.
\end{proposition}

\begin{theorem} [Comparison Lemma, cf. {{\cite[Expos\'e III, Th\'eor\`eme 4.1]{SGA4_I}}}] \label{theorem:comparison_lemma_of_sheaves_on_a_site_and_sheaves_on_a_small_topologically_generating_family}
    Let $\calC$ be a \CrefAndHyperrefIfExist{definition:locally_small_category}{small category}, let $(\calD, K)$ be a \CrefAndHyperrefIfExist{definition:grothendieck_topology_on_a_category_site_covering_sieve_topologically_generating_family}{site}, let $u: \calC \to \calD$ be a \CrefAndHyperrefIfExist{definition:full_and_faithful_functor_between_locally_small_categories}{fully faithful functor}, and equip $\calC$ with the \CrefAndHyperrefIfExist{definition:grothendieck_topology_on_a_category_site_covering_sieve_topologically_generating_family}{Grothendieck topology} $J$ \CrefAndHyperrefIfExist{definition:induced_topology_on_a_category_by_a_functor_to_a_site}{induced by} $u$. Let $\calA$ be a \CrefAndHyperref{definition:locally_small_category}{locally small} \CrefAndHyperrefIfExist{definition:complete_and_cocomplete_category}{complete category}, i.e. a category which admits all \CrefAndHyperrefIfExist{definition:small_and_finite_limits_and_colimits_in_a_category}{small limits}.

    If the objects of $\calC$ form a \CrefAndHyperrefIfExist{definition:grothendieck_topology_on_a_category_site_covering_sieve_topologically_generating_family}{topologically generating family}\TODO{I need to check that the condition of covering in SGA really matches the notion of topologically generating family} of $\calD$, then the \CrefAndHyperrefIfExist{definition:direct_image_of_a_sheaf_on_a_site_under_a_continuous_functor_of_sites_or_a_site_morphism}{direct image}/restriction functor 
    $$u_s: \Sh(\calD, K; \calA) \to \Sh(\calC, J; \calA)$$
    is an \CrefAndHyperrefIfExist{definition:equivalence_of_categories}{equivalence of categories}. In particular, $\Sh(\calD, K; \calA)$ is a \CrefAndHyperrefIfExist{definition:locally_small_category}{locally small category}.
\end{theorem}
\begin{proof}
We explicitly construct the inverse functor and prove the equivalence without assuming \textit{a priori} that $\text{Sh}(\mathcal{D}, K)$ is locally small.

Step 1: Construction of the Inverse Functor $u_*$

We define the functor $u_*: \text{Sh}(\mathcal{C}, J) \to \text{Sh}(\mathcal{D}, K)$ as the \textbf{Right Kan Extension} of a sheaf $F$ along $u$.

For a sheaf $F \in \text{Sh}(\mathcal{C}, J)$ and an object $d \in \mathcal{D}$, define:
\[ u_*F(d) = \lim_{(c, f) \in (u \downarrow d)^{op}} F(c) \]
Here, the limit is taken over the opposite of the comma category $(u \downarrow d)$, whose objects are pairs $(c, f: u(c) \to d)$.

\textit{Verification of Well-Definedness (Size):} Since $\mathcal{C}$ is a small category and $\mathcal{D}$ is locally small, the collection of objects in $(u \downarrow d)$ is a set (indexed by objects of $\mathcal{C}$ and hom-sets of $\mathcal{D}$). Therefore, the limit defining $u_*F(d)$ is a \textbf{small limit of sets}, which exists in $\mathbf{Set}$. Thus, $u_*F$ takes values in $\mathbf{Set}$ rather than proper classes.

\textit{Verification that $u_*F$ is a Sheaf:} Since limits commute with limits, and the sheaf condition is a limit condition, the Right Kan Extension of a sheaf along a continuous functor is a sheaf. The density condition ensures that covers in $\mathcal{D}$ are "seen" by $\mathcal{C}$, ensuring the sheaf condition is preserved.

Step 2: The Unit of Adjunction ($u^* u_* \cong \text{Id}$)

We examine $u^* u_* F$ for $F \in \text{Sh}(\mathcal{C}, J)$. Evaluating at an object $c_0 \in \mathcal{C}$:
\[ (u^* u_* F)(c_0) = u_*F(u(c_0)) = \lim_{(c, f) \in (u \downarrow u(c_0))^{op}} F(c) \]
Since $u$ is \textbf{fully faithful}, the comma category $(u \downarrow u(c_0))$ has a terminal object: $(c_0, \text{id}_{u(c_0)})$. The limit over a category with a terminal object is isomorphic to the value at that object. Thus, $(u^* u_* F)(c_0) \cong F(c_0)$, concluding $u^* u_* \cong \text{Id}_{\text{Sh}(\mathcal{C})}$.

Step 3: The Counit of Adjunction ($H \cong u_* u^* H$)

This is the critical step that uses the \textbf{Density Condition} to control the size of sheaves on $\mathcal{D}$. Let $H \in \text{Sh}(\mathcal{D}, K)$. We construct a map $\eta_d: H(d) \to u_*(u^*H)(d)$. By definition:
\[ u_*(u^*H)(d) = \lim_{u(c) \to d} H(u(c)) \]
There is a canonical map $\eta_d$ induced by the morphisms $H(f): H(d) \to H(u(c))$ for each $f: u(c) \to d$. 

To see that $\eta_d$ is an isomorphism:
\begin{enumerate}
    \item \textbf{Density as a Cover:} By hypothesis, the family of morphisms $\mathcal{S} = \{ f: u(c) \to d \mid c \in \mathcal{C} \}$ generates a covering sieve $S$ on $d$.
    \item \textbf{Sheaf Property:} Since $H$ is a sheaf on $(\mathcal{D}, K)$, $H(d) \cong \text{Match}(S, H)$.
    \item \textbf{Matching Families:} The limit over the comma category $(u \downarrow d)$ is exactly the set of compatible families indexed by the generators of the sieve. Because $\mathcal{S}$ generates $S$, the data of a matching family on $\mathcal{S}$ extends uniquely to the whole sieve $S$.
\end{enumerate}
The canonical map $H(d) \to \lim_{u(c) \to d} H(u(c))$ is therefore an isomorphism, so $H \cong u_* (u^* H)$.

Step 4: Conclusion of Equivalence and Local Smallness

We have established natural isomorphisms $u^* u_* \cong \text{Id}$ and $\text{Id} \cong u_* u^*$, establishing an equivalence. Since $\mathcal{C}$ is small, $\text{Sh}(\mathcal{C}, J)$ is a locally small category. Since $\text{Sh}(\mathcal{D}, K)$ is equivalent to it, $\text{Sh}(\mathcal{D}, K)$ is itself locally small. Specifically:
\[ \text{Hom}_{\mathcal{D}}(H, G) \cong \text{Hom}_{\mathcal{C}}(u^*H, u^*G) \]
where the latter is a set.



    Since $\calC$ is small and $\calA$ is locally small, the category of presheaves on $\calC$ valued in $\calA$ is locally small by \Cref{lemma:category_of_presheaves_on_a_small_category_of_locally_small_value_is_locally_small}. Therefore, the category of sheaves is locally small.
\end{proof}


\begin{proposition}[{\cite[Expos\'e II, Proposition 6.3.1]{SGA4_I}}] \label{proposition:category_of_sheaves_with_values_in_gamma_sets_is_equivalent_to_category_of_sheaves_of_gamma_objects_in_the_category_of_sheaves_of_sets}
    Let $\calC$ be a \CrefAndHyperrefIfExist{definition:grothendieck_topology_on_a_category_site_covering_sieve_topologically_generating_family}{site}, and let $\gamma$ be a ``kind of algebraic structure defined by finite projective limits'' \TODO{Probably the correct terminology for this is ``essentially algebraic categories'' or ``Lawvere theories''} (e.g. $\gamma$ can be the notion of \CrefAndHyperrefIfExist{definition:group_object_in_a_category_with_a_final_object}{group objects or abelian group objects}, \CrefAndHyperrefIfExist{definition:ring_object_in_a_category_with_a_terminal_object}{ring objects}, or \CrefAndHyperrefIfExist{definition:module_object_over_a_ring_object_in_a_category_with_a_final_object}{module objects}).
    The following categories are \CrefAndHyperrefIfExist{definition:equivalence_of_categories}{equivalent}:
    \begin{enumerate}
        \item The category of \CrefAndHyperrefIfExist{definition:sheaf_on_a_site}{sheaves} on $\calC$ with values in the category of $\gamma$-sets,
        \item The category of $\gamma$-objects of the category of sheaves of sets on $\calC$, and
        \item The category of $\gamma$-objects of the category of \CrefAndHyperrefIfExist{definition:presheaf_on_a_category}{presheaves} of sets on $\calC$ whose underlying presheaf is a sheaf of sets.
    \end{enumerate}
\end{proposition}


\begin{definition} \label{definition:quasi_compact_finitely_presentable_object_in_a_locally_small_category}
Let $\mathcal{C}$ be a \CrefAndHyperrefIfExist{definition:locally_small_category}{locally small} \CrefAndHyperrefIfExist{definition:category}{category}.
An object $X$ in $\mathcal{C}$ is called \hldef{quasi-compact} (or \hldef{compact}, or \hldef{finitely presentable} depending on context) if the functor $h^X = \hom_{\mathcal{C}}(X, -)$ \CrefAndHyperrefIfExist{definition:representable_functor_on_a_category_enriched_in_a_monoidal_category}{represented by} $X$, preserves \CrefAndHyperrefIfExist{definition:projective_and_inductive_limits_in_categories}{filtered colimits}.

Explicitly, this means that for any \CrefAndHyperrefIfExist{definition:diagram_in_a_category_indexed_by_a_small_category}{filtered diagram} $D: \mathcal{I} \to \mathcal{C}$ such that the \CrefAndHyperrefIfExist{definition:limit_and_colimit_of_a_diagram_in_a_category}{colimit} $\colim_{i \in \mathcal{I}} D_i$ exists in $\mathcal{C}$, the canonical map
$$\colim_{i \in \mathcal{I}} \hom_{\mathcal{C}}(X, D_i) \xrightarrow{\cong} \hom_{\mathcal{C}}(X, \colim_{i \in \mathcal{I}} D_i)$$
is a bijection.
\end{definition}

% ---------------------------------------------------------------------------
% 1. Quasi-compact Object in a Site
% ---------------------------------------------------------------------------

\begin{definition} \label{definition:quasi_compact_object_in_a_locally_small_site}
Let $(\mathcal{C}, J)$ be a \CrefAndHyperrefIfExist{definition:grothendieck_topology_on_a_category_site_covering_sieve_topologically_generating_family}{site}. Assume that $\mathcal{C}$ is \CrefAndHyperrefIfExist{definition:locally_small_category}{locally small}.

An object $U$ in $\mathcal{C}$ is called \hldef{quasi-compact (relative to the topology $J$)} if every \CrefAndHyperrefIfExist{definition:grothendieck_topology_on_a_category_site_covering_sieve_topologically_generating_family}{covering sieve} $S \in J(U)$ contains a finite set of morphisms $\{f_i: V_i \to U\}_{i=1}^n$ which \CrefAndHyperrefIfExist{definition:sieve_on_an_object_of_a_category_generated_by_a_family_of_morphisms}{generates} a \CrefAndHyperrefIfExist{definition:grothendieck_topology_on_a_category_site_covering_sieve_topologically_generating_family}{covering sieve}.

Equivalently, if coverings are defined via a \CrefAndHyperrefIfExist{definition:basis_and_grothendieck_pretopology_for_a_grothendieck_topology_on_a_category}{basis (or pretopology)}, an object $U$ is quasi-compact if every covering family $\{U_i \to U\}_{i \in I}$ admits a finite subfamily $\{U_i \to U\}_{i \in I_0}$ (with $I_0 \subseteq I$ finite) which is also a covering family.
\end{definition}
\begin{definition} \label{definition:quasi_compact_morphism_in_a_locally_small_site}
    \TODO{does this need local smallness}
Let $(\mathcal{C}, J)$ be a \CrefAndHyperrefIfExist{definition:locally_small_category}{locally small} \CrefAndHyperrefIfExist{definition:grothendieck_topology_on_a_category_site_covering_sieve_topologically_generating_family}{site}.
A morphism $f: X \to Y$ in $\mathcal{C}$ is called \hldef{quasi-compact} if for every morphism $Z \to Y$ where the source $Z$ is a \CrefAndHyperrefIfExist{definition:quasi_compact_object_in_a_locally_small_site}{quasi-compact object with respect to $J$}, the \CrefAndHyperrefIfExist{definition:cartesian_product_of_two_objects_in_a_category_over_an_object}{fiber product} $X \times_Y Z$ exists and is also a quasi-compact object with respect to $J$.
\end{definition}
\begin{definition}  \label{definition:quasi_seperated_morphism_in_a_locally_small_site}
Let $(\mathcal{C}, J)$ be a \CrefAndHyperrefIfExist{definition:locally_small_category}{locally small} \CrefAndHyperrefIfExist{definition:grothendieck_topology_on_a_category_site_covering_sieve_topologically_generating_family}{site}.

An object $X$ in $\mathcal{C}$ is called \hldef{quasi-separated with respect to $J$} if the \CrefAndHyperrefIfExist{definition:product_and_coproduct_of_objects_in_a_category}{product} $X \times X$ exists and the \CrefAndHyperrefIfExist{definition:diagonal_morphism_of_an_object_over_an_object_in_a_category}{diagonal morphism} \hl{$\Delta: X \to X \times X$} is a \CrefAndHyperrefIfExist{definition:quasi_compact_morphism_in_a_site}{quasi-compact morphism with respect to $J$}.

Explicitly, this means that for any two morphisms $U \to X$ and $V \to X$ from quasi-compact objects $U$ and $V$, the fiber product $U \times_X V$ is a quasi-compact object, assuming that the relevant fiber products exist.
\end{definition}
\begin{definition} \label{definition:coherent_object_in_a_locally_small_site}
Let $(\mathcal{C}, J)$ be a site where $\mathcal{C}$ is \CrefAndHyperrefIfExist{definition:locally_small_category}{locally small}. An object $X \in \mathcal{C}$ is called a \hldef{coherent object} if it satisfies two conditions:
\begin{enumerate}
    \item $X$ is \CrefAndHyperrefIfExist{definition:quasi_compact_object_in_a_locally_small_site}{quasi-compact} relative to the topology $J$.
    \item $X$ is \CrefAndHyperrefIfExist{definition:quasi_seperated_morphism_in_a_locally_small_site}{quasi-separated} relative to the topology $J$.
\end{enumerate}
\end{definition}
\begin{definition} \label{definition:coherent_site}
A \CrefAndHyperrefIfExist{definition:grothendieck_topology_on_a_category_site_covering_sieve_topologically_generating_family}{site} $(\mathcal{C}, J)$ is called a \hldef{coherent site} if it satisfies the following conditions:
\begin{enumerate}
    \item The category $\mathcal{C}$ is locally small with a small \CrefAndHyperrefIfExist{definition:grothendieck_topology_on_a_category_site_covering_sieve_topologically_generating_family}{topologically generating family}.
    \item The category $\mathcal{C}$ admits all finite \CrefAndHyperrefIfExist{definition:limit_and_colimit_of_a_diagram_in_a_category}{limits}.
    \item Every object in $\mathcal{C}$ is \CrefAndHyperrefIfExist{definition:quasi_compact_object_in_a_locally_small_site}{quasi-compact relative to} the topology $J$.
    \item The collection of quasi-compact objects relative to $J$ is closed under \CrefAndHyperrefIfExist{definition:cartesian_product_of_two_objects_in_a_category_over_an_object}{fiber products}. Explicitly, if $X \to Z$ and $Y \to Z$ are morphisms where $X, Y, Z$ are quasi-compact, then $X \times_Z Y$ is also quasi-compact.
\end{enumerate}
\end{definition}
\begin{definition}[Slice site] \label{definition:site_induced_by_a_site_on_an_over_category}
Let $(\mathcal{C}, \tau)$ be a \CrefAndHyperrefIfExist{definition:grothendieck_topology_on_a_category_site_covering_sieve_topologically_generating_family}{site}, where $\tau$ is a Grothendieck topology on the (\CrefAndHyperrefIfExist{definition:locally_small_category}{locally small or $U$-locally small}, if a \CrefAndHyperrefIfExist{definition:grothendieck_universe}{universe} $U$ is available) category $\mathcal{C}$. For a fixed object $X$ in $\mathcal{C}$, the \hldef{slice site} (or the \hldef{over site}, the \hldef{site on the slice category $\mathcal{C}_{/X}$}, the \hldef{site induced on the over category $\mathcal{C}_{/X}$}, the \hldef{localization of the site $\calC$ at the object $X$}, etc.) $(\mathcal{C}_{/X}, \tau_{/X})$ is the site whose underlying category is the \CrefAndHyperrefIfExist{definition:category_of_objects_over_under_a_fixed_object_in_a_category}{slice category $\mathcal{C}_{/X}$}, and whose Grothendieck topology \hl{$\tau_{/X}$} (also denoted by notations such as \hl{$\tau|_{X}$} or \hl{$\tau/X$}) is defined by declaring a family of morphisms $\{f_i : Y_i \to Y\}$ in $\mathcal{C}_{/X}$ to be a covering if and only if the family $\{f_i : Y_i \to Y\}$ is a covering in $(\mathcal{C}, \tau)$.

% The forgetful functor 
% $$\hlin{j_X: \calC/X \to \calC}$$
% is \CrefAndHyperrefIfExist{definition:continuous_cocontinuous_functor_between_categories}{cocontinuous and continuous} 

\end{definition}

\begin{definition} \label{definition:site_of_opens_on_a_topological_space}
    Let $(X, \tau_X)$ be a topological space. The \hldef{small site associated to $X$} or \hldef{the site of open covers of $X$} or \hldef{the canonical site on $\operatorname{Open} X$} is the \CrefAndHyperrefIfExist{definition:category_of_opens_of_a_topological_space}{category $\operatorname{Open}(X)$ of open subsets} of $X$ with inclusion morphisms, equipped with the canonical \CrefAndHyperrefIfExist{definition:grothendieck_topology_on_a_category_site_covering_sieve_topologically_generating_family}{Grothendieck topology} \CrefAndHyperrefIfExist{definition:grothendieck_topology_generated_by_a_pretopology}{generated by} the \CrefAndHyperrefIfExist{definition:basis_and_grothendieck_pretopology_for_a_grothendieck_topology_on_a_category}{Grothendieck pretopology} whose covering families $\{U_i \to U\}_{i \in I}$, for $U \in \operatorname{Open}(X)$ are families of morphisms in $\operatorname{Open}(X)$ such that $\bigcup_{i \in I} U_i = U$. In other words, $\{U_i \to U\}_{i \in I}$ is a covering for the pretopology if it is an \CrefAndHyperrefIfExist{definition:open_covering_of_a_topological_space}{open coverings}.
\end{definition}
\begin{proposition} \label{proposition:common_examples_of_sites}
    \TODO{many definitions}
    The following are common examples of sites:
    \begin{enumerate}

        \item Let $X$ be a scheme. The following categories equipped with their respective Grothendieck topologies form \hldef{small sites associated to $X$}:
        \begin{itemize}
            \item The \hldef{small Zariski site} $\text{Zar}(X)$: the category of open immersions $U \to X$ with the Zariski topology given by open covers.

            \item The \hldef{small étale site} $\text{\'Et}(X)$: the category of \CrefAndHyperrefIfExist{definition:etale_morphism_of_schemes}{étale morphisms} $U \to X$ equipped with the étale topology defined by étale coverings.

            \item The \hldef{small Nisnevich site} $\text{Nis}(X)$: the category of étale morphisms $U \to X$ with the Nisnevich topology given by Nisnevich covers.

            \item The \hldef{small fppf site} $\text{fppf}(X)$: the category of flat morphisms locally of finite presentation $U \to X$ equipped with the fppf topology.

            \item The \hldef{small fpqc site} $\text{fpqc}(X)$: the category of flat morphisms $U \to X$ (not necessarily locally of finite presentation), with the fpqc topology.

            \item The \hldef{small crystalline site} $\text{Cris}(X)$: the category of PD-thickenings over $X$ equipped with the crystalline topology.
        \end{itemize}
    \end{enumerate}
\end{proposition}

\subsubsection{Sheaf on a site}

\begin{definition}[Sheaf on a site] \label{definition:sheaf_on_a_site}

% \TODO{There might be some need to say that $\calA$ is a category for which sheaves on the site ``can be defined''}
% \TODO{go through statements using the notion of sheaves and make sure that the value categories have small products and that the categories have small generating families.}

Let $(\calC, J)$ be a \CrefAndHyperrefIfExist{definition:grothendieck_topology_on_a_category_site_covering_sieve_topologically_generating_family}{site}. Let $\calA$ be a \CrefAndHyperrefIfExist{definition:category}{(large) category}.
\begin{enumerate}
    \item A \CrefAndHyperrefIfExist{definition:presheaf_on_a_category}{presheaf} $\calF: \calC^{\op} \to \calA$\CrefIfExists{definition:opposite_category_of_a_category} is called a \hldef{sheaf on the site $(\calC, J)$ valued in $\calA$} if, for every object $U$ of $\calC$ and every \CrefAndHyperrefIfExist{definition:grothendieck_topology_on_a_category_site_covering_sieve_topologically_generating_family}{covering sieve} $S \in J(U)$, the \CrefAndHyperrefIfExist{definition:limit_and_colimit_of_a_diagram_in_a_category}{limit}
    $$\varprojlim_{(V \to U) \in (\calD_S)^{\op}} \calF|_{\calD_S}(V),$$
    exists and the canonical natural morphism
    $$\calF(U) \to \varprojlim_{(V \to U) \in (\calD_S)^{\op}} \calF|_{\calD_S}(V)$$
    is an isomorphism. Here, $\calD_S \hookrightarrow \calC/U$\CrefIfExists{definition:category_of_objects_over_under_a_fixed_object_in_a_category} is the full \CrefAndHyperrefIfExist{definition:downward_upward_closed_subcategory_of_a_category}{downward-closed subcategory} such that $\operatorname{Ob}(\calD_S) = \{(f: V \to U): f \in S(V)\}$,

    In particular, when we are working with a \CrefAndHyperref{definition:basis_and_grothendieck_pretopology_for_a_grothendieck_topology_on_a_category}{Grothendieck pretopology} $K$ on a category $\calC$, we may speak of sheaves on the site whose Grothendieck topology is the \CrefAndHyperref{definition:grothendieck_topology_generated_by_a_pretopology}{one generated by} $K$.

    \item Given sheaves $\calF, \calG: \calC^{\op} \to \calA$ on the site $(\calC, J)$, a \hldef{morphism between the sheaves} is a \CrefAndHyperrefIfExist{definition:presheaf_on_a_category}{morphism} between $\calF$ and $\calG$ as presheaves.


    \item Let $U$ be a \hyperrefIfExists{definition:grothendieck_universe}{universe}\CrefIfExists{definition:grothendieck_universe}. A \hldef{$U$-sheaf} typically refers to a $U$-presheaf that is a sheaf for a $U$-site. In other words, a $U$-sheaf is a sheaf on a site whose underlying category is \hyperrefIfExists{definition:locally_small_category}{$U$-locally small}\CrefIfExists{definition:locally_small_category} and which has a $U$-small topologically generating family such that the sheaf is valued in $U$-sets.

    \item The \hldef{sheaf category/category of $\calA$-valued sheaves on $\calC$} is the (large) category defined as the full subcategory of $\PreShv(\calC, \calA)$ whose objects are the sheaves on $\calC$ with values in $\calA$. Common notations for the sheaf category include \hl{$\Shv(\calC, \calA)$}, \hl{$\Shv(\calC, J, \calA)$}, \hl{$\Sh(\calC, \calA)$}, \hl{$\Sh(\calC, J, \calA)$}. If the value category $\calA$ is clear from context, then notations such as \hl{$\Shv(\calC)$}, \hl{$\Shv(\calC, J)$}, \hl{$\Sh(\calC)$}, \hl{$\Sh(\calC, J)$} are also common.

\end{enumerate}

% Let $(\calC, J)$ be a \CrefAndHyperrefIfExist{definition:grothendieck_topology_on_a_category_site_covering_sieve_topologically_generating_family}{site} with a small \CrefAndHyperrefIfExist{definition:grothendieck_topology_on_a_category_site_covering_sieve_topologically_generating_family}{topological generating family} (or a $U$-small topologically generating family if a \CrefAndHyperrefIfExist{definition:grothendieck_universe}{universe} $U$ is available) and let $\mathcal{A}$ be a \CrefAndHyperrefIfExist{definition:category}{(large) category} that has all \CrefAndHyperrefIfExist{definition:locally_small_category}{small} \CrefAndHyperrefIfExist{definition:product_and_coproduct_of_objects_in_a_category}{products} (Some common examples of categories that have small products and thus play the role of $\calA$ here include $\mathcal{A} = \text{Set}$, $\text{Ab}$, $R\mathbf{-mod}$ for a fixed ring $R$, $\text{rings}$). 
% \begin{enumerate}

%     \item For any object $U$ of $\calC$ and every covering $\{U_i \to U\}_{i \in I}$ in $J$, note that there are morphisms $U_i \times_U U_j \to U_i$ for every $i,j \in I$. 
%     % Consider the subcategory of $C$ consisting of the objects $U_i$ and $U_i \times_U U_j$, together with these morphisms.
%     Given any presheaf $\calF: C^{\op} \to \calA$, there is a \CrefAndHyperrefIfExist{definition:diagram_in_a_category_indexed_by_a_small_category}{diagram} in $\calA$ consisting of objects $\calF(U_i)$ and $\calF(U_i \times_U U_j)$ and morphisms $\calF(U_i) \to \calF(U_i \times_U U_j)$. The presheaf $\calF$ is called a \hldef{sheaf on the site $(\calC, J)$ valued in $\calA$} if, for every object $U$ of $\calC$ and every covering $\{U_i \to U\}_{i \in I}$ in $J$, the sections object $\calF(U)$ is the \CrefAndHyperrefIfExist{definition:limit_and_colimit_of_a_diagram_in_a_category}{limit} of the aforementioned diagram:
    
%     % A \hyperrefIfExists{definition:presheaf_on_a_category}{presheaf}\CrefIfExists{definition:presheaf_on_a_category} $\mathcal{F}: C^{\mathrm{op}} \to \mathcal{A}$ is a \hldef{sheaf on the site $(\calC,J)$ valued in $\calA$} if, for every object $U$ of $\calC$ and every covering $\{U_i \to U\}_{i \in I}$ in $J$, the sections object $\calF(U)$ is the \CrefAndHyperrefIfExist{definition:limit_and_colimit_of_a_diagram_in_a_category}{limit} of the sections objects $\calF(U_i)$:
%     % $$\calF(U) \cong \varprojlim_{}$$
    
%     % following sequence is an \CrefAndHyperrefIfExist{definition:equalizer_and_coequalizer_of_morphisms_in_a_category}{equalizer} in $\mathcal{A}$:
%     % \[
%     % \mathcal{F}(U) \to \prod_{i} \mathcal{F}(U_i) \rightrightarrows \prod_{i, j} \mathcal{F}(U_i \times_U U_j)
%     % \]
%     % where the first map sends $s$ to $(\mathcal{F}(U_i \to U)(s))_i$ and the arrows to $(\mathcal{F}(U_i \times_U U_j \to U_i)(s_i))_{i,j}$ and $(\mathcal{F}(U_i \times_U U_j \to U_j)(s_j))_{i,j}$, respectively.

%     % \item A \hyperrefIfExists{definition:presheaf_on_a_category}{presheaf}\CrefIfExists{definition:presheaf_on_a_category} $\mathcal{F}: C^{\mathrm{op}} \to \mathcal{A}$ is a \hldef{sheaf on the site $(\calC,J)$ valued in $\calA$} if, for every object $U$ of $\calC$ and every covering $\{U_i \to U\}_{i \in I}$ in $J$, the following sequence is an \CrefAndHyperrefIfExist{definition:equalizer_and_coequalizer_of_morphisms_in_a_category}{equalizer} in $\mathcal{A}$:
%     % \[
%     % \mathcal{F}(U) \to \prod_{i} \mathcal{F}(U_i) \rightrightarrows \prod_{i, j} \mathcal{F}(U_i \times_U U_j)
%     % \]
%     % where the first map sends $s$ to $(\mathcal{F}(U_i \to U)(s))_i$ and the arrows to $(\mathcal{F}(U_i \times_U U_j \to U_i)(s_i))_{i,j}$ and $(\mathcal{F}(U_i \times_U U_j \to U_j)(s_j))_{i,j}$, respectively.

%     \item A \hldef{morphism of sheaves} $\calF: \calC^{\op} \to \calA$ is a \hyperrefIfExists{definition:presheaf_on_a_category}{morphism as presheaves}\CrefIfExists{definition:presheaf_on_a_category}. 


%     \item Let $U$ be a \hyperrefIfExists{definition:grothendieck_universe}{universe}\CrefIfExists{definition:grothendieck_universe}. A \hldef{$U$-sheaf} typically refers to a $U$-presheaf that is a sheaf for a $U$-site. In other words, a $U$-sheaf is a sheaf on a site whose underlying category is \hyperrefIfExists{definition:locally_small_category}{$U$-locally small}\CrefIfExists{definition:locally_small_category} and which has a $U$-small topologically generating family such that the sheaf is valued in $U$-sets.

%     \item The \hldef{sheaf category/category of $\calA$-valued sheaves on $\calC$} is the (large) category defined as the full subcategory of $\PreShv(\calC, \calA)$ whose objects are the sheaves on $C$ with values in $\calA$. Common notations for the sheaf category include \hl{$\Shv(\calC, \calA)$}, \hl{$\Shv(\calC, J, \calA)$}, \hl{$\Sh(\calC, \calA)$}, \hl{$\Sh(\calC, J, \calA)$}. If the value category $\calA$ is clear from context, then notations such as \hl{$\Shv(\calC)$}, \hl{$\Shv(\calC, J)$}, \hl{$\Sh(\calC)$}, \hl{$\Sh(\calC, J)$} are also common.

% \end{enumerate}
\end{definition}


\begin{definition}[Sheaf on a topological space] \label{definition:sheaf_on_a_topological_space_valued_in_a_category_with_a_terminal_object}

    Let $X$ be a \CrefAndHyperrefIfExist{definition:topological_space}{topological space}, let $\calD$ be a \CrefAndHyperrefIfExist{definition:category}{category} with a \CrefAndHyperrefIfExist{lemma:initial_or_final_object_in_a_category_that_is_also_in_a_full_subcategory_is_initial_or_final_in_the_subcategory}{terminal object}, and let $\mathcal{F}$ be a \CrefAndHyperrefIfExist{definition:presheaf_on_a_topological_space}{presheaf valued in $\calD$ on $X$}.  
    Then $\mathcal{F}$ is a \hldef{sheaf} if it satisfies the following additional condition (known as the \hldef{sheaf axioms}):

    For every open set $U \subseteq X$ and every \CrefAndHyperrefIfExist{definition:open_covering_of_a_topological_space}{open cover} $\{U_i\}_{i \in I}$ of $U$, let $\calJ$ be the \CrefAndHyperrefIfExist{definition:diagram_in_a_category_indexed_by_a_small_category}{diagram} in the \CrefAndHyperrefIfExist{definition:category_of_opens_of_a_topological_space}{category of opens} of $U$ consisting of the inclusions $U_i \cap U_j \hookrightarrow U_i$ for all $i,j \in I$. Then $\calF$ is a sheaf if the \CrefAndHyperrefIfExist{definition:limit_and_colimit_of_a_diagram_in_a_category}{limit} of the diagram $\calF \circ \calJ$ exists in $\calD$ and the natural morphism
    $$\calF(U) \to \lim_{j \in \calJ} \calF(j)$$
    is an isomorphism. More precisely, $\calJ:J \to \operatorname{Open}(U)$ should be the functor whose index category $J$ consists of
    \begin{enumerate}
        \item An object $i$ for every $i \in I$ and an object $(i,j)$ for every pair $i,j \in I$,
        \item Morphisms $p_1: (i,j) \to i$ and $p_2: (i,j) \to j$ for every pair $i,j \in I$
    \end{enumerate}
    and which sends the objects and morphisms as follows:
    \begin{enumerate}
        \item $\calJ(i) = U_i$
        \item $\calJ(i,j) = U_i \cap U_j$
        \item $\calJ(p_1): U_i \cap U_j \hookrightarrow U_i$
        \item $\calJ(p_2): U_i \cap U_j \hookrightarrow U_j$.
    \end{enumerate}
    In particular, taking $U = \emptyset$ and taking the empty open cover of the empty set, $\calF(\emptyset)$ must be the \CrefAndHyperrefIfExist{lemma:initial_or_final_object_in_a_category_that_is_also_in_a_full_subcategory_is_initial_or_final_in_the_subcategory}{terminal object} of $\calD$

    In the case that $\calD$ admits all \CrefAndHyperrefIfExist{definition:small_and_finite_limits_and_colimits_in_a_category}{small limits}, the sheaf condition is equivalent to the following: For every open set $U \subset X$ and every open cover $\{U_i\}_{i \in I}$ of $U$, the following \CrefAndHyperrefIfExist{definition:equalizer_and_coequalizer_of_morphisms_in_a_category}{equalizer diagram is exact}:
    $$\calF(U) \to \prod_{i \in I} \calF(U_i) \rightrightarrows \prod_{i,j \in I} \calF(U_i \cap U_j).$$
    Here, the morphism $\calF(U) \to \prod_{i \in I} \calF(U_i)$ and the two morphisms $\prod_{i \in I} \calF(U_i) \rightrightarrows \prod_{i,j \in I} \calF(U_i \cap U_j)$ are induced by the \CrefAndHyperrefIfExist{definition:presheaf_on_a_topological_space}{restriction maps} $\calF(U) \to \calF(U_i)$ and $\calF(U_i) \to \calF(U_i \cap U_j)$.

    In the case that $\calD$ is some \CrefAndHyperrefIfExist{definition:subcategory_of_a_category}{subcategory} of the \CrefAndHyperrefIfExist{definition:category_of_sets}{category of sets}, the sheaf condition is equivalent to the following: For every open set $U \subseteq X$ and every \CrefAndHyperrefIfExist{definition:open_covering_of_a_topological_space}{open cover} $\{U_i\}_{i \in I}$ of $U$, 
    \begin{itemize}
        \item (Locality) If $s, t \in \mathcal{F}(U)$ are such that $s|_{U_i} = t|_{U_i}$ for all $i$, then $s = t$.
        \item (Gluing) If for each $i$ there is $s_i \in \mathcal{F}(U_i)$ such that for all $i,j$ one has $s_i|_{U_i \cap U_j} = s_j|_{U_i \cap U_j}$, then there exists a unique $s \in \mathcal{F}(U)$ such that $s|_{U_i} = s_i$ for all $i$.
    \end{itemize}

    \TextIfExists{definition:sheaf_on_a_site}{
        Equivalently, a sheaf on a topological space $X$ may be defined as a \CrefAndHyperrefIfExist{definition:sheaf_on_a_site}{sheaf on} the \CrefAndHyperrefIfExist{definition:grothendieck_topology_on_a_category_site_covering_sieve_topologically_generating_family}{site} \CrefAndHyperrefIfExist{definition:site_of_opens_on_a_topological_space}{of opens on $X$}. 
        % whose objects are open subsets $U$ of $X$ and whose morphisms are inclusions $U_1 \hookrightarrow U_2$ of open subsets of $X$, and whose Grothendieck topology is the in which a cover of an open subset $U \subseteq X$ is given by a collection $\{U_i \to U\}_{i \in I}$ in which $\bigcup U_i = U$. 
    }

\end{definition}


\begin{definition}[Morphism of presheaves on a topological space] \label{definition:morphism_of_presheaves_on_a_topological_space}
    Let $X$ be a \CrefAndHyperrefIfExist{definition:topological_space}{topological space}, let $\calD$ be a \CrefAndHyperrefIfExist{definition:category}{category} and let $\mathcal{F}$, and $\mathcal{G}$ be \CrefAndHyperrefIfExist{definition:presheaf_on_a_topological_space}{presheaves valued in $\calD$ on $X$}.  

    A \hldef{morphism of presheaves} $\varphi: \mathcal{F} \to \mathcal{G}$ is a collection of maps
    $$\varphi(U): \mathcal{F}(U) \to \mathcal{G}(U)$$
    in $\calD$ defined for every open set $U \subseteq X$, such that the maps are compatible with restriction: for every inclusion $V \subseteq U$ of open sets, the diagram
    \[
    \begin{array}{ccc}
    \mathcal{F}(U) & \xrightarrow{\varphi(U)} & \mathcal{G}(U) \\
    \downarrow \rho^U_V & & \downarrow \rho^U_V \\
    \mathcal{F}(V) & \xrightarrow{\varphi(V)} & \mathcal{G}(V)
    \end{array}
    \]
    commutes.  

    Equivalently, $\varphi$ is a \CrefAndHyperrefIfExist{definition:natural_transformation_between_functors_between_categories}{natural transformation} of $\calF, \calG$, \CrefAndHyperrefIfExist{definition:presheaf_on_a_topological_space}{regarded as} functors 
    $$\mathcal{F}, \mathcal{G}: \mathbf{Open}(X)^{\op} \to \calD.$$

    If $\calF$ and $\calG$ are sheaves, then a \hldef{morphism of sheaves} is a morphism between $\calF$ and $\calG$ as presheaves.
\end{definition}


\begin{definition}[Presheaf on a topological space] \label{definition:presheaf_on_a_topological_space}
    Let $X$ be a \CrefAndHyperrefIfExist{definition:topological_space}{topological space}. Let $\calD$ be a category.
    
    A \hldef{presheaf (of objects of $\calD$/valued in $\calD$) on $X$} is a rule $\mathcal{F}$ that assigns:
    \begin{itemize}
        \item to each open set $U \subseteq X$, an object $\mathcal{F}(U) \in \Ob \calD$, called the \hldef{sections of $\mathcal{F}$ over $U$},
        \item to each inclusion of open sets $V \subseteq U$, a morphism
        $$\rho^U_V: \mathcal{F}(U) \to \mathcal{F}(V), \quad s \mapsto s|_V,$$
    \end{itemize}
        in the category $\calD$ called the \hldef{restriction map} such that the following conditions hold:
    \begin{itemize}
        \item (Identity) For every open set $U \subseteq X$, the restriction map $\rho^U_U$ is the identity on $\mathcal{F}(U)$.
        \item (Transitivity) For inclusions $W \subseteq V \subseteq U$ of open sets, one has
        $$\rho^U_W = \rho^V_W \circ \rho^U_V.$$
    \end{itemize}
    For instance, we may speak of a \hldef{presheaf of sets/groups/rings/etc. on the topological space $X$}.


    Equivalently, a presheaf on $X$ (of objects in a category $\calD$) is a \CrefAndHyperrefIfExist{definition:functor_between_categories}{functor}
    $$\mathbf{Open}(X)^{\op} \to \calD$$
    from the opposite of the category \CrefAndHyperrefIfExist{definition:category_of_opens_of_a_topological_space}{$\mathbf{Open}(X)$} of open subsets of $X$\TextIfExists{definition:presheaf_on_a_category}{ (see also \Cref{definition:presheaf_on_a_category})}.

    \TextIfExists{definition:presheaf_on_a_category}{Equivalently, a presheaf on $X$ is a presheaf on the category $\mathbf{Open}(X)$ in the sense of \Cref{definition:presheaf_on_a_category}}.


    The sections object $\calF(U)$ is also denoted by \hl{$\Gamma(U, \calF)$}\TextIfExists{definition:sections_of_a_presheaf_on_a_category_valued_in_a_category}{ (see \Cref{definition:sections_of_a_presheaf_on_a_category_valued_in_a_category})}.
    Moreover, the object $\calF(X) = \Gamma(X, \calF)$ is called the \hldef{global sections object of $\calF$}. \TextIfExists{definition:sections_of_a_presheaf_on_a_category_valued_in_a_category}{This agrees with the notion of global sections as discussed in \Cref{definition:sections_of_a_presheaf_on_a_category_valued_in_a_category}.}

\end{definition}


\begin{definition}[$\mathcal{P}$-torsion sheaf of abelian groups] \label{definition:torsion_for_a_set_of_primes_on_a_sheaf_of_abelian_groups_on_a_site}
    \TODO{must the site have a small topological generating family?}
Let $(\mathcal{C}, J)$ be a \CrefAndHyperrefIfExist{definition:grothendieck_topology_on_a_category_site_covering_sieve_topologically_generating_family}{site} with a small \CrefAndHyperrefIfExist{definition:grothendieck_topology_on_a_category_site_covering_sieve_topologically_generating_family}{topological generating family} (or a $U$-small topologically generating family if a \CrefAndHyperrefIfExist{definition:grothendieck_universe}{universe} $U$ is available), and let $\mathcal{P}$ be a collection of prime numbers. A \CrefAndHyperrefIfExist{definition:sheaf_on_a_site}{sheaf} of abelian groups $\mathcal{F}$ on $(\mathcal{C}, J)$ is called a \hldef{$\mathcal{P}$-torsion sheaf} if for every object $U$ of $\mathcal{C}$, every section $s \in \mathcal{F}(U)$, and every $p \in \mathcal{P}$, there exists some $n \geq 1$ such that $p^n s = 0$ in $\mathcal{F}(U)$.
\end{definition}
\begin{definition}[Torsion sheaf of abelian groups] \label{definition:torsion_sheaf_of_abelian_groups_on_a_site}
    \TODO{must the site have a small topological generating family?}
Let $(\mathcal{C}, J)$ be a \CrefAndHyperrefIfExist{definition:grothendieck_topology_on_a_category_site_covering_sieve_topologically_generating_family}{site} with a small \CrefAndHyperrefIfExist{definition:grothendieck_topology_on_a_category_site_covering_sieve_topologically_generating_family}{topological generating family} (or a $U$-small topologically generating family if a \CrefAndHyperrefIfExist{definition:grothendieck_universe}{universe} $U$ is available). A \CrefAndHyperrefIfExist{definition:sheaf_cohomology_group_of_a_sheaf_of_modules_over_a_sheaf_of_rings_on_a_site}{sheaf} of abelian groups $\mathcal{F}$ on $(\mathcal{C}, J)$ is called a \hldef{torsion sheaf of abelian groups} if for every object $U$ in $\mathcal{C}$ and every section $s \in \mathcal{F}(U)$, there exists a nonzero integer $n$ such that $n s = 0$ in $\mathcal{F}(U)$.
\end{definition}

\subsubsection{Sheafy functors}

\begin{definition} \label{definition:sheafy_functor_of_a_functor_between_two_categories_for_the_categories_of_sheaves_on_a_site}
    Let $\calA$, $\calB$ be \CrefAndHyperrefIfExist{definition:category}{categories}, let $F: \calA \to \calB$ be a \CrefAndHyperrefIfExist{definition:functor_between_categories}{functor}, and let $(\calD, K)$ be a \CrefAndHyperrefIfExist{definition:grothendieck_topology_on_a_category_site_covering_sieve_topologically_generating_family}{site}. 

    Assume at least one of the following:
    \begin{enumerate}
        \item A \CrefAndHyperrefIfExist{definition:sheafification_functor_on_a_site}{sheafification functor}
        $$a: \operatorname{PreSh}(\calD, K; \calB) \to \Sh(\calD,K; \calB) $$
        exists.
        \item $F$ is a \CrefAndHyperrefIfExist{definition:continuous_cocontinuous_functor_between_categories}{continuous functor} and $\calB$ admits \CrefAndHyperrefIfExist{theorem:limit_and_colimit_are_left_right_adjoint_to_diagonal_functor_for_locally_small_base_and_small_index}{limits} indexed by \CrefAndHyperrefIfExist{definition:grothendieck_topology_on_a_category_site_covering_sieve_topologically_generating_family}{covering sieves} of $K$.
    \end{enumerate}
    % Assume that a \CrefAndHyperrefIfExist{definition:sheafification_functor_on_a_site}{sheafification functor}
    % $$a: \operatorname{PreSh}(\calD, K; \calB) \to \Sh(\calD,K; \calB) $$
    % exists.

    Define the \hldef{sheafy functor} \hl{$\mathscr{F}$} of $F$ under each of the above assumptions as follows: 
    \begin{enumerate}
        \item 
        As the functor
        $$\mathscr{F}: \operatorname{PreSh}(\calD, \calA) \to \Sh(\calD, K; \calB)$$
        given by $S \mapsto a(F \circ S)$. By restriction, there is an induced functor 
        $$\operatorname{Sh}(\calD,K; \calA) \to \Sh(\calD, K; \calB)$$ 
        on the category of sheaves on $\calD$ valued in $\calA$; we usually denote this restricted functor by \hl{$\mathscr{F}$} as well. 

        \item 
        As the functor
        $$\mathscr{F}: \operatorname{Sh}(\calD,K; \calA) \to \Sh(\calD, K; \calB)$$
        given by $S \mapsto F \circ S$. Due to the assumptions on $F$ and $\calB$, $F \circ S$ is a sheaf.
    \end{enumerate}
    
    Note that if both assumptions hold, then the two functors 
    $$\operatorname{Sh}(\calD,K; \calA) \to \Sh(\calD, K; \calB)$$ 
    are naturally isomorphic.

    Sheafy functors are often notated with scripted letters.
\end{definition}

\begin{proposition} \label{proposition:category_of_sheaves_on_a_site_valued_in_product_category_is_equivalent_to_product_of_categories_of_sheaves_on_the_site}
    Let $(\calD, K)$ be a \CrefAndHyperrefIfExist{definition:grothendieck_topology_on_a_category_site_covering_sieve_topologically_generating_family}{site}. Let $\calA, \calB$ be \CrefAndHyperrefIfExist{definition:category}{categories}. There is an \CrefAndHyperrefIfExist{definition:equivalence_of_categories}{equivalence}
    $$\Sh(\calD, K; \calA \times \calB) \cong \Sh(\calD, K; \calA) \times \Sh(\calD, K; \calB)$$
    \CrefIfExists{definition:sheaf_on_a_site}\CrefIfExists{definition:product_category_of_a_family_of_categories}.
\end{proposition}

\begin{proof}
    \TODO{}
\end{proof}

\begin{theorem} \label{theorem:adjunction_between_sheafy_functors_from_adjoint_bifunctors}
    Let $(\calD, K)$ be a \CrefAndHyperrefIfExist{definition:grothendieck_topology_on_a_category_site_covering_sieve_topologically_generating_family}{site}. Let $\calA, \calB, \calC$ be \CrefAndHyperrefIfExist{definition:category}{categories}. Let $F: \calA \times \calB \to \calC$ and $G: \calA^{\op} \times \calC \to \calB$ be functors such that for any fixed $A \in \calA$, there is an \CrefAndHyperrefIfExist{definition:adjoint_functors_between_categories_unit_counit_of_adjoint_functors}{adjunction} $F(A,-) \dashv G(A,-)$. Assume the following:
    \begin{itemize}
        \item there is a \CrefAndHyperrefIfExist{definition:sheafification_functor_on_a_site}{sheafification functor}
        $$a: \operatorname{PreSh}(\calD, \calC) \to \Sh(\calD, K; \calC)$$

        \item $\calB$ admits \CrefAndHyperrefIfExist{definition:limit_and_colimit_of_a_diagram_in_a_category}{limits} indexed by the covering sieves in the \CrefAndHyperrefIfExist{definition:grothendieck_topology_on_a_category_site_covering_sieve_topologically_generating_family}{Grothendieck topology} $K$.
    \end{itemize}

    Let $\mathscr{F}$ be the composition
    $$\Sh(D,K;\calA) \times \Sh(D,K;\calB) \cong \Sh(D,K;\calA \times \calB) \to \Sh(D,K;\calC)$$
    of the equivalence of \Cref{proposition:category_of_sheaves_on_a_site_valued_in_product_category_is_equivalent_to_product_of_categories_of_sheaves_on_the_site} with the \CrefAndHyperrefIfExist{definition:sheafy_functor_of_a_functor_between_two_categories_for_the_categories_of_sheaves_on_a_site}{sheafy functor} of $F$. 

    For a fixed sheaf $\mathcal{R} \in \Sh(\calD, K; \calA)$, let $\mathscr{G}_{\mathcal{R}}: \Sh(\calD, K; \calC) \to \Sh(\calD, K; \calB)$ be the functor defined section-wise by 
    \[ \mathcal{T} \mapsto \left( U \mapsto G(\mathcal{R}(U), \mathcal{T}(U)) \right). \]
    Note that since $G(\mathcal{R}(U), -)$ is a \CrefAndHyperrefIfExist{definition:adjoint_functors_between_categories_unit_counit_of_adjoint_functors}{right adjoint}, \CrefAndHyperrefIfExist{theorem:right_left_adjoints_commute_with_limits_colimits}{it preserves} the \CrefAndHyperrefIfExist{definition:limit_and_colimit_of_a_diagram_in_a_category}{limits} indexed by the covering sieves of $K$; thus, $\mathscr{G}_{\mathcal{R}}(\mathcal{T})$ is a sheaf whenever $\mathcal{T}$ is a sheaf. 

    Furthermore, the assignment $(\mathcal{R}, \mathcal{T}) \mapsto \mathscr{G}_{\mathcal{R}}(\mathcal{T})$ defines a functor 
    \[ \mathscr{G}: \Sh(\calD, K; \calA)^{\op} \times \Sh(\calD, K; \calC) \to \Sh(\calD, K; \calB). \]

    Then for all $\mathcal{R} \in \Sh(\calD, K; \calA)$, there is an adjunction 
    \[ \mathscr{F}(\mathcal{R}, -) \dashv \mathscr{G}_{\mathcal{R}}(-) \]
    induced by the pointwise adjunctions $F(\mathcal{R}(U), -) \dashv G(\mathcal{R}(U), -)$ and the universal property of sheafification.
        
    % Similarly, let $\mathscr{G}$ be the composition
    % $$\Sh(D,K;\calA^{\op}) \times \Sh(D,K;\calC) \cong \Sh(D,K;\calA^{\op} \times \calB) \to \Sh(D,K;\calC).$$

    % For all $\mathcal{R} \in \Sh(D,K;\calA)$, there is an adjunction $\mathscr{F}(\mathcal{R},-) \dashv \mathscr{G}(\mathcal{R},-)$

\end{theorem}

\begin{proof}
    \TODO{}
\end{proof}

\begin{definition} \label{definition:module_over_a_sheaf_of_rings_on_a_site}

    \begin{enumerate}
        \item 
        Let $\mathcal{C}$ be a \CrefAndHyperrefIfExist{definition:grothendieck_topology_on_a_category_site_covering_sieve_topologically_generating_family}{site}, and let $\mathcal{A}$ and $\mathcal{B}$ be \CrefAndHyperrefIfExist{definition:sheaf_on_a_site}{sheaves} of (not necessarily commutative) \CrefAndHyperrefIfExist{definition:ring}{rings} on $\mathcal{C}$. 
        
        \begin{enumerate}
            \item 
            An \hldef{$(\mathcal{A}, \mathcal{B})$-bimodule} (or a \hldef{bimodule over $(\mathcal{A}, \mathcal{B})$}) is a \CrefAndHyperrefIfExist{definition:sheaf_on_a_site}{sheaf} $\mathcal{M}$ of abelian groups on $\mathcal{C}$ equipped with a left $\mathcal{A}$-module structure given by a \CrefAndHyperrefIfExist{definition:sheaf_on_a_site}{morphism of sheaves} of sets
            $$ \lambda: \mathcal{A} \times \mathcal{M} \longrightarrow \mathcal{M}, $$
            and a right $\mathcal{B}$-module structure given by a morphism of sheaves of sets
            $$ \rho: \mathcal{M} \times \mathcal{B} \longrightarrow \mathcal{M}, $$
            such that the actions are compatible. Specifically, for every object $U$ in $\mathcal{C}$, every section $m \in \mathcal{M}(U)$, every $a \in \mathcal{A}(U)$, and every $b \in \mathcal{B}(U)$, the equality
            $$ \lambda_U(a, \rho_U(m, b)) = \rho_U(\lambda_U(a, m), b) $$
            holds in $\mathcal{M}(U)$. In standard multiplicative notation where $\lambda(a,m)$ is denoted $a \cdot m$ and $\rho(m,b)$ is denoted $m \cdot b$, this condition is the associativity axiom
            $$ (a \cdot m) \cdot b = a \cdot (m \cdot b). $$

            In particular, for every object $U \in \calC$, the abelian group $\calM(U)$ has the structure of an \CrefAndHyperrefIfExist{definition:module_of_a_ring}{$\calA(U)-\calB(U)$-bimodule}.

            \item Let $\mathcal{M}$ and $\mathcal{N}$ be $(\mathcal{A}, \mathcal{B})$-bimodules. A \hldef{homomorphism of $(\mathcal{A}, \mathcal{B})$-bimodules} (or an \hldef{$(\mathcal{A}, \mathcal{B})$-linear morphism}) is a morphism of sheaves of abelian groups $f: \mathcal{M} \to \mathcal{N}$ such that for every object $U$ of $\mathcal{C}$, every section $m \in \mathcal{M}(U)$, every $a \in \mathcal{A}(U)$, and every $b \in \mathcal{B}(U)$, the following compatibility conditions hold:
            $$ f_U(a \cdot m) = a \cdot f_U(m) \quad \text{and} \quad f_U(m \cdot b) = f_U(m) \cdot b. $$


        \end{enumerate}

        \noindent We denote the category of $(\mathcal{A}, \mathcal{B})$-bimodules, with morphisms being morphisms of sheaves of abelian groups compatible with both the left $\mathcal{A}$-action and the right $\mathcal{B}$-action, by
        \hl{$ \mathcal{A}\text{-}\mathcal{B}\text{-}\mathsf{Mod} $}
        or sometimes by
        \hl{$ {}_{\mathcal{A}}\mathsf{Mod}_{\mathcal{B}} $}
        \TODO{talk about how bimodules can be identifies with left/right modules}

        \item 

        Let $(\mathcal{C}, J)$ be a \CrefAndHyperrefIfExist{definition:grothendieck_topology_on_a_category_site_covering_sieve_topologically_generating_family}{site}. Let $\mathcal{O}$ be a \CrefAndHyperrefIfExist{definition:sheaf_on_a_site}{sheaf of (not necessarily commutative) rings on $(\mathcal{C}, J)$}, i.e. $((\calC, J), \calO)$ is a \CrefAndHyperrefIfExist{definition:ringed_site}{ringed site}.  

        \begin{enumerate}
            \item An \hldef{(left/right/two-sided) $\mathcal{O}$-module} consists of the following data:
            \begin{itemize}
                \item A sheaf $\mathcal{F}$ of abelian groups on $(\mathcal{C}, J)$,
            \item for every object $U \in \mathcal{C}$, the structure of an (left/right/two-sided) $\mathcal{O}(U)$-module on $\mathcal{F}(U)$,
            \end{itemize}
            such that for every morphism $f: V \to U$ in $\mathcal{C}$, the restriction map 
            $$\rho_{U,V}: \mathcal{F}(U) \to \mathcal{F}(V)$$ 
            is $\mathcal{O}(U)$-linear when the $\mathcal{O}(U)$-action on $\mathcal{F}(V)$ is defined via the natural ring homomorphism 
            $$\mathcal{O}(U) \to \mathcal{O}(V)$$
            induced by $f$.


            \item Let $\mathcal{F}$ and $\mathcal{G}$ be \CrefAndHyperrefIfExist{definition:module_over_a_sheaf_of_rings_on_a_site}{$\mathcal{O}$-modules}.

            A \hldef{morphism of $\mathcal{O}$-modules} $\varphi: \mathcal{F} \to \mathcal{G}$ is a \CrefAndHyperrefIfExist{definition:sheaf_on_a_site}{morphism of sheaves} of abelian groups such that, for every object $U \in \mathcal{C}$, the component map
            $$\varphi_U : \mathcal{F}(U) \to \mathcal{G}(U)$$
            is $\mathcal{O}(U)$-linear, i.e. it satisfies
            $$\varphi_U(r \cdot s) = r \cdot \varphi_U(s) \quad \text{for all } r \in \mathcal{O}(U), \, s \in \mathcal{F}(U).$$

            The collection of all $\mathcal{O}$-modules together with their morphisms of $\mathcal{O}$-modules forms the \hldef{category of $\mathcal{O}$-modules}, denoted \hl{$\mathbf{Mod}(\mathcal{O})$}.

            \TextIfExists{definition:algebra_over_a_sheaf_of_rings_on_a_site}{See also \Cref{definition:algebra_over_a_sheaf_of_rings_on_a_site}.}
        \end{enumerate}

        \noindent In case that a \CrefAndHyperrefIfExist{definition:sheafification_functor_on_a_site}{sheafification functor} 
        $$\PreShv(\calC, \mathbf{Rings}) \to \Shv(\calC, \mathbf{Rings})$$ 
        exists, a left, right, two-sided $\calO$-module (and morphisms thereof) is equivalent to a $(\calO,\bbZ)$-bimodule, $(\bbZ,\calO)$-bimodule, and $(\calO, \calO)$-bimodule (and morphisms thereof) respectively, where $\bbZ$ is the \CrefAndHyperrefIfExist{definition:constant_sheaf_on_a_site_with_sheafification}{constant sheaf} of the integer ring $\bbZ$.

\end{enumerate}


\end{definition}


% See Also
% theorem:category_of_modules_over_a_sheaf_of_rings_on_a_site_on_an_essentially_small_category_has_enough_injectives
\begin{definition}[Tensor product of bimodules] \label{definition:tensor_product_of_bimodules_of_rings}
Let $R,S,T$ be \CrefAndHyperrefIfExist{definition:ring}{(not necessarily commutative) rings}, let $M$ be an \CrefAndHyperrefIfExist{definition:module_of_a_ring}{$R$-$S$ bimodule}, and let $N$ be an $S$-$T$ bimodule. In the \CrefAndHyperrefIfExist{definition:free_abelian_group_generated_by_a_set}{free abelian group} $\bbZ[M \times N]$ generated by the \CrefAndHyperrefIfExist{definition:product_of_sets}{Cartesian product $M \times N$}, let $U$ be the subgroup generated by elements of the form
\TODO{subgroup generated}
\begin{align*}
&(m+m',n) - (m,n) - (m',n),\\
&(m,n+n') - (m,n) - (m,n'),\\
&(m \cdot s, n) - (m, s \cdot n),
\end{align*}
for all $m,m' \in M$, $n,n' \in N$, and $s \in S$. The \hldef{tensor product of $M$ and $N$ over $S$} is the \CrefAndHyperrefIfExist{definition:quotient_of_a_group_by_a_normal_subgroup}{quotient} abelian group
$$M \otimes_S N := \mathbb{Z}[M \times N] / U.$$
The image of an element of the form $(m,n) \in M \times N$ in $M \otimes_S N$ is denoted \hl{$m \otimes n$} and called a \hldef{pure tensor}. In general, the elements of $M \otimes_S N$ are finite sums 
$$\sum_{i=1}^n m_i \otimes n_i \quad m_i \in M, n_i \in N$$
of pure tensors. Thus, the pure tensors satisfy the following relations:
\begin{align*}
    (m + m') \otimes n &= m \otimes n + m' \otimes n \\ 
    m \otimes (n + n') &= m \otimes n + m \otimes n' \\
    (m \cdot s) \otimes n &= m \otimes (s \cdot n)
\end{align*}

This tensor product becomes naturally an $R$-$T$ bimodule with left action and right action defined by
\begin{align*}
r \cdot (m \otimes n) &= (r \cdot m) \otimes n, \\
(m \otimes n) \cdot t &= m \otimes (n \cdot t),
\end{align*}
for all $r \in R$, $t \in T$, $m \in M$, and $n \in N$.

Inductively, given rings $R_0,\ldots,R_k$ and $R_{i-1}-R_i$-bimodules $M_i$ for $i = 1,\ldots,k$, we may speak of the tensor product
$$M_0 \otimes_{R_1} M_1 \otimes_{R_2} \cdots \otimes_{R_{k-1}} M_k;$$
tensor products are associative\TODO{}, so parentheses are not strictly needed to notate them. Its \hldef{pure tensors} are elements of the form $m_0 \otimes m_1 \otimes \cdots \otimes m_k$ for $m_i \in M_i$, and its general elements are finite sums
$$\sum_{j=1}^n m_{0j} \otimes m_{1j} \otimes \cdots m_{kj} \quad m_{ij} \in M_i.$$
of pure tensors. It also has a natural $R_0-R_k$-bimodule structure.

\TextIfExists{definition:n_ary_additive_functor_between_additive_categories}{In general, $(M_0,\ldots,M_k) \mapsto M_0 \otimes_{R_1} M_1 \otimes_{R_2} \cdots \otimes_{R_{k-1}} M_k$ defines a \CrefAndHyperrefIfExist{definition:n_ary_additive_functor_between_additive_categories}{$(k+1)$-ary additive functor}
$${}_{R_0}\mathbf{Mod}_{R_1} \times \cdots \times {}_{R_{k-1}}\mathbf{Mod}_{R_k} \to {}_{R_0} \mathbf{Mod}_{R_k}$$
(\Cref{theorem:the_category_of_R_S_bimodules_is_a_grothendieck_abelian_category_and_AB4_star}).}


Given a ring $R$ and a two-sided $R$-module $M$, we may also speak of the \hldef{$n$-fold tensor product} \hl{$M^{\otimes n} = M^{\otimes_R n}$}

\end{definition}

\begin{definition}[Sheafified Tensor Product] \label{definition:sheafified_tensor_product_of_sheaves_of_modules_of_sheaves_of_rings_on_a_site_with_a_small_topologically_generating_family}
    Let $(\calD, K)$ be a \CrefAndHyperrefIfExist{definition:grothendieck_topology_on_a_category_site_covering_sieve_topologically_generating_family}{site} that admits a \CrefAndHyperrefIfExist{definition:grothendieck_topology_on_a_category_site_covering_sieve_topologically_generating_family}{small topologically generating family} (e.g. by being \CrefAndHyperrefIfExist{definition:essentially_small_site}{essentially small}) and let $\calR, \calS, \calT$ be sheaves of rings on $(\calD, K)$. Let $\calM$ be a \CrefAndHyperrefIfExist{definition:sheaf_on_a_site}{sheaf} of \CrefAndHyperrefIfExist{definition:module_over_a_sheaf_of_rings_on_a_site}{$\calR$-$\calS$ bimodules} and $\calN$ be a sheaf of $\calS$-$\calT$ bimodules. 
    The \hldef{sheafified tensor product}, denoted \hl{$\calM \otimes_{\calS} \calN$}, is the sheaf of $\calR$-$\calT$ bimodules defined as the \CrefAndHyperrefIfExist{definition:sheafification_functor_on_a_site}{sheafification} of the section-wise tensor product presheaf:
    \[ U \mapsto \calM(U) \otimes_{\calS(U)} \calN(U). \]
    \CrefIfExists{definition:tensor_product_of_bimodules_of_rings}
    Explicitly, $\calM \otimes_{\calS} \calN = a(U \mapsto \calM(U) \otimes_{\calS(U)} \calN(U))$, where $a$ is the sheafification functor.

    \TextIfExists{definition:sheafy_functor_of_a_functor_between_two_categories_for_the_categories_of_sheaves_on_a_site}{
    In the framework of \Cref{definition:sheafy_functor_of_a_functor_between_two_categories_for_the_categories_of_sheaves_on_a_site}, the sheafified tensor product is the evaluation of the sheafy functor $\mathscr{F}$ induced by the algebraic tensor product $F: \calA \times \calB \to \calC$ under assumption (1). Here, $\calA, \calB, \calC$ are the categories of all bimodules over all rings \TODO{category of all bimodules over all rings}. Specifically, if $\mathbb{M} := (\calR, \calS, \calM) \in \Sh(\calD, K; \calA)$ and $\mathbb{N} := (\calS, \calT, \calN) \in \Sh(\calD, K; \calB)$, then the sheafified tensor product is the module component of the sheaf $\mathscr{F}(\mathbb{M}, \mathbb{N}) \in \Sh(\calD, K; \calC)$, which is the triple $(\calR, \calT, \calM \otimes_{\calS} \calN)$.
    }

    % In the framework of \Cref{definition:sheafy_functor_of_a_functor_between_two_categories_for_the_categories_of_sheaves_on_a_site}, the sheafified tensor product is the sheafy functor $\mathscr{F}$ induced by the algebraic tensor product functor $F: \calA \times \calB \to \calC$ where $\calA, \calB, \calC$ are the categories of bimodules over rings. It satisfies assumption (1) of said definition, as the section-wise tensor product generally fails to be a sheaf and requires the sheafification functor $a$.
\end{definition}

 

\begin{definition}[Sheafified Hom] \label{definition:sheaf_hom_for_sheaves_of_modules_over_sheaves_of_rings_on_a_site_with_a_small_topologically_generating_family}
    Let $(\calD, K)$ be a \CrefAndHyperrefIfExist{definition:grothendieck_topology_on_a_category_site_covering_sieve_topologically_generating_family}{site} that admits a \CrefAndHyperrefIfExist{definition:grothendieck_topology_on_a_category_site_covering_sieve_topologically_generating_family}{small topologically generating family} (e.g. by being \CrefAndHyperrefIfExist{definition:essentially_small_site}{essentially small}) and let $\calR, \calS, \calT$ be sheaves of rings on $(\calD, K)$. Let $\calM$ be a \CrefAndHyperrefIfExist{definition:sheaf_on_a_site}{sheaf} of \CrefAndHyperrefIfExist{definition:module_over_a_sheaf_of_rings_on_a_site}{$\calR$-$\calS$ bimodules} and $\calN$ be a sheaf of $\calR$-$\calT$ bimodules.

    The \hldef{sheafified Hom}, denoted by notations such as \hl{$\mathcal{H}om_{\calR}(\calM, \calN)$}, \hl{$\mathscr{H}om_{\calR}(\calM, \calN)$}, or \hl{$\underline{\mathrm{Hom}}_{\calR}(\calM, \calN)$}, is the sheaf of $\calS$-$\calT$ bimodules defined section-wise by:
    \[ \mathcal{H}om_{\calR}(\calM, \calN)(U) = \operatorname{Hom}_{\Sh(\calD/U, K|_U; \operatorname{Mod}(\calR|_U))}(\calM|_U, \calN|_U). \]
    (\Cref{definition:category_of_objects_over_under_a_fixed_object_in_a_category}) (\Cref{definition:site_induced_by_a_site_on_an_over_category}) (\Cref{definition:restriction_of_a_sheaf_on_a_site_to_an_object_of_the_underlying_category_of_the_site}) 

    Equivalently, it is the unique sheaf which satisfies the right adjoint property relative to the sheafified tensor product (\Cref{theorem:tensor_hom_adjucntion_for_sheaves_of_modules_over_sheaves_of_rings_on_a_site_with_a_small_topologically_generating_family}). 

    \TextIfExists{definition:sheafy_functor_of_a_functor_between_two_categories_for_the_categories_of_sheaves_on_a_site}{
    In the framework of \Cref{definition:sheafy_functor_of_a_functor_between_two_categories_for_the_categories_of_sheaves_on_a_site}, the sheafified Hom corresponds to the right adjoint $\mathscr{G}$ to the sheafified tensor product $\mathscr{F}$. While the general framework defines sheafy functors via section-wise application, the internal Hom for sheaves of modules requires the restriction-based definition (using $\calD/U$) to correctly represent the right adjoint in the category of sheaves, ensuring compatibility with the varying ring structure $\calR$.

    Also in the framework of \Cref{definition:sheafy_functor_of_a_functor_between_two_categories_for_the_categories_of_sheaves_on_a_site}, the sheafified Hom is the evaluation of the sheafy functor $\mathscr{G}$ induced by the algebraic Hom functor $G: \calA^{\op} \times \calC \to \calB$ under assumption (2). Given input sheaves $\mathbb{M} := (\calR, \calS, \calM)$ and $\mathbb{N} := (\calR, \calT, \calN)$, the result is the sheaf $\mathscr{G}(\mathbb{M}, \mathbb{N}) = (\calS, \calT, \mathcal{H}om_{\calR}(\calM, \calN))$. Because $G$ is a right adjoint in the base categories, it is a continuous functor (i.e., it preserves limits); thus, $\mathscr{G}$ can be defined section-wise as in Case (2) of the framework. This section-wise construction $U \mapsto \operatorname{Hom}_{\calR(U)}(\calM(U), \calN(U))$ is naturally isomorphic to the restriction-based internal Hom defined above, as both satisfy the same local-to-global universal property.

    }

    % \TextIfExists{definition:sheafy_functor_of_a_functor_between_two_categories_for_the_categories_of_sheaves_on_a_site}{
    % In the framework of \Cref{definition:sheafy_functor_of_a_functor_between_two_categories_for_the_categories_of_sheaves_on_a_site}, the sheafified Hom is the evaluation of the sheafy functor $\mathscr{G}$ induced by the algebraic Hom functor $G: \calA^{\op} \times \calC \to \calB$ under assumption (2). Given input sheaves $\mathbb{M} := (\calR, \calS, \calM)$ and $\mathbb{P} := (\calR, \calT, \calP)$, the result is the sheaf $\mathscr{G}(\mathbb{M}, \mathbb{P}) = (\calS, \calT, \mathcal{H}om_{\calR}(\calM, \calP))$. Because $G$ is a right adjoint in the base categories, it is a continuous functor; thus, $\mathscr{G}$ is defined section-wise as in Case (2) of the framework, which is naturally isomorphic to the restriction-based internal Hom defined above.
    % }
\end{definition}

\begin{theorem}[Tensor-Hom Adjunction for Sheaves of Bimodules] \label{theorem:tensor_hom_adjucntion_for_sheaves_of_modules_over_sheaves_of_rings_on_a_site_with_a_small_topologically_generating_family}
    Let $(\calD, K)$ be a \CrefAndHyperrefIfExist{definition:grothendieck_topology_on_a_category_site_covering_sieve_topologically_generating_family}{site} that admits a \CrefAndHyperrefIfExist{definition:grothendieck_topology_on_a_category_site_covering_sieve_topologically_generating_family}{small topologically generating family} (e.g. by being \CrefAndHyperrefIfExist{definition:essentially_small_site}{essentially small}). Let $\calR, \calS, \calT$ be \CrefAndHyperrefIfExist{definition:sheaf_on_a_site}{sheaves} of rings on $(\calD, K)$. Let $\calM$ be a sheaf of $\calR$-$\calS$ \CrefAndHyperrefIfExist{definition:module_over_a_sheaf_of_rings_on_a_site}{bimodules}, $\calN$ be a sheaf of $\calS$-$\calT$ bimodules, and $\calP$ be a sheaf of $\calR$-$\calT$ bimodules.
    
    Then there is a natural isomorphism:
    \[ \operatorname{Hom}_{\Sh(\calR\text{-}\calT)}(\calM \otimes_{\calS} \calN, \calP) \cong \operatorname{Hom}_{\Sh(\calS\text{-}\calT)}(\calN, \mathcal{H}om_{\calR}(\calM, \calP)) \]
    where $\calM \otimes_{\calS} \calN$ is the sheafified tensor product and $\mathcal{H}om_{\calR}(\calM, \calP)$ is the sheaf of local $\calR$-linear homomorphisms.
\end{theorem}
\begin{proof}
    \TODO{Verify this proof
    We apply \Cref{theorem:adjunction_between_sheafy_functors_from_adjoint_bifunctors} by specifying the categories $\calA, \calB, \calC$ and the functors $F, G$ as follows:

    \begin{enumerate}
        \item \textbf{Categories:} Let $\calA$ be the category whose objects are triples $(R, S, M)$ where $M$ is an $R$-$S$ bimodule. Define $\calB$ and $\calC$ similarly for $S$-$T$ and $R$-$T$ bimodules.
        \item \textbf{Base Functors:} Define $F: \calA \times \calB \to \calC$ by $(M, N) \mapsto M \otimes_S N$ and $G: \calA^{\op} \times \calC \to \calB$ by $(M, P) \mapsto \operatorname{Hom}_R(M, P)$. 
        At each component, $F(A, -) \dashv G(A, -)$ is the standard algebraic tensor-hom adjunction for bimodules.
        \item \textbf{Sheaf Interpretation:} The sheaves $\calM, \calN, \calP$ are viewed as objects in $\Sh(\calD, K; \calA)$, $\Sh(\calD, K; \calB)$, and $\Sh(\calD, K; \calC)$ respectively, where the ring components of these sheaves are fixed to be the sheaves $\calR, \calS, \calT$.
    \end{enumerate}

    By \Cref{theorem:adjunction_between_sheafy_functors_from_adjoint_bifunctors}, the sheafy functor $\mathscr{F}$ associated to the tensor product is:
    \[ \mathscr{F}(\calM, \calN) = a(U \mapsto \calM(U) \otimes_{\calS(U)} \calN(U)) \]
    which is precisely the definition of the sheaf of $\calR$-$\calT$ bimodules $\calM \otimes_{\calS} \calN$.

    The right adjoint functor $\mathscr{G}_{\calM}$ is defined section-wise by:
    \[ \mathscr{G}_{\calM}(\calP) : U \mapsto \operatorname{Hom}_{\calR(U)}(\calM(U), \calP(U)) \]
    Per the theorem, since $G$ is a right adjoint, $\mathscr{G}_{\calM}(\calP)$ is a sheaf. In the category of modules, this section-wise Hom sheaf is naturally isomorphic to the internal sheaf $\mathcal{H}om_{\calR}(\calM, \calP)$ because any morphism of sheaves $\calM|_U \to \calP|_U$ is uniquely determined by its action on sections when the sheafification of the section-wise Hom agrees with the internal Hom.

    Thus, the general adjunction $\mathscr{F}(\calM, -) \dashv \mathscr{G}_{\calM}(-)$ specializes to the required isomorphism of Hom-sets in the category of sheaves of bimodules.
    }
\end{proof}


\subsubsection{Topos}

% \begin{definition}[Topos] \label{definition:topos}
%     There are a multitude of notions of topos. Here are some that we consider; more notions may be added later.
%     \begin{enumerate}
%         \item A \hldef{(sheaf/Grothendieck) topos} is a \CrefAndHyperrefIfExist{definition:category}{category} \CrefAndHyperrefIfExist{definition:equivalence_of_categories}{equivalent} to the category of \CrefAndHyperrefIfExist{definition:sheaf_on_a_site}{sheaves} of sets on some \CrefAndHyperrefIfExist{definition:grothendieck_topology_on_a_category_site_covering_sieve_topologically_generating_family}{site}. That is, there exists a site $(C, J)$ such that the category is equivalent to $\operatorname{Sh}(C, J)$, the category of sheaves of sets on $(C, J)$.
%         \item Let $U$ be a universe. A \hldef{$U$-(sheaf )topos} is a category equivalent to the category of \hyperrefIfExists{definition:sheaf_on_a_site}{$U$-sheaves}\CrefIfExists{definition:sheaf_on_a_site} (valued in $U$-sets) \cite[Expos\'e IV D\'efinition 1.1]{SGA4_I}

%         \item An \hldef{elementary topos} is a cateogry which has all finite \CrefAndHyperrefIfExist{definition:limit_and_colimit_of_a_diagram_in_a_category}{limits}, is cartesian closed, and has a subobject classifier \TODO{cartesian closed, subobject classifier}
%     \end{enumerate}
% \end{definition}

\begin{definition}[Topos] \label{definition:topos}
    There are multiple notions of a topos depending on the context (geometric vs. logical).
    \begin{enumerate}
        \item A \hldef{Grothendieck topos} (or \hldef{sheaf topos}) is a \CrefAndHyperrefIfExist{definition:category}{category} \CrefAndHyperrefIfExist{definition:equivalence_of_categories}{equivalent} to the category of \CrefAndHyperrefIfExist{definition:sheaf_on_a_site}{sheaves} of sets on a \hldef{small} \CrefAndHyperrefIfExist{definition:grothendieck_topology_on_a_category_site_covering_sieve_topologically_generating_family}{site}. That is, there exists a small site $(\mathcal{C}, J)$ such that the category is equivalent to $\operatorname{Sh}(\mathcal{C}, J)$.
        
        \item Let $\mathscr{U}$ be a \hyperrefIfExists{definition:grothendieck_universe}{universe}\CrefIfExists{definition:grothendieck_universe}. A \hldef{$\mathscr{U}$-topos} is a category equivalent to the category of sheaves of sets on a $\mathscr{U}$-small site $(\mathcal{C}, J)$, where the sheaves take values in the category of $\mathscr{U}$-sets ($\mathbf{Set}_{\mathscr{U}}$). \cite[Expos\'e IV D\'efinition 1.1]{SGA4_I}

        \item An \hldef{elementary topos} is a category which has all finite \CrefAndHyperrefIfExist{definition:limit_and_colimit_of_a_diagram_in_a_category}{limits}, is \CrefAndHyperrefIfExist{definition:cartesian_closed_category}{cartesian closed}, and has a \CrefAndHyperrefIfExist{definition:subobject_classifier_in_a_category_with_a_final_object}{subobject classifier}.
    \end{enumerate}
    \textit{Remark:} Every Grothendieck topos is an elementary topos, but the converse is not true (e.g., the category of finite sets is an elementary topos but not a Grothendieck topos).
\end{definition}


% {\cite[Expos\'e IV D\'efinition 1.1]{SGA4_I}}
% Let $\scrU$ be a fixed universe. A \hldef{$\scrU$-topos}, or simply \hldef{topos} if there is no confusion, $E$ is a category that is equivalent to the category $\Shv(T)$ of sheaves of sets on a fixed site $T$ in $\scrU$.
\begin{definition} \label{definition:geometric_morphism_between_topoi}
Let $\mathcal{E}$ and $\mathcal{F}$ be \CrefAndHyperrefIfExist{definition:topos}{topoi}. A \hldef{geometric morphism from $\mathcal{F}$ to $\mathcal{E}$}, denoted $f: \mathcal{F} \to \mathcal{E}$, is a pair of adjoint functors \hl{$(f^*, f_*)$}
$$ f^*: \mathcal{E} \rightleftarrows \mathcal{F} : f_* $$
where:
\begin{enumerate}
    \item The functor \hl{$f^*$} (called the \hldef{inverse image functor}) is left adjoint to the functor \hl{$f_*$} (called the \hldef{direct image functor}).
    \item The inverse image functor $f^*$ is left exact, meaning it preserves finite \CrefAndHyperrefIfExist{definition:limit_and_colimit_of_a_diagram_in_a_category}{limits}.
\end{enumerate}
A \hldef{morphism of topoi} usually refers to a geometric morphism between topoi.
\end{definition}

\begin{lemma}[see e.g. {\cite[Expos\'e IV 5.1]{SGA4_I}}] \label{lemma:slice_category_of_a_topos_is_a_topos}
    Let $E$ be a \hyperrefIfExists{definition:topos}{topos}\CrefIfExists{definition:topos} and let $X$ be an object of $E$. The \hyperrefIfExists{definition:category_of_objects_over_under_a_fixed_object_in_a_category}{slice category $E_{/X}$}\CrefIfExists{definition:category_of_objects_over_under_a_fixed_object_in_a_category} is itself a topos. 
\end{lemma}

\begin{definition} \label{definition:topos_of_objects_over_a_fixed_object_of_a_topos}
    Let $C$ be a \hyperrefIfExists{definition:category}{(large) category}\CrefIfExists{definition:category}. In view of \Cref{lemma:slice_category_of_a_topos_is_a_topos}, the \hyperrefIfExists{definition:category_of_objects_over_under_a_fixed_object_in_a_category}{slice category $E_{/X} = E/X$}\CrefIfExists{definition:category_of_objects_over_under_a_fixed_object_in_a_category} is often called the \hldef{slice topos} or \hldef{over topos}, etc. 
\end{definition}


\begin{definition}[Generator of a category] \label{definition:generator_of_a_category}
Let \(\mathcal{C}\) be a \CrefAndHyperrefIfExist{definition:category}{category}. 
\begin{enumerate}
    \item  An object \(G \in \mathcal{C}\) is called a \hldef{generator} (or \hldef{separator}) if for every pair of distinct morphisms \(f, g : X \to Y\) in \(\mathcal{C}\), there exists a morphism \(h : G \to X\) such that
    \[
    f \circ h \neq g \circ h.
    \]
    In case that $\calC$ is \CrefAndHyperrefIfExist{definition:locally_small_category}{locally small}, this is equivalent to the condition that the \CrefAndHyperrefIfExist{definition:representable_functor_on_a_category_enriched_in_a_monoidal_category}{representable functor}
    \[
    \mathrm{Hom}_{\mathcal{C}}(G, -) : \mathcal{C} \to \mathbf{Set}
    \]
    is \CrefAndHyperrefIfExist{definition:full_and_faithful_functor_between_locally_small_categories}{faithful}, 
    %
    % In other words, for every pair of distinct morphisms \(f, g : X \to Y\) in \(\mathcal{C}\), there exists a morphism \(h : G \to X\) such that
    % \[
    % f \circ h \neq g \circ h.
    % \]
    %
    which in turn is equivalent to the condition that for every object \(X \in \mathcal{C}\), there exists an epimorphism
    \[
    \bigoplus_{i \in I} G \twoheadrightarrow X
    \]
    for some indexing set \(I\), where \(\bigoplus\) denotes the \CrefAndHyperrefIfExist{definition:product_and_coproduct_of_objects_in_a_category}{coproduct} in \(\mathcal{C}\).

    \item A family \(\{G_i\}_{i \in I}\) is called a \hldef{generating family} if for every pair of distinct morphisms \(f, g : X \to Y\) in \(\mathcal{C}\), there exists some index \(i \in I\) and a morphism \(h : G_i \to X\) such that
    \[
    f \circ h \neq g \circ h.
    \]
    In case $\calC$ is locally small, this is equivalent to the condition that the collection of representable functors
    \[
    \{\mathrm{Hom}_{\mathcal{C}}(G_i, -) : \mathcal{C} \to \mathbf{Set}\}_{i \in I}
    \]
    is jointly faithful, which in turn is equivalent to the condition that for every object \(X \in \mathcal{C}\), there exists a family of objects \(\{G_i\}_{i \in J}\) from the generating set indexed by some set \(J\), and an epimorphism
    \[
    \bigoplus_{i \in J} G_i \twoheadrightarrow X.
    \]

\end{enumerate}
\end{definition}

\begin{proposition} \label{proposition:family_in_a_topos_with_small_coproducts_is_a_generating_family_if_and_only_if_every_object_has_an_epimoprhism_from_a_product_from_the_family}
    \TODO{find in SGA}
Let $\mathcal{E}$ be a \CrefAndHyperrefIfExist{definition:topos}{topos} that has all small \CrefAndHyperrefIfExist{definition:product_and_coproduct_of_objects_in_a_category}{coproducts} (which is true for any Grothendieck topos).
A family of objects $\{G_i\}_{i \in I}$ is a \CrefAndHyperrefIfExist{definition:generator_of_a_category}{generating family} if and only if for every object $X \in \mathcal{E}$, there exists a set $J$ and an epimorphism
$$ \coprod_{j \in J} G_{i_j} \twoheadrightarrow X $$
where each $i_j \in I$.
\end{proposition}

% ---------------------------------------------------------------------------
% 3. Generators of a Grothendieck Topos
% ---------------------------------------------------------------------------

\begin{theorem} \label{theorem:sheaves_represented_by_objects_of_small_site_is_a_generating_family_for_the_topos_of_the_site}
    \TODO{find in SGA}
Let $(\mathcal{C}, J)$ be a \CrefAndHyperrefIfExist{definition:locally_small_category}{small} \CrefAndHyperrefIfExist{definition:grothendieck_topology_on_a_category_site_covering_sieve_topologically_generating_family}{site}, and let $\mathcal{E} = \text{Sh}(\mathcal{C}, J)$ be the \CrefAndHyperrefIfExist{definition:topos}{topos} of \CrefAndHyperrefIfExist{definition:sheaf_on_a_site}{sheaves} of sets on it. 
Let \hl{$a \circ y : \mathcal{C} \to \mathcal{E}$} be the composite of the \CrefAndHyperrefIfExist{corollary:yoneda_embedding_on_a_locally_small_category}{Yoneda embedding} followed by the \CrefAndHyperrefIfExist{definition:sheafification_functor_on_a_site}{sheafification functor}.

The family of sheaves $\{ (a \circ y)(U) \mid U \in \operatorname{Ob}(\mathcal{C}) \}$ is a \CrefAndHyperrefIfExist{definition:generator_of_a_category}{generating family} for the topos $\mathcal{E}$.
\end{theorem}


\TODO{}

\begin{definition} \label{definition:coherent_topos}
A \CrefAndHyperrefIfExist{definition:topos}{Grothendieck topos} $\mathcal{E}$ is called a \hldef{coherent topos} if there exists a \CrefAndHyperrefIfExist{definition:coherent_site}{coherent site} $(\mathcal{C}, J)$ such that $\mathcal{E}$ is equivalent to the \CrefAndHyperrefIfExist{definition:sheaf_on_a_site}{category of sheaves} $\mathcal{E} \simeq \text{Sh}(\mathcal{C}, J)$ on that site.

Alternatively, an intrinsic characterization 
\cite[Expos\'e VI]{SGA4_II}
is that $\mathcal{E}$ is coherent if there exists a \CrefAndHyperrefIfExist{definition:generator_of_a_category}{generating family} $\mathcal{G}$ of objects in $\mathcal{E}$ such that:
\begin{enumerate}
    \item Every object in $\mathcal{G}$ is \CrefAndHyperrefIfExist{definition:coherent_object_in_a_locally_small_site}{coherent} (\CrefAndHyperrefIfExist{definition:quasi_compact_object_in_a_locally_small_site}{quasi-compact} and \CrefAndHyperrefIfExist{definition:quasi_seperated_morphism_in_a_locally_small_site}{quasi-separated}).
    \item The full subcategory generated by $\mathcal{G}$ is closed under \CrefAndHyperrefIfExist{definition:cartesian_product_of_two_objects_in_a_category_over_an_object}{fiber products}.
\end{enumerate}
It is a theorem that the full subcategory of \textit{all} coherent objects in a coherent topos is closed under finite limits and finite colimits (forming a so-called \hldef{pretopos}).
\end{definition}



\begin{definition}[{\cite[Definition 1.1]{nlab:point_of_a_topos}}] \label{definition:point_of_a_topos}
    A \hldef{point of a topos $E$} is a \CrefAndHyperrefIfExist{definition:geometric_morphism_between_topoi}{geometric morphism} $x: \Sets \to E$\CrefIfExists{definition:category_of_sets}. 

    For an object $A \in E$, its inverse image $x^* A \in \Sets$ under such a point $x$ is the \hldef{stalk of $A$ at $x$ of $A$ at $x$}.
\end{definition}


\begin{definition} \label{definition:stalk_of_a_topos_at_a_point}
Let $\mathcal{E}$ be a \CrefAndHyperrefIfExist{definition:topos}{topos}, and let \hl{$p: \mathbf{Set} \to \mathcal{E}$} be a \CrefAndHyperrefIfExist{definition:point_of_a_topos}{point of the topos}. By definition, $p$ is a \CrefAndHyperrefIfExist{definition:geometric_morphism}{geometric morphism}, so it consists of an adjoint pair of functors $(p^*, p_*)$ where the inverse image functor $p^*: \mathcal{E} \to \mathbf{Set}$ is left exact.
\begin{enumerate}
    \item The inverse image functor \hl{$p^*: \mathcal{E} \to \mathbf{Set}$} is called the \hldef{stalk functor associated with the point $p$}.
    
    \item For any object (sheaf) $F \in \mathcal{E}$, the set \hl{$p^*(F)$} is called the \hldef{stalk of $F$ at the point $p$}.
\end{enumerate}
\end{definition}

\begin{remark}
    If $\mathcal{E}$ is the topos of sheaves on a topological space $X$, a classical point $x \in X$ induces a point of the topos $p_x: \mathbf{Set} \to \text{Sh}(X)$. In this case, the stalk functor $p_x^*$ is equivalent to the classical stalk functor at $x$, which is defined as the colimit of sections over open neighborhoods of $x$.
    $$ p_x^*(F) \cong \varinjlim_{x \in U} F(U) $$
\end{remark}

\begin{definition} \label{definition:has_enough_points_for_a_topos}
A \CrefAndHyperrefIfExist{definition:topos}{topos} $\mathcal{E}$ is said to \hldef{have enough points} if the collection of all points of $\mathcal{E}$ is jointly faithful. 

Explicitly, this means that for any two distinct morphisms $f, g: X \to Y$ in $\mathcal{E}$, there exists a point $p: \mathbf{Set} \to \mathcal{E}$ such that the induced maps on stalks are distinct:
$$ p^*(f) \neq p^*(g). $$
Equivalently, a morphism $f$ in $\mathcal{E}$ is an isomorphism if and only if for every point $p$, the map $p^*(f)$ is an isomorphism of sets.
\end{definition}

\begin{theorem} \label{theorem:a_coherent_topos_has_enough_points}
Let $\mathcal{E}$ be a \CrefAndHyperrefIfExist{definition:coherent_topos}{coherent topos}.
Then $\mathcal{E}$ \CrefAndHyperrefIfExist{definition:has_enough_points_for_a_topos}{has enough points}.
\end{theorem}

\begin{theorem}[Deligne's Completeness Theorem for Coherent Topoi] \label{theorem:delignes_completeness_theorem_for_coherent_topoi}
Let $\mathcal{E}$ be a \CrefAndHyperrefIfExist{definition:coherent_topos}{coherent topos}.
Then there exists a \CrefAndHyperrefIfExist{definition:compact_topological_space}{compact} \CrefAndHyperrefIfExist{definition:separation_axioms_of_topology}{Hausdorff} topological space $X$ and an exact functor $\mathcal{E} \to \text{Sh}(X)$ that reflects isomorphisms.
\end{theorem}


\subsection{Morphism of sites and topoi}



\begin{definition} \label{definition:continuous_functor_of_sites}
Let $(\calC,J)$ and $(\calD,K)$ be \CrefAndHyperrefIfExist{definition:grothendieck_topology_on_a_category_site_covering_sieve_topologically_generating_family}{sites}. 

A functor $u : \calC \to \calD$ is said to be a \hldef{continuous functor of sites} if, for every object $U \in \operatorname{Ob}(\calD)$ and every \CrefAndHyperrefIfExist{definition:grothendieck_topology_on_a_category_site_covering_sieve_topologically_generating_family}{covering sieve} $S \in K(U)$, the \CrefAndHyperrefIfExist{definition:pullback_sieve_of_an_object_in_a_category_via_a_morphism_to_the_object}{pullback sieve $u^*S$} belongs to $J(V)$ for all $V \in \calC$ with a morphism $u(V) \to U$ in $\calD$.

Equivalently, $u$ is continuous if for every \CrefAndHyperrefIfExist{definition:sheaf_on_a_site}{sheaf} of sets $F$ on $\calD$, the \CrefAndHyperrefIfExist{definition:presheaf_on_a_category}{presheaf} $\calC^{\op} \to \Sets, X \mapsto F(u(X))$ is a sheaf on $\calC$. 
\TODO{show these are equivalent}
\TODO{define morphism of sites and recheck ref's to this definition}

% A \hldef{morphism of sites} $f: (\calD, K) \to (\calC, J)$ 

% Synonymously, we call a continuous functor $u: C \to D$ a \hldef{morphism of sites}.
\end{definition}

\begin{definition} \label{definition:continuous_functor_between_sites_of_opens_on_topological_spaces_induced_by_continuous_map}
    Let $(X, \tau_X)$ and $(Y, \tau_Y)$ be \CrefAndHyperrefIfExist{definition:topological_space}{topological spaces}, and let $f : X \to Y$ be a \CrefAndHyperrefIfExist{definition:continuous_map_of_topological_spaces}{continuous map}.
    Let $\operatorname{Open}(X)$ and $\operatorname{Open}(Y)$ be their respective \CrefAndHyperrefIfExist{definition:category_of_opens_of_a_topological_space}{categories of open sets} with inclusion morphisms, equipped with the \CrefAndHyperrefIfExist{definition:site_of_opens_on_a_topological_space}{canonical} \CrefAndHyperrefIfExist{definition:grothendieck_topology_on_a_category_site_covering_sieve_topologically_generating_family}{Grothendieck topologies} given by open coverings.

    Define the functor 
    $$\hlin{f^{-1} : \operatorname{Open}(Y) \to \operatorname{Open}(X), \quad U \mapsto f^{-1}(U).}$$
    It is a \CrefAndHyperrefIfExist{definition:continuous_functor_of_sites}{continuous functor of sites} from $\operatorname{Open}(Y)$ to $\operatorname{Open}(Y)$ which induces a \CrefAndHyperrefIfExist{definition:morphism_of_sites}{site morphism} 
    $$f: (\operatorname{Open}(X), \text{can}) \to (\operatorname{Open}(Y), \text{can})$$.
\end{definition}
\begin{definition} \label{definition:continuous_functors_on_sites_on_schemes_induced_by_scheme_morphism}
        Let $f : X \to Y$ be a morphism of schemes, and consider one of the common Grothendieck topologies on schemes such as the Zariski, étale, Nisnevich, fppf, fpqc, or crystalline topology. Denote by $\mathbf{C}(X)$ and $\mathbf{C}(Y)$ the corresponding small \CrefAndHyperrefIfExist{definition:grothendieck_topology_on_a_category_site_covering_sieve_topologically_generating_family}{sites} of $X$ and $Y$ (i.e., categories of morphisms to $X$ and $Y$ respectively equipped with one of these topologies).

        Then the \CrefAndHyperrefIfExist{definition:base_change_of_a_morphism_in_a_category_by_a_morphism}{base change functor}
        \[
        f^{-1} : \mathbf{C}(Y) \to \mathbf{C}(X), \quad (V \to Y) \mapsto (V \times_Y X \to X)
        \]
        \CrefIfExists{definition:cartesian_product_of_two_objects_in_a_category_over_an_object} is a \CrefAndHyperrefIfExist{definition:continuous_functor_of_sites }{continuous functor}. It in fact induces a \CrefAndHyperrefIfExist{definition:morphism_of_sites}{morphism of sites} 
        $$f: (\mathbf{C}(X), \tau_{\mathbf{C}}) \to (\mathbf{C}(Y), \tau_{\mathbf{C}})$$
        where $\tau_{\mathbf{C}}$ denotes the chosen topology (Zariski, étale, Nisnevich, fppf, fpqc, crystalline).
\end{definition}

\begin{definition} \label{definition:morphism_of_sites}
    Let $(\calC,J)$ and $(\calD,K)$ be \CrefAndHyperrefIfExist{definition:grothendieck_topology_on_a_category_site_covering_sieve_topologically_generating_family}{sites}. 
    A \hldef{morphism of sites} $f: (\calD, K) \to (\calC, J)$ consists of a \CrefAndHyperrefIfExist{definition:continuous_functor_of_sites}{continuous functor} $u: \calC \to \calD$ such that the inverse image \TODO{carefully piece together in what sense this is an inverse image functor} functor $\Sh(\calD, K; \Sets) \to \Sh(\calC, J; \Sets)$ (\Cref{definition:sheaf_on_a_site}) is left exact \TODO{left exact}, i.e. commutes with finite \CrefAndHyperrefIfExist{definition:projective_and_inductive_limits_in_categories}{projective limits}. 
\end{definition}

\begin{proposition} \label{proposition:identity_functor_from_finer_site_to_coarser_site_on_the_same_category_is_continuous_and_induces_a_site_morphism}
Let $(\mathcal{C}, J)$ and $(\mathcal{C}, J')$ be two \CrefAndHyperrefIfExist{definition:grothendieck_topology_on_a_category_site_covering_sieve_topologically_generating_family}{sites} on the same underlying category $\mathcal{C}$. 
If the topology $J$ is \CrefAndHyperrefIfExist{definition:coarser_finer_than_for_sites_on_a_category}{finer} than $J'$ on the category $\mathcal{C}$, then the identity functor
$$ \text{id}_{\mathcal{C}}: (\mathcal{C}, J) \to (\mathcal{C}, J') $$
is a \CrefAndHyperrefIfExist{definition:continuous_functor_of_sites}{continuous functor} between sites and in fact induces a \CrefAndHyperrefIfExist{definition:morphism_of_sites}{site morphism}.
$$(\calC, J') \to (\calC, J).$$
\end{proposition}

\subsubsection{Sheafification}
\begin{definition} \label{definition:sheafification_functor_on_a_site}
    Let $\calC$ be a \CrefAndHyperrefIfExist{definition:grothendieck_topology_on_a_category_site_covering_sieve_topologically_generating_family}{site} and let $\calA$ be a \CrefAndHyperrefIfExist{definition:category}{(large) category}.

    Assuming that the \CrefAndHyperrefIfExist{definition:presheaf_on_a_category}{presheaf} category $\PreShv(\calC, \calA)$ (and hence the \CrefAndHyperrefIfExist{definition:sheaf_on_a_site}{sheaf} category $\Shv(\calC, \calA)$) is \CrefAndHyperrefIfExist{definition:locally_small_category}{locally small} (or $U$-locally small if a \CrefAndHyperrefIfExist{definition:grothendieck_universe}{Grothendieck universe} $U$ is available), a \hldef{sheafification functor} refers to a functor
    $$a: \PreShv(\calC, \calA) \to \Shv(\calC, \calA) $$
    that is \CrefAndHyperrefIfExist{definition:adjoint_functors_between_categories_unit_counit_of_adjoint_functors}{left adjoint} to the inclusion functor 
    $$i:\Shv(\calC, \calA) \hookrightarrow \PreShv(\calC, \calA)  .$$
    If such a sheafification functor exists, then it is unique up to unique natural isomorphism. Given a presheaf $P$, the sheafification $a(P)$ is also sometimes called the \hldef{sheaf associated to $P$}.
    \TextIfExists{theorem:sheafification_of_a_presheaf_of_sets_on_a_small_enough_site}{See \Cref{theorem:sheafification_of_a_presheaf_of_sets_on_a_small_enough_site} for common conditions under which sheafification exists.} 
\end{definition}

% See Also
%theorem:sheafification_of_a_presheaf_of_sets_on_a_small_enough_site

\begin{definition} \label{definition:sheafification_of_a_presheaf_on_a_topological_space}
    Let $\calA$ be a \CrefAndHyperrefIfExist{definition:category}{category}  with a \CrefAndHyperrefIfExist{lemma:initial_or_final_object_in_a_category_that_is_also_in_a_full_subcategory_is_initial_or_final_in_the_subcategory}{terminal object}. Let $X$ be a \CrefAndHyperrefIfExist{definition:topological_space}{topological space}.

    \begin{enumerate}
    \item  Let $\mathcal{F}$ be a \CrefAndHyperrefIfExist{definition:presheaf_on_a_topological_space}{presheaf} on $X$ valued in $\calA$. 
    
    A \hldef{sheafification of $\mathcal{F}$} is a pair $(\mathcal{F}^+, \theta)$, where $\mathcal{F}^+$ is a sheaf on $X$ and $\theta: \mathcal{F} \to \mathcal{F}^+$ is a morphism of presheaves, satisfying the following universal property:

    For any sheaf $\mathcal{G}$ on $X$ and any morphism of presheaves $\phi: \mathcal{F} \to \mathcal{G}$, there exists a unique morphism of sheaves $\psi: \mathcal{F}^+ \to \mathcal{G}$ such that $\psi \circ \theta = \phi$.
    $$ \begin{array}{ccc}
    \mathcal{F} & \xrightarrow{\theta} & \mathcal{F}^+ \\
    & \searrow_{\phi} & \downarrow \psi \\
    & & \mathcal{G}
    \end{array} $$

    \item Assuming that the presheaf category $\PreShv(X, \calA)$ (and hence the \CrefAndHyperrefIfExist{definition:sheaf_on_a_topological_space_valued_in_a_category_with_a_terminal_object}{sheaf} category $\Shv(X, \calA)$) are \CrefAndHyperrefIfExist{definition:locally_small_category}{locally small}, a \hldef{sheafification functor} refers to a functor
    $$a: \PreShv(X, \calA) \to \Shv(X, \calA) $$
    that is \CrefAndHyperrefIfExist{definition:adjoint_functors_between_categories_unit_counit_of_adjoint_functors}{left adjoint} to the inclusion functor 
    $$i:\Shv(X, \calA) \hookrightarrow \PreShv(X, \calA)  .$$
    If such a sheafification functor exists, then it is unique up to unique natural isomorphism. Given a presheaf $P$, the sheafification $a(P)$ is also sometimes called the \hldef{sheaf associated to $P$}.
    \TextIfExists{definition:sheafification_functor_on_a_site}{The sheafification functor on $\PreShv(X, \calA)$ is a special case of the general definition of \CrefAndHyperrefIfExist{definition:sheafification_functor_on_a_site}{sheafification functor}, applied to the \CrefAndHyperrefIfExist{definition:grothendieck_topology_on_a_category_site_covering_sieve_topologically_generating_family}{site} whose underlying category is \CrefAndHyperrefIfExist{definition:category_of_opens_of_a_topological_space}{$\operatorname{Open}(X)$} and whose Grothendieck topology is the \CrefAndHyperrefIfExist{definition:site_of_opens_on_a_topological_space}{canonical one}.}
    \end{enumerate}
\end{definition}

\begin{theorem} \label{theorem:sheafification_for_sheaves_of_sets_on_a_topological_space_exist}
    Let $X$ be a \CrefAndHyperrefIfExist{definition:topological_space}{topological space}. 
    \begin{enumerate}
        \item Let $\calF$ be a \CrefAndHyperrefIfExist{definition:presheaf_on_a_topological_space}{presheaf} of sets (or sets with structure, e.g. groups, rings, abelian groups, $R$-modules, etc.). The \CrefAndHyperrefIfExist{theorem:sheafification_of_a_presheaf_of_sets_on_a_small_enough_site}{sheafification} $(\mathcal{F}^+, \theta)$ exists and is unique up to unique isomorphism. The sheaf $\mathcal{F}^+$ can be constructed as the sheaf of sections of the espace étalé (space of stalks) associated to $\mathcal{F}$. Specifically, for any open $U \subseteq X$, $\mathcal{F}^+(U)$ is the set of functions $s: U \to \coprod_{x \in U} \mathcal{F}_x$ such that:
        \begin{enumerate}
            \item For each $x \in U$, $s(x) \in \mathcal{F}_x$.
            \item For each $x \in U$, there exists an open neighborhood $V \subseteq U$ of $x$ and a section $t \in \mathcal{F}(V)$ such that $s(y) = t_y$ for all $y \in V$ (where $t_y$ denotes the germ of $t$ at $y$).
        \end{enumerate}

    \item A \CrefAndHyperrefIfExist{definition:sheafification_of_a_presheaf_on_a_topological_space}{sheafification functor} 
    $$a: \PreShv(X, \calA) \to \Shv(X, \calA) $$
    exists.
    \end{enumerate}
\end{theorem}


The following expresses the idea that sheafification functors exist for small enough sites.

\begin{theorem}{cf. {\cite[Expos\'e II, Th\'eor\`eme 3.4]{SGA4_I}}} \label{theorem:sheafification_of_a_presheaf_of_sets_on_a_small_enough_site}
    \begin{enumerate}
        \item Let $U$ be a universe. Let $\calC$ be a \hyperrefIfExists{definition:grothendieck_topology_on_a_category_site_covering_sieve_topologically_generating_family}{$U$-site}\CrefIfExists{definition:grothendieck_topology_on_a_category_site_covering_sieve_topologically_generating_family}. A \CrefAndHyperrefIfExist{definition:sheafification_of_a_presheaf_on_a_topological_space_valued_in_a_category_admitting_direct_colimits}{sheafification functor}
        $$a: \Shv(\calC, \USets) \to \PreShv(\calC, \USets).$$
        exists. 
        % The inclusion functor 
        % $$i: \PreShv(\calC, \USets) \hookrightarrow \Shv(\calC, \USets)$$
        % has a \hyperrefIfExists{definition:adjoint_functors_between_categories_unit_counit_of_adjoint_functors}{left adjoint functor}\CrefIfExists{definition:adjoint_functors_between_categories_unit_counit_of_adjoint_functors}

        \item Let $\calC$ be a site whose underlying category is \CrefAndHyperrefIfExist{definition:locally_small_category}{locally small} and which has a \CrefAndHyperrefIfExist{definition:grothendieck_topology_on_a_category_site_covering_sieve_topologically_generating_family}{topologically generating family} that is a set (rather than a proper class). A sheafification functor 
        $$a: \Shv(\calC, \Sets) \to \PreShv(\calC, \Sets)$$
        exists.

        \item (see e.g. {\cite[3]{nlab:sheafification}}) Let $(\calC, J)$ be a \CrefAndHyperrefIfExist{definition:grothendieck_topology_on_a_category_site_covering_sieve_topologically_generating_family}{site} on an \CrefAndHyperrefIfExist{definition:essentially_small_category}{essentially small category} $\calC$. Suppose that the category $\calA$ is \CrefAndHyperrefIfExist{definition:complete_and_cocomplete_category}{complete, cocomplete}, that small \CrefAndHyperrefIfExist{definition:projective_and_inductive_limits_in_categories}{filtered colimits} in $\calA$ are exact, and that $\calA$ satisfies the IPC-property. A \CrefAndHyperrefIfExist{definition:sheafification_functor_on_a_site}{sheafification functor} 
        $$a: \PreShv(\calC, \calA) \to \Shv(\calC, \calA) $$
        exists.
        \TODO{IPC-property, exactess in this context.}

        \TODO{state as a fact that these categories are complete, cocomplete, with small filtered colimits that are exact}
        This is true for instance of $\calA = \mathbf{Set}, \mathbf{Grp}$, $k-\mathbf{Alg}$ for a field $k$, or $\mathbf{Mod}_R$ for a \CrefAndHyperrefIfExist{definition:ring}{(not necessarily commutative unital) ring $R$}.
    \end{enumerate}
\end{theorem}
\begin{remark}
    If the presheaf is valued in nice ``algebraic category'', e.g. groups, abelian groups, rings, modules over a ring, etc., then the sheafification is again valued in that category. \TODO{Make this more precise.}
\end{remark}


\begin{definition}[Constant sheaf on a site] \label{definition:constant_sheaf_on_a_site_with_sheafification}
    Let $\calC$ be a \hyperrefIfExists{definition:category}{(large) category}\CrefIfExists{definition:category}, let $\calA$ be a (large category), and let $A$ be an object of $\calA$. %a set (or more generally, an abelian group, ring, etc.).
    
    \begin{enumerate}
        \item The \hldef{constant presheaf on $\calC$ with value $A$} is the \hyperrefIfExists{definition:presheaf_on_a_category}{presheaf}\CrefIfExists{definition:presheaf_on_a_category} $P$ defined by
        \[
        P(U) = A
        \]
        for every object $U$ of $\calC$ such that every morphism $f: V \to U$ in $\calC$ induces the identity map $A = P(U)\to P(V) = A$. 

        \item Let $\calC$ be a \CrefAndHyperrefIfExist{definition:grothendieck_topology_on_a_category_site_covering_sieve_topologically_generating_family}{site} and assume that a \CrefAndHyperrefIfExist{definition:sheafification_functor_on_a_site}{sheafification functor} 
        $$a: \Shv(\calC, \calA) \to \PreShv(\calC, \calA)$$
        exists\TextIfExists{theorem:sheafification_of_a_presheaf_of_sets_on_a_small_enough_site}{~(e.g. see \Cref{theorem:sheafification_of_a_presheaf_of_sets_on_a_small_enough_site})}.
        The \hldef{constant sheaf on $\calC$ with value $A$}, or the \hldef{constant sheaf on $\calC$ associated to $A$} commonly denoted \hl{$\underline{A}$} or sometimes just \hl{$A$} by abuse of notation, is the \hyperrefIfExists{theorem:sheafification_of_a_presheaf_of_sets_on_a_small_enough_site}{sheaf associated to}\CrefIfExists{theorem:sheafification_of_a_presheaf_of_sets_on_a_small_enough_site} the constant presheaf $P$ with value $A$ above.

        \item Let $\calC$ be a site. Let $\calO$ be a sheaf of (not-necessarily commutative) rings on $\calC$. Assume that the \CrefAndHyperrefIfExist{definition:sections_of_a_presheaf_on_a_category_valued_in_a_category}{global sections ring $\Gamma(\calO)$} exists. A \hldef{constant $\calO$-module} is an \CrefAndHyperrefIfExist{definition:module_over_a_sheaf_of_rings_on_a_site}{$\calO$-module} $\calF$ which is isomorphic as a sheaf to the constant sheaf on $\calC$ with value $M$ where $M$ is a module of the ring $\Gamma(\calO)$. Note that sheafification functors exist for presheaves/sheaves valued in $\Ab$ (\Cref{theorem:sheafification_of_a_presheaf_of_sets_on_a_small_enough_site}).

        In case that $\calO$ is the constant sheaf associated to $A$ for some (not-necessarily commutative) ring $A$, a constant $\calO$-module is simply called a \hldef{constant $A$-module}.
    \end{enumerate}
\end{definition}


\begin{definition} \label{definition:restriction_of_a_sheaf_on_a_site_to_an_object_of_the_underlying_category_of_the_site}
    Let $(\calC, J)$ be a \CrefAndHyperrefIfExist{definition:grothendieck_topology_on_a_category_site_covering_sieve_topologically_generating_family}{site}, let $\calF$ be a \CrefAndHyperrefIfExist{definition:sheaf_on_a_site}{sheaf} on $(\calC, J)$ valued in some category, and let $U \in \Ob(\calC)$ be some object. 
    The \hldef{restriction of $\calF$ to $U$} is the sheaf \hl{$\calF|_U$} on the \CrefAndHyperrefIfExist{definition:site_induced_by_a_site_on_an_over_category}{induced site $(\calC/U, J/U)$} defined by 
    $$\calF|_U(f: V \to U) = F(V).$$

    Equivalently, the forgetful functor $j_U: \calC/U \to \calC$ forgetful functor is a \CrefAndHyperrefIfExist{definition:continuous_functor_of_sites}{continuous functor} of sites and $\calF|_U$ is given by the \CrefAndHyperrefIfExist{definition:direct_image_of_a_sheaf_on_a_site_under_a_continuous_functor_of_sites_or_a_site_morphism}{inverse image $j_U^* \calF$}.
\end{definition}


\begin{definition}[Locally constant sheaf on a site] \label{definition:locally_constant_sheaf_on_a_site_with_sheafification}

    Let $(\calC, J)$ be a \CrefAndHyperrefIfExist{definition:grothendieck_topology_on_a_category_site_covering_sieve_topologically_generating_family}{site}. 
\begin{enumerate}
    \item 

    Let $\calA$ be a (large) category, and let $A$ be an object of $\calA$. %a set (or more generally, an abelian group, ring, etc.).
    \TODO{If such a sheafification functor exist, does a sheafification functor exist when restricted to an object $U$?}
    Assume that a \CrefAndHyperrefIfExist{definition:sheafification_functor_on_a_site}{sheafification functor} 
    $$a: \PreShv(\calC, J, \calA) \to \Shv(\calC,J, \calA)$$
    \CrefIfExists{definition:sheaf_on_a_site}
    \CrefIfExists{definition:presheaf_on_a_category}
    exists\TextIfExists{theorem:sheafification_of_a_presheaf_of_sets_on_a_small_enough_site}{~(e.g. see \Cref{theorem:sheafification_of_a_presheaf_of_sets_on_a_small_enough_site})}. Let $U$ be an object of $\calC$.

    A sheaf $\mathcal{F}$ on $C$ with values in $\calA$ is said to be a \hldef{locally constant sheaf on $U$ with value $A$} if there exists a \CrefAndHyperrefIfExist{definition:grothendieck_topology_on_a_category_site_covering_sieve_topologically_generating_family}{covering sieve} $\{ U_i \to U \}$ of every object $U$ in $C$ such that for each $i$, the \CrefAndHyperrefIfExist{definition:restriction_of_a_sheaf_on_a_site_to_an_object_of_the_underlying_category_of_the_site}{restriction $\mathcal{F}|_{U_i}$} is isomorphic to the \CrefAndHyperrefIfExist{definition:constant_sheaf_on_a_site_with_sheafification}{constant sheaf} $\underline{A}$ on the \CrefAndHyperrefIfExist{definition:site_induced_by_a_site_on_an_over_category}{slice site $\calC_{/{U_i}}$}. 

    If $\calC$ has a final object, then we may say that $\calF$ is a \hldef{locally constant sheaf with value $A$} if it is a locally constants sheaf on the final object with value $A$.
    
    %constant sheaf $\underline{A}|_{U_i}$.

    % In other words,
    % \[
    % \forall U \in C,\, \exists\, \text{cover} \{ U_i \to U \} :\, \forall i,\, \mathcal{F}|_{U_i} \cong \underline{A}|_{U_i}.
    % \]

    \item Let $\calO$ be a sheaf of (not-necessarily commutative) rings on $\calC$. 
    % Assume that the \CrefAndHyperrefIfExist{definition:sections_of_a_presheaf_on_a_category_valued_in_a_category}{global sections ring $\Gamma(\calO)$} exists. 
    A \hldef{locally constant $\calO$-module} is a \CrefAndHyperrefIfExist{definition:module_over_a_sheaf_of_rings_on_a_site}{$\calO$-module} $\calF$ such that there exists a \CrefAndHyperrefIfExist{definition:grothendieck_topology_on_a_category_site_covering_sieve_topologically_generating_family}{covering sieve} $\{U_i \to U\}$ for every object $U$ in $C$ such that for each $i$, the \CrefAndHyperrefIfExist{definition:restriction_of_a_sheaf_on_a_site_to_an_object_of_the_underlying_category_of_the_site}{restriction} $\calF|_{U_i}$ is isomorphic, as an $\calO|_{U_i}$-module, to a \CrefAndHyperrefIfExist{definition:constant_sheaf_on_a_site_with_sheafification}{constant $\calO|_{U_i}$-module}\footnote{The \CrefAndHyperrefIfExist{definition:sections_of_a_presheaf_on_a_category_valued_in_a_category}{global sections rings $\Gamma(\calO|_{U_i})$} of the sheaves $\calO|_{U_i}$ on \CrefAndHyperrefIfExist{definition:site_induced_by_a_site_on_an_over_category}{the slice sites $\calC_{/U_i}$} exist because each $\calC_{/U_i}$ has a \CrefAndHyperrefIfExist{definition:initial_final_zero_objects_of_a_category}{final object}(\Cref{lemma:slice_category_has_final_object}), so we may speak of constant $\calO|_{U_i}$-modules.}.

    We additionally say that $\calF$ is 
    \begin{enumerate}
        \item \hldef{locally free of rank $r$ over $\calO$} if there exists a covering $\{U_i \to U\}$ such that $\calF|_{U_i}$ is isomorphic as a $\calO|_{U_i}$-module to $(\calO|_{U_i})^{\oplus r}$ for each $i$.

        \item \hldef{free of rank $r$ over $\calO$} if $\calF \cong \calO^{\oplus r}$ as $\calO$-modules.

        \item \hldef{of finite type} if there exists a covering $\{U_i \to U\}$ such that $\calF|_{U_i}$ generated by finitely many sections over $U_i$ as an $\calO|_{U_i}$-module. In other words, there is an epimorphism 
        $$(\calO|_{U_i})^{\oplus n_i} \to \calF|_{U_i}$$
        of $\calO|_{U_i}$-modules for each $i$.
    \end{enumerate}

    In case that $\calO$ is the \CrefAndHyperrefIfExist{definition:constant_sheaf_on_a_site_with_sheafification}{constant sheaf on $\calC$ associated to $A$} for some (not-necessarily commutative) ring $A$, a locally constant $\calO$-module is simply called a \hldef{locally constant $A$-module}.

    

    % \item Let $\Lambda$ be a commutative ring. A locally constant sheaf $\calF$ of $\Lambda$-modules is said to be 
    % \begin{enumerate}
    %     \item \hldef{locally free of rank $r$ over $\Lambda$} if it is valued in $\Lambda^{\oplus r}$.
    %     \item \hldef{of finite type} if it is locally valued in some finitely generated $\Lambda$-modules. 
    % \end{enumerate}
\end{enumerate}
\end{definition}


% \begin{definition}[Locally Constant Sheaf on a General Site]
% Let \((\mathcal{C}, J)\) be a site and let \(\Lambda\) be a commutative ring. A sheaf \(\mathcal{F}\) of \(\Lambda\)-modules on \(\mathcal{C}\) is said to be \hldef{locally constant} if for every object \(U \in \mathcal{C}\) there exists a covering \(\{U_i \to U\}\) in the topology \(J\) such that for each \(i\), the restricted sheaf \(\mathcal{F}|_{U_i}\) is isomorphic to a constant sheaf on \(U_i\) associated to a \(\Lambda\)-module \(M_i\).

% Moreover, \(\mathcal{F}\) is said to be of \hldef{finite rank over $\Lambda$} if for each such \(U_i\), the \(\Lambda\)-module \(M_i\) is a finitely generated free \(\Lambda\)-module.
% \end{definition}



\subsection{Inverse image and direct image of sheaves}

\begin{definition}[Relative Slice Category] \label{definition:relative_slice_category_of_objects_over_an_object_relative_to_a_functor}
    Let $\calC$, $\calD$ be \CrefAndHyperrefIfExist{definition:category}{categories}, let $u: \mathcal{D} \to \mathcal{C}$ be a \CrefAndHyperrefIfExist{definition:functor_between_categories}{functor} and let $U$ be an object of $\mathcal{C}$. The \hldef{relative slice category} (also called the \hldef{relative over-category} or \hldef{category of objects over $U$ relative to $u$}), denoted \hl{$(u \downarrow U)$} or \hl{$u/U$}, is defined as follows:
    \begin{itemize}
        \item \textbf{Objects:} Pairs $(V, \varphi)$, where $V$ is an object of $\mathcal{D}$ and $\varphi: u(V) \to U$ is a morphism in $\mathcal{C}$.
        \item \textbf{Morphisms:} A morphism from $(V, \varphi)$ to $(V', \varphi')$ is a morphism $f: V \to V'$ in $\mathcal{D}$ such that the following triangle in $\mathcal{C}$ commutes:
        $$
        \begin{tikzcd}
            u(V) \arrow[rr, "u(f)"] \arrow[dr, "\varphi"'] & & u(V') \arrow[dl, "\varphi'"] \\
            & U &
        \end{tikzcd}
        $$
        That is, $\varphi' \circ u(f) = \varphi$.
        \item \textbf{Composition:} Inherited from $\mathcal{D}$.
    \end{itemize}
    \TextIfExists{definition:comma_category_of_two_functors_to_a_category}{Slice categories are special cases of \CrefAndHyperrefIfExist{definition:comma_category_of_two_functors_to_a_category}{comma categories}.}
\end{definition}


\begin{definition} \label{definition:direct_image_of_a_sheaf_on_a_site_under_a_continuous_functor_of_sites_or_a_site_morphism}
Let $(\calC,J)$ and $(\calD,K)$ be \CrefAndHyperrefIfExist{definition:grothendieck_topology_on_a_category_site_covering_sieve_topologically_generating_family}{sites}  with small \CrefAndHyperrefIfExist{definition:grothendieck_topology_on_a_category_site_covering_sieve_topologically_generating_family}{topological generating families} (or $U$-small topologically generating families if a \CrefAndHyperrefIfExist{definition:grothendieck_universe}{universe} $U$ is available), and let $u : \calC \to \calD$ be a \CrefAndHyperrefIfExist{definition:continuous_functor_of_sites}{continuous functor of sites}. 
%Let $\mathcal{A}$ be a (large) category which has all small (or $U$-small) \CrefAndHyperrefIfExist{definition:product_and_coproduct_of_objects_in_a_category}{products}.

For any \CrefAndHyperrefIfExist{definition:sheaf_on_a_site}{sheaf} 
\[
\mathcal{F} \in \operatorname{Sh}(\calD,K;\mathcal{A}),
\]
Define the \hldef{pushforward/direct image sheaf} \hl{$u^s \calF$} by 
$$\hlin{u^s \mathcal{F} := \mathcal{F} \circ u : \calC^{\mathrm{op}} \to \mathcal{A}.}$$
Because $u$ is continuous, $u^s\mathcal{F}$ is a sheaf on $(\calC,J)$ valued in $\mathcal{A}$. The assignment $\mathcal{F} \mapsto u^s\mathcal{F}$ defines the \hldef{direct image/pushforward functor}
$$\hlin{u^s : \operatorname{Sh}(\calD,K;\mathcal{A}) \to \operatorname{Sh}(\calC,J;\mathcal{A}).}$$

If $u$ is the functor underlying a \CrefAndHyperrefIfExist{definition:morphism_of_sites}{site morphism} $f: (\calD, K) \to (\calC, J)$, we may alternatively denote $u^s \calF$ by \hl{$f_* \calF$} and call it the \hldef{direct image/pushforward of $\calF$ under $f$}; the assignment $\calF \mapsto f_* \calF$ is then the \hldef{direct image/pushforward functor}.
$$\hlin{f_* : \operatorname{Sh}(\calD,K;\mathcal{A}) \to \operatorname{Sh}(\calC,J;\mathcal{A}).}$$

Note that while the continuous functor $u$ and the site morphism $f$ point in opposite directions, the definition $f_* := u^s$ ensures that $f_*$ corresponds to the standard geometric pushforward used in topology and algebraic geometry.



% We further note that $u^*$ is ``categorical'' notation whereas $f_*$ is ``geometric'' notation; loosely speaking, given a morphism $f: X \to Y$ of topological spaces or schemes, 
% \begin{itemize}
%     \item we may have a continuous functor $u: \mathbf{C}(Y) \to \mathbf{C}(X)$ where $\mathbf{C}(X), \mathbf{C}(Y)$ are appropriate sites induced by $X$ and $Y$ respectively,
%     \item $u$ may underlie a site morphism $f: \mathbf{C}(X) \to \mathbf{C}(Y)$ roughly given by pullbacks under the morphism $f: X \to Y$, and
%     \item given a sheaf $\calF$ on $\mathbf{C}(X)$, we may speak of its direct image $f_* \calF$ on $\mathbf{C}(Y)$.
% \end{itemize}
\end{definition}
% \begin{definition} \label{definition:inverse_image_of_a_sheaf_under_a_continuous_functor_of_sites_or_a_site_morphism}
% Let $(\calC,J)$ and $(\calD,K)$ be \CrefAndHyperrefIfExist{definition:grothendieck_topology_on_a_category_site_covering_sieve_topologically_generating_family}{sites}  with small \CrefAndHyperrefIfExist{definition:grothendieck_topology_on_a_category_site_covering_sieve_topologically_generating_family}{topological generating families} (or $U$-small topologically generating families if a \CrefAndHyperrefIfExist{definition:grothendieck_universe}{universe} $U$ is available), and let $u : \calC \to \calD$ be a \CrefAndHyperrefIfExist{definition:continuous_functor_of_sites}{continuous functor of sites}. Let $\mathcal{A}$ be a (large) category which has all small (or $U$-small) \CrefAndHyperrefIfExist{definition:product_and_coproduct_of_objects_in_a_category}{products}. For any \CrefAndHyperrefIfExist{definition:sheaf_on_a_site}{sheaf} 
% \[
% \mathcal{G} \in \operatorname{Sh}(\calC,J;\mathcal{A}),
% \]
% the \hldef{direct image/pushforward sheaf of $\mathcal{G}$ under $u$} is defined by
% $$\hlin{u_*\mathcal{G} : \calD^{\mathrm{op}} \to \mathcal{A}, \quad V \mapsto \varprojlim_{(u \downarrow V)^{\op}} \mathcal{G}(U),}$$
% where the \CrefAndHyperrefIfExist{definition:projective_and_inductive_limits_in_categories}{limit} is taken over the \CrefAndHyperrefIfExist{definition:opposite_category_of_a_category}{opposite} of the \CrefAndHyperrefIfExist{definition:comma_category_of_two_functors_to_a_category}{comma category $(u \downarrow V)$} of whose objects are pairs $(U, u(U) \to V)$ with $U \in \calC$ and $u(U) \to V$ a moprhism in $\calD$. 

% The assignment $\mathcal{G} \mapsto u_*\mathcal{G}$ defines the \hldef{direct image functor}
% $$\hlin{u_* : \operatorname{Sh}(\calC,J;\mathcal{A}) \to \operatorname{Sh}(\calD,K;\mathcal{A})}.$$

% If $u$ is the functor underlying a \CrefAndHyperrefIfExist{definition:morphism_of_sites}{site morphism} $f: (\calD, K) \to (\calC, J)$, we may alternatively denote $u_* \calG$ by \hl{$f^* \calG$} and call it the \hldef{inverse image/pullback of $\calG$ under $f$}; the assignment $\calG \mapsto f^* \calG$ is then the \hldef{inverse image/pullback functor}.
% $$\hlin{f^* : \operatorname{Sh}(\calC,J;\mathcal{A}) \to \operatorname{Sh}(D,K;\mathcal{A}).}$$

% We further note that $u_*$ is ``categorical'' notation whereas $f^*$ is ``geometric'' notation; loosely speaking, given a morphism $f: X \to Y$ of topological spaces or schemes, 
% \begin{itemize}
%     \item we may have a continuous functor $u: \mathbf{C}(Y) \to \mathbf{C}(X)$ where $\mathbf{C}(X), \mathbf{C}(Y)$ are appropriate sites induced by $X$ and $Y$ respectively,
%     \item $u$ may underlie a site morphism $f: \mathbf{C}(X) \to \mathbf{C}(Y)$ roughly given by pullbacks under the morphism $f: X \to Y$, and
%     \item given a sheaf $\calG$ on $\mathbf{C}(Y)$, we may speak of its direct image $f^* \calG$ on $\mathbf{C}(X)$.
% \end{itemize}

% \end{definition}

\begin{definition} \label{definition:inverse_image_of_a_sheaf_under_a_continuous_functor_of_sites_or_a_site_morphism}
    % \begin{definition} \label{definition:inverse_image_of_a_sheaf_on_a_site_under_a_continuous_functor_of_sites}
Let $(\calC,J)$ and $(\calD,K)$ be \CrefAndHyperrefIfExist{definition:grothendieck_topology_on_a_category_site_covering_sieve_topologically_generating_family}{sites} with small \CrefAndHyperrefIfExist{definition:grothendieck_topology_on_a_category_site_covering_sieve_topologically_generating_family}{topological generating families}, and let $u : \calC \to \calD$ be a \CrefAndHyperrefIfExist{definition:continuous_functor_of_sites}{continuous functor of sites}. Let $\mathcal{A}$ be a (large) category such that the \CrefAndHyperrefIfExist{definition:presheaf_on_a_category}{presheaf category} $\operatorname{PreSh}(\calD,K;\mathcal{A})$ has \CrefAndHyperrefIfExist{definition:sheafification_functor_on_a_site}{sheafification}.

% Let $\mathcal{A}$ be a category admitting all small colimits and finite limits. 
For any \CrefAndHyperrefIfExist{definition:sheaf_on_a_site}{sheaf} 
\[
\mathcal{G} \in \operatorname{Sh}(\calC,J;\mathcal{A}),
\]
the \hldef{inverse image/pullback sheaf of $\mathcal{G}$ under $u$} is defined, assuming that all colimits below exist, as:
$$\hlin{u_s \mathcal{G} : \calD^{\mathrm{op}} \to \mathcal{A}, \quad V \mapsto a \left( \varinjlim_{(V \downarrow u)} \mathcal{G}(U) \right),}$$
where $a$ is the \CrefAndHyperrefIfExist{definition:definition:sheafification_functor_on_a_site}{sheafification} functor of presheaves and the \CrefAndHyperrefIfExist{definition:limit_and_colimit_of_a_diagram_in_a_category}{colimit} is taken over the \CrefAndHyperrefIfExist{definition:comma_category_of_two_functors_to_a_category}{comma category $(V \downarrow u)$} of pairs $(U, V \to u(U))$ with $U \in \calC$.

The assignment $\mathcal{G} \mapsto u_s\mathcal{G}$ defines the \hldef{inverse image/pullback functor}
$$\hlin{u_s : \operatorname{Sh}(\calC,J;\mathcal{A}) \to \operatorname{Sh}(\calD,K;\mathcal{A})}.$$

If $u$ is the functor underlying a \CrefAndHyperrefIfExist{definition:morphism_of_sites}{site morphism} $f: (\calD, K) \to (\calC, J)$, we may alternatively denote $u_s \calG$ by \hl{$f^{*} \calG$} (or sometimes by \hl{$f^{-1} \calG$}) and call it the \hldef{inverse image/pullback of $\calG$ under $f$}.


Note that while the continuous functor $u$ and the site morphism $f$ point in opposite directions, the identification $f^* := u_s$ ensures that $f^*$ corresponds to the standard geometric pullback. In the case of topological spaces, this recovers the usual construction involving colimits over open neighborhoods to obtain stalks followed by sheafification.

\end{definition}
% \end{definition}
\begin{theorem} \label{theorem:adjunction_between_inverse_images_and_direct_images_of_sheaves_under_continuous_functor_of_sites}
    \TODO{It is likely that some more restrictions are needed; e.g. must $C$ and $D$ be small and $\calA$ locally small to ensure that we can talk about hom's between sheaves?}
Let $(C,J)$ and $(D,K)$ be \CrefAndHyperrefIfExist{definition:grothendieck_topology_on_a_category_site_covering_sieve_topologically_generating_family}{sites}, and let $u : C \to D$ be a \CrefAndHyperrefIfExist{definition:continuous_functor_of_sites}{continuous functor} of sites. Let $\mathcal{A}$ be a (large) category such that the \CrefAndHyperrefIfExist{definition:presheaf_on_a_category}{presheaf category} $\operatorname{PreSh}(\calD,K;\mathcal{A})$ has \CrefAndHyperrefIfExist{definition:sheafification_functor_on_a_site}{sheafification}.


% with all small limits and colimits in which \CrefAndHyperrefIfExist{definition:sheaf_on_a_site}{sheaves} are valued.

% Let $\mathcal{A}$ be a (large) category with all small limits and colimits in which \CrefAndHyperrefIfExist{definition:sheaf_on_a_site}{sheaves} are valued.

Then the \CrefAndHyperrefIfExist{definition:direct_image_of_a_sheaf_on_a_site_under_a_continuous_functor_of_sites_or_a_site_morphism}{inverse image functor}
\[
u_s : \operatorname{Sh}(D,K;\mathcal{A}) \to \operatorname{Sh}(C,J;\mathcal{A})
\]
(assuming that the inverse images of all sheaves on $(D,K)$ valued in $\calA$ exist by virtue of $\calA$ admitting enough \CrefAndHyperrefIfExist{definition:projective_and_inductive_limits_in_categories}{colimits}) and the \CrefAndHyperrefIfExist{definition:inverse_image_of_a_sheaf_under_a_continuous_functor_of_sites_or_a_site_morphism}{direct image functor}
\[
u^s : \operatorname{Sh}(C,J;\mathcal{A}) \to \operatorname{Sh}(D,K;\mathcal{A})
\]
form an \CrefAndHyperrefIfExist{definition:adjoint_functors_between_categories_unit_counit_of_adjoint_functors}{adjoint pair}, i.e., for any sheaves $\mathcal{F} \in \operatorname{Sh}(D,K;\mathcal{A})$ and $\mathcal{G} \in \operatorname{Sh}(C,J;\mathcal{A})$, there is a natural isomorphism
\[
\operatorname{Hom}_{\operatorname{Sh}(C,J)}(u_s\mathcal{F}, \mathcal{G}) \cong \operatorname{Hom}_{\operatorname{Sh}(D,K)}(\mathcal{F}, u^s\mathcal{G}).
\]
\end{theorem}


\begin{definition}[Pushforward (direct image) of a sheaf] \label{definition:direct_image_of_a_sheaf_on_a_topological_space}
    Let $f : X \to Y$ be a \CrefAndHyperrefIfExist{definition:continuous_map_between_open_subsets_of_euclidean_spaces}{continuous map} between \CrefAndHyperrefIfExist{definition:topological_space}{topological spaces}, and let $\mathcal{F}$ be a \CrefAndHyperrefIfExist{definition:presheaf_on_a_topological_space}{presheaf on $X$ valued in a category $\mathcal{D}$} with a \CrefAndHyperrefIfExist{lemma:initial_or_final_object_in_a_category_that_is_also_in_a_full_subcategory_is_initial_or_final_in_the_subcategory}{terminal object}.  
    The \hldef{pushforward} or \hldef{direct image presheaf} 
    \hl{$f_* \mathcal{F}$} on $Y$ is the \CrefAndHyperrefIfExist{definition:presheaf_on_a_category}{presheaf valued in $\calD$ on $Y$} defined as follows: For every open set $V \subseteq Y$, the value of the pushforward is given by
    \[ 
    f_* \mathcal{F}(V) := \mathcal{F}(f^{-1}(V)). 
    \]
    For an inclusion of open sets $V' \subseteq V$ in $Y$, the restriction morphism 
    \[ 
    \operatorname{res}_{V, V'}^{f_* \mathcal{F}} : f_* \mathcal{F}(V) \to f_* \mathcal{F}(V') 
    \]
    is defined as the restriction morphism of $\mathcal{F}$ associated with the inclusion of preimages $f^{-1}(V') \subseteq f^{-1}(V)$ in $X$:
    \[ 
    \operatorname{res}_{f^{-1}(V), f^{-1}(V')}^{\mathcal{F}} : \mathcal{F}(f^{-1}(V)) \to \mathcal{F}(f^{-1}(V')). 
    \]
    
    % by
    % $$f_* \mathcal{F}(V) := \mathcal{F}(f^{-1}(V))$$
    % for every open set $V \subseteq Y$, with restriction maps induced from those of $\mathcal{F}$ via preimages.  

    If $\calF$ is a \CrefAndHyperrefIfExist{definition:sheaf_on_a_topological_space_valued_in_a_category_with_a_terminal_object}{sheaf}, then so is $f_* \calF$. \TextIfExists{definition:direct_image_of_a_sheaf_on_a_site_under_a_continuous_functor_of_sites_or_a_site_morphism}{In this case, it is equivalent to define $f_* \calF$ as the \CrefAndHyperrefIfExist{definition:direct_image_of_a_sheaf_on_a_site_under_a_continuous_functor_of_sites_or_a_site_morphism}{direct image} $(f^{-1})^s \calF$ of $\calF$ under the \CrefAndHyperrefIfExist{definition:continuous_functor_between_sites_of_opens_on_topological_spaces_induced_by_continuous_map}{continuous functor $f^{-1}: \operatorname{Open} Y \to \operatorname{Open} X$}\CrefIfExists{definition:continuous_functor_between_sites_of_opens_on_topological_spaces_induced_by_continuous_map}\CrefIfExists{definition:site_of_opens_on_a_topological_space}, which is also equivalent to the \CrefAndHyperrefIfExist{definition:direct_image_of_a_sheaf_on_a_site_under_a_continuous_functor_of_sites_or_a_site_morphism}{direct image} of $\calF$ under the \CrefAndHyperrefIfExist{definition:morphism_of_sites}{site morphism} $\operatorname{Open} X \to \operatorname{Open} Y$ whose underlying continuous functor is $f^{-1}$.}
    
\end{definition}


\begin{definition}[Pullback (inverse image) of a sheaf] \label{definition:inverse_image_of_a_sheaf_on_a_topological_space}
    Let $f : X \to Y$ be a \CrefAndHyperrefIfExist{definition:continuous_map_between_open_subsets_of_euclidean_spaces}{continuous map} between \CrefAndHyperrefIfExist{definition:topological_space}{topological spaces}. Let $\calD$ be a category  with a \CrefAndHyperrefIfExist{lemma:initial_or_final_object_in_a_category_that_is_also_in_a_full_subcategory_is_initial_or_final_in_the_subcategory}{terminal object}.
    
    \begin{enumerate}
        \item  Let $\mathcal{G}$ be a \CrefAndHyperrefIfExist{definition:presheaf_on_a_topological_space}{presheaf on $Y$ valued in a $\calD$}.  
        The \hldef{pullback} or \hldef{inverse image presheaf} 
        \hl{$f^{-1} \mathcal{G}$} on $X$ is defined as the presheaf 
        $$U \mapsto \varinjlim_{V \supseteq f(U)} \mathcal{G}(V)$$
        where $U$ ranges over open subsets of $X$ and the colimit is taken over all open subsets $V \subseteq Y$ containing $f(U)$.  
        \TextIfExists{definition:stalk_of_a_presheaf_on_a_topological_space_at_a_point}{This construction admits a natural isomorphism
        $$(f^{-1}\mathcal{G})_x \to \mathcal{G}_{f(x)}$$
        of \CrefAndHyperrefIfExist{definition:stalk_of_a_presheaf_on_a_topological_space_at_a_point}{stalks} for every $x \in X$.}  

        \item 
        If $\calG$ is a \CrefAndHyperrefIfExist{definition:sheaf_on_a_topological_space_valued_in_a_category_with_a_terminal_object}{sheaf} valued in $\calD$, then we can define the \hldef{pullback} or \hldef{inverse image sheaf} \hl{$f^* \calG$} on $X$ as the \CrefAndHyperrefIfExist{definition:sheafification_functor_on_a_site}{sheaf associated to the presheaf} $f^{-1} \calG$, assuming it exists.
        \TextIfExistsElse{definition:direct_image_of_a_sheaf_on_a_site_under_a_continuous_functor_of_sites_or_a_site_morphism}{

            Assuming that a \CrefAndHyperrefIfExist{definition:sheafification_functor_on_a_site}{sheafification functor} exists, one may equivalently define $f^* \calG$ via \Cref{definition:inverse_image_of_a_sheaf_under_a_continuous_functor_of_sites_or_a_site_morphism} --- More concretely, $f^* \calG$ is the following equivalent constructions:
            \begin{itemize}
                \item The \CrefAndHyperrefIfExist{definition:inverse_image_of_a_sheaf_under_a_continuous_functor_of_sites_or_a_site_morphism}{direct image} $(f^{-1})_s \calG$ of $\calG$ under the continuous functor $f^{-1}: \operatorname{Open} Y \to \operatorname{Open} X, \quad W \mapsto f^{-1}(W)$\CrefIfExists{definition:site_of_opens_on_a_topological_space}\CrefIfExists{definition:continuous_functor_between_sites_of_opens_on_topological_spaces_induced_by_continuous_map}. 

                \item The \CrefAndHyperrefIfExist{definition:inverse_image_of_a_sheaf_under_a_continuous_functor_of_sites_or_a_site_morphism}{inverse image} of $\calG$ under the \CrefAndHyperrefIfExist{definition:morphism_of_sites}{site morphism} $\operatorname{Open} X \to \operatorname{Open} Y$ whose underlying \CrefAndHyperrefIfExist{definition:continuous_functor_of_sites}{continuous functor} is $f^{-1}$

            \end{itemize}
            
        }{
        }

    \end{enumerate}
\end{definition}

\begin{theorem} \label{theorem:adjunction_between_inverse_image_and_direct_image_functors_for_sheaves_on_topological_spaces_under_a_continuous_map}
    Let $f : X \to Y$ be a \CrefAndHyperrefIfExist{definition:continuous_map_between_open_subsets_of_euclidean_spaces}{continuous map} between \CrefAndHyperrefIfExist{definition:topological_space}{topological spaces}. Let $\calA$ be a \CrefAndHyperrefIfExist{definition:locally_small_category}{locally small category} such that $\calA$ is \CrefAndHyperrefIfExist{definition:complete_and_cocomplete_category}{cocomplete}, i.e. admits \CrefAndHyperrefIfExist{definition:small_and_finite_limits_and_colimits_in_a_category}{small colimits}, and such that $\PreShv(Y, \calA)$ admits \CrefAndHyperrefIfExist{definition:sheafification_of_a_presheaf_on_a_topological_space}{sheafification}.
    
    The \CrefAndHyperrefIfExist{definition:inverse_image_of_a_sheaf_on_a_topological_space}{inverse image functor} 
    $$f^*: \Sh(Y, \calA) \to \Sh(X, \calA)$$
    and the \CrefAndHyperrefIfExist{definition:direct_image_of_a_sheaf_on_a_topological_space}{direct image functor}
    $$f_*: \Sh(X, \calA) \to \Sh(Y, \calA)$$
    form an \CrefAndHyperrefIfExist{definition:adjoint_functors_between_categories_unit_counit_of_adjoint_functors}{adjoint pair} $f^* \dashv f_*$, i.e., for any sheaves $\calF \in \Sh(Y, \calA)$ and $\calG \in \Sh(X, \calA)$, there is a natural isomorphism
    $$\Hom_{\Sh(X, \calA)}(f^* \calF, \calG) \cong \Hom_{\Sh(Y, \calA)}(\calF, f_* \calG).$$
\end{theorem}


\begin{definition}  \label{definition:inverse_image_and_direct_image_of_sheaves_on_schemes_for_various_topologies}
    \TODO{think about if assumptions on the small site and $\calA$ are needed}
Let $f : X \to Y$ be a morphism of schemes, and let $\mathbf{C}(X)$ and $\mathbf{C}(Y)$ be \CrefAndHyperrefIfExist{definition:big_site_on_the_category_of_schemes_over_a_scheme_and_small_site}{small sites} associated to $X$ and $Y$ respectively, equipped with any common Grothendieck topologies such as Zariski, étale, Nisnevich, fppf, fpqc, or crystalline. Let $\calA$ be a (large) category.

\begin{enumerate}
    \item Given a sheaf $\calG \in \operatorname{Sh}(\mathbf{C}(Y); \mathcal{A})$, the \hldef{inverse image} \hl{$f^* \calG$} along $f$ is defined to be the \CrefAndHyperrefIfExist{definition:inverse_image_of_a_sheaf_under_a_continuous_functor_of_sites_or_a_site_morphism}{inverse image} of $\calG$ under the \CrefAndHyperrefIfExist{definition:continuous_functor_of_sites}{continuous functor} \CrefAndHyperrefIfExist{definition:continuous_functors_on_sites_on_schemes_induced_by_scheme_morphism}{$f^{-1} : \mathbf{C}(Y) \to \mathbf{C}(X)$} on sites. In particular, it is an object of $\operatorname{Sh}(\mathbf{C}(X); \mathcal{A})$ and $f^*$ yields a functor
    $$f^* : \operatorname{Sh}(\mathbf{C}(Y); \mathcal{A}) \to \operatorname{Sh}(\mathbf{C}(X); \mathcal{A}),$$

    \item Given a sheaf $\calF \in \operatorname{Sh}(\mathbf{C}(X); \mathcal{A})$, the \hldef{direct image} \hl{$f_* \calF$} along $f$ is defined to be the \CrefAndHyperrefIfExist{definition:direct_image_of_a_sheaf_on_a_site_under_a_continuous_functor_of_sites_or_a_site_morphism}{direct image} of $\calF$ under the \CrefAndHyperrefIfExist{definition:continuous_functor_of_sites}{continuous functor} \CrefAndHyperrefIfExist{definition:continuous_functors_on_sites_on_schemes_induced_by_scheme_morphism}{$f^{-1} : \mathbf{C}(Y) \to \mathbf{C}(X)$} on sites. In particular, it is an object of $\operatorname{Sh}(\mathbf{C}(Y); \mathcal{A})$ if it exists. If $f_* \calF$ exists for all sheaves $\calF \in \operatorname{Sh}(\mathbf{C}(X); \mathcal{A})$, then $f_*$ yields a functor
    $$f_* : \operatorname{Sh}(\mathbf{C}(X); \mathcal{A}) \to \operatorname{Sh}(\mathbf{C}(Y); \mathcal{A}).$$
\end{enumerate}
\end{definition}



\subsubsection{Sheaf of rings on a site and modules over a sheaf of rings on a site}


\begin{definition} \label{definition:module_over_a_sheaf_of_rings_on_a_site}

    \begin{enumerate}
        \item 
        Let $\mathcal{C}$ be a \CrefAndHyperrefIfExist{definition:grothendieck_topology_on_a_category_site_covering_sieve_topologically_generating_family}{site}, and let $\mathcal{A}$ and $\mathcal{B}$ be \CrefAndHyperrefIfExist{definition:sheaf_on_a_site}{sheaves} of (not necessarily commutative) \CrefAndHyperrefIfExist{definition:ring}{rings} on $\mathcal{C}$. 
        
        \begin{enumerate}
            \item 
            An \hldef{$(\mathcal{A}, \mathcal{B})$-bimodule} (or a \hldef{bimodule over $(\mathcal{A}, \mathcal{B})$}) is a \CrefAndHyperrefIfExist{definition:sheaf_on_a_site}{sheaf} $\mathcal{M}$ of abelian groups on $\mathcal{C}$ equipped with a left $\mathcal{A}$-module structure given by a \CrefAndHyperrefIfExist{definition:sheaf_on_a_site}{morphism of sheaves} of sets
            $$ \lambda: \mathcal{A} \times \mathcal{M} \longrightarrow \mathcal{M}, $$
            and a right $\mathcal{B}$-module structure given by a morphism of sheaves of sets
            $$ \rho: \mathcal{M} \times \mathcal{B} \longrightarrow \mathcal{M}, $$
            such that the actions are compatible. Specifically, for every object $U$ in $\mathcal{C}$, every section $m \in \mathcal{M}(U)$, every $a \in \mathcal{A}(U)$, and every $b \in \mathcal{B}(U)$, the equality
            $$ \lambda_U(a, \rho_U(m, b)) = \rho_U(\lambda_U(a, m), b) $$
            holds in $\mathcal{M}(U)$. In standard multiplicative notation where $\lambda(a,m)$ is denoted $a \cdot m$ and $\rho(m,b)$ is denoted $m \cdot b$, this condition is the associativity axiom
            $$ (a \cdot m) \cdot b = a \cdot (m \cdot b). $$

            In particular, for every object $U \in \calC$, the abelian group $\calM(U)$ has the structure of an \CrefAndHyperrefIfExist{definition:module_of_a_ring}{$\calA(U)-\calB(U)$-bimodule}.

            \item Let $\mathcal{M}$ and $\mathcal{N}$ be $(\mathcal{A}, \mathcal{B})$-bimodules. A \hldef{homomorphism of $(\mathcal{A}, \mathcal{B})$-bimodules} (or an \hldef{$(\mathcal{A}, \mathcal{B})$-linear morphism}) is a morphism of sheaves of abelian groups $f: \mathcal{M} \to \mathcal{N}$ such that for every object $U$ of $\mathcal{C}$, every section $m \in \mathcal{M}(U)$, every $a \in \mathcal{A}(U)$, and every $b \in \mathcal{B}(U)$, the following compatibility conditions hold:
            $$ f_U(a \cdot m) = a \cdot f_U(m) \quad \text{and} \quad f_U(m \cdot b) = f_U(m) \cdot b. $$


        \end{enumerate}

        \noindent We denote the category of $(\mathcal{A}, \mathcal{B})$-bimodules, with morphisms being morphisms of sheaves of abelian groups compatible with both the left $\mathcal{A}$-action and the right $\mathcal{B}$-action, by
        \hl{$ \mathcal{A}\text{-}\mathcal{B}\text{-}\mathsf{Mod} $}
        or sometimes by
        \hl{$ {}_{\mathcal{A}}\mathsf{Mod}_{\mathcal{B}} $}
        \TODO{talk about how bimodules can be identifies with left/right modules}

        \item 

        Let $(\mathcal{C}, J)$ be a \CrefAndHyperrefIfExist{definition:grothendieck_topology_on_a_category_site_covering_sieve_topologically_generating_family}{site}. Let $\mathcal{O}$ be a \CrefAndHyperrefIfExist{definition:sheaf_on_a_site}{sheaf of (not necessarily commutative) rings on $(\mathcal{C}, J)$}, i.e. $((\calC, J), \calO)$ is a \CrefAndHyperrefIfExist{definition:ringed_site}{ringed site}.  

        \begin{enumerate}
            \item An \hldef{(left/right/two-sided) $\mathcal{O}$-module} consists of the following data:
            \begin{itemize}
                \item A sheaf $\mathcal{F}$ of abelian groups on $(\mathcal{C}, J)$,
            \item for every object $U \in \mathcal{C}$, the structure of an (left/right/two-sided) $\mathcal{O}(U)$-module on $\mathcal{F}(U)$,
            \end{itemize}
            such that for every morphism $f: V \to U$ in $\mathcal{C}$, the restriction map 
            $$\rho_{U,V}: \mathcal{F}(U) \to \mathcal{F}(V)$$ 
            is $\mathcal{O}(U)$-linear when the $\mathcal{O}(U)$-action on $\mathcal{F}(V)$ is defined via the natural ring homomorphism 
            $$\mathcal{O}(U) \to \mathcal{O}(V)$$
            induced by $f$.


            \item Let $\mathcal{F}$ and $\mathcal{G}$ be \CrefAndHyperrefIfExist{definition:module_over_a_sheaf_of_rings_on_a_site}{$\mathcal{O}$-modules}.

            A \hldef{morphism of $\mathcal{O}$-modules} $\varphi: \mathcal{F} \to \mathcal{G}$ is a \CrefAndHyperrefIfExist{definition:sheaf_on_a_site}{morphism of sheaves} of abelian groups such that, for every object $U \in \mathcal{C}$, the component map
            $$\varphi_U : \mathcal{F}(U) \to \mathcal{G}(U)$$
            is $\mathcal{O}(U)$-linear, i.e. it satisfies
            $$\varphi_U(r \cdot s) = r \cdot \varphi_U(s) \quad \text{for all } r \in \mathcal{O}(U), \, s \in \mathcal{F}(U).$$

            The collection of all $\mathcal{O}$-modules together with their morphisms of $\mathcal{O}$-modules forms the \hldef{category of $\mathcal{O}$-modules}, denoted \hl{$\mathbf{Mod}(\mathcal{O})$}.

            \TextIfExists{definition:algebra_over_a_sheaf_of_rings_on_a_site}{See also \Cref{definition:algebra_over_a_sheaf_of_rings_on_a_site}.}
        \end{enumerate}

        \noindent In case that a \CrefAndHyperrefIfExist{definition:sheafification_functor_on_a_site}{sheafification functor} 
        $$\PreShv(\calC, \mathbf{Rings}) \to \Shv(\calC, \mathbf{Rings})$$ 
        exists, a left, right, two-sided $\calO$-module (and morphisms thereof) is equivalent to a $(\calO,\bbZ)$-bimodule, $(\bbZ,\calO)$-bimodule, and $(\calO, \calO)$-bimodule (and morphisms thereof) respectively, where $\bbZ$ is the \CrefAndHyperrefIfExist{definition:constant_sheaf_on_a_site_with_sheafification}{constant sheaf} of the integer ring $\bbZ$.

\end{enumerate}


\end{definition}


% See Also
% theorem:category_of_modules_over_a_sheaf_of_rings_on_a_site_on_an_essentially_small_category_has_enough_injectives

\begin{lemma} \label{lemma:category_of_modules_over_the_constant_sheaf_of_a_ring}
    Let $(\mathcal{C}, J)$ be a \CrefAndHyperrefIfExist{definition:grothendieck_topology_on_a_category_site_covering_sieve_topologically_generating_family}{site} on a locally small category $\calC$ or be a $U$-site for some \CrefAndHyperrefIfExist{definition:grothendieck_universe}{universe $U$}. Let $R$ be a \CrefAndHyperrefIfExist{definition:ring}{not-necessarily commutative unital ring}. Let $\mathcal{O}$ be the \CrefAndHyperrefIfExist{definition:constant_sheaf_on_a_site_with_sheafification}{constant sheaf} valued in $R$, which exists by \Cref{theorem:sheafification_of_a_presheaf_of_sets_on_a_small_enough_site}; it is also a \CrefAndHyperrefIfExist{definition:sheaf_on_a_site}{sheaf of rings on $(\mathcal{C}, J)$}. 
    
        The following categories are equivalent:
        \begin{itemize}
            \item $\mathbf{Mod}(\mathcal{O})$ of \CrefAndHyperrefIfExist{definition:module_over_a_sheaf_of_rings_on_a_site}{$\calO$-modules} is \CrefAndHyperrefIfExist{definition:equivalence_of_categories}{equivalent} 
            \item $\Shv(\calC, R\mathbf{-mod})$ of \CrefAndHyperrefIfExist{definition:sheaf_on_a_site}{sheaves} on $\calC$ valued in \CrefAndHyperrefIfExist{definition:examples_of_common_categories}{$R\mathbf{-mod}$}.
        \end{itemize}
    
        In particular, the categories of modules over the constant ring $\bbZ$ on $\calC$ is equivalent to the category $\Shv(\calC, \mathbf{Ab})$ of sheaves on $\calC$ valued in $\mathbf{Ab}$. 

\end{lemma}




\subsubsection{Sections}

\begin{definition} \label{definition:sections_of_a_presheaf_on_a_category_valued_in_a_category}
Let $\mathcal{C}$ be a \CrefAndHyperrefIfExist{definition:category}{(large) category}, and let $\mathcal{D}$ be a \CrefAndHyperrefIfExist{definition:category}{(large) category}. Let $\mathcal{F} : \mathcal{C}^{op} \to \mathcal{D}$ be a \CrefAndHyperrefIfExist{definition:presheaf_on_a_category}{presheaf valued in $\mathcal{D}$}.

\begin{enumerate}
    \item For an object $U \in \mathcal{C}$, the \hldef{sections functor evaluated at $U$} is the functor
    $$\hlin{\Gamma(U, -) : \mathrm{PSh}(\mathcal{C}, \mathcal{D}) \to \mathcal{D}}$$
    defined by
    $$\Gamma(U, \mathcal{F}) := \mathcal{F}(U),$$
    i.e., the value of the presheaf $\mathcal{F}$ at the object $U$.

    \item The \hldef{global sections of $\calF$} is the object \hl{$\Gamma(\calF)$} of $\calD$ defined as the \CrefAndHyperrefIfExist{definition:limit_and_colimit_of_a_diagram_in_a_category}{limit}
    $$\Gamma(\calF) = \varprojlim_{U \in \calC^{\op}} \calF(U)$$
    assuming that such a limit exists, where the limit is taken over objects $U \in \calC$ and the restriction morphisms $\calF(V) \to \calF(U)$ in $\calD$ for  morphisms $U \to V$ in $\calC$. 
    
    If a \CrefAndHyperrefIfExist{definition:initial_final_zero_objects_of_a_category}{final object} $\ast \in \mathcal{C}$ exists, then $\Gamma(\calF)$ exists and coincides with $\Gamma(\ast, \calF) = \calF(\ast)$. The construction $\Gamma(\calF)$ is functorial; in particular, if $\Gamma(\calF)$ exists for all $\calF$ in $\mathrm{PSh}(\calC, \calD)$, e.g. if \CrefAndHyperrefIfExist{definition:limit_and_colimit_of_a_diagram_in_a_category}{limits of} diagrams in $\calD$ indexed by $\calC$ exist, then $\Gamma$ is a functor 
    $$\hlin{\Gamma : \mathrm{PSh}(\mathcal{C}, \mathcal{D}) \to \mathcal{D}}$$
    called the \hldef{global sections functor on $\mathrm{PSh}(\mathcal{C}, \mathcal{D})$}.
\end{enumerate}
\end{definition}



\begin{proposition} \label{proposition:sections_functors_on_presheaves_vlaued_in_an_abelian_category_are_left_exact}
Let $\calC$ be an \CrefAndHyperrefIfExist{definition:essentially_small_category}{essentially small category} and let $\mathcal{A}$ be an \CrefAndHyperrefIfExist{definition:abelian_category}{abelian category}. 
\begin{enumerate}
    \item Let $U \in \Ob(\calC)$ be some fixed object. The \CrefAndHyperrefIfExist{definition:sections_of_a_presheaf_on_a_category_valued_in_a_category}{sections functor}
    $$\Gamma(U, -) : \mathrm{PSh}(\mathcal{C}, \mathcal{A}) \to \mathcal{A}$$ 
    is \CrefAndHyperrefIfExist{definition:exact_functor_between_abelian_categories}{left exact}. \TODO{state that presheaves and sheaves valued in an abelian category form abelian categories}

    \item Assume that $\Gamma(\calF)$ exists for all $\calF$ in $\mathrm{PSh}(\calC, \calA)$ so that $\Gamma$ is a functor 
    $$\Gamma: \mathrm{PSh}(\mathcal{C}, \calA) \to \calA.$$
    The functor $\Gamma$ is left exact.

\end{enumerate}
\end{proposition}
\begin{proof}
    Recall that $\mathrm{PSh}(\mathcal{C}, \mathcal{D})$ is an abelian category (\Cref{proposition:examples_of_abelian_categories}) since $\calC$ is essentially small.  \TODO{Talk about how limits are left exact and }
\end{proof}



\subsection{Sheaf cohomology}

\begin{lemma}[cf. {\cite[Porism 2.2.7]{weibel}}] \label{lemma:projective_injective_complex_with_map_to_from_object_with_left_right_resolution_lifts_uniquely_up_to_chain_homotopy}
    Let $\calA$ be an \CrefAndHyperrefIfExist{definition:abelian_category}{abelian category}.
    \begin{enumerate}
        \item Let
        $$\cdots \to P_2 \to P_1 \to P_0 \to M \to 0$$
        be a \CrefAndHyperrefIfExist{definition:chain_complex_of_objects_in_an_additive_category}{chain complex} with $P_i$ \CrefAndHyperrefIfExist{definition:injective_and_projective_objects_in_a_category}{projective}. For every \CrefAndHyperrefIfExist{definition:left_right_resolution_of_a_class_of_objects_in_an_abelian_category}{left resolution} $Q_\bullet \to N$ of an object $N$, every map $M \to N$ lifts to a \CrefAndHyperrefIfExist{definition:chain_complex_of_objects_in_an_additive_category}{complex map} $P_\bullet \to Q_\bullet$ unique up to \CrefAndHyperrefIfExist{definition:chain_homotopy_between_chain_maps_between_complexes}{chain homotopy}.

        \item Let
        $$0 \to M \to I^0 \to I^1 \to I^2 \to \cdots$$
        be a \CrefAndHyperrefIfExist{definition:chain_complex_of_objects_in_an_additive_category}{(co)chain complex} with $I^i$ \CrefAndHyperrefIfExist{definition:injective_and_projective_objects_in_a_category}{injective}. For every \CrefAndHyperrefIfExist{definition:left_right_resolution_of_a_class_of_objects_in_an_abelian_category}{right resolution} $N \to Q^\bullet$ of an object $N$, every map $N \to M$ lifts to a \CrefAndHyperrefIfExist{definition:chain_complex_of_objects_in_an_additive_category}{complex map} $Q^\bullet \to I^\bullet$ unique up to \CrefAndHyperrefIfExist{definition:chain_homotopy_between_chain_maps_between_complexes}{chain homotopy}.
    \end{enumerate}
\end{lemma}

\begin{proof}
    \begin{enumerate}
        \item The map $P_0 \to M \to N$ lifts to a map $P_0 \to Q_0$ because $P_0$ is projective and $Q_0 \to N$ is an epimorphism. 
        Inductively suppose that there are morphisms $P_i \to Q_i$ for $0 \leq i \leq n$, where $n \geq 0$ that make 
        \begin{center}
        \begin{tikzcd}
            P_n \ar[r] \ar[d] & P_{n-1} \ar[r] \ar[d] & \cdots \ar[r] & P_0 \ar[r] \ar[d] & M \ar[r] \ar[d] & 0 \\
            Q_n \ar[r] & Q_{n-1} \ar[r] & \cdots \ar[r] & Q_0 \ar[r] & N \ar[r] & 0 \\
        \end{tikzcd}
        \end{center}
        into a commuting diagram are established. The morphism $Q_{n} \to Q_{n-1}$ (where we let $Q_{-1} = N$ and $P_{-1} = M$ here in case that $n = 0$) acts as $0$ when restricted to \CrefAndHyperrefIfExist{definition:image_coimage_of_a_morphism_in_a_category}{$\mathfrak{I} \coloneq \operatorname{im} (P_{n+1} \to P_n \to Q_n)$} because the composition 
        $$P_{n+1} \to P_n \to Q_n \to Q_{n-1}$$
        equals the composition 
        $$P_{n+1} \to P_n \to P_{n-1} \to Q_{n-1}.$$
        In other words, $\mathfrak{I}$ is a \CrefAndHyperrefIfExist{definition:subobject_of_an_object_of_an_additive_category}{subobject} of \CrefAndHyperrefIfExist{definition:kernel_and_cokernel_of_a_morphism_in_a_category}{$\ker(Q_n \to Q_{n-1})$}, which is isomorphic to $\operatorname{im}(Q_{n+1} \to Q_n)$ by the acyclicity of the sequence of the $Q_i$'s. Therefore, we have a map $P_{n+1} \twoheadrightarrow \mathfrak{I}\hookrightarrow \operatorname{im}(Q_{n+1} \to Q_n)$ along with an epimorphism $Q_{n+1} \twoheadrightarrow \operatorname{im}(Q_{n+1} \to Q_n)$. Since $P_{n+1}$ is projective, the former map lifts to a map $P_{n+1} \to Q_{n+1}$ in a way that is compatible with the latter, i.e. the following commutes:
        \begin{center}
        \begin{tikzcd}
            P_{n+1} \ar[rd] \ar[d,dotted] & \\
            Q_{n+1} \ar[r] & \operatorname{im}(Q_{n+1} \to Q_n).
        \end{tikzcd}
        \end{center}
        By induction, this shows thta $M \to N$ lifts to a morphism $P_\bullet \to Q_\bullet$ of complexes.

        We show that the morphism of complexes is unique up to chain homotopy, i.e. if $f_1, f_2: P_\bullet \to Q_\bullet$ are two morphisms of complexes, then $h \coloneq f_1 - f_2$ is null homotopic. We construct a \CrefAndHyperrefIfExist{definition:chain_homotopy_between_chain_maps_between_complexes}{chain contraction} $\{s_n: P_n \to Q_{n+1}\}$ of $h$ by induction on $n$. If $n < 0$, then set $s_n = 0$. If $n = 0$, note that the composition $P_0 \xrightarrow{h_0} Q_0 \to N$ equals the composition $P_0 \to M \xrightarrow{0} N$, so $\operatorname{im}(h_0)$ is a subobject of $\ker(Q_0 \to N) \cong \operatorname{im}(Q_1 \to Q_0)$. The projectivitiy of $P_0$ thus yields a lift $s_0: P_0 \to Q_1$ such that $h_0$ equals the composition $P_0 \xrightarrow{s_0} X_1 \xrightarrow{d} Q_0$:
        \begin{center}
        \begin{tikzcd}
           &  P_0 \ar[dl, "s_0", dotted] \ar[d, "h_0"] \\
           Q_1 \ar[r, "d"] & Q_0 
        \end{tikzcd}
        \end{center}
        Note moreover that $h_0 = ds_0 + s_{-1} d$ because $s_{-1} = 0$. Inductively suppose that we have maps $s_i$ for $i \leq n$  such that $h_n = d s_{n} + s_{n-1} d$ or equivalently that $ds_{n} = h_n - s_{n-1} d$. Consider the map $h_{n+1} - s_{n} d: P_{n+1} \to Q_{n+1}$. Compute
        $$d(h_{n+1} - s_{n} d) = dh_{n+1} - d s_{n} d = dh_{n+1} - (h_n - s_{n-1}d)d = (dh_{n+1} - h_n d) + s_{n-1} d d = 0$$
        Therefore, $\operatorname{im} (h_{n+1} - s_n d)$ is a subobject of $\ker(Q_{n+1} \to Q_{n}) \cong \operatorname{im}(Q_{n+2} \to Q_{n+1})$, which is in turn a quotient of $Q_{n+2}$. Since $P_{n+1}$ is projective, there is a morphism $s_{n+1}: P_{n+1} \to Q_{n+2}$ such that $d s_{n+1} = h_{n+1} - s_{n} d$. 
        \begin{center}
        \begin{tikzcd}
           &  P_{n+1} \ar[dl, dotted, "s_{n+1}"] \ar[d, "h_n - s_{n-1} d = ds_{n}"] \\
           Q_{n+2} \ar[r, "d"]  &\operatorname{im}(Q_{n+2} \to Q_{n+1}) \cong \ker(Q_{n+1} \to Q_{n})
        \end{tikzcd}
        \end{center}
        The $s_n$ thus form a chain contraction as needed.

        \item This is simply dual to the previous part.
    \end{enumerate}
\end{proof}
\begin{proposition}[cf.{\cite[Lemma 2.4.1]{weibel}}] \label{proposition:left_right_derived_objects_for_a_right_left_exact_functor_between_abelian_categories_are_well_defined}
    Let $F: \calA \to \calB$ be an \CrefAndHyperrefIfExist{definition:additive_functor_between_additive_categories}{additive functor} between \CrefAndHyperrefIfExist{definition:abelian_category}{abelian categories}. Let $A$ be an object of $\calA$. 
    \begin{enumerate}
        \item Suppose that $F$ is \CrefAndHyperrefIfExist{definition:exact_functor_between_abelian_categories}{right exact}, and suppose that a \CrefAndHyperrefIfExist{definition:left_right_resolution_of_a_class_of_objects_in_an_abelian_category}{projective resolution}
        $$\cdots \to P_2 \to P_1 \to P_0 \to A \to 0$$
        of $A$ exists in $\calA$. Let 
        $$\cdots \to Q_2 \to Q_1 \to Q_0 \to A \to 0$$
        be any projective resolution of $A$ in $\calA$. For all $n$, there are natural isomorphisms
        $$H_n(F(P_\bullet)) \cong H_n(F(Q_\bullet)).$$
        In other words, the \CrefAndHyperrefIfExist{definition:left_right_derived_functors_of_a_right_left_exact_functor_between_abelian_categories_where_source_has_enough_projectives_injectives}{left derived objects $L_n F(A)$} is well defined.

        \item Suppose that $F$ is \CrefAndHyperrefIfExist{definition:exact_functor_between_abelian_categories}{left exact}, and suppose that a \CrefAndHyperrefIfExist{definition:left_right_resolution_of_a_class_of_objects_in_an_abelian_category}{injective resolution}
        $$0 \to A \to I^0 \to I^1 \to I^2 \to \cdots$$
        of $A$ exists in $\calA$. Let 
        $$0 \to A \to Q^0 \to Q^1 \to Q^2 \to \cdots$$
        be any injective resolution of $A$ in $\calA$. For all $n$, there are natural isomorphisms
        $$H_n(F(I^\bullet)) \cong H_n(F(Q^\bullet)).$$
        In other words, the \CrefAndHyperrefIfExist{definition:left_right_derived_functors_of_a_right_left_exact_functor_between_abelian_categories_where_source_has_enough_projectives_injectives}{right derived objects $R_n F(A)$} is well defined.
    \end{enumerate}
\end{proposition}

\begin{proof}
    \begin{enumerate}
        \item By \Cref{lemma:projective_injective_complex_with_map_to_from_object_with_left_right_resolution_lifts_uniquely_up_to_chain_homotopy}, there is a lift $f: P_\bullet \to Q_\bullet$ of the identity map $A \to A$ unique up to chain homotopy. There are then induced natural maps $H_n(F(f)): H_n(F(P_\bullet)) \to H_n(F(Q_\bullet))$. There is also a lift $f': Q_\bullet \to P_\bullet$ of the identity map $A \to A$ unique up to chain homotopy, and this also induces natural maps $H_n(F(f')): H_n(F(Q_\bullet)) \to H_n(F(P_\bullet))$. The chain maps $f$ and $f'$ are in fact \CrefAndHyperrefIfExist{definition:chain_homotopy_between_chain_maps_between_complexes}{chain homotopy inverses} because \Cref{lemma:projective_injective_complex_with_map_to_from_object_with_left_right_resolution_lifts_uniquely_up_to_chain_homotopy} also implies that any lifts $P_\bullet \to P_\bullet$ and $Q_\bullet \to Q_\bullet$ of the identity map $A \to A$ are chain homotopic to the identity chain maps. Therefore, $H_n(F(f))$ and $H_n(F(f'))$ are inverses of each other as morphisms in $\calB$.  \TODO{prove basic facts about the fucntoriality of homology/cohomology of chain complexes}

        \item This is dual to the previous part.
    \end{enumerate}
\end{proof}
\begin{definition} \label{definition:left_right_resolution_of_a_class_of_objects_in_an_abelian_category}
Let $\mathcal{A}$ be an \CrefAndHyperrefIfExist{definition:abelian_category}{abelian category} and let $\mathcal{X}$ be a class of objects in $\mathcal{A}$. Let $M$ be an object of $\calA$.

\begin{enumerate}
    \item A \hldef{right resolution of $M$} is a \CrefAndHyperrefIfExist{definition:chain_complex_of_objects_in_an_additive_category}{cochain complex} $I^\bullet$ with $I^i = 0$ for $i < 0$ and a map $M \to I^0$ such that the augmented complex
    $$0 \to M \to I^0 \to I^1 \to I^2 \to \cdots$$
    is \CrefAndHyperrefIfExist{definition:acyclic_complex_of_objects_in_an_abelian_category}{exact}.

    % \item An \hldef{injective resolution of $M$} is a right resolution $I^\bullet$ for which the objects $I^i$ are all \CrefAndHyperrefIfExist{definition:injective_and_projective_objects_in_a_category}{injective}.

    \item A \hldef{left resolution of $M$} is a \CrefAndHyperrefIfExist{definition:chain_complex_of_objects_in_an_additive_category}{chain complex} $P_\bullet$ with $P_i = 0$ for $i < 0$ and a map $P_0 \to M$ such that the augmented complex
    $$\cdots P_2 \to P_1 \to P_0 \to M \to 0$$
    is \CrefAndHyperrefIfExist{definition:acyclic_complex_of_objects_in_an_abelian_category}{exact}.
    
    \item An \hldef{$\mathcal{X}$-left resolution} of an object $M \in \mathcal{A}$ a \CrefAndHyperrefIfExist{definition:left_right_resolution_of_a_class_of_objects_in_an_abelian_category}{left resolution} by objects of $X$, i.e. an exact complex
    $$ \cdots \to X_2 \to X_1 \to X_0 \to M \to 0 $$
    with each $X_i \in \mathcal{X}$.

    \item An \hldef{$\mathcal{X}$-right resolution} of an object $M \in \mathcal{A}$ a \CrefAndHyperrefIfExist{definition:left_right_resolution_of_a_class_of_objects_in_an_abelian_category}{right resolution} by objects of $X$, i.e. an exact complex
    $$ 0 \to M \to X^0 \to X^1 \to X^2 \to \cdots $$
    with each $X_i \in \mathcal{X}$.

    \item A \hldef{projective resolution of $M$} is a left resolution $P^\bullet$ for which the objects $P^i$ are all \CrefAndHyperrefIfExist{definition:injective_and_projective_objects_in_a_category}{projective}.

    \item An \hldef{injective resolution of $M$} is a right resolution $I^\bullet$ for which the objects $I^i$ are all \CrefAndHyperrefIfExist{definition:injective_and_projective_objects_in_a_category}{injective}.
\end{enumerate}

\end{definition}
\begin{definition} \label{definition:has_enough_injectives_or_projectives_for_an_abelian_category}
Let $\mathcal{A}$ be an \CrefAndHyperrefIfExist{definition:abelian_category}{abelian category}.
\begin{enumerate}
    \item $\mathcal{A}$ is said to \hldef{have enough injectives} if for every object $A$ in $\calA$, there is an \CrefAndHyperrefIfExist{definition:monomorphism_and_epimorphism_in_categories}{monomorphism} $A \to I$ with $I$ an \CrefAndHyperrefIfExist{definition:injective_and_projective_objects_in_a_category}{injective object} of $\calA$. \TextIfExistsElse{definition:has_enough_objects_of_a_class_on_the_left_right_for_an_abelian_category}{Equivalently, $\calA$ has enough injectives if it has enough objects of the class of injectives on the right (\Cref{definition:has_enough_objects_of_a_class_on_the_left_right_for_an_abelian_category})}

    \item $\mathcal{A}$ is said to \hldef{have enough projectives} if for every object $A$ in $\calA$, there is a \CrefAndHyperrefIfExist{definition:monomorphism_and_epimorphism_in_categories}{epimorphism} $P \to A$ with $P$ a \CrefAndHyperrefIfExist{definition:injective_and_projective_objects_in_a_category}{projective object} of $\calA$. \TextIfExistsElse{definition:has_enough_objects_of_a_class_on_the_left_right_for_an_abelian_category}{Equivalently, $\calA$ has enough projectives if it has enough objects of the class of projectives on the left (\Cref{definition:has_enough_objects_of_a_class_on_the_left_right_for_an_abelian_category})}

\end{enumerate}
\end{definition}

% See Also
% definition:left_right_derived_functors_of_a_right_left_exact_functor_between_abelian_categories_where_source_has_enough_projectives_injectives 



\begin{theorem}[e.g. see {\cite[Tag 01DU]{stacks-project}}] \label{theorem:category_of_modules_over_a_sheaf_of_rings_on_a_site_on_an_essentially_small_category_has_enough_injectives}
    For any \CrefAndHyperrefIfExist{definition:grothendieck_topology_on_a_category_site_covering_sieve_topologically_generating_family}{site} $(\calC, J)$ on an \CrefAndHyperrefIfExist{definition:essentially_small_category}{essentially small category} $\mathcal{C}$ and a \CrefAndHyperrefIfExist{definition:sheaf_on_a_site}{sheaf of rings} $\calO$ on $\calC$, the category $\mathbf{Mod}(\mathcal{O})$ of \CrefAndHyperrefIfExist{definition:module_over_a_sheaf_of_rings_on_a_site}{$\calO$-modules} is an abelian category that \CrefAndHyperrefIfExist{definition:has_enough_injectives_or_projectives_for_an_abelian_category}{has enough injectives}. In fact, there is a functorial injective embedding \TODO{ what does this mean?}
\end{theorem}

\TODO{Apparently, if an abelian category satisfies AB5 and admits a generator (whatever that means), then it has enough injectives. Find a reference for this. Moreover, apparently, the category of sheaves of abelian groups on a general $U$-site satisfies these criteria and hence has enough injectives.}

\TODO{According to Weibel exercise 2.3.7, if $\calA$ is complete and has enough injectives and $I$ is (essentially) small, then the functor category $\calA^I$ has enough injectives}
\begin{theorem} \label{theorem:examples_of_abelian_categories_with_enough_injectives_or_projectives}
\begin{enumerate}
    \item Examples of abelian categories with enough injectives include:
    \begin{itemize}
        \item The category of abelian groups.
        \item The category of modules over a ring.
        \item The category of sheaves of abelian groups on a ringed space or on an essentially small site.
    \end{itemize}

    \item Examples of abelian categories with enough projectives include:
    \begin{itemize}
        \item The category of modules over a ring with enough projectives (e.g., rings with unity and suitable properties). \TODO{make this more precise}
        \item The category of finitely generated modules over a semisimple ring.
    \end{itemize}
\end{enumerate}
% These conditions ensure the existence of derived functors such as Ext and Tor.
\end{theorem}

\begin{lemma} [cf. {\cite[Lemma 2.2.5, Lemma 2.3.6]{weibel}}] \label{lemma:an_object_of_abelian_category_with_enough_objects_of_a_class_on_the_right_left_has_right_left_resolution_by_the_class}
Let $\mathcal{A}$ be an \CrefAndHyperrefIfExist{definition:abelian_category}{abelian category} and let $\calX$ be a class of objects in $\calA$.

\begin{enumerate}
    \item If $\mathcal{A}$ \CrefAndHyperrefIfExist{definition:has_enough_objects_of_a_class_on_the_left_right_for_an_abelian_category}{has enough objects of class $\calX$ on the right}, then for every object $A \in \mathcal{A}$ there exists an \CrefAndHyperrefIfExist{definition:left_right_resolution_of_a_class_of_objects_in_an_abelian_category}{$\calX$-right resolution of $A$}.

    \item If $\mathcal{A}$ \CrefAndHyperrefIfExist{definition:has_enough_objects_of_a_class_on_the_left_right_for_an_abelian_category}{has enough objects of class $\calX$ on the left}, then for every object $A \in \mathcal{A}$ there exists an \CrefAndHyperrefIfExist{definition:left_right_resolution_of_a_class_of_objects_in_an_abelian_category}{$\calX$-left resolution of $A$}.

\end{enumerate}

Note that this is a special case of \Cref{proposition:abelian_category_with_enough_objets_of_a_class_on_the_right_left_has_resolutions_of_complexes} obtained by letting the complex $M^\bullet$ be the complex such that
$$M^i = \begin{cases} A &\text{if } i = 0 \\ 0 &\text{otherwise} \end{cases}.$$


In particular,
\begin{itemize}
    \item If $\mathcal{A}$ \CrefAndHyperrefIfExist{definition:has_enough_injectives_or_projectives_for_an_abelian_category}{has enough injective objects}, then for every object $A \in \mathcal{A}$ there exists an \CrefAndHyperrefIfExist{definition:left_right_resolution_of_a_class_of_objects_in_an_abelian_category}{injective resolution of $A$}.
    \item If $\mathcal{A}$ \CrefAndHyperrefIfExist{definition:has_enough_injectives_or_projectives_for_an_abelian_category}{has enough projective objects}, then for every object $A \in \mathcal{A}$ there exists a \CrefAndHyperrefIfExist{definition:left_right_resolution_of_a_class_of_objects_in_an_abelian_category}{projective resolution of $A$}.
    \item If $F: \calA \to \calB$ is a \CrefAndHyperrefIfExist{definition:exact_functor_between_abelian_categories}{left (resp. right) exact functor} between abelian categories and $\calA$ has enough $F$-acyclic objects on the right (resp. left), then for every object $A \in \calA$, there exists an \CrefAndHyperrefIfExist{definition:F_acyclic_resolution_for_a_right_left_exact_functor_between_abelian_categories}{right (resp. left) $F$-acyclic resolution} of $A$.
\end{itemize}

\end{lemma}
\begin{proof}
    \begin{enumerate}
        \item Let $A \in \calA$ be an object. Since $\calA$ has enough objects of class $\calX$ of the right, there is an object $X_0$ of $\calX$ and a monomorphism $\varepsilon_0: A \to X_0$. Let \CrefAndHyperrefIfExist{definition:kernel_and_cokernel_of_a_morphism_in_a_category}{$A_0 = \operatorname{coker} \varepsilon_0$}. Inductively, given an object $A_{n-1}$ of $\calA$, choose an object $X_n$ of $\calX$ and a monomorphism $\varepsilon_{n}: A_{n-1} \hookrightarrow X_n$. Let $A_n = \operatorname{coker} \varepsilon_n$. In particular, there is a surjection $X_n \twoheadrightarrow A_n$. Let $d_n$ be the composition
        $$X_{n-1} \twoheadrightarrow A_{n-1} \xrightarrow{\varepsilon_n} X_n.$$
        The chain complex
        $$0 \to A \xrightarrow{\varepsilon_0} X_0 \xrightarrow{d_0} X_1 \xrightarrow{d_1} \cdots$$
        is thus an $\calX$-right resolution of $A$.

        \item This is simply the dual statement of the next statement.

    \end{enumerate} 
\end{proof}

\begin{definition} \label{definition:left_right_derived_functors_of_a_right_left_exact_functor_between_abelian_categories_where_source_has_enough_projectives_injectives}
    \TODO{I think that the definition of derived categories might be doable for more general kinds of resolutions? Perhaps it is that if I have a right exact functor $F$, then $L^i F$ can be computed with resolutions of $F$-acyclic objects? \CrefIfExists{definition:F_acyclic_object_for_a_left_or_right_functor_between_abelian_categories}}
    \TODO{Apparently, left/right derived functors may be defined for functors that are additive and preserve finite coproducts, and not necessarily right/left exact; the exactness condition ensures that the zeroth derived functor agrees with $F$.}
Let $\mathcal{A}$ and $\mathcal{B}$ be \CrefAndHyperrefIfExist{definition:abelian_category}{abelian categories}, and let 
$$F: \mathcal{A} \to \mathcal{B}$$ 
be an \CrefAndHyperrefIfExist{definition:additive_functor_between_additive_categories}{additive functor}.

\begin{enumerate}
    \item Suppose that the functor $F$ is \CrefAndHyperrefIfExist{definition:exact_functor_between_abelian_categories}{right exact} and suppose that $A \in \calA$ is an object for which a \CrefAndHyperrefIfExist{definition:left_right_resolution_of_a_class_of_objects_in_an_abelian_category}{projective resolution}
    $$\cdots \to P_2 \to P_1 \to P_0 \to A \to 0$$
    exists in $\calA$. We define the \hldef{left derived object} \hl{$L_n F A \in \calB$} by applying $F$ to obtain a complex
    $$\cdots \to F(P_2) \to F(P_1) \to F(P_0) \to 0$$
    and letting $L_n F(A)$ be the \CrefAndHyperrefIfExist{definition:homology_and_cohomology_objects_for_a_chain_complex_in_an_additive_category}{$n$-th homology object} of this complex in $\mathcal{B}$:
    $$L_n F(A) := H_n(F(P_\bullet)).$$
    The object $L_n F(A)$ is independent of the choice of projective resolution up to natural isomorphism (\Cref{proposition:left_right_derived_objects_for_a_right_left_exact_functor_between_abelian_categories_are_well_defined}). 

    By convention, set $L_n F = 0$ for $n < 0$.

    The \hldef{higher left derived objects} refer to the object $L_n F(A)$ for $n > 0$. 

    \item  Suppose that the functor $F$ is \CrefAndHyperrefIfExist{definition:exact_functor_between_abelian_categories}{right exact} and that $\calA$ \CrefAndHyperrefIfExist{definition:has_enough_injectives_or_projectives_for_an_abelian_category}{has enough projectives}. The \hldef{left derived functors} refer to the family of functors
    $$\hlin{L_n F : \mathcal{A} \to \mathcal{B}, \quad A \mapsto L_n F(A).}$$
    The \hldef{higher left derived functors} refer to the functors $L_n F$ for $n > 0$. 

    \item Suppose that the functor $F$ is \CrefAndHyperrefIfExist{definition:exact_functor_between_abelian_categories}{right exact} and suppose that $A \in \calA$ is an object for which a \CrefAndHyperrefIfExist{definition:left_right_resolution_of_a_class_of_objects_in_an_abelian_category}{injective resolution}
    $$0 \to A \to I^0 \to I^1 \to I^2 \to \cdots$$
    exists in $\calA$. We define the \hldef{right derived object} \hl{$R_n F A \in \calB$}, also often denoted by \hl{$R^n FA$}, by applying $F$ to obtain a complex
    $$0 \to F(I^0) \to F(I^1) \to F(I^2) \to \cdots.$$
    and letting $R_n F(A)$ be the \CrefAndHyperrefIfExist{definition:homology_and_cohomology_objects_for_a_chain_complex_in_an_additive_category}{$n$-th cohomology object} of this complex in $\mathcal{B}$:
    $$R_n F(A) := H^n(F(I_\bullet)).$$
    The object $R_n F(A)$ is independent of the choice of injective resolution up to natural isomorphism (\Cref{proposition:left_right_derived_objects_for_a_right_left_exact_functor_between_abelian_categories_are_well_defined}). 

    By convention, set $R_n F = 0$ for $n < 0$.

    The \hldef{higher right derived objects} refer to the object $R_n F(A)$ for $n > 0$. 

    \item  Suppose that the functor $F$ is \CrefAndHyperrefIfExist{definition:exact_functor_between_abelian_categories}{right exact} and that $\calA$ \CrefAndHyperrefIfExist{definition:has_enough_injectives_or_projectives_for_an_abelian_category}{has enough injectives}. The \hldef{right derived functors} refer to the family of functors
    $$\hlin{R_n F : \mathcal{A} \to \mathcal{B}, \quad A \mapsto R_n F(A).}$$
    The right derived functors are also often denoted by \hl{$R^n F$}.
    The \hldef{higher right derived functors} refer to the functors $R_n F$ for $n > 0$. 

    
    % If the functor $F$ is right exact and $\calA$ \CrefAndHyperrefIfExist{definition:has_enough_injectives_or_projectives_for_an_abelian_category}{has enough projectives}, then its \hldef{left derived functors} are a family of functors
    % $$\hlin{L_n F : \mathcal{A} \to \mathcal{B}, \quad n \geq 0,}$$
    % which are defined for each object $A$ in $\mathcal{A}$ by choosing (\Cref{lemma:an_object_of_abelian_category_with_enough_objects_of_a_class_on_the_right_left_has_right_left_resolution_by_the_class}) a \CrefAndHyperrefIfExist{definition:left_right_resolution_of_a_class_of_objects_in_an_abelian_category}{projective resolution}
    % $$\cdots \to P_2 \to P_1 \to P_0 \to A \to 0$$
    % in $\mathcal{A}$ and applying $F$ to obtain a complex
    % $$\cdots \to F(P_2) \to F(P_1) \to F(P_0) \to 0.$$
    % Then $L_n F(A)$ is defined to be the \CrefAndHyperrefIfExist{definition:homology_and_cohomology_objects_for_a_chain_complex_in_an_additive_category}{$n$-th homology object} of this complex in $\mathcal{B}$:
    % $$L_n F(A) := H_n(F(P_\bullet)).$$
    % The functors $L_n F$ are independent of the choice of projective resolution up to natural isomorphism. 

    % By convention, set $L_n F = 0$ for $n < 0$.

    % \item If the functor $F$ is \CrefAndHyperrefIfExist{definition:exact_functor_between_abelian_categories}{left exact} and $\calA$ \CrefAndHyperrefIfExist{definition:has_enough_injectives_or_projectives_for_an_abelian_category}{has enough injectives}, then its \hldef{right derived functors} are a family of functors
    % $$\hlin{R^n F : \mathcal{A} \to \mathcal{B}, \quad n \geq 0,}$$
    % which are defined for each object $A$ in $\mathcal{A}$ by choosing (\Cref{lemma:an_object_of_abelian_category_with_enough_objects_of_a_class_on_the_right_left_has_right_left_resolution_by_the_class}) an \CrefAndHyperrefIfExist{definition:left_right_resolution_of_a_class_of_objects_in_an_abelian_category}{injective resolution}
    % $$0 \to A \to I^0 \to I^1 \to I^2 \to \cdots$$
    % in $\mathcal{A}$ and applying $F$ to obtain a complex
    % $$0 \to F(I^0) \to F(I^1) \to F(I^2) \to \cdots.$$
    % Then $R^n F(A)$ is defined to be the \CrefAndHyperrefIfExist{definition:homology_and_cohomology_objects_for_a_chain_complex_in_an_additive_category}{$n$-th cohomology object} of this complex in $\mathcal{B}$:
    % $$R^n F(A) := H^n(F(I^\bullet)).$$
    % The functors $R^n F$ are independent of the choice of injective resolution up to natural isomorphism.

    % By convention, set $R^n F = 0$ for $n < 0$.
\end{enumerate}
\end{definition}

\begin{definition} \label{definition:ringed_site}
    \TODO{there are places where sites and sheaves of rings on them are used, but it would be better to just have them be ringed sites.}

    A \hldef{ringed site} is a \CrefAndHyperrefIfExist{definition:grothendieck_topology_on_a_category_site_covering_sieve_topologically_generating_family}{site} $(\mathcal{C}, J)$ with a small \CrefAndHyperrefIfExist{definition:grothendieck_topology_on_a_category_site_covering_sieve_topologically_generating_family}{topological generating family} equipped with a \CrefAndHyperrefIfExist{definition:sheaf_on_a_site}{sheaf} of (not necessarily commutative) rings $\mathcal{O}$. If the Grothendieck topology $J$ is clear in context, one may even write that $(C, \calO)$ is a ringed site.

    A \hldef{morphism of ringed sites}
    $$ ((\mathcal{C},J),\mathcal{O}) \to ((\mathcal{C}',J'),\mathcal{O}') $$
    consists of a \CrefAndHyperrefIfExist{definition:morphism_of_sites}{morphism of sites} $f : (\mathcal{C},J) \to (\mathcal{C}',J')$ and a \CrefAndHyperrefIfExist{definition:sheaf_on_a_site}{morphism of sheaves} of rings $f^\# : \mathcal{O}' \to f_*\mathcal{O}$ \CrefIfExists{definition:inverse_image_of_a_sheaf_under_a_continuous_functor_of_sites_or_a_site_morphism}.
\end{definition}
\begin{definition} \label{definition:sheaf_cohomology_group_of_a_sheaf_of_modules_over_a_sheaf_of_rings_on_a_site}

Let $(\mathcal{C}, J)$ be a \CrefAndHyperrefIfExist{definition:grothendieck_topology_on_a_category_site_covering_sieve_topologically_generating_family}{site} on a \CrefAndHyperrefIfExist{definition:locally_small_category}{locally small category} or a $U$-site for some \CrefAndHyperrefIfExist{definition:grothendieck_universe}{universe} $U$. Let $\calO$ be a \CrefAndHyperrefIfExist{definition:sheaf_on_a_site}{sheaf of rings} on $\calC$, so that $(\calC, J, \calO)$ is a \CrefAndHyperrefIfExist{definition:ringed_site}{ringed site}. Recall that the category \CrefAndHyperrefIfExist{definition:module_over_a_sheaf_of_rings_on_a_site}{$\mathbf{Mod}(\mathcal{O})$} of $\calO$-modules is abelian and \CrefAndHyperrefIfExist{definition:has_enough_injectives_or_projectives_for_an_abelian_category}{has enough injectives} (\Cref{theorem:category_of_modules_over_a_sheaf_of_rings_on_a_site_on_an_essentially_small_category_has_enough_injectives}).

Assume that \CrefAndHyperrefIfExist{definition:sections_of_a_presheaf_on_a_category_valued_in_a_category}{global sections objects $\Gamma(\calG)$} exist for all objects $\calG$ of $\mathrm{Sh}(\mathcal{C}, \mathbf{Ab})$\footnote{for example, this occurs when $\calC$ is \CrefAndHyperrefIfExist{definition:essentially_small_category}{essentially small}} so that $\Gamma$ is a functor
$$\Sh(\calC, \mathbf{Ab}) \to \mathbf{Ab},$$
which is a \CrefAndHyperrefIfExist{definition:exact_functor_between_abelian_categories}{left exact functor} (\Cref{proposition:sections_functors_on_presheaves_vlaued_in_an_abelian_category_are_left_exact}).  Note that $\Gamma$ restricts to a left exact functor 
$$\mathbf{Mod}(\mathcal{O}) \to \mathbf{Ab}.$$
If $\calC$ has a \CrefAndHyperrefIfExist{definition:initial_final_zero_objects_of_a_category}{final object} $\ast$ as well, then recall that $\Gamma(\calF) = \calF(\ast)$. 

Let $\calF$ be an object of $\mathbf{Mod}(\mathcal{O})$. 
\begin{enumerate}
    \item For each integer $n \geq 0$, the \hldef{$n$-th (abelian) (global) sheaf cohomology group of $\mathcal{F}$} is
    $$\hlin{H^n(\mathcal{C}, J; \mathcal{F}) := R^n \Gamma(\mathcal{F}),}$$
    where $R^n \Gamma$ is the $n$-th \CrefAndHyperrefIfExist{definition:left_right_derived_functors_of_a_right_left_exact_functor_between_abelian_categories_where_source_has_enough_projectives_injectives}{right derived functor} of the \CrefAndHyperrefIfExist{definition:sections_of_a_presheaf_on_a_category_valued_in_a_category}{global sections functor $\Gamma$}.

    In particular, each $H^n$ is a functor 
    $$H^n: \mathbf{Mod}(\mathcal{O})  \to \mathbf{Ab}.$$

    \item Given an object $U \in \calC$ and for each integer $n \geq 0$, the \hldef{$n$-th (abelian) sheaf cohomology group of $\mathcal{F}$ of sections at $U$} is
    $$\hlin{H^n(U, \mathcal{F}) := (R^n \Gamma(U,-))(\mathcal{F}),}$$
    where $R^n \Gamma(U,-)$ is the $n$-th \CrefAndHyperrefIfExist{definition:left_right_derived_functors_of_a_right_left_exact_functor_between_abelian_categories_where_source_has_enough_projectives_injectives}{right derived functor} of the \CrefAndHyperrefIfExist{definition:sections_of_a_presheaf_on_a_category_valued_in_a_category}{sections functor $\Gamma(U,-)$ evaluated at $U$}.

    In particular, $H^n(U,\calF)$ can be regarded as the $n$th global sheaf cohomology group of the \CrefAndHyperrefIfExist{definition:restriction_of_a_sheaf_on_a_site_to_an_object_of_the_underlying_category_of_the_site}{restriction $\calF|_U$} of $\calF$ to $U$.
    % to the 
    % \CrefAndHyperrefIfExist{definition:site_induced_by_a_site_on_an_over_category}{site induced by $(\calC, J)$} on the \CrefAndHyperrefIfExist{definition:category_of_objects_over_under_a_fixed_object_in_a_category}{over category $\calC_{/U}$}.

\end{enumerate}

% In the case that $\calA = R\mathbf{-mod}$, the category of (left/right/two-sided)modules over some fixed (not necessarily commutative) ring $R$, recall that $R\mathbf{-Mod}$ is complete (and cocomplete) \TODO{Is it precisely completeness that I need or cocompleteness? In other words, for the limits defining $\Gamma$, are those limits projective limits or colimtis?} \TODO{talk about how $R$-modules are complete and cocomplete}, so all global sections objects $\Gamma(\calG)$ exist. \TODO{continue talking about this context of modules}

% The \hldef{sheaf cohomology groups of $\mathcal{F}$} on the site $(\mathcal{C}, J)$ are defined as the \CrefAndHyperrefIfExist{definition:left_right_derived_functors_of_a_right_left_exact_functor_between_abelian_categories_where_source_has_enough_projectives_injectives}{right derived functors} of the global sections functor
% $$\Gamma : \mathrm{Sh}(\mathcal{C}, J) \to \mathbf{Ab}$$
% where $\mathrm{Sh}(\mathcal{C}, J)$ \CrefAndHyperrefIfExist{definition:sheaf_on_a_site}{denotes} the category of sheaves of abelian groups on $(\mathcal{C}, J)$, and $\ast$ denotes the \CrefAndHyperrefIfExist{definition:initial_final_zero_objects_of_a_category}{final object} in $\mathcal{C}$ if it exists.

% More precisely, 

% If $\mathcal{C}$ has no final object, $H^n(\mathcal{C}, J; \mathcal{F})$ is defined by choosing an injective resolution of $\mathcal{F}$ and taking cohomology of the resulting complex obtained by applying $\Gamma$.

% These groups measure the extent to which global sections fail to be exact, and generalize classical sheaf cohomology defined on topological spaces to arbitrary sites.
\end{definition}

% \begin{proposition}[See {\cite[Tag 03F2]{stacks-project}} for a statement]
%     Let $(\mathcal{C}, J)$ be a \CrefAndHyperrefIfExist{definition:grothendieck_topology_on_a_category_site_covering_sieve_topologically_generating_family}{site} on a \CrefAndHyperrefIfExist{definition:locally_small_category}{locally small category} or a $U$-site for some \CrefAndHyperrefIfExist{definition:grothendieck_universe}{universe} $U$. Let $\calO$ be a \CrefAndHyperrefIfExist{definition:sheaf_on_a_site}{sheaf of rings} on $\calC$, so that $(\calC, J, \calO)$ is a \CrefAndHyperrefIfExist{definition:ringed_site}{ringed site}. There is a functorial isomorphism
%     $$
% \end{proposition}

\subsubsection{\v{C}ech cohomology of a sheaf of groups}

\begin{definition}[\v{C}ech Cohomology of a Sheaf of Groups]
Let $(\mathcal{C},J)$ be a \CrefAndHyperrefIfExist{definition:grothendieck_topology_on_a_category_site_covering_sieve_topologically_generating_family}{site}, and $\mathcal{G}$ a \CrefAndHyperrefIfExist{definition:sheaf_on_a_site}{sheaf} of groups (not necessarily abelian) on it. 
\TODO{define cech cohomology of the covering and cech cohomology}
\begin{enumerate}
    \item For an object $U \in \mathcal{C}$, and a covering family $\mathfrak{U} = \{ U_i \to U \}_{i \in I} \in J(U)$, the \hldef{\v{C}ech complex} \hl{$C^\bullet(\mathfrak{U},\mathcal{G})$} is defined by:
    $$ C^n(\mathfrak{U},\mathcal{G}) := \prod_{(i_0,\ldots,i_n) \in I^{n+1}} \mathcal{G}(U_{i_0} \times_U \cdots \times_U U_{i_n}), $$
    with face maps defined by restriction and group operations giving a cosimplicial object.
\end{enumerate}
\end{definition}

\subsubsection{Sheaf cohomology of a sheaf of groups}



\begin{definition}[Torsors under a Sheaf of Groups] \label{definition:torsor_principal_homogeneous_space_of_a_sheaf_of_groups_on_a_site_over_an_object}
    \TODO{group object, action by a group object}
    Let $(\mathcal{C},J)$ be a \CrefAndHyperrefIfExist{definition:grothendieck_topology_on_a_category_site_covering_sieve_topologically_generating_family}{site} and $\mathcal{G}$ a \CrefAndHyperrefIfExist{definition:sheaf_on_a_site}{sheaf of groups on it}. A \hldef{$\mathcal{G}$-torsor} (or synonymously a \hldef{principal $\calG$-bundle} or \hldef{principal $\calG$-homogeneous space}, etc.) is a sheaf $\mathcal{P}$ of sets on $(\mathcal{C},J)$ equipped with a (right) action of $\mathcal{G}U$ such that 
        \TODO{This needs some genuine fixing to be definable on a general site, not just one given by a pretopology}
\CrefIfExists{definition:site_induced_by_a_site_on_an_over_category} 
    \begin{enumerate}
        \item For each object $U$ of $\calC$, the action of the group $G(U)$ on $\calP(U)$ is free and transitive whenever $\calP(U)$ is nonempty.

        \item $\calP$ is locally trivial. More explicitly, for each object $U$ of $\calC$, there exists some covering $\{V_i \to U\}_{i \in I}$ in $J$ such that there is an isomorphism 
        $$\calP|_{U_i} \calG|_{U_i}$$
        as a sheaves with $\calG|_{U_i}$ actions for each $i$.
        
    \end{enumerate}
    A $\calG$-torsor is called \hldef{trivial on $U$} if it is isomorphic (as a sheaf with a right action of $\calG$) to $\calG|_U$ acting on itself by right multiplication, or equivalently, if $\calG(U) \neq \emptyset$. In particular, any $\calG$-torsor is \emph{locally} trivial.
    % This local condition ensures that $\mathcal{P}$ "looks like" the trivial torsor $\mathcal{G}|_U$ when viewed on suitably small neighborhoods covering $U$.

\end{definition}


\begin{definition}[Nonabelian Sheaf Cohomology: $H^0$ and $H^1$] \label{definition:zeroth_and_first_nonabelian_sheaf_cohomology_of_a_sheaf_of_groups_on_a_site}
   Let $(\mathcal{C},J)$ be a \CrefAndHyperrefIfExist{definition:grothendieck_topology_on_a_category_site_covering_sieve_topologically_generating_family}{site} and $\mathcal{G}$ a \CrefAndHyperrefIfExist{definition:sheaf_on_a_site}{sheaf} of groups on $(\mathcal{C},J)$. For $U \in \mathcal{C}$, we define the \hldef{$0$th and $1$st (nonabelian) sheaf cohomology sets} as follows:

        \begin{itemize}
        \item The zeroth sheaf cohomology group \hl{$\displaystyle H^0(\calC, J, \mathcal{G}) := \Gamma(\calG)$} is the group of \CrefAndHyperrefIfExist{definition:sections_of_a_presheaf_on_a_category_valued_in_a_category}{global sections} of $\calG$, assuming that it exists (which is always the case when $\calC$ is \CrefAndHyperrefIfExist{definition:essentially_small_category}{essentially small} for example since the category of groups is closed under small projective limits \TODO{}).

        \item The first sheaf cohomology set \hl{$\displaystyle H^1(\calC, J, \mathcal{G})$} is the pointed set of isomorphism classes of \CrefAndHyperrefIfExist{definition:torsor_principal_homogeneous_space_of_a_sheaf_of_groups_on_a_site_over_an_object}{$\mathcal{G}$-torsors} on the site $(\mathcal{C}, J)$.
        % , where a \hldef{$\mathcal{G}$-torsor} is a sheaf $\mathcal{P}$ on $(\mathcal{C}/U,J|_{U})$ with a free and transitive right action of $\mathcal{G}|_{U}$ such that there exists a covering $\{V_i \to U\}$ with $\mathcal{P}(V_i) \neq \emptyset$.
        \end{itemize}
\end{definition}

\begin{proposition}{See {\cite[Tag 03AJ]{stacks-project}} for a statement\footnote{While the Stacks Project defines a site as a category eqiupped with a Grothendieck \emph{pre}topology, its proof of this statement should be applicable for more general sites}} \label{proposition:zeroth_and_first_nonabelian_and_abelian_sheaf_cohomologies_of_a_sheaf_of_abelian_groups_on_}
    Let $(\calC, J)$ be a \CrefAndHyperrefIfExist{definition:grothendieck_topology_on_a_category_site_covering_sieve_topologically_generating_family}{site}. 
    Assume that \CrefAndHyperrefIfExist{definition:sections_of_a_presheaf_on_a_category_valued_in_a_category}{global sections objects $\Gamma(\calG)$} exist for all objects $\calG$ of $\mathrm{Sh}(\mathcal{C}, \mathbf{Ab})$\footnote{for example, this occurs when $\calC$ is \CrefAndHyperrefIfExist{definition:essentially_small_category}{essentially small}} so that $\Gamma$ is a functor
    $$\Sh(\calC, \mathbf{Ab}) \to \mathbf{Ab}.$$

    Let $\calG$ be a \CrefAndHyperrefIfExist{definition:sheaf_on_a_site}{sheaf} of abelian groups on $(\calC, J)$. For $i = 0,1$, there is a canonical bijection between the set of isomorphism classes of \CrefAndHyperrefIfExist{definition:torsor_principal_homogeneous_space_of_a_sheaf_of_groups_on_a_site_over_an_object}{$\calG$-torsors} and the \CrefAndHyperrefIfExist{definition:sheaf_cohomology_group_of_a_sheaf_of_modules_over_a_sheaf_of_rings_on_a_site}{(abelian) sheaf cohomology group $H^i(\calC, J; \calG)$}. In other words, for $i = 0,1$, there is a canonical bijection between the \CrefAndHyperrefIfExist{definition:zeroth_and_first_nonabelian_sheaf_cohomology_of_a_sheaf_of_groups_on_a_site}{$i$th nonabelian sheaf cohomology} and the \CrefAndHyperrefIfExist{definition:sheaf_cohomology_group_of_a_sheaf_of_modules_over_a_sheaf_of_rings_on_a_site}{$i$th abelian sheaf cohomology} of $\calG$. 
\end{proposition}

\begin{proof}
    In the case of $i = 0$, both are identifiable with the groups of global sections. In the case of $i = 1$, see {\cite[Tag 03AJ]{stacks-project}}. 
\end{proof}


\section{Homological Algebra}

% \begin{definition}[Chain complex in an additive category] \label{definition:chain_complex_of_objects_in_an_additive_category}
% Let $\mathcal{A}$ be an \hyperrefIfExists{definition:additive_category_preadditive_category}{preadditive category} and let $I$ be a totally ordered set (typically $\mathbb{Z}$, but $I \subseteq \mathbb{Z}$ is also allowed). 
% \begin{enumerate}
%     \item A \hldef{chain complex} $(K^\bullet, d^\bullet)$ in $\mathcal{A}$ indexed by $I$ consists of:
%     \begin{itemize}
%         \item Objects $\{ K^i \}_{i \in I}$ in $\mathcal{A}$, called the \hldef{terms in degree $i$},
%         \item Morphisms $d^i: K^i \to K^{i+1}$ in $\mathcal{A}$, called the \hldef{differentials in degree $i$},
%     \end{itemize}
%     such that for every $i \in I$, $d^{i+1} \circ d^i = 0$. That is,
%     $$ K^\bullet: \cdots \xrightarrow{d^{i-2}} K^{i-1} \xrightarrow{d^{i-1}} K^i \xrightarrow{d^i} K^{i+1} \xrightarrow{d^{i+1}} \cdots $$
%     with $d^{i+1}d^i = 0$ for all $i$. We might typically use notation such as \hl{$K^\bullet = (K^i, d^i)_{i \in I}$} to denote a chain complex in $\mathcal{A}$.

%     A cochain complex can be defined similarly/dually.

%     \item Let $K^\bullet = (K^i, d_K^i)$ and $L^\bullet = (L^i, d_L^i)$ be \CrefAndHyperrefIfExist{definition:chain_complex_of_objects_in_an_additive_category}{chain complexes} in $\mathcal{A}$ indexed by the same set $I$. 
%     A \hldef{morphism of chain complexes} (or \hldef{chain map})
%     $$ f^\bullet: K^\bullet \to L^\bullet $$
%     consists of morphisms $f^i: K^i \to L^i$ for all $i \in I$, such that for every $i \in I$,
%     $$ d_L^i \circ f^i = f^{i+1} \circ d_K^i, $$
%     i.e., the following diagram commutes for all $i$:

%     $$ \begin{array}{ccc} K^i & \xrightarrow{d_K^i} & K^{i+1} \\ \downarrow{f^i} && \downarrow{f^{i+1}} \\ L^i & \xrightarrow{d_L^i} & L^{i+1} \end{array}.$$

% \end{enumerate}

% There is then a category, often denoted by \hl{$\mathrm{Ch}(\mathcal{A})$} or \hl{$\mathbf{Ch}(\mathcal{A})$}, whose objects are chain complexes in $\calA$ and whose morphisms are morphisms of chain complexes. In particular, we may denote by 
% $$\hlin{\operatorname{Hom}(K^\bullet, L^\bullet)=  \operatorname{Hom}_{\mathrm{Ch}(\mathcal{A})}(K^\bullet, L^\bullet)}$$
% the set of chain maps $K^\bullet \to L^\bullet$; it is in fact an abelian group.

% A \hldef{morphism of cochain complexes} is defined similarly, and we similarly denote by \hl{$\mathrm{Ch}(\mathcal{A})$} or \hl{$\mathbf{Ch}(\mathcal{A})$} the caetgory of cochain complexes in $\calA$. 


% \TextIfExists{definition:dg_category_over_a_ring}{
% If $k$ is a \CrefAndHyperrefIfExist{definition:commutative_ring}{commutative ring} such that $\Hom_\calA(X,Y)$ is \CrefAndHyperrefIfExist{definition:category_enriched_in_a_monoidal_category}{enriched in} the category of \CrefAndHyperrefIfExist{definition:module_of_a_ring}{$k$-modules}, then $\mathrm{Ch}(\calA)$ \CrefAndHyperref{definition:category_of_chain_complexes_of_objects_in_an_additive_category_as_a_dg_category}{can be equipped with} the structure of a \CrefAndHyperrefIfExist{definition:dg_category_over_a_ring}{dg-category over $k$}.
% }


% \end{definition}

\begin{definition}[Chain complex in a preadditive category] \label{definition:chain_complex_of_objects_in_an_additive_category}
Let $\mathcal{A}$ be a \hyperrefIfExists{definition:additive_category_preadditive_category}{preadditive category} and let $I$ be a totally ordered set (typically $\mathbb{Z}$, but $I \subseteq \mathbb{Z}$ is also allowed). 
\begin{enumerate}
    \item A \hldef{chain complex} $(K_\bullet, d_\bullet)$ in $\mathcal{A}$ indexed by $I$ is the homological convention for sequences with decreasing degrees. It consists of:
    \begin{itemize}
        \item Objects $\{ K_i \}_{i \in I}$ in $\mathcal{A}$, called the \hldef{terms in degree $i$},
        \item Morphisms $d_i: K_i \to K_{i-1}$ in $\mathcal{A}$, called the \hldef{boundary maps} or \hldef{differentials in degree $i$},
    \end{itemize}
    such that for every $i \in I$, $d_{i-1} \circ d_i = 0$. That is,
    $$ K_\bullet: \cdots \xrightarrow{d_{i+1}} K_i \xrightarrow{d_i} K_{i-1} \xrightarrow{d_{i-1}} K_{i-2} \xrightarrow{} \cdots $$
    with $d_{i-1}d_i = 0$ for all $i$. We typically use the notation \hl{$K_\bullet = (K_i, d_i)_{i \in I}$}.



    \item Dually, a \hldef{cochain complex} $(K^\bullet, d^\bullet)$ in $\mathcal{A}$ follows the \hldef{cohomological convention} with increasing degrees. It consists of objects $\{ K^i \}_{i \in I}$ and \hldef{coboundary maps} $d^i: K^i \to K^{i+1}$ such that $d^{i+1} \circ d^i = 0$:
    $$ K^\bullet: \cdots \xrightarrow{d^{i-1}} K^i \xrightarrow{d^i} K^{i+1} \xrightarrow{d^{i+1}} K^{i+2} \xrightarrow{} \cdots $$
    We typically use the notation \hl{$K^\bullet = (K^i, d^i)_{i \in I}$}.

    \item Let $K_\bullet = (K_i, d_i^K)$ and $L_\bullet = (L_i, d_i^L)$ be \CrefAndHyperrefIfExist{definition:chain_complex_of_objects_in_an_additive_category}{chain complexes} in $\mathcal{A}$ indexed by the same set $I$. A \hldef{morphism of chain complexes} (or \hldef{chain map})
    $$ f_\bullet: K_\bullet \to L_\bullet $$
    consists of morphisms $f_i: K_i \to L_i$ for all $i \in I$, such that for every $i \in I$, the following diagram commutes:
    $$ \begin{array}{ccc} K_i & \xrightarrow{d_i^K} & K_{i-1} \\ \downarrow{f_i} && \downarrow{f_{i-1}} \\ L_i & \xrightarrow{d_i^L} & L_{i-1} \end{array} $$
    i.e., $d_i^L \circ f_i = f_{i-1} \circ d_i^K$. 



    A \hldef{morphism of cochain complexes} $f^\bullet: K^\bullet \to L^\bullet$ is defined similarly, satisfying the commutativity condition $d_L^i \circ f^i = f^{i+1} \circ d_K^i$.
\end{enumerate}

The collection of these objects and morphisms forms a category. Notation for these categories is as follows:
\begin{itemize}
    \item \hl{$\mathrm{Ch}(\mathcal{A})$} or \hl{$\mathbf{Ch}(\mathcal{A})$} is often used as a general term.
    \item To be explicit about the indexing convention, one uses \hl{$\mathrm{Ch}_\bullet(\mathcal{A})$} for chain complexes and \hl{$\mathrm{Ch}^\bullet(\mathcal{A})$} (or sometimes $\mathrm{CoCh}(\mathcal{A})$) for cochain complexes.
    \item The set of chain maps between two complexes is denoted by $\hlin{\operatorname{Hom}_{\mathrm{Ch}(\mathcal{A})}(K_\bullet, L_\bullet)}$; it is an abelian group under pointwise addition $(f+g)_i = f_i + g_i$.
\end{itemize}

\TextIfExists{definition:dg_category_over_a_ring}{
If $k$ is a \CrefAndHyperrefIfExist{definition:commutative_ring}{commutative ring} such that $\Hom_\calA(X,Y)$ is \CrefAndHyperrefIfExist{definition:category_enriched_in_a_monoidal_category}{enriched in} the category of \CrefAndHyperrefIfExist{definition:module_of_a_ring}{$k$-modules}, then $\mathrm{Ch}(\calA)$ \CrefAndHyperref{definition:category_of_chain_complexes_of_objects_in_an_additive_category_as_a_dg_category}{can be equipped with} the structure of a \CrefAndHyperrefIfExist{definition:dg_category_over_a_ring}{dg-category over $k$}.
}
\end{definition}

% 
\begin{remark} \label{remark:cohomological_vs_homological_conventions}
    The convention used to define chain complexes in \Cref{definition:chain_complex_of_objects_in_an_additive_category} is a \emph{cohomological one} --- note that indices are written as superscripts and increase when ``following the arrows''. Such a chain complex may also be referred to as a \hldef{cochain complex} or a \hldef{cohomological chain complex} to emphasize an adoption of a cohomological convention. 

    The dual convention would be a \emph{homological one}, in which indices are written as subscripts and decrease when ``following the arrow''. As such, one may speak of a \hldef{(homological) chain complex} $(K_\bullet, d_\bullet)$ indexed by $I$ as consisting of:

    \begin{itemize}
    \item Objects $\{ K_i \}_{i \in I}$ in $\mathcal{A}$, called the \hldef{terms in degree $i$},
    \item Morphisms $d_i: K_i \to K_{i-1}$ in $\mathcal{A}$, called the \hldef{differentials in degree $i$},
    \end{itemize}
    such that for every $i \in I$, $d_{i-1} \circ d_i = 0$. That is,
    $$ 
    K_\bullet: \cdots \xrightarrow{d_{i+1}} K_i \xrightarrow{d_i} K_{i-1} \xrightarrow{d_{i-1}} K_{i-2} \xrightarrow{d_{i-2}} \cdots
    $$
    with $d_{i-1} d_i = 0$ for all $i$. We might typically use notation such as \hl{$K_\bullet = (K_i, d_i)_{i \in I}$} to denote a chain complex in $\mathcal{A}$.

    The differences between the conventions persist --- for example, cohomological objects are usually written with superscript indicees whereas homological objects are usually written with subscript indicees.
\end{remark}
%\begin{convention} \label{convention:homological_algebra_is_discussed_in_cohomological_terms}
    When discussing homological algebra in abstract terms, we may often adopt the homological convention in some discussions and the cohomological convention in others \CrefIfExists{remark:cohomological_vs_homological_conventions}.
    %; for instance, indices are written as superscripts and increase along the direction of the arrows in chain complexes. 
\end{convention}

% \begin{definition}[Morphisms of chain complexes] \label{definition:chain_complex_of_objects_in_an_additive_category}
Let $\mathcal{A}$ be an \CrefAndHyperrefIfExist{definition:additive_category}{additive category}, and let $K^\bullet = (K^i, d_K^i)$ and $L^\bullet = (L^i, d_L^i)$ be \CrefAndHyperrefIfExist{definition:chain_complex_of_objects_in_an_additive_category}{chain complexes} in $\mathcal{A}$ indexed by the same set $I$. 
A \hldef{morphism of chain complexes} (or \hldef{chain map})
$$ f^\bullet: K^\bullet \to L^\bullet $$
consists of morphisms $f^i: K^i \to L^i$ for all $i \in I$, such that for every $i \in I$,
$$ d_L^i \circ f^i = f^{i+1} \circ d_K^i, $$
i.e., the following diagram commutes for all $i$:

$$ \begin{array}{ccc} K^i & \xrightarrow{d_K^i} & K^{i+1} \\ \downarrow{f^i} && \downarrow{f^{i+1}} \\ L^i & \xrightarrow{d_L^i} & L^{i+1} \end{array}.$$

There is then a category, often denoted by \hl{$\mathrm{Ch}(\mathcal{A})$} or \hl{$\mathbf{Ch}(\mathcal{A})$}, whose objects are chain complexes in $\calA$ and whose morphisms are morphisms of chain complexes. In particular, we may denote by 
$$\hlin{\operatorname{Hom}(K^\bullet, L^\bullet)=  \operatorname{Hom}_{\mathrm{Ch}(\mathcal{A})}(K^\bullet, L^\bullet)}$$
the set of chain maps $K^\bullet \to L^\bullet$; it is in fact an abelian group.

A \hldef{morphism of cochain complexes} is defined similarly, and we similarly denote by \hl{$\mathrm{Ch}(\mathcal{A})$} or \hl{$\mathbf{Ch}(\mathcal{A})$} the caetgory of cochain complexes in $\calA$. 
\end{definition}

% See Also
% 
\begin{proposition} \label{proposition:category_of_chain_complexes_in_an_additive_category_is_additive}
Let $\mathcal{A}$ be an \hyperrefIfExists{definition:additive_category}{additive category}. 
\begin{enumerate}
    \item The category \hyperrefIfExists{definition:chain_complex_of_objects_in_an_additive_category}{$\mathrm{Ch}(\calA)$} of chain complexes is itself and additive category.

    \item If $\calA$ is an \hyperrefIfExists{definition:abelian_category}{abelian category}, then $\mathrm{Ch}(\calA)$ is an abelian category.

    \item If $\calA$ is an \hyperrefIfExists{definition:abelian_category}{abelian category} satisfying Grothendieck's axiom \CrefAndHyperrefIfExist{definition:grothendiecks_additional_axioms_for_abelian_categories}{AB$n$ (resp. AB$n^*$)} for $n \in \{3,4,5,6\}$, then $\mathrm{Ch}(\calA)$ also satisfies AB$n$ (resp. AB$n^*$). If $\calA$ is a \CrefAndHyperrefIfExist{definition:grothendiecks_additional_axioms_for_abelian_categories}{Grothendieck abelian category}, then so is $\mathrm{Ch}(\calA)$
\end{enumerate}
\end{proposition}
\begin{proof}
    Combine \Cref{proposition:category_of_chain_complexes_of_objects_in_a_preadditive_category_is_equivalent_to_the_category_of_additive_functors_from_the_walking_chain_complex_category} and \Cref{lemma:additive_functor_category_from_small_preadditive_categories_preserves}.
\end{proof}

% \begin{corollary}
% Let $\calB$ be a \CrefAndHyperrefIfExist{definition:additive_category}{preadditive category}.
% \begin{enumerate}
%     \item The \CrefAndHyperrefIfExist{definition:chain_complex_of_objects_in_an_additive_category}{(co)chain complex category} $\text{Ch}(\calB)$ is preadditive. If $\calB$ is additionally \CrefAndHyperrefIfExist{definition:additive_category}{additive}/\CrefAndHyperrefIfExist{definition:abelian_category}{abelian}, then so is $\text{Ch}(\calB)$.

%     \item If $\calB$ is an abelian category with property \CrefAndHyperrefIfExist{definition:grothendiecks_additional_axioms_for_abelian_categories}{$ABn$ for $n = 3,4,5,6$ or $ABn^*$ for $n = 3,4,5$}, then $\text{Add}(\calA, \calB)$ possesses the same property.
% \end{enumerate}

% \end{corollary}
\begin{definition} \label{definition:quiver}
A \hldef{quiver} is a quadruple $Q = (Q_0, Q_1, s, t)$, where:
\begin{itemize}
    \item $Q_0$ is a collection of \hldef{vertices}.
    \item $Q_1$ is a collection of \hldef{arrows}.
    \item $s, t: Q_1 \to Q_0$ are functions assigning to each arrow $\alpha \in Q_1$ its \hldef{source} $s(\alpha)$ and its \hldef{target} $t(\alpha)$.
\end{itemize}
\end{definition}
\begin{definition} \label{definition:path_category_generated_by_a_quiver}
Let $Q$ be a \CrefAndHyperrefIfExist{definition:quiver}{quiver}.
The \hldef{path category generated by $Q$}, denoted \hl{$\mathcal{F}(Q)$}, is the \CrefAndHyperrefIfExist{definition:category}{category} defined as follows:
\begin{itemize}
    \item The objects of $\mathcal{F}(Q)$ are the vertices $Q_0$.
    \item For any two objects $x, y \in Q_0$, the set of morphisms $\text{Hom}_{\mathcal{F}(Q)}(x, y)$ consists of all paths from $x$ to $y$ --- A \hldef{path of length $n \ge 1$ from $x$ to $y$} is a sequence of arrows $\alpha_n \dots \alpha_1$ such that $s(\alpha_1) = x$, $t(\alpha_n) = y$, and $s(\alpha_{i+1}) = t(\alpha_i)$ for all $1 \le i < n$. Additionally, for each vertex $x$, there is a path $e_x$ of length $0$, which serves as the identity morphism.
    \item Composition of morphisms is defined by the concatenation of paths.
\end{itemize}
\end{definition}
\begin{definition} \label{definition:preadditive_category_generated_by_a_quiver}
Let $Q$ be a \CrefAndHyperrefIfExist{definition:quiver}{quiver} whose collection of arrows is small.

The \hldef{preadditive category generated by $Q$}, denoted \hl{$\mathbb{Z}Q$}, is the \CrefAndHyperrefIfExist{definition:additive_category_preadditive_category}{preadditive category}, i.e. the \CrefAndHyperrefIfExist{definition:category_enriched_in_a_monoidal_category}{category enriched over} the \CrefAndHyperrefIfExist{definition:category_of_groups_of_abelian_groups}{category of abelian groups} defined as follows:
\begin{itemize}
    \item The objects of $\mathbb{Z}Q$ are the vertices $Q_0$.
    \item For any objects $x, y \in Q_0$, the morphism set $\text{Hom}_{\mathbb{Z}Q}(x, y)$ is the free abelian group generated by the set of all paths from $x$ to $y$ in $Q$.
    \item Composition is the unique bilinear extension of the path concatenation in $\mathcal{F}(Q)$. That is, for paths $u, v, w$ where concatenation is defined, composition satisfies $(u + v) \circ w = u \circ w + v \circ w$ and $w \circ (u + v) = w \circ u + w \circ v$.
\end{itemize}
\end{definition}
\begin{definition} \label{definition:walking_chain_complex_category}
Let $Q_{\text{chain}}$ be the \CrefAndHyperrefIfExist{definition:quiver}{quiver} with vertex set $Q_0 = \mathbb{Z}$ and arrow set $Q_1 = \{ d_n : n \to n-1 \mid n \in \mathbb{Z} \}$. 
\TODO{quotient of a category}
The \hldef{walking chain complex category}, denoted \hl{$\mathcal{I}$}, is the quotient of the \CrefAndHyperrefIfExist{definition:additive_category}{preadditive category} $\mathbb{Z}Q_{\text{chain}}$ by the ideal generated by the relations $d_{n-1} \circ d_n = 0$ for all $n \in \mathbb{Z}$. Explicitly:
\begin{itemize}
    \item Objects are the integers $\mathbb{Z}$.
    \item Morphisms are $\mathbb{Z}$-linear combinations of paths, subject to the relation that any path containing a subsegment $d_{n-1}d_n$ is identified with the zero morphism.
\end{itemize}
\end{definition}
\begin{definition} \label{definition:additive_functor_category_between_preadditive_categories}
Let $\mathcal{A}$ and $\mathcal{B}$ be \CrefAndHyperrefIfExist{definition:additive_category_preadditive_category}{preadditive categories} (\CrefAndHyperrefIfExist{definition:category_enriched_in_a_monoidal_category}{categories enriched over} the \CrefAndHyperrefIfExist{definition:category_of_groups_of_abelian_groups}{category of abelian groups}). The \hldef{additive functor category} \hl{$\text{Add}(\mathcal{A}, \mathcal{B})$} is the functor category where:
\begin{itemize}
    \item Objects are {additive functors} $F: \mathcal{A} \to \mathcal{B}$. An \CrefAndHyperrefIfExist{definition:additive_functor_between_additive_categories}{additive functor} is a functor such that for any $x, y \in \text{Ob}(\mathcal{A})$, the map $F: \text{Hom}_{\mathcal{A}}(x, y) \to \text{Hom}_{\mathcal{B}}(F(x), F(y))$ is a group homomorphism.
    \item Morphisms are \CrefAndHyperrefIfExist{definition:natural_transformation_between_functors_between_categories}{natural transformations} between \CrefAndHyperrefIfExist{definition:additive_functor_between_additive_categories}{additive functors}.
\end{itemize}
\end{definition}
\input{../_concepts/proposition_category_of_chain_complexes_of_objects_in_a_preadditive_category_is_equivalent_to_the_category_of_additive_functors_from_the_walking_chain_complex_category.tex}
\begin{lemma} \label{lemma:additive_functor_category_from_small_preadditive_categories_preserves}
Let $\calA, \calB$ be \CrefAndHyperrefIfExist{definition:additive_category}{preadditive categories} with $\calA$ \CrefAndHyperrefIfExist{definition:locally_small_category}{small}.
\begin{enumerate}
    \item The \CrefAndHyperrefIfExist{definition:additive_functor_category_between_preadditive_categories}{additive functor category} $\text{Add}(\calA, \calB)$ is preadditive. If $\calB$ is additionally \CrefAndHyperrefIfExist{definition:additive_category}{additive}/\CrefAndHyperrefIfExist{definition:abelian_category}{abelian}, then so is $\text{Add}(\calA, \calB)$.

    \item If $\calB$ is an abelian category with property \CrefAndHyperrefIfExist{definition:grothendiecks_additional_axioms_for_abelian_categories}{$ABn$ for $n = 3,4,5,6$ or $ABn^*$ for $n = 3,4,5$}, then $\text{Add}(\calA, \calB)$ possesses the same property.
\end{enumerate}
\end{lemma}

\begin{proposition} \label{proposition:category_of_chain_complexes_in_an_additive_category_is_additive}
Let $\mathcal{A}$ be an \hyperrefIfExists{definition:additive_category}{additive category}. 
\begin{enumerate}
    \item The category \hyperrefIfExists{definition:chain_complex_of_objects_in_an_additive_category}{$\mathrm{Ch}(\calA)$} of chain complexes is itself and additive category.

    \item If $\calA$ is an \hyperrefIfExists{definition:abelian_category}{abelian category}, then $\mathrm{Ch}(\calA)$ is an abelian category.

    \item If $\calA$ is an \hyperrefIfExists{definition:abelian_category}{abelian category} satisfying Grothendieck's axiom \CrefAndHyperrefIfExist{definition:grothendiecks_additional_axioms_for_abelian_categories}{AB$n$ (resp. AB$n^*$)} for $n \in \{3,4,5,6\}$, then $\mathrm{Ch}(\calA)$ also satisfies AB$n$ (resp. AB$n^*$). If $\calA$ is a \CrefAndHyperrefIfExist{definition:grothendiecks_additional_axioms_for_abelian_categories}{Grothendieck abelian category}, then so is $\mathrm{Ch}(\calA)$
\end{enumerate}
\end{proposition}
\begin{proof}
    Combine \Cref{proposition:category_of_chain_complexes_of_objects_in_a_preadditive_category_is_equivalent_to_the_category_of_additive_functors_from_the_walking_chain_complex_category} and \Cref{lemma:additive_functor_category_from_small_preadditive_categories_preserves}.
\end{proof}

% \begin{corollary}
% Let $\calB$ be a \CrefAndHyperrefIfExist{definition:additive_category}{preadditive category}.
% \begin{enumerate}
%     \item The \CrefAndHyperrefIfExist{definition:chain_complex_of_objects_in_an_additive_category}{(co)chain complex category} $\text{Ch}(\calB)$ is preadditive. If $\calB$ is additionally \CrefAndHyperrefIfExist{definition:additive_category}{additive}/\CrefAndHyperrefIfExist{definition:abelian_category}{abelian}, then so is $\text{Ch}(\calB)$.

%     \item If $\calB$ is an abelian category with property \CrefAndHyperrefIfExist{definition:grothendiecks_additional_axioms_for_abelian_categories}{$ABn$ for $n = 3,4,5,6$ or $ABn^*$ for $n = 3,4,5$}, then $\text{Add}(\calA, \calB)$ possesses the same property.
% \end{enumerate}

% \end{corollary}


\section{Derived categories}

\import{../_excerpts}{excerpts_derived_categories.tex}


\section{Simplicial Categories}

\TODO{}

\begin{notation}
Let $\mathbf{Cat}$ denote the category of (small) categories.  
Let $\Delta$ denote the simplex category, whose objects are finite nonempty totally ordered sets $[n] := \{0, 1, \dots, n\}$ for $n \ge 0$, and whose morphisms are all order-preserving functions $\theta : [m] \to [n]$.  
Let $\Delta^{\mathrm{op}}$ denote its opposite category.
\end{notation}


\begin{definition} \label{definition:geometric_simplex_of_independent_points_in_a_real_vector_space}
Let $V$ be a real vector space of finite dimension.  
A \hldef{$k$-simplex in topology} (or a \hldef{geometric $k$-simplex}) is the convex hull of $k+1$ affinely independent points $v_0, v_1, \dots, v_k \in V$, and is denoted by
$$\hlin{[v_0, v_1, \dots, v_k] := \left\{ \sum_{i=0}^k t_i v_i \ \middle| \ t_i \ge 0, \ \sum_{i=0}^k t_i = 1 \right\}.}$$


It is also standard to talk of the \hldef{standard topological $n$-simplex} ---  the topological space \hl{$|\Delta^n|$} defined as the subset of Euclidean space $\mathbb{R}^{n+1}$ given by
$$ |\Delta^n| = \Big\{ (t_0, t_1, \ldots, t_n) \in \mathbb{R}^{n+1} : \sum_{i=0}^n t_i = 1, \text{ and } t_i \geq 0 \text{ for all } i \Big\} $$
equipped with the induced topology from the usual Euclidean topology on $\mathbb{R}^{n+1}$.

\TODO{comment on how $|\Delta^n|$ makes sense via a geometric realization}
% Eqiuvalently, $|\Delta^n|$ 

\end{definition}

\begin{definition} \label{definition:simplex_category}
The \hldef{simplex category}, or \hldef{nonempty finite ordinal category}, denoted by \hl{$\Delta$}, is the \CrefAndHyperrefIfExist{definition:locally_small_category}{locally small} \CrefAndHyperrefIfExist{definition:category}{category} whose
\begin{itemize}
  \item objects are the finite nonempty totally ordered sets \hl{$[n] := \{0, 1, 2, \dots, n\}$} for each integer $n \ge 0$;
  \item morphisms are all order-preserving (non-decreasing) \CrefAndHyperrefIfExist{definition:function_of_sets}{functions} $\theta : [m] \to [n]$.
\end{itemize}
Composition in $\Delta$ is given by composition of functions.

\end{definition}

\begin{definition} \label{definition:simplicial_cosimplicial_object_in_a_category}
    Let $\mathcal{C}$ be a \CrefAndHyperrefIfExist{definition:category}{category}.
    \begin{enumerate}
        \item   
        The \hldef{category of simplicial objects in $\mathcal{C}$}, commonly denoted by notations such as \hl{$\mathbf{s}\mathcal{C}$}, \hl{$\operatorname{Simp} \mathcal{C}$}, \hl{$\calC_\Delta$}, or \hl{$(\Delta^{\op})^{\calC}$} (cf. \Cref{definition:diagram_in_a_category_indexed_by_a_small_category}), or \hl{$\Deltaop C$}, is the \CrefAndHyperrefIfExist{definition:diagram_in_a_category_indexed_by_a_small_category}{functor category}
        $$ \mathbf{s}\mathcal{C} := \mathbf{Fun}(\Delta^{\mathrm{op}}, \mathcal{C}).$$
        \CrefIfExists{definition:opposite_category_of_a_category} \CrefIfExists{definition:simplex_category} In particular, a morphism between objects $X,Y:\Delta^{\op} \to \calC$ in this category is a natural transform $X \Rightarrow Y$ from $X$ to $Y$ as functors. 

        An object $X$ of $\mathbf{s}\mathcal{C}$ is called a \hldef{simplicial object of $\mathcal{C}$}, and, by \CrefAndHyperrefIfExist{definition:functor_between_categories}{definition}, consists of a family of objects $\{X_n\}_{n \ge 0}$ in $\mathcal{C}$ together with morphisms
        $$ X(\theta) : X_n \to X_m, \quad \text{for each } \theta : [m] \to [n] \text{ in } \Delta, $$
        satisfying the functoriality conditions
        $$ X(\mathrm{id}_{[n]}) = \mathrm{id}_{X_n}, \qquad X(\theta \circ \psi) = X(\psi) \circ X(\theta).  $$

        \item Dually, the \hldef{category of cosimplicial objects in $\mathcal{C}$}, commonly denoted by notatoins such as \hl{$\mathbf{c}\mathcal{C}$}, \hl{$\operatorname{Cosimp} \mathcal{C}$}, or \hl{$\Delta^{\calC}$} (cf. \Cref{definition:diagram_in_a_category_indexed_by_a_small_category}) is the functor category
        $$
        \mathbf{c}\mathcal{C} := \mathbf{Fun}(\Delta, \mathcal{C}).
        $$
        In particular, a morphism between objects $X,Y:\Delta \to \calC$ in this category is a natural transform $X \Rightarrow Y$ from $X$ to $Y$ as functors. 

        An object $Y$ of $\mathbf{c}\mathcal{C}$ is called a \hldef{cosimplicial object of $\mathcal{C}$}, and consists of a family of objects $\{Y^n\}_{n \ge 0}$ in $\mathcal{C}$ together with morphisms
        $$
        Y(\theta) : Y^m \to Y^n, \quad \text{for each } \theta : [m] \to [n] \text{ in } \Delta,
        $$
        satisfying the functoriality conditions
        $$
        Y(\mathrm{id}_{[n]}) = \mathrm{id}_{Y^n}, \qquad Y(\theta \circ \psi) = Y(\theta) \circ Y(\psi).
        $$
    \end{enumerate}
For instance, a \hldef{(co)simplicial set, group, topological space, ring, etc.} refers to a (co)simplicial object in the category of sets, of groups, of topological spaces, of rings, etc. and such categories are denoted by notations such as \hl{$\Sets_\Delta$}, \hl{$\mathbf{Grps}_\Delta$}, \hl{$\mathbf{Top}_\Delta$}, \hl{$\mathbf{Rings}_\Delta$}, etc. Accordingly, a \hl{(co)simplicial map} between such (co)simplicial objects refers to a morphism in the appropriate (co)simplicial category.

If $\mathcal{C}$ is \CrefAndHyperrefIfExist{definition:locally_small_category}{locally small}, then both $\mathbf{s}\mathcal{C}$ and $\mathbf{c}\mathcal{C}$ are locally small as well.
\end{definition}



\subsubsection{Geometric realization and singular complex adjunction}

\begin{definition} \label{definition:geometric_realization_of_a_simplicial_set}
    Let $S$ be a \CrefAndHyperrefIfExist{definition:simplicial_cosimplicial_object_in_a_category}{simplicial set}. The \hldef{geometric realization of $S$}, denoted \hl{$|S|$}, is the \CrefAndHyperrefIfExist{definition:compactly_generated_topological_space}{compactly generated} \CrefAndHyperrefIfExist{definition:separation_axioms_of_topology}{Hausdorff} topological space obtained as the coend
    \TODO{coend}
    $$ |S| \;=\; \int^{[n] \in \Delta} S_n \times |\Delta^n|, $$
    \CrefIfExists{definition:simplex_of_a_simplicial_object_in_a_category_and_face_and_degeneracy_maps} where \CrefAndHyperrefIfExist{definition:geometric_simplex_of_independent_points_in_a_real_vector_space}{$|\Delta^n|$ is the standard topological $n$-simplex}, and the equivalence relation is generated by the simplicial structure maps in $S$ and the face and degeneracy maps of the simplices $|\Delta^n|$. Concretely, $|S|$ is given by the quotient space
    $$
    \left(\coprod_{n \geq 0} S_n \times |\Delta^n|\right) \bigg/ \sim,
    $$
    where for every morphism $\varphi : [m] \to [n]$ in the simplex category $\Delta$, the identification
    $$
    (s, \varphi^*(x)) \sim (\varphi_*(s), x)
    $$
    holds for all $s \in S_n$ and $x \in |\Delta^m|$, with $\varphi^* : |\Delta^m| \to |\Delta^n|$ induced by $\varphi$, and $\varphi_* : S_n \to S_m$ the simplicial map induced by $\varphi$. This construction endows $|S|$ with the quotient topology from the disjoint union.
\end{definition}

\begin{definition} \label{definition:singular_complex_of_a_topological_space}
    Let $\mathcal{C}$ be a \CrefAndHyperrefIfExist{definition:topological_space}{topological space}. The \hldef{singular complex of $\mathcal{C}$}, denoted \hl{$\mathrm{Sing}(\mathcal{C})$}, is the \CrefAndHyperrefIfExist{definition:simplicial_cosimplicial_object_in_a_category}{simplicial set} defined as follows:

    \begin{itemize}
    \item For each integer $n \geq 0$, the set of \CrefAndHyperrefIfExist{definition:simplex_of_a_simplicial_object_in_a_category_and_face_and_degeneracy_maps}{$n$-simplices} is the set of \CrefAndHyperrefIfExist{definition:continuous_map_of_topological_spaces}{continuous maps} from the \CrefAndHyperrefIfExist{definition:geometric_simplex_of_independent_points_in_a_real_vector_space}{standard topological $n$-simplex $|\Delta^n|$} to $\mathcal{C}$:
    $$ \mathrm{Sing}(\mathcal{C})_n = \Hom_{\mathrm{Top}}(|\Delta^n|, \mathcal{C}).  $$

    \item The \CrefAndHyperrefIfExist{definition:simplex_of_a_simplicial_object_in_a_category_and_face_and_degeneracy_maps}{face and degeneracy maps} of $\mathrm{Sing}(\mathcal{C})$ are induced by the standard coface and codegeneracy maps of the simplices $|\Delta^n|$ in the usual way, making \hl{$\mathrm{Sing}(\mathcal{C})$} a simplicial set.
    \end{itemize}
\end{definition}

\begin{proposition} \label{proposition:geometric_realization_singular_complex_adjunction}
    \begin{enumerate}
        \item 
        The \CrefAndHyperrefIfExist{definition:geometric_realization_of_a_simplicial_set}{geometric realization functor} and the \CrefAndHyperrefIfExist{definition:singular_complex_of_a_topological_space}{singular complex functor} form an \CrefAndHyperrefIfExist{definition:adjoint_functors_between_categories_unit_counit_of_adjoint_functors}{adjoint pair}
        $$|\cdot| \dashv \operatorname{Sing}$$
        $$|\cdot|: \mathbf{s}\Sets \rightleftarrows \mathrm{Top}: \operatorname{Sing}.$$
        (\Cref{definition:simplicial_cosimplicial_object_in_a_category}) which infact restrict to an adjoint pair
        $$|\cdot|: \mathbf{s}\Sets \rightleftarrows \mathcal{CG}\mathrm{Haus}: \operatorname{Sing}.$$
        where $\mathcal{CG}\mathrm{Haus}$ is the category of \CrefAndHyperrefIfExist{definition:compactly_generated_topological_space}{compactly generated} \CrefAndHyperrefIfExist{definition:separation_axioms_of_topology}{Hausdorff} \CrefAndHyperrefIfExist{definition:topological_space}{topological spaces}. 

        \item Moreover, for every topological space $X$, the \CrefAndHyperrefIfExist{definition:adjoint_functors_between_categories_unit_counit_of_adjoint_functors}{counit map} 
        $$|\operatorname{Sing} X| \to X$$
        is a \CrefAndHyperrefIfExist{definition:weak_homotopy_equivalence_of_topological_spaces}{weak homotopy equivalence} of topological spaces.
    \end{enumerate}
\end{proposition}

\begin{theorem}[Quillen] \label{theorem:unit_and_counit_morphisms_for_geometric_realization_singular_complex_adjunction_are_weak_homotopy_equivalences_for_compactly_generated_spaces_and_simplicial_sets}
    The \CrefAndHyperrefIfExist{definition:adjoint_functors_between_categories_unit_counit_of_adjoint_functors}{unit and counit morphisms}
    \begin{align*}
    S \to \operatorname{Sing}|S|\\
    |\operatorname{Sing} X| \to X
    \end{align*}
    for \CrefAndHyperrefIfExist{definition:simplicial_cosimplicial_object_in_a_category}{simplicial sets} $S$ and topological spaces $X$ are weak homotopy equivalences (\CrefAndHyperrefIfExist{definition:standard_model_structure_on_the_category_of_simplicial_sets}{of simplicial sets} and \CrefAndHyperrefIfExist{definition:weak_homotopy_equivalence_of_topological_spaces}{of topological spaces} respectively).
\end{theorem}


\subsubsection{Homotopy groups of a Kan Simplicial Set}

\begin{definition}[Basepoint of a Simplicial Set] \label{definition:basepoint_of_a_simplicial_set}
Let $X$ be a \CrefAndHyperrefIfExist{definition:simplicial_cosimplicial_object_in_a_category}{simplicial set}. A \hldef{(base)point} or \hldef{basepoint vertex} of $X$ is a chosen element
$$ x \in X_0.$$
A \hldef{pointed simplicial set} is a simplicial set equipped with a choice of base point.
\end{definition}

\begin{definition}[Homotopy Groups of a Kan Simplicial Set] \label{definition:homotopy_groups_of_a_kan_simplicial_set}
Let $X$ be a \CrefAndHyperrefIfExist{definition:kan_complex}{Kan simplicial set}, and fix a \CrefAndHyperrefIfExist{definition:basepoint_of_a_simplicial_set}{basepoint} $x \in X_0$.

For each integer $n \geq 1$, define the \hldef{$n$-th homotopy group of $X$ at $x$}, denoted \hl{$\pi_n(X,x)$}, as the set of homotopy classes of maps
\TODO{homotopy, boundary of $\Delta^n$} 
$$ f: \partial \Delta^{n+1} \to X $$
that restrict to the constant map at $x$ on the basepoint simplex, modulo homotopies relative to the boundary.

Equivalently, $\pi_n(X,x)$ can be described as the set of equivalence classes of \CrefAndHyperrefIfExist{definition:simplicial_cosimplicial_object_in_a_category}{$n$-simplices} whose faces are degenerate at $x$, with composition induced by the combinatorial structure of simplices.

These sets carry natural group structures for $n \geq 1$, with $\pi_1(X,x)$ being the fundamental group and $\pi_n(X,x)$ for $n \geq 2$ being abelian groups.
\end{definition}



\subsubsection{}


\begin{definition} \label{definition:simplex_of_a_simplicial_object_in_a_category_and_face_and_degeneracy_maps}
Let $\mathcal{C}$ be a \CrefAndHyperrefIfExist{definition:category}{category}. 
\begin{enumerate}
    \item Let $X$ be a \CrefAndHyperrefIfExist{definition:simplicial_cosimplicial_object_in_a_category}{simplicial object} in $\mathcal{C}$. An object \hl{$X_n := X([n])$} of $\mathcal{C}$ is called the \hldef{$n$-simplices of $X$}. In case that $\calC$ is some kind of category of sets, an element of $X_n$ is called an \hldef{$n$-simplex of $X$}, so $X_n$ is the \hldef{set of $n$-simplices of $X$}. In this case, a \hldef{vertex of $X$} moreover refers to a $0$-simplex of $X$ and a \hldef{edge of $X$} refers to a $1$-simplex of $X$. 



    For each morphism $\theta : [m] \to [n]$ in $\Delta$, the induced morphism
    $$
    X(\theta) : X_n \to X_m
    $$
    in $\mathcal{C}$ is called a \hldef{simplicial morphism}. 

    For each $0 \leq j \leq n$, the \hldef{$j$th face map of the $n$-simplicies} refers to the map 
    $$\hlin{d_j = X(p_j): X_n \to X_{n-1}, \quad p_j: [n-1] \to [n], \quad p_j(i) = \begin{cases} i &\text{if } i < j  \\ i+1 &\text{if } i \geq j \end{cases}}.$$
    $d_j$ is also denoted by \hl{$\partial_j$}.

    For each $0 \leq j \leq n$, the \hldef{$j$th degeneracy map of the $n$-simplicies} refers to the map 
    $$\hlin{s_j = X(q_j): X_n \to X_{n+1}, \quad q_j: [n+1] \to [n], \quad q_j(i) = \begin{cases} i &\text{if } i \leq j  \\ i-1 &\text{if } i > j \end{cases}}.$$
    
    \item Let $Y : \Delta \to \mathcal{C}$ be a \CrefAndHyperrefIfExist{definition:simplicial_cosimplicial_object_in_a_category}{cosimplicial object} of $\mathcal{C}$. An object \hl{$Y^n := Y([n])$} of $\calC$ is called the \hldef{$n$-cosimplicies of $Y$}. In case that $\calC$ is some kind of category of sets, an element of $Y_n$ is called an \hldef{$n$-cosimplex of $Y$}, so $Y^n$ is the \hldef{set of $n$-cosimplices of $Y$}.
. 

    
    For each morphism $\theta : [m] \to [n]$ in $\Delta$, the induced morphism
    $$ Y(\theta) : Y^m \to Y^n $$
    is called a \hldef{cosimplicial morphism}.  

    For each $0 \leq j \leq n$, the \hldef{$j$th coface map of the $n$-cosimplicies} refers to the map
    $$\hlin{d^j = Y(p_j) : Y^n \to Y^{n+1}, \quad p_j : [n] \to [n+1], \quad p_j(i) = 
    \begin{cases}
    i & \text{if } i < j \\
    i + 1 & \text{if } i \geq j
    \end{cases}
    }$$
    $d^j$ is also denoted by \hl{$\partial^j$}.

    For each $0 \leq j \leq n$, the \hldef{$j$th codegeneracy map of the $n$-cosimplicies} refers to the map
    $$\hlin{s^j = Y(q_j): Y^n \to Y^{n-1}, \quad q_j: [n] \to [n-1], \quad q_j(i) =
    \begin{cases}
    i & \text{if } i \leq j \\
    i - 1 & \text{if } i > j
    \end{cases}
    }$$
    % If $\theta$ is injective, then $Y(\theta)$ is called a \hldef{coface map}; if $\theta$ is surjective, then $Y(\theta)$ is called a \hldef{codegeneracy map}.
\end{enumerate}
A \hldef{(co)face/degeneracy of a the $n$-(co)simplicies of a (co)simplicial object} may also refer to the images of the (co)face/degeneracy maps.
\end{definition}




\begin{definition} \label{definition:standard_n_simplex}
    Let $n \geq 0$ be an integer. The \CrefAndHyperrefIfExist{definition:representable_functor_on_a_category_enriched_in_a_monoidal_category}{representable functor}
    $$\Hom_\Delta(-,[n]): \Delta^{\op} \to \Sets$$
    \CrefIfExists{definition:simplex_category} \CrefIfExists{definition:opposite_category_of_a_category} \CrefIfExists{definition:category_of_sets}
    is a \CrefAndHyperrefIfExist{definition:simplicial_cosimplicial_object_in_a_category}{simplicial set} often denoted by \hl{$\Delta^n$}, \hl{$\Delta^{[n]}$}, or \hl{$\Delta[n]$}, and is called the \hldef{standard $n$-simplex}. More generally, if $J$ is a finite nonempty linearly ordered set, then we may speak of the simplicial set \hldef{$\Delta^J$} given by the representable functor $\Hom_\Delta(-, J)$. 

    The \hldef{Standard simplicial $n$-simplex functor} refers to the functor
    $$\Delta^\bullet: \Delta \to \mathbf{s}\Sets$$
    given by $[n] \mapsto \Delta^n$. By construct, note that $\Delta^\bullet$ is a \CrefAndHyperrefIfExist{definition:simplicial_cosimplicial_object_in_a_category}{cosimplicial object} in the category of simplicial sets. 

    Dually, the functor
    $$\Hom_{\Delta}([n], -): \Delta \to \Sets$$
    is a \CrefAndHyperrefIfExist{definition:simplicial_cosimplicial_object_in_a_category}{cosimplicial set} called the \hldef{standard $n$-cosimplex}.
\end{definition}


$0$-simplices of a simplicial set may be regarded like objects in a category and $1$-simplices between such objects may be regarded like morphisms between the objects --- however, unless the simplicial set satisfies more conditions, composition of morphisms may not exist and may not be unique. 

\begin{definition} \label{definition:categorical_terminology_for_simplices_in_a_simplicial_set_objects_morphisms_compositions_of_morphisms_in_a_simplicial_set}
    Let $X$ be a \CrefAndHyperrefIfExist{definition:simplicial_cosimplicial_object_in_a_category}{simplicial set}. 
    \begin{enumerate}
        \item An \hldef{object of $X$} refers to a \CrefAndHyperrefIfExist{definition:simplex_of_a_simplicial_object_in_a_category_and_face_and_degeneracy_maps}{vertex of $X$}, i.e. a \CrefAndHyperrefIfExist{definition:simplex_of_a_simplicial_object_in_a_category_and_face_and_degeneracy_maps}{$0$-simplex of $X$}. We may write \hl{$x \in X$} or \hl{$x \in \Ob X$} to denote that $x$ is an object of $X$. 

        \item A \hldef{morphism} $x \to y$ where $x,y: \Delta^0 \to X$\CrefIfExists{definition:standard_n_simplex} (\Cref{lemma:n_simplicies_on_a_simplicial_set_are_naturally_isomorphic_to_morphisms_from_standard_n_simplex}) are objects of $X$ is an \CrefAndHyperrefIfExist{definition:simplex_of_a_simplicial_object_in_a_category_and_face_and_degeneracy_maps}{edge} $f:\Delta^1 \to S$ such that $d_1(f) = x$ and $d_0(f) = y$, or equivalently $f(0) = x$ and $f(1) = y$. We may write such a morphism by \hl{$f: x \to y$}.

        \item Given an object $x$ of $X$, the \hldef{identity morphism of $x$} is the edge \hl{$\operatorname{id}_X = s_0(X): \Delta^1 \to X$}. 

        \item Let $f: x \to y$ and $g: y \to z$ be morphisms in $X$. A \hldef{composition of $f$ and $g$} (which may not exist and may not be unique) is a \CrefAndHyperrefIfExist{definition:simplex_of_a_simplicial_object_in_a_category_and_face_and_degeneracy_maps}{$2$-simplex} $F: \Delta^2 \to S$ such that  
        $$d_2(F) = f, \quad \text{and} \quad d_0(F) = g.$$
        \CrefIfExists{definition:simplex_of_a_simplicial_object_in_a_category_and_face_and_degeneracy_maps}. The \hl{composite of $f$ and $g$ with respect to the composition $F$} is then the morphism $d_1(F): x \to z$. 

        In particular, compositions of morphisms exist whenever $X$ is a \CrefAndHyperrefIfExist{definition:infty_category_quasi_category}{quasi-category} and are unique whenever $X$ satisfies the conditions of \Cref{proposition:simplicial_set_is_the_nerve_of_a_small_category_if_and_only_if_maps_from_inner_horns_lift_uniquely}, i.e. when $X$ is isomorphic to the \CrefAndHyperrefIfExist{definition:nerve_of_a_category}{nerve} of a small category.
        \TODO{comment on how different composites of $f$ and $g$ become equivalent in the homotopy category of $X$ in case that $X$ is a quasi-category.}

    \end{enumerate}
\end{definition}

\begin{definition}[Functor between quasi-categories] \label{definition:functor_between_infty_categories}
Let $C$ and $D$ be \CrefAndHyperrefIfExist{definition:infty_category_quasi_category}{$\infty$-categories/quasi-categories}. A \hldef{functor of quasi-categories from $C$ to $D$} is a \CrefAndHyperrefIfExist{definition:simplicial_cosimplicial_object_in_a_category}{simplicial map} $F\colon C\to D$. 

Equivalently, a functor $C \to D$ is a \CrefAndHyperrefIfExist{definition:simplex_of_a_simplicial_object_in_a_category_and_face_and_degeneracy_maps}{$0$-simplex} of the \CrefAndHyperrefIfExist{definition:simplicial_category_of_simplicial_sets}{simplicial mapping object $D^C$}.
\end{definition}


% \begin{notation}[Size, simplicial sets, exponentials, cores, coherent nerve]
% Fix a Grothendieck universe $\mathcal U$. Let $\mathrm{sSet}_{\mathcal U}$ denote the category of $\mathcal U$-small simplicial sets, and for each integer $n\ge 0$ let $\Delta^n$ be the standard $n$-simplex in $\mathrm{sSet}_{\mathcal U}$. For simplicial sets $X,Y\in \mathrm{sSet}_{\mathcal U}$, write $\mathrm{sSet}_{\mathcal U}(X,Y)$ for the set of simplicial maps $X\to Y$, $X\times Y$ for the cartesian product, and $Y^X$ for the exponential characterized by $(Y^X)_n=\mathrm{sSet}_{\mathcal U}(X\times \Delta^n,Y)$. For any simplicial set $K$, let $K^{\simeq}\subseteq K$ denote the largest Kan subcomplex of $K$ (the union of all Kan subcomplexes of $K$). Let $N_\Delta(-)$ denote the homotopy coherent nerve functor from simplicial categories to simplicial sets. We use the following notations:
% \begin{itemize}
% \item \hl{$\mathcal U$}
% \item \hl{$\mathrm{sSet}_{\mathcal U}$}
% \item \hl{$\Delta^n$}
% \item \hl{$\mathrm{sSet}_{\mathcal U}(X,Y)$}
% \item \hl{$X\times Y$}
% \item \hl{$Y^X$}
% \item \hl{$K^{\simeq}$}
% \item \hl{$N_\Delta(-)$}
% \end{itemize}
% \end{notation}

% \begin{definition}[Quasi-category]
% A simplicial set $C\in \mathrm{sSet}_{\mathcal U}$ is a \hldef{quasi-category} if it has fillers for all inner horns, i.e., for every $n\ge 2$ and $0<i<n$, every map $\Lambda_i^n\to C$ extends to a map $\Delta^n\to C$.
% \end{definition}

\begin{definition}[Simplicial category of quasi-categories] \label{definition:simplicial_category_of_quasi_categories}
    \TODO{read this and make it more precise}
Let $\mathbf{QCat}_{\mathcal U}^\Delta$ be the simplicial category defined as follows. Its objects are the $\mathcal U$-small quasi-categories. For objects $C,D$, the mapping space is the Kan complex
\[
\mathrm{Map}_{\mathbf{QCat}_{\mathcal U}^\Delta}(C,D)\coloneqq (D^C)^{\simeq},
\]
where $D^C$ is the exponential simplicial set (the functor simplicial set) and $(-)^{\simeq}$ denotes the largest Kan subcomplex. Composition
\[
\mathrm{Map}(C,D)\times \mathrm{Map}(D,E)\longrightarrow \mathrm{Map}(C,E)
\]
is induced by the exponential composition $E^D\times D^C\to E^C$, and identities by the units $C\to C$ under the exponential adjunction; these maps restrict to Kan subcomplexes and equip $\mathbf{QCat}_{\mathcal U}^\Delta$ with a simplicial category structure.
\end{definition}

\begin{definition}[Infinity category of infinity categories] \label{definition:infty_category_of_infty_categories}
    \TODO{reformulate this definition}
    Let $\calU$ be a Grothendieck universe.
    The \hldef{$\infty$-category of $\infty$-categories} (of $\mathcal U$-small quasi-categories) is the \CrefAndHyperrefIfExist{definition:infty_category_quasi_category}{quasi-category}
    $$\hlin{\mathrm{Cat}_\infty^{\mathcal U}\coloneqq N_\Delta\big(\mathbf{QCat}_{\mathcal U}^\Delta\big).}$$
    (\Cref{definition:simplicial_category_of_quasi_categories})
    Concretely, the objects of $\mathrm{Cat}_\infty^{\mathcal U}$ are $\mathcal U$-small quasi-categories, and for $C,D$ the mapping space in $\mathrm{Cat}_\infty^{\mathcal U}$ is equivalent to $(D^C)^{\simeq}$, the largest \CrefAndHyperrefIfExist{definition:kan_complex}{Kan} subcomplex of the \CrefAndHyperrefIfExist{definition:simplicial_category_of_simplicial_sets}{exponential simplicial set $D^C$}, so that $0$-simplices correspond to functors $C\to D$ and higher simplices encode their coherent homotopies.
\end{definition}







\begin{lemma} \label{lemma:n_simplicies_on_a_simplicial_set_are_naturally_isomorphic_to_morphisms_from_standard_n_simplex}
    % Let $\calC$ be a \CrefAndHyperrefIfExist{definition:category}{category}. 
    \begin{enumerate}

        \item Let $X$ be a \CrefAndHyperrefIfExist{definition:simplicial_cosimplicial_object_in_a_category}{simplicial set}. The \CrefAndHyperrefIfExist{definition:simplex_of_a_simplicial_object_in_a_category_and_face_and_degeneracy_maps}{$n$-simplicies} $X^n$ is naturally isomorphic to $\Hom_{\operatorname{Simp} \Sets}(\Delta^n, X)$\CrefIfExists{definition:standard_n_simplex}.

        \item Let $Y$ be a \CrefAndHyperrefIfExist{definition:simplicial_cosimplicial_object_in_a_category}{cosimplicial set}. The \CrefAndHyperrefIfExist{definition:simplex_of_a_simplicial_object_in_a_category_and_face_and_degeneracy_maps}{$n$-cosimplicies} $Y^n$ is naturally isomorphic to $\Hom_{\operatorname{Cosimp} \Sets}(\Hom_\Delta([n], -), Y)$\CrefIfExists{definition:standard_n_simplex}.
    \end{enumerate}
\end{lemma}

\begin{proof}
    This is a consequence of the \CrefAndHyperrefIfExist{theorem:enriched_yoneda_lemma}{Yoneda lemma}.
\end{proof}





\begin{definition} \label{definition:simplicial_cosimplicial_subset_of_a_simplicial_cosimplicial_set}
    \begin{enumerate}
        \item Let $X$ and $Y$ be \CrefAndHyperrefIfExist{definition:simplicial_cosimplicial_object_in_a_category}{simplicial sets}, that is, functors $X, Y : \Delta^{\mathrm{op}} \to \mathbf{Set}$.  
        A \hldef{simplicial subset of $X$} is a simplicial set $Y$ together with a family of \CrefAndHyperrefIfExist{definition:injective_surjective_bijective_map_of_sets}{injective} maps
        $$ \iota_n : Y_n \hookrightarrow X_n \quad (n \ge 0) $$
        that form a \CrefAndHyperrefIfExist{definition:natural_transformation_between_functors_between_categories}{natural transformation} $\iota : Y \Rightarrow X$.  
        Equivalently, $Y$ is a simplicial subset of $X$ if and only if for every morphism $\theta : [m] \to [n]$ in $\Delta$, the following diagram commutes:
        $$
        \begin{array}{ccc}
        Y_n & \xrightarrow{Y(\theta)} & Y_m \\
        \downarrow \iota_n & & \downarrow \iota_m \\
        X_n & \xrightarrow{X(\theta)} & X_m
        \end{array}
        $$
        and each $\iota_n$ is injective.

        \item Let $X$ and $Y$ be cosimplicial sets, that is, functors $X, Y : \Delta \to \mathbf{Set}$.  
        A \hldef{cosimplicial subset} of $X$ is a cosimplicial set $Y$ together with a family of injective maps
        $$
        \iota^n : Y^n \hookrightarrow X^n \quad (n \ge 0)
        $$
        that form a natural transformation $\iota : Y \Rightarrow X$.  
        Equivalently, $Y$ is a cosimplicial subset of $X$ if and only if for every morphism $\theta : [m] \to [n]$ in $\Delta$, the following diagram commutes:
        $$
        \begin{array}{ccc}
        Y^m & \xrightarrow{Y(\theta)} & Y^n \\
        \downarrow \iota^m & & \downarrow \iota^n \\
        X^m & \xrightarrow{X(\theta)} & X^n
        \end{array}
        $$
        and each $\iota^n$ is injective.
    \end{enumerate}
\end{definition}





\begin{definition} \label{definition:horn_of_the_nth_standard_simplex}
Let $n \ge 1$ be an integer and $\Delta[n]$ be the standard $n$-simplex.  
For each $i$ with $0 \le i \le n$, the \hldef{$i$th horn of the $n$th standard simplex}, denoted by \hl{$\Lambda^i[n]$}, is the simplicial subset of \CrefAndHyperrefIfExist{definition:standard_n_simplex}{$\Delta[n]$} defined as follows:
the set of \CrefAndHyperrefIfExist{definition:simplex_of_a_simplicial_object_in_a_category_and_face_and_degeneracy_maps}{$k$-simplices} of $\Lambda^i[n]$ is
$$ \Lambda^i[n]_k := \{\, \theta \in \mathrm{Hom}_{\Delta}([k], [n]) \mid \mathrm{Im}(\theta) \text{ does not contain all of } [n] \setminus \{i\} \,\}.  $$
Equivalently,
% is the union of all $(n-1)$-dimensional \CrefAndHyperrefIfExist{definition:simplex_of_a_simplicial_object_in_a_category_and_face_and_degeneracy_maps}{faces} of $\Delta[n]$ except the $i$th face, that is,
$$ \Lambda^i[n] = \bigcup_{j \ne i} \mathrm{Im}(d^j : \Delta[n-1] \to \Delta[n]), $$
where each $d^j$ is the \CrefAndHyperrefIfExist{definition:simplex_of_a_simplicial_object_in_a_category_and_face_and_degeneracy_maps}{coface map} in the simplex category corresponding to the injective order-preserving inclusion $[n-1] \hookrightarrow [n]$ that omits $j$.
\end{definition}


\begin{definition} \label{definition:nerve_of_a_category}
    Let $\calC$ be a \CrefAndHyperrefIfExist{definition:category}{(large) category}. The \hldef{nerve of $\calC$} is the \CrefAndHyperrefIfExist{definition:simplicial_cosimplicial_object_in_a_category}{simplicial ``class''} \hl{$N(\calC)$} given as follows: for $n \geq 0$, the \CrefAndHyperrefIfExist{definition:simplex_of_a_simplicial_object_in_a_category_and_face_and_degeneracy_maps}{$n$-simplicies} $N(\calC)_n$ is the class of all functors \CrefAndHyperrefIfExist{definition:simplex_category}{$[n] \to \calC$}. In other words, $N(\calC)_n$ is the class of all composable sequences
    $$ C_0 \stackrel{f_1}{\rightarrow} C_1 \ldots \stackrel{f_n}{\rightarrow} C_n$$
    of morphisms of length $n$. The \CrefAndHyperrefIfExist{definition:simplex_of_a_simplicial_object_in_a_category_and_face_and_degeneracy_maps}{face map} $d_i: N(\calC)_n \to N(\calC)_{n-1}$ carries the above sequence to 
    $$C_0 \stackrel{f_1}{\rightarrow} C_1 \ldots \stackrel{f_{i-1}}{\rightarrow} C_{i-1} \stackrel{ f_{i+1} \circ f_i }{\rightarrow} C_{i+1} \stackrel{f_{i+2}}{\rightarrow} \ldots \stackrel{f_{n}}{\rightarrow} C_n$$
    while the degeneracy $s_i$ carries it to
    $$C_0 \stackrel{f_1}{\rightarrow} C_1 \ldots \stackrel{f_i}{\rightarrow} C_i \stackrel{id_{C_i}}{\rightarrow} C_i \stackrel{f_{i+1}}{\rightarrow} C_{i+1} \stackrel{f_{i+2}}{\rightarrow} \ldots \stackrel{f_n}{\rightarrow} C_n.$$

    If $\calC$ is a \CrefAndHyperrefIfExist{definition:locally_small_category}{small category}, then $N(\calC)$ is a \CrefAndHyperrefIfExist{definition:simplicial_cosimplicial_object_in_a_category}{simplicial set}.

    Note that the category $\calC$ can be recovered, roughly up to isomoprhism, from its nerve $N(\calC)$. The objects of $\calC$ are simply the objects of $N(\calC)_0$. The morphisms $C_0 \to C_1$ are given by objects $\phi \in N(\calC)_1$ such that $d_1(\phi) = C_0$ and $d_0(\phi) = C_1$. The identity morphism $C \to C$ is given by the degenerate simplex $s_0(C)$. Moreover, given a diagram $C_0 \xrightarrow{\phi} C_1 \xrightarrow{\psi} C_2$, the object $N(\calC)_1$ corresponding to $\psi \circ \phi: C_0 \to C_2$ is uniquely characterized by the $2$-simplex $\sigma \in N(\calC)_2$ such that $d_2(\sigma) = \phi, d_0(\sigma) = \psi$, and $d_1(\sigma) = \psi \circ \phi$. 
\end{definition}


Simplicial categories provide one framework for $\infty$-categories.

\begin{definition} \label{definition:simplicial_category}
    A \hldef{simplicial category} is a \CrefAndHyperrefIfExist{definition:category_enriched_in_a_monoidal_category}{category enriched over} the \CrefAndHyperrefIfExist{definition:simplicial_cosimplicial_object_in_a_category}{category of simplicial sets}. Common notation for the category of simplicial categories include \hl{$\mathbf{Cat}_\Delta$} or \hl{$\mathbf{sCat}$}. However, note that these would overlap with \CrefAndHyperrefIfExist{definition:simplicial_cosimplicial_object_in_a_category}{notation for the category of simplicial objects of the category $\mathbf{Cat}$}. Further note that all simplicial categories are regardable as simplicial objects in $\mathbf{Cat}$, but not vice versa. Unless otherwise specified, the notations here will take precedence; in other words, $\mathbf{Cat}_\Delta$ or $\mathbf{sCat}$ will mean the caetgory of simplicial categories unless otherwise stated.
\end{definition}

\begin{definition} \label{definition:simplicial_category_of_simplicial_sets}
    The category $\mathbf{s}\Sets$ of \CrefAndHyperrefIfExist{definition:simplicial_cosimplicial_object_in_a_category}{simplicial sets} can be enriched with the structure of a \CrefAndHyperrefIfExist{definition:simplicial_category}{simplicial category} 
    (i.e. $\mathbf{s}\Sets$ can be regarded as a category \CrefAndHyperrefIfExist{definition:category_enriched_in_a_monoidal_category}{enriched} in $\mathbf{s}\Sets$) as follows: given simplicial sets $X$ and $Y$, we let $\operatorname{Mor}(X,Y)$, which is also often denoted by notations such as \hl{$Y^X$} and referred to as the \hldef{exponential simplicial set from $X$ to $Y$} or the \hldef{simplicial mapping object from $X$ to $Y$}
    $$\operatorname{Mor}(X,Y)_n = \Hom_{\mathbf{s}\Sets}(X \times \Delta^n, Y)$$
    \CrefIfExists{definition:simplex_of_a_simplicial_object_in_a_category_and_face_and_degeneracy_maps}\CrefIfExists{definition:standard_n_simplex} 
    \TODO{discuss what the face and degeneracy maps do}
    and we let composition be given by the \CrefAndHyperrefIfExist{definition:simplicial_cosimplicial_object_in_a_category}{simplicial set morphisms}
    $$\circ: \operatorname{Mor}(Y,Z) \times \operatorname{Mor}(X,Y) \to \operatorname{Mor}(X,Z)$$
    where $X,Y,Z$ are simplicial sets as follows:
    \begin{enumerate}
        \item The $n$-simplices of the product simplicial set $\operatorname{Mor}(Y,Z) \times \operatorname{Mor}(X,Y)$ is the product set $\operatorname{Mor}(Y,Z)_n \times \operatorname{Mor}(X,Y)_n$

        \item Since the $n$-simplices of $\operatorname{Mor}(Y,Z)$ and the $n$-simplicies of $\operatorname{Mor}(X,Y)$ are respectively given by simplicial set morphisms
        $$f: Y \times \Delta^n \to Z$$
        $$g: X \times \Delta^n \to Y$$
        respectively, the $n$-simplices of the product simplicial set are given by pairs $(f,g)$ of such morphisms.
        \item The composition $\circ(f,g)$ is the simplicial set morphism
        $$X \times \Delta^n \xrightarrow{\id_X \times \Delta} X \times (\Delta^n \times \Delta^n) \cong (X \times \Delta^n) \times \Delta^n \xrightarrow{g, \id_{\Delta^n}} Y \times \Delta^n \xrightarrow{f} Z$$
        where $\Delta: \Delta^n \to \Delta^n \times \Delta^n$ is the diagonal morphism of the standard $n$-simplex $\Delta^n$.
    \end{enumerate}

    
    This simplicial category may be called the \hldef{simplicial category of simplicial sets} and the same notations, such as \hl{$\mathbf{s}\mathbf{Sets}$} or \hl{$\mathbf{Sets}_\Delta$}, used to denote the (ordinary) category of simplicial sets may be used to denote this simplicial category.

    When the simplicial set $Y$ is \CrefAndHyperrefIfExist{definition:infty_category_quasi_category}{$\infty$-categories}, it is also customary to denote $\mathrm{Mor}(X,Y)$ by \hl{$\Fun(X,Y)$}, and to call this the \hldef{functor $\infty$-category from $X$ to $Y$}, especially when both $X$ and $Y$ are $\infty$-categories; it is itself an $\infty$-category. See also \Cref{definition:functor_between_infty_categories}.
\end{definition}


\begin{definition} \label{definition:mapping_space_between_objects_of_a_simplicial_set}
    Let $X$ be a \CrefAndHyperrefIfExist{definition:simplicial_cosimplicial_object_in_a_category}{simplicial set}. Let \CrefAndHyperrefIfExist{definition:categorical_terminology_for_simplices_in_a_simplicial_set_objects_morphisms_compositions_of_morphisms_in_a_simplicial_set}{$x,y \in X$ be objects}. The \hldef{mapping space between $x$ and $y$} is the simplicial set \hl{$\operatorname{Mor}(x,y)$} whose \CrefAndHyperrefIfExist{definition:simplex_of_a_simplicial_object_in_a_category_and_face_and_degeneracy_maps}{$n$-simplicies} are given by 
    $$\operatorname{Mor}(x,y)_n = \{F: \Delta^n \times \Delta^1 \to X: F|_{\Delta^n \times \{0\}} = x, F|_{\Delta^n \times \{1\}} = y \}.$$
    \CrefIfExists{definition:standard_n_simplex}
    \TODO{comment on how composition of morphisms becomes well defined up to homotopy; it seems reasonable that compositions to be well defined on the homotopy categories $\mathrm{h}\operatorname{Mor}$ of the mapping spaces }
\end{definition}


\begin{definition} \label{definition:simplicial_path_category_of_a_simplex_of_a_finite_nonempty_linearly_ordered_set}
    Let $J$ be a finite nonempty linearly ordered set. Define the \hldef{simplicial path category} \hl{$\mathfrak{C}[\Delta^J]$} of \CrefAndHyperrefIfExist{definition:standard_n_simplex}{$\Delta^J$} as the following \CrefAndHyperrefIfExist{definition:simplicial_category}{simplicial category}:
    \begin{itemize}
        \item The objects of $\mathfrak{C}[\Delta^J]$ are the elements of $J$.
        \item If $i,j \in J$, then 
        $$\Hom_{\mathfrak{C}[\Delta^J]}(i,j) = \begin{cases} \emptyset &\text{if } j < i \\ N(P_{i,j}) &\text{if } i \leq j \end{cases}$$
        \CrefIfExists{definition:nerve_of_a_category} where $P_{i,j}$ is the partially ordered set 
        $$P_{i,j} = \{I \subseteq J: (i,j \in I) \wedge (\forall k \in I) [i \leq k \leq j])\}.$$

        \item If $i_0 \leq i_1 \cdots i_n$, then the composition
        $$\Hom_{\mathfrak{C}[\Delta^J]}(i_0,i_1) \times \cdots \times \Hom_{\mathfrak{C}[\Delta^J]}(i_{n-1},i_n) \to \Hom_{\mathfrak{C}[\Delta^J]}(i_0,i_n)$$
        is induced by the map of partially ordered sets
        \begin{align*}
        P_{i_0,i_1} \times \ldots  \times P_{i_{n-1},i_n} &\rightarrow P_{i_0,i_n} \\
        ( I_1, \ldots, I_n ) &\mapsto I_1 \cup \ldots \cup I_n.
        \end{align*}
    \end{itemize}
    The functor $\mathfrak{C}: \Delta \to \mathbf{Cat}_\Delta$ (\Cref{definition:simplex_category}, \Cref{definition:simplicial_category}) extends uniquely (up to unique isomorphism) to a (small) colimit-preserving functor \hl{$\mathfrak{C}: \Sets_{\Delta} \to \mathbf{Cat}_\Delta$} (\Cref{definition:simplicial_cosimplicial_object_in_a_category}, \Cref{lemma:functor_from_category_of_simplicial_sets_to_category_of_simplicial_categories})
\end{definition}

\begin{definition} \label{definition:simplicial_nerve_of_a_simplicial_category}
    Let $\mathcal{C}$ be a \CrefAndHyperref{definition:simplicial_category}{simplicial category}. The \hldef{(simplicial) nerve} \hl{$N(\mathcal{C})$} of $\calC$ is the \CrefAndHyperrefIfExist{definition:simplicial_cosimplicial_object_in_a_category}{simplicial set} described by the formula
    $$ N(\calC)_{n} \cong Hom_{ \Sets_\Delta}( \Delta^n, N(\mathcal{C})) = Hom_{\mathcal{C}at_{\Delta}}( \mathfrak{C}[ \Delta^n ], \mathcal{C}).$$
    (\Cref{lemma:n_simplicies_on_a_simplicial_set_are_naturally_isomorphic_to_morphisms_from_standard_n_simplex}, \Cref{definition:simplicial_cosimplicial_object_in_a_category}, \Cref{definition:simplex_of_a_simplicial_object_in_a_category_and_face_and_degeneracy_maps}, \Cref{definition:simplicial_category}, \Cref{definition:simplicial_path_category_of_a_simplex_of_a_finite_nonempty_linearly_ordered_set}) 
    It is synonymously called the \hldef{homotopy coherent nerve of $\calC$}.
\end{definition}

\begin{definition} \label{definition:infty_category_of_spaces}
    Let $Kan$ be the \CrefAndHyperrefIfExist{definition:full_subcategory_of_a_category}{full subcategory} of \CrefAndHyperrefIfExist{definition:simplicial_cosimplicial_object_in_a_category}{$\mathbf{s}\Sets$} spanned by the collection of \CrefAndHyperrefIfExist{definition:kan_complex}{Kan complexes}. Regard $Kan$ as a \CrefAndHyperrefIfExist{definition:simplicial_category}{simplicial category}. \TODO{how exactly is $Kan$ regarded as a simplicial category?} 
    In the context of $\infty$-category theory, the \CrefAndHyperrefIfExist{definition:simplicial_nerve_of_a_simplicial_category}{simplicial nerve} of $Kan$ is called the \hldef{$\infty$-category of spaces}.
\end{definition}




\subsection{Model categories}

\begin{definition}[Model Category] \label{definition:model_category}
    \TODO{I don't like the axioms as stated here}
A \hldef{model category}, or synonymously a \hldef{closed model category}, is a \CrefAndHyperrefIfExist{definition:complete_and_cocomplete_category}{complete and cocomplete category} \(\mathcal{M}\) equipped with three distinguished classes of morphisms:

\begin{itemize}
  \item \hldef{Weak equivalences} \(\mathcal{W}\),
  \item \hldef{Fibrations} \(\mathcal{F}\),
  \item \hldef{Cofibrations} \(\mathcal{C}\),
\end{itemize}

\begin{enumerate}
  \item \textbf{(Two-out-of-three)}: For any composable morphisms \(f: X \to Y\), \(g: Y \to Z\), if any two of \(f\), \(g\), or \(g \circ f\) lie in \(\mathcal{W}\), then so does the third.

  \item \textbf{(Retracts)}: Each of the classes \(\mathcal{W}, \mathcal{F}, \mathcal{C}\) is closed under retracts in the arrow category \(\mathcal{M}^2\). That is, if \(f\) is a retract of \(g\) and \(g\) belongs to one of these classes, then \(f\) also belongs to that class.

  \item \textbf{(Lifting)}: Given any commutative square
  \[
  \begin{tikzcd}
  A \arrow[r] \arrow[d, "i"'] & X \arrow[d, "p"] \\
  B \arrow[r] & Y
  \end{tikzcd}
  \]
  where \(i \in \mathcal{C}\) and \(p \in \mathcal{F}\), a diagonal filler (lift) exists making both triangles commute provided either
  \begin{itemize}
    \item \(i\) is also a weak equivalence (called an acyclic cofibration), or
    \item \(p\) is also a weak equivalence (called an acyclic fibration).
  \end{itemize}

  Formally, acyclic cofibrations have the left lifting property with respect to all fibrations, and cofibrations have the left lifting property with respect to all acyclic fibrations.

  \item \textbf{(Factorization)}: Every morphism \(f : X \to Y\) in \(\mathcal{M}\) admits two functorial factorizations:
  \begin{itemize}
    \item \(f = p \circ i\), where \(i \in \mathcal{C}\) is a cofibration and \(p \in \mathcal{F} \cap \mathcal{W}\) is an acyclic fibration.
    \item \(f = q \circ j\), where \(j \in \mathcal{C} \cap \mathcal{W}\) is an acyclic cofibration and \(q \in \mathcal{F}\) is a fibration.
  \end{itemize}
  
\end{enumerate}

Here, an \hldef{acyclic fibration} (or \hldef{trivial fibration}) is a morphism in \(\mathcal{F} \cap \mathcal{W}\), and an \hldef{acyclic cofibration} (or \hldef{trivial cofibration}) is a morphism in \(\mathcal{C} \cap \mathcal{W}\).
\end{definition}

\begin{definition}[Proper Model Category] \label{definition:proper_model_category}
Let $\mathcal{M}$ be a \CrefAndHyperrefIfExist{definition:model_category}{model category} with classes of weak equivalences $\mathcal{W}$, cofibrations $\mathcal{C}of$, and fibrations $\mathcal{F}ib$. 
\TODO{pushout, pullback}
\begin{enumerate}
    \item Then $\mathcal{M}$ is called \hldef{left proper} if weak equivalences are preserved under pushouts along cofibrations, i.e., if for every pushout diagram
    $$
    \begin{array}{ccc}
    A & \xrightarrow{i} & B \\
    \downarrow^{f \in \mathcal{W}} && \downarrow \\
    C & \xrightarrow{} & D
    \end{array}
    $$
    with $i \in \mathcal{C}of$, the induced map $C \to D$ lies in $\mathcal{W}$. 
    \item Dually, $\mathcal{M}$ is called \hldef{right proper} if weak equivalences are preserved under pullbacks along fibrations, i.e., if for every pullback diagram
    $$
    \begin{array}{ccc}
    P & \xrightarrow{} & X \\
    \downarrow^{g} && \downarrow^{p \in \mathcal{F}ib} \\
    Y & \xrightarrow{h \in \mathcal{W}} & Z
    \end{array}
    $$
    the induced map $P \to Y$ is in $\mathcal{W}$.

    \item A model category $\mathcal{M}$ is \hldef{proper} if it is both left proper and right proper.
\end{enumerate}
\end{definition}


\begin{lemma} \label{lemma:model_category_has_an_initial_object_and_a_final_object}
    A \CrefAndHyperrefIfExist{definition:model_category}{model category} always has a \CrefAndHyperrefIfExist{definition:initial_final_zero_objects_of_a_category}{initial object and a final object}.
\end{lemma}

\begin{proof}
    This is because a model category, by definition, is \CrefAndHyperrefIfExist{definition:complete_and_cocomplete_category}{complete and cocomplete}.
\end{proof}


\begin{definition}[Fibrant Object] \label{definition:fibrant_cofibrant_object_in_a_model_category}
Let $(\mathcal{C}, \mathcal{W}, \mathcal{C}of, \mathcal{F}ib)$ be a \CrefAndHyperrefIfExist{definition:model_category}{model category}. 
\begin{enumerate}
    \item An object $X \in \mathcal{C}$ is called \hldef{fibrant} if the unique morphism
    $$ X \longrightarrow * $$
    to the \CrefAndHyperrefIfExist{definition:initial_final_zero_objects_of_a_category}{terminal object} (\Cref{lemma:model_category_has_an_initial_object_and_a_final_object}) $*$ of $\mathcal{C}$ is a fibration (i.e., belongs to $\mathcal{F}ib$).

    \item An object $X \in \mathcal{C}$ is called \hldef{cofibrant} if the unique morphism
    $$
    \emptyset \longrightarrow X
    $$
    from the \CrefAndHyperrefIfExist{definition:initial_final_zero_objects_of_a_category}{initial object} (\Cref{lemma:model_category_has_an_initial_object_and_a_final_object}) $\emptyset$ of $\mathcal{C}$ is a cofibration (i.e., belongs to $\mathcal{C}of$).
\end{enumerate}
\end{definition}

\begin{definition}[Standard model category structure on simplicial sets] \label{definition:standard_model_structure_on_the_category_of_simplicial_sets}
    Let $\mathsf{sSet}$ denote the category of \CrefAndHyperrefIfExist{definition:simplicial_cosimplicial_object_in_a_category}{simplicial sets}. The \hldef{standard model category structure on $\mathsf{sSet}$}, also known as the \hldef{Kan model structure} or the \hldef{Quillen model structure} or the \hldef{Kan-Quillen model structure}, is the \CrefAndHyperrefIfExist{definition:model_category}{model category} given by specifying three classes of morphisms:

    \begin{itemize}
    \item \textbf{Weak equivalences:} A map $f : X \to Y$ in $\mathsf{sSet}$ is a weak equivalence, also called a \hldef{weak homotopy equivalence between the simplicial sets $X$ and $Y$}, if the induced map of \CrefAndHyperrefIfExist{definition:geometric_realization_of_a_simplicial_set}{geometric realizations} $|f| : |X| \to |Y|$ is a \CrefAndHyperrefIfExist{definition:weak_homotopy_equivalence_of_topological_spaces}{weak homotopy equivalence} of topological spaces (i.e., induces isomorphisms on all homotopy groups for all choices of basepoint).
    \item \textbf{Fibrations:} A map $f : X \to Y$ is a \hldef{(Kan) fibration} if it has the right lifting property with respect to the horn inclusions $\Lambda_k^n \hookrightarrow \Delta^n$ for all $n \ge 1$ and $0 \le k \le n$.
    \item \textbf{Cofibrations:} A map $f : X \to Y$ is a cofibration if it is a \CrefAndHyperrefIfExist{definition:monomorphism_and_epimorphism_in_categories}{monomorphism} (i.e., injective at each level).
    \end{itemize}
    For this model category structure, all objects are \CrefAndHyperrefIfExist{definition:fibrant_cofibrant_object_in_a_model_category}{cofibrant}, and the \CrefAndHyperrefIfExist{definition:fibrant_cofibrant_object_in_a_model_category}{fibrant objects} are exactly the \CrefAndHyperrefIfExist{definition:kan_complex}{Kan complexes}.
    % With these choices, $\mathsf{sSet}$ is a model category in the sense of Quillen. All objects are cofibrant, and the fibrant objects are exactly the Kan complexes.
\end{definition}



\begin{definition}[Model Structure on $\mathcal{T}op$] \label{definition:quillen_serre_model_structure_on_the_category_of_topological_spaces}
The category $\Top$ of topological spaces has a \CrefAndHyperrefIfExist{definition:model_category}{model structure}, referred to such names as the \hldef{Quillen model structure (on the category of topological spaces)}, \hldef{classical model structure on the category of topological spaces}, \hldef{Serre model structure}, or \hldef{Quillen-Serre model structure}, given by the following three classes of morphisms
$$ \mathcal{W}, \quad \mathcal{C}of, \quad \mathcal{F}ib $$
such that:
\begin{itemize}
    \item $\mathcal{W}$ is the class of \CrefAndHyperrefIfExist{definition:weak_homotopy_equivalence_of_topological_spaces}{weak homotopy equivalences}.
    \TODO{cofibrations, Serre fibrations of topological spaces}
    \item $\mathcal{C}of$ is the class of cofibrations: maps that are retracts of relative cell complexes (e.g., inclusions satisfying the homotopy extension property).
    \item $\mathcal{F}ib$ is the class of Serre fibrations: maps having the right lifting property with respect to all inclusions of disks into disks times the interval.
\end{itemize}
% The triple $(\mathcal{T}op, \mathcal{W}, \mathcal{C}of, \mathcal{F}ib)$ satisfies the axioms of a model category, providing a framework for homotopy theory in topological spaces.
\end{definition}



\subsubsection{Homotopy categories of model categories}

% \begin{definition}[Homotopy Category of a Model Category] \label{definition:homotopy_category_of_a_model_category}
% The \hldef{homotopy category} \(\mathrm{Ho}(\mathcal{M})\) of a \CrefAndHyperrefIfExist{definition:model_category}{model category} \(\mathcal{M}\) is the localization \TODO{need to make this localization precise.}
% \[ \mathrm{Ho}(\mathcal{M}) = \mathcal{M}[\mathcal{W}^{-1}] \]
% where the morphisms are formally inverted weak equivalences.
% \end{definition}

\begin{definition}[Homotopy category of a model category] \label{definition:homotopy_category_of_a_model_category}
    Let $\mathcal{M}$ be a \CrefAndHyperrefIfExist{definition:model_category}{model category}. The \hldef{homotopy category of $\mathcal{M}$}, denoted by notations such as \hl{$\mathrm{Ho}(\mathcal{M})$}, \hl{$\mathrm{h}(\calM)$}, etc. , is the \CrefAndHyperrefIfExist{definition:category}{category} $\calM[\calW^{-1}]$ whose objects are those of $\mathcal{M}$, and whose morphisms are equivalence classes of morphisms in $\mathcal{M}$ under the relation of left and right homotopy, \CrefAndHyperrefIfExist{definition:localization_of_a_category_by_a_multiplicative_system}{localized} at the weak equivalences. Explicitly,
    \begin{itemize}
        \item The objects of $\mathrm{Ho}(\mathcal{M})$ are the same as those in $\mathcal{M}$.
        \item For objects $X,Y$ in $\mathcal{M}$, the morphism set $\operatorname{Hom}_{\mathrm{Ho}(\mathcal{M})}(X,Y)$ consists of maps in $\mathcal{M}$ modulo homotopy, with weak equivalences formally inverted.
    \end{itemize}
\end{definition}


\begin{definition}[Homotopy category of the category of simplicial sets] \label{definition:homotopy_category_of_the_category_of_simplicial_sets}
    Let $\mathbf{s}\Sets$ denote the category of \CrefAndHyperrefIfExist{definition:simplicial_cosimplicial_object_in_a_category}{simplicial sets}, equipped with its \CrefAndHyperrefIfExist{definition:standard_model_structure_on_the_category_of_simplicial_sets}{standard model structure}. The \hldef{homotopy category of (the category of) simplicial sets}, denoted by notations such as \hl{$\mathrm{Ho}(\mathbf{s}\Sets)$}, \hl{$\mathrm{h}(\Sets_\Delta)$}, etc., is the \CrefAndHyperrefIfExist{definition:homotopy_category_of_a_model_category}{homotopy category} of the \CrefAndHyperrefIfExist{definition:standard_model_structure_on_the_category_of_simplicial_sets}{standard model category of simplicial sets}. 
    In other words, It is the category whose objects are simplicial sets, and whose morphisms are homotopy classes of maps between simplicial sets, with weak equivalences inverted. That is,
    \begin{itemize}
        \item Objects: simplicial sets $X$.
        \item Morphisms: equivalence classes of maps $f: X \to Y$ under simplicial homotopy, where weak equivalences (maps inducing isomorphism on all homotopy groups) are made invertible.
    \end{itemize}
    % Up to equivalence, $\mathrm{Ho}(\mathsf{sSet})$ models the classical homotopy category of topological spaces.
    The homotopy category of the category of simplicial sets is \CrefAndHyperrefIfExist{definition:equivalence_of_categories}{equivalent} to the 
    category obtained from the category $\calC\calG$ of \CrefAndHyperrefIfExist{definition:compactly_generated_topological_space}{compactly generated}, \CrefAndHyperrefIfExist{definition:weakly_hausdorff_topological_space}{weakly Hausdorff} topological spaces by \CrefAndHyperrefIfExist{definition:localization_of_a_category_by_a_multiplicative_system}{inverting} \CrefAndHyperrefIfExist{definition:weak_homotopy_equivalence_of_topological_spaces}{weak homotopy equivalences} of spaces (\Cref{corollary:homotopy_category_of_simplicial_sets_is_equivalent_to_the_homotopy_category_of_compactly_generated_weakly_hausdorff_spaces}).


    The homotopy category of simplicial sets should not be confused with the \CrefAndHyperrefIfExist{definition:homotopy_category_of_a_simplicial_set}{homotopy category of a simplicial set}.
\end{definition}

\begin{corollary} \label{corollary:homotopy_category_of_simplicial_sets_is_equivalent_to_the_homotopy_category_of_compactly_generated_weakly_hausdorff_spaces}
    The category obtained from the category $\calC\calG$ of \CrefAndHyperrefIfExist{definition:compactly_generated_topological_space}{compactly generated}, \CrefAndHyperrefIfExist{definition:weakly_hausdorff_topological_space}{weakly Hausdorff} topological spaces by \CrefAndHyperrefIfExist{definition:localization_of_a_category_by_a_multiplicative_system}{inverting} \CrefAndHyperrefIfExist{definition:weak_homotopy_equivalence_of_topological_spaces}{weak homotopy equivalences} of spaces  is \CrefAndHyperrefIfExist{definition:equivalence_of_categories}{equivalent} to the \CrefAndHyperrefIfExist{definition:homotopy_category_of_the_category_of_simplicial_sets}{homotopy category of the category of simplicial sets}. 
\end{corollary}
\begin{proof}
    This follows from \Cref{theorem:unit_and_counit_morphisms_for_geometric_realization_singular_complex_adjunction_are_weak_homotopy_equivalences_for_compactly_generated_spaces_and_simplicial_sets}.
\end{proof}


\begin{definition}[{\cite[Before Definition 1.1.4.4]{lurie_htt}}] \label{definition:homotopy_category_of_a_simplicial_category}
    Let $\calC$ be a \CrefAndHyperrefIfExist{definition:simplicial_category}{simplicial category}. Let $\calH$ be the \CrefAndHyperrefIfExist{definition:homotopy_category_of_the_category_of_simplicial_sets}{homotopy category of the category of simplicial sets}. The \hldef{homotopy category of $\calC$} is the \CrefAndHyperrefIfExist{definition:category_enriched_in_a_monoidal_category}{$\calH$-enriched category} \hl{$\mathrm{h}\calC$} obtained by applying the localization functor $\mathrm{h}: \Sets_\Delta \to \calH$ \CrefIfExists{definition:simplicial_cosimplicial_object_in_a_category} to the Hom's of $\calC$. In other words, $\mathrm{h}\calC$ is the $\calH$-enriched category category whose 
    \begin{enumerate}
        \item Objects are the objects of $\calC$
        \item Hom's are given for $X, Y \in \Ob \mathrm{h} \calC$ by
        $$\Hom_{\mathrm{h}\calC}(X,Y) = \mathrm{h} \Hom_{\calC}(X,Y).$$
    \end{enumerate}
    
    \TODO{homotopy category of the cateogry of simplicial sets, equivalently the homotopy category of the category of compactly generated spaces}
\end{definition}


\begin{definition} \label{definition:homotopy_category_of_a_simplicial_set}
    Let $S$ be a \CrefAndHyperrefIfExist{definition:simplicial_cosimplicial_object_in_a_category}{simplicial set}. Its \hldef{homotopy category} \hl{$\mathrm{h}S$} is defined to be the \CrefAndHyperrefIfExist{definition:homotopy_category_of_a_simplicial_category}{homotopy category} $h\mathfrak{C}[S]$ of the \CrefAndHyperrefIfExist{definition:simplicial_category}{simplicial category} \CrefAndHyperrefIfExist{definition:simplicial_path_category_of_a_simplex_of_a_finite_nonempty_linearly_ordered_set}{$\mathfrak{C}[S]$}.

    The homotopy category of a simplicial set should not be confused with the \CrefAndHyperrefIfExist{definition:homotopy_category_of_the_category_of_simplicial_sets}{homotopy category of the category of simplicial sets}.
\end{definition}

\begin{proposition}[{\cite[Proposition 1.2.3.1]{lurie_htt}}]
    \TODO{}
    There is an \CrefAndHyperrefIfExist{definition:adjoint_functors_between_categories_unit_counit_of_adjoint_functors}{adjunction} $N \dashv h$ between the \CrefAndHyperrefIfExist{definition:nerve_of_a_category}{functor of nerves of a small categories} and the \CrefAndHyperrefIfExist{definition:homotopy_category_of_a_simplicial_set}{functor of homotopy categories} of \CrefAndHyperrefIfExist{definition:simplicial_cosimplicial_object_in_a_category}{simplicial sets}
    $$N: \mathbf{Cat} \leftrightarrows \Sets_\Delta: h$$
    (\Cref{definition:simplicial_cosimplicial_object_in_a_category}) (where we ignore that \CrefAndHyperrefIfExist{definition:category_enriched_in_a_monoidal_category}{enriched category structure} of $hS$ for \CrefAndHyperrefIfExist{definition:simplicial_cosimplicial_object_in_a_category}{simplicial sets} $S$)
\end{proposition}




\begin{definition}[Joyal Model Structure, {\cite[Theorem 2.2.5.1]{lurie_htt}}] \label{definition:joyal_model_structure_on_the_category_of_simplicial_sets}
  \TODO{This description seems a bit incomplete; it should be necessary to say that the model structure is the left proper combinatorial model structure with these properties.}
The \hldef{Joyal model structure} on the category $s\mathbf{Set}$ of \CrefAndHyperrefIfExist{definition:simplicial_cosimplicial_object_in_a_category}{simplicial sets} is characterized by:
    \TODO{fibrant/cofibrant objects}
\begin{itemize}
  \item \CrefAndHyperrefIfExist{definition:fibrant_cofibrant_object_in_a_model_category}{Fibrant objects} are exactly the \CrefAndHyperrefIfExist{definition:infty_category_quasi_category}{quasi-categories},
  \item Cofibrations are the monomorphisms,
  \item Weak equivalences are categorical equivalences, i.e., maps inducing equivalences of \(\infty\)-categories.
\end{itemize}

This model category satisfies the axioms of a \CrefAndHyperrefIfExist{definition:model_category}{model category} and presents the homotopy theory of \((\infty,1)\)-categories.
\end{definition}

\begin{definition}[Model for the Homotopy Theory of \((\infty,1)\)-Categories] \label{definition:model_for_the_homotopy_theory_of_infty_1_categories}
A \CrefAndHyperrefIfExist{definition:model_category}{model category} \(\mathcal{M}\) is called a \hldef{model for the homotopy theory of $(\infty,1)$-categories} if:
\begin{itemize}
  \item The fibrant objects of \(\mathcal{M}\) are equivalent (under suitable equivalences) to \((\infty,1)\)-categories,
  \item The weak equivalences correspond to equivalences of \((\infty,1)\)-categories,
  \item \(\mathcal{M}\) is Quillen equivalent to other established models of \(\infty\)-categories (e.g., \CrefAndHyperrefIfExist{definition:simplicial_category}{simplicial categories}, Segal categories, complete Segal spaces).
  \TODO{Quillen equivalent}
\end{itemize}
\end{definition}


\subsection{Quasi-categories}

Quasi-categories provide one framework for $\infty$-categories.

\begin{definition} \label{definition:kan_extension_condition_for_a_horn_for_a_simplicial_set}
    Let $K$ be a \CrefAndHyperrefIfExist{definition:simplicial_cosimplicial_object_in_a_category}{simplicial set}. We may say that $K$ \hldef{satisfies the Kan condition/extension condition} for the \CrefAndHyperrefIfExist{definition:horn_of_the_nth_standard_simplex}{horn} $\Lambda_i^n \subset \Delta^n$ (\CrefIfExists{definition:standard_n_simplex}) if any morphism $\lambda_i^n \to K$ of simplicial sets admits an extension $f: \Delta^n \to K$. 
\end{definition}

\begin{definition} \label{definition:infty_category_quasi_category}
    A \hldef{quasi-category} is a \CrefAndHyperrefIfExist{definition:simplicial_cosimplicial_object_in_a_category}{simplicial set} $K$ satisfying the \CrefAndHyperrefIfExist{definition:kan_extension_condition_for_a_horn_for_a_simplicial_set}{Kan extension condition} for the horns $\Lambda_i^n \subset \Delta^n$ for all $n$ and all $0 < i < n$. More explicitly, for any $0 < i < n$, any map $f_0: \Lambda_i^n \to K$ \CrefIfExists{definition:horn_of_the_nth_standard_simplex} admits an extension $f: \Delta^n \to K$ \CrefIfExists{definition:standard_n_simplex}.

    In $\infty$-category theory, a quasi-category is synonymously referred to as an \hldef{$(\infty, 1)$-category}. Moreover, an \hldef{$\infty$-category}, without further qualifications, often refers to a quasi-category.
\end{definition}

A basic class of \CrefAndHyperrefIfExist{definition:infty_category_quasi_category}{quasi-categories} would be \CrefAndHyperrefIfExist{definition:nerve_of_a_category}{nerves of categories} (\Cref{proposition:simplicial_set_is_the_nerve_of_a_small_category_if_and_only_if_maps_from_inner_horns_lift_uniquely}).

\begin{proposition}[see {\cite[Proposition 1.1.2.2]{lurie_htt}}] \label{proposition:simplicial_set_is_the_nerve_of_a_small_category_if_and_only_if_maps_from_inner_horns_lift_uniquely}
        Let $K$ be a \CrefAndHyperrefIfExist{definition:simplicial_cosimplicial_object_in_a_category}{simplicial set}. Then the following conditions are equivalent:

    \begin{itemize}

    \item There exists a \CrefAndHyperrefIfExist{definition:locally_small_category}{small category} $\mathcal{C}$ and an isomorphism \CrefAndHyperrefIfExist{definition:nerve_of_a_category}{$K \simeq N(\mathcal{C})$}.

    \item For each $0 < i < n$ and each diagram
    $$ \xymatrix{ \Lambda^n_i \ar@{^{(}->}[d] \ar[r] & K \\
    \Delta^n \ar@{-->}[ur], & \\}$$
    there exists a {\em unique} dotted arrow rendering the diagram commutative. 
    \end{itemize}
\end{proposition}

\begin{corollary} \label{corollary:the_nerve_of_a_small_category_is_a_quasi_category}
    The \CrefAndHyperrefIfExist{definition:nerve_of_a_category}{nerve $N(\calC)$} of any \CrefAndHyperrefIfExist{definition:locally_small_category}{small category} is a \CrefAndHyperrefIfExist{definition:infty_category_quasi_category}{quasi-category}.
\end{corollary}

\begin{proof}
    This is immediate by the definition of a quasi-category and \Cref{proposition:simplicial_set_is_the_nerve_of_a_small_category_if_and_only_if_maps_from_inner_horns_lift_uniquely}.
\end{proof}



\begin{definition} \label{definition:infty_groupoid_infty_0_category}
    An \hldef{$\infty$-groupoid} is an \CrefAndHyperrefIfExist{definition:infty_category_quasi_category}{$\infty$-category} whose \CrefAndHyperrefIfExist{definition:homotopy_category_of_a_simplicial_set}{homotopy category} $\mathrm{h} \calC$ is a groupoid, i.e. every morphism in $\calC$ is an equivalence.
    \TODO{objects and morphisms, equivalence in an $\infty$-category, which is a simplicial set}
    An $\infty$-groupoid is also often referred to as an \hldef{$(\infty, 0)$-category}.

    $\infty$-groupoids are equivalent to \CrefAndHyperrefIfExist{definition:kan_complex}{Kan-complexes} (\Cref{proposition:infty_groupoids_are_equivalent_to_kan_complexes}).

    An $\infty$-groupoid may be referred to as an \hldef{anima} (in the Clausen-Scholze termionology), especially if the $\infty$-groupoid is of the form \CrefAndHyperrefIfExist{definition:mapping_space_between_objects_of_a_simplicial_set}{$\mathrm{Mor}_\calC(x,y)$} for an \CrefAndHyperrefIfExist{definition:infty_category_quasi_category}{$\infty$-category} $\calC$ and \CrefAndHyperrefIfExist{definition:categorical_terminology_for_simplices_in_a_simplicial_set_objects_morphisms_compositions_of_morphisms_in_a_simplicial_set}{objects}/\CrefAndHyperrefIfExist{definition:simplex_of_a_simplicial_object_in_a_category_and_face_and_degeneracy_maps}{$0$-simplicies} $x,y$ of $\calC$, cf. \Cref{proposition:mapping_space_between_objects_of_a_quasi_category_is_an_infty_groupoid}.
\end{definition}



\begin{definition} \label{definition:kan_complex}
    A \hldef{Kan complex} or a \hldef{Kan simplicial set} is a \CrefAndHyperrefIfExist{definition:simplicial_cosimplicial_object_in_a_category}{simplicial set} $K$ satisfying the \CrefAndHyperrefIfExist{definition:kan_extension_condition_for_a_horn_for_a_simplicial_set}{Kan extension condition} for the horns $\Lambda_i^n \subset \Delta^n$ for all $n$ and all $0 \leq i < n$. More explicitly, for any $0 \leq i \leq n$, any map $f_0: \Lambda_i^n \to K$ \CrefIfExists{definition:horn_of_the_nth_standard_simplex} admits an extension $f: \Delta^n \to K$ \CrefIfExists{definition:standard_n_simplex}.

    Kan-complexes are equivalent to \CrefAndHyperrefIfExist{definition:infty_groupoid_infty_0_category}{$\infty$-groupoids} (\Cref{proposition:infty_groupoids_are_equivalent_to_kan_complexes}).

    % A \hldef{quasi-category}, or synonymously an \hldef{$\infty$-category}, depending on the context, is a \CrefAndHyperrefIfExist{definition:simplicial_cosimplicial_object_in_a_category}{simplicial set} $K$ satisfying the \CrefAndHyperrefIfExist{definition:kan_extension_condition_for_a_horn_for_a_simplicial_set}{Kan extension condition} for the horns $\Lambda_i^n \subset \Delta^n$ for all $n$ and all $0 < i < n$. More explicitly, for any $0 < i < n$, any map $f_0: \Lambda_i^n \to K$ \CrefIfExists{definition:horn_of_the_nth_standard_simplex} admits an extension $f: \Delta^n \to K$ \CrefIfExists{definition:standard_n_simplex}.
\end{definition}

\begin{lemma} \label{lemma:every_kan_complex_is_a_quasi_category}
    Every \CrefAndHyperrefIfExist{definition:kan_complex}{Kan complex} is a \CrefAndHyperrefIfExist{definition:infty_category_quasi_category}{quasi-category}.
\end{lemma}

\begin{proof}
    This is immediate by definition.
\end{proof}

\begin{proposition}[{\cite[Corollary 1.4]{joyal_qckc}}, cf. {\cite[Proposition 1.2.5.1]{lurie_htt}}] \label{proposition:infty_groupoids_are_equivalent_to_kan_complexes}
    Let $\calC$ be a \CrefAndHyperrefIfExist{definition:simplicial_cosimplicial_object_in_a_category}{simplicial set}. The following are equivalent:
    \begin{enumerate}
        \item $\calC$ is an \CrefAndHyperrefIfExist{definition:infty_groupoid_infty_0_category}{$\infty$-groupoid}.
        \item $\calC$ satisfies by \CrefAndHyperrefIfExist{definition:kan_extension_condition_for_a_horn_for_a_simplicial_set}{Kan extension condition} for all \CrefAndHyperrefIfExist{definition:horn_of_the_nth_standard_simplex}{horns} $\Lambda_i^n \subseteq \Delta^n$\CrefIfExists{definition:standard_n_simplex} for all $0 \leq i < n$.
        \item $\calC$ satisfies by \CrefAndHyperrefIfExist{definition:kan_extension_condition_for_a_horn_for_a_simplicial_set}{Kan extension condition} for all \CrefAndHyperrefIfExist{definition:horn_of_the_nth_standard_simplex}{horns} $\Lambda_i^n \subseteq \Delta^n$\CrefIfExists{definition:standard_n_simplex} for all $0 < i \leq n$.
        \item $\calC$ is a \CrefAndHyperrefIfExist{definition:kan_complex}{Kan complex}.
    \end{enumerate}
\end{proposition}

\begin{proposition} \label{proposition:mapping_space_between_objects_of_a_quasi_category_is_an_infty_groupoid}
    Let $X$ be a \CrefAndHyperrefIfExist{definition:infty_category_quasi_category}{quasi-category} and let $x,y \in X$ be \CrefAndHyperrefIfExist{definition:categorical_terminology_for_simplices_in_a_simplicial_set_objects_morphisms_compositions_of_morphisms_in_a_simplicial_set}{objects}. The \CrefAndHyperrefIfExist{definition:mapping_space_between_objects_of_a_simplicial_set}{mapping space $\mathrm{Mor}(x,y)$} is an \CrefAndHyperrefIfExist{definition:infty_groupoid_infty_0_category}{$\infty$-groupoid}.
\end{proposition}



\section{Higher Category Theory}

\subsection{$n$-categories}

\begin{notation}
Let $\mathbf{Cat}$ denote the category of (small) categories and functors between them.  
For integers $n \ge 0$, we define inductively the category of $n$-categories.  
% We denote by $\hl{$n\text{-}\mathbf{Cat}$}$ the category (or higher category) of all strict $n$-categories, and by $\hl{$\infty\text{-}\mathbf{Cat}$}$ the category (or higher category) of all $\infty$-categories.  
\end{notation}

\begin{definition} \label{definition:strict_n_category}
A \hldef{strict $0$-category} is a set.  
A \hldef{strict $(n+1)$-category} is a \CrefAndHyperrefIfExist{definition:category_enriched_in_a_monoidal_category}{category enriched in} the category of strict $n$-categories.  
Explicitly, a strict $(n+1)$-category $\mathcal{C}$ consists of:
\begin{itemize}
  \item a collection of objects $\mathrm{Ob}(\mathcal{C})$; such an object is called a \hldef{$0$-cell} or synonymously a \hldef{$0$-morphism},
  \item for each pair of objects $x, y \in \mathrm{Ob}(\mathcal{C})$, a strict $n$-category $\mathrm{Hom}_{\mathcal{C}}(x, y)$; inductively, this strict $n$-category itself contains \hldef{$k$-cells}, or synonymously \hldef{$k$-morphisms}, for $0 \leq k \leq n$; the $k$-cells of $\mathrm{Hom}_{\mathcal{C}}(x, y)$ are exactly the $k+1$-cells of $\calC$.
  \begin{itemize}
    \item For each object/$0$-cell $x$ of $\calC$, there is an \hldef{identity} \hl{$\id_x \in \Hom_\calC(x,x)$}, which is an $n+1$ cell of $\calC$ and or equivalently an $n$ cell of $\Hom_\calC(x,x)$. 
  \end{itemize}
    
  \item composition functors 
  $$\hlin{\circ_{x,y,z}: \mathrm{Hom}_{\mathcal{C}}(y, z) \times \mathrm{Hom}_{\mathcal{C}}(x, y) \to \mathrm{Hom}_{\mathcal{C}}(x, z)}$$
  that satisfy the following:

\begin{enumerate}
  \item \textbf{(Strict associativity)}  
  For any quadruple of objects $w, x, y, z$, the two possible compositions
  $$ \circ_{w,y,z} \circ (1 \times \circ_{w,x,y}), \qquad \circ_{w,x,z} \circ (\circ_{x,y,z} \times 1) $$
  are \emph{equal} as functors
  \TODO{what is $1$?}
  $$ \mathrm{Hom}_{\mathcal{C}}(y,z) \times \mathrm{Hom}_{\mathcal{C}}(x,y) \times \mathrm{Hom}_{\mathcal{C}}(w,x) \to \mathrm{Hom}_{\mathcal{C}}(w,z).  $$

  \item \textbf{(Strict unit laws)}  
  For each object $x \in \mathrm{Ob}(\mathcal{C})$, there exists a distinguished identity element
  $$ \mathrm{id}_x \in \mathrm{Hom}_{\mathcal{C}}(x,x), $$
  such that for all $x, y \in \mathrm{Ob}(\mathcal{C})$,
  $$ \circ_{x,x,y}(\mathrm{id}_x, -) = 1_{\mathrm{Hom}_{\mathcal{C}}(x,y)}, \qquad \circ_{x,y,y}(-, \mathrm{id}_y) = 1_{\mathrm{Hom}_{\mathcal{C}}(x,y)}.  $$
  That is, composing with an identity morphism acts as the identity functor.

  \item \textbf{(Functoriality)}  
  \TODO{strict $n$-functor}
  Each composition map $\circ_{x,y,z}$ is a strict $n$-functor; it preserves all
  $k$-cell compositions and identities in the strict $n$-category structure of
  each $\mathrm{Hom}_{\mathcal{C}}(x,y)$ strictly (i.e., on the nose).
\end{enumerate}
  
\end{itemize}
In particular, collecting all the strict $0$-categories yields the \CrefAndHyperrefIfExist{definition:category_of_sets}{category of sets}, i.e. the $1$-category whose objects are sets and morphisms are functions between these sets. Moreover, a strict $1$-category is equivalent to a \CrefAndHyperrefIfExist{definition:locally_small_category}{locally small category}.

Morphisms between strict $(n+1)$-categories are functors that preserve all levels of composition strictly.  
\end{definition}

\begin{definition} \label{definition:bicategory}
A \hldef{bicategory} or a \hldef{weak $2$-category} $\mathcal{B}$ consists of the following data:
\begin{enumerate}
    \item A collection of \hldef{0-cells} (or \hldef{objects}), denoted by $A, B, C, \dots$;
    \item For each pair of 0-cells $A, B$, a \CrefAndHyperrefIfExist{definition:category}{category} \hl{$\mathcal{B}(A, B)$}, whose objects are called \hldef{1-cells} (denoted $f: A \to B$) and whose morphisms are called \hldef{2-cells} (denoted $\alpha: f \Rightarrow g$);
    \item For each triple of 0-cells $A, B, C$, a composition functor
    $$
    \hlin{\odot_{ABC}: \mathcal{B}(B, C) \times \mathcal{B}(A, B) \to \mathcal{B}(A, C)},
    $$
    denoted by $(g, f) \mapsto \hlin{g \odot f}$;
    \item For each 0-cell $A$, an identity 1-cell \hl{$I_A \in \mathcal{B}(A, A)$};
    \item For each quadruple of 0-cells $A, B, C, D$ and 1-cells $f: A \to B$, $g: B \to C$, $h: C \to D$, a natural isomorphism called the \hldef{associator},
    $$
    a_{h,g,f}: (h \odot g) \odot f \xrightarrow{\cong} h \odot (g \odot f);
    $$
    \item For each pair of 0-cells $A, B$ and 1-cell $f: A \to B$, natural isomorphisms called the \hldef{left and right unitors},
    $$
    \lambda_f: I_B \odot f \xrightarrow{\cong} f \quad \text{and} \quad \rho_f: f \odot I_A \xrightarrow{\cong} f.
    $$
\end{enumerate}
These must satisfy the following two coherence axioms:
\begin{enumerate}
    \item \textbf{The Pentagon Identity:} For 1-cells $f, g, h, k$, the following diagram of 2-cells commutes:
    $$
    \begin{tikzcd}[column sep=small]
        & (k \circ h) \circ (g \circ f) \arrow[dr, "\alpha"] & \\
        ((k \circ h) \circ g) \circ f \arrow[ur, "\alpha"] \arrow[d, "\alpha \circ 1"'] & & k \circ (h \circ (g \circ f)) \\
        (k \circ (h \circ g)) \circ f \arrow[rr, "\alpha"] & & k \circ ((h \circ g) \circ f) \arrow[u, "1 \circ \alpha"']
    \end{tikzcd}
    $$
    \item \textbf{The Triangle Identity:} For 1-cells $f: A \to B$ and $g: B \to C$, the following diagram of 2-cells commutes:
    $$
    \begin{tikzcd}
        (g \circ I_B) \circ f \arrow[rr, "\alpha"] \arrow[dr, "\rho \circ 1"'] & & g \circ (I_B \circ f) \arrow[dl, "1 \circ \lambda"] \\
        & g \circ f &
    \end{tikzcd}
    $$
\end{enumerate}
\end{definition}

\begin{definition} \label{definition:weak_n_category_tamsamani_simpson}
A \hldef{weak $n$-category} (in the sense of Tamsamani and Simpson) is defined inductively on $n \ge 0$.

\textbf{Base Case ($n=0$):} A weak 0-category is a set.

\textbf{Inductive Step:} For $n \ge 1$, a \hldef{weak $n$-category} is a simplicial object $X_\bullet$ in the category of weak $(n-1)$-categories, denoted $X: \Delta^{op} \to \text{Weak-}(n-1)\text{-Cat}$, satisfying the Segal condition (composition) and the discreteness condition (identities).

Explicitly, $X$ consists of:
\begin{itemize}
    \item A sequence of weak $(n-1)$-categories $X_k$ for each integer $k \ge 0$ (where $X_0$ represents the objects, $X_1$ the morphisms, $X_2$ composable pairs, etc.);
    \item Face maps $d_i: X_k \to X_{k-1}$ and degeneracy maps $s_i: X_k \to X_{k+1}$ in the category of weak $(n-1)$-categories, satisfying the simplicial identities.
\end{itemize}
These data must satisfy the following properties:
\begin{enumerate}
    \item \textbf{Discreteness (Units):} The object of 0-simplices $X_0$ is "discrete" in the sense that the underlying simplicial set of $X_0$ is constant (homotopically equivalent to a set).
    
    \item \textbf{Segal Condition (Composition):} For every $k \ge 2$, the map induced by the simplicial segment inclusions
    $$
    \mu_k: X_k \xrightarrow{\approx} X_1 \times_{X_0} X_1 \times_{X_0} \dots \times_{X_0} X_1 \quad (k \text{ times})
    $$
    is an \hldef{equivalence} of weak $(n-1)$-categories. 
    
    Here, the fiber product is formed using the source and target maps ($d_1, d_0: X_1 \to X_0$). The "equivalence" is the notion of equivalence appropriate for $(n-1)$-categories (defined inductively as maps inducing weak homotopy equivalences on all hom-spaces).
\end{enumerate}
\end{definition}

\begin{definition} \label{definition:tricategory}
A \hldef{weak 3-category}\CrefIfExists{definition:weak_n_category_tamsamani_simpson} (or \hldef{tricategory} in the sense of Gordon-Power-Street) consists of:
\begin{itemize}
    \item A collection of 0-cells (objects);
    \item For every pair of 0-cells $A, B$, a \CrefAndHyperrefIfExist{definition:bicategory_explicit_axioms}{bicategory} $\mathcal{C}(A, B)$;
    \item \textbf{Composition:} Pseudofunctors $\otimes: \mathcal{C}(B, C) \times \mathcal{C}(A, B) \to \mathcal{C}(A, C)$;
    \item \textbf{Units:} Identity 1-cells $I_A$ (objects in $\mathcal{C}(A, A)$);
\end{itemize}
Equipped with the following coherent equivalence data (constraints):
\begin{itemize}
    \item \textbf{Associator Adjoint Equivalence:} A pseudonatural equivalence in the bicategory of bicategories:
    $$
    \mathbf{a}: \otimes \circ (\otimes \times 1) \xrightarrow{\simeq} \otimes \circ (1 \times \otimes);
    $$
    \item \textbf{Unitor Adjoint Equivalences:} Pseudonatural equivalences $\mathbf{l}, \mathbf{r}$ for left and right units.
    \item \textbf{Modifications (The 3-dimensional Axioms):} Invertible 3-cells satisfying specific axioms:
    \begin{itemize}
        \item The \hldef{pentagon modification} $\pi$ (relating the various 3-cell associators);
        \item The \hldef{middle triangle modification} $\mu$;
        \item The \hldef{left/right triangle modifications} $\lambda, \rho$.
    \end{itemize}
\end{itemize}
These modifications must satisfy the non-abelian 4-cocycle conditions, specifically the \hldef{non-abelian pentagon equation} (often called the associahedron equation $K_5$).
\end{definition}


\begin{definition}
    Let $\mathcal{C}$ and $\mathcal{D}$ be \CrefAndHyperrefIfExist{definition:strict_n_category}{strict $n$-categories}.  
    For each pair of objects $x, y \in \mathrm{Ob}(\mathcal{C})$, write 
    $$ \mathrm{Hom}_{\mathcal{C}}(x, y), \quad \mathrm{Hom}_{\mathcal{D}}(Fx, Fy) $$
    for the corresponding hom $((n-1)$)-categories.  

    A \hldef{functor between $n$-categories} (or \hldef{$n$-functor})
    $$ F : \mathcal{C} \longrightarrow \mathcal{D} $$
    consists of the following data:
    \begin{itemize}
    \item a function on objects
    $$
    F_0 : \mathrm{Ob}(\mathcal{C}) \to \mathrm{Ob}(\mathcal{D}),
    $$
    \item for each pair of objects $x, y \in \mathrm{Ob}(\mathcal{C})$, a strict $(n-1)$-functor
    $$
    F_{x,y} : \mathrm{Hom}_{\mathcal{C}}(x, y) \to \mathrm{Hom}_{\mathcal{D}}(Fx, Fy),
    $$
    \item these assignments must satisfy the following axioms.
\end{itemize}

\begin{enumerate}
  \item \textbf{(Functoriality of composition)}  
  For all $x, y, z \in \mathrm{Ob}(\mathcal{C})$, the following square of strict $(n-1)$-functors commutes strictly:
  $$
  \begin{array}{ccc}
  \mathrm{Hom}_{\mathcal{C}}(y,z) \times \mathrm{Hom}_{\mathcal{C}}(x,y) & \xrightarrow{\circ_\mathcal{C}} & \mathrm{Hom}_{\mathcal{C}}(x,z) \\
  \downarrow F_{y,z} \times F_{x,y} & & \downarrow F_{x,z} \\
  \mathrm{Hom}_{\mathcal{D}}(Fy,Fz) \times \mathrm{Hom}_{\mathcal{D}}(Fx,Fy) & \xrightarrow{\circ_\mathcal{D}} & \mathrm{Hom}_{\mathcal{D}}(Fx,Fz)
  \end{array}
  $$
  That is,
  $$
  F_{x,z} \circ \circ_\mathcal{C} \;=\; \circ_\mathcal{D} \circ (F_{y,z} \times F_{x,y}).
  $$

  \item \textbf{(Preservation of identities)}  
  For every object $x \in \mathrm{Ob}(\mathcal{C})$, the identity $n$-cell is preserved strictly:
  $$
  F_{x,x}(\mathrm{id}_x) = \mathrm{id}_{Fx}.
  $$
\end{enumerate}

Morphisms between functors of $n$-categories are $(n+1)$-natural transformations, defined inductively in the same manner.
\end{definition}



\begin{definition}
    \TODO{this definition is not very precise}
    A \hldef{weak $n$-category} is a structure that generalizes the notion of a strict $n$-category by relaxing the associativity and unit laws to hold only up to specified coherent equivalences.  
    It can be defined inductively or through several equivalent formal models.

    Formally, a weak $n$-category $\mathcal{C}$ consists of:
    \begin{itemize}
    \item for each $0 \le k \le n$, a collection of $k$-morphisms;
    \item source and target operations assigning to each $k$-morphism its $(k-1)$-boundary morphisms;
    \item composition operations of $k$-morphisms along $(k-1)$-boundaries;
    \item identity $k$-morphisms for each $(k-1)$-morphism;
    \item and {coherent associativity, unit, and interchange equivalences}
    expressing that composition is associative and unital up to higher morphisms, 
    with these higher coherence conditions themselves satisfying higher coherence laws.
    \end{itemize}

    Several equivalent frameworks make this notion precise:

    \begin{itemize}
    \item In the \textbf{algebraic approach} (Batanin, Leinster), a weak $n$-category is encoded as a system of operations on $n$-cells governed by a globular operad, whose algebras satisfy the necessary coherence conditions.
    \item In the \textbf{simplicial or topological approach}, weak $n$-categories are modeled by higher Segal-type objects, such as Tamsamani–Simpson $n$-categories or Rezk’s $(n+k,n)$–$\Theta$–spaces, which are presheaves of simplicial sets on the category $\Theta_n$ satisfying Segal and completeness conditions.
    \item In the \textbf{truncated $\infty$-categorical approach} (Baez–Dolan), a weak $n$-category is a weak $\infty$-category with trivial cells above dimension $n$.
    \end{itemize}

    Thus, a weak $n$-category can be viewed as a category-like structure containing morphisms of all orders up to $n$, where composition is associative and unital only up to a coherent family of higher isomorphisms organized by the same structure.
\end{definition}



\begin{definition}
    \TODO{this definition is not very precise}
A \hldef{weak $n$-category} (or \hldef{weak higher category}) is a structure similar to a strict $n$-category, except that the associativity and unit axioms hold only up to coherent higher equivalences.  
That is, an $n$-category $\mathcal{C}$ consists of
\begin{itemize}
  \item objects, $1$-morphisms, $2$-morphisms, …, $n$-morphisms,
  \item composition operations defined at each level,
  \item associativity, unit, and interchange data holding up to coherent $(k+1)$-morphisms for $k < n$.
\end{itemize}
Different models exist for weak $n$-categories (e.g. bicategories for $n=2$, tricategories, quasicategories, Segal spaces, etc.), which satisfy equivalent homotopy-coherent conditions.
\end{definition}

\begin{definition} \label{definition:weak_infty_category}
    \TODO{this definition is not very precise}
An \hldef{$\infty$-category} (or \hldef{weak $\infty$-category}) is a category-like structure that encodes morphisms of all dimensions, where composition is associative and unital only up to coherent higher homotopies.  
More precisely, one usually works with the notion of an \hldef{$(\infty,1)$-category}, in which all $k$-morphisms for $k>1$ are invertible up to higher homotopy.  

Formally, an $\infty$-category can be defined in several equivalent ways, each specifying a precise model of this higher structure.  
Each of the standard models forms an \textit{$\infty$-cosmos} — a simplicially enriched category whose hom-objects are themselves $\infty$-categories — and all such models are equivalent up to categorical equivalence.  
Common frameworks include:

\begin{itemize}
  \item \textbf{Quasicategories:} Simplicial sets $C$ satisfying the \hldef{inner horn-filling condition}:  
  for every $n \ge 2$ and $0 < i < n$, every map $\Lambda_i[n] \to C$ (where $\Lambda_i[n]$ is the $i$th inner horn of the standard simplex $\Delta[n]$) extends to a map $\Delta[n] \to C$.  
  Such objects are the fibrant objects in the Joyal model structure on simplicial sets.

  \item \textbf{Complete Segal spaces:} Simplicial spaces $W_\bullet: \Delta^{\mathrm{op}} \to \mathbf{sSet}$ satisfying the Segal condition  
  $$
  W_n \simeq W_1 \times_{W_0} \cdots \times_{W_0} W_1
  $$
  (encoding associative composition up to homotopy) together with the completeness condition ensuring that equivalences are detected correctly.  
  These are the fibrant objects in the Rezk model structure on simplicial spaces.

  \item \textbf{Simplicial categories:} Categories enriched in simplicial sets whose hom-objects are Kan complexes and satisfy homotopy-coherent associativity laws.  
  These are related to quasicategories via the homotopy coherent nerve construction.

  \item \textbf{Other equivalent models:} Segal categories, naturally marked simplicial sets, and $\Theta$-spaces, each giving a distinct but equivalent encoding of higher categorical coherence.
\end{itemize}

Morphisms between $\infty$-categories, called \hldef{$\infty$-functors}, are maps in the chosen model that preserve the higher compositional structure up to coherent homotopy.  
The totality of $\infty$-categories and $\infty$-functors forms an $(\infty,2)$-category commonly denoted by \hl{$\infty\text{-}\mathbf{Cat}$}.

In summary, an $\infty$-category formalizes the idea of a category enriched in homotopy types (or spaces) — a “category up to higher coherent homotopies.”  
All standard models of $\infty$-categories (quasicategories, complete Segal spaces, simplicial categories, etc.) are equivalent in the sense of $\infty$-cosmoi theory.
\end{definition}



\subsection{Topological categories}

\CrefAndHyperrefIfExist{definition:topological_category}{Topological categories} provide one framework for $\infty$-categories.

\begin{definition} \label{definition:compactly_generated_topological_space}
    \TODO{final  topology}
A \CrefAndHyperrefIfExist{definition:topological_space}{topological space} $X$ is said to be \hldef{compactly generated} (or a \hldef{k-space}) if a subset $U \subseteq X$ is open whenever for every \CrefAndHyperrefIfExist{definition:compact_topological_space}{compact} subset $K \subseteq X$, the intersection $U \cap K$ is open in the subspace $K$.  
Equivalently, $X$ is compactly generated if and only if the topology of $X$ is the final topology with respect to the collection of inclusions $K \hookrightarrow X$ for compact $K \subseteq X$.
\end{definition}
\begin{definition} \label{definition:weakly_hausdorff_topological_space}
A \CrefAndHyperrefIfExist{definition:topological_space}{topological space} $X$ is said to be \hldef{weakly Hausdorff} if for every \CrefAndHyperrefIfExist{definition:continuous_map_of_topological_spaces}{continuous map} $f : K \to X$ from a \CrefAndHyperrefIfExist{definition:compact_topological_space}{compact} \CrefAndHyperrefIfExist{definition:compact_subset_of_a_topological_space_alternate_definition}{Hausdorff} space $K$, the image $\operatorname{Im}(f)$ is closed in $X$.  
Equivalently, $X$ is weakly Hausdorff if and only if for every compact Hausdorff $K$, the induced map $f : K \to X$ is a closed map.
\end{definition}

\begin{definition}  \label{definition:topological_category}
    A \hldef{topological category} is a \CrefAndHyperrefIfExist{definition:category}{category} which is \CrefAndHyperrefIfExist{definition:category_enriched_in_a_monoidal_category}{enriched over} $\mathcal{C}\mathcal{G}$, the category of \CrefAndHyperrefIfExist{definition:compactly_generated_topological_space}{compactly generated} (and \CrefAndHyperrefIfExist{definition:weakly_hausdorff_topological_space}{weakly Hausdorff}) topological spaces. The category of topological categories will often be denoted by \hl{$\mathcal{C}at_{\text{top}}$}.

    More explicitly, a topological category $\mathcal{C}$ consists of a collection of objects, together with a (compactly generated) topological space ${Map}_{\mathcal{C}}(X,Y)$ for any pair of objects $X,Y \in \mathcal{C}$. These mapping spaces must be equipped with an associative composition law, given by continuous maps
    $${Map}_{\mathcal{C}}(X_0, X_1) \times {Map}_{\mathcal{C}}(X_1, X_2) \times \ldots {Map}_{\mathcal{C}}(X_{n-1},X_n) \rightarrow {Map}_{\mathcal{C}}(X_0,X_n)$$
    (defined for all $n \geq 0$). Here the product is taken in the category of compactly generated topological spaces.
\end{definition}

There are analogues to the \CrefAndHyperrefIfExist{definition:geometric_realization_of_a_simplicial_set}{geometric realization functor} of simplicial sets and the \CrefAndHyperrefIfExist{definition:singular_complex_of_a_topological_space}{singular complex functor} of topological spaces:

\begin{definition} \label{definition:geometric_realization_of_a_simplicial_category_as_a_topological_category}
    Let $\mathcal{C}$ be a \CrefAndHyperrefIfExist{definition:simplicial_category}{simplicial category}. The \hldef{geometric realization of $\mathcal{C}$}, denoted \hl{$|\mathcal{C}|$}, is the \CrefAndHyperrefIfExist{definition:topological_category}{topological category} defined as follows:

    \begin{itemize}
    \item The objects of $|\mathcal{C}|$ are the same as the objects of $\mathcal{C}$:
    $$ \mathrm{Ob}(|\mathcal{C}|) = \mathrm{Ob}(\mathcal{C}).  $$

    \item For any objects $X, Y \in \mathcal{C}$, the morphism space is the \CrefAndHyperrefIfExist{definition:geometric_realization_of_a_simplicial_set}{geometric realization} of the simplicial mapping space:
    $$ \mathrm{Hom}_{|\mathcal{C}|}(X, Y) = |\Hom_\mathcal{C}(X, Y)|.  $$
    (\Cref{definition:category_enriched_in_a_monoidal_category})

    \item The composition maps 
    $$
    \Hom_{|\mathcal{C}|}(Y,Z) \times \Hom_{|\mathcal{C}|}(X,Y) \to \Hom_{|\mathcal{C}|}(X,Z)
    $$
    are induced by applying the geometric realization functor to the composition maps of $\mathcal{C}$, using that geometric realization commutes with finite products.
    \end{itemize}
\end{definition}


\begin{definition} \label{definition:singular_simplicial_category_of_a_topological_category}
    Let $\calC$ be a \CrefAndHyperrefIfExist{definition:topological_category}{topological category}. Its \hldef{singular simplicial category} \hl{$\operatorname{Sing} \calC$} is the \CrefAndHyperrefIfExist{definition:simplicial_category}{simplicial category} defined as follows:

        \begin{itemize}
        \item The objects of $\mathrm{Sing} \, \mathcal{C}$ are the same as those of $\mathcal{C}$:
        $$ \mathrm{Ob}(\mathrm{Sing} \, \mathcal{C}) = \mathrm{Ob}(\mathcal{C}).  $$

        \item For any objects $X, Y \in \mathcal{C}$, the morphism simplicial set is given by the \CrefAndHyperrefIfExist{definition:singular_complex_of_a_topological_space}{singular complex} of the topological morphism space:
        $$ \Hom_{\mathrm{Sing} \, \mathcal{C}}(X, Y) = \mathrm{Sing}(\Hom_{\mathcal{C}}(X, Y)). $$
        Explicitly, this singular complex is given by
        $$ \mathrm{Sing}_n(\Hom_{\mathcal{C}}(X,Y)) = \mathrm{Top}(|\Delta^n|, \Hom_{\mathcal{C}}(X, Y)), $$
        where \CrefAndHyperrefIfExist{definition:geometric_simplex_of_independent_points_in_a_real_vector_space}{$|\Delta^n|$ is the standard topological $n$-simplex}.

        \item The composition maps in $\mathrm{Sing} \, \mathcal{C}$ are induced by the composition maps in $\mathcal{C}$ via functoriality of the singular complex:
        $$ \mathrm{Sing}(\Hom_{\mathcal{C}}(Y,Z)) \times \mathrm{Sing}(\Hom_{\mathcal{C}}(X,Y)) \to \mathrm{Sing}(\Hom_{\mathcal{C}}(X,Z)).  $$
        \end{itemize}
\end{definition}

There is an adjunction analogous to that of \Cref{proposition:geometric_realization_singular_complex_adjunction}:

\begin{proposition}  \label{proposition:geometric_realization_of_simplicial_category_and_singular_simplicial_category_of_topological_category_adjunction}
        The \CrefAndHyperrefIfExist{definition:geometric_realization_of_a_simplicial_category_as_a_topological_category}{geometric realization functor} and the \CrefAndHyperrefIfExist{definition:singular_simplicial_category_of_a_topological_category}{singular simplicial category functor} form an \CrefAndHyperrefIfExist{definition:adjoint_functors_between_categories_unit_counit_of_adjoint_functors}{adjoint pair}
        $$|\cdot| \dashv \operatorname{Sing}$$
        $$|\cdot|: \mathbf{Cat}_\Delta \rightleftarrows \mathcal{C}at_{\text{top}}: \operatorname{Sing}$$
        (\Cref{definition:simplicial_category}, \Cref{definition:topological_category}).
\end{proposition}


\begin{definition} \label{definition:topological_nerve_of_a_topological_category}
If $\mathcal{C}$ is a \CrefAndHyperrefIfExist{definition:topological_category}{topological category}, we define the \hldef{(topological) nerve} \hl{$N(\mathcal{C})$} of $\mathcal{C}$ to be the \CrefAndHyperrefIfExist{definition:simplicial_nerve_of_a_simplicial_category}{simplicial nerve} of \CrefAndHyperrefIfExist{definition:singular_simplicial_category_of_a_topological_category}{$\operatorname{Sing} \mathcal{C}$}.
\end{definition}

\begin{lemma}[{\cite[between Examples 1.1.5.8 and 1.1.5.9]{lurie_htt}}] \label{lemma:functor_from_category_of_simplicial_sets_to_category_of_simplicial_categories}
   The functor $\mathfrak{C}: \Delta \to \mathbf{Cat}_\Delta$ (\Cref{definition:simplicial_path_category_of_a_simplex_of_a_finite_nonempty_linearly_ordered_set}, \Cref{definition:simplex_category}, \Cref{definition:simplicial_category}) extends uniquely (up to unique isomorphism) to a (small) colimit-preserving functor \hl{$\mathfrak{C}: \Sets_{\Delta} \to \mathbf{Cat}_\Delta$} (\Cref{definition:simplicial_cosimplicial_object_in_a_category}). 
\end{lemma}


\begin{theorem}[{\cite[Theorem 1.1.5.13]{lurie_htt}}]
    Let $\calC$ be a \CrefAndHyperrefIfExist{definition:topological_category}{topological category}, and let $X,Y \in \calC$ be objects. The \CrefAndHyperrefIfExist{definition:adjoint_functors_between_categories_unit_counit_of_adjoint_functors}{counit map} (obtained via the \CrefAndHyperrefIfExist{definition:adjoint_functors_between_categories_unit_counit_of_adjoint_functors}{adjunction} of \TODO{the adjunction should come from $\mathfrak{C}$ and $N$})
    $$|\Hom_{\mathfrak{C}[N(\calC)]}(X,Y)| \to \Hom_{\calC}(X,Y) $$
    (\Cref{definition:geometric_realization_of_a_simplicial_set}, \Cref{lemma:functor_from_category_of_simplicial_sets_to_category_of_simplicial_categories}, \Cref{definition:topological_nerve_of_a_topological_category})
    is a \CrefAndHyperrefIfExist{definition:weak_homotopy_equivalence_of_topological_spaces}{weak homotopy equivalence} of topological spaces.
\end{theorem}


\subsection{Derived categories via \texorpdfstring{$\infty$}{infinity}-categories}

\begin{definition}[$\infty$-category localization]
    \TODO{$\infty$-category in what sense}
    Let $\mathcal{C}$ be an $\infty$-category and let $W \subseteq \mathrm{Mor}(\mathcal{C})$ be a collection of morphisms in $\mathcal{C}$, called \hldef{weak equivalences} or \hldef{localizing morphisms}.

    A \hldef{localization of the $\infty$-category $\mathcal{C}$ at $W$} is an $\infty$-category $\mathcal{C}[W^{-1}]$ together with an $\infty$-functor
    $$\hlin{L: \mathcal{C} \to \mathcal{C}[W^{-1}]}$$
    satisfying the following universal property:
    \begin{itemize}
        \item For every morphism $w \in W$, the image $L(w)$ is an equivalence in $\mathcal{C}[W^{-1}]$.
        \item For any $\infty$-category $\mathcal{D}$, composition with $L$ induces a fully faithful embedding of $\infty$-categories
        \[
        \mathrm{Fun}(\mathcal{C}[W^{-1}], \mathcal{D}) \hookrightarrow \mathrm{Fun}(\mathcal{C}, \mathcal{D})
        \]
        whose essential image consists precisely of the $\infty$-functors that send every morphism in $W$ to an equivalence in $\mathcal{D}$.
    \end{itemize}
\end{definition}



\begin{definition}[dg-category] \label{definition:dg_category_over_a_ring}
Let $k$ be a commutative ring. A \hldef{dg-category} (differential graded category) $\mathcal{C}$ over $k$ is a \CrefAndHyperrefIfExist{definition:category_enriched_in_a_monoidal_category}{category enriched in} the category of \CrefAndHyperrefIfExist{definition:chain_complex_of_objects_in_an_additive_category}{chain complexes} of \CrefAndHyperrefIfExist{definition:module_of_a_ring}{$k$-modules}. Explicitly, a dg-category $\calC$ consists of the following data:
\begin{itemize}
    \item A class of objects $\mathrm{Ob}(\mathcal{C})$.
    \item For each pair of objects $X,Y \in \mathrm{Ob}(\mathcal{C})$, a \CrefAndHyperrefIfExist{definition:chain_complex_of_objects_in_an_additive_category}{complex} of $k$-modules \hl{$\mathrm{Hom}_\mathcal{C}(X,Y) = (\mathrm{Hom}^n_\mathcal{C}(X,Y), d)$}; that is, a $\mathbb{Z}$-graded $k$-module $$\mathrm{Hom}_\mathcal{C}(X,Y) = \bigoplus_{n \in \mathbb{Z}} \mathrm{Hom}_\mathcal{C}^n(X,Y)$$ equipped with a differential $d : \mathrm{Hom}_\mathcal{C}^n(X,Y) \to \mathrm{Hom}_\mathcal{C}^{n+1}(X,Y)$ satisfying $d^2 = 0$.
    \item For each triple of objects $X,Y,Z \in \mathrm{Ob}(\mathcal{C})$, a morphism of complexes (composition)
    $$\hlin{\circ : \mathrm{Hom}_\mathcal{C}(Y,Z) \otimes_k \mathrm{Hom}_\mathcal{C}(X,Y) \to \mathrm{Hom}_\mathcal{C}(X,Z)}$$
    which is associative and satisfies the graded Leibniz rule with respect to the differential $d$:
    \[
    d(f \circ g) = d(f) \circ g + (-1)^{|f|} f \circ d(g),
    \]
    for homogeneous $f,g$ with $|f|$ denoting the degree of $f$.
    \item For each object $X$, an identity morphism \hl{$1_X \in \mathrm{Hom}_\mathcal{C}^0(X,X)$} such that for any $f \in \mathrm{Hom}_\mathcal{C}(X,Y)$,
    \[
    1_Y \circ f = f, \quad f \circ 1_X = f.
    \]
\end{itemize}
\end{definition}


\begin{definition}[category of chain complexes as a dg-category] \label{definition:category_of_chain_complexes_of_objects_in_an_additive_category_as_a_dg_category}
Let $\mathcal{A}$ be an \CrefAndHyperrefIfExist{definition:additive_category}{additive category} and suppose that $\calA$ is \CrefAndHyperrefIfExist{definition:category_enriched_in_a_monoidal_category}{enriched over} a commutative ring $k$. The category \hl{$\mathrm{Ch}(\calA)$}\CrefIfExists{definition:chain_complex_of_objects_in_an_additive_category} can be given the structure of a \CrefAndHyperrefIfExist{definition:dg_category_over_a_ring}{dg-category over $k$} as follows:
\begin{itemize}
    \item An object in $\mathrm{Ch}(\mathcal{A})$ is, as usual, a \CrefAndHyperrefIfExist{definition:chain_complex_of_objects_in_an_additive_category}{chain complex} of objects in $\calA$.    

    \item For a pair of chain complexes $X^\bullet, Y^\bullet \in \mathrm{Ch}(\mathcal{A})$, the hom-complex $\mathrm{Hom}_{\mathrm{Ch}(\mathcal{A})}(X^\bullet, Y^\bullet) \in \Ob \mathrm{Ch}(\calA)$ is the complex of $k$-modules with
    \[
    \mathrm{Hom}^n_{\mathrm{Ch}(\mathcal{A})}(X^\bullet, Y^\bullet) := \prod_{m \in \mathbb{Z}} \mathrm{Hom}_\mathcal{A}\bigl(X^m, Y^{m+n}\bigr).
    \]
    The differential on this complex, $\delta: \mathrm{Hom}^n_{\mathrm{Ch}(\mathcal{A})}(X^\bullet, Y^\bullet) \to \mathrm{Hom}^{n+1}_{\mathrm{Ch}(\mathcal{A})}(X^\bullet, Y^\bullet)$, is given by
    \[
    \delta(f)^m := d_Y^{m+n} \circ f^m - (-1)^n f^{m+1} \circ d_X^m,
    \]
    for $f = (f^m)_{m \in \mathbb{Z}}$.
    
    \item Composition of morphisms
    \[
    \circ : \mathrm{Hom}^p_{\mathrm{Ch}(\mathcal{A})}(Y^\bullet, Z^\bullet) \otimes \mathrm{Hom}^q_{\mathrm{Ch}(\mathcal{A})}(X^\bullet, Y^\bullet) \to \mathrm{Hom}^{p+q}_{\mathrm{Ch}(\mathcal{A})}(X^\bullet, Z^\bullet)
    \]
    is given by component-wise composition in $\mathcal{A}$ with the Koszul sign rule:
    \[
    (g \circ f)^m := g^{m+q} \circ f^m.
    \]
    
    \item For each $X^\bullet$, the identity morphism is the family $(1_{X^m})_{m \in \mathbb{Z}}$ in degree zero.
\end{itemize}
\end{definition}

\begin{definition}[Moore complex of a simplicial object] \label{definition:moore_complex_of_a_simplicial_object_in_an_abelian_category}
    Let $\mathcal{A}$ be an \CrefAndHyperrefIfExist{definition:abelian_category}{abelian category} and let $X_\bullet$ be a \CrefAndHyperrefIfExist{definition:simplicial_cosimplicial_object_in_a_category}{simplicial object} in $\mathcal{A}$, i.e., a functor
    $$\hlin{X_\bullet: \Delta^{op} \to \mathcal{A}}$$
    where $\Delta$ is the simplex category.

    The \hldef{Moore complex}, synonymously called the \hldef{normalized chain complex} or \hldef{complex of normalized chains}, \hl{$M(X_\bullet) \in \mathrm{Ch}_{\geq 0}(\mathcal{A})$} associated to $X_\bullet$ is the \CrefAndHyperrefIfExist{definition:chain_complex_of_objects_in_an_additive_category}{chain complex} given by:
    \begin{itemize}
        \item For each integer $n \geq 0$, the degree $n$ component is
        $$\hlin{M_n(X_\bullet) := \bigcap_{i=1}^n \ker (d_i: X_n \to X_{n-1}),}$$
        where $d_i$ are the \CrefAndHyperrefIfExist{definition:simplex_of_a_simplicial_object_in_a_category_and_face_and_degeneracy_maps}{face maps} of the simplicial object.
        \item The differential $d_n: M_n(X_\bullet) \to M_{n-1}(X_\bullet)$ is given by the restriction of the face map $d_0$:
        $$\hlin{d_n := d_0|_{M_n(X_\bullet)}.}$$
    \end{itemize}
\end{definition}


\begin{theorem}[Dold-Kan correspondence]
    Let $\mathcal{A}$ be an abelian category with enough projectives. There exists an equivalence of categories, called the \hldef{Dold-Kan equivalence},
    \TODO{normalized chain complex functor, moore complex functor}
    $$\hlin{N \colon \mathbf{s}\Ab(\mathcal{A}) \leftrightarrows \mathrm{Ch}_{\geq 0}(\mathcal{A}) : \Gamma,}$$
    (\Cref{definition:moore_complex_of_a_simplicial_object_in_an_abelian_category})
    between the category of \CrefAndHyperrefIfExist{definition:simplicial_cosimplicial_object_in_a_category}{simplicial objects} in $\Ab$, and the category of connective (non-negatively graded) chain complexes in $\mathcal{A}$.

    This equivalence is exact, functorial, and compatible with the homotopy structures on both categories:
    \begin{itemize}
        \item The \hldef{normalized chain complex functor} $N$ sends a simplicial object to its normalized chain complex.
        \item The inverse functor $\Gamma$ takes a chain complex supported in degrees $\geq 0$ to a simplicial object.
        \item The adjunction $(N, \Gamma)$ restricts to an equivalence of categories.
    \end{itemize}
\end{theorem}

\begin{theorem}[Stable enhancement of the Dold-Kan correspondence]
    \TODO{verify that the following is true}
    Let $\mathcal{A}$ be an abelian category with enough projectives. There exists an equivalence of stable $\infty$-categories
    $$\mathcal{D}(\mathcal{A}) \simeq \mathrm{Sp}(\mathrm{S}_{\geq 0}(\mathcal{A})),$$
    where
    \begin{itemize}
        \item $\mathcal{D}(\mathcal{A})$ denotes the derived $\infty$-category of unbounded chain complexes in $\mathcal{A}$, obtained by inverting quasi-isomorphisms in $\mathrm{Ch}(\mathcal{A})$ and passing to the homotopy-coherent enhancement.
        \item $\mathrm{S}_{\geq 0}(\mathcal{A})$ is the $\infty$-category of connective (nonnegatively graded) simplicial objects in $\mathcal{A}$.
        \item $\mathrm{Sp}(\mathrm{S}_{\geq 0}(\mathcal{A}))$ denotes the stable $\infty$-category of spectrum objects in $\mathrm{S}_{\geq 0}(\mathcal{A})$, i.e., the stabilization of the connective simplicial objects.
    \end{itemize}

    This equivalence \hldef{categorifies and stabilizes} the classical Dold-Kan correspondence and extends it from connective chain complexes to all unbounded chain complexes:
    \begin{itemize}
        \item The classical Dold-Kan equivalence identifies nonnegatively graded chain complexes with connective simplicial objects.
        \item The stable enhancement takes spectra in the simplicial world to model all chain complexes, thus producing a stable $\infty$-category representing $\mathrm{Ch}(\mathcal{A})$ internally as an $\infty$-category.
    \end{itemize}
\end{theorem}


\section{$6$-functor formalism}
\TODO{read through these}

\begin{notation}[Ambient data, sizes, pullbacks, spans]
Fix a symmetric monoidal $\infty$-category $(\mathcal{E},\otimes,\mathds{1}_{\mathcal{E}})$ and a class of morphisms $\mathcal{P}$ in an $\infty$-category $\mathcal{C}$ such that:
\begin{itemize}
\item $\mathcal{C}$ admits pullbacks along morphisms in $\mathcal{P}$, and $\mathcal{P}$ is stable under pullback and composition.
\item There is a functor $\omega\colon \mathcal{C}\to \mathcal{E}$ (often a coefficient or “sheaf/transfer” functor) that sends $\mathcal{P}$-pullback squares in $\mathcal{C}$ to $\otimes$-Beck–Chevalley squares in $\mathcal{E}$ (so that base-change along $\mathcal{P}$ exists and is coherent).
\end{itemize}
Write objects of $\mathcal{C}$ by $X,Y,Z$, and spans by diagrams $X\xleftarrow{p} U\xrightarrow{q} Y$. For a morphism $f\colon X\to Y$ in $\mathcal{C}$, its graph span is $X\xleftarrow{\mathrm{id}_X} X\xrightarrow{f} Y$. We use the following notation:
\begin{itemize}
\item \hl{$(\mathcal{E},\otimes,\mathds{1}_{\mathcal{E}})$}
\item \hl{$\mathcal{C}$}
\item \hl{$\mathcal{P}\subseteq \mathrm{Mor}(\mathcal{C})$}
\item \hl{$\omega\colon \mathcal{C}\to \mathcal{E}$}
\item \hl{$X\xleftarrow{p} U\xrightarrow{q} Y$}
\end{itemize}
\end{notation}

\begin{definition}[Category of $\mathcal{P}$-correspondences]
Given $\mathcal{C}$ and a class $\mathcal{P}$ as in the notation above, the \hldef{$\infty$-category of $\mathcal{P}$-correspondences} $\mathrm{Corr}_{\mathcal{P}}(\mathcal{C})$ is defined as follows. Its objects are those of $\mathcal{C}$. For $X,Y\in \mathcal{C}$, the mapping anima $\mathrm{Map}_{\mathrm{Corr}_{\mathcal{P}}(\mathcal{C})}(X,Y)$ is the space of spans $X\xleftarrow{p} U\xrightarrow{q} Y$ with $p\in \mathcal{P}$, and morphisms given by equivalences of such spans. Composition is given by pullback: the composite of $X\xleftarrow{p} U\xrightarrow{q} Y$ and $Y\xleftarrow{p'} V\xrightarrow{q'} Z$ is
\[
X \xleftarrow{\;\mathrm{pr}_1\;} U\times_Y V \xrightarrow{\;q'\circ \mathrm{pr}_2\;} Z,
\]
where $\mathrm{pr}_1\in \mathcal{P}$ by stability of $\mathcal{P}$ under pullback, and associativity/unitality hold up to coherent homotopy. The unit at $X$ is the identity span $X\xleftarrow{\mathrm{id}_X} X \xrightarrow{\mathrm{id}_X} X$.
\end{definition}

\begin{definition}[Coefficient systems on correspondences]
A \hldef{coefficient system} (with values in $\mathcal{E}$ along $\mathcal{P}$) is the data of $(\omega,\mathcal{P})$ as in the notation, together with:
\begin{itemize}
\item for every span $X\xleftarrow{p\in \mathcal{P}} U\xrightarrow{q} Y$, a morphism in $\mathcal{E}$ written $\omega_!(p)\circ \omega^*(q)\colon \omega(X)\to \omega(Y)$, where $\omega^*(q)$ is pullback along $q$ and $\omega_!(p)$ is pushforward along $p$;
\item for every pullback square with left arrow in $\mathcal{P}$, a canonical base-change equivalence exhibiting Beck–Chevalley for $\omega_!$ and $\omega^*$;
\item coherences making these assignments functorial for composition of spans.
\end{itemize}
We denote such a system by \hl{$(\omega,\mathcal{P})$}.
\end{definition}

\begin{definition}[Symmetric monoidal structure on correspondences]
Assume $\mathcal{C}$ admits finite products $\times$ preserved by pullbacks along $\mathcal{P}$, and that $(\mathcal{E},\otimes,\mathds{1}_{\mathcal{E}})$ is symmetric monoidal with $\omega$ lax symmetric monoidal. The \hldef{symmetric monoidal structure} on $\mathrm{Corr}_{\mathcal{P}}(\mathcal{C})$ is given on objects by $X\otimes Y \coloneqq X\times Y$, unit $\mathds{1}\coloneqq *$ (a terminal object of $\mathcal{C}$), and on morphisms by external product of spans:
\[
\big(X\xleftarrow{p} U\xrightarrow{q} Y\big)\otimes \big(X'\xleftarrow{p'} U'\xrightarrow{q'} Y'\big)
\;\coloneqq\;
X\times X' \xleftarrow{p\times p'} U\times U' \xrightarrow{q\times q'} Y\times Y',
\]
which is well-defined since $p\times p'\in \mathcal{P}$ by closure of $\mathcal{P}$ under products and pullbacks. Symmetry, associativity, and unit constraints are induced from those of $(\mathcal{C},\times,*)$ and satisfy the $\infty$-categorical coherences.
\end{definition}

\begin{definition}[Symmetric monoidal $\infty$-category of correspondences with coefficients]
Given $(\omega,\mathcal{P})$ and assumptions as above, the \hldef{symmetric monoidal $\infty$-category of correspondences} is the symmetric monoidal $\infty$-category
\[
\mathrm{Corr}(\mathcal{C},\mathcal{E}) \coloneqq \big(\mathrm{Corr}_{\mathcal{P}}(\mathcal{C}),\;\otimes=\times,\;\mathds{1}=*\big),
\]
equipped with the symmetric monoidal functor $\Omega\colon \mathrm{Corr}_{\mathcal{P}}(\mathcal{C})\to \mathcal{E}$ determined on objects by $\Omega(X)=\omega(X)$ and on a span $X\xleftarrow{p\in \mathcal{P}} U\xrightarrow{q} Y$ by
\[
\Omega\big(X\xleftarrow{p} U\xrightarrow{q} Y\big)\coloneqq \omega_!(p)\circ \omega^*(q)\colon \omega(X)\to \omega(Y),
\]
with monoidal structure maps induced by the external product and the lax symmetric monoidal structure on $\omega$ together with Beck–Chevalley for base-change. This realizes $\mathrm{Corr}(\mathcal{C},\mathcal{E})$ as a symmetric monoidal $\infty$-category encoding spans in $\mathcal{C}$ and their functorial “push–pull” action on coefficients in $\mathcal{E}$.
\end{definition}





\appendix

\section{Set Theory}

\begin{definition} \label{definition:function_of_sets}
Let $X$ and $Y$ be sets. A \hldef{map} (or \hldef{function}) from $X$ to $Y$ is a rule $f$ assigning to each element $x \in X$ exactly one element $f(x) \in Y$. We write \hl{$f : X \to Y$}.

We say that $X$ is the \hldef{domain} and that $Y$ is the \hldef{codomain of $f$}.

% Given any sets $X$ and $Y$, the collection of maps $X \to Y$ is a set; the collection of all sets along with maps between them form a \CrefAndHyperrefIfExist{definition:locally_small_category}{locally small} \CrefAndHyperrefIfExist{definition:category}{category}, usually called the \hldef{category of sets}, and often denoted by notations such as \hl{$\mathrm{Set}$}, \hl{$\mathbf{Set}$}, \hl{$\mathrm{Sets}$}, \hl{$\mathbf{Sets}$}, \hl{$(\mathrm{Set})$}, \hl{$(\mathbf{Set})$}, \hl{$(\mathrm{Sets})$}, \hl{$(\mathbf{Sets})$}.

\end{definition}

\begin{definition} \label{definition:injective_surjective_bijective_map_of_sets}
Let $X$ and $Y$ be sets and let $f: X \to Y$ be a function.
\begin{itemize}
  \item The function $f$ is said to be \hldef{injective} (or \hldef{one-to-one}) if for all $x_1, x_2 \in X$, $f(x_1) = f(x_2)$ implies $x_1 = x_2$.
  \item The function $f$ is said to be \hldef{surjective} (or \hldef{onto}) if for every $y \in Y$ there exists $x \in X$ such that $f(x) = y$.

  \item The map $f$ is \hldef{bijective} if it is both injective and surjective. In this case, there exists a unique \hldef{inverse map} \hl{$f^{-1} : Y \to X$} such that for all $x \in X$ and $y \in Y$,
    $$f^{-1}(f(x)) = x \text{ and } f(f^{-1}(y)) = y.$$
\end{itemize}
\end{definition}
% \input{../_definitions/}

\section{Topology}


\begin{definition}[Topology] \label{definition:topological_space}
Let $X$ be a set. A \hldef{topology on $X$} is a collection $\mathcal{T}$ of subsets of $X$ such that:
\begin{enumerate}
    \item $\emptyset \in \mathcal{T}$ and $X \in \mathcal{T}$,
    \item For any collection $\{ U_i \}_{i \in I} \subseteq \mathcal{T}$ (with $I$ arbitrary), the union $\bigcup_{i \in I} U_i \in \mathcal{T}$,
    \item For any finite collection $\{ U_1, \ldots, U_n \} \subseteq \mathcal{T}$, the intersection $U_1 \cap \cdots \cap U_n \in \mathcal{T}$.
\end{enumerate}
If $\mathcal{T}$ is a topology on $X$, the pair $(X, \mathcal{T})$ is called a \hldef{topological space}. Members of $\mathcal{T}$ are called \hldef{open sets}. 

A subset $C \subseteq X$ is \hldef{closed} if its complement $X \setminus C$ is an open set in $\mathcal{T}$

One very often refers to $X$ as a topological spcae, omitting the notation of the topology $\mathcal{T}$. 

The collection of all topologies on a set $X$ may be denoted by notations such as \hl{$\mathrm{Top}(X)$}, \hl{$\mathbf{Top}(X)$}, or \hl{$\mathsf{Top}(X)$}.
\end{definition}





\begin{definition} \label{definition:open_covering_of_a_topological_space}
Let $(X, \tau)$ be a \CrefAndHyperrefIfExist{definition:topological_space}{topological space}. An \hldef{open covering of $X$} is a family of open sets 
\[ \mathcal{U} = \{ U_i \}_{i \in I} \] 
such that 
\[
\bigcup_{i \in I} U_i = X.
\]
Here, each $U_i \in \tau$ is an open subset of $X$ indexed by a set $I$, which can be finite or infinite.
\end{definition}
\begin{definition} \label{definition:continuous_map_of_topological_spaces}
Let $(X,\mathcal{T}_X)$ and $(Y,\mathcal{T}_Y)$ be \CrefAndHyperrefIfExist{definition:topological_space}{topological spaces}. A map $f : X \to Y$ is called \hldef{continuous} if for every open set $V \in \mathcal{T}_Y$, the preimage $f^{-1}(V)$ is an open set in $X$, that is,
$$\hlin{\forall V \in \mathcal{T}_Y, \; f^{-1}(V) \in \mathcal{T}_X.}$$
Equivalently, $f$ is continuous if and only if for every closed set $C \subseteq Y$, the preimage $f^{-1}(C)$ is closed in $X$. 


A \hldef{map of topological spaces} usually refers to a continuous map between the topological spaces.

The set of continuous maps from $X$ to $Y$ is sometimes denoted by \hl{$C(X,Y)$}. Other standard notation include \hl{$\operatorname{Hom}_{\mathrm{Top}}(X,Y)$} or \hl{$\operatorname{Top}(X,Y)$} coming from more general notation for morphisms between objects in a \CrefAndHyperrefIfExist{definition:category}{category}.

% The collection of topological spaces along with continuous maps form a \CrefAndHyperrefIfExist{definition:locally_small_category}{locally small} \CrefAndHyperrefIfExist{definition:category}{category}, usually called the \hldef{category of topological spaces} and often denoted by notations such as $\mathrm{Top}$, $\mathbf{Top}$, etc. 

% The set of continuous maps from $X$ to $Y$ is sometimes denoted by \hl{$C(X,Y)$}. Other standard notation include \hl{$\operatorname{Hom}_{\mathrm{Top}}(X,Y)$} or \hl{$\operatorname{Top}(X,Y)$} coming from more general notation for morphisms between objects in a category.
\end{definition}

\begin{definition}[Separation axioms] \label{definition:separation_axioms_of_topology}
Let $(X,\mathcal{T})$ be a \CrefAndHyperrefIfExist{definition:topological_space}{topological space}.
\begin{itemize}
  \item $(X,\mathcal{T})$ is \hldef{T$_0$} (\hldef{Kolmogorov}) if for every pair of distinct points $x, y \in X$, there exists an open set $U \in \mathcal{T}$ such that, without loss of generality, $x \in U$ and $y \notin U$.
  \item $(X,\mathcal{T})$ is \hldef{T$_1$} (\hldef{Fréchet}) if for every pair of distinct points $x, y \in X$, there exist open sets $U, V \in \mathcal{T}$ such that $x \in U$, $y \notin U$, and $y \in V$, $x \notin V$.
  \item $(X,\mathcal{T})$ is \hldef{T$_2$} or \hldef{Hausdorff} if for every pair of distinct points $x, y \in X$, there exist disjoint open sets $U, V \in \mathcal{T}$ such that $x \in U$ and $y \in V$.
  \item $(X,\mathcal{T})$ is \hldef{regular} if it is T$_1$ and for each point $x \in X$ and closed set $F \subseteq X$ with $x \notin F$, there exist disjoint open sets $U, V \in \mathcal{T}$ such that $x \in U$ and $F \subseteq V$.
  \item $(X,\mathcal{T})$ is \hldef{T$_3$} (regular Hausdorff) if it is T$_1$ and regular.
  \item $(X,\mathcal{T})$ is \hldef{completely regular} if for each closed set $F \subseteq X$ and $x \notin F$, there exists a continuous function $f : X \to [0,1]$ such that $f(x) = 0$ and $f|_F = 1$.
  \item $(X,\mathcal{T})$ is \hldef{T$_{3\frac{1}{2}}$} (completely regular Hausdorff) if it is T$_1$ and completely regular.
  \item $(X,\mathcal{T})$ is \hldef{normal} if it is T$_1$ and for each pair of disjoint closed sets $A, B \subseteq X$, there exist disjoint open sets $U, V \in \mathcal{T}$ such that $A \subseteq U$ and $B \subseteq V$.
  \item $(X,\mathcal{T})$ is \hldef{T$_4$} (normal Hausdorff) if it is T$_1$ and normal.
   \item $(X,\mathcal{T})$ is \hldef{T$_5$} (completely normal Hausdorff) if it is T$1$ and completely normal.
    \item $(X,\mathcal{T})$ is \hldef{perfectly normal} if every closed set is a \CrefAndHyperrefIfExist{definition:G_delta_set_and_F_sigma_set_of_a_topological_space}{$G\delta$} (\CrefAndHyperrefIfExist{definition:countable_finite_uncountable_sets}{countable} intersection of open sets) and the space is normal.
    \item $(X,\mathcal{T})$ is \hldef{T$_6$} (perfectly normal Hausdorff) if it is T$_1$ and perfectly normal.
\end{itemize}
\end{definition}
\begin{definition}[Homotopy of maps of topological spaces] \label{definition:homotopy_of_maps_of_topological_spaces_relative_to_a_subset}

    Let $X$ and $Y$ be topological spaces and let $K \subseteq X$ be a subset. Let $C(X,Y)$ denote the set of all continuous maps $f : X \to Y$.  

    \begin{enumerate}
        \item A \hldef{homotopy between two maps $f,g \in C(X,Y)$ relative to $K$ } is a continuous map
        $$H : X \times [0,1] \to Y$$
        such that for all $x \in X$,
        $$H(x,0) = f(x), \quad H(x,1) = g(x),$$
        and for all $x \in K$ and $t \in [0,1]$,
        $$H(x,t) = f(x) = g(x).$$

        If such an $H$ exists, we say $f$ and $g$ are \hldef{homotopic relative to $K$}, and we write \hl{$f \simeq g \text{ rel } K$}; this is an equivalence relation.

        A \hldef{homotopy between two maps $f,g \in C(X,Y)$} is simply a homotopy relative to $\emptyset$. We write we write \hl{$f \simeq g$} if a homotopy between them exists.

        \item Let $(X, x_0)$ and $(Y, y_0)$ be \CrefAndHyperrefIfExist{definition:pointed_topological_space}{pointed topological spaces} and let $K \subseteq X$ be a subset with $x_0 \in K$. Let $C_*(X,Y)$ denote the set of all continuous based maps $f : X \to Y$ satisfying $f(x_0) = y_0$.

        A \hldef{homotopy of based maps $f,g \in C_*(X,Y)$ relative to $K$} is a continuous map
        $$H : X \times [0,1] \to Y$$
        such that for all $x \in X$,
        $$H(x,0) = f(x), \quad H(x,1) = g(x),$$
        and for all $k \in K$ and $t \in [0,1]$,
        $$H(k,t) = f(k) = g(k),$$
        in particular fixing the basepoint throughout,
        $$H(x_0, t) = y_0 \quad \text{for all } t \in [0,1].$$

        If such an $H$ exists, we say $f$ and $g$ are \hldef{based homotopic relative to $K$}, and we write \hl{$f \simeq g \text{ rel } K$}. This is an equivalence relation.

        A \hldef{homotopy of based maps $f,g \in C_*(X,Y)$} without relative condition is the special case $K = \{x_0\}$ and is called a \hldef{homotopy of based maps} or \hldef{based homotopy}. We write \hl{$f \simeq g$} if such a homotopy exists.

    \end{enumerate}
\end{definition}
\begin{definition}[Homotopy groups] \label{definition:homotopy_groups_of_a_pointed_topological_space}
For any \CrefAndHyperrefIfExist{definition:pointed_topological_space}{pointed topological space} $(X,x_0)$ and integer $n \ge 0$, the \hldef{$n$-th homotopy group of $X$ at $x_0$}, denoted \hl{$\pi_n(X,x_0)$}, is defined as the set of all \CrefAndHyperrefIfExist{definition:homotopy_class_of_maps_of_topological_spaces_relative_to_a_subset}{homotopy classes (rel.\ $\partial I^n$)} of based maps
$$f : (I^n, \partial I^n) \to (X,x_0),$$
where $I^n = [0,1]^n$. For $n \ge 1$, $\pi_n(X,x_0)$ is a group under concatenation of based maps, and for $n \ge 2$, it is abelian.

The \hldef{fundamental group of $(X,x_0)$} refers to $\pi_1(X,x_0)$. Equivalently, it is the group of homotopy classes (rel.\ endpoints) of \CrefAndHyperrefIfExist{definition:path_and_loop_in_a_topological_space}{loops} $\gamma : [0,1] \to X$ satisfying $\gamma(0)=\gamma(1)=x_0$. 
\end{definition}
\begin{definition}[Weak homotopy equivalence] \label{definition:weak_homotopy_equivalence_of_topological_spaces}
Let $f : X \to Y$ be a \CrefAndHyperrefIfExist{definition:continuous_map_of_topological_spaces}{continuous map} between topological spaces. The map $f$ is a \hldef{weak homotopy equivalence} if, for every choice of basepoint $x_0 \in X$ with $y_0 = f(x_0)$, the induced maps on \CrefAndHyperrefIfExist{definition:homotopy_groups_of_a_pointed_topological_space}{homotopy groups}
$$f_* : \pi_n(X,x_0) \to \pi_n(Y,y_0)$$
are \CrefAndHyperrefIfExist{definition:group_homomorphism}{isomorphisms} for all integers $n \ge 0$.
\end{definition}

\section{Algebra}


\begin{definition}\label{definition:ring}
    A \hldef{ring} is a triple $(R, +, \cdot)$ where 
    \begin{enumerate}
        \item $(R,+)$ is a \CrefAndHyperrefIfExist{definition:group}{commutative group}, and
        \item $(R, \cdot)$ is a \CrefAndHyperrefIfExist{definition:monoid}{monoid}. 
        \item $\cdot$ is distributive over $+$, i.e. for all $a,b,c \in R$, we have
        $$a \cdot (b+c) = a \cdot b + a \cdot c \quad \text{and} \quad (a+b) \cdot c = a \cdot c + b \cdot c.$$
    \end{enumerate}

    Equivalently, a ring is a triple $(R,+,\cdot)$ where $+,\cdot: R \times R \to R$ are binary operations satisfying
    \begin{enumerate}
        \item $(a+b)+c = a+(b+c)$ and $(ab)c = a(bc)$ for all $a,b,c \in R$
        \item There exists an element \hl{$0 \in R$} such that $a+0 = a = 0 + a$ for all $a \in R$.
        \item For every $a \in R$, there exists an element \hl{$-a \in R$} such that $a+(-a) = 0 = (-a) + a$ for all $a \in R$.
        \item There exists an element \hl{$1 \in R$} such that $a \cdot 1 = a = 1 \cdot a$ for all $a \in R$.
        \item For all $a,b,c \in R$, we have
        $$a \cdot (b+c) = a \cdot b + a \cdot c \quad \text{and} \quad (a+b) \cdot c = a \cdot c + b \cdot c.$$
    \end{enumerate} 

    The operation $+$ is often called \hldef{addition} and the operation $\cdot$ is often called \hldef{multiplication}. Accordingly, the identity element $0$ of $+$ is often called the \hldef{additive identity} and the identity element $1$ of $\cdot$ is often called the \hldef{multiplicative identity}.

    % If $\cdot$ is additionally a \CrefAndHyperrefIfExist{definition:commutative_binary_operation}{commutative operation}, i.e. $a \cdot b = b \cdot a$ for all $a,b \in R$, then we call the ring \hldef{commutative}.  


\end{definition}
\begin{remark}
    Some writers might not require a ring to have a multiplicative identity element, i.e. would define a ring so that $(R,+)$ is a commutative group, $(R, \cdot)$ is a semigroup, and $\cdot$ is distributive over $+$. Such writers would call the notion of ring in \Cref{definition:ring} a \hldef{unitary ring} to emphasize the existence of the multiplicative identity $1$. 
\end{remark}


\begin{definition} \label{definition:commutative_ring}
   A \hldef{commutative (unital) ring} is a \CrefAndHyperrefIfExist{definition:ring}{ring} $(R, +, \cdot)$ such that $\cdot$ is a \CrefAndHyperrefIfExist{definition:commutative_binary_operation}{commutative operation}, i.e. $a \cdot b = b \cdot a$. 

   For many writers (e.g. ``commutative'' algebraists or number theorists), a \hldef{ring} refers to a commutative ring as above.
\end{definition}


\begin{definition} \label{definition:ring_homomorphism}
Let $(R,+,\cdot)$ and $(S,+,\cdot)$ be \CrefAndHyperrefIfExist{definition:ring}{rings}, not assumed to be commutative. A function $f: R \to S$ is called a \hldef{ring homomorphism} if for all $r_1,r_2 \in R$ the following properties hold:
\begin{enumerate}
    \item $f(r_1 + r_2) = f(r_1) + f(r_2)$,
    \item $f(r_1 r_2) = f(r_1) f(r_2)$,
    \item $f(1_R) = 1_S$ where $1_R$ and $1_S$ denote the multiplicative identities in $R$ and $S$, respectively.
\end{enumerate}
A ring homomorphism is said to be a \hldef{ring isomorphism} if it is invertible as a map of sets.

An \hldef{$R$-ring} refers to a ring $S$ equipped with a ring homomorphism $f: R \to S$. 

We note that a ring homomorphism $f: R \to S$ yields a natural \CrefAndHyperrefIfExist{definition:module_of_a_ring}{left $R$-module} structure on $S$ and a natural right $R$-module structure on $S$ respectively as follows for $r \in R$ and $s \in S$:
$$r \cdot s = f(r) \cdot s$$
$$s \cdot r = s \cdot f(r).$$
However, these left and right module structures need not yield a two-sided $R$--module structure.
% A ring $S$ equipped with a ring homomorphism $f: R \to S$ is called an \hldef{$R$-algebra}.
\end{definition}


\begin{definition}[Groups] \label{definition:group}
A \hldef{group} is a pair $(G,\cdot)$ where $G$ is a set and $\cdot : G \times G \to G$ is a binary operation, subject to the following conditions:

1. (Associativity) For all $g,h,k \in G$ one has 
$$ (g \cdot h) \cdot k = g \cdot (h \cdot k). $$

2. (Identity element) There exists an element \hl{$e \in G$} such that for all $g \in G$, 
$$ e \cdot g = g \cdot e = g. $$

3. (Inverse element) For all $g \in G$ there exists an element \hl{$g^{-1} \in G$} such that 
$$ g \cdot g^{-1} = g^{-1} \cdot g = e. $$

The element $e$ is called the \hldef{identity element of $G$}, and $g^{-1}$ is called the \hldef{inverse of $g$}.

Equivalently, a group is a \CrefAndHyperrefIfExist{definition:monoid}{monoid} with inverse elements.

\TextIfExists{definition:group_object_in_a_category_with_a_final_object}{Equivalently, a group is a \CrefAndHyperrefIfExist{definition:group_object_in_a_category_with_a_final_object}{group object} in the \CrefAndHyperrefIfExist{definition:category_of_sets}{category of sets}.}

A group $(G, \cdot)$ is often simply written as $G$, when the notation for the binary operation $\cdot$ is clear. 

An \hldef{abelian group} or synonymously, a \hldef{commutative group}, is a group $(G,\cdot)$ whose binary operation $\cdot$ is \CrefAndHyperrefIfExist{definition:commutative_binary_operation}{\hldef{abelian} or \hldef{commutative}}, i.e. satisfies 
$$g \cdot h = h \cdot g$$
for all $g,h \in G$. 


\TextIfExists{definition:module_of_a_ring}{An abelian group is equivalent to a $\bbZ$-module.}

\end{definition}


\begin{definition}[Group homomorphism] \label{definition:group_homomorphism}
Let $(G,\cdot)$ and $(H,\ast)$ be \CrefAndHyperrefIfExist{definition:group}{groups}. A map $f : G \to H$ is called a \hldef{group homomorphism} if for all $g_{1},g_{2} \in G$ one has
$$ f(g_{1} \cdot g_{2}) = f(g_{1}) \ast f(g_{2}). $$

The collection of all groups with the group homomorphisms forms a \CrefAndHyperrefIfExist{definition:locally_small_category}{locally small} \CrefAndHyperrefIfExist{definition:category}{category}, called the \hldef{category of groups}.

If $f$ is \CrefAndHyperrefIfExist{definition:injective_surjective_bijective_map_of_sets}{bijective}, then $f$ is called a \hldef{group isomorphism}. 
\TextIfExists{definition:isomorphism_in_a_category}{Equivalently, a group isomorphism is an \CrefAndHyperrefIfExist{definition:isomorphism_in_a_category}{isomorphism} in the category of groups}.
\end{definition}
\begin{definition} \label{definition:homomorphism_of_modules_over_a_ring}
Let $R,S$ be \CrefAndHyperrefIfExist{definition:ring}{(not-necessarily commutative) rings}. 
\begin{enumerate}
    \item Let $M$ and $N$ be \CrefAndHyperrefIfExist{definition:module_of_a_ring}{$R$-$S$-bimodules}. A function $\varphi: M \to N$ is called an \hldef{$R$-$S$-bimodule homomorphism} or \hldef{$R$-$S$-linear} if it is a \CrefAndHyperrefIfExist{definition:group_homomorphism}{group homomorphism} of the underlying abelian groups of $M$ and $N$ and respects the scalar actions as follows: 
    for all $m_1,m_2 \in M$, $r \in R$, and $s \in S$,
        \begin{align*}
        % \varphi(m_1 + m_2) &= \varphi(m_1) + \varphi(m_2), \\
        \varphi(r \cdot m_1) &= r \cdot \varphi(m_1), \\
        \varphi(m_1 \cdot s) &= \varphi(m_1) \cdot s.
        \end{align*}

    \item Let $M$ and $N$ be \CrefAndHyperrefIfExist{definition:module_of_a_ring}{left/right/two-sided $R$-modules}. A function $\varphi: M \to N$ is called a \hldef{left/right/two-sided $R$-module homomorphism} if it is an bimodule homomorphism on the \CrefAndHyperrefIfExist{definition:module_of_a_ring}{natural bimodule structures} of $M$ and $N$.
    %  $R$-$\bbZ$/$\bbZ$-$R$/$R$-$R$-bimodule homomorphism. 
     Such a function is also called \hldef{$R$-linear}.

\end{enumerate}

Modules and homomorphisms of a fixed type (i.e. $R$-$S$-bimodules or left/righ/two-sided $R$-modules) form a \CrefAndHyperrefIfExist{definition:locally_small_category}{locally small} \CrefAndHyperrefIfExist{definition:category}{category}.

% Let $M$ and $N$ be \CrefAndHyperrefIfExist{definition:module_of_a_ring}{left/right/two-sided $R$-modules or $R$-$S$-bidmodules}. 

% \begin{enumerate}
%     \item A function $\varphi : M \to N$ is called a \hldef{left/right/two-sided module homomorphism} or \hldef{$R$-linear} if it is additive (more precisely, a \CrefAndHyperrefIfExist{definition:group_homomorphism}{group homomorphism} of \CrefAndHyperrefIfExist{definition:group}{abelian groups}) and respects the scalar action(s) as follows: for all $m_1,m_2 \in M$, $r \in R$, and $s \in S$,
%     \begin{align*}
%     % \varphi(m_1 + m_2) &= \varphi(m_1) + \varphi(m_2), \\
%     \varphi(r \cdot m_1) &= r \cdot \varphi(m_1), \\
%     \varphi(m_1 \cdot s) &= \varphi(m_1) \cdot s.
%     \end{align*}

%     \item 
% \end{enumerate}

% \begin{enumerate}
%     \item If $M$ and $N$ are left $R$-modules, then for all $m_1,m_2 \in M$ and $r \in R$,
%     \begin{align*}
%     \varphi(m_1 + m_2) &= \varphi(m_1) + \varphi(m_2), \\
%     \varphi(r \cdot m_1) &= r \cdot \varphi(m_1).
%     \end{align*}

%     \item If $M$ and $N$ are right $R$-modules, then for all $m_1,m_2 \in M$ and $r \in R$,
%     \begin{align*}
%     \varphi(m_1 + m_2) &= \varphi(m_1) + \varphi(m_2), \\
%     \varphi(m_1 \cdot r) &= \varphi(m_1) \cdot r.
%     \end{align*}

%     \item If $M$ and $N$ are two-sided $R$-modules, then for all $m_1,m_2 \in M$, and $r_1,r_2 \in R$,
%     \begin{align*}
%     \varphi(m_1 + m_2) &= \varphi(m_1) + \varphi(m_2), \\
%     \varphi(r_1 \cdot m_1) &= r_1 \cdot \varphi(m_1), \\
%     \varphi(m_1 \cdot r_2) &= \varphi(m_1) \cdot r_2.
%     \end{align*}

%     \item If $M$ and $N$ are $(R,S)$-bimodules, then for all $m_1,m_2 \in M$, $r \in R$, and $s \in S$,
%     \begin{align*}
%     \varphi(m_1 + m_2) &= \varphi(m_1) + \varphi(m_2), \\
%     \varphi(r \cdot m_1) &= r \cdot \varphi(m_1), \\
%     \varphi(m_1 \cdot s) &= \varphi(m_1) \cdot s.
%     \end{align*}
% \end{enumerate}
\end{definition}


\begin{definition} \label{definition:module_of_a_ring}
Let $R$ be a \CrefAndHyperrefIfExist{definition:ring}{not-necessarily commutative ring}. 
\begin{enumerate}
    \item A \hldef{left $R$-module} is an abelian group $(M,+)$ together with an operation $R \times M \to M$, denoted $(r,m) \mapsto rm$, such that for all $r,s \in R$ and $m,n \in M$:
    \begin{itemize}
        \item $r(m+n) = rm + rn$,
        \item $(r+s)m = rm + sm$,
        \item $(rs)m = r(sm)$,
        \item $1_R m = m$ where $1_R$ is the multiplicative identity of $R$.
    \end{itemize}

    \item A \hldef{right $R$-module} is defined similarly as an abelian group $(M,+)$ with an operation $M \times R \to M$, denoted $(m,r) \mapsto mr$, such that for all $r,s \in R$ and $m,n \in M$:
    \begin{itemize}
        \item $(m+n)r = mr + nr$,
        \item $m(r+s) = mr + ms$,
        \item $m(rs) = (mr)s$,
        \item $m 1_R = m$.
    \end{itemize}

    \item Let $R$ and $S$ be  (not necessarily commutative) \CrefAndHyperrefIfExist{definition:ring}{rings}.

    An \hldef{$R$-$S$-bimodule} (or an \hldef{$R$-$S$-module} or an $(R,S)$-module, etc.)is an \CrefAndHyperrefIfExist{definition:group}{abelian group} $(M,+)$ equipped with
    \begin{enumerate}
        \item a left action of $R$:
        $$\hlin{R \times M \to M, \quad (r,m) \mapsto r \cdot m},$$
        making $M$ a \CrefAndHyperrefIfExist{definition:module_of_a_ring}{left $R$-module},
        \item a right action of $S$:
        $$\hlin{M \times S \to M, \quad (m,s) \mapsto m \cdot s},$$
        making $M$ a right $S$-module,
    \end{enumerate}
    such that the left and right actions commute; that is, for all $r \in R$, $s \in S$, and $m \in M$,
    $$ r \cdot (m \cdot s) = (r \cdot m) \cdot s.  $$

    \item A \hldef{two-sided $R$-module} (or \hldef{$R$-bimodule}) is an $R$-$R$-bimodule.
    
    % an abelian group $(M,+)$ which is simultaneously a left $R$-module and a right $R$-module, such that $(rm)s = r(ms)$ for all $r,s \in R$, $m \in M$. Equivalently, a two-sided $R$-module is an \hldef{$R$-$R$-bimodule}\CrefIfExists{definition:module_of_a_ring}


\end{enumerate}
If $R$ is a \CrefAndHyperrefIfExist{definition:commutative_ring}{commutative ring}, then a left/right $R$-module can automatically be regarded as a two-sided $R$-module. As such, we simply talk about \hldef{$R$-modules} in this case. 

Any abelian group is equivalent to a two-sided $\bbZ$-module. Moreover, any left $R$-module is equivalent to an \CrefAndHyperrefIfExist{definition:module_of_a_ring}{$R-\bbZ$-bimodule} and any right $R$-module is equivalent to an \CrefAndHyperrefIfExist{definition:module_of_a_ring}{$\bbZ-R$-bimodule}. Given a left/right/two-sided $R$-module, its \hldef{natural bimodule structure} will refer to its structure as a $R$-$\bbZ$/$\bbZ$-$R$/$R$-$R$ bimodule. In this way, many definitions associated with the notions of left/right/two-sided $R$-modules can be defined as special cases for definitions for $R$-$S$-bimodules.
\end{definition}


\section{Schemes}
\begin{definition} \label{definition:etale_morphism_of_schemes}
A \CrefAndHyperrefIfExist{definition:morphism_of_schemes}{morphism of schemes} $f : X \to Y$ is called \hldef{\'etale} if it satisfies the following conditions:
\TODO{sheaf of relative differentials}
\begin{itemize}
  \item $f$ is \CrefAndHyperrefIfExist{definition:locally_of_finite_presentation_finite_presentation_morphism_of_schemes}{locally of finite presentation},
  \item $f$ is \CrefAndHyperrefIfExist{definition:flat_morphism_of_schemes}{flat},
  \item $f$ is \CrefAndHyperrefIfExist{definition:unramified_morphism_of_schemes}{unramified}, i.e., the sheaf of relative differentials $\Omega_{X/Y}$ equals $0$.
\end{itemize}
\TODO{relative dimension}
Equivalently, a morphism of schemes is \'etale if and only if it is \CrefAndHyperrefIfExist{definition:smooth_morphism_of_schemes}{smooth} of relative dimension $0$.
A \CrefAndHyperrefIfExist{definition:finite_morphism_of_schemes}{finite} \'etale morphism is synonymously called a \hldef{finite \'etale cover}.
\end{definition}

\section{Grothendieck universes}

Recall that the collection of all sets within Zermelo-Fraenkel set theory is a proper class and not a set. We often use Grothendieck universes in categorical discussions as a way to restrict the sets considered so as to guarantee that only sets result in the categorical constructions that matter (e.g. to guarantee that the hom's in a newly constructed category form a set).

\begin{definition}[Grothendieck Universe] \label{definition:grothendieck_universe}
    Let $U$ be a set. We say $U$ is a \hldef{Grothendieck universe} (or just a \hldef{universe}) if the following conditions hold:
    \begin{enumerate}
        \item If $x \in U$ and $y \in x$, then $y \in U$ (transitivity).
        \item If $x,y \in U$, then $\{x,y\} \in U$ (closed under pair formation).
        \item If $x \in U$, then the power set $\mathcal{P}(x) \in U$.
        \item If $I \in U$ and $(x_\alpha)_{\alpha \in I}$ is a family with each $x_\alpha \in U$, then $\bigcup_{\alpha \in I} x_\alpha \in U$.
    \end{enumerate}
    A set $X$ is called \hldef{$U$-small} or a \hldef{$U$-set} if $X \in U$.
\end{definition}

\begin{remark}
    Given a Grothendieck universe $U$, the collection of all $U$-small sets is $U$ itself, which contrasts against the fact that the collection of all sets (for the ZF (and ZFC) axioms) is a proper class.
\end{remark}

\begin{remark}
    Within ZF/ZFC alone, there is no guarantee that every given set $S$ satisfies $S \in U$ for some universe $S$.
\end{remark}

\begin{remark}
    While universes are a powerful tool for studying category theory, they are not universally used or necessary for doing so. Usage depends on the context and foundational preferences. For instance, Grothendieck universes can be used to simplify foundational technicalities when one needs to work with such things as ``categories of all categories'' or ``functor categories'' of large size. 
\end{remark}







\begin{definition}[Compact topological space] \label{definition:compact_topological_space}
A topological space $(X, \mathcal{T})$ is \hldef{compact} if every open cover of $X$ admits a finite subcover; that is, for every collection $\{ U_i \}_{i \in I}$ of open sets in $\mathcal{T}$ such that $X = \bigcup_{i \in I} U_i$, there exists a finite subcollection $\{ U_{i_j} \}_{j=1}^n$ such that $X = \bigcup_{j=1}^n U_{i_j}$.

Some mathematicians, e.g. algebraic geometers, would refer to this property as \hldef{quasi-compactness}.
\end{definition}
\begin{definition}[Separation axioms] \label{definition:separation_axioms_of_topology}
Let $(X,\mathcal{T})$ be a \CrefAndHyperrefIfExist{definition:topological_space}{topological space}.
\begin{itemize}
  \item $(X,\mathcal{T})$ is \hldef{T$_0$} (\hldef{Kolmogorov}) if for every pair of distinct points $x, y \in X$, there exists an open set $U \in \mathcal{T}$ such that, without loss of generality, $x \in U$ and $y \notin U$.
  \item $(X,\mathcal{T})$ is \hldef{T$_1$} (\hldef{Fréchet}) if for every pair of distinct points $x, y \in X$, there exist open sets $U, V \in \mathcal{T}$ such that $x \in U$, $y \notin U$, and $y \in V$, $x \notin V$.
  \item $(X,\mathcal{T})$ is \hldef{T$_2$} or \hldef{Hausdorff} if for every pair of distinct points $x, y \in X$, there exist disjoint open sets $U, V \in \mathcal{T}$ such that $x \in U$ and $y \in V$.
  \item $(X,\mathcal{T})$ is \hldef{regular} if it is T$_1$ and for each point $x \in X$ and closed set $F \subseteq X$ with $x \notin F$, there exist disjoint open sets $U, V \in \mathcal{T}$ such that $x \in U$ and $F \subseteq V$.
  \item $(X,\mathcal{T})$ is \hldef{T$_3$} (regular Hausdorff) if it is T$_1$ and regular.
  \item $(X,\mathcal{T})$ is \hldef{completely regular} if for each closed set $F \subseteq X$ and $x \notin F$, there exists a continuous function $f : X \to [0,1]$ such that $f(x) = 0$ and $f|_F = 1$.
  \item $(X,\mathcal{T})$ is \hldef{T$_{3\frac{1}{2}}$} (completely regular Hausdorff) if it is T$_1$ and completely regular.
  \item $(X,\mathcal{T})$ is \hldef{normal} if it is T$_1$ and for each pair of disjoint closed sets $A, B \subseteq X$, there exist disjoint open sets $U, V \in \mathcal{T}$ such that $A \subseteq U$ and $B \subseteq V$.
  \item $(X,\mathcal{T})$ is \hldef{T$_4$} (normal Hausdorff) if it is T$_1$ and normal.
   \item $(X,\mathcal{T})$ is \hldef{T$_5$} (completely normal Hausdorff) if it is T$1$ and completely normal.
    \item $(X,\mathcal{T})$ is \hldef{perfectly normal} if every closed set is a \CrefAndHyperrefIfExist{definition:G_delta_set_and_F_sigma_set_of_a_topological_space}{$G\delta$} (\CrefAndHyperrefIfExist{definition:countable_finite_uncountable_sets}{countable} intersection of open sets) and the space is normal.
    \item $(X,\mathcal{T})$ is \hldef{T$_6$} (perfectly normal Hausdorff) if it is T$_1$ and perfectly normal.
\end{itemize}
\end{definition}