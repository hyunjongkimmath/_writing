%% Delete this \nocite command invocation to make the references section only list out the bibitems that are actually cited.
\nocite{*}

\section{Basic category theory}

\begin{definition}[Category] \label{definition:category}
    A 
    \defin{category}{category}{
        name={Category},
        description={A nice enough collection of objects and morphisms (\Cref{definition:category})},
    }
    \hldef{category} $\mathcal{C}$ consists of the following data:
    \begin{itemize}
        \item A class of \defin{objects}{object_of_a_category}{
            name={Object of a category},
            description={\Cref{definition:category}},
        }
        denoted \notat{\operatorname{Ob}(\mathcal{C})}{class_of_objects_of_a_category}{
            name={$\operatorname{Ob}(\mathcal{C})$},
            description={Class of objects of a category $\calC$ \Cref{definition:category}},
            sort={Ob},
        }.
        % \hl{$\operatorname{Ob}(\mathcal{C})$}.
        \item For each pair of objects $X, Y \in \operatorname{Ob}(\mathcal{C})$, a class
        \notatin{\operatorname{Hom}_{\mathcal{C}}(X,Y)}{class_of_morphisms_between_two_objects_of_a_category}
        {
            name={$\operatorname{Hom}_{\mathcal{C}}(X,Y)$},
            description={Class of morphisms between objects $X$ and $Y$ of the category $\calC$ (\Cref{definition:category})},
            sort={Hom},
        }
        % $$\hlin{\operatorname{Hom}_{\mathcal{C}}(X,Y)}$$
        of \defin{morphisms}{morphism_between_objects_of_a_category}{
            name={Morphism between objects of a category},
            description={(\Cref{definition:category})},
        }
        (also called 
        \defin{arrows}{arrow_between_objects_of_a_category}{
            name={Arrow between objects of a category},
            description={Synonym for morphism (\Cref{definition:category})},
        }
        or
        \defin{homs}{hom_between_objects_of_a_category}{
            name={Hom between objects of a category},
            description={Synonym for morphism (\Cref{definition:category})},
        }). If the category $\calC$ is clear, then this \hldef{hom-class} is also denoted by \hl{$\operatorname{Hom}(X,Y)$}. It may also be denoted by \hl{$\operatorname{hom}_{\mathcal{C}}(X,Y)$} or \hl{$\operatorname{hom}(X,Y)$}, especially to distinguish from other types of hom's (e.g. \hyperrefIfExists{definition:internal_hom_object_in_a_category}{internal hom's})
        \item For each triple of objects $X,Y,Z$, a composition law
        $$ \circ : \operatorname{Hom}_{\mathcal{C}}(Y,Z) \times \operatorname{Hom}_{\mathcal{C}}(X,Y) \to \operatorname{Hom}_{\mathcal{C}}(X,Z), $$
        denoted \hl{$(g,f) \mapsto g \circ f$}.
        \item For each object $X$, an \hldef{identity morphism}
        $$\hlin{\operatorname{id}_X \in \operatorname{Hom}_{\mathcal{C}}(X,X).}$$
    \end{itemize}
    These data satisfy the following axioms:
    \begin{itemize}
        \item (Associativity) For all morphisms $f \in \operatorname{Hom}_{\mathcal{C}}(X,Y)$, $g \in \operatorname{Hom}_{\mathcal{C}}(Y,Z)$, and $h \in \operatorname{Hom}_{\mathcal{C}}(Z,W)$, 
        $$
        h \circ (g \circ f) = (h \circ g) \circ f.
        $$
        \item (Identity) For all $f \in \operatorname{Hom}_{\mathcal{C}}(X,Y)$,
        $$
        \operatorname{id}_Y \circ f = f = f \circ \operatorname{id}_X.
        $$
    \end{itemize}
    One often writes \hl{$X \in \calC$} synonymously with $X \in \Ob(\calC)$, i.e. to denote that $X$ is an object of of $\calC$. 

    We may call a category as above an \hldef{ordinary category} to distinguish this notion from the notions of \hyperrefIfExists{definition:category_enriched_in_a_monoidal_category}{\emph{categories enriched in monoidal categories}} or higher/$n$-categories.
    \TODO{TODO: define $n$-categories}

    A category as defined above may be called called a \hldef{large category} or a \hldef{class category} to emphasize that the hom-classes may be proper classes rather than sets (note, however, that the possibility that hom-classes are sets is not excluded for large categories). Accordingly, a \hldef{category} may often refer to a \hyperrefIfExists{definition:locally_small_category}{locally small category}\CrefIfExists{definition:locally_small_category}, which is a category whose hom-classes are all sets.
\end{definition}

% Later on, we refer to the \gls{category} again.


\begin{definition}[Isomorphism in a category] \label{definition:isomorphism_in_a_category}
Let $\mathcal{C}$ be a \CrefAndHyperrefIfExist{definition:category}{(large) category}, and let $x,y \in \mathrm{Ob}(\mathcal{C})$.  
A morphism $f \in \mathcal{C}(x,y)$ is called an \hldef{isomorphism} if there exists a morphism $g \in \mathcal{C}(y,x)$ such that
$$ g \circ f = 1_x \qquad \text{and} \qquad f \circ g = 1_y.  $$
In this case, $g$ is called the \hldef{inverse of $f$}, and $x$ and $y$ are said to be \hldef{isomorphic objects} in $\mathcal{C}$. It is standard to write \hl{$x \cong y$} if there exists an isomorphism $f : x \to y$.

In practice, isomorphisms in specific categories may be defined in different, yet equivalent, ways.
\end{definition}





\begin{definition}[Locally small category] \label{definition:locally_small_category}
A \hyperrefIfExists{definition:category}{(large) category}\CrefIfExists{definition:category} $\mathcal{C}$ is called a \hldef{locally small category} if for every pair of objects $X, Y \in \operatorname{Ob}(\mathcal{C})$, the collection $\operatorname{Hom}_{\mathcal{C}}(X,Y)$ of morphisms between them is a (\CrefAndHyperrefIfExist{definition:small_set}{small}) \emph{set} (as opposed to a proper class). In other words, each hom-class is a set and may even be called a \hldef{hom-set}.

In some contexts, a locally small category may simply be called a \hldef{category}, especially when genuinely large categories are not considered.

A category $\mathcal{C}$ is called a \hldef{small category} if it is a locally small category and the class $\operatorname{Ob}(\mathcal{C})$ of objects is a set.

\TextIfExists{definition:grothendieck_universe}{
Given a \hyperrefIfExists{definition:grothendieck_universe}{universe}\CrefIfExists{definition:grothendieck_universe} $U$, we can define the notion of a \hldef{$U$-locally small category} and of a \hldef{$U$-small category} similarly. More explicitly, 
\begin{enumerate}
    \item a $U$-locally small category is a category such that for every pair of objects $X, Y \in \operatorname{Ob}(\mathcal{C})$, the collection $\operatorname{Hom}_{\mathcal{C}}(X,Y)$ of morphisms between them is a $U$-set.
    \item a $U$-small category is a category such that $\operatorname{Ob}(\mathcal{C})$ is a $U$-set and for every pair of objects $X, Y \in \operatorname{Ob}(\mathcal{C})$, the collection $\operatorname{Hom}_{\mathcal{C}}(X,Y)$ of morphisms between them is a $U$-set; in particular the collection of all objects and morhpisms in a $U$-small category is a $U$-set.
\end{enumerate}
}
\end{definition}

\begin{remark}
    Many ``concrete'' categories considered in ``classical mathematics'' or outside of more ``abstract'' category theory tend to be locally small. For example, the categories of sets, groups, $R$-modules, vector spaces, topological spaces, schemes, manifolds, sheaves on ``small enough'' sites are all locally small.
\end{remark}

\begin{convention}
    In this document, care should be taken to make it clear whether a category is locally small or large. However, currently such care has not been fully taken, so most statements that just speak of ``categories'' may not be applicable to large categories.
\end{convention}

\begin{definition}[Opposite category] \label{definition:opposite_category_of_a_category}

    Let $\mathcal{C}$ be a \hyperrefIfExists{definition:category}{(large) category}\CrefIfExists{definition:category}. The \hldef{opposite category} of $\mathcal{C}$, denoted \hl{$\mathcal{C}^{\mathrm{op}}$}, is defined as follows:
    \begin{itemize}
        \item The objects of $\mathcal{C}^{\mathrm{op}}$ are the same as those of $\mathcal{C}$.
        \item For any pair of objects $X,Y \in \mathcal{C}$, the morphisms from $X$ to $Y$ in $\mathcal{C}^{\mathrm{op}}$ are given by the morphisms from $Y$ to $X$ in $\mathcal{C}$:
        \[
        \mathrm{Hom}_{\mathcal{C}^{\mathrm{op}}}(X,Y) := \mathrm{Hom}_{\mathcal{C}}(Y,X).
        \]
        \item Composition in $\mathcal{C}^{\mathrm{op}}$ is defined by reversing the order of composition in $\mathcal{C}$. That is, for morphisms $f \in \mathrm{Hom}_{\mathcal{C}^{\mathrm{op}}}(X,Y)$ and $g \in \mathrm{Hom}_{\mathcal{C}^{\mathrm{op}}}(Y,Z)$, their composition is
        \[
        g \circ_{\mathcal{C}^{\mathrm{op}}} f := f \circ_{\mathcal{C}} g.
        \]
    \end{itemize}
    Intuitively, the category $\mathcal{C}^{\mathrm{op}}$ thus "reverses" the direction of all morphisms in $\mathcal{C}$.

\end{definition}


\TODO{TODO: define and notate many common categories}

\subsection{Functors between categories}

\begin{definition} \label{definition:functor_between_categories}
Let $\mathcal{C}$ and $\mathcal{D}$ be \CrefAndHyperrefIfExist{definition:category}{(large) categories}. 
\begin{enumerate}
  \item A \hldef{functor $F: \calC \to \calD$ (from $\mathcal{C}$ to $\mathcal{D}$)} consists of :
  \begin{itemize}
    \item For each object $X$ in $\mathcal{C}$, an object $F(X)$ in $\mathcal{D}$.
    \item For each morphism $f: X \to Y$ in $\mathcal{C}$, a morphism $F(f): F(X) \to F(Y)$ in $\mathcal{D}$,
  \end{itemize}
  such that:
  \begin{align*}
    F(\mathrm{id}_X) &= \mathrm{id}_{F(X)} \quad \text{for all objects } X \text{ in } \mathcal{C}, \\
    F(g \circ f) &= F(g) \circ F(f) \quad \text{for all } X,Y,Z \in \Ob(\calC) \text{ and all } f: X \to Y, g: Y \to Z \text{ in } \mathcal{C}.
  \end{align*}

  Functors as defined above are also referred to as \hldef{covariant functors} to distinguish them from contravariant functors

  \item A \hldef{contravariant functor from $\calC$ to $\calD$} refers to a covariant functor $F:\calC^{\op} \to \calD$. Equivalently, such a functor consists of 
  \begin{itemize}
    \item For each object $X$ in $\mathcal{C}$, an object $F(X)$ in $\mathcal{D}$.
    \item For each morphism $f: X \to Y$ in $\mathcal{C}$, a morphism $F(f): F(Y) \to F(X)$ in $\mathcal{D}$,
  \end{itemize}
  such that:
  \begin{align*}
    F(\mathrm{id}_X) &= \mathrm{id}_{F(X)} \quad \text{for all objects } X \text{ in } \mathcal{C}, \\
    F(g \circ f) &= F(f) \circ F(g) \quad \text{for all } X,Y,Z \in \Ob(\calC) \text{ and all } f: X \to Y, g: Y \to Z \text{ in } \mathcal{C}.
  \end{align*}
  \TextIfExists{definition:presheaf_on_a_category}{A synonym for a ``contravariant functor from $\calC$ to $\calD$'' is a ``\CrefAndHyperrefIfExist{definition:presheaf_on_a_category}{presheaf on $\calC$ with values in $\calD$}''.}
  
\end{enumerate}
Note that declarations such as ``Let $F: \calC^{\op} \to \calD$ be a contravariant functor'' can be common; such declarations usually mean ``Let $F$ be a contravariant functor from $\calC$ to $\calD$'' as opposed to ``Let $F$ be a contravariant functor from $\calC^{\op}$ to $\calD$''. further note that a contravariant functor from $\calC$ to $\calD$ is equivalent to a covariant functor from $\calC^{\op}$ to $\calD$.
\end{definition}



\subsubsection{Natural transformations between functors}
\begin{definition} \label{definition:natural_transformation_between_functors_between_categories}
Let $\mathcal{C}$ and $\mathcal{D}$ be \CrefAndHyperrefIfExist{definition:category}{(large) categories}. 
Let $F, G : \mathcal{C} \to \mathcal{D}$ be \CrefAndHyperrefIfExist{definition:functor_between_categories}{functors}.

A \hldef{natural transformation $\eta$ between $F$ and $G$} is a family of morphisms $\eta_X: F(X) \to G(X)$ in $\mathcal{D}$, one for each object $X$ in $\mathcal{C}$, such that for every morphism $f: X \to Y$ in $\mathcal{C}$,
\begin{align*}
G(f) \circ \eta_X = \eta_Y \circ F(f)
\end{align*}
in $\mathcal{D}$. In other words, the following diagram commutes:
\begin{center}
\begin{tikzcd}
    F(X) \arrow[r, "F(f)"] \arrow[d, "\eta_X"']
    & F(Y) \arrow[d, "\eta_Y"] \\
    G(X) \arrow[r, "G(f)"']
    & G(Y)
\end{tikzcd}
\end{center}

We write such a natural transformation by \hl{$\eta: F \Rightarrow G$}.

If $\eta_X$ is an \CrefAndHyperrefIfExist{definition:isomorphism_in_a_category}{isomorphism} for all objects $X$ of $\calC$, then $\eta$ is said to be a \hldef{natural isomorphism}.
\end{definition}


\subsubsection{Adjoint functors between categories}

\begin{definition} \label{definition:adjoint_functors_between_categories_unit_counit_of_adjoint_functors}
Let $\mathcal{C}$ and $\mathcal{D}$ be \CrefAndHyperrefIfExist{definition:category}{categories}. Let $F : \mathcal{C} \to \mathcal{D}$ and $G : \mathcal{D} \to \mathcal{C}$ be functors. 

An \hldef{adjunction between $F$ and $G$} consists of two \CrefAndHyperrefIfExist{definition:natural_transformation_between_functors_between_categories}{natural transformations}: $\eta : \mathrm{Id}_{\mathcal{C}} \implies GF$ (the \hldef{unit}), and  $\varepsilon : FG \implies \mathrm{Id}_{\mathcal{D}}$ (the \hldef{counit})

These must satisfy the triangle identities: For every object $X \in \mathcal{C}$ 
and $Y \in \mathcal{D}$, 
$$\varepsilon_{FX} \circ F(\eta_X) = \text{id}_{FX}$$
$$G(\varepsilon_Y) \circ \eta_{GY} = \text{id}_{GY}.$$
In diagrammatic form, the triangle identities assert that the following are commutative diagrams:
\begin{center}
\begin{tikzcd}
F(X) \arrow[r, "F(\eta_X)"] \arrow[rd, "\text{id}_{F(X)}"'] & FGF(X) \arrow[d, "\varepsilon_{F(X)}"] \\
& F(X)
\end{tikzcd}
\begin{tikzcd}
G(Y) \arrow[r, "\eta_{G(Y)}"] \arrow[rd, "\text{id}_{G(Y)}"'] & GFG(Y) \arrow[d, "G(\varepsilon_Y)"] \\
& G(Y)
\end{tikzcd}
\end{center}

We say that $F$ is a \hldef{left adjoint to $G$} and $G$ is a \hldef{right adjoint to $F$} (written \hl{$F \dashv G$}). 

% for every object $A$ in $\mathcal{C}$ and $B$ in $\mathcal{D}$ there is a \CrefAndHyperrefIfExist{definition:natural_transformation_between_functors_between_categories}{natural isomorphism}
% \begin{align*}
% \operatorname{Hom}_{\mathcal{D}}(F(A), B) \cong \operatorname{Hom}_{\mathcal{C}}(A, G(B))
% \end{align*}
% that is natural in both $A$ and $B$.


In the case that $\mathcal{C}$ and $\mathcal{D}$ are \CrefAndHyperrefIfExist{definition:locally_small_category}{locally small} categories (or $U$-locally small categories if a \CrefAndHyperrefIfExist{definition:grothendieck_universe}{universe} $U$ is available), we have an adjunction $F \dashv G$ if and only if for every object $X$ in $\mathcal{C}$ and $Y$ in $\mathcal{D}$ there is a \CrefAndHyperrefIfExist{definition:natural_transformation_between_functors_between_categories}{natural isomorphism}
\begin{align*}
\operatorname{Hom}_{\mathcal{D}}(F(X), Y) \cong \operatorname{Hom}_{\mathcal{C}}(X, G(Y))
\end{align*}
that is natural in both $X$ and $Y$. In this case, the \hldef{unit of the adjunction} is the natural transformation $\eta : \mathrm{Id}_{\mathcal{C}} \Rightarrow G F$ such that, 
\begin{enumerate}
    \item for every $X \in \calC$, the morphism $\eta_X: X \to GF(X)$ (each called a \hldef{unit morphism}) in $\calC$ is obtained as the image of $\id_{F(X)}$ via the adjoint isomorphism
    $$\Hom_\calD(F(X), F(X)) \cong \Hom_\calC(X, GF(X)). $$

    \item for every $Y \in \calD$, the morphism $\epsilon_Y: FG(Y) \to Y$ (each called a \hldef{counit morphism}) in $\calD$ is obtained as the image of $\id_{G(Y)}$ via the adjoint isomorphism 
    $$\Hom_\calC(G(Y), G(Y)) \cong \Hom_\calD(FG(Y), Y).$$

\end{enumerate}


% Let $F : \mathcal{C} \to \mathcal{D}$ and $G : \mathcal{D} \to \mathcal{C}$ be functors. 
% $F$ is a \hldef{left adjoint to $G$} and $G$ is a \hldef{right adjoint to $F$} (written \hl{$F \dashv G$}) if for every object $A$ in $\mathcal{C}$ and $B$ in $\mathcal{D}$ there is a \CrefAndHyperrefIfExist{definition:natural_transformation_between_functors_between_categories}{natural isomorphism}
% \begin{align*}
% \operatorname{Hom}_{\mathcal{D}}(F(A), B) \cong \operatorname{Hom}_{\mathcal{C}}(A, G(B))
% \end{align*}
% that is natural in both $A$ and $B$.
\end{definition}

\begin{definition} \label{definition:adjoint_functors_between_categories_unit_counit_of_adjoint_functors}
Let $\mathcal{C}$ and $\mathcal{D}$ be (large) categories. 
Let $F: \mathcal{C} \to \mathcal{D}$ and $G: \mathcal{D} \to \mathcal{C}$ be functors such that $F$ is \hyperrefIfExists{definition:adjoint_functors_between_categories_unit_counit_of_adjoint_functors}{left adjoint to $G$}.

The \hldef{unit of the adjunction} is the natural transformation $\eta : \mathrm{Id}_{\mathcal{C}} \Rightarrow G F$ such that, for every $A \in \calC$, the morphism $\eta_A: A \to GF(A)$ (each called a \hldef{unit morphism}) in $\calC$ is obtained as the image of $\id_{F(A)}$ via the adjoint isomorphism
$$\Hom_\calD(F(A), F(A)) \cong \Hom_\calC(A, GF(A)). $$


The \hldef{counit of the adjunction} is the natural transformation
$\epsilon : F G \Rightarrow \mathrm{Id}_{\mathcal{D}}$, that such that, for every $B \in \calD$, the morphism $\epsilon_B: FG(B) \to B$ (each called a \hldef{counit morphism}) in $\calD$ is obtained as the image of $\id_{G(B)}$ via the adjoint isomorphism 
$$\Hom_\calC(G(B), G(B)) \cong \Hom_\calD(FG(B), B).$$

These transformations satisfy the triangular identities:
\begin{align*}
G(\epsilon_B) \circ \eta_{G(B)} &= \mathrm{id}_{G(B)} \quad \forall B \in \mathrm{Ob}(\mathcal{D}), \\
\epsilon_{F(A)} \circ F(\eta_A) &= \mathrm{id}_{F(A)} \quad \forall A \in \mathrm{Ob}(\mathcal{C}).
\end{align*}
\end{definition}


\begin{proposition}[Factorization through the counit and the unit] \label{proposition:factorization_of_morphisms_through_counit_and_unit_morphisms_for_adjunctions_between_categories}
Let $\mathcal{C}$ and $\mathcal{D}$ be categories, and let 
$$
L : \mathcal{C} \to \mathcal{D}, 
\qquad 
R : \mathcal{D} \to \mathcal{C}
$$ 
be \CrefAndHyperrefIfExist{definition:adjoint_functors_between_categories_unit_counit_of_adjoint_functors}{adjoint functors} with $L \dashv R$. Denote by 
$$ 
\eta : \mathrm{Id}_{\mathcal{C}} \Rightarrow R \circ L, 
\qquad 
\varepsilon : L \circ R \Rightarrow \mathrm{Id}_{\mathcal{D}} 
$$
the \CrefAndHyperrefIfExist{definition:adjoint_functors_between_categories_unit_counit_of_adjoint_functors}{unit and counit of the adjunction}.

\medskip

(1) For every $X' \in \mathrm{Ob}(\mathcal{C})$ and $X \in \mathrm{Ob}(\mathcal{D})$, a morphism 
$$ 
f : L(X') \to X \quad \text{in } \mathcal{D} 
$$
factors uniquely as
$$ 
f = \varepsilon_X \circ L(g), 
$$
for some morphism $g : X' \to R(X)$ in $\mathcal{C}$.
Equivalently, the assignment $g \mapsto \varepsilon_X \circ L(g)$ gives a bijection
$$ 
\Hom_\mathcal{C}(X', R(X)) \;\cong\; \Hom_\mathcal{D}(L(X'), X), 
$$
which is precisely the adjunction isomorphism.

\medskip

(2) Dually, for every $X \in \mathrm{Ob}(\mathcal{D})$ and $X' \in \mathrm{Ob}(\mathcal{C})$, a morphism 
$$ 
h : X \to R(X') \quad \text{in } \mathcal{C}
$$
factors uniquely as
$$ 
h = R(f) \circ \eta_X, 
$$
for some morphism $f : L(X) \to X'$ in $\mathcal{D}$.
Equivalently, the assignment $f \mapsto R(f) \circ \eta_X$ gives a bijection
$$ 
\Hom_\mathcal{D}(L(X), X') \;\cong\; \Hom_\mathcal{C}(X, R(X')), 
$$
again exhibiting the adjunction isomorphism.
\end{proposition}



% \begin{proposition}[Factorization through the counit]
% Let $\mathcal{C}$ and $\mathcal{D}$ be categories, and let 
% $$
% L : \mathcal{C} \to \mathcal{D}, 
% \qquad 
% R : \mathcal{D} \to \mathcal{C}
% $$ 
% be \CrefAndHyperrefIfExist{definition:adjoint_functors_between_categories_unit_counit_of_adjoint_functors}{adjoint functors} with $L \dashv R$. Denote by 
% $$ \eta : \mathrm{Id}_{\mathcal{C}} \Rightarrow R \circ L, \qquad \varepsilon : L \circ R \Rightarrow \mathrm{Id}_{\mathcal{D}} $$
% the \CrefAndHyperrefIfExist{definition:adjoint_functors_between_categories_unit_counit_of_adjoint_functors}{unit and counit of the adjunction}.

% Then for every $X' \in \mathrm{Ob}(\mathcal{C})$ and $X \in \mathrm{Ob}(\mathcal{D})$, a morphism 
% $$ f : L(X') \to X \quad \text{in } \mathcal{D} $$
% factors uniquely as
% $$ f = \varepsilon_X \circ L(g), $$
% for some morphism $g : X' \to R(X)$ in $\mathcal{C}$.

% Equivalently, the assignment $g \mapsto \varepsilon_X \circ L(g)$ gives a bijection
% $$ \Hom_\mathcal{C}(X', R(X)) \;\cong\; \Hom_\mathcal{D}(L(X'), X), $$
% which is precisely the adjunction isomorphism.
% \end{proposition}


\subsection{Diagrams, systems, and limits in categories}

\import{../_excerpts}{excerpts_diagrams_systems_and_limits_in_categories.tex}

\subsection{Miscellaneous categorical constructions and definitions}

\subsubsection{Subcategory of a category}

\begin{definition}[Subcategory] \label{definition:subcategory_of_a_category}
    Let $\mathcal{C}$ be a \CrefAndHyperrefIfExist{definition:category}{(large) category}. A \hldef{subcategory} $\mathcal{D}$ of $\mathcal{C}$ consists of:
    \begin{itemize}
        \item a subclass of objects $\mathrm{Ob}(\mathcal{D}) \subseteq \mathrm{Ob}(\mathcal{C})$,
        \item for each pair of objects $X, Y \in \mathrm{Ob}(\mathcal{D})$, a subclass of morphisms
        $$\mathrm{Hom}_{\mathcal{D}}(X,Y) \subseteq \mathrm{Hom}_{\mathcal{C}}(X,Y),$$
    \end{itemize}
    such that
    \begin{itemize}
        \item for every object $X \in \mathrm{Ob}(\mathcal{D})$, the identity morphism $\mathrm{id}_X$ of $X$ in $\mathcal{C}$ lies in $\mathrm{Hom}_{\mathcal{D}}(X,X)$,
        \item the composition of morphisms in $\mathcal{D}$ is inherited from $\mathcal{C}$ and is closed in $\mathcal{D}$: for morphisms $f \in \mathrm{Hom}_{\mathcal{D}}(X,Y)$ and $g \in \mathrm{Hom}_{\mathcal{D}}(Y,Z)$, their composition $g \circ f \in \mathrm{Hom}_{\mathcal{D}}(X,Z)$.
    \end{itemize}
\end{definition}



\begin{definition}[Full subcategory] \label{definition:full_subcategory_of_a_category}
    Let $\mathcal{C}$ be a \CrefAndHyperrefIfExist{definition:category}{(large) category}. A \hldef{full subcategory} $\mathcal{D}$ of $\mathcal{C}$ is a \CrefAndHyperrefIfExist{definition:subcategory_of_a_category}{subcategory} such that for every pair of objects $X, Y \in \mathrm{Ob}(\mathcal{D})$, the morphism classes coincide:
    $$\mathrm{Hom}_{\mathcal{D}}(X,Y) = \mathrm{Hom}_{\mathcal{C}}(X,Y).$$
    In other words, a full subcategory includes all morphisms between its objects that exist in the ambient category $\mathcal{C}$.
\end{definition}


\begin{definition}[Replete subcategory] \label{definition:replete_subcategory_of_a_category}
Let $\mathcal{C}$ be a \CrefAndHyperrefIfExist{definition:category}{(large) category}.  
A \CrefAndHyperrefIfExist{definition:full_subcategory_of_a_category}{full subcategory} $\mathcal{D} \subseteq \mathcal{C}$ is called \hldef{replete} if it satisfies the following property:
\begin{itemize}
    \item For all $x \in \mathrm{Ob}(\mathcal{C})$ and $y \in \mathrm{Ob}(\mathcal{D})$, if $x \cong y$ in $\mathcal{C}$, then $x \in \mathrm{Ob}(\mathcal{D})$ as well.
\end{itemize}
Equivalently, a replete subcategory of $\mathcal{C}$ is a full subcategory closed under isomorphisms in $\mathcal{C}$.
\end{definition}



\begin{definition}[Category of objects over a fixed object] \label{definition:category_of_objects_over_under_a_fixed_object_in_a_category}
Let $\mathcal{C}$ be a \hyperrefIfExists{definition:category}{category}\CrefIfExists{definition:category} and let $X \in \operatorname{Ob}(\mathcal{C})$ be a fixed object.
\begin{enumerate}
    \item 
        The \hldef{category of objects over $X$} (or synonymously the \hldef{slice category of $X$ in $\calC$} or the \hldef{over category of $X$ in $\calC$}), commonly denoted \hl{$\mathcal{C}/X$}, \hl{$\mathcal{C}_{/X}$}, or \hl{$(\mathcal{C} \downarrow X)$} is the category defined as follows:
        \begin{itemize}
            \item An object of $\mathcal{C}/X$ is a morphism $f \colon A \to X$ in $\mathcal{C}$, where $A \in \operatorname{Ob}(\mathcal{C})$.
            \item A morphism from $f \colon A \to X$ to $g \colon B \to X$ in $\mathcal{C}/X$ is a morphism $h \colon A \to B$ in $\mathcal{C}$ such that the following diagram commutes:
            $$
            \begin{aligned}
            \xymatrix{
            A \ar[dr]_f \ar[r]^h & B \ar[d]^g \\
            & X
            }
            \end{aligned}
            $$
            i.e. such that $g \circ h = f$.
            \item The identity morphisms and composition in $\mathcal{C}/X$ are inherited from $\mathcal{C}$.
        \end{itemize}

    \item 
    The \hldef{category of objects under $X$} (or synonymously the \hldef{coslice category of $X$ in $\calC$} or the \hldef{under category of $X$ in $\calC$}), commonly denoted \hl{$X/\mathcal{C}$}, \hl{$X \backslash \calC$}, \hl{$\mathcal{C}_{X/}$}, or \hl{$(X \downarrow \calC)$}, is the category defined as follows:
    \begin{itemize}
        \item An object of $X/\mathcal{C}$ is a morphism $f \colon X \to A$ in $\mathcal{C}$, where $A \in \operatorname{Ob}(\mathcal{C})$.
        \item A morphism from $f \colon X \to A$ to $g \colon X \to B$ in $X/\mathcal{C}$ is a morphism $h \colon A \to B$ in $\mathcal{C}$ such that the following diagram commutes:
        $$
        \begin{aligned}
        \xymatrix{
        X \ar[dr]^g \ar[r]^f & A \ar[d]^h \\
        & B
        }
        \end{aligned}
        $$
        i.e. such that $h \circ f = g$.
        \item The identity morphisms and composition in $X/\mathcal{C}$ are inherited from $\mathcal{C}$.
    \end{itemize}

\end{enumerate} 
\TextIfExists{definition:comma_category_of_two_functors_to_a_category}{Both notions are special cases of \CrefAndHyperrefIfExist{definition:comma_category_of_two_functors_to_a_category}{comma categories}.}
\end{definition}



\begin{definition}[Replete subcategory] \label{definition:replete_subcategory_of_a_category}
Let $\mathcal{C}$ be a \CrefAndHyperrefIfExist{definition:category}{(large) category}.  
A \CrefAndHyperrefIfExist{definition:full_subcategory_of_a_category}{full subcategory} $\mathcal{D} \subseteq \mathcal{C}$ is called \hldef{replete} if it satisfies the following property:
\begin{itemize}
    \item For all $x \in \mathrm{Ob}(\mathcal{C})$ and $y \in \mathrm{Ob}(\mathcal{D})$, if $x \cong y$ in $\mathcal{C}$, then $x \in \mathrm{Ob}(\mathcal{D})$ as well.
\end{itemize}
Equivalently, a replete subcategory of $\mathcal{C}$ is a full subcategory closed under isomorphisms in $\mathcal{C}$.
\end{definition}


\begin{definition}[Image of a functor] \label{definition:image_of_a_functor_between_categories}
Let $F : \mathcal{C} \to \mathcal{D}$ be a \CrefAndHyperrefIfExist{definition:functor_between_categories}{functor} between \CrefAndHyperrefIfExist{definition:category}{(large) categories}.  
The \hldef{image of $F$} is the subcategory \hl{$\mathrm{Im}(F)$} of $\mathcal{D}$ defined by:
\begin{itemize}
    \item $\mathrm{Ob}(\mathrm{Im}(F)) = \{\, F(c) \mid c \in \mathrm{Ob}(\mathcal{C}) \,\}$,  
    the collection of objects of $\mathcal{D}$ that arise as images of objects of $\mathcal{C}$.
    \item For $F(x),F(y) \in \mathrm{Ob}(\mathrm{Im}(F))$, the hom-set is
    $$ \mathrm{Im}(F)(F(x),F(y)) = \{\, F(f) \mid f \in \mathcal{C}(x,y) \,\}.  $$
    \item Composition and identities are inherited from $\mathcal{D}$ and are well-defined by functoriality of $F$.
\end{itemize}
Thus $\mathrm{Im}(F)$ is the smallest (not necessarily \CrefAndHyperrefIfExist{definition:full_subcategory_of_a_category}{full}) subcategory of $\mathcal{D}$ containing all $F(c)$ for $c \in \mathrm{Ob}(\mathcal{C})$ and all $F(f)$ for $f$ a morphism of $\mathcal{C}$.
\end{definition}


\begin{definition}[Reflective subcategory] \label{definition:reflective_subcategory_of_a_category}
Let $\mathcal{C}$ be a \CrefAndHyperrefIfExist{definition:category}{(large) category} and let $\mathcal{D} \subseteq \mathcal{C}$ be a \CrefAndHyperrefIfExist{definition:full_subcategory_of_a_category}{full subcategory}.  
The subcategory $\mathcal{D}$ is called \hldef{reflective} if the inclusion functor
$$ I : \mathcal{D} \hookrightarrow \mathcal{C} $$
admits a left adjoint
$$ L : \mathcal{C} \to \mathcal{D}.  $$
In this case, $L$ is called the \hldef{reflector}, and for each $c \in \mathrm{Ob}(\mathcal{C})$, the unit morphism of the adjunction
$$ \eta_c : c \to I(L(c)) $$
is called the \hldef{reflection arrow of $c$ into $\mathcal{D}$}.
\end{definition}






\TODO{describe what monomorphisms and epimorphisms are like in additive or abelian categories}

\subsubsection{Types of morphisms in a category}


\begin{definition}[Monomorphism and Epimorphism in Categories] \label{definition:monomorphism_and_epimorphism_in_categories}
Let $\mathcal{C}$ be a \CrefAndHyperrefIfExist{definition:category}{category}. For objects $A, B \in \mathcal{C}$, let $f: A \to B$ be a morphism in $\mathcal{C}$.  
\begin{itemize}
    \item The morphism $f$ is called a \hldef{monomorphism} (or a \hldef{monic morphism}) if for every object $X$ and every pair of morphisms $g_1, g_2 : X \to A$, the equality $f \circ g_1 = f \circ g_2$ implies $g_1 = g_2$.  
    \item The morphism $f$ is called an \hldef{epimorphism} (or an \hldef{epic morphism}) if for every object $Y$ and every pair of morphisms $h_1, h_2: B \to Y$, the equality $h_1 \circ f = h_2 \circ f$ implies $h_1 = h_2$.  
\end{itemize}
\end{definition}



\begin{definition} \label{definition:kernel_and_cokernel_of_a_morphism_in_a_category}
Let $\mathcal{C}$ be a \CrefAndHyperrefIfExist{definition:category}{(large)} \CrefAndHyperrefIfExist{definition:pointed_category}{pointed category}, i.e. a category with a \CrefAndHyperrefIfExist{definition:initial_final_zero_objects_of_a_category}{zero object} $0$. Let $X,Y \in \mathrm{Ob}(\mathcal{C})$ be an object and let $f: X \to Y$ be a morphism. 

\begin{enumerate}
    \item A morphism $i: K \to X$ is called the \hldef{kernel of $f$} if:
    \begin{enumerate}
        \item $f \circ i = 0$, where $0$ is the \CrefAndHyperrefIfExist{definition:zero_morphism_in_a_pointed_category}{zero morphism} $K \to Y$,
        \item for any morphism $g: Z \to X$ such that $f \circ g = 0$, there exists a unique morphism $u: Z \to K$ such that $g = i \circ u$.
    \end{enumerate}
    The kernel, if it exists, is unique up to unique \CrefAndHyperrefIfExist{definition:isomorphism_in_a_category}{isomorphism}. \hl{$\ker(f)$} denotes the object $K$ determined (up to isomorphism) by a kernel of $f$.

    \TextIfExists{definition:equalizer_and_coequalizer_of_morphisms_in_a_category}{
        Equivalently, $\ker(f)$ is the \CrefAndHyperref{definition:equalizer_and_coequalizer_of_morphisms_in_a_category}{equalizer} of $f$ and the $0$ morphism $X \to Y$.
    }

    \item a morphism $p: Y \to Q$ is called the \hldef{cokernel of $f$} if:
    \begin{enumerate}
        \item $p \circ f = 0$, where $0$ is the \CrefAndHyperrefIfExist{definition:initial_final_zero_objects_of_a_category}{zero morphism} $X \to Q$,
        \item for any morphism $g: Y \to Z$ such that $g \circ f = 0$, there exists a unique morphism $v: Q \to Z$ such that $g = v \circ p$.
    \end{enumerate}
    The cokernel, if it exists, is unique up to unique isomorphism. \hl{$\operatorname{coker}(f)$} denotes the object $Q$ determined (up to isomorphism) by a cokernel of $f$.

    \TextIfExists{definition:equalizer_and_coequalizer_of_morphisms_in_a_category}{
        Equivalently, $\coker(f)$ is the \CrefAndHyperref{definition:equalizer_and_coequalizer_of_morphisms_in_a_category}{coequalizer} of $f$ and the $0$ morphism $X \to Y$.
    }

\end{enumerate}

\end{definition}


\subsubsection{Essential image of a functor between categories}


\begin{definition}[Essential image of a functor] \label{definition:essential_image_of_a_functor_between_categories}
Let $F : \mathcal{C} \to \mathcal{D}$ be a functor between \CrefAndHyperrefIfExist{definition:category}{(large) categories}.  
The \hldef{essential image of $F$} is the \CrefAndHyperrefIfExist{definition:full_subcategory_of_a_category}{full subcategory} of $\mathcal{D}$ whose objects are those $d \in \mathrm{Ob}(\mathcal{D})$ for which there exists an object $c \in \mathrm{Ob}(\mathcal{C})$ such that
$$ F(c) \cong d.  $$
\CrefIfExists{definition:isomorphism_in_a_category} Equivalently, the essential image is given by
$$\hlin{\mathrm{EssIm}(F) = \{\, d \in \mathrm{Ob}(\mathcal{D}) \mid \exists c \in \mathrm{Ob}(\mathcal{C}), \, F(c) \cong d \,\},}$$
endowed with all morphisms $\mathcal{D}(d,d')$ between such objects.

\TextIfExists{definition:replete_subcategory_of_a_category}{Equivalently, the essential image of $F$ is the smallest \CrefAndHyperrefIfExist{definition:replete_subcategory_of_a_category}{replete} \CrefAndHyperrefIfExist{definition:full_subcategory_of_a_category}{full subcategory} of $\calD$ containing the \CrefAndHyperrefIfExist{definition:image_of_a_functor_between_categories}{image} of $F$}
\end{definition}




\subsubsection{Types of categories}


\begin{definition}[Complete and Cocomplete Category] \label{definition:complete_and_cocomplete_category}
Let $\mathcal{C}$ be a \CrefAndHyperrefIfExist{definition:category}{category}.  
\begin{itemize}
    \item The category $\mathcal{C}$ is called \hldef{complete} (resp. \hldef{finitely complete}) if all \CrefAndHyperrefIfExist{definition:small_and_finite_limits_and_colimits_in_a_category}{small limits} (resp. finite limits) exist in $\mathcal{C}$; that is, for every small diagram $D : J \to \mathcal{C}$ (with $J$ a small category), the limit $\lim D$ exists and is an object of $\mathcal{C}$.
    \item The category $\mathcal{C}$ is called \hldef{cocomplete} (resp. \hldef{finitely cocomplete}) if all \CrefAndHyperrefIfExist{definition:small_and_finite_limits_and_colimits_in_a_category}{small colimits} (resp. finite colimits) exist in $\mathcal{C}$; that is, for every small diagram $D : J \to \mathcal{C}$, the colimit $\mathrm{colim}\ D$ exists and is an object of $\mathcal{C}$.
\end{itemize}
\end{definition}


\subsubsection{Types of functors}


\begin{definition} \label{definition:full_and_faithful_functor_between_locally_small_categories}

Let $\mathcal{C}$ and $\mathcal{D}$ be \CrefAndHyperrefIfExist{definition:category}{(large)) categories}. Let $F : \mathcal{C} \to \mathcal{D}$ be a \CrefAndHyperrefIfExist{definition:functor_between_categories}{functor}. 
\begin{enumerate}
    \item $F$ is called \hldef{full} if for every pair of objects $x,y \in \mathrm{Ob}(\mathcal{C})$, the induced rule/assignment/class function
    $$ F_{x,y} : \Hom_\mathcal{C}(x,y) \to \Hom_\mathcal{D}(F(x), F(y)) $$
    on Hom-collections is ``surjective'', i.e. for all morphisms $g:F(x) \to F(y)$, there exists some morphism $f: x \to y$ such that $F(f) = g$. 

    \item $F$ is called \hldef{faithful} if for every pair of objects $x,y \in \mathrm{Ob}(\mathcal{C})$, 
    the induced class function (assignment)
    $$ F_{x,y} : \mathrm{Hom}_\mathcal{C}(x,y) \to \mathrm{Hom}_\mathcal{D}(F(x), F(y)) $$
    on Hom-collections is ``injective'', i.e., for any morphisms $f_1, f_2 \in \mathrm{Hom}_\mathcal{C}(x,y)$, 
    if $F(f_1) = F(f_2)$ in $\mathrm{Hom}_\mathcal{D}(F(x), F(y))$, then $f_1 = f_2$.

    \item $F$ is called \hldef{fully faithful} if it is both full and faithful.
\end{enumerate}

\end{definition}



% \input{../_definitions/definition_reflects_isomorphisms_for_a_functor_between_categories.tex}

\begin{definition}[Reflecting a type of morphism] \label{definition:reflects_a_type_of_morphism_for_a_functor_between_categories}
Let $F : \mathcal{C} \to \mathcal{D}$ be a \CrefAndHyperrefIfExist{definition:functor_between_categories}{functor between (large) categories}, and let $\mathcal{P}$ be a property of morphisms (or more generally a property of sequences or families of morphisms) that is stable under \CrefAndHyperrefIfExist{definition:isomorphism_in_a_category}{isomorphism} (e.g. \CrefAndHyperrefIfExist{definition:monomorphism_and_epimorphism_in_categories}{monomorphism, epimorphism}, isomorphism, etc.). We say that $F$ \hldef{reflects $\mathcal{P}$-morphisms} if for every morphism $f : x \to y$ in $\mathcal{C}$, whenever $F(f)$ has property $\mathcal{P}$ in $\mathcal{D}$, it follows that $f$ has property $\mathcal{P}$ in $\mathcal{C}$.
\end{definition}


\begin{proposition}[Reflection of morphism properties by full or faithful functors] \label{proposition:reflection_of_monomorphism_and_epimorphisms_by_faithful_functors}
Let $F : \mathcal{C} \to \mathcal{D}$ be a \CrefAndHyperrefIfExist{definition:functor_between_categories}{functor} between \CrefAndHyperrefIfExist{definition:locally_small_category}{locally small categories}.

(1) If $F$ is \CrefAndHyperrefIfExist{definition:full_and_faithful_functor_between_locally_small_categories}{faithful}, then $F$ \CrefAndHyperrefIfExist{definition:reflects_a_type_of_morphism_for_a_functor_between_categories}{reflects} \CrefAndHyperrefIfExist{definition:monomorphism_and_epimorphism_in_categories}{monomorphisms and epimorphisms}.  
That is, if $f : x \to y$ in $\mathcal{C}$ is such that $F(f)$ is a monomorphism (resp. epimorphism) in $\mathcal{D}$, then $f$ is a monomorphism (resp. epimorphism) in $\mathcal{C}$.

\TODO{define split monos and epis}
(2) If $F$ is fully faithful, then $F$ reflects isomorphisms, split monomorphisms, and split epimorphisms.  
That is, if $f : x \to y$ in $\mathcal{C}$ is such that $F(f)$ is an isomorphism in $\mathcal{D}$, then $f$ is an isomorphism in $\mathcal{C}$.

\end{proposition}



\subsubsection{Categories constructed from other categories}

\begin{definition}[Ind-category] \label{definition:ind_pro_category_of_a_locally_small_category}
Let $\mathcal{C}$ be a \CrefAndHyperrefIfExist{definition:locally_small_category}{locally small category}.

\begin{enumerate}
    \item  The \hldef{Ind-category of $\mathcal{C}$}, denoted \hl{$\mathrm{Ind}(\mathcal{C})$}, is defined as follows:
    \begin{itemize}
        \item Objects of $\mathrm{Ind}(\mathcal{C})$ are formal \CrefAndHyperrefIfExist{definition:projective_and_inductive_limits_in_categories}{filtered colimits} of objects in $\mathcal{C}$. More precisely, an object is given by a \CrefAndHyperrefIfExist{definition:filtered_cofiltered_category}{filtered} small category $I$ and a functor 
        $$ X : I \to \mathcal{C}.  $$
        \item Morphisms between objects $X : I \to \mathcal{C}$ and $Y : J \to \mathcal{C}$ are defined by
        $$ \mathrm{Hom}_{\mathrm{Ind}(\mathcal{C})}(X,Y) \;:=\; \varprojlim_{i \in I} \varinjlim_{j \in J} \mathrm{Hom}_{\mathcal{C}}(X_i, Y_j), $$
        \CrefIfExists{definition:projective_and_inductive_limits_in_categories}
        where $X_i$ and $Y_j$ denote the images of $i \in I$ and $j \in J$ under $X$ and $Y$, respectively.
    \end{itemize}
    The composition of morphisms is induced naturally from composition in $\mathcal{C}$.  
    Hence, $\mathrm{Ind}(\mathcal{C})$ is the completion of $\mathcal{C}$ under filtered colimits. Objects of $\mathrm{Ind}(\mathcal{C})$ are called \hldef{Ind-objects of $\calC$}.
    
    \item 
    The \hldef{Pro-category of $\mathcal{C}$}, denoted \hl{$\mathrm{Pro}(\mathcal{C})$}, is defined as follows:
    \begin{itemize}
        \item Objects of \(\mathrm{Pro}(\mathcal{C})\) are formal \CrefAndHyperrefIfExist{definition:projective_and_inductive_limits_in_categories}{cofiltered limits} of objects in \(\mathcal{C}\). More precisely, an object is given by a \CrefAndHyperrefIfExist{definition:cofiltered_cofiltered_category}{cofiltered} small category \(I\) and a functor
        \[
        X : I \to \mathcal{C}.
        \]
        \item Morphisms between objects \(X : I \to \mathcal{C}\) and \(Y : J \to \mathcal{C}\) are defined by
        \[
        \mathrm{Hom}_{\mathrm{Pro}(\mathcal{C})}(X,Y) := \varinjlim_{j \in J} \varprojlim_{i \in I} \mathrm{Hom}_{\mathcal{C}}(X_i, Y_j),
        \]
        where \(X_i\) and \(Y_j\) denote the images of \(i \in I\) and \(j \in J\) under \(X\) and \(Y\), respectively.
    \end{itemize}
    The composition of morphisms is induced naturally from composition in \(\mathcal{C}\).

    Hence, \(\mathrm{Pro}(\mathcal{C})\) is the completion of \(\mathcal{C}\) under cofiltered limits. Objects of $\mathrm{Pro}(\mathcal{C})$ are called \hldef{Pro-objects of $\calC$}.


\end{enumerate}

    Since $\Sets$ has all limits and colimits \TODO{} and hence has all projective and inductive limits and since $\calC$ is locally small, $\mathrm{Ind}(\calC)$ and $\mathrm{Pro}(\calC)$ are locally small.

\end{definition}


\section{Monoidal categories}
\TODO{Carefully check what statements need to assume that the categories are locally small and which do not}

\begin{definition} \label{definition:monoidal_category}
A \hldef{monoidal category} is a \CrefAndHyperrefIfExist{definition:category}{(large) category} $\mathcal{C}$ equipped with:
\begin{itemize}
    \item a \CrefAndHyperrefIfExist{definition:n_ary_functor}{bifunctor} $\otimes : \mathcal{C} \times \mathcal{C} \to \mathcal{C}$\CrefIfExists{definition:product_category_of_a_family_of_categories} (called the \hldef{tensor product});
    \item an object $\mathbb{I} \in \mathrm{Ob}(\mathcal{C})$ (often called the \hldef{unit object}); common notations for the unit object include \hl{$\mathbb{I}$} and \hl{$\mathds{1}$}.
    \item natural isomorphisms (\hldef{associator}) $\alpha_{X,Y,Z} : (X \otimes Y) \otimes Z \to X \otimes (Y \otimes Z)$ for all $X, Y, Z \in \mathcal{C}$;
    \item natural isomorphisms (\hldef{left and right unitors}) $\lambda_X : \mathbb{I} \otimes X \to X$, $\rho_X : X \otimes \mathbb{I} \to X$ for all $X \in \mathcal{C}$;
\end{itemize}
% \TODO{TODO: add the pentagon and triangle coherence diagrams}

% such that the pentagon and triangle coherence diagrams commute.

such that the following coherence diagrams commute:

\bigskip

\textbf{Pentagon coherence:} For all $W,X,Y,Z \in \mathcal{C}$, the diagram
\[
\begin{tikzcd}[column sep=huge]
((W \otimes X) \otimes Y) \otimes Z \arrow[r, "\alpha_{W\otimes X, Y, Z}"] \arrow[d, "\alpha_{W,X,Y} \otimes \mathrm{id}_Z"'] & (W \otimes X) \otimes (Y \otimes Z) \arrow[r, "\alpha_{W,X,Y \otimes Z}"] & W \otimes (X \otimes (Y \otimes Z)) \\
(W \otimes (X \otimes Y)) \otimes Z \arrow[rr, "\alpha_{W,X \otimes Y, Z}"'] & & W \otimes ((X \otimes Y) \otimes Z) \arrow[u, "\mathrm{id}_W \otimes \alpha_{X,Y,Z}"'].
\end{tikzcd}
\]

\bigskip

\textbf{Triangle coherence:} For all $X,Y \in \mathcal{C}$, the diagram
\[
\begin{tikzcd}[column sep=large]
(X \otimes \mathbb{I}) \otimes Y \arrow[r, "\alpha_{X, \mathbb{I}, Y}"] \arrow[dr, "\rho_X \otimes \mathrm{id}_Y"'] & X \otimes (\mathbb{I} \otimes Y) \arrow[d, "\mathrm{id}_X \otimes \lambda_Y"] \\
& X \otimes Y.
\end{tikzcd}
\]
\end{definition}


\begin{definition} \label{definition:internal_hom_object_in_a_category}
Let $(\mathcal{C}, \otimes, \mathbb{I})$ be a \hyperrefIfExists{definition:monoidal_category}{monoidal category}. Given objects $X, Y \in \mathrm{Ob}(\mathcal{C})$, an \hldef{internal hom object from $X$ to $Y$} is an object $\underline{\mathrm{Hom}}(X,Y) \in \mathrm{Ob}(\mathcal{C})$ together with a morphism
$$\hlin{\mathrm{ev}_{X,Y}: \underline{\mathrm{Hom}}(X,Y) \otimes X \to Y}$$

such that for every object $Z \in \mathcal{C}$, the assignment
\[
\hom_{\mathcal{C}}(Z, \underline{\mathrm{Hom}}(X,Y)) \to \hom_{\mathcal{C}}(Z \otimes X, Y), \quad f \mapsto \mathrm{ev}_{X,Y} \circ (f \otimes \mathrm{id}_X)
\]
is a natural isomorphism of sets.
\end{definition}



% \begin{definition} \label{definition:closed_category}
% A category $\mathcal{C}$ equipped with a bifunctor
% \TODO{TODO: define opposite category, product category}
% \[
% \hlin{\otimes : \mathcal{C}^{\mathrm{op}} \times \mathcal{C} \to \mathcal{C}}
% \]
% is called a \hldef{closed category} if for every pair of objects $X, Y$ in $\mathcal{C}$, the internal hom object $\underline{\mathrm{Hom}}(X,Y)$ exists.
% \end{definition}

% \begin{definition} \label{definition:closed_monoidal_category}
% A \hldef{closed monoidal category} is a \hyperrefIfExists{definition:monoidal_category}{monoidal category} $(\mathcal{C}, \otimes, \mathbb{I})$ such that $\mathcal{C}$ is a \hyperrefIfExists{definition:closed_category}{closed category} with respect to the tensor product $\otimes$, i.e., for every $X,Y \in \mathrm{Ob}(\mathcal{C})$, there exists an internal hom object
% \[
% \hlin{\underline{\mathrm{Hom}}(X,Y)}
% \]
% with evaluation map
% \[
% \hlin{\mathrm{ev}_{X,Y} : \underline{\mathrm{Hom}}(X,Y) \otimes X \to Y.}
% \]
% The isomorphisms 
% \[
% \hom_{\mathcal{C}}(Z, \underline{\mathrm{Hom}}(X,Y)) \cong \hom_{\mathcal{C}}(Z \otimes X, Y)
% \]
% natural in $Z$, $X$, and $Y$ must hold.
% \end{definition}

\begin{definition} \label{definition:closed_category}
A \hldef{closed category} is a category $\mathcal{C}$ equipped with the following data:
\begin{itemize}
    \item A functor 
    $$
    \hlin{\underline{\mathrm{Hom}} : \mathcal{C}^{\mathrm{op}} \times \mathcal{C} \to \mathcal{C}}
    $$
    called the \hldef{internal hom-functor}.
    \item A \hldef{left evaluation morphism} 
    $$
    \hlin{j_X : I \to \underline{\mathrm{Hom}}(X,X)}
    $$
    for each object $X \in \mathcal{C}$, where \hl{$I$} is a fixed object serving as a "unit" of the internal hom-structure.
    \item A family of natural \hldef{left composition morphisms}
    $$ \hlin{L^X_{Y,Z} : \underline{\mathrm{Hom}}(Y,Z) \otimes \underline{\mathrm{Hom}}(X,Y) \to \underline{\mathrm{Hom}}(X,Z)} $$
    \TODO{TODO: the axioms}
    satisfying appropriate associativity and unit axioms that generalize composition.
\end{itemize}
\end{definition}
\begin{definition} \label{definition:closed_monoidal_category}
A \hldef{closed monoidal category} is a \CrefAndHyperrefIfExist{definition:monoidal_category}{monoidal category} that is a \CrefAndHyperrefIfExist{definition:closed_category}{closed category} where the \CrefAndHyperrefIfExist{definition:internal_hom_object_in_a_category}{internal hom} is right adjoint to the monoidal tensor.

In other words, for all objects $Y, Z \in \mathrm{Ob}(\mathcal{C})$ there is a natural isomorphism 
$$ \hom_{\mathcal{C}}(Y \otimes X, Z) \cong \hom_{\mathcal{C}}(Y, \underline{\mathrm{Hom}}(X,Z)) $$
of Hom-sets natural in $Y$ and $Z$.
\end{definition}

\begin{definition} \label{definition:symmetric_monoidal_category}
A \hldef{symmetric monoidal category} is a \hyperrefIfExists{definition:monoidal_category}{monoidal category} $(\mathcal{C}, \otimes, \mathbb{I})$ together with a natural isomorphism (symmetry)
$$\hlin{\gamma_{X,Y}: X \otimes Y \xrightarrow{\cong} Y \otimes X}$$
for all $X, Y \in \mathcal{C}$, such that for all $X, Y, Z \in \mathcal{C}$ the following holds:
\begin{itemize}
    \item $\gamma_{Y,X} \circ \gamma_{X,Y} = \mathrm{id}_{X \otimes Y}$ (involutivity);
    % % \TODO{TODO: add the hexagon and symmetry coherence diagrams}
    % \item the hexagon and symmetry coherence diagrams commute.
        \item the \textbf{hexagon coherence diagrams} commute:
        \[
        \begin{tikzcd}[column sep=large]
        (X \otimes Y) \otimes Z \arrow[r, "\alpha_{X,Y,Z}"] \arrow[d, "\gamma_{X,Y} \otimes \mathrm{id}_Z"'] & X \otimes (Y \otimes Z) \arrow[r, "\gamma_{X, Y \otimes Z}"] & (Y \otimes Z) \otimes X \\
        (Y \otimes X) \otimes Z \arrow[rr, "\alpha_{Y,X,Z}"'] & & Y \otimes (X \otimes Z) \arrow[u, "\mathrm{id}_Y \otimes \gamma_{X,Z}"']
        \end{tikzcd}
        \]
        and the analogous hexagon with inverse braiding:
        \[
        \begin{tikzcd}[column sep=large]
        X \otimes (Y \otimes Z) \arrow[r, "\alpha^{-1}_{X,Y,Z}"] \arrow[d, "\mathrm{id}_X \otimes \gamma_{Y,Z}"'] & (X \otimes Y) \otimes Z \arrow[r, "\gamma_{X \otimes Y, Z}"] & Z \otimes (X \otimes Y) \\
        X \otimes (Z \otimes Y) \arrow[rr, "\alpha^{-1}_{X,Z,Y}"'] & & (X \otimes Z) \otimes Y \arrow[u, "\gamma_{X,Z} \otimes \mathrm{id}_Y"']
        \end{tikzcd}
        \]
    \item the \textbf{symmetry coherence diagram} commutes:
        \[
        \begin{tikzcd}
        X \otimes Y \arrow[r, "\gamma_{X,Y}"] \arrow[dr, swap, "\mathrm{id}_{X \otimes Y}"] & Y \otimes X \arrow[d, "\gamma_{Y,X}"] \\
        & X \otimes Y
        \end{tikzcd}
        \]
\end{itemize}
A \hldef{closed symmetric monoidal category} usually refers to a symmetric monoidal category that is \hyperrefIfExists{definition:closed_monoidal_category}{closed as a monoidal category}. 
\end{definition}

% \section{Tannakian formalism}
% \input{../_definitions/definition_k_linear_category_over_a_field.tex}
% \begin{definition} \label{definition:rigid_category}
A \hldef{rigid category} is a \hyperrefIfExists{definition:monoidal_category}{monoidal category} $(\mathcal{C}, \otimes, \mathbb{I})$ in which every object $X \in \mathcal{C}$ admits a \hldef{left dual} \hl{$X^{\vee}$} and a \hldef{right dual} \hl{${}^{\vee}\!X$}, i.e., there exist objects and morphisms
\hlalign{
\begin{align*}
&\mathrm{ev}_X: X^{\vee} \otimes X \to \mathbb{I}, \qquad \mathrm{coev}_X: \mathbb{I} \to X \otimes X^{\vee}, \\
&\widetilde{\mathrm{ev}}_X: X \otimes {}^{\vee}\!X \to \mathbb{I}, \qquad \widetilde{\mathrm{coev}}_X: \mathbb{I} \to {}^{\vee}\!X \otimes X,
\end{align*}
}
% \TODO{TODO: add the zig-zag identities}
% satisfying the standard zig-zag identities (triangle identities) for duals.

satisfying the \hldef{zig-zag identities} (triangle identities), i.e., the following compositions are the respective identity morphisms:

\bigskip

\noindent\textbf{Left dual zig-zag identities:}
\[
\begin{tikzcd}
X \arrow[r, "\cong"] \arrow[rr, bend left=40, "\mathrm{id}_X"] & 
X \otimes \mathbb{I} \arrow[r, "\mathrm{id}_X \otimes \mathrm{coev}_X"] & 
X \otimes (X^{\vee} \otimes X) \arrow[r, "\alpha_{X, X^\vee, X}"] & 
(X \otimes X^{\vee}) \otimes X \arrow[r, "\mathrm{ev}_X \otimes \mathrm{id}_X"] & 
\mathbb{I} \otimes X \arrow[r, "\cong"] & X
\end{tikzcd}
\]
and
\[
\begin{tikzcd}
X^{\vee} \arrow[r, "\cong"] \arrow[rr, bend left=40, "\mathrm{id}_{X^\vee}"] & 
\mathbb{I} \otimes X^{\vee} \arrow[r, "\mathrm{coev}_X \otimes \mathrm{id}_{X^\vee}"] & 
(X^{\vee} \otimes X) \otimes X^{\vee} \arrow[r, "\alpha^{-1}_{X^\vee, X, X^\vee}"] & 
X^{\vee} \otimes (X \otimes X^{\vee}) \arrow[r, "\mathrm{id}_{X^\vee} \otimes \mathrm{ev}_X"] & 
X^{\vee} \otimes \mathbb{I} \arrow[r, "\cong"] & X^{\vee}
\end{tikzcd}
\]

\bigskip

\noindent\textbf{Right dual zig-zag identities:}
\[
\begin{tikzcd}
X \arrow[r, "\cong"] \arrow[rr, bend left=40, "\mathrm{id}_X"] & 
\mathbb{I} \otimes X \arrow[r, "\widetilde{\mathrm{coev}}_X \otimes \mathrm{id}_X"] & 
({}^{\vee}\!X \otimes X) \otimes X \arrow[r, "\alpha^{-1}_{{}^{\vee}\!X, X, X}"] & 
{}^{\vee}\!X \otimes (X \otimes X) \arrow[r, "\mathrm{id}_{{}^{\vee}\!X} \otimes \widetilde{\mathrm{ev}}_X"] & 
{}^{\vee}\!X \otimes \mathbb{I} \arrow[r, "\cong"] & X
\end{tikzcd}
\]
and
\[
\begin{tikzcd}
{}^{\vee}\!X \arrow[r, "\cong"] \arrow[rr, bend left=40, "\mathrm{id}_{{}^{\vee}\!X}"] & 
{}^{\vee}\!X \otimes \mathbb{I} \arrow[r, "\mathrm{id}_{{}^{\vee}\!X} \otimes \widetilde{\mathrm{coev}}_X"] & 
{}^{\vee}\!X \otimes ({}^{\vee}\!X \otimes X) \arrow[r, "\alpha_{{}^{\vee}\!X, {}^{\vee}\!X, X}"] & 
({}^{\vee}\!X \otimes {}^{\vee}\!X) \otimes X \arrow[r, "\widetilde{\mathrm{ev}}_X \otimes \mathrm{id}_X"] & 
\mathbb{I} \otimes X \arrow[r, "\cong"] & {}^{\vee}\!X
\end{tikzcd}
\]

It is reasonable to call $\mathrm{ev}_X$ the \hldef{evaluation map for the left dual}, $\widetilde{\mathrm{ev}}_X$ the \hldef{evaluation map for the right dual}, $\mathrm{coev}_X$ the \hldef{coevaluation map for the left dual}, and $\widetilde{\mathrm{coev}}_X$ the \hldef{coevaluation map for the right dual}. 

\end{definition}

\subsubsection{Types of objects in a category}

\begin{definition} \label{definition:initial_final_zero_object_in_a_category}
Let $\mathcal{C}$ be a \CrefAndHyperrefIfExist{definition:category}{category}.

\begin{enumerate}
    \item An object $I \in \mathcal{C}$ is called an \hldef{initial object} if for every object $X \in \mathcal{C}$ there exists a unique morphism
    $$I \to X.$$

    \item An object $F \in \mathcal{C}$ is called a \hldef{final object} (or \hldef{terminal object}) if for every object $X \in \mathcal{C}$ there exists a unique morphism
    $$X \to F.$$

    \item An object $Z \in \mathcal{C}$ is called a \hldef{zero object} if $Z$ is both initial and final in $\mathcal{C}$. In particular, for every object $X \in \mathcal{C}$ there exist unique morphisms
    $$Z \to X \quad \text{and} \quad X \to Z.$$
\end{enumerate}
\end{definition}



\section{Additive and abelian categories}

\TODO{Carefully check what statements need to assume that the categories are locally small and which do not}
\begin{definition}[Additive category] \label{definition:additive_category}
Let $\mathcal{A}$ be a \CrefAndHyperrefIfExist{definition:locally_small_category}{locally small category}. 
\begin{enumerate}
    \item $\calA$ is said to be a \hldef{preadditive category} if the following hold:
    \begin{itemize}
        \item For any two objects $A, B$ in $\mathcal{A}$, the set $\operatorname{Hom}_{\mathcal{A}}(A, B)$ is an \CrefAndHyperrefIfExist{definition:group}{abelian group}, and composition of morphisms is bilinear.
        \item There is a \CrefAndHyperrefIfExist{definition:initial_final_zero_objects_of_a_category}{zero object} $0$ in $\mathcal{A}$.
    \end{itemize}
    \TextIfExists{definition:category_enriched_in_a_monoidal_category}{Equvialently, a preadditive cateogry $\calA$ is a (necessarily locally small) category \CrefAndHyperrefIfExist{definition:category_enriched_in_a_monoidal_category}{enriched in} the \CrefAndHyperrefIfExist{definition:monoidal_category}{monoidal category} $\Ab$ that also possesses a zero object.}

    \item
    If $\calA$ is preadditive, then it is called \hldef{additive} if it additionally satisfies the following:
    \begin{itemize}
        \item For any two objects $A, B$ in $\mathcal{A}$, there exists a \CrefAndHyperrefIfExist{definition:product_and_coproduct_of_objects_in_a_category}{product object $A \times B$}, often written \hl{$A \oplus B$}, called the \hldef{direct sum of $A$ and $B$}. In fact, $A \oplus B$ is not only a product but also a \CrefAndHyperrefIfExist{definition:coproduct_of_modules_of_rings}{coproduct} of $A$ and $B$\CrefIfExists{lemma:finite_products_and_finite_coproducts_coincide_in_preadditive_categories}.
    \end{itemize}

    Given a finite collection $\{A_i\}_i$ of objects $A_i$ in an additive category $\calA$, we may more generally speak of the \hldef{direct sum} \hl{$\bigoplus_i A_i$}; it has canonical injections from and projections to each $A_i$.


\end{enumerate}
\end{definition}

\begin{definition}[Additive functor] \label{definition:additive_functor_between_additive_categories}

    \begin{enumerate}
        \item Let $\mathcal{A}$ and $\mathcal{B}$ be \hyperrefIfExists{definition:additive_category_preadditive_category}{pre-additive categories}. A functor
        $$ F: \mathcal{A} \to \mathcal{B} $$
        is an \hldef{additive functor} if for every pair of objects $A, A' \in \mathcal{A}$, the induced map
        $$ F_{A,A'}: \operatorname{Hom}_{\mathcal{A}}(A, A') \to \operatorname{Hom}_{\mathcal{B}}(F(A), F(A')) $$
        is a group homomorphism of abelian groups, or equvialently if it is \CrefAndHyperrefIfExist{definition:category_enriched_in_a_monoidal_category}{enriched over the category $\Ab$ of abelian groups}.
        
        \item Let $\mathcal{A}$ and $\mathcal{B}$ be \hyperrefIfExists{definition:additive_category_preadditive_category}{additive categories}. A functor
        $$ F: \mathcal{A} \to \mathcal{B} $$
        is an \hldef{additive functor} if it an additive functor of pre-additive categories and satisfies the following:
        \begin{itemize}
            \item $F$ sends the zero object $0_{\mathcal{A}}$ of $\mathcal{A}$ to the zero object $0_{\mathcal{B}}$ of $\mathcal{B}$, i.e.,
            $$ F(0_{\mathcal{A}}) = 0_{\mathcal{B}}.  $$
            \item $F$ preserves finite direct sums: For any finite family of objects $\{A_i\}_{i=1}^n$ in $\mathcal{A}$,
            $$ F\left(\bigoplus_{i=1}^n A_i\right) \cong \bigoplus_{i=1}^n F(A_i) $$
            via the canonical isomorphism induced by $F$ applied to the canonical injections and projections.
        \end{itemize}
        In other words, $F$ is a functor that is compatible with the additive structures on $\mathcal{A}$ and $\mathcal{B}$.
    \end{enumerate}
\end{definition}

\begin{definition}[Abelian category] \label{definition:abelian_category}
Let $\mathcal{A}$ be a category. The category $\mathcal{A}$ is an \hldef{abelian category} if:
\begin{itemize}
    \item $\mathcal{A}$ is an \CrefAndHyperrefIfExist{definition:additive_category_preadditive_category}{additive category}.

    \item Every morphism $f: A \to B$ has a \CrefAndHyperrefIfExist{definition:kernel_and_cokernel_of_a_morphism_in_a_category}{kernel $\ker(f)$ and a cokernel $\operatorname{coker}(f)$}.

    \item For every morphism $f: A \to B$, the canonical morphism $\operatorname{coim}(f) \to \operatorname{im}(f)$ is an isomorphism, where
    $$
    \operatorname{coim}(f) = \operatorname{coker}(\ker(f) \to A),\quad \operatorname{im}(f) = \ker(B \to \operatorname{coker}(f)).
    $$
    \TODO{I think I need to re-check this defintion}
    \TODO{coimage}
\end{itemize}

\TextIfExists{definition:pre_abelian_category}{In particular, every abelian category is \Cref{definition:pre_abelian_category}{pre-abelian}}.

It is also worth considering Grothendieck's additional axioms for abelian categories\CrefIfExists{definition:grothendiecks_additional_axioms_for_abelian_categories}.

\end{definition}

% 
\begin{definition} \label{definition:length_of_an_object_of_an_abelian_category}
Let $\mathcal{A}$ be an \hyperrefIfExists{definition:abelian_category}{abelian category} and $X \in \mathrm{Ob}(\mathcal{A})$. The \hldef{length of $X$} is the supremum of the lengths $n$ of chains of subobjects
\[ 0 = X_0 \subsetneq X_1 \subsetneq \cdots \subsetneq X_n = X, \]
where each inclusion is strict. If the supremum is finite, $X$ is said to have \hldef{finite length}.
Common notaions for the length of $X$ include: \hl{$\mathrm{length} X$}, \hl{$\mathrm{len} X$}.
\end{definition}

\subsection{Simple and semisimple objects in an additive category}


\begin{definition} \label{definition:subobject_of_an_object_of_an_additive_category}
Let $\mathcal{C}$ be an \CrefAndHyperrefIfExist{definition:additive_category}{additive category}. Let $X \in \Ob(\calC)$ be an object. 
A \hldef{subobject of $X$} refers to a \CrefAndHyperrefIfExist{definition:monomorphism_and_epimorphism_in_categories}{monomorphism} $i: Y \hookrightarrow X$ in $\mathcal{C}$. We regard two subobjects $(Y,i)$ and $(Y',i')$ of $X$ as isomorphic if there exists an isomorphism $f: Y \to Y'$ such that $i = i' \circ f$. One often leaves the monomorphism $i$ implicit, suprressing it from the notation.
\end{definition}


Unlike the notion of a subobject, the notion of a quotient object is more appropriate to speak of for an abelian category rather than a more general additive category.


\begin{definition} \label{definition:quotient_object_of_an_object_of_an_abelian_category_by_a_subobject}
Let $\mathcal{C}$ be an \CrefAndHyperrefIfExist{definition:abelian_category}{abelian category}. Let $X \in \Ob(\calC)$ be an object. Let $i: A \hookrightarrow X$ be a \CrefAndHyperrefIfExist{definition:subobject_of_an_object_of_an_additive_category}{subobject}. The \CrefAndHyperrefIfExist{definition:kernel_and_cokernel_of_a_morphism_in_a_category}{cokernel} $\pi: X \twoheadrightarrow X/A := \operatorname{coker}(i)$ is called the \hldef{quotient object of $X$ by $A$}. The object $X/A$ is determined up to canonical isomorphism.
\end{definition}


\begin{definition} \label{definition:subquotient_of_an_object_in_an_abelian_category}
Let $\mathcal{C}$ be an \CrefAndHyperrefIfExist{definition:abelian_category}{abelian category}. Let $X \in \mathrm{Ob}(\mathcal{C})$, and let $A \hookrightarrow B \hookrightarrow X$ be subobjects of $X$. The \CrefAndHyperrefIfExist{definition:quotient_object_of_an_object_of_an_abelian_category_by_a_subobject}{quotient object} $B/A := \operatorname{coker}(A \hookrightarrow B)$ is called a \hldef{subquotient object of $X$}. 
\end{definition}


\begin{definition} \label{definition:direct_summand_of_an_object_of_an_additive_category}
Let $\mathcal{C}$ be an \CrefAndHyperrefIfExist{definition:additive_category}{additive category} and $X \in \mathrm{Ob}(\mathcal{C})$. A \CrefAndHyperrefIfExist{definition:subcategory_of_a_category}{subobject} $i: Y \hookrightarrow X$ is called a \hldef{direct summand of $X$} if there exists a morphism $p: X \to Y$ such that $p \circ i = \mathrm{id}_Y$. In this case, $X$ is isomorphic to the \CrefAndHyperrefIfExist{definition:additive_category}{direct sum} $Y \oplus Z$ for some object $Z$, with $i$ the canonical inclusion.
\end{definition}


\begin{definition} \label{definition:simple_object_of_an_additive_category}
Let $\mathcal{C}$ be an \CrefAndHyperrefIfExist{definition:additive_category}{additive category}. An object $S \in \mathrm{Ob}(\mathcal{C})$ is called \hldef{simple} (or \hldef{irreducible}) if $S \neq 0$ and the only \CrefAndHyperrefIfExist{definition:subobject_of_an_object_of_an_additive_category}{subobjects} of $S$ are $0$ and $S$ itself (up to isomorphism of subobjects).
\end{definition}


\begin{definition} \label{definition:semisimple_object_of_an_additive_category}
Let $\mathcal{C}$ be an \CrefAndHyperrefIfExist{definition:additive_category_preadditive_category}{additive category}. An object $X \in \mathrm{Ob}(\mathcal{C})$ is called \hldef{semisimple} if it is isomorphic to a finite \CrefAndHyperrefIfExist{definition:additive_category_preadditive_category}{direct sum} of \CrefAndHyperrefIfExist{definition:simple_object_of_an_additive_category}{simple objects} in $\mathcal{C}$.
\end{definition}


\begin{definition} \label{definition:semisimple_additive_category}
An \CrefAndHyperrefIfExist{definition:additive_category_preadditive_category}{additive category} $\mathcal{C}$ is called a \hldef{semisimple category} if every object of $\mathcal{C}$ is \CrefAndHyperrefIfExist{definition:semisimple_object_of_an_additive_category}{semisimple}.
\end{definition}






\TODO{TODO: Give examples of monoidal categoriess}

\section{Categories enriched in monoidal categories}

\TODO{Carefully check what statements need to assume that the categories are locally small and which do not}

We might want to study (locally small) categories whose hom-sets carry additional structure beyond that of a set (e.g. an abelian group, vector space); the theory of categories enriched in monoidal categories accommodates such desires.

\begin{definition}[Category enriched in a monoidal category] \label{definition:category_enriched_in_a_monoidal_category}
Let $(\mathcal{V}, \otimes, \mathbf{1})$ be a \CrefAndHyperrefIfExist{definition:monoidal_category}{monoidal category}. A \hldef{category enriched in $\mathcal{V}$} (or a \hldef{$\mathcal{V}$-enriched category} or a \hldef{$\mathcal{V}$-category}) $\mathcal{C}$ consists of the following data:
\begin{itemize}
    \item A class \hl{$\operatorname{Ob}(\mathcal{C})$} of \hldef{objects}. As with \hyperrefIfExists{definition:category}{regular categories}, we may write \hl{$X \in \operatorname{Ob}(\mathcal{C})$} or \hl{$X \in \calC$} to mean that $X$ is an object of $\calC$.  
    \item For each pair of objects $X, Y \in \operatorname{Ob}(\mathcal{C})$, an object \hl{$\underline{\operatorname{Hom}}_{\mathcal{C}}(X,Y) \in \operatorname{Ob}(\mathcal{V})$} of \hldef{morphisms}; it is an object of the monoidal category $\mathcal{V}$. It is also often denoted by notations such as \hl{$\calC(X,Y)$}, \hl{$\Hom(X,Y) = \Hom_\calC(X,Y)$}, or \hl{$\operatorname{Mor}(X,Y) = \operatorname{Mor}_{\calC}(X,Y)$}.
    \item For each triple $X,Y,Z \in \operatorname{Ob}(\mathcal{C})$, a \hldef{composition morphism} 
    $$\mu_{X,Y,Z} : \underline{\operatorname{Hom}}_{\mathcal{C}}(Y,Z) \otimes \underline{\operatorname{Hom}}_{\mathcal{C}}(X,Y) \to \underline{\operatorname{Hom}}_{\mathcal{C}}(X,Z).$$
    It is a morphism in $\mathcal{V}$.
    \item For each object $X$, a \hldef{unit morphism} \hl{$\eta_X : \mathbf{1} \to \underline{\operatorname{Hom}}_{\mathcal{C}}(X,X)$} in $\mathcal{V}$.
\end{itemize}
These data satisfy the following axioms:
\begin{itemize}
    \item (Associativity) For all $W,X,Y,Z \in \operatorname{Ob}(\mathcal{C})$, the following diagram in $\mathcal{V}$ commutes:
    $$
    \begin{tikzcd}[column sep=large,row sep=large]
    \bigl(\underline{\operatorname{Hom}}_{\mathcal{C}}(Z,W) \otimes \underline{\operatorname{Hom}}_{\mathcal{C}}(Y,Z)\bigr) \otimes \underline{\operatorname{Hom}}_{\mathcal{C}}(X,Y) \ar[r,"\alpha"] \ar[d,"\mu \otimes \mathrm{id}"]
    & \underline{\operatorname{Hom}}_{\mathcal{C}}(Z,W) \otimes \bigl(\underline{\operatorname{Hom}}_{\mathcal{C}}(Y,Z) \otimes \underline{\operatorname{Hom}}_{\mathcal{C}}(X,Y)\bigr) \ar[d,"\mathrm{id} \otimes \mu"] \\
    \underline{\operatorname{Hom}}_{\mathcal{C}}(Y,W) \otimes \underline{\operatorname{Hom}}_{\mathcal{C}}(X,Y) \ar[d,"\mu"] 
    & \underline{\operatorname{Hom}}_{\mathcal{C}}(Z,W) \otimes \underline{\operatorname{Hom}}_{\mathcal{C}}(X,Z) \ar[d,"\mu"] \\
    \underline{\operatorname{Hom}}_{\mathcal{C}}(X,W) \ar[r,equal] & \underline{\operatorname{Hom}}_{\mathcal{C}}(X,W)
    \end{tikzcd}
    $$
    where $\alpha$ is the associativity constraint in $\mathcal{V}$.
    \item (Unit) For all $X,Y \in \operatorname{Ob}(\mathcal{C})$, the following diagrams commute:
    \begin{center}
    \begin{tikzcd}[column sep=large]
    \mathbf{1} \otimes \underline{\operatorname{Hom}}_{\mathcal{C}}(X,Y) \ar[r,"\eta_Y \otimes \mathrm{id}"] \ar[dr,"\lambda"']
    & \underline{\operatorname{Hom}}_{\mathcal{C}}(Y,Y) \otimes \underline{\operatorname{Hom}}_{\mathcal{C}}(X,Y) \ar[d,"\mu"] \\
    & \underline{\operatorname{Hom}}_{\mathcal{C}}(X,Y)
    \end{tikzcd}
    \begin{tikzcd}[column sep=large]
    \underline{\operatorname{Hom}}_{\mathcal{C}}(X,Y) \otimes \mathbf{1} \ar[r,"\mathrm{id} \otimes \eta_X"] \ar[dr,"\rho"']
    & \underline{\operatorname{Hom}}_{\mathcal{C}}(X,Y) \otimes \underline{\operatorname{Hom}}_{\mathcal{C}}(X,X) \ar[d,"\mu"] \\
    & \underline{\operatorname{Hom}}_{\mathcal{C}}(X,Y)
    \end{tikzcd}
    \end{center}
    % $$
    % \quad\quad
    % $$
    where $\lambda$ and $\rho$ are the left and right unit constraints in $\mathcal{V}$.
\end{itemize}
\end{definition}

\begin{example}
    \begin{enumerate}
        \item A \hyperrefIfExists{definition:category_enriched_in_a_monoidal_category}{category enriched in} $\Sets$ is equivalent to a \hyperrefIfExists{definition:locally_small_category}{locally small category}. 
        \item A category enriched in $\Ab$ is equivalent to a \hyperrefIfExists{definition:additive_category}{pre-additive category}. 
        \TODO{TODO: define an $R$-linear category and a $k$-linear category}
        \item Let $R$ be a commutative ring. A category enriched in $R\mathrm{-Mod}$ is equivalent to a $R$-linear category. 
        \item Let $k$ be a field. A category enriched in $k\mathrm{-Vect}$ is equivalent to a $k$-linear category. 
    \end{enumerate}
\end{example}


\section{Categories enriched in monoidal categories}

\section{Sites, sheaves and topoi}

\begin{definition}[Presheaf on a category] \label{definition:presheaf_on_a_category}
    Let $C$ and $\mathcal{A}$ be \hyperrefIfExists{definition:category}{(large) categories}\CrefIfExists{definition:category}. 
    \begin{enumerate}
        \item A \hldef{presheaf $\mathcal{F}$ on $C$ with values in $\mathcal{A}$} is a functor
        \[
        \mathcal{F}: C^{\mathrm{op}} \to \mathcal{A}.
        \]
        In other words, a presheaf $\calF$ on $C$ with values in $\calA$ is simply a \CrefAndHyperrefIfExist{definition:functor_between_categories}{contravariant functor} from $C$ to $\calA$. 
        Explicitly, for every object $U$ in $C$, one has an object $\mathcal{F}(U)$ in $\mathcal{A}$ (called the \hldef{$U$-valued sections/sections evaluated at $U$ of $\calF$}\TextIfExists{definition:sections_of_a_presheaf_on_a_category_valued_in_a_category}{, cf. \Cref{definition:sections_of_a_presheaf_on_a_category_valued_in_a_category}}), and for every morphism $f: V \to U$ in $C$, one has a morphism (called the \hldef{restriction map})
        \[
        \mathcal{F}(f): \mathcal{F}(U) \to \mathcal{F}(V)
        \]
        in $\mathcal{A}$, such that for all composable morphisms $W \xrightarrow{g} V \xrightarrow{f} U$ in $C$, the following diagram in $\mathcal{A}$ commutes:
        \[
        \begin{tikzcd}
        \mathcal{F}(U) \arrow[r, "\mathcal{F}(f)"] \arrow[rr, bend left, "\mathcal{F}(f \circ g)"] & \mathcal{F}(V) \arrow[r, "\mathcal{F}(g)"] & \mathcal{F}(W)
        \end{tikzcd}
        \]
        That is,
        \[
        \mathcal{F}(g) \circ \mathcal{F}(f) = \mathcal{F}(f \circ g),
        \]
        and for every object $U$ in $C$, $\mathcal{F}(\mathrm{id}_U) = \mathrm{id}_{\mathcal{F}(U)}$.


        \item 
        Let $\mathcal{F},\mathcal{G}: C^{\mathrm{op}} \to \mathcal{A}$ be two presheaves on $C$ with values in $\mathcal{A}$. A \hldef{morphism of presheaves}
        \[
        \varphi: \mathcal{F} \to \mathcal{G}
        \]
        is a \hyperrefIfExists{definition:natural_transformation_between_functors_between_categories}{natural transformation of functors}\CrefIfExists{definition:natural_transformation_between_functors_between_categories}: for each object $U$ of $C$, one has a morphism
        \[
        \varphi_U: \mathcal{F}(U) \to \mathcal{G}(U)
        \]
        in $\mathcal{A}$, such that for every morphism $f: V \to U$ in $C$, the diagram
        \[
        \begin{tikzcd}
        \mathcal{F}(U) \arrow[r, "\mathcal{F}(f)"] \arrow[d, "\varphi_U"'] & \mathcal{F}(V) \arrow[d, "\varphi_V"] \\
        \mathcal{G}(U) \arrow[r, "\mathcal{G}(f)"'] & \mathcal{G}(V)
        \end{tikzcd}
        \]
        commutes, i.e.,
        \[
        \varphi_V \circ \mathcal{F}(f) = \mathcal{G}(f) \circ \varphi_U
        \]
        for all objects and morphisms in $C$.

        \item Given a \hyperrefIfExists{definition:grothendieck_universe}{universe}\CrefIfExists{definition:grothendieck_universe} $U$, a \hldef{$U$-presheaf on $\calC$} typically refers to a presheaf of $U$-sets on $C$.

        \item The \hldef{presheaf category/category of $\calA$-valued presheaves on $\calC$} is the (large) category whose objects are the presheaves on $C$ with values in $\calA$ and whose morphisms are the presheaf morphisms. Common notations for the presheaf category include, but are not limited to: \hl{$\calA^{\calC^{\op}}$}, \hl{$\PreShv(\calC, \calA)$}, \hl{$[\calC^{\op}, \calA]$}. If the value category $\calA$ is clear from context, then notations such as \hl{$\PreShv(\calC)$} are also common. \TextIfExists{definition:diagram_in_a_category_indexed_by_a_small_category}{Note that the presheaf category $\PreShv(\calC, \calA)$ is equivalent to the \CrefAndHyperrefIfExist{definition:diagram_in_a_category_indexed_by_a_small_category}{category of functors} $\calC^{\op} \to \calA$ and hence notations for the functor categories are applicable as notations for presheaf categories.}

    \end{enumerate}
\end{definition}

\begin{lemma} \label{lemma:category_of_presheaves_on_a_small_category_of_locally_small_value_is_locally_small}
    Let $\calC$ be a \hyperrefIfExists{definition:locally_small_category}{small category}\CrefIfExists{definition:locally_small_category} (resp. $U$-small category where $U$ is some \hyperrefIfExists{definition:grothendieck_universe}{universe}\CrefIfExists{definition:grothendieck_universe}) and let $\calA$ be a \CrefAndHyperrefIfExist{definition:locally_small_category}{locally small} category (resp. $U$-locally small category). The \hyperrefIfExists{definition:presheaf_on_a_category}{presheaf category $\PreShv(\calC, \calA)$}\CrefIfExists{definition:presheaf_on_a_category} is locally small (resp. $U$-locally small).
\end{lemma}
\begin{proof}
    A morphism $\calF \to \calG$ in $\PreShv(\calC, \calA)$ is a \hyperrefIfExists{definition:natural_transformation_between_functors_between_categories}{natural transformation}\CrefIfExists{definition:natural_transformation_between_functors_between_categories} of the functors $\calF, \calG: \calC^{\op} \to \calA$. Such a natural transformation is encoded by a family $(\eta_C)_C$ of morphisms (satisfying certain conditions) $\eta_C: \calF(C) \to \calG(C)$ in $\calA$ over objects $C$ of $\calC^{\op}$. The product $\prod_{C \in \Ob \calC^{\op}} \Hom_{\calA}(\calF(C), \calG(C))$ is a product of ($U$-small) sets indexed by a ($U$-small) set, and the collection of natural transformations is a subset of this set. Therefore, $\Hom_{\PreShv(\calC, \calA)}(\calF, \calG)$ is a ($U$-small) set.  
\end{proof}

\begin{remark}
    Even when $\calC$ is ($U$-)locally small and ($\calA$ is ($U$-)locally small), $\PreShv(\calC, \calA)$ may not be locally small.
\end{remark}
\begin{remark}
    In practice, one might treat $\calC$ as if it were a small category, even when it is technically not a small category, to treat $\PreShv(\calC, \calA)$ as if it were a locally small category. For example, in algebraic geometry, we might take $\calC$ to be the \hyperrefIfExists{definition:small_etale_site_of_a_scheme}{small \'etale site $X_{\et}$}\CrefIfExists{definition:small_etale_site_of_a_scheme} of some scheme $X$; this is not a small category. Instead, we might technically replace $X_{\et}$ with a small category, e.g. the set of \'etale maps $U \to X$ of the following form: $U$ is obtained by patching the schemes attached to quotients of rings of the form $A[T_1,T_2,\ldots]$ where $A = \Gamma(V, \calO_X)$ for some open affine $V \subset X$ and $\{T_1,T_2,\ldots\}$ is a fixed countable set of symbols. Such technicalities of replacing $\calC$ with a small category are often glossed over.
\end{remark}


\begin{definition}[Small étale site of a scheme] \label{definition:small_etale_site_of_a_scheme}
Let $X$ be a fixed scheme. The \hldef{small étale site on $X$}, commonly denoted by notations including \hl{$X_{\et}$}, \hl{$X_{\mathrm{\acute{e}tale}}$}, or \hl{$\mathrm{Et}_{/X}$}, is defined as the following \hyperrefIfExists{definition:grothendieck_topology_on_a_category_site_covering_sieve_topologically_generating_family}{site}\CrefIfExists{definition:grothendieck_topology_on_a_category_site_covering_sieve_topologically_generating_family}:
\begin{itemize}
    \item The underlying category is the full subcategory of the \hyperrefIfExists{definition:big_etale_site_of_a_scheme}{big étale site $(\Sch/X)_{\et}$}\CrefIfExists{definition:big_etale_site_of_a_scheme} whose objects are schemes $U$ equipped with an étale morphism $U \to X$.
    \item The Grothendieck topology is the one \CrefAndHyperrefIfExist{definition:grothendieck_topology_generated_by_a_pretopology}{generated by} the \CrefAndHyperrefIfExist{definition:basis_and_grothendieck_pretopology_for_a_grothendieck_topology_on_a_category}{pretopology} whose covering families are families $\{g_j : U_j \to U\}_{j \in J}$ of morphisms such that each $g_j$ is an \CrefAndHyperrefIfExist{definition:etale_morphism_of_schemes}{\'etale} and the family is jointly surjective on the underlying topological spaces.
\end{itemize}
\TextIfExists{definition:big_site_on_the_category_of_schemes_over_a_scheme_and_small_site}{Equivalently, the small \'etale site on $X$ is the \CrefAndHyperrefIfExist{definition:big_site_on_the_category_of_schemes_over_a_scheme_and_small_site}{small site for \'etale morphisms on $X$}.}
\TODO{state this as a fact}
$X_{\et}$ is an \CrefAndHyperrefIfExist{definition:essentially_small_category}{essentially small category}.
\end{definition}


% \begin{definition}[Grothendieck topology] \label{definition:grothendieck_topology_on_a_category_site_covering_sieve_topologically_generating_family}
%     Let $\mathscr{U}$ be a \hyperrefIfExists{definition:grothendieck_universe}{universe}\CrefIfExists{definition:grothendieck_universe} and let $\calC$ be a \hyperrefIfExists{definition:locally_small_category}{locally small category}\CrefIfExists{definition:locally_small_category}.

%     \begin{enumerate}
%         \item \textbf{(Grothendieck Topology via Sieves)}
%         A \hldef{Grothendieck topology} $J$ on $\calC$ is an assignment to each object $U \in \calC$ of a collection $J(U)$ of \CrefAndHyperrefIfExist{definition:sieve_on_an_object_in_a_category}{sieves} on $U$, called \hldef{covering sieves}, satisfying:
%         \begin{enumerate}
%             \item (Maximality) The maximal \CrefAndHyperrefIfExist{definition:sieve_on_an_object_in_a_category}{sieve} $\{ f : V \to U \mid V \in \calC \}$ is in $J(U)$.
%             \item (Stability) If $S \in J(U)$ and $f : V \to U$ is any morphism, then the \CrefAndHyperrefIfExist{definition:pullback_sieve_of_an_object_in_a_category_via_a_morphism_to_the_object}{pullback sieve} $f^{*}S$ is in $J(V)$.
%             \item (Transitivity/Local Character) If $S$ is a sieve on $U$ and there exists a covering sieve $R \in J(U)$ such that for every morphism $f : V \to U$ in $R$, the pullback sieve $f^{*}S$ is in $J(V)$, then $S \in J(U)$.
%         \end{enumerate}

%         % \item \textbf{(Grothendieck Pretopology / Basis)}
%         % If $\calC$ admits fiber products, one can define a topology via \hldef{covering families}. A \hldef{Grothendieck pretopology} (or basis) is a collection $K(U)$ of families $\{U_i \to U\}_{i \in I}$ for each object $U$, satisfying:
%         % \begin{itemize}
%         %     \item (Isomorphism) $\{U' \xrightarrow{\sim} U\} \in K(U)$ for any isomorphism.
%         %     \item (Stability) If $\{U_i \to U\} \in K(U)$ and $V \to U$ is a morphism, then $\{U_i \times_U V \to V\} \in K(V)$.
%         %     \item (Composition) If $\{U_i \to U\} \in K(U)$ and for each $i$, $\{V_{ij} \to U_i\} \in K(U_i)$, then the composite family $\{V_{ij} \to U\} \in K(U)$.
%         % \end{itemize}
%         % Every pretopology generates a unique Grothendieck topology $J$, where $S \in J(U)$ iff $S$ contains a covering family from the pretopology.

%         \item A \hldef{site} is a pair $(\calC, J)$ consisting of a category $\calC$ and a Grothendieck topology $J$.

%         \item A family of objects $\mathcal{G} = \{G_\alpha\}$ in a site $(\calC, J)$ is called a \hldef{topologically generating family} if for every object $X \in \calC$, there exists a covering sieve $S \in J(X)$ \CrefAndHyperrefIfExist{definition:sieve_on_an_object_of_a_category_generated_by_a_family_of_morphisms}{generated by} morphisms with domains in $\mathcal{G}$. Equivalently, every object $X$ admits a cover $\{U_i \to X\}$ where each $U_i \in \mathcal{G}$.

%         \item A \hldef{$\mathscr{U}$-site} is a site whose underlying category is $\mathscr{U}$-locally small and which admits a $\mathscr{U}$-small topologically generating family.
%     \end{enumerate}
% \end{definition}

\begin{definition}[Grothendieck topology] \label{definition:grothendieck_topology_on_a_category_site_covering_sieve_topologically_generating_family}
    Let $\mathscr{U}$ be a \hyperrefIfExists{definition:grothendieck_universe}{universe}\CrefIfExists{definition:grothendieck_universe}.
    \begin{enumerate}
        % \item Let $C$ be a \hyperrefIfExists{definition:locally_small_category}{locally small category}\CrefIfExists{definition:locally_small_category}. A \hldef{Grothendieck topology on $C$} assigns to each object $U$ of $C$ a collection of families of morphisms $\{U_i \to U\}_{i \in I}$, called \hldef{coverings of $U$}, satisfying:
        % \begin{itemize}
        %     \item (Isomorphism) If $f: V \to U$ is an isomorphism in $C$, then $\{f: V \to U\}$ is a covering of $U$.
        %     \item (Stability under base change) If $\{U_i \to U\}_{i \in I}$ is a covering of $U$ and $V \to U$ is any morphism, then the family $\{ U_i \times_U V \to V \}_{i \in I}$ is a covering of $V$.
        %     \item (Transitivity) If $\{U_i \to U\}_{i \in I}$ is a covering of $U$ and for each $i$, $\{V_{ij} \to U_i\}_{j \in J_i}$ is a covering of $U_i$, then the family $\{ V_{ij} \to U \}_{i \in I,\, j \in J_i}$ is a covering of $U$.
        % \end{itemize}

        \item (See \cite[Expos\'e II, D\'efinition 1.1]{SGA4_I}) Let $\calC$ be a \CrefAndHyperrefIfExist{definition:category}{category}. A \hldef{Grothendieck topology on $\calC$} assigns to each object $U$ of $\calC$ a collection \hl{$J(U)$} of \CrefAndHyperrefIfExist{definition:sieve_on_an_object_in_a_category}{sieves} $\{U_i \to U\}_{i \in I}$, each called a \hldef{covering sieve of $U$}, satisfying:
        \begin{enumerate}
            \item (Stability under ``base change''): If $S \in J(U)$ is a covering sieve of an object $U$, and $f: V \to U$ is any morphism in $\calC$, then the \CrefAndHyperrefIfExist{definition:pullback_sieve_of_an_object_in_a_category_via_a_morphism_to_the_object}{pullback sieve} $f^* S$ is a covering sieve of $U$.
            % \item (Local character condition) If $F$ is a sieve on $U$ such that the sieve $\bigcup_...$ \TODO{}
            \item (Local character condition) If $S$ is a sieve on $U$, and if there exists a covering sieve $R \in J(U)$ such that for all $f: V \to U$ in $R$ the \CrefAndHyperrefIfExist{definition:pullback_sieve_of_an_object_in_a_category_via_a_morphism_to_the_object}{pullback sieve} $f^* S$ is in $J(V)$, then $S \in J(U)$. 
            
            \item The \CrefAndHyperrefIfExist{definition:maximal_sieve_on_an_object_in_a_category}{maximal sieve} is a covering sieve.
        \end{enumerate}


        % Equivalently, a Grothendieck topology $J$ on a category $C$ is an assignment of a collection $J(U)$ of \CrefAndHyperrefIfExist{definition:sieve_on_an_object_in_a_category}{sieves} on each object $U \in \operatorname{Ob}(C)$ such that:
        % \begin{enumerate}
        %     \item the maximal \CrefAndHyperrefIfExist{definition:sieve_on_an_object_in_a_category}{sieve} $\{ f : V \to U \mid f \in \operatorname{Mor}(C) \}$ belongs to $J(U)$,
        %     \item if $S \in J(U)$ and $f : V \to U$, then the \CrefAndHyperrefIfExist{definition:pullback_sieve_of_an_object_in_a_category_via_a_morphism_to_the_object}{pullback sieve $f^{*}S$} on $V$ belongs to $J(V)$,
        %     \item if $S$ is a sieve on $U$, and if there exists $R \in J(U)$ such that for all $f : V \to U$ in $R$ the \CrefAndHyperrefIfExist{definition:pullback_sieve_of_an_object_in_a_category_via_a_morphism_to_the_object}{pullback sieve $f^{*}S$} is in $J(V)$, then $S \in J(U)$.
        % \end{enumerate}

        Some will refer to a Grothendieck topology as simply a \hldef{topology}, not to be confused with the related, but less general, notion of a \CrefAndHyperrefIfExist{definition:topological_space}{topology on a set}.


        \item (See \cite[Expos\'e II, 1.1.5]{SGA4_I}) A \hldef{site} is a category $\calC$ equipped with a Grothendieck topology.

        When we are working with a \CrefAndHyperref{definition:basis_and_grothendieck_pretopology_for_a_grothendieck_topology_on_a_category}{Grothendieck pretopology} $K$ on a category $\calC$, we may regard $\calC$ as a site by equipping it with the \CrefAndHyperref{definition:grothendieck_topology_generated_by_a_pretopology}{Grothendieck topology generated by} $K$. 

        \item (See \cite[Expos\'e II, D\'efinition 1.2]{SGA4_I}) Let $(\calC, J)$ be a site. A family of morphisms $(U_i \to U)_{i \in I}$ is called a \hldef{covering family of $U$ (with respect to the site/topology)} or a \hldef{cover of $U$ (with respect to the site/topology)} if the \CrefAndHyperrefIfExist{definition:sieve_on_an_object_of_a_category_generated_by_a_family_of_morphisms}{sieve generated by} the family is a covering sieve of $U$. 

        \item (See \cite[Expos\'e II, D\'efinition 3.0.1]{SGA4_I}) Let $(\calC, J)$ be a \CrefAndHyperrefIfExist{definition:grothendieck_topology_on_a_category_site_covering_sieve_topologically_generating_family}{site}, where $J$ is a Grothendieck topology on $\calC$.

        A family $G$ of objects $\calC$ is called a \hldef{topologically generating family of the site/topology} or a \hldef{generating family/collection of the site/topology} if for every object $X \in \calC$, there is a covering family $\{X_\alpha \to X\}_{\alpha \in A}$ of $X$ such that every $X_\alpha$ is a member of $G$.  

        Equivalently, the Grothendieck topology $J$ is the smallest Grothendieck topology containing all covers of the $U_i$. Also equivalently, for any $S \in J(X)$, the sieve $S$ contains a covering family $\{V_i \to X\}$ such that each morphism $V_i \to X$ factors through some member of $G$. \TODO{Verify that these claimed equivalences are indeed equivalences}
        
        % A family of objects $\{U_i\}_{i \in I}$ in $\calC$ is called a \hldef{topologically generating family} if for every object $X \in \calC$ and every covering sieve $S \in J(X)$, the sieve $S$ is \CrefAndHyperrefIfExist{definition:sieve_on_an_object_of_a_category_generated_by_a_family_of_morphisms}{generated by} pullbacks of covering families from the family $\{U_i\}$.

        % More precisely, this means that for any $S \in J(X)$, the sieve $S$ contains a covering family $\{V_j \to X\}$ such that each morphism $V_j \to X$ factors through some $U_i$, and the covering families of the $U_i$ generate the topology $J$. 
        % Equivalently, the Grothendieck topology $J$ is the smallest Grothendieck topology containing all coverings of the $U_i$.

        % When one speaks of a \hldef{generating family/collection} of a site, one usually refers to the above notion of a topologically generating family.

        \item (See \cite[Expos\'e II, D\'efinition 3.0.2]{SGA4_I}) A \hldef {$\mathscr{U}$-site} is a site whose underlying category $\calC$ is \hyperrefIfExists{definition:locally_small_category}{$\mathscr{U}$-locally small}\CrefIfExists{definition:locally_small_category} and which has a $\mathscr{U}$-small topologically generating family. A $\mathscr{U}$-site is called \hldef{$\mathscr{U}$-small} if its underlying category is $\mathscr{U}$-small. Similarly, a \hldef{small site} is a site whose underlying category is a set and a \hldef{locally small site} is a site whose underlying category is \CrefAndHyperrefIfExist{definition:locally_small_category}{locally small}.
    \end{enumerate}
\end{definition}


\begin{definition}[Sheaf on a site] \label{definition:sheaf_on_a_site}

% \TODO{There might be some need to say that $\calA$ is a category for which sheaves on the site ``can be defined''}
% \TODO{go through statements using the notion of sheaves and make sure that the value categories have small products and that the categories have small generating families.}

Let $(\calC, J)$ be a \CrefAndHyperrefIfExist{definition:grothendieck_topology_on_a_category_site_covering_sieve_topologically_generating_family}{site}. Let $\calA$ be a \CrefAndHyperrefIfExist{definition:category}{(large) category}.
\begin{enumerate}
    \item A \CrefAndHyperrefIfExist{definition:presheaf_on_a_category}{presheaf} $\calF: \calC^{\op} \to \calA$\CrefIfExists{definition:opposite_category_of_a_category} is called a \hldef{sheaf on the site $(\calC, J)$ valued in $\calA$} if, for every object $U$ of $\calC$ and every \CrefAndHyperrefIfExist{definition:grothendieck_topology_on_a_category_site_covering_sieve_topologically_generating_family}{covering sieve} $S \in J(U)$, the \CrefAndHyperrefIfExist{definition:limit_and_colimit_of_a_diagram_in_a_category}{limit}
    $$\varprojlim_{(V \to U) \in (\calD_S)^{\op}} \calF|_{\calD_S}(V),$$
    exists and the canonical natural morphism
    $$\calF(U) \to \varprojlim_{(V \to U) \in (\calD_S)^{\op}} \calF|_{\calD_S}(V)$$
    is an isomorphism. Here, $\calD_S \hookrightarrow \calC/U$\CrefIfExists{definition:category_of_objects_over_under_a_fixed_object_in_a_category} is the full \CrefAndHyperrefIfExist{definition:downward_upward_closed_subcategory_of_a_category}{downward-closed subcategory} such that $\operatorname{Ob}(\calD_S) = \{(f: V \to U): f \in S(V)\}$,

    In particular, when we are working with a \CrefAndHyperref{definition:basis_and_grothendieck_pretopology_for_a_grothendieck_topology_on_a_category}{Grothendieck pretopology} $K$ on a category $\calC$, we may speak of sheaves on the site whose Grothendieck topology is the \CrefAndHyperref{definition:grothendieck_topology_generated_by_a_pretopology}{one generated by} $K$.

    \item Given sheaves $\calF, \calG: \calC^{\op} \to \calA$ on the site $(\calC, J)$, a \hldef{morphism between the sheaves} is a \CrefAndHyperrefIfExist{definition:presheaf_on_a_category}{morphism} between $\calF$ and $\calG$ as presheaves.


    \item Let $U$ be a \hyperrefIfExists{definition:grothendieck_universe}{universe}\CrefIfExists{definition:grothendieck_universe}. A \hldef{$U$-sheaf} typically refers to a $U$-presheaf that is a sheaf for a $U$-site. In other words, a $U$-sheaf is a sheaf on a site whose underlying category is \hyperrefIfExists{definition:locally_small_category}{$U$-locally small}\CrefIfExists{definition:locally_small_category} and which has a $U$-small topologically generating family such that the sheaf is valued in $U$-sets.

    \item The \hldef{sheaf category/category of $\calA$-valued sheaves on $\calC$} is the (large) category defined as the full subcategory of $\PreShv(\calC, \calA)$ whose objects are the sheaves on $\calC$ with values in $\calA$. Common notations for the sheaf category include \hl{$\Shv(\calC, \calA)$}, \hl{$\Shv(\calC, J, \calA)$}, \hl{$\Sh(\calC, \calA)$}, \hl{$\Sh(\calC, J, \calA)$}. If the value category $\calA$ is clear from context, then notations such as \hl{$\Shv(\calC)$}, \hl{$\Shv(\calC, J)$}, \hl{$\Sh(\calC)$}, \hl{$\Sh(\calC, J)$} are also common.

\end{enumerate}

% Let $(\calC, J)$ be a \CrefAndHyperrefIfExist{definition:grothendieck_topology_on_a_category_site_covering_sieve_topologically_generating_family}{site} with a small \CrefAndHyperrefIfExist{definition:grothendieck_topology_on_a_category_site_covering_sieve_topologically_generating_family}{topological generating family} (or a $U$-small topologically generating family if a \CrefAndHyperrefIfExist{definition:grothendieck_universe}{universe} $U$ is available) and let $\mathcal{A}$ be a \CrefAndHyperrefIfExist{definition:category}{(large) category} that has all \CrefAndHyperrefIfExist{definition:locally_small_category}{small} \CrefAndHyperrefIfExist{definition:product_and_coproduct_of_objects_in_a_category}{products} (Some common examples of categories that have small products and thus play the role of $\calA$ here include $\mathcal{A} = \text{Set}$, $\text{Ab}$, $R\mathbf{-mod}$ for a fixed ring $R$, $\text{rings}$). 
% \begin{enumerate}

%     \item For any object $U$ of $\calC$ and every covering $\{U_i \to U\}_{i \in I}$ in $J$, note that there are morphisms $U_i \times_U U_j \to U_i$ for every $i,j \in I$. 
%     % Consider the subcategory of $C$ consisting of the objects $U_i$ and $U_i \times_U U_j$, together with these morphisms.
%     Given any presheaf $\calF: C^{\op} \to \calA$, there is a \CrefAndHyperrefIfExist{definition:diagram_in_a_category_indexed_by_a_small_category}{diagram} in $\calA$ consisting of objects $\calF(U_i)$ and $\calF(U_i \times_U U_j)$ and morphisms $\calF(U_i) \to \calF(U_i \times_U U_j)$. The presheaf $\calF$ is called a \hldef{sheaf on the site $(\calC, J)$ valued in $\calA$} if, for every object $U$ of $\calC$ and every covering $\{U_i \to U\}_{i \in I}$ in $J$, the sections object $\calF(U)$ is the \CrefAndHyperrefIfExist{definition:limit_and_colimit_of_a_diagram_in_a_category}{limit} of the aforementioned diagram:
    
%     % A \hyperrefIfExists{definition:presheaf_on_a_category}{presheaf}\CrefIfExists{definition:presheaf_on_a_category} $\mathcal{F}: C^{\mathrm{op}} \to \mathcal{A}$ is a \hldef{sheaf on the site $(\calC,J)$ valued in $\calA$} if, for every object $U$ of $\calC$ and every covering $\{U_i \to U\}_{i \in I}$ in $J$, the sections object $\calF(U)$ is the \CrefAndHyperrefIfExist{definition:limit_and_colimit_of_a_diagram_in_a_category}{limit} of the sections objects $\calF(U_i)$:
%     % $$\calF(U) \cong \varprojlim_{}$$
    
%     % following sequence is an \CrefAndHyperrefIfExist{definition:equalizer_and_coequalizer_of_morphisms_in_a_category}{equalizer} in $\mathcal{A}$:
%     % \[
%     % \mathcal{F}(U) \to \prod_{i} \mathcal{F}(U_i) \rightrightarrows \prod_{i, j} \mathcal{F}(U_i \times_U U_j)
%     % \]
%     % where the first map sends $s$ to $(\mathcal{F}(U_i \to U)(s))_i$ and the arrows to $(\mathcal{F}(U_i \times_U U_j \to U_i)(s_i))_{i,j}$ and $(\mathcal{F}(U_i \times_U U_j \to U_j)(s_j))_{i,j}$, respectively.

%     % \item A \hyperrefIfExists{definition:presheaf_on_a_category}{presheaf}\CrefIfExists{definition:presheaf_on_a_category} $\mathcal{F}: C^{\mathrm{op}} \to \mathcal{A}$ is a \hldef{sheaf on the site $(\calC,J)$ valued in $\calA$} if, for every object $U$ of $\calC$ and every covering $\{U_i \to U\}_{i \in I}$ in $J$, the following sequence is an \CrefAndHyperrefIfExist{definition:equalizer_and_coequalizer_of_morphisms_in_a_category}{equalizer} in $\mathcal{A}$:
%     % \[
%     % \mathcal{F}(U) \to \prod_{i} \mathcal{F}(U_i) \rightrightarrows \prod_{i, j} \mathcal{F}(U_i \times_U U_j)
%     % \]
%     % where the first map sends $s$ to $(\mathcal{F}(U_i \to U)(s))_i$ and the arrows to $(\mathcal{F}(U_i \times_U U_j \to U_i)(s_i))_{i,j}$ and $(\mathcal{F}(U_i \times_U U_j \to U_j)(s_j))_{i,j}$, respectively.

%     \item A \hldef{morphism of sheaves} $\calF: \calC^{\op} \to \calA$ is a \hyperrefIfExists{definition:presheaf_on_a_category}{morphism as presheaves}\CrefIfExists{definition:presheaf_on_a_category}. 


%     \item Let $U$ be a \hyperrefIfExists{definition:grothendieck_universe}{universe}\CrefIfExists{definition:grothendieck_universe}. A \hldef{$U$-sheaf} typically refers to a $U$-presheaf that is a sheaf for a $U$-site. In other words, a $U$-sheaf is a sheaf on a site whose underlying category is \hyperrefIfExists{definition:locally_small_category}{$U$-locally small}\CrefIfExists{definition:locally_small_category} and which has a $U$-small topologically generating family such that the sheaf is valued in $U$-sets.

%     \item The \hldef{sheaf category/category of $\calA$-valued sheaves on $\calC$} is the (large) category defined as the full subcategory of $\PreShv(\calC, \calA)$ whose objects are the sheaves on $C$ with values in $\calA$. Common notations for the sheaf category include \hl{$\Shv(\calC, \calA)$}, \hl{$\Shv(\calC, J, \calA)$}, \hl{$\Sh(\calC, \calA)$}, \hl{$\Sh(\calC, J, \calA)$}. If the value category $\calA$ is clear from context, then notations such as \hl{$\Shv(\calC)$}, \hl{$\Shv(\calC, J)$}, \hl{$\Sh(\calC)$}, \hl{$\Sh(\calC, J)$} are also common.

% \end{enumerate}
\end{definition}

\begin{definition} \label{definition:sheafification_functor_on_a_site}
    Let $\calC$ be a \CrefAndHyperrefIfExist{definition:grothendieck_topology_on_a_category_site_covering_sieve_topologically_generating_family}{site} and let $\calA$ be a \CrefAndHyperrefIfExist{definition:category}{(large) category}.

    Assuming that the \CrefAndHyperrefIfExist{definition:presheaf_on_a_category}{presheaf} category $\PreShv(\calC, \calA)$ (and hence the \CrefAndHyperrefIfExist{definition:sheaf_on_a_site}{sheaf} category $\Shv(\calC, \calA)$) is \CrefAndHyperrefIfExist{definition:locally_small_category}{locally small} (or $U$-locally small if a \CrefAndHyperrefIfExist{definition:grothendieck_universe}{Grothendieck universe} $U$ is available), a \hldef{sheafification functor} refers to a functor
    $$a: \PreShv(\calC, \calA) \to \Shv(\calC, \calA) $$
    that is \CrefAndHyperrefIfExist{definition:adjoint_functors_between_categories_unit_counit_of_adjoint_functors}{left adjoint} to the inclusion functor 
    $$i:\Shv(\calC, \calA) \hookrightarrow \PreShv(\calC, \calA)  .$$
    If such a sheafification functor exists, then it is unique up to unique natural isomorphism. Given a presheaf $P$, the sheafification $a(P)$ is also sometimes called the \hldef{sheaf associated to $P$}.
    \TextIfExists{theorem:sheafification_of_a_presheaf_of_sets_on_a_small_enough_site}{See \Cref{theorem:sheafification_of_a_presheaf_of_sets_on_a_small_enough_site} for common conditions under which sheafification exists.} 
\end{definition}

% See Also
%theorem:sheafification_of_a_presheaf_of_sets_on_a_small_enough_site

The following expresses the idea that sheafification functors exist for small enough sites.
\begin{theorem}{cf. {\cite[Expos\'e II, Th\'eor\`eme 3.4]{SGA4_I}}} \label{theorem:sheafification_of_a_presheaf_of_sets_on_a_small_enough_site}
    \begin{enumerate}
        \item Let $U$ be a universe. Let $\calC$ be a \hyperrefIfExists{definition:grothendieck_topology_on_a_category_site_covering_sieve_topologically_generating_family}{$U$-site}\CrefIfExists{definition:grothendieck_topology_on_a_category_site_covering_sieve_topologically_generating_family}. A \CrefAndHyperrefIfExist{definition:sheafification_of_a_presheaf_on_a_topological_space_valued_in_a_category_admitting_direct_colimits}{sheafification functor}
        $$a: \Shv(\calC, \USets) \to \PreShv(\calC, \USets).$$
        exists. 
        % The inclusion functor 
        % $$i: \PreShv(\calC, \USets) \hookrightarrow \Shv(\calC, \USets)$$
        % has a \hyperrefIfExists{definition:adjoint_functors_between_categories_unit_counit_of_adjoint_functors}{left adjoint functor}\CrefIfExists{definition:adjoint_functors_between_categories_unit_counit_of_adjoint_functors}

        \item Let $\calC$ be a site whose underlying category is \CrefAndHyperrefIfExist{definition:locally_small_category}{locally small} and which has a \CrefAndHyperrefIfExist{definition:grothendieck_topology_on_a_category_site_covering_sieve_topologically_generating_family}{topologically generating family} that is a set (rather than a proper class). A sheafification functor 
        $$a: \Shv(\calC, \Sets) \to \PreShv(\calC, \Sets)$$
        exists.

        \item (see e.g. {\cite[3]{nlab:sheafification}}) Let $(\calC, J)$ be a \CrefAndHyperrefIfExist{definition:grothendieck_topology_on_a_category_site_covering_sieve_topologically_generating_family}{site} on an \CrefAndHyperrefIfExist{definition:essentially_small_category}{essentially small category} $\calC$. Suppose that the category $\calA$ is \CrefAndHyperrefIfExist{definition:complete_and_cocomplete_category}{complete, cocomplete}, that small \CrefAndHyperrefIfExist{definition:projective_and_inductive_limits_in_categories}{filtered colimits} in $\calA$ are exact, and that $\calA$ satisfies the IPC-property. A \CrefAndHyperrefIfExist{definition:sheafification_functor_on_a_site}{sheafification functor} 
        $$a: \PreShv(\calC, \calA) \to \Shv(\calC, \calA) $$
        exists.
        \TODO{IPC-property, exactess in this context.}

        \TODO{state as a fact that these categories are complete, cocomplete, with small filtered colimits that are exact}
        This is true for instance of $\calA = \mathbf{Set}, \mathbf{Grp}$, $k-\mathbf{Alg}$ for a field $k$, or $\mathbf{Mod}_R$ for a \CrefAndHyperrefIfExist{definition:ring}{(not necessarily commutative unital) ring $R$}.
    \end{enumerate}
\end{theorem}
\begin{remark}
    If the presheaf is valued in nice ``algebraic category'', e.g. groups, abelian groups, rings, modules over a ring, etc., then the sheafification is again valued in that category. \TODO{Make this more precise.}
\end{remark}
% \begin{definition}[Topos] \label{definition:topos}
%     There are a multitude of notions of topos. Here are some that we consider; more notions may be added later.
%     \begin{enumerate}
%         \item A \hldef{(sheaf/Grothendieck) topos} is a \CrefAndHyperrefIfExist{definition:category}{category} \CrefAndHyperrefIfExist{definition:equivalence_of_categories}{equivalent} to the category of \CrefAndHyperrefIfExist{definition:sheaf_on_a_site}{sheaves} of sets on some \CrefAndHyperrefIfExist{definition:grothendieck_topology_on_a_category_site_covering_sieve_topologically_generating_family}{site}. That is, there exists a site $(C, J)$ such that the category is equivalent to $\operatorname{Sh}(C, J)$, the category of sheaves of sets on $(C, J)$.
%         \item Let $U$ be a universe. A \hldef{$U$-(sheaf )topos} is a category equivalent to the category of \hyperrefIfExists{definition:sheaf_on_a_site}{$U$-sheaves}\CrefIfExists{definition:sheaf_on_a_site} (valued in $U$-sets) \cite[Expos\'e IV D\'efinition 1.1]{SGA4_I}

%         \item An \hldef{elementary topos} is a cateogry which has all finite \CrefAndHyperrefIfExist{definition:limit_and_colimit_of_a_diagram_in_a_category}{limits}, is cartesian closed, and has a subobject classifier \TODO{cartesian closed, subobject classifier}
%     \end{enumerate}
% \end{definition}

\begin{definition}[Topos] \label{definition:topos}
    There are multiple notions of a topos depending on the context (geometric vs. logical).
    \begin{enumerate}
        \item A \hldef{Grothendieck topos} (or \hldef{sheaf topos}) is a \CrefAndHyperrefIfExist{definition:category}{category} \CrefAndHyperrefIfExist{definition:equivalence_of_categories}{equivalent} to the category of \CrefAndHyperrefIfExist{definition:sheaf_on_a_site}{sheaves} of sets on a \hldef{small} \CrefAndHyperrefIfExist{definition:grothendieck_topology_on_a_category_site_covering_sieve_topologically_generating_family}{site}. That is, there exists a small site $(\mathcal{C}, J)$ such that the category is equivalent to $\operatorname{Sh}(\mathcal{C}, J)$.
        
        \item Let $\mathscr{U}$ be a \hyperrefIfExists{definition:grothendieck_universe}{universe}\CrefIfExists{definition:grothendieck_universe}. A \hldef{$\mathscr{U}$-topos} is a category equivalent to the category of sheaves of sets on a $\mathscr{U}$-small site $(\mathcal{C}, J)$, where the sheaves take values in the category of $\mathscr{U}$-sets ($\mathbf{Set}_{\mathscr{U}}$). \cite[Expos\'e IV D\'efinition 1.1]{SGA4_I}

        \item An \hldef{elementary topos} is a category which has all finite \CrefAndHyperrefIfExist{definition:limit_and_colimit_of_a_diagram_in_a_category}{limits}, is \CrefAndHyperrefIfExist{definition:cartesian_closed_category}{cartesian closed}, and has a \CrefAndHyperrefIfExist{definition:subobject_classifier_in_a_category_with_a_final_object}{subobject classifier}.
    \end{enumerate}
    \textit{Remark:} Every Grothendieck topos is an elementary topos, but the converse is not true (e.g., the category of finite sets is an elementary topos but not a Grothendieck topos).
\end{definition}


% {\cite[Expos\'e IV D\'efinition 1.1]{SGA4_I}}
% Let $\scrU$ be a fixed universe. A \hldef{$\scrU$-topos}, or simply \hldef{topos} if there is no confusion, $E$ is a category that is equivalent to the category $\Shv(T)$ of sheaves of sets on a fixed site $T$ in $\scrU$.


\begin{lemma}[see e.g. {\cite[Expos\'e IV 5.1]{SGA4_I}}] \label{lemma:slice_category_of_a_topos_is_a_topos}
    Let $E$ be a \hyperrefIfExists{definition:topos}{topos}\CrefIfExists{definition:topos} and let $X$ be an object of $E$. The \hyperrefIfExists{definition:category_of_objects_over_under_a_fixed_object_in_a_category}{slice category $E_{/X}$}\CrefIfExists{definition:category_of_objects_over_under_a_fixed_object_in_a_category} is itself a topos. 
\end{lemma}


\begin{definition} \label{definition:topos_of_objects_over_a_fixed_object_of_a_topos}
    Let $C$ be a \hyperrefIfExists{definition:category}{(large) category}\CrefIfExists{definition:category}. In view of \Cref{lemma:slice_category_of_a_topos_is_a_topos}, the \hyperrefIfExists{definition:category_of_objects_over_under_a_fixed_object_in_a_category}{slice category $E_{/X} = E/X$}\CrefIfExists{definition:category_of_objects_over_under_a_fixed_object_in_a_category} is often called the \hldef{slice topos} or \hldef{over topos}, etc. 
\end{definition}


\subsection{Sheaf cohomology}

\begin{definition} \label{definition:left_right_resolution_of_a_class_of_objects_in_an_abelian_category}
Let $\mathcal{A}$ be an \CrefAndHyperrefIfExist{definition:abelian_category}{abelian category} and let $\mathcal{X}$ be a class of objects in $\mathcal{A}$. Let $M$ be an object of $\calA$.

\begin{enumerate}
    \item A \hldef{right resolution of $M$} is a \CrefAndHyperrefIfExist{definition:chain_complex_of_objects_in_an_additive_category}{cochain complex} $I^\bullet$ with $I^i = 0$ for $i < 0$ and a map $M \to I^0$ such that the augmented complex
    $$0 \to M \to I^0 \to I^1 \to I^2 \to \cdots$$
    is \CrefAndHyperrefIfExist{definition:acyclic_complex_of_objects_in_an_abelian_category}{exact}.

    % \item An \hldef{injective resolution of $M$} is a right resolution $I^\bullet$ for which the objects $I^i$ are all \CrefAndHyperrefIfExist{definition:injective_and_projective_objects_in_a_category}{injective}.

    \item A \hldef{left resolution of $M$} is a \CrefAndHyperrefIfExist{definition:chain_complex_of_objects_in_an_additive_category}{chain complex} $P_\bullet$ with $P_i = 0$ for $i < 0$ and a map $P_0 \to M$ such that the augmented complex
    $$\cdots P_2 \to P_1 \to P_0 \to M \to 0$$
    is \CrefAndHyperrefIfExist{definition:acyclic_complex_of_objects_in_an_abelian_category}{exact}.
    
    \item An \hldef{$\mathcal{X}$-left resolution} of an object $M \in \mathcal{A}$ a \CrefAndHyperrefIfExist{definition:left_right_resolution_of_a_class_of_objects_in_an_abelian_category}{left resolution} by objects of $X$, i.e. an exact complex
    $$ \cdots \to X_2 \to X_1 \to X_0 \to M \to 0 $$
    with each $X_i \in \mathcal{X}$.

    \item An \hldef{$\mathcal{X}$-right resolution} of an object $M \in \mathcal{A}$ a \CrefAndHyperrefIfExist{definition:left_right_resolution_of_a_class_of_objects_in_an_abelian_category}{right resolution} by objects of $X$, i.e. an exact complex
    $$ 0 \to M \to X^0 \to X^1 \to X^2 \to \cdots $$
    with each $X_i \in \mathcal{X}$.

    \item A \hldef{projective resolution of $M$} is a left resolution $P^\bullet$ for which the objects $P^i$ are all \CrefAndHyperrefIfExist{definition:injective_and_projective_objects_in_a_category}{projective}.

    \item An \hldef{injective resolution of $M$} is a right resolution $I^\bullet$ for which the objects $I^i$ are all \CrefAndHyperrefIfExist{definition:injective_and_projective_objects_in_a_category}{injective}.
\end{enumerate}

\end{definition}
\begin{definition} \label{definition:has_enough_injectives_or_projectives_for_an_abelian_category}
Let $\mathcal{A}$ be an \CrefAndHyperrefIfExist{definition:abelian_category}{abelian category}.
\begin{enumerate}
    \item $\mathcal{A}$ is said to \hldef{have enough injectives} if for every object $A$ in $\calA$, there is an \CrefAndHyperrefIfExist{definition:monomorphism_and_epimorphism_in_categories}{monomorphism} $A \to I$ with $I$ an \CrefAndHyperrefIfExist{definition:injective_and_projective_objects_in_a_category}{injective object} of $\calA$. \TextIfExistsElse{definition:has_enough_objects_of_a_class_on_the_left_right_for_an_abelian_category}{Equivalently, $\calA$ has enough injectives if it has enough objects of the class of injectives on the right (\Cref{definition:has_enough_objects_of_a_class_on_the_left_right_for_an_abelian_category})}

    \item $\mathcal{A}$ is said to \hldef{have enough projectives} if for every object $A$ in $\calA$, there is a \CrefAndHyperrefIfExist{definition:monomorphism_and_epimorphism_in_categories}{epimorphism} $P \to A$ with $P$ a \CrefAndHyperrefIfExist{definition:injective_and_projective_objects_in_a_category}{projective object} of $\calA$. \TextIfExistsElse{definition:has_enough_objects_of_a_class_on_the_left_right_for_an_abelian_category}{Equivalently, $\calA$ has enough projectives if it has enough objects of the class of projectives on the left (\Cref{definition:has_enough_objects_of_a_class_on_the_left_right_for_an_abelian_category})}

\end{enumerate}
\end{definition}

% See Also
% definition:left_right_derived_functors_of_a_right_left_exact_functor_between_abelian_categories_where_source_has_enough_projectives_injectives 

\begin{definition} \label{definition:left_right_derived_functors_of_a_right_left_exact_functor_between_abelian_categories_where_source_has_enough_projectives_injectives}
    \TODO{I think that the definition of derived categories might be doable for more general kinds of resolutions? Perhaps it is that if I have a right exact functor $F$, then $L^i F$ can be computed with resolutions of $F$-acyclic objects? \CrefIfExists{definition:F_acyclic_object_for_a_left_or_right_functor_between_abelian_categories}}
    \TODO{Apparently, left/right derived functors may be defined for functors that are additive and preserve finite coproducts, and not necessarily right/left exact; the exactness condition ensures that the zeroth derived functor agrees with $F$.}
Let $\mathcal{A}$ and $\mathcal{B}$ be \CrefAndHyperrefIfExist{definition:abelian_category}{abelian categories}, and let 
$$F: \mathcal{A} \to \mathcal{B}$$ 
be an \CrefAndHyperrefIfExist{definition:additive_functor_between_additive_categories}{additive functor}.

\begin{enumerate}
    \item Suppose that the functor $F$ is \CrefAndHyperrefIfExist{definition:exact_functor_between_abelian_categories}{right exact} and suppose that $A \in \calA$ is an object for which a \CrefAndHyperrefIfExist{definition:left_right_resolution_of_a_class_of_objects_in_an_abelian_category}{projective resolution}
    $$\cdots \to P_2 \to P_1 \to P_0 \to A \to 0$$
    exists in $\calA$. We define the \hldef{left derived object} \hl{$L_n F A \in \calB$} by applying $F$ to obtain a complex
    $$\cdots \to F(P_2) \to F(P_1) \to F(P_0) \to 0$$
    and letting $L_n F(A)$ be the \CrefAndHyperrefIfExist{definition:homology_and_cohomology_objects_for_a_chain_complex_in_an_additive_category}{$n$-th homology object} of this complex in $\mathcal{B}$:
    $$L_n F(A) := H_n(F(P_\bullet)).$$
    The object $L_n F(A)$ is independent of the choice of projective resolution up to natural isomorphism (\Cref{proposition:left_right_derived_objects_for_a_right_left_exact_functor_between_abelian_categories_are_well_defined}). 

    By convention, set $L_n F = 0$ for $n < 0$.

    The \hldef{higher left derived objects} refer to the object $L_n F(A)$ for $n > 0$. 

    \item  Suppose that the functor $F$ is \CrefAndHyperrefIfExist{definition:exact_functor_between_abelian_categories}{right exact} and that $\calA$ \CrefAndHyperrefIfExist{definition:has_enough_injectives_or_projectives_for_an_abelian_category}{has enough projectives}. The \hldef{left derived functors} refer to the family of functors
    $$\hlin{L_n F : \mathcal{A} \to \mathcal{B}, \quad A \mapsto L_n F(A).}$$
    The \hldef{higher left derived functors} refer to the functors $L_n F$ for $n > 0$. 

    \item Suppose that the functor $F$ is \CrefAndHyperrefIfExist{definition:exact_functor_between_abelian_categories}{right exact} and suppose that $A \in \calA$ is an object for which a \CrefAndHyperrefIfExist{definition:left_right_resolution_of_a_class_of_objects_in_an_abelian_category}{injective resolution}
    $$0 \to A \to I^0 \to I^1 \to I^2 \to \cdots$$
    exists in $\calA$. We define the \hldef{right derived object} \hl{$R_n F A \in \calB$}, also often denoted by \hl{$R^n FA$}, by applying $F$ to obtain a complex
    $$0 \to F(I^0) \to F(I^1) \to F(I^2) \to \cdots.$$
    and letting $R_n F(A)$ be the \CrefAndHyperrefIfExist{definition:homology_and_cohomology_objects_for_a_chain_complex_in_an_additive_category}{$n$-th cohomology object} of this complex in $\mathcal{B}$:
    $$R_n F(A) := H^n(F(I_\bullet)).$$
    The object $R_n F(A)$ is independent of the choice of injective resolution up to natural isomorphism (\Cref{proposition:left_right_derived_objects_for_a_right_left_exact_functor_between_abelian_categories_are_well_defined}). 

    By convention, set $R_n F = 0$ for $n < 0$.

    The \hldef{higher right derived objects} refer to the object $R_n F(A)$ for $n > 0$. 

    \item  Suppose that the functor $F$ is \CrefAndHyperrefIfExist{definition:exact_functor_between_abelian_categories}{right exact} and that $\calA$ \CrefAndHyperrefIfExist{definition:has_enough_injectives_or_projectives_for_an_abelian_category}{has enough injectives}. The \hldef{right derived functors} refer to the family of functors
    $$\hlin{R_n F : \mathcal{A} \to \mathcal{B}, \quad A \mapsto R_n F(A).}$$
    The right derived functors are also often denoted by \hl{$R^n F$}.
    The \hldef{higher right derived functors} refer to the functors $R_n F$ for $n > 0$. 

    
    % If the functor $F$ is right exact and $\calA$ \CrefAndHyperrefIfExist{definition:has_enough_injectives_or_projectives_for_an_abelian_category}{has enough projectives}, then its \hldef{left derived functors} are a family of functors
    % $$\hlin{L_n F : \mathcal{A} \to \mathcal{B}, \quad n \geq 0,}$$
    % which are defined for each object $A$ in $\mathcal{A}$ by choosing (\Cref{lemma:an_object_of_abelian_category_with_enough_objects_of_a_class_on_the_right_left_has_right_left_resolution_by_the_class}) a \CrefAndHyperrefIfExist{definition:left_right_resolution_of_a_class_of_objects_in_an_abelian_category}{projective resolution}
    % $$\cdots \to P_2 \to P_1 \to P_0 \to A \to 0$$
    % in $\mathcal{A}$ and applying $F$ to obtain a complex
    % $$\cdots \to F(P_2) \to F(P_1) \to F(P_0) \to 0.$$
    % Then $L_n F(A)$ is defined to be the \CrefAndHyperrefIfExist{definition:homology_and_cohomology_objects_for_a_chain_complex_in_an_additive_category}{$n$-th homology object} of this complex in $\mathcal{B}$:
    % $$L_n F(A) := H_n(F(P_\bullet)).$$
    % The functors $L_n F$ are independent of the choice of projective resolution up to natural isomorphism. 

    % By convention, set $L_n F = 0$ for $n < 0$.

    % \item If the functor $F$ is \CrefAndHyperrefIfExist{definition:exact_functor_between_abelian_categories}{left exact} and $\calA$ \CrefAndHyperrefIfExist{definition:has_enough_injectives_or_projectives_for_an_abelian_category}{has enough injectives}, then its \hldef{right derived functors} are a family of functors
    % $$\hlin{R^n F : \mathcal{A} \to \mathcal{B}, \quad n \geq 0,}$$
    % which are defined for each object $A$ in $\mathcal{A}$ by choosing (\Cref{lemma:an_object_of_abelian_category_with_enough_objects_of_a_class_on_the_right_left_has_right_left_resolution_by_the_class}) an \CrefAndHyperrefIfExist{definition:left_right_resolution_of_a_class_of_objects_in_an_abelian_category}{injective resolution}
    % $$0 \to A \to I^0 \to I^1 \to I^2 \to \cdots$$
    % in $\mathcal{A}$ and applying $F$ to obtain a complex
    % $$0 \to F(I^0) \to F(I^1) \to F(I^2) \to \cdots.$$
    % Then $R^n F(A)$ is defined to be the \CrefAndHyperrefIfExist{definition:homology_and_cohomology_objects_for_a_chain_complex_in_an_additive_category}{$n$-th cohomology object} of this complex in $\mathcal{B}$:
    % $$R^n F(A) := H^n(F(I^\bullet)).$$
    % The functors $R^n F$ are independent of the choice of injective resolution up to natural isomorphism.

    % By convention, set $R^n F = 0$ for $n < 0$.
\end{enumerate}
\end{definition}

\begin{definition}
Let $(\mathcal{C}, J)$ be a site. Let $\mathcal{F}$ be a sheaf of abelian groups (more generally of modules over a fixed \CrefAndHyperrefIfExist{definition:ring}{ring}) on $(\mathcal{C}, J)$.

\TODO{global sections functor}
For each integer $n \geq 0$, the \hldef{$n$-th sheaf cohomology group} of $\mathcal{F}$ is
$$\hlin{H^n(\mathcal{C}, J; \mathcal{F}) := R^n \Gamma(\mathcal{F}),}$$
where $R^n \Gamma$ is the $n$-th \CrefAndHyperrefIfExist{definition:left_right_derived_functors_of_a_right_left_exact_functor_between_abelian_categories_where_source_has_enough_projectives_injectives}{right derived functor} of the global sections functor $\Gamma$, which is a left exact functor.


The \hldef{sheaf cohomology groups of $\mathcal{F}$} on the site $(\mathcal{C}, J)$ are defined as the \CrefAndHyperrefIfExist{definition:left_right_derived_functors_of_a_right_left_exact_functor_between_abelian_categories_where_source_has_enough_projectives_injectives}{right derived functors} of the global sections functor
$$\Gamma : \mathrm{Sh}(\mathcal{C}, J) \to \mathbf{Ab}, \quad \Gamma(\mathcal{F}) = \mathcal{F}(\ast),$$
where $\mathrm{Sh}(\mathcal{C}, J)$ denotes the category of sheaves of abelian groups on $(\mathcal{C}, J)$, and $\ast$ denotes the final object in $\mathcal{C}$ if it exists or a chosen terminal object.

More precisely, 

If $\mathcal{C}$ has no final object, $H^n(\mathcal{C}, J; \mathcal{F})$ is defined by choosing an injective resolution of $\mathcal{F}$ and taking cohomology of the resulting complex obtained by applying $\Gamma$.

% These groups measure the extent to which global sections fail to be exact, and generalize classical sheaf cohomology defined on topological spaces to arbitrary sites.
\end{definition}


\appendix

\section{Grothendieck universes}

Recall that the collection of all sets within Zermelo-Fraenkel set theory is a proper class and not a set. We often use Grothendieck universes in categorical discussions as a way to restrict the sets considered so as to guarantee that only sets result in the categorical constructions that matter (e.g. to guarantee that the hom's in a newly constructed category form a set).

\begin{definition}[Grothendieck Universe] \label{definition:grothendieck_universe}
    Let $U$ be a set. We say $U$ is a \hldef{Grothendieck universe} (or just a \hldef{universe}) if the following conditions hold:
    \begin{enumerate}
        \item If $x \in U$ and $y \in x$, then $y \in U$ (transitivity).
        \item If $x,y \in U$, then $\{x,y\} \in U$ (closed under pair formation).
        \item If $x \in U$, then the power set $\mathcal{P}(x) \in U$.
        \item If $I \in U$ and $(x_\alpha)_{\alpha \in I}$ is a family with each $x_\alpha \in U$, then $\bigcup_{\alpha \in I} x_\alpha \in U$.
    \end{enumerate}
    A set $X$ is called \hldef{$U$-small} or a \hldef{$U$-set} if $X \in U$.
\end{definition}

% \begin{definition}[U-Smallness] \label{definition:U_small}
%     Let $U$ be a \hyperrefIfExists{definition:grothendieck_universe}{Grothendieck universe}\CrefIfExists{definition:grothendieck_universe}. A set $X$ is called \hldef{$U$-small} if $X \in U$. 
% \end{definition}
\begin{remark}
    Given a Grothendieck universe $U$, the collection of all $U$-small sets is $U$ itself, which contrasts against the fact that the collection of all sets (for the ZF (and ZFC) axioms) is a proper class.
\end{remark}

\begin{remark}
    Within ZF/ZFC alone, there is no guarantee that every given set $S$ satisfies $S \in U$ for some universe $S$.
\end{remark}

\begin{remark}
    While universes are a powerful tool for studying category theory, they are not universally used or necessary for doing so. Usage depends on the context and foundational preferences. For instance, Grothendieck universes can be used to simplify foundational technicalities when one needs to work with such things as ``categories of all categories'' or ``functor categories'' of large size. 
\end{remark}





