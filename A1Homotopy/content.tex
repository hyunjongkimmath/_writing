%% Delete this \nocite command invocation to make the references section only list out the bibitems that are actually cited.

\section{\texorpdfstring{$\bbA^1$}{A1}-homotopy categories}

\section{Simplicial sheaves in the Nisnevich topology as a category of spaces}


\hl{$S$}: A base scheme. This is assumed to be \CrefAndHyperrefIfExist{definition:locally_noetherian_and_noetherian_scheme}{Noetherian} of finite \CrefAndHyperrefIfExist{definition:dimension_of_a_scheme}{dimension} when the $\bbA^1$-homotopy category $\calH(S)$ is discussed.

\hl{$\Sch/S$}: The category of schemes over $S$.

\hl{$\Sm/S$}: The category of smooth schemes over $S$.  

\hl{$\Sets$}: The category of sets.

\begin{notation} \label{notation:sheaves_of_sets_on_a_site}
Given a site $T$, let \hl{$\Shv(T) = \Shv(T, \Sets)$} denote its category of \CrefAndHyperrefIfExist{definition:sheaf_on_a_site}{sheaves of sets}. The category $\Shv(T)$ has all small limits and colimits and internal function objects. Moreover, $\Shv(T)$ has a final object \CrefIfExists{lemma:final_object_is_projective_limit_of_empty_diagram}, denoted by \hl{$\pt$}, \hl{$*$}, or \hl{$\bullet$}. 
\end{notation}

\begin{definition} \label{definition:pointed_object_of_a_category_with_a_final_object}
    Let $\calC$ be a \CrefAndHyperrefIfExist{definition:category}{(large) category} with a \CrefAndHyperrefIfExist{definition:initial_final_zero_objects_of_a_category}{final object} $*$. By a \hldef{pointed object of $\calC$}, we mean an object $X$ of $\calC$ equipped with a morphism $* \to X$. The \hldef{category of pointed objects of $\calC$} refers to the \CrefAndHyperrefIfExist{definition:category_of_objects_over_under_a_fixed_object_in_a_category}{under category/coslice category} of $*$, i.e. the category whose 
    \begin{itemize}
        \item objects are morphisms $* \to X$ in $\calC$ and 
        \item morphisms between objects $* \to X$ and $* \to Y$ are morphisms $X \to Y$ in $\calC$ such that the following diagram commutes:
        $$
        \begin{aligned}
        \xymatrix{
        \ast \ar[dr] \ar[r] & X \ar[d] \\
        & Y
        }
        \end{aligned}
        $$
    \end{itemize}
    The category of pointed objects of $C$ may be denoted by notations such as \hl{$C_\bullet$} or \hl{$C_\ast$}s.
\end{definition}

\begin{definition} \label{definition:pointed_sheaves_of_sets_on_a_site_simplicial_pointed_sheaves_of_sets_on_a_site}
    Let $T$ be a \CrefAndHyperrefIfExist{definition:grothendieck_topology_on_a_category_site_covering_sieve_topologically_generating_family}{site}. 
    The category of \hldef{pointed sheaves} (resp. of \hldef{simplicial pointed sheaves}) of sets on $T$ is the category \hl{$\Shv(T)_\bullet = \Shv(T, \Sets)_\bullet$} (resp. \hl{$\Deltaop\Shv(T)_\bullet = \Deltaop\Shv(T, \Sets)_\bullet$}) of \CrefAndHyperrefIfExist{definition:pointed_object_of_a_category_with_a_final_object}{pointed objects in} $\Shv(T) = \Shv(T, \Sets)$ (resp. $\Deltaop\Shv(T) = \Deltaop\Shv(T, \Sets)$); note that the category $\Shv(T)$ (resp. $\Deltaop\Shv(T)$) has a final object \CrefIfExists{lemma:final_object_is_projective_limit_of_empty_diagram} \TODO{simplicial sheaves of sets has a final object}.
    
    % is the category whose objects are \CrefAndHyperrefIfExist{definition:sheaf_on_a_site}{sheaves} (resp. \CrefAndHyperrefIfExist{definition:simplicial_cosimplicial_object_in_a_category}{simplicial sheaves}) $X$ of sets together with a morphism $x: \pt \to X$. 

    It is also equivalent to refer to pointed sheaves of sets (resp. pointed simplicial sheaves of sets) as \hldef{sheaves of pointed sets} (resp. \hldef{simplicial sheaves of pointed sets}).
\end{definition}

\begin{definition} \label{definition:pointification_of_an_object_in_a_category_with_a_final_object}
    Let $\calC$ be a \CrefAndHyperrefIfExist{definition:category}{(large) category} with a \CrefAndHyperrefIfExist{definition:initial_final_zero_objects_of_a_category}{final object} $*$. By the \hldef{pointification of an object $X$ of $\calC$}, we mean the \CrefAndHyperrefIfExist{definition:pointed_object_of_a_category_with_a_final_object}{pointed object} \CrefAndHyperrefIfExist{definition:product_and_coproduct_of_objects_in_a_category}{coproduct} \hl{$X_+ \coloneq X \coprod *$}, if such an object exists, equipped with the natural morphism $* \to X_+$. If $X_+$ exists for all objects $X$ of $\calC$, then note that $X \mapsto X_+$ specifies a functor $\calC \to \calC_\bullet$ to the \CrefAndHyperrefIfExist{definition:pointed_object_of_a_category_with_a_final_object}{category of pointed objects of $\calC$}.

    In particular, we may talk about the pointification of an object in the following contexts:
    \begin{itemize}
        \item For a \CrefAndHyperrefIfExist{definition:grothendieck_topology_on_a_category_site_covering_sieve_topologically_generating_family}{site} $T$, we may talk about the pointification of a \CrefAndHyperrefIfExist{definition:sheaf_on_a_site}{sheaf} $X \in \Shv(T)$ of sets or of a \CrefAndHyperrefIfExist{definition:simplicial_cosimplicial_object_in_a_category}{simplicial} \CrefAndHyperrefIfExist{definition:sheaf_on_a_site}{sheaf} $X \in \Deltaop\Shv(T)$ of sets; note that the categories $\Shv(T)$ and $\Deltaop\Shv(T)$ both have final objects and (small) coproducts \TODO{}.
        
        \item We may talk about the pointification of \CrefAndHyperrefIfExist{definition:topological_space}{topological spaces}; note that the category of topological spaces has a final object and (small) coproducts.
    \end{itemize}

    Moreover, in the context of discussing a pointification functor, the \hldef{forgetful functor} is the functor $\calC_\bullet \to \calC$ given by sending a pointed space $(X, * \to X)$ to $X$.
\end{definition}

\begin{lemma} \label{lemma:pointification_and_forgetful_functor_adjunction}
    Let $\calC$ be a \CrefAndHyperrefIfExist{definition:locally_small_category}{locally small} \CrefAndHyperrefIfExist{definition:homotopy_category_of_a_model_category}{category} with a \CrefAndHyperrefIfExist{definition:initial_final_zero_objects_of_a_category}{final object} $*$ and such that a \CrefAndHyperrefIfExist{definition:pointification_of_an_object_in_a_category_with_a_final_object}{pointification $X_+$} of $X$ exists for all objects $X$ of $\calC$, so that pointifcation is a functor $\calC \to \calC_\bullet$ to the \CrefAndHyperrefIfExist{definition:pointed_object_of_a_category_with_a_final_object}{category of pointed objects} of $\calC$. 
    
    The pointification functor is \CrefAndHyperrefIfExist{definition:adjoint_functors_between_categories_unit_counit_of_adjoint_functors}{left adjoint} to the forgetful functor. In other words, for every object $X$ of $\calC$ and $(Y,y: * \to Y)$ of $\calC_\bullet$, we have natural isomorphisms
    $$\Hom_{\calC_\bullet}(X_+, (Y, y)) \cong \Hom_{\calC}(X, Y).$$
\end{lemma}
\begin{proof}
    A morphism $(X_+) = (X \coprod *, *) \to (Y,y)$ of pointed spaces consists of a morphism $\left( X \coprod * \right) \to Y$ such that the composition $* \to \left(X \coprod * \right) \to Y$ coincides with $y$. Included in this data is a morphism $X \to \left( X \coprod * \right) \to Y$, so there is a set map 
    $$\Hom_{\calC_\bullet}(X_+, (Y, y)) \to \Hom_{\calC}(X, Y).$$
    Conversely, given a morphism $X \to Y$, we can extend it uniquely to a morphism $X_+ \to (Y,y)$, yielding a set map
    $$\Hom_{\calC_\bullet}(X_+, (Y, y)) \leftarrow \Hom_{\calC}(X, Y).$$
    These two set maps are in fact inverses.
\end{proof}


\subsection{Elementary distinguished squares and the Nisnevich topology}


\begin{definition} \label{definition:elementary_distinguished_square_in_the_category_of_somoth_schemes_over_a_scheme}
    [See {\cite[Definition 2.1]{voevodsky_A1}}, {\cite[Section 3 Definition 1.3]{morel_voevodsky_1999}}]
    Let $S$ be a \CrefAndHyperrefIfExist{definition:scheme}{scheme}. An \hl{elementary distinguished square in the category $\Sm/S$} of \CrefAndHyperrefIfExist{definition:smooth_morphism_of_schemes}{smooth schemes} over $S$ is a square of the form
    \begin{equation}  \label{center:elementary_distinguished_square_of_nisnevich_topology}
    \begin{tikzcd} 
    p^{-1}(U) \ar[r] \ar[d, "p"] & V \ar[d, "p"] \\ U \ar[r,"j"] & X
    \end{tikzcd}
    \end{equation}
    \TODO{open embedding, reduced subscheme, support}
    such that $p$ is an \CrefAndHyperrefIfExist{definition:etale_morphism_of_schemes}{\'etale morphism}, $j$ is an open embedding, and $p^{-1}(X-U) \to X-U$ is an isomorphism (where $X-U$ is the maximal reduced subscheme with support in the closed subset $X-U$).
\end{definition}


\begin{definition} \label{definition:finite_type_morphism_of_schemes}
Let $f : X \to Y$ be a \CrefAndHyperrefIfExist{definition:morphism_of_schemes}{morphism of schemes}. We say that $f$ is a \hldef{finite type morphism} if for every \CrefAndHyperrefIfExist{definition:affine_open_subscheme_of_a_scheme}{affine open} $V = \operatorname{Spec} B \subseteq Y$ with $U = f^{-1}(V)$ affine, say $U = \operatorname{Spec} A$, the ring $A$ is a \CrefAndHyperrefIfExist{definition:finitely_generated_algebra_over_a_not_necessarily_commutative_ring}{finitely generated $B$-algebra}.

When $X$ is equipped with a finite type morphism $f: X \to Y$, we say that $X$ is a \hldef{finite type scheme over $Y$} or a \hldef{finite type $Y$-scheme} or a \hldef{$Y$-scheme of finite type} \CrefIfExists{definition:scheme_over_a_scheme}, etc.
\end{definition}


\begin{proposition}[{\cite[Section 3 Proposition 1.1]{morel_voevodsky_1999}}] \label{proposition:nisnevich_pretopology_on_the_category_of_smooth_schemes_over_a_scheme}
    Let $S$ be a \CrefAndHyperrefIfExist{definition:locally_noetherian_and_noetherian_scheme}{Noetherian} scheme of finite \CrefAndHyperrefIfExist{definition:dimension_of_a_scheme}{dimension}. Let $X$ be a scheme of \CrefAndHyperrefIfExist{definition:finite_type_morphism_of_schemes}{finite type} over $S$ and let $\{U_i \to X\}$ be a finite family of \CrefAndHyperrefIfExist{definition:etale_morphism_of_schemes}{\'etale morphisms} in \CrefAndHyperrefIfExist{definition:scheme_over_a_scheme}{$\Sch/S$}. The following conditions are equivalent:
    \TODO{residue field}
    \begin{enumerate}
        \item For any point $x$ of $X$ there is an $i$ and a point $u$ of $U_i$ over $x$ such that the corresponding morphism of residue fields is an isomorphism which maps to $x$ with the same residue field.
        \item For any point $x \in X$, the morphism
        $$\coprod_i (U_i \times_X \Spec \scrO_{X,x}^h) \to \Spec \scrO_{X,x}^h$$
        of $S$-schemes admits a section.
    \end{enumerate}

    Moroever, the collection of families of \'etale morphisms $\{U_i \to X\}$ in $\Sm/S$ satisfying the equivalent conditions above forms a pretopology on \CrefAndHyperrefIfExist{definition:smooth_morphism_of_schemes}{$\Sm/S$}.
\end{proposition}

\begin{definition} \label{definition:nisnevich_topology_on_a_noetherian_scheme_of_finite_dimension}
    [{See \cite[Section 3 Definition 1.2]{morel_voevodsky_1999}}]
    Let $S$ be a \CrefAndHyperrefIfExist{definition:locally_noetherian_and_noetherian_scheme}{Noetherian} scheme of finite \CrefAndHyperrefIfExist{definition:dimension_of_a_scheme}{dimension}. 
    The \CrefAndHyperrefIfExist{definition:grothendieck_topology_on_a_category_site_covering_sieve_topologically_generating_family}{Grothendieck topology} generated by the \CrefAndHyperrefIfExist{definition:basis_and_grothendieck_pretopology_for_a_grothendieck_topology_on_a_category}{pretopology} of \Cref{proposition:nisnevich_pretopology_on_the_category_of_smooth_schemes_over_a_scheme} is called the \hldef{Nisnevich topology on $\Sm/S$}. The site whose underlying category is $\Sm/S$ and whose Grothendieck topology is the Nisnevich topology is called the \hldef{(big) Nisnevich site of $S$} and is denoted by notations such as \hl{$(\mathrm{Sm}/S)_{\mathrm{Nis}}$}, \hl{$(\mathbf{Sm}/S)_{\mathrm{Nis}}$}, etc. A covering family in the Nisnevich site of $S$ is called a \hldef{Nisnevich covering}.

    \TODO{establish that a nisnevich covering is a family of etale moprhisms such that there is an isomorphism of residue fields at every point}
\end{definition}


\begin{proposition}[See {\cite[Definition 2.2]{voevodsky_A1}}, {\cite[Section 3 Proposition 1.4]{morel_voevodsky_1999}}] \label{proposition:elementary_distinguished_square_condition_for_a_presheaf_to_be_sheaf_in_the_nisnevich_topology_over_a_noetherian_scheme_of_finite_dimension}
    Let $S$ be a \CrefAndHyperrefIfExist{definition:locally_noetherian_and_noetherian_scheme}{Noetherian} scheme of finite \CrefAndHyperrefIfExist{definition:dimension_of_a_scheme}{dimension}. 
A \CrefAndHyperrefIfExist{definition:presheaf_on_a_category}{presheaf} $F: \Sm/S \to \Sets$ is a sheaf in the \CrefAndHyperrefIfExist{definition:nisnevich_topology_on_a_noetherian_scheme_of_finite_dimension}{Nisnevich topology} if and only if for any \CrefAndHyperrefIfExist{center:elementary_distinguished_square_of_nisnevich_topology}{elementary distinguished square} 
\begin{center}
    \begin{tikzcd} 
    p^{-1}(U) \ar[r] \ar[d, "p"] & V \ar[d, "p"] \\ U \ar[r,"j"] & X
    \end{tikzcd}
\end{center}
the induced square
    \begin{center}
    \begin{tikzcd}
        F(X) \ar[r] \ar[d] & F(V) \ar[d] \\ F(U) \ar[r] & F(p^{-1}(U))
    \end{tikzcd}
    \end{center}
    is \CrefAndHyperrefIfExist{definition:cartesian_product_of_two_objects_in_a_category_over_an_object}{Cartesian}. 
\end{proposition}

\subsection{Simplicial objects in a category}

\begin{definition} \label{definition:simplex_category}
The \hldef{simplex category}, or \hldef{nonempty finite ordinal category}, denoted by \hl{$\Delta$}, is the \CrefAndHyperrefIfExist{definition:locally_small_category}{locally small} \CrefAndHyperrefIfExist{definition:category}{category} whose
\begin{itemize}
  \item objects are the finite nonempty totally ordered sets \hl{$[n] := \{0, 1, 2, \dots, n\}$} for each integer $n \ge 0$;
  \item morphisms are all order-preserving (non-decreasing) \CrefAndHyperrefIfExist{definition:function_of_sets}{functions} $\theta : [m] \to [n]$.
\end{itemize}
Composition in $\Delta$ is given by composition of functions.

\end{definition}

\begin{definition} \label{definition:simplicial_cosimplicial_object_in_a_category}
    Let $\mathcal{C}$ be a \CrefAndHyperrefIfExist{definition:category}{category}.
    \begin{enumerate}
        \item   
        The \hldef{category of simplicial objects in $\mathcal{C}$}, commonly denoted by notations such as \hl{$\mathbf{s}\mathcal{C}$}, \hl{$\operatorname{Simp} \mathcal{C}$}, \hl{$\calC_\Delta$}, or \hl{$(\Delta^{\op})^{\calC}$} (cf. \Cref{definition:diagram_in_a_category_indexed_by_a_small_category}), or \hl{$\Deltaop C$}, is the \CrefAndHyperrefIfExist{definition:diagram_in_a_category_indexed_by_a_small_category}{functor category}
        $$ \mathbf{s}\mathcal{C} := \mathbf{Fun}(\Delta^{\mathrm{op}}, \mathcal{C}).$$
        \CrefIfExists{definition:opposite_category_of_a_category} \CrefIfExists{definition:simplex_category} In particular, a morphism between objects $X,Y:\Delta^{\op} \to \calC$ in this category is a natural transform $X \Rightarrow Y$ from $X$ to $Y$ as functors. 

        An object $X$ of $\mathbf{s}\mathcal{C}$ is called a \hldef{simplicial object of $\mathcal{C}$}, and, by \CrefAndHyperrefIfExist{definition:functor_between_categories}{definition}, consists of a family of objects $\{X_n\}_{n \ge 0}$ in $\mathcal{C}$ together with morphisms
        $$ X(\theta) : X_n \to X_m, \quad \text{for each } \theta : [m] \to [n] \text{ in } \Delta, $$
        satisfying the functoriality conditions
        $$ X(\mathrm{id}_{[n]}) = \mathrm{id}_{X_n}, \qquad X(\theta \circ \psi) = X(\psi) \circ X(\theta).  $$

        \item Dually, the \hldef{category of cosimplicial objects in $\mathcal{C}$}, commonly denoted by notatoins such as \hl{$\mathbf{c}\mathcal{C}$}, \hl{$\operatorname{Cosimp} \mathcal{C}$}, or \hl{$\Delta^{\calC}$} (cf. \Cref{definition:diagram_in_a_category_indexed_by_a_small_category}) is the functor category
        $$
        \mathbf{c}\mathcal{C} := \mathbf{Fun}(\Delta, \mathcal{C}).
        $$
        In particular, a morphism between objects $X,Y:\Delta \to \calC$ in this category is a natural transform $X \Rightarrow Y$ from $X$ to $Y$ as functors. 

        An object $Y$ of $\mathbf{c}\mathcal{C}$ is called a \hldef{cosimplicial object of $\mathcal{C}$}, and consists of a family of objects $\{Y^n\}_{n \ge 0}$ in $\mathcal{C}$ together with morphisms
        $$
        Y(\theta) : Y^m \to Y^n, \quad \text{for each } \theta : [m] \to [n] \text{ in } \Delta,
        $$
        satisfying the functoriality conditions
        $$
        Y(\mathrm{id}_{[n]}) = \mathrm{id}_{Y^n}, \qquad Y(\theta \circ \psi) = Y(\theta) \circ Y(\psi).
        $$
    \end{enumerate}
For instance, a \hldef{(co)simplicial set, group, topological space, ring, etc.} refers to a (co)simplicial object in the category of sets, of groups, of topological spaces, of rings, etc. and such categories are denoted by notations such as \hl{$\Sets_\Delta$}, \hl{$\mathbf{Grps}_\Delta$}, \hl{$\mathbf{Top}_\Delta$}, \hl{$\mathbf{Rings}_\Delta$}, etc. Accordingly, a \hl{(co)simplicial map} between such (co)simplicial objects refers to a morphism in the appropriate (co)simplicial category.

If $\mathcal{C}$ is \CrefAndHyperrefIfExist{definition:locally_small_category}{locally small}, then both $\mathbf{s}\mathcal{C}$ and $\mathbf{c}\mathcal{C}$ are locally small as well.
\end{definition}

\begin{definition} \label{definition:simplex_of_a_simplicial_object_in_a_category_and_face_and_degeneracy_maps}
Let $\mathcal{C}$ be a \CrefAndHyperrefIfExist{definition:category}{category}. 
\begin{enumerate}
    \item Let $X$ be a \CrefAndHyperrefIfExist{definition:simplicial_cosimplicial_object_in_a_category}{simplicial object} in $\mathcal{C}$. An object \hl{$X_n := X([n])$} of $\mathcal{C}$ is called the \hldef{$n$-simplices of $X$}. In case that $\calC$ is some kind of category of sets, an element of $X_n$ is called an \hldef{$n$-simplex of $X$}, so $X_n$ is the \hldef{set of $n$-simplices of $X$}. In this case, a \hldef{vertex of $X$} moreover refers to a $0$-simplex of $X$ and a \hldef{edge of $X$} refers to a $1$-simplex of $X$. 



    For each morphism $\theta : [m] \to [n]$ in $\Delta$, the induced morphism
    $$
    X(\theta) : X_n \to X_m
    $$
    in $\mathcal{C}$ is called a \hldef{simplicial morphism}. 

    For each $0 \leq j \leq n$, the \hldef{$j$th face map of the $n$-simplicies} refers to the map 
    $$\hlin{d_j = X(p_j): X_n \to X_{n-1}, \quad p_j: [n-1] \to [n], \quad p_j(i) = \begin{cases} i &\text{if } i < j  \\ i+1 &\text{if } i \geq j \end{cases}}.$$
    $d_j$ is also denoted by \hl{$\partial_j$}.

    For each $0 \leq j \leq n$, the \hldef{$j$th degeneracy map of the $n$-simplicies} refers to the map 
    $$\hlin{s_j = X(q_j): X_n \to X_{n+1}, \quad q_j: [n+1] \to [n], \quad q_j(i) = \begin{cases} i &\text{if } i \leq j  \\ i-1 &\text{if } i > j \end{cases}}.$$
    
    \item Let $Y : \Delta \to \mathcal{C}$ be a \CrefAndHyperrefIfExist{definition:simplicial_cosimplicial_object_in_a_category}{cosimplicial object} of $\mathcal{C}$. An object \hl{$Y^n := Y([n])$} of $\calC$ is called the \hldef{$n$-cosimplicies of $Y$}. In case that $\calC$ is some kind of category of sets, an element of $Y_n$ is called an \hldef{$n$-cosimplex of $Y$}, so $Y^n$ is the \hldef{set of $n$-cosimplices of $Y$}.
. 

    
    For each morphism $\theta : [m] \to [n]$ in $\Delta$, the induced morphism
    $$ Y(\theta) : Y^m \to Y^n $$
    is called a \hldef{cosimplicial morphism}.  

    For each $0 \leq j \leq n$, the \hldef{$j$th coface map of the $n$-cosimplicies} refers to the map
    $$\hlin{d^j = Y(p_j) : Y^n \to Y^{n+1}, \quad p_j : [n] \to [n+1], \quad p_j(i) = 
    \begin{cases}
    i & \text{if } i < j \\
    i + 1 & \text{if } i \geq j
    \end{cases}
    }$$
    $d^j$ is also denoted by \hl{$\partial^j$}.

    For each $0 \leq j \leq n$, the \hldef{$j$th codegeneracy map of the $n$-cosimplicies} refers to the map
    $$\hlin{s^j = Y(q_j): Y^n \to Y^{n-1}, \quad q_j: [n] \to [n-1], \quad q_j(i) =
    \begin{cases}
    i & \text{if } i \leq j \\
    i - 1 & \text{if } i > j
    \end{cases}
    }$$
    % If $\theta$ is injective, then $Y(\theta)$ is called a \hldef{coface map}; if $\theta$ is surjective, then $Y(\theta)$ is called a \hldef{codegeneracy map}.
\end{enumerate}
A \hldef{(co)face/degeneracy of a the $n$-(co)simplicies of a (co)simplicial object} may also refer to the images of the (co)face/degeneracy maps.
\end{definition}

\begin{definition} \label{definition:standard_n_simplex}
    Let $n \geq 0$ be an integer. The \CrefAndHyperrefIfExist{definition:representable_functor_on_a_category_enriched_in_a_monoidal_category}{representable functor}
    $$\Hom_\Delta(-,[n]): \Delta^{\op} \to \Sets$$
    \CrefIfExists{definition:simplex_category} \CrefIfExists{definition:opposite_category_of_a_category} \CrefIfExists{definition:category_of_sets}
    is a \CrefAndHyperrefIfExist{definition:simplicial_cosimplicial_object_in_a_category}{simplicial set} often denoted by \hl{$\Delta^n$}, \hl{$\Delta^{[n]}$}, or \hl{$\Delta[n]$}, and is called the \hldef{standard $n$-simplex}. More generally, if $J$ is a finite nonempty linearly ordered set, then we may speak of the simplicial set \hldef{$\Delta^J$} given by the representable functor $\Hom_\Delta(-, J)$. 

    The \hldef{Standard simplicial $n$-simplex functor} refers to the functor
    $$\Delta^\bullet: \Delta \to \mathbf{s}\Sets$$
    given by $[n] \mapsto \Delta^n$. By construct, note that $\Delta^\bullet$ is a \CrefAndHyperrefIfExist{definition:simplicial_cosimplicial_object_in_a_category}{cosimplicial object} in the category of simplicial sets. 

    Dually, the functor
    $$\Hom_{\Delta}([n], -): \Delta \to \Sets$$
    is a \CrefAndHyperrefIfExist{definition:simplicial_cosimplicial_object_in_a_category}{cosimplicial set} called the \hldef{standard $n$-cosimplex}.
\end{definition}

\begin{definition}[Model Category] \label{definition:model_category}
    \TODO{I don't like the axioms as stated here}
A \hldef{model category}, or synonymously a \hldef{closed model category}, is a \CrefAndHyperrefIfExist{definition:complete_and_cocomplete_category}{complete and cocomplete category} \(\mathcal{M}\) equipped with three distinguished classes of morphisms:

\begin{itemize}
  \item \hldef{Weak equivalences} \(\mathcal{W}\),
  \item \hldef{Fibrations} \(\mathcal{F}\),
  \item \hldef{Cofibrations} \(\mathcal{C}\),
\end{itemize}

\begin{enumerate}
  \item \textbf{(Two-out-of-three)}: For any composable morphisms \(f: X \to Y\), \(g: Y \to Z\), if any two of \(f\), \(g\), or \(g \circ f\) lie in \(\mathcal{W}\), then so does the third.

  \item \textbf{(Retracts)}: Each of the classes \(\mathcal{W}, \mathcal{F}, \mathcal{C}\) is closed under retracts in the arrow category \(\mathcal{M}^2\). That is, if \(f\) is a retract of \(g\) and \(g\) belongs to one of these classes, then \(f\) also belongs to that class.

  \item \textbf{(Lifting)}: Given any commutative square
  \[
  \begin{tikzcd}
  A \arrow[r] \arrow[d, "i"'] & X \arrow[d, "p"] \\
  B \arrow[r] & Y
  \end{tikzcd}
  \]
  where \(i \in \mathcal{C}\) and \(p \in \mathcal{F}\), a diagonal filler (lift) exists making both triangles commute provided either
  \begin{itemize}
    \item \(i\) is also a weak equivalence (called an acyclic cofibration), or
    \item \(p\) is also a weak equivalence (called an acyclic fibration).
  \end{itemize}

  Formally, acyclic cofibrations have the left lifting property with respect to all fibrations, and cofibrations have the left lifting property with respect to all acyclic fibrations.

  \item \textbf{(Factorization)}: Every morphism \(f : X \to Y\) in \(\mathcal{M}\) admits two functorial factorizations:
  \begin{itemize}
    \item \(f = p \circ i\), where \(i \in \mathcal{C}\) is a cofibration and \(p \in \mathcal{F} \cap \mathcal{W}\) is an acyclic fibration.
    \item \(f = q \circ j\), where \(j \in \mathcal{C} \cap \mathcal{W}\) is an acyclic cofibration and \(q \in \mathcal{F}\) is a fibration.
  \end{itemize}
  
\end{enumerate}

Here, an \hldef{acyclic fibration} (or \hldef{trivial fibration}) is a morphism in \(\mathcal{F} \cap \mathcal{W}\), and an \hldef{acyclic cofibration} (or \hldef{trivial cofibration}) is a morphism in \(\mathcal{C} \cap \mathcal{W}\).
\end{definition}

% \begin{definition}[Homotopy Category of a Model Category] \label{definition:homotopy_category_of_a_model_category}
% The \hldef{homotopy category} \(\mathrm{Ho}(\mathcal{M})\) of a \CrefAndHyperrefIfExist{definition:model_category}{model category} \(\mathcal{M}\) is the localization \TODO{need to make this localization precise.}
% \[ \mathrm{Ho}(\mathcal{M}) = \mathcal{M}[\mathcal{W}^{-1}] \]
% where the morphisms are formally inverted weak equivalences.
% \end{definition}

\begin{definition}[Homotopy category of a model category] \label{definition:homotopy_category_of_a_model_category}
    Let $\mathcal{M}$ be a \CrefAndHyperrefIfExist{definition:model_category}{model category}. The \hldef{homotopy category of $\mathcal{M}$}, denoted by notations such as \hl{$\mathrm{Ho}(\mathcal{M})$}, \hl{$\mathrm{h}(\calM)$}, etc. , is the \CrefAndHyperrefIfExist{definition:category}{category} $\calM[\calW^{-1}]$ whose objects are those of $\mathcal{M}$, and whose morphisms are equivalence classes of morphisms in $\mathcal{M}$ under the relation of left and right homotopy, \CrefAndHyperrefIfExist{definition:localization_of_a_category_by_a_multiplicative_system}{localized} at the weak equivalences. Explicitly,
    \begin{itemize}
        \item The objects of $\mathrm{Ho}(\mathcal{M})$ are the same as those in $\mathcal{M}$.
        \item For objects $X,Y$ in $\mathcal{M}$, the morphism set $\operatorname{Hom}_{\mathrm{Ho}(\mathcal{M})}(X,Y)$ consists of maps in $\mathcal{M}$ modulo homotopy, with weak equivalences formally inverted.
    \end{itemize}
\end{definition}


\subsection{The category of spaces in \texorpdfstring{$\bbA^1$}{A1}-homotopy theory}

\begin{definition} \label{definition:A1_homotopy_theoretic_space_over_a_noetherian_scheme_of_finite_dimension}
    Let $S$ be a \CrefAndHyperrefIfExist{definition:locally_noetherian_and_noetherian_scheme}{Noetherian} base scheme of finite \CrefAndHyperrefIfExist{definition:dimension_of_a_scheme}{dimension}. In the context of $\bbA^1$-homotopy theory, the category of \hldef{spaces} is the category \hl{$\simpnissheaves$} of \CrefAndHyperrefIfExist{definition:simplicial_cosimplicial_object_in_a_category}{simplicial sheaves} for the \CrefAndHyperrefIfExist{definition:nisnevich_topology_on_a_noetherian_scheme_of_finite_dimension}{Nisnevich site} on $\Sm/S$. This category is also sometimes denoted by \hl{$\Spc$}. We may also call a space by names such as an \hldef{$\bbA^1$-homotopy (theoretic) space}, or a \hldef{motivic space}.
\end{definition}

\begin{remark}
    In \cite{voevodsky_A1}, Voevodsky developed the theory of the (unstable $\bbA^1$)-homotopy category by letting the category of spaces {\cite[Definition 2.2]{voevodsky_A1}} be merely the category $\Shv(\SmSNis)$ of sheaves for the Nisnevich topology over $S$, instead of simplicial sheaves as in \Cref{definition:A1_homotopy_theoretic_space_over_a_noetherian_scheme_of_finite_dimension}. In \cite[Theorem 3.6]{voevodsky_A1}, it is established that the \CrefAndHyperrefIfExist{definition:A1_model_structure_on_the_category_of_simp_sheaves_of_sets_on_the_nis_site_over_a_noeth_scheme_of_fin_dim_and_the_unstable_A1_homotopy_category}{(unstable) $\bbA^1$-homotopy category} \TODO{} is equivalent to the homotopy category obtainable with sheaves, see \Cref{theorem:homotopy_category_of_ordinary_sheaves_on_nisnevich_site_is_equivalent_to_unstable_homotopy_category} for a statement
    
    % It is now standard to let the category of spaces be the category of simplicial sheaves for the Nisnevich site.
\end{remark}


\begin{convention} \label{convention:objects_of_a_site_as_sheaves_and_constant_simplicial_object_of_a_category}
    Given a site $T$, there is a canonical functor, in fact a fully faithful embedding, $T \to \Shv(T)$ sending an object of $T$ to the associated sheaf, i.e. the functor is given by 
    $$X \mapsto (Y \mapsto \Hom_{T}(Y, X));$$
    more informally, this is to say that every object $X \in T$ embeds as a representable functor in $\Shv(T)$. As such, we often speak of objects of $T$ as objects of $\Shv(T)$.

    Moreover, given any category $C$, there is a ``constant simplicial object functor'' $C \to \Deltaop C$ given by sending an object $X$ of $C$ to the simplicial object $\Deltaop \to C$ given by $[n] \mapsto C$ for all $n \geq 0$ and $([n] \to [m]) \mapsto (C \xrightarrow{\id} C)$. Thus, given a sheaf $X \in \Shv(T)$, we can speak of $X$ as though it were a simplicial sheaf $\Deltaop \to \Shv(T)$. 

    \TODO{Does this convention identify sipmlicial sets as simplicial sheaves?}

    \TODO{I might need to go fixing "Noetherian scheme" to "Noetherian scheme of finite dimension" to make sure the Nisnevich site is defined.}

    Therefore, given a \CrefAndHyperrefIfExist{definition:locally_noetherian_and_noetherian_scheme}{Noetherian} base scheme $S$, one can consider $S$ as a space (i.e. an object of $\simpnissheaves$). In fact, $S$ is the final object of $\Shv(\SmSNis)$ and of $\simpnissheaves$. As per \Cref{notation:sheaves_of_a_site}, this final object of $\Shv(\SmSNis)$ may again be denoted by $\pt$, $*$, or $\bullet$; note that $\pt$ also makes sense as a space by the above discussion.  

\end{convention}

See \Cref{definition:point_of_a_topos} for the definition of a point of a topos. For our purposes, we can alternatively define a point (of a site $T$) as a functor $x^*: \Shv(T) \to \Sets$ which commutes with finite limits and arbitrary colimits.
\begin{lemma}
    Let $T$ be a site. A point $x = (x_*, x^*, \varphi): \Sets \to Shv(T)$, as a morphism of topoi, is equivalent to the data of a functor $\Shv(T) \to \Sets$ that commutes with finite limits and all colimits.
\end{lemma}
\begin{proof}
    See \cite[\href{https://stacks.math.columbia.edu/tag/00Y3}{Tag 00Y3}]{stacks-project} 
\end{proof}
\begin{convention}
    A point of a site $T$ can be identified with a point of the topos $\Shv(T)$. 
\end{convention}


\begin{definition} \label{definition:pointed_space_in_motivic_homotopy_theory}
    Let $S$ be a \CrefAndHyperrefIfExist{definition:locally_noetherian_and_noetherian_scheme}{Noetherian} base scheme. In the context of $\A^1$-homotopy theory, a \hldef{point} of a space $X \in \simpnissheaves$ is a morphism $\pt \to X$. A \hldef{pointed space} is a space equipped with a point --- as such, the category of pointed spaces is the category $\pointedsimpnissheaves$ of pointed simplicial sheaves in the Nisnevich topology on $\Sm/S$.
\end{definition}

\begin{remark}
    Note that we have two notions of points --- one is the notion of a point of a topos (\Cref{definition:point_of_a_topos}), say of the topos $\ShvNisSmS$. Another is the notion of a point of a space (\Cref{definition:pointed_sheaves_of_sets_on_a_site_simplicial_pointed_sheaves_of_sets_on_a_site}). It may be fitting to think of the former type of point as a ``point of the base scheme $S$''.
\end{remark}

\section{How to define the \texorpdfstring{$\bbA^1$}{A1}-homotopy category}

\subsection{The simplicial model structure on the category of simplicial sheaves on a site}


For a general small site $T$, there is a model category structure on $\Deltaop\Shv(T)$ called the simplicial model category structure. Defining this for $T = \SmSNis$ is a preliminary step towards defining the $\bbA^1$-homotopy categories $\calH(S)$ and $\calH_\bullet(S)$.

\begin{definition}[Standard model category structure on simplicial sets] \label{definition:standard_model_structure_on_the_category_of_simplicial_sets}
    Let $\mathsf{sSet}$ denote the category of \CrefAndHyperrefIfExist{definition:simplicial_cosimplicial_object_in_a_category}{simplicial sets}. The \hldef{standard model category structure on $\mathsf{sSet}$}, also known as the \hldef{Kan model structure} or the \hldef{Quillen model structure} or the \hldef{Kan-Quillen model structure}, is the \CrefAndHyperrefIfExist{definition:model_category}{model category} given by specifying three classes of morphisms:

    \begin{itemize}
    \item \textbf{Weak equivalences:} A map $f : X \to Y$ in $\mathsf{sSet}$ is a weak equivalence, also called a \hldef{weak homotopy equivalence between the simplicial sets $X$ and $Y$}, if the induced map of \CrefAndHyperrefIfExist{definition:geometric_realization_of_a_simplicial_set}{geometric realizations} $|f| : |X| \to |Y|$ is a \CrefAndHyperrefIfExist{definition:weak_homotopy_equivalence_of_topological_spaces}{weak homotopy equivalence} of topological spaces (i.e., induces isomorphisms on all homotopy groups for all choices of basepoint).
    \item \textbf{Fibrations:} A map $f : X \to Y$ is a \hldef{(Kan) fibration} if it has the right lifting property with respect to the horn inclusions $\Lambda_k^n \hookrightarrow \Delta^n$ for all $n \ge 1$ and $0 \le k \le n$.
    \item \textbf{Cofibrations:} A map $f : X \to Y$ is a cofibration if it is a \CrefAndHyperrefIfExist{definition:monomorphism_and_epimorphism_in_categories}{monomorphism} (i.e., injective at each level).
    \end{itemize}
    For this model category structure, all objects are \CrefAndHyperrefIfExist{definition:fibrant_cofibrant_object_in_a_model_category}{cofibrant}, and the \CrefAndHyperrefIfExist{definition:fibrant_cofibrant_object_in_a_model_category}{fibrant objects} are exactly the \CrefAndHyperrefIfExist{definition:kan_complex}{Kan complexes}.
    % With these choices, $\mathsf{sSet}$ is a model category in the sense of Quillen. All objects are cofibrant, and the fibrant objects are exactly the Kan complexes.
\end{definition}

\begin{definition}[See e.g. {\cite[Section 2 Definition 1.2]{morel_voevodsky_1999}}] \label{definition:simplicial_weak_equivalences_cofibrations_and_fibrations_for_simplicial_sheaves_on_a_site}
    Let $f: X \to Y$ be a morphism of simplicial sheaves on some site $T$.
    \begin{enumerate}
        \item $f$ is called a \hldef{(simplicial) weak eqiuvalence} if for any \CrefAndHyperrefIfExist{definition:point_of_a_topos}{point} $x$ of the site $T$, the morphism $x^*(f) : x^*(Y) \to x^*(Y)$ is a \CrefAndHyperrefIfExist{definition:standard_model_structure_on_the_category_of_simplicial_sets}{weak equivalence} of simplicial sets.
        \item $f$ is called a \hldef{(simplicial) cofibration} if it is a \CrefAndHyperrefIfExist{definition:monomorphism_and_epimorphism_in_categories}{monomorphism}.
        \TODO{right lifting property}
        \item $f$ is called a \hldef{(simplicial) fibration} if it has the right lifting property with respect to any cofibration that is also a weak equivalence.
    \end{enumerate}
    A morphism $f: X \to Y$ of \CrefAndHyperrefIfExist{definition:pointed_object_of_a_category_with_a_final_object}{pointed} \CrefAndHyperrefIfExist{definition:simplicial_cosimplicial_object_in_a_category}{simplicial} \CrefAndHyperrefIfExist{definition:sheaf_on_a_site}{sheaves} of sets (on some site) is called a \hldef{simplicial weak equivalence} (resp. \hldef{simplicial cofibration}, resp. \hldef{simplicial fibration}) if it is a simplicial weak equivalence (resp. simplicial cofibration, resp. simplicial fibration) as a morphism of simplicial sheaves.
\end{definition}


\begin{theorem}[See e.g. {\cite[Section 2 Theorem 1.4]{morel_voevodsky_1999}}] \label{theorem:simplicial_model_category_structure_on_simplicial_sheaves}
    For any \CrefAndHyperrefIfExist{definition:locally_small_category}{small} \CrefAndHyperrefIfExist{definition:grothendieck_topology_on_a_category_site_covering_sieve_topologically_generating_family}{site} $T$, the classes of \CrefAndHyperrefIfExist{definition:simplicial_weak_equivalences_cofibrations_and_fibrations_for_simplicial_sheaves_on_a_site}{simplicial weak equivalences, simplicial cofibrations, and simplicial fibrations} give $\Deltaop \Shv(T)$ the structure of a \CrefAndHyperrefIfExist{definition:model_category}{model category}. 

    There is a similar model category structure on $\Deltaop \Shv(T)_\bullet$.
\end{theorem}

\begin{definition} \label{definition:simplicial_model_category_struture_on_the_category_of_simplicial_sheaves_on_a_small_site_and_simplicial_homotopy_category_of_a_site}
    Let $T$ be a \CrefAndHyperrefIfExist{definition:locally_small_category}{small} \CrefAndHyperrefIfExist{definition:grothendieck_topology_on_a_category_site_covering_sieve_topologically_generating_family}{site}. The model category structure on $\Deltaop \Shv(T)$ discussed in \Cref{theorem:simplicial_model_category_structure_on_simplicial_sheaves} is referred to as the \hldef{simplicial model category structure on $\Deltaop\Shv(T)$}. We call the associated \CrefAndHyperrefIfExist{definition:homotopy_category_of_a_model_category}{homotopy category} the \hldef{simplicial homotopy category of $T$ (or of $\Shv(T)$ or of $\Deltaop\Shv(T)$)} and denote it by \hl{$\calH_s(T)$}. 

    Similarly, the model category structure on $\Deltaop \Shv(T)_\bullet$ is referred to as the \hldef{(pointed) simplicial model category structure on $\Deltaop \Shv(T)_\bullet$}. We call the associated homotopy category the \hldef{pointed simplicial homotopy category of $T$ (or of $\Shv(T)$ or of $\Deltaop\Shv(T)$)} and denote it by \hl{$\calH_s(T)_\bullet$} or \hl{$\calH_{s\bullet}(T)$}. 
\end{definition}

\subsection{General model structures on categories of simplicial sheaves}

There are two more general model category structures on $\Deltaop(\Shv(T))$ whose theories \cite{morel_voevodsky_1999} relies upon to define the $\bbA^1$-model category structure. One (\Cref{definition:A_local_object_A_weak_equivalence_A_fibration_of_simplicial_homotopy_category_of_sheaves_of_a_small_site}) is defined with respect to a set $A$ of morphisms in \CrefAndHyperrefIfExist{definition:simplicial_model_category_struture_on_the_category_of_simplicial_sheaves_on_a_small_site_and_simplicial_homotopy_category_of_a_site}{$\calH_s(T)$} and the other (\Cref{definition:I_local_weak_I_weak_equivalence_I_weak_fibration_for_morphisms_of_simplicial_sheaves_on_a_site_with_interval}) is defined with respect to a site with interval. The $\bbA^1$-simplicial model structure is a special case of the latter, which in turn is a special case of the former. 

In defining these model category structures, there is a general pattern --- we define ``local'' objects in $\calH_s(T)$, then define what it means for a morphism in $\Deltaop(\Shv(T))$ to be a ``weak-equivalence'' or to be a ``fibration''.

\begin{definition}[See e.g. {\cite[Section 2 Definition 2.1, Definition 2.2]{morel_voevodsky_1999}}] \label{definition:A_local_object_A_weak_equivalence_A_fibration_of_simplicial_homotopy_category_of_sheaves_of_a_small_site}
Let $T$ be a \CrefAndHyperrefIfExist{definition:locally_small_category}{small} \CrefAndHyperrefIfExist{definition:grothendieck_topology_on_a_category_site_covering_sieve_topologically_generating_family}{site} and let $A$ be a set of morphisms in \CrefAndHyperrefIfExist{definition:simplicial_model_category_struture_on_the_category_of_simplicial_sheaves_on_a_small_site_and_simplicial_homotopy_category_of_a_site}{$\calH_s(T)$}. 
\begin{enumerate}
    \item An object $X$ of \CrefAndHyperrefIfExist{definition:simplicial_model_category_struture_on_the_category_of_simplicial_sheaves_on_a_small_site}{$\calH_s(T)$} is called \hldef{$A$-local} if for any $Y \in \calH_s(T)$ and any $f: Z_1 \to Z_2$ in $A$, the map 
    $$\Hom_{\calH_s(T)}(Y \times Z_2, X) \to \Hom_{\calH_s(T)}(Y \times Z_1, X)$$
    is a bijection. Write \hl{$\calH_{s,A}(T)$} for the full subcategory of $\calH_s(T)$ of $A$-local objects. 
    \item A morphism $f: X_1 \to X_2$ in $\Deltaop(\Shv(T))$ is called an \hldef{$A$-weak equivalence} if for any $A$-local object $Y$ the map
    $$\Hom_{\calH_s(T)}(X_2, Y) \to \Hom_{\calH_s(T)}(X_1, Y)$$
    induced by $f$ is a bijection. 
    \item A morphism $f: X_1 \to X_2$ in $\Deltaop(\Shv(T))$ is called an \hldef{$A$-fibration} if it has the right lifting property with respect to monomorphisms (i.e. simplicial cofibrations) that are also $A$-weak equivalences.
\end{enumerate}
A morphism $f: X \to Y$ in $\calH_s(T)_\bullet$ is called an \hldef{$A$-weak equivalence} (resp. \hldef{$A$-fibration}) if it is an $A$-weak equivalence (resp. $A$-fibration) as an unpointed morphism.
\end{definition}

\begin{theorem}[{\cite[Section 2 Theorem 2.5]{morel_voevodsky_1999}}] \label{theorem:A_model_category_and_A_localization_fnctor}
Let $T$ be a \CrefAndHyperrefIfExist{definition:locally_small_category}{small} \CrefAndHyperrefIfExist{definition:grothendieck_topology_on_a_category_site_covering_sieve_topologically_generating_family}{site} and let $A$ be a set of morphisms in \CrefAndHyperrefIfExist{definition:simplicial_model_category_struture_on_the_category_of_simplicial_sheaves_on_a_small_site_and_simplicial_homotopy_category_of_a_site}{$\calH_s(T)$}. The classes of \CrefAndHyperrefIfExist{definition:A_local_object_A_weak_equivalence_A_fibration_of_simplicial_homotopy_category_of_sheaves_of_a_small_site}{$A$-weak equivalences, monomorphisms, and $A$-fibrations} give $\Deltaop(\Shv(T))$ \CrefIfExists{definition:simplicial_cosimplicial_object_in_a_category} \CrefIfExists{definition:sheaf_on_a_site} the structure of a \CrefAndHyperrefIfExist{definition:model_category}{model category}. 

The inclusion functor \CrefAndHyperrefIfExist{definition:A_local_object_A_weak_equivalence_A_fibration_of_simplicial_homotopy_category_of_sheaves_of_a_small_site}{$\calH_{s,A}(T) \hookrightarrow \calH_{s}(T)$} has a \CrefAndHyperrefIfExist{definition:adjoint_functors_between_categories_unit_counit_of_adjoint_functors}{left adjoint functor} 
$$\hlin{L_A: \calH_{s}(T) \to \calH_{s,A}(T),}$$
which we call the \hldef{$A$-localization functor}, which identifies $\calH_{s,A}(T)$ with the localization of $\calH_{s}(T)$ with respect to $A$-weak equivalences.
\TODO{localizatoin}
\end{theorem}


\begin{definition}[See {\cite[Section 2.3]{morel_voevodsky_1999}}] \label{definition:interval_in_a_site_with_enough_points}
    Let $T$ be a \CrefAndHyperrefIfExist{definition:grothendieck_topology_on_a_category_site_covering_sieve_topologically_generating_family}{site} \CrefAndHyperrefIfExist{definition:site_with_enough_points}{with enough points}. An \hldef{interval in $T$} is a \CrefAndHyperrefIfExist{definition:sheaf_on_a_site}{sheaf} $I$ of sets on $T$ together with \CrefAndHyperrefIfExist{definition:sheaf_on_a_site}{sheaf morphisms}

    \hlalign{
    \begin{align*}
        \mu: I \times I \to I \\
        i_0,i_1: \pt \to I
    \end{align*}
    }
    \CrefIfExists{definition:product_and_coproduct_of_objects_in_a_category}

    satisfying the following:
    \begin{itemize}
        \item Writing $p$ be the canonical morphism $I \to \pt$, 
        \begin{align*}
            \mu(i_0 \times \id) &= \mu(\id \times i_0) = i_0 p \\
            \mu(i_1 \times \id) &= \mu(\id \times i_1) = \id
        \end{align*}
        \item The morphism $i_0 \coprod i_1: \pt \coprod \pt \to I$ \CrefIfExists{definition:product_and_coproduct_of_objects_in_a_category} is a monomorphism.
    \end{itemize}
    A site $T$ with enough points equipped with an interval is called a \hldef{site with interval}.
\end{definition}

\begin{definition}[See {\cite[Section 2 Definition 3.1]{morel_voevodsky_1999}}] \label{definition:I_local_weak_I_weak_equivalence_I_weak_fibration_for_morphisms_of_simplicial_sheaves_on_a_site_with_interval}
    Let $(T,I)$ be a \CrefAndHyperrefIfExist{definition:interval_in_a_site_with_enough_points}{site with interval}. 
    \begin{enumerate}
        \item A \CrefAndHyperrefIfExist{definition:simplicial_cosimplicial_object_in_a_category}{simplicial} \CrefAndHyperrefIfExist{definition:sheaf_on_a_site}{sheaf} $X$ (of sets) is called \hldef{$I$-local} if for any simplicial sheaf $Y$ the map
        $$\Hom_{\calH_s(T)}(Y \times I, X) \to \Hom_{\calH_s(T)}(Y, X)$$
        induced by $i_0: \pt \to I$ is a bijection. Write \hl{$\calH_{s,I}(T)$} for the full subcategory of \CrefAndHyperrefIfExist{definition:simplicial_model_category_struture_on_the_category_of_simplicial_sheaves_on_a_small_site_and_simplicial_homotopy_category_of_a_site}{$\calH_s(T)$} of $I$-local objects. 

        \item A morphism $f: X_1 \to Y_2$ in \CrefAndHyperrefIfExist{definition:simplicial_cosimplicial_object_in_a_category}{$\Deltaop(\Shv(T))$} is called an \hldef{$I$-weak equivalence} if for any $I$-local object $Y$ the map
        $$\Hom_{\calH_s(T)}(X_2, Y) \to \Hom_{\calH_s(T)}(X_1, Y)$$
        is a bijection.

        \item A morphism $f: X_1 \to Y_2$ in $\Deltaop(\Shv(T))$ is called an \hldef{$I$-weak fibration} if it has the right lifting property with respect to monomorphisms (i.e. simplicial cofibrations) that are also $I$-weak equivalences.

        \TODO{What is an $I$-fibration and how does it differ from an $I$-weak fibration?}
    \end{enumerate}
    A morphism $f: X \to Y$ in $\Deltaop(\Shv(T))_\bullet$ is called an \hldef{$I$-weak equivalence} (resp. \hldef{$I$-fibration}) if it is an $I$-weak equivalence (resp. $I$-fibration) as an unpointed morphism.
\end{definition}

\begin{lemma}
    Let $(T,I)$ be a \CrefAndHyperrefIfExist{definition:locally_small_category}{small} \CrefAndHyperrefIfExist{definition:grothendieck_topology_on_a_category_site_covering_sieve_topologically_generating_family}{site} \CrefAndHyperrefIfExist{definition:interval_in_a_site_with_enough_points}{with interval}. Let $A = \{i_0\}$.
    \begin{itemize}
        \item A \CrefAndHyperrefIfExist{definition:simplicial_cosimplicial_object_in_a_category}{simplicial} \CrefAndHyperrefIfExist{definition:sheaf_on_a_site}{sheaf} (of sets) $X \in \Deltaop(\Shv(T))$ is \CrefAndHyperrefIfExist{definition:I_local_weak_I_weak_equivalence_I_weak_fibration_for_morphisms_of_simplicial_sheaves_on_a_site_with_interval}{$I$-local} if and only if it is \CrefAndHyperrefIfExist{definition:A_local_object_A_weak_equivalence_A_fibration_of_simplicial_homotopy_category_of_sheaves_of_a_small_site}{$A$-local}.

        \item A morphism $f: X_1 \to X_2$ in $\Deltaop(\Shv(T))$ is an \CrefAndHyperrefIfExist{definition:I_local_weak_I_weak_equivalence_I_weak_fibration_for_morphisms_of_simplicial_sheaves_on_a_site_with_interval}{$I$-weak equivalence} if and only if it is an \CrefAndHyperrefIfExist{definition:A_local_object_A_weak_equivalence_A_fibration_of_simplicial_homotopy_category_of_sheaves_of_a_small_site}{$A$-weak equivalence}.

        \TODO{What is an $I$-fibration and how does it differ from an $I$-weak fibration?}
        \item A morphism $f: X_1 \to X_2$ in $\Deltaop(\Shv(T))$ is an \CrefAndHyperrefIfExist{definition:I_local_weak_I_weak_equivalence_I_weak_fibration_for_morphisms_of_simplicial_sheaves_on_a_site_with_interval}{$I$-fibration} if and only if it is an \CrefAndHyperrefIfExist{definition:A_local_object_A_weak_equivalence_A_fibration_of_simplicial_homotopy_category_of_sheaves_of_a_small_site}{$A$-weak fibration}.
    \end{itemize}
\end{lemma}
\begin{proof}
    These are all clear by the definitions. 
\end{proof}

\begin{theorem}[{\cite[Section 2 Theorem 3.2]{morel_voevodsky_1999}}] \label{theorem:I_model_category_of_simplicial_sheaves_on_a_small_site_with_interval}
    Let $(T,I)$ be a \CrefAndHyperrefIfExist{definition:locally_small_category}{small} \CrefAndHyperrefIfExist{definition:interval_in_a_site_with_enough_points}{site with interval}. The classes of \CrefAndHyperrefIfExist{definition:I_local_weak_I_weak_equivalence_I_weak_fibration_for_morphisms_of_simplicial_sheaves_on_a_site_with_interval}{$I$-weak equivalences, monomorphisms, and $I$-fibrations} give $\Deltaop(\Shv(T))$ the structure of a \CrefAndHyperrefIfExist{definition:model_category}{model category}. We may refer to this model category as the \hldef{$I$-model category on $\Delta^{\op}\Shv(T)$}.

    The inclusion functor $\calH_{s,I}(T) \hookrightarrow \calH_{s}(T)$ (\Cref{definition:I_local_weak_I_weak_equivalence_I_weak_fibration_for_morphisms_of_simplicial_sheaves_on_a_site_with_interval}, \Cref{definition:simplicial_model_category_struture_on_the_category_of_simplicial_sheaves_on_a_small_site_and_simplicial_homotopy_category_of_a_site}) has a \CrefAndHyperrefIfExist{definition:adjoint_functors_between_categories_unit_counit_of_adjoint_functors}{left adjoint} functor \hl{$L_I: \calH_{s}(T) \to \calH_{s,I}(T)$}, which we call the \hldef{$I$-localization functor}, which identifies $\calH_{s,I}(T)$ with the \CrefAndHyperrefIfExist{definition:homotopy_category_of_a_model_category}{homotopy category} of $\calH_s(T)$, i.e. the \CrefAndHyperrefIfExist{definition:localization_of_a_category_by_a_multiplicative_system}{localization} of $\calH_{s}(T)$ with respect to $I$-weak equivalences.
\end{theorem}


\begin{remark}{\cite[Section 2 Remark 3.3]{morel_voevodsky_1999}}
    The \CrefAndHyperrefIfExist{theorem:I_model_category_of_simplicial_sheaves_on_a_small_site_with_interval}{$I$-model category structure} on $\Deltaop\Shv(T)$ only depends on the object $I$ and not on the morphism $i_0$.
\end{remark}


\subsection{The \texorpdfstring{$\bbA^1$}{A1}-simplicial model structure on the category of spaces and the \texorpdfstring{$\bbA^1$}{A1}-homotopy category}

% \TODO{Mkae a defintion for this?}
\begin{convention} \label{convention:standard_simplicial_n_simplex_functor_as_a_functor_from_Delta_to_Delta_op_Shv_T}
    Let $T$ be a \CrefAndHyperrefIfExist{definition:grothendieck_topology_on_a_category_site_covering_sieve_topologically_generating_family}{site}.
    Recall that the \CrefAndHyperrefIfExist{definition:standard_n_simplex}{standard simplicial $n$-simplex functor $\Delta^\bullet$} is the functor
    $$\Delta^\bullet: \Delta \to \mathbf{s}\Sets = \Deltaop \Sets$$
    \CrefIfExists{definition:simplex_category} \CrefIfExists{definition:simplicial_cosimplicial_object_in_a_category}
    given by $[n] \mapsto \Delta^n = \Hom_{\Delta}(-, [n])$.
    Identifying objects of $\Deltaop \Sets$ as objects of $\Deltaop \Shv(T)$ (\Cref{convention:objects_of_a_site_as_sheaves_and_constant_simplicial_object_of_a_category}), we identify $\Delta^\bullet$ as a functor
    $$\Delta^\bullet: \Delta \to \Deltaop \Shv(T).$$
    
    % Also write $\Delta^\bullet: \Delta \to \Deltaop\Shv(T)$ be the cosimplicial object given by $[n] \mapsto \Delta^n$.
\end{convention}

\begin{definition}[{\cite[Section 3 Definition 2.1]{morel_voevodsky_1999}}] \label{definition:A1_local_weak_equivalence_fibration_of_simplicial_sheaves_of_sets_on_the_nisnevich_site_over_a_noetherian_finite_dimensional_scheme}
    Let $S$ be a \CrefAndHyperrefIfExist{definition:locally_noetherian_and_noetherian_scheme}{Noetherian} scheme of finite \CrefAndHyperrefIfExist{definition:dimension_of_a_scheme}{dimension}. 
    \begin{enumerate}
        \item A \CrefAndHyperrefIfExist{definition:simplicial_cosimplicial_object_in_a_category}{simplicial} \CrefAndHyperrefIfExist{definition:sheaf_on_a_site}{sheaf} $X$ on \CrefAndHyperrefIfExist{definition:nisnevich_topology_on_a_noetherian_scheme_of_finite_dimension}{$\SmSNis$} is called \hldef{$\bbA^1$-local} if for any simplicial sheaf $Y$ the map
        $$\Hom_{\calH_s(\SmSNis)}(Y, X) \to \Hom_{\calH_s(\SmSNis)}(Y \times \bbA^1, X)$$
        (\Cref{definition:A_local_object_A_weak_equivalence_A_fibration_of_simplicial_homotopy_category_of_sheaves_of_a_small_site}, \Cref{definition:simplicial_model_category_struture_on_the_category_of_simplicial_sheaves_on_a_small_site_and_simplicial_homotopy_category_of_a_site}, \Cref{definition:nisnevich_topology_on_a_noetherian_scheme_of_finite_dimension})
        induced by the projection $Y \times \bbA^1 \to Y$ is a bijection. Write 
        $$\hlin{\calH_{s,\bbA^1}(S) = \calH_{s,\bbA^1}(\ShvNisSmS)}$$
        for the full subcategory of $\calH_{s}(\ShvNisSmS)$ (\Cref{definition:simplicial_model_category_struture_on_the_category_of_simplicial_sheaves_on_a_small_site_and_simplicial_homotopy_category_of_a_site}) of $\bbA^1$-local objects. A posteriori, \CrefAndHyperref{corollary:A1_model_structure_on_simplicial_nisnevich_sheaves_over_a_noetherian_finite_dimensional_scheme_is_an_instance_of_an_I_model_structure}{it turns out} that $\calH_{s,\bbA^1}(S)$ is equivalent to the unstable \CrefAndHyperrefIfExist{definition:A1_model_structure_on_the_category_of_simp_sheaves_of_sets_on_the_nis_site_over_a_noeth_scheme_of_fin_dim_and_the_unstable_A1_homotopy_category}{$\bbA^1$-homotopy category $\calH_{s}^{\bbA^1}(S)$}.

        \item A morphism $f: X_1 \to X_2$ in $\simpnissheaves$ is called an \hldef{$\bbA^1$-weak equivalence} if for any $\bbA^1$-local, simplicially fibrant (i.e. the morphism $Z \to \pt$ is a simplicial fibration) sheaf $Z$, the map 
        $$\Hom_{\simpnissheaves}(X_2 \times \Delta^\bullet,Z) \to \Hom_{\simpnissheaves}(X_1 \times \Delta^\bullet, Z)$$
        (\Cref{convention:standard_simplicial_n_simplex_functor_as_a_functor_from_Delta_to_Delta_op_Shv_T}, \Cref{definition:simplicial_cosimplicial_object_in_a_category})
        induced by $f$ is a weak equvialence of simplicial sets.

        \item A morphism $f: X \to Y$ in $\simpnissheaves$ is called an \hldef{$\bbA^1$-fibration} if it has the right lifting property with respect to monomorphisms that are also $\bbA^1$-weak equivalences.

    \end{enumerate}
    A morphism $f: X \to Y$ in $\Deltaop(\Shv(T))_\bullet$ is called an \hldef{$\bbA^1$-weak equivalence} (resp. \hldef{$\bbA^1$-fibration}) if it is an $\bbA^1$-weak equivalence (resp. $\bbA^1$-fibration) as an unpointed morphism.
\end{definition}

The following lemma shows that $\bbA^1$-homotopy theory is an instance of homotopy theory for sites with interval where the interval is given by the affine line $\bbA^1$.
\begin{lemma} \label{lemma:A1_homotopy_theory_is_a_kind_of_interval_homotopy_theory}
    Let $S$ be a \CrefAndHyperrefIfExist{definition:locally_noetherian_and_noetherian_scheme}{Noetherian} scheme of finite \CrefAndHyperrefIfExist{definition:dimension_of_a_scheme}{dimension}. The following specifies an \CrefAndHyperrefIfExist{definition:interval_in_a_site_with_enough_points}{interval} $I$ in $\SmSNis$:
    \TODO{Why does the nisnevich site have enough points}
    \begin{itemize}
        \item The sheaf $I$ of sets is the sheaf represented by $\bbA^1_S$.
        \item $\mu: \bbA^1 \times \bbA^1 \to \bbA^1$ is given by $(x,y) \mapsto (xy)$.
        \item $i_0,i_1: \pt = S \to \bbA^1_S$ are given by embedding the point into $t = 0$ and $t = 1$ respectively where $t$ is the coordinate of $\bbA^1$. 
    \end{itemize}

    For this interval $I$,
    \begin{enumerate}
        \item a simplicial sheaf $X$ on $\SmSNis$ is \CrefAndHyperrefIfExist{definition:A1_local_weak_equivalence_fibration_of_simplicial_sheaves_of_sets_on_the_nisnevich_site_over_a_noetherian_finite_dimensional_scheme}{$\bbA^1$-local} if and only if it is \CrefAndHyperrefIfExist{definition:I_local_weak_I_weak_equivalence_I_weak_fibration_for_morphisms_of_simplicial_sheaves_on_a_site_with_interval}{$I$-local},

        \item a morphism $f: X_1 \to X_2$ in $\simpnissheaves$ is an \CrefAndHyperrefIfExist{definition:A1_weak_equivalence_of_simplicial_sheaves_of_sets_on_the_nisnevich_topology_over_a_noetherian_finite_dimensional_scheme}{$\bbA^1$-weak equivalence} if and only if it is an \CrefAndHyperrefIfExist{definition:I_local_weak_I_weak_equivalence_I_weak_fibration_for_morphisms_of_simplicial_sheaves_on_a_site_with_interval}{$I$-weak equivalence}, and

        \item a morphism $f: X_1 \to X_2$ in $\simpnissheaves$ is an \CrefAndHyperrefIfExist{definition:A1_weak_equivalence_of_simplicial_sheaves_of_sets_on_the_nisnevich_topology_over_a_noetherian_finite_dimensional_scheme}{$\bbA^1$-fibration} if and only if it is an \CrefAndHyperrefIfExist{definition:I_local_weak_I_weak_equivalence_I_weak_fibration_for_morphisms_of_simplicial_sheaves_on_a_site_with_interval}{$I$-weak fibration}.
    \end{enumerate}
\end{lemma}
\begin{proof}
    It is not difficult to see that the proposed interval is indeed an interval. The notions of $I$-locality and $\bbA^1$-locality coincide by definition. See \cite[Lemma 2.5.5]{choudury_htsA1fg} for an argument that shows that $I$-weak equivalence and $\bbA^1$-weak equivalence are equivalent. The notions of $I$-fibrations and of $\bbA^1$-fibrations then coincide by definition.
\end{proof}


\begin{definition}[Simplicial Model Structure] \label{definition:simplicial_model_structure_on_a_category}
Let $\mathcal{M}$ be a category equipped with a \CrefAndHyperrefIfExist{definition:model_category}{model structure}. Suppose $\mathcal{M}$ is also \CrefAndHyperrefIfExist{definition:category_enriched_in_a_monoidal_category}{enriched over} the category of \CrefAndHyperrefIfExist{definition:simplicial_cosimplicial_object_in_a_category}{simplicial sets $\mathbf{sSet}$}: that is, there is a bifunctor
\[
\mathrm{Map}(-,-) : \mathcal{M}^{op} \times \mathcal{M} \to \mathbf{sSet}
\]
\TODO{tensor, cotensor}
satisfying the standard enrichment axioms. Denote by $-\otimes K$ the tensor (or simplicial action) of an object of $\mathcal{M}$ by a simplicial set $K \in \mathbf{sSet}$ and by $(-)^K$ the cotensor.

Then \hldef{simplicial model structure  on $\mathcal{M}$} consists of the following data and axioms:
\begin{itemize}
  \item The model category structure on $\mathcal{M}$,
  \item The simplicial enrichment $\mathrm{Map}(-,-)$,
  \item The tensoring $-\otimes K : \mathcal{M} \times \mathbf{sSet} \to \mathcal{M}$ and cotensoring $(-)^K : \mathcal{M} \times \mathbf{sSet}^{op} \to \mathcal{M}$,
\end{itemize}
such that the following axioms hold:
\begin{enumerate}
  \item (Compatibility) For all \CrefAndHyperrefIfExist{definition:standard_model_structure_on_the_category_of_simplicial_sets}{cofibrations} $i : A \to B$ in $\mathcal{M}$ and all cofibrations $j : K \to L$ in $\mathbf{sSet}$, the induced map
  $$
  i \, \square \, j : (B \otimes K) \coprod_{A \otimes K} (A \otimes L) \to B \otimes L
  $$
  is a cofibration in $\mathcal{M}$ which is a \CrefAndHyperrefIfExist{definition:standard_model_structure_on_the_category_of_simplicial_sets}{weak equivalence} if either $i$ or $j$ is.

  \item (Simplicial hom) For all cofibrations $i: A \to B$ in $\mathcal{M}$ and all fibrations $p: X \to Y$ in $\mathcal{M}$, the induced map of simplicial sets
  $$
  \mathrm{Map}(B,X) \to \mathrm{Map}(A,X) \times_{\mathrm{Map}(A,Y)} \mathrm{Map}(B,Y)
  $$
  is a \CrefAndHyperrefIfExist{definition:standard_model_structure_on_the_category_of_simplicial_sets}{fibration} of simplicial sets which is a weak equivalence if either $i$ or $p$ is.

\end{enumerate}
\end{definition}



\begin{proposition}[See {\cite[Section 3 Definition 2.1, Section 3 after Proposition 2.12]{morel_voevodsky_1999}}] \label{proposition:A1_model_structure_on_category_of_simplicial_nisnevich_sheaves}

Let $S$ be a \CrefAndHyperrefIfExist{definition:locally_noetherian_and_noetherian_scheme}{Noetherian} base scheme of finite \CrefAndHyperrefIfExist{definition:dimension_of_a_scheme}{dimension}. The classes of \CrefAndHyperrefIfExist{definition:A1_local_weak_equivalence_fibration_of_simplicial_sheaves_of_sets_on_the_nisnevich_site_over_a_noetherian_finite_dimensional_scheme}{$\bbA^1$-weak equivalences}, \CrefAndHyperrefIfExist{definition:monomorphism_and_epimorphism_in_categories}{monomorphisms}, and \CrefAndHyperrefIfExist{definition:A1_local_weak_equivalence_fibration_of_simplicial_sheaves_of_sets_on_the_nisnevich_site_over_a_noetherian_finite_dimensional_scheme}{$\bbA^1$-fibrations} form a \CrefAndHyperrefIfExist{definition:proper_model_category}{proper} \CrefAndHyperrefIfExist{definition:simplicial_model_structure_on_a_category}{simplicial} \CrefAndHyperrefIfExist{definition:model_category}{model structure} on the category $\simpnissheaves$ \CrefIfExists{definition:simplicial_cosimplicial_object_in_a_category} \CrefIfExists{definition:nisnevich_topology_on_a_noetherian_scheme_of_finite_dimension}. 

Similarly, the classes of $\bbA^1$-weak equivalences, monomorphisms, and $\bbA^1$-fibrations form a proper simplicial model structure on the category $\pointedsimpnissheaves$. 

The inclusion functor $\calH_{s,\bbA^1}(S) \hookrightarrow \calH_{s}(\ShvNisSmS)$ (\Cref{definition:A1_local_weak_equivalence_fibration_of_simplicial_sheaves_of_sets_on_the_nisnevich_site_over_a_noetherian_finite_dimensional_scheme}, \Cref{definition:simplicial_model_category_struture_on_the_category_of_simplicial_sheaves_on_a_small_site_and_simplicial_homotopy_category_of_a_site}) has a \CrefAndHyperrefIfExist{definition:adjoint_functors_between_categories_unit_counit_of_adjoint_functors}{left adjoint} 
$$\hlin{L_{\bbA^1}: \calH_{s}(\ShvNisSmS) \to \calH_{s,\bbA^1}(S),}$$
which we call the \hldef{$\bbA^1$-localization functor of spaces}, which identifies $\calH_{s,\bbA^1}(S)$ with the localization of $\calH_{s}(\ShvNisSmS)$ with respect to $\bbA^1$-weak equivalences. 

Similarly, there is an \hldef{$\bbA^1$-localization functor of pointed spaces} which we also denote by \hl{$L_{\bbA^1}$}.
\end{proposition}

\begin{proof}
    This follows by Theorem \ref{theorem:I_model_category_of_simplicial_sheaves_on_a_small_site_with_interval} and Lemma \ref{lemma:A1_homotopy_theory_is_a_kind_of_interval_homotopy_theory}.
\end{proof}


\begin{corollary} \label{corollary:A1_model_structure_on_simplicial_nisnevich_sheaves_over_a_noetherian_finite_dimensional_scheme_is_an_instance_of_an_I_model_structure}
    Let $S$ be a \CrefAndHyperrefIfExist{definition:locally_noetherian_and_noetherian_scheme}{Noetherian} base scheme of finite \CrefAndHyperrefIfExist{definition:dimension_of_a_scheme}{dimension}.
    Letting $I$ be the \CrefAndHyperrefIfExist{definition:interval_in_a_site_with_enough_points}{interval} given in \Cref{lemma:A1_homotopy_theory_is_a_kind_of_interval_homotopy_theory}, the \CrefAndHyperrefIfExist{theorem:I_model_category_of_simplicial_sheaves_on_a_small_site_with_interval}{$I$-model category on $\Deltaop \Shv(\SmSNis)$} coincides with the \CrefAndHyperrefIfExist{definition:A1_model_structure_on_the_category_of_simp_sheaves_of_sets_on_the_nis_site_over_a_noeth_scheme_of_fin_dim_and_the_unstable_A1_homotopy_category}{$\bbA^1$-model structure on $\Deltaop \Shv(\SmSNis)$}. Moreover, the category \CrefAndHyperrefIfExist{definition:I_local_weak_I_weak_equivalence_I_weak_fibration_for_morphisms_of_simplicial_sheaves_on_a_site_with_interval}{$\calH_{s,I}(\SmSNis)$} is equivalent to the category \CrefAndHyperrefIfExist{definition:A1_local_weak_equivalence_fibration_of_simplicial_sheaves_of_sets_on_the_nisnevich_site_over_a_noetherian_finite_dimensional_scheme}{$\calH_{s,\bbA^1}(S)$} and the \CrefAndHyperrefIfExist{theorem:I_model_category_of_simplicial_sheaves_on_a_small_site_with_interval}{$I$-localization} and \CrefAndHyperrefIfExist{proposition:A1_model_structure_on_category_of_simplicial_nisnevich_sheaves}{$\bbA^1$-localization functors}
    $$L_I: \calH_{s}(\ShvNisSmS) \to \calH_{s,I}(\ShvNisSmS)$$
    $$L_{\bbA^1}: \calH_{s}(\ShvNisSmS) \to \calH_{s,\bbA^1}(S),$$
    are equivalent. Moreover, $\calH_{s,\bbA^1}(S)$ is identified with the homotopy category of $\calH_{s}(\ShvNisSmS)$.
\end{corollary}
\begin{proof}
    \Cref{lemma:A1_homotopy_theory_is_a_kind_of_interval_homotopy_theory} implies that $\calH_{s,I}(\ShvNisSmS)$ and $\calH_{s,{\bbA}^1}(S)$ are eqiuvalent by definition. Moreover, since $L_I$ and $L_{{\bbA}^1}$ are each left adjoint to their corresponding right adjoint inclusion functors, \Cref{theorem:I_model_category_of_simplicial_sheaves_on_a_small_site_with_interval} and \Cref{proposition:A1_model_structure_on_category_of_simplicial_nisnevich_sheaves} imply that $L_I$ and $L_{{\bbA}^1}$ are equivalent. Since $\calH_{s,I}(\ShvNisSmS)$ is identified with the homotopy category of $\calH_s(T)$, $\calH_{s,{\bbA}^1}(S)$ is identified with the homotopy category of $\calH_{s,\bbA^1}(S)$.
\end{proof}


\begin{definition}[Unstable motivic homotopy category over a scheme, See {\cite[Section 3 Definition 2.1, Section 3 after Proposition 2.12]{morel_voevodsky_1999}}] \label{definition:A1_model_structure_on_the_category_of_simp_sheaves_of_sets_on_the_nis_site_over_a_noeth_scheme_of_fin_dim_and_the_unstable_A1_homotopy_category}
    Let $S$ be a \CrefAndHyperrefIfExist{definition:locally_noetherian_and_noetherian_scheme}{Noetherian} base scheme of finite \CrefAndHyperrefIfExist{definition:dimension_of_a_scheme}{dimension}. 
    
    \begin{enumerate}
        \item The \CrefAndHyperrefIfExist{definition:simplicial_model_structure_on_a_category}{simplicial} \CrefAndHyperrefIfExist{definition:model_category}{model} structure on $\simpnissheaves$ discussed in Proposition \ref{proposition:A1_model_structure_on_category_of_simplicial_nisnevich_sheaves} is called the \hldef{$\bbA^1$-model structure on $\simpnissheaves$}. 

        \item The \CrefAndHyperrefIfExist{definition:homotopy_category_of_a_model_category}{associated homotopy category} is called the \hldef{(unstable) $\bbA^1$-homotopy category (of smooth schemes over $S$)} or the \hldef{(unstable) motivic category over $S$} and may be denoted by notations such as \hl{$\calH(S)$} or \hl{$\calH^{\bbA^1}(S)$}.

    Note that $\calH^{\bbA^1}(S)$ is equivalent to the category \CrefAndHyperrefIfExist{definition:A1_local_weak_equivalence_fibration_of_simplicial_sheaves_of_sets_on_the_nisnevich_site_over_a_noetherian_finite_dimensional_scheme}{$\calH_{s,\bbA^1}(S)$} by \Cref{corollary:A1_model_structure_on_simplicial_nisnevich_sheaves_over_a_noetherian_finite_dimensional_scheme_is_an_instance_of_an_I_model_structure}

        \item Similarly, the simplicial model structure on $\pointedsimpnissheaves$ discussed in \Cref{proposition:A1_model_structure_on_category_of_simplicial_nisnevich_sheaves} is called the \hldef{(pointed) $\bbA^1$-model structure on $\pointedsimpnissheaves$}. 

        \item The \CrefAndHyperrefIfExist{definition:homotopy_category_of_a_model_category}{associated homotopy category} is called the \hldef{(unstable) pointed $\bbA^1$-homotopy category (of smooth scehemes over $S$)} or the \hldef{(unstable) motivic homotopy category over $S$} and may be denoted by notations such as \hl{$\calH_\bullet(S)$} \hl{$\calH_\bullet^{\bbA^1}(S)$}, \hldef{$\calH(S)_\bullet$}, or \hldef{$\calH^{\bbA^1}(S)_\bullet$}. 
    \end{enumerate}


    Given two objects $X,Y \in \calH_\bullet(S)$, denote by \hl{$[X,Y]_\bullet$} the (pointed) set $\Hom_{\calH_\bullet(S)}(X,Y)$ of morphisms. Given two objects $X,Y \in \calH(S)$, denote by \hl{$[X,Y]$} the (pointed) set $\Hom_{\calH(S)}(X,Y)$ of morphisms.
    \TODO{Why are the homotopy groups actual sheaves of groups?}
\end{definition}

\begin{lemma} \label{lemma:pointification_forgetful_functorson_the_unstable_motivic_homotopy_categories_over_a_noetherian_finite_dimensional_scheme}
    Let $S$ be a \CrefAndHyperrefIfExist{definition:locally_noetherian_and_noetherian_scheme}{Noetherian} base scheme of finite \CrefAndHyperrefIfExist{definition:dimension_of_a_scheme}{dimension}. 
    The \CrefAndHyperrefIfExist{definition:pointification_of_an_object_in_a_category_with_a_final_object}{pointification and forgetful functors}
    $$(X \mapsto X_+): \simpsheaves \leftrightarrows \pointedsimpnissheaves: ((Y,y) \mapsto Y)$$
    \CrefIfExists{definition:simplicial_cosimplicial_object_in_a_category}\CrefIfExists{definition:pointed_sheaves_of_sets_on_a_site_simplicial_pointed_sheaves_of_sets_on_a_site}
    Both functors preserve \CrefAndHyperrefIfExist{definition:simplicial_weak_equivalences_cofibrations_and_fibrations_for_simplicial_sheaves_on_a_site}{simplicial weak equivalences} and \CrefAndHyperrefIfExist{definition:A1_local_weak_equivalence_fibration_of_simplicial_sheaves_of_sets_on_the_nisnevich_site_over_a_noetherian_finite_dimensional_scheme}{$\bbA^1$-weak equivalences} and thus induce \CrefAndHyperrefIfExist{definition:adjoint_functors_between_categories_unit_counit_of_adjoint_functors}{adjoint pairs} of functors 
    $$(X \mapsto X_+): \calH_s(\SmSNis) \leftrightarrows \calH_s(\SmSNis)_\bullet: ((Y,y) \mapsto Y)$$
    (\Cref{definition:simplicial_model_category_struture_on_the_category_of_simplicial_sheaves_on_a_small_site_and_simplicial_homotopy_category_of_a_site}) and
    $$(X \mapsto X_+): \calH^{\bbA^1}(S) \leftrightarrows \calH^{\bbA^1}(S)_\bullet: ((Y,y) \mapsto Y)$$
    (\Cref{definition:A1_model_structure_on_the_category_of_simp_sheaves_of_sets_on_the_nis_site_over_a_noeth_scheme_of_fin_dim_and_the_unstable_A1_homotopy_category})
\end{lemma}

\begin{proof}
    \TODO{establish that the fucntors preserve weak equivalences}
    The adjunctions follow from that of \Cref{lemma:pointification_and_forgetful_functor_adjunction}.
\end{proof}


\section{Functors and operations on spaces}

See \cite[After Section 2 Lemma 2.30]{morel_voevodsky_1999}.


\begin{definition} \label{definition:wedge_and_smash_products_of_pointed_simplicial_sheaves_of_sets_on_a_site}
    Let $T$ be a \CrefAndHyperrefIfExist{definition:grothendieck_topology_on_a_category_site_covering_sieve_topologically_generating_family}{site}.
    For $(X,x), (Y,y) \in \Deltaop(\Shv(T))_\bullet$\CrefIfExists{definition:simplicial_cosimplicial_object_in_a_category}\CrefIfExists{definition:pointed_object_of_a_category_with_a_final_object} \CrefIfExists{definition:sheaf_on_a_site}, define their \hldef{wedge product} \hl{$(X,x) \vee (Y,y)$} and their \hldef{smash product} \hl{$(X,x) \wedge (Y,y)$} as follows:
    \begin{align*}
        (X,x) \vee (Y,y) &\coloneq \left(X \coprod_{\pt} Y, x = y \right) \\
        (X,x) \wedge (Y,y) &\coloneq \left(X \times Y / \left( (X, x) \coprod (Y,y) \right), x \times y \right).
    \end{align*}
\end{definition}

\TODO{TODO: prove this}


\begin{claim}
    Let $T$ be a \CrefAndHyperrefIfExist{definition:grothendieck_topology_on_a_category_site_covering_sieve_topologically_generating_family}{site} and fix $(Y,y) \in \Deltaop(\Shv(T))$.
    The functor 
    $$\Deltaop(\Shv(T))_\bullet \to \Deltaop(\Shv(T))_\bullet, \quad (X,x) \mapsto (X,x) \wedge (Y,y)$$
    has a \CrefAndHyperrefIfExist{definition:adjoint_functors_between_categories_unit_counit_of_adjoint_functors}{right adjoint}, denoted by
    $$\hlin{(Z,z) \mapsto \underline{\Hom}_\bullet((Y,y), (Z,z)).}$$
\end{claim}

\begin{definition} \label{definition:simplicial_suspension_functor_and_simplicial_loop_functor_on_pointed_simplicial_sheaves_of_sets_on_a_site}
    Let $T$ be a \CrefAndHyperrefIfExist{definition:grothendieck_topology_on_a_category_site_covering_sieve_topologically_generating_family}{site}.
    \TODO{$\Delta^n, \partial \Delta^n$}
    \begin{enumerate}

        \item Let \hl{$S_s^1$} the quotient sheaf $\Delta^1 / \partial \Delta^1$, pointed by the image of $\partial \Delta^1$. It is called the \hldef{simplicial circle (on $T$)}. As per \Cref{convention:objects_of_a_site_as_sheaves_and_constant_simplicial_object_of_a_category}, it may also denote the constant pointed simplicial sheaf induced by $S_s^1$. 

        \item Define the \hldef{simplicial suspension functor}
        $$\hlin{\Sigma_s: \Deltaop(\Shv(T))_\bullet \to \Deltaop(\Shv(T))_\bullet, \quad (X,x) \mapsto S_s^1 \wedge (X,x)}.$$
        \CrefIfExists{definition:wedge_and_smash_products_of_pointed_simplicial_sheaves_of_sets_on_a_site}

        \item Define the \hldef{simplicial loop functor} \hl{$\Omega_s^1(-) := \underline{\Hom}_\bullet(S_s^1, -)$}, which is right adjoint to $\Sigma_s(-)$. 

        \item We write by \hl{$\Sigma^n = \Sigma_s^n$} to be the $n$-time composition of $\Sigma$. We write by \hl{$\Omega^n = \Omega_s^n$} to be the $n$-time composition of $\Omega_s^1$.
    \end{enumerate}

\end{definition}


\begin{definition}[Multiplicative group scheme] \label{definition:multiplicative_group_scheme_over_a_scheme}
Let $S$ be a \CrefAndHyperrefIfExist{definition:scheme}{scheme}.
The \hldef{multiplicative group scheme over $S$}, denoted \hl{$\mathbb{G}_{m,S}$}, is the group scheme over $S$ defined as the open subscheme
$$\mathbb{G}_{m,S} := \mathbb{A}^1_S \setminus \{0\}_S$$
of the \CrefAndHyperrefIfExist{definition:affine_space_of_dimension_n_over_a_scheme}{affine line} $\mathbb{A}^1_S$, equipped with the group law given by multiplication of functions:
$$m: \mathbb{G}_{m,S} \times_S \mathbb{G}_{m,S} \to \mathbb{G}_{m,S}, \quad (x,y) \mapsto xy.$$
The identity section is the morphism
$$e: S \to \mathbb{G}_{m,S}, \quad s \mapsto 1,$$
and the inversion morphism is given by
$$i: \mathbb{G}_{m,S} \to \mathbb{G}_{m,S}, \quad x \mapsto x^{-1}.$$
\end{definition}


\begin{definition} \label{definition:tate_sphere_over_a_noetherian_finite_dimensional_scheme}
    \TODO{figure out other formulations of $T$, e.g. it might be isomorphic to $\bbA^1 / (\bbA^1 - \{0\}) \cong \bbP^1$}
    Let $S$ be a \CrefAndHyperrefIfExist{definition:locally_noetherian_and_noetherian_scheme}{Noetherian scheme} of \CrefAndHyperrefIfExist{definition:dimension_of_a_scheme}{finite dimension}. 

    The \hldef{Tate sphere over $S$}, or \hldef{motivic sphere over $S$}, is the smash product
    $$\hlin{T :=S_s^1 \wedge \bbG_m}$$
    where $S_s^1$ is the \CrefAndHyperrefIfExist{definition:simplicial_suspension_functor_and_simplicial_loop_functor_on_pointed_simplicial_sheaves_of_sets_on_a_site}{simplicial circle} on \CrefAndHyperrefIfExist{definition:nisnevich_topology_on_a_noetherian_scheme_of_finite_dimension}{$(\Sm/S)_{\Nis}$}, and $\bbG_m/S$ is the \CrefAndHyperrefIfExist{definition:multiplicative_group_scheme_over_a_scheme}{multiplicative group} over $S$ regarded as a (simplicial) sheaf on the $(\Sm/S)_{\Nis}$.

    In the homotopy category, $T$ is $\mathbb{A}^1$-weakly equivalent to the pointed projective line $(\mathbb{P}^1, \infty)$. \TODO{verify that this is true and also which homotopy category? Stable or unstable?}
\end{definition}



\subsection{The \texorpdfstring{$\bbA^1$}{A1}-homotopy sheaves of a pointed space}


\begin{definition}[See e.g. {\cite[Before Definition 1.7]{morel_a1at}}, cf. {\cite[After Section 3 Lemma 2.13]{morel_voevodsky_1999}}] \label{definition:A1_homotopy_group_sheaf_of_an_object_of_the_pointed_unstable_A1_homotopy_category_over_a_noetherian_finite_dimensional_scheme}
    Let $S$ be a \CrefAndHyperrefIfExist{definition:locally_noetherian_and_noetherian_scheme}{Noetherian} base scheme of finite \CrefAndHyperrefIfExist{definition:dimension_of_a_scheme}{dimension}. Given an object \CrefAndHyperrefIfExist{definition:A1_model_structure_on_the_category_of_simp_sheaves_of_sets_on_the_nis_site_over_a_noeth_scheme_of_fin_dim_and_the_unstable_A1_homotopy_category}{$X \in \calH_\bullet(S)$}, define the \hldef{$n$th $\bbA^1$-homotopy sheaf} \hl{$\pinAone(X)$} as the Nisnevich sheaf of sets \CrefAndHyperrefIfExist{definition:sheafification_functor_on_a_site}{associated} to the presheaf $U \mapsto [\Sigma^n(U_+), X]_\bullet$ (\Cref{definition:A1_model_structure_on_the_category_of_simp_sheaves_of_sets_on_the_nis_site_over_a_noeth_scheme_of_fin_dim_and_the_unstable_A1_homotopy_category}, \Cref{definition:simplicial_suspension_functor_and_simplicial_loop_functor_on_pointed_simplicial_sheaves_of_sets_on_a_site}, \Cref{definition:pointification_of_an_object_in_a_category_with_a_final_object}). It is a sheaf of groups of $n = 1$ and a sheaf of abelian groups for $n \geq 2$.
\end{definition}

We can also define the sheaf of connected components for an unpointed space $X$:
\begin{definition} \label{definition:sheaf_of_A1_connected_components_of_an_unbased_space}
    Let $S$ be a \CrefAndHyperrefIfExist{definition:locally_noetherian_and_noetherian_scheme}{Noetherian} base scheme of finite \CrefAndHyperrefIfExist{definition:dimension_of_a_scheme}{dimension}. Given an object $X \in \calH(S)$, define \hl{$\pinAone[0](X)$} as the \CrefAndHyperrefIfExist{definition:nisnevich_topology_on_a_noetherian_scheme_of_finite_dimension}{Nisnevich} \CrefAndHyperrefIfExist{definition:sheaf_on_a_site}{sheaf} of sets \CrefAndHyperrefIfExist{definition:sheafification_functor_on_a_site}{associated} to the presheaf \CrefAndHyperrefIfExist{definition:A1_model_structure_on_the_category_of_simp_sheaves_of_sets_on_the_nis_site_over_a_noeth_scheme_of_fin_dim_and_the_unstable_A1_homotopy_category}{$U \mapsto [U, X]$}. It is called the \hldef{sheaf of $\bbA^1$-connected components of $X$}. 
\end{definition}



\begin{lemma} \label{lemma:A1_connected_components_of_pointed_space_equals_that_of_underlying_space}
    Let $S$ be a \CrefAndHyperrefIfExist{definition:locally_noetherian_and_noetherian_scheme}{Noetherian} base scheme of finite \CrefAndHyperrefIfExist{definition:dimension_of_a_scheme}{dimension}. Given a pointed space $X$, the sheaf $\pinAone[0](X)$ in the sense of Definition \ref{definition:A1_homotopy_group_sheaf_of_an_object_of_the_pointed_unstable_A1_homotopy_category_over_a_noetherian_finite_dimensional_scheme} is isomorphic to that of the underlying space of $X$ in the sense of \Cref{definition:sheaf_of_A1_connected_components_of_an_unbased_space}. 
\end{lemma}
\begin{proof}
    It suffices to show that the presheaves 
    $$\SmSNis \to \Sets, \quad U \mapsto [U_+, X]_\bullet$$
    and 
    $$\SmSNis \to \Sets, \quad U \mapsto [U, X]$$
    are naturally isomorphic. This is true by the adjunction bewteen the functor $+$ and the forgetful functor stated in Claim \ref{lemma:pointification_and_forgetful_functor_adjunction}.
\end{proof}

\begin{remark}
    \TODO{The morl voevodsky notion of A1-homotopy group}
    The definitions of the $\bbA^1$-homotopy group of a pointed space as given in \cite{morel_voevodsky_1999} and in \cite[Section 6.3 after Conjecture 6.34]{morel_a1at} are different from that of Definition \ref{definition:A1_homotopy_group_sheaf_of_an_object_of_the_pointed_unstable_A1_homotopy_category_over_a_noetherian_finite_dimensional_scheme}. The former definition uses the theory of resolution functors. The latter ostensibly defines $\pinAone(X)$ as the sheafification of the presheaf $U \mapsto \pi_n(L_{\bbA^1}(X)(U))$ where $\pi_n$ is the $n$th homotopy of a simplicial set. We nevertheless accept, without proof, that all three definitions are equivalent in this current writing for convenience. 
\end{remark}

\begin{definition} \label{definition:geometric_simplex_of_independent_points_in_a_real_vector_space}
Let $V$ be a real vector space of finite dimension.  
A \hldef{$k$-simplex in topology} (or a \hldef{geometric $k$-simplex}) is the convex hull of $k+1$ affinely independent points $v_0, v_1, \dots, v_k \in V$, and is denoted by
$$\hlin{[v_0, v_1, \dots, v_k] := \left\{ \sum_{i=0}^k t_i v_i \ \middle| \ t_i \ge 0, \ \sum_{i=0}^k t_i = 1 \right\}.}$$


It is also standard to talk of the \hldef{standard topological $n$-simplex} ---  the topological space \hl{$|\Delta^n|$} defined as the subset of Euclidean space $\mathbb{R}^{n+1}$ given by
$$ |\Delta^n| = \Big\{ (t_0, t_1, \ldots, t_n) \in \mathbb{R}^{n+1} : \sum_{i=0}^n t_i = 1, \text{ and } t_i \geq 0 \text{ for all } i \Big\} $$
equipped with the induced topology from the usual Euclidean topology on $\mathbb{R}^{n+1}$.

\TODO{comment on how $|\Delta^n|$ makes sense via a geometric realization}
% Eqiuvalently, $|\Delta^n|$ 

\end{definition}
\begin{definition} \label{definition:standard_topological_geometric_simplicial_topological_space}
    The \hldef{standard topological/geometric simplicial (topological) space} refers to the \CrefAndHyperrefIfExist{definition:simplicial_cosimplicial_object_in_a_category}{simplicial} \CrefAndHyperrefIfExist{definition:topological_space}{topological space}, denoted by notations such as \hl{$\Delta^\bullet$} or \hl{$|\Delta^\bullet|$} whose $n$-simplices, for each $n \geq 0$, is the \CrefAndHyperrefIfExist{definition:geometric_simplex_of_independent_points_in_a_real_vector_space}{standard topological $n$-simplex $|\Delta^n|$} and whose face and degeneracy maps are given in the ``obvious'' way \TODO{}.
\end{definition}

\begin{definition} \label{definition:standard_simplicial_scheme_over_a_scheme}
    Let $S$ be a \CrefAndHyperrefIfExist{definition:scheme}{scheme}. By the \hldef{``standard'' simplicial scheme over $S$}, we will mean the \CrefAndHyperrefIfExist{definition:simplicial_cosimplicial_object_in_a_category}{simplicial} scheme \hl{$\Delta_S^\bullet$} whose \CrefAndHyperrefIfExist{definition:simplex_of_a_simplicial_object_in_a_category_and_face_and_degeneracy_maps}{$n$-simplices}, for each $n \geq 0$, is the closed subschemes \hl{$\Delta_S^n \subset \bbA_S^{n+1}$} given by the equation $\sum_{i=0}^n x_i = 1$ and whose face and degeneracy maps are given in the ``obvious'' way \TODO{}
    \TODO{this might actually be a cosimplicial scheme}

    We will call $\Delta_S^\bullet$ the ``standard'' simplicial scheme over $S$ becaues it is analogous to the \CrefAndHyperrefIfExist{definition:standard_topological_geometric_simplicial_topological_space}{standard simplicial topological space $|\Delta^\bullet|$}.
\end{definition}

\begin{definition} \label{definition:algebraic_singular_simplicial_set_of_morphisms_from_a_smooth_scheme_to_a_nisnevich_sheaf_of_sets_over_a_noetherian_scheme_of_finite_dimension}
    Let $S$ be a \CrefAndHyperrefIfExist{definition:locally_noetherian_and_noetherian_scheme}{noetherian} \CrefAndHyperrefIfExist{definition:scheme}{scheme} of finite \CrefAndHyperrefIfExist{definition:dimension_of_a_scheme}{dimension}. Let $X \in \ShvNisSmS$ be a \CrefAndHyperrefIfExist{definition:sheaf_on_a_site}{sheaf of sets} on the \CrefAndHyperrefIfExist{definition:nisnevich_topology_on_a_noetherian_scheme_of_finite_dimension}{Nisnevich site over $S$}, and let $U/S$ be a \CrefAndHyperrefIfExist{definition:smooth_morphism_of_schemes}{smooth scheme}.

    \TODO{figure out if $\Delta_S^n$ is a simplicial scheme or a cosimplicial scheme.}
    Let $\operatorname{Sing}_n(X)(U) = \Hom_{\ShvNisSmS}(U \times \Delta_S^n, X)$ where $\Delta_S^n$ is the sheaf represented by the scheme \CrefAndHyperrefIfExist{definition:standard_simplicial_scheme_over_a_scheme}{$\Delta_S^n$}. The $\operatorname{Sing}_n(X)(U)$ form a simplicial set \hl{$\operatorname{Sing}_\bullet(X)(U)$}. We may refer to this as the \hldef{algebraic singular simplicial set (of morphisms from $U$ to $X$)}. 
\end{definition}

\begin{definition} \label{definition:kan_complex}
    A \hldef{Kan complex} or a \hldef{Kan simplicial set} is a \CrefAndHyperrefIfExist{definition:simplicial_cosimplicial_object_in_a_category}{simplicial set} $K$ satisfying the \CrefAndHyperrefIfExist{definition:kan_extension_condition_for_a_horn_for_a_simplicial_set}{Kan extension condition} for the horns $\Lambda_i^n \subset \Delta^n$ for all $n$ and all $0 \leq i < n$. More explicitly, for any $0 \leq i \leq n$, any map $f_0: \Lambda_i^n \to K$ \CrefIfExists{definition:horn_of_the_nth_standard_simplex} admits an extension $f: \Delta^n \to K$ \CrefIfExists{definition:standard_n_simplex}.

    Kan-complexes are equivalent to \CrefAndHyperrefIfExist{definition:infty_groupoid_infty_0_category}{$\infty$-groupoids} (\Cref{proposition:infty_groupoids_are_equivalent_to_kan_complexes}).

    % A \hldef{quasi-category}, or synonymously an \hldef{$\infty$-category}, depending on the context, is a \CrefAndHyperrefIfExist{definition:simplicial_cosimplicial_object_in_a_category}{simplicial set} $K$ satisfying the \CrefAndHyperrefIfExist{definition:kan_extension_condition_for_a_horn_for_a_simplicial_set}{Kan extension condition} for the horns $\Lambda_i^n \subset \Delta^n$ for all $n$ and all $0 < i < n$. More explicitly, for any $0 < i < n$, any map $f_0: \Lambda_i^n \to K$ \CrefIfExists{definition:horn_of_the_nth_standard_simplex} admits an extension $f: \Delta^n \to K$ \CrefIfExists{definition:standard_n_simplex}.
\end{definition}
\begin{definition}[Homotopy Groups of a Kan Simplicial Set] \label{definition:homotopy_groups_of_a_kan_simplicial_set}
Let $X$ be a \CrefAndHyperrefIfExist{definition:kan_complex}{Kan simplicial set}, and fix a \CrefAndHyperrefIfExist{definition:basepoint_of_a_simplicial_set}{basepoint} $x \in X_0$.

For each integer $n \geq 1$, define the \hldef{$n$-th homotopy group of $X$ at $x$}, denoted \hl{$\pi_n(X,x)$}, as the set of homotopy classes of maps
\TODO{homotopy, boundary of $\Delta^n$} 
$$ f: \partial \Delta^{n+1} \to X $$
that restrict to the constant map at $x$ on the basepoint simplex, modulo homotopies relative to the boundary.

Equivalently, $\pi_n(X,x)$ can be described as the set of equivalence classes of \CrefAndHyperrefIfExist{definition:simplicial_cosimplicial_object_in_a_category}{$n$-simplices} whose faces are degenerate at $x$, with composition induced by the combinatorial structure of simplices.

These sets carry natural group structures for $n \geq 1$, with $\pi_1(X,x)$ being the fundamental group and $\pi_n(X,x)$ for $n \geq 2$ being abelian groups.
\end{definition}

\begin{definition} \label{definition:homotopy_groups_of_maps_from_a_smooth_scheme_to_a_sheaf_of_sets_on_the_nisnevich_site_with_respect_to_a_base_point_over_a_noetherian_scheme_of_finite_dimension}
    Let $S$ be a \CrefAndHyperrefIfExist{definition:locally_noetherian_and_noetherian_scheme}{noetherian} \CrefAndHyperrefIfExist{definition:scheme}{scheme} of finite \CrefAndHyperrefIfExist{definition:dimension_of_a_scheme}{dimension}. Let $X \in \ShvNisSmS$ be a sheaf of sets on the Nisnevich site, let $U/S$ be a \CrefAndHyperrefIfExist{definition:smooth_morphism_of_schemes}{smooth scheme}, and let $x \in \Hom(U,X)$. Define \hldef{homotopy groups} \hl{$\pi_{i,U}^{\bbA^1}(X,x)$} to be the \CrefAndHyperrefIfExist{definition:homotopy_groups_of_a_kan_simplicial_set}{homotopy groups} of the \CrefAndHyperrefIfExist{definition:kan_complex}{Kan simplicial set} $\operatorname{Sing}_\bullet(Ex^\infty(X))(U)$ with respect to the base point $x$.
    \TODO{$Ex^\infty$}
\end{definition}

\begin{definition}[{\cite[Definition 3.4]{voevodsky_A1}}] \label{definition:A1_weak_equivalence_of_simplicial_sheaves_of_sets_on_the_nisnevich_topology_over_a_noetherian_finite_dimensional_scheme}
    Let $S$ be a \CrefAndHyperrefIfExist{definition:locally_noetherian_and_noetherian_scheme}{noetherian} \CrefAndHyperrefIfExist{definition:scheme}{scheme} of finite \CrefAndHyperrefIfExist{definition:dimension_of_a_scheme}{dimension}. A morphism $f: X \to Y$ in the category $\Shv(\SmSNis)$ of \CrefAndHyperrefIfExist{definition:sheaf_on_a_site}{sheaves} of sets on \CrefAndHyperrefIfExist{definition:nisnevich_topology_on_a_noetherian_scheme_of_finite_dimension}{$\SmSNis$} is called an \hldef{$\bbA^1$-weak equivalence} or a \hldef{weak equivalence} if for any smooth scheme $U/S$, any $x \in \operatorname{Hom}_{\Deltaop(\SmSNis)}(U,X)$, and any $i \geq 0$, the corresponding map
    $$\pi_{i,U}^{\bbA^1}(X,x) \to \pi_{i,U}^{\bbA}(Y,f(x))$$
    (\Cref{definition:homotopy_groups_of_maps_from_a_smooth_scheme_to_a_sheaf_of_sets_on_the_nisnevich_site_with_respect_to_a_base_point_over_a_noetherian_scheme_of_finite_dimension}) is a bijection.     
\end{definition}



The unstable $\bbA^1$-homotopy category can alternatively be obtained as a homotopy category not of simplicial sheaves, but of ordinary sheaves on $\SmSNis$. 

\begin{definition}[Proper Model Category] \label{definition:proper_model_category}
Let $\mathcal{M}$ be a \CrefAndHyperrefIfExist{definition:model_category}{model category} with classes of weak equivalences $\mathcal{W}$, cofibrations $\mathcal{C}of$, and fibrations $\mathcal{F}ib$. 
\TODO{pushout, pullback}
\begin{enumerate}
    \item Then $\mathcal{M}$ is called \hldef{left proper} if weak equivalences are preserved under pushouts along cofibrations, i.e., if for every pushout diagram
    $$
    \begin{array}{ccc}
    A & \xrightarrow{i} & B \\
    \downarrow^{f \in \mathcal{W}} && \downarrow \\
    C & \xrightarrow{} & D
    \end{array}
    $$
    with $i \in \mathcal{C}of$, the induced map $C \to D$ lies in $\mathcal{W}$. 
    \item Dually, $\mathcal{M}$ is called \hldef{right proper} if weak equivalences are preserved under pullbacks along fibrations, i.e., if for every pullback diagram
    $$
    \begin{array}{ccc}
    P & \xrightarrow{} & X \\
    \downarrow^{g} && \downarrow^{p \in \mathcal{F}ib} \\
    Y & \xrightarrow{h \in \mathcal{W}} & Z
    \end{array}
    $$
    the induced map $P \to Y$ is in $\mathcal{W}$.

    \item A model category $\mathcal{M}$ is \hldef{proper} if it is both left proper and right proper.
\end{enumerate}
\end{definition}

\begin{theorem}[{\cite[Theorems 3.6, 3.7]{morel_voevodsky_1999}}] \label{theorem:homotopy_category_of_ordinary_sheaves_on_nisnevich_site_is_equivalent_to_unstable_homotopy_category}
     Let $S$ be a \CrefAndHyperrefIfExist{definition:locally_noetherian_and_noetherian_scheme}{Noetherian} base scheme of finite \CrefAndHyperrefIfExist{definition:dimension_of_a_scheme}{dimension}. The following specifies a \CrefAndHyperrefIfExist{definition:proper_model_category}{proper} \CrefAndHyperrefIfExist{definition:homotopy_category_of_a_model_category}{model category structure} on the category $\Shv(\SmSNis)$ of \CrefAndHyperrefIfExist{definition:sheaf_on_a_site}{sheaves of sets} on \CrefAndHyperrefIfExist{definition:nisnevich_topology_on_a_noetherian_scheme_of_finite_dimension}{$\SmSNis$}:
     \begin{enumerate}
        \item The weak equivalences are the \CrefAndHyperref{definition:A1_weak_equivalence_of_simplicial_sheaves_of_sets_on_the_nisnevich_topology_over_a_noetherian_finite_dimensional_scheme}{$\bbA^1$-weak equivalence}
        \item The cofibrations are the \CrefAndHyperrefIfExist{definition:monomorphism_and_epimorphism_in_categories}{monomorphisms}.
        \item The fibrations are the class of morphisms having the right lifting property with respect to morphisms that are simultaneously weak equivalences and cofibrations.
     \end{enumerate}
     Moreover, the \CrefAndHyperrefIfExist{definition:homotopy_category_of_a_model_category}{homotopy category} of this model category structure is  \CrefAndHyperrefIfExist{definition:equivalence_of_categories}{equivalent} to the \CrefAndHyperrefIfExist{definition:A1_homotopy_group_sheaf_of_an_object_of_the_pointed_unstable_A1_homotopy_category_over_a_noetherian_finite_dimensional_scheme}{unstable $\bbA^1$-homotopy category $\calH(S)$}.
\end{theorem}

\begin{proposition}[{\cite[Section 3 Proposition 2.14]{morel_voevodsky_1999}}]
    Let $S$ be a \CrefAndHyperrefIfExist{definition:locally_noetherian_and_noetherian_scheme}{Noetherian} base scheme of finite \CrefAndHyperrefIfExist{definition:dimension_of_a_scheme}{dimension}. A morphism $f: X \to Y$ in $\simpnissheaves_\bullet$ is an $\bbA^1$-weak equivalence if and only if the natural morphisms
    $$\pinAone(X) \to \pinAone(Y)$$
    of the \CrefAndHyperrefIfExist{definition:A1_homotopy_group_sheaf_of_an_object_of_the_pointed_unstable_A1_homotopy_category_over_a_noetherian_finite_dimensional_scheme}{$\bbA^1$-homotopy sheaves} are isomorphisms for all $n \geq 0$.
\end{proposition}


\section{\texorpdfstring{$\bbA^1$}{A1}-invariance of Nisnevich sheaves}

\TODO{TODO: discuss the merit of A1-invariance}


\begin{notation} \label{notation:sheaf_cohomology_for_sheaves_of_groups_on_the_nisnevich_site_over_a_noetherian_finite_dimensional_scheme}
    Whenever $S$ is a \CrefAndHyperrefIfExist{definition:locally_noetherian_and_noetherian_scheme}{Noetherian} base scheme of finite \CrefAndHyperrefIfExist{definition:dimension_of_a_scheme}{dimension} let \hl{$H^i_{\Nis}$} denote the $i$th sheaf cohomology functor for $\SmSNis$. More explicitly, given a sheaf $G$ of groups on $\SmSNis$ and a $X \in \Sm/S$, we may speak of the \CrefAndHyperrefIfExist{definition:zeroth_and_first_nonabelian_sheaf_cohomology_of_a_sheaf_of_groups_on_a_site}{group $H^0_{\mathrm{Nis}}(X; G)$ and the set $H^1_{\mathrm{Nis}}(X;G)$} as the zeroth and first nonabelian sheaf cohomology sets on the \CrefAndHyperrefIfExist{definition:site_induced_by_a_site_on_an_over_category}{site $(\SmSNis)_{/X}$ over $X$} and if $G$ is further a sheaf of abelian groups, then we may speak of the groups $H^i_{\mathrm{Nis}}(X;G)$ as the \CrefAndHyperrefIfExist{definition:sheaf_cohomology_group_of_a_sheaf_of_modules_over_a_sheaf_of_rings_on_a_site}{abelian sheaf cohomology groups} of $G$ on $(\SmSNis)_{/X}$. Note that $H^0_{\mathrm{Nis}}(X;G)$ and $H^1_{\mathrm{Nis}}(X;G)$ are unambiguous when $G$ is a sheaf of abelian groups by \Cref{proposition:zeroth_and_first_nonabelian_and_abelian_sheaf_cohomologies_of_a_sheaf_of_abelian_groups_on_}
\end{notation}

\begin{definition}[cf. {\cite[Definition 1.7]{morel_a1at}}, {\cite[Definition 3.1]{morel_a1at_2006}}] \label{definition:A1_invariant_strongly_A1_invariant_strictly_A1_invariant_sheaves_of_sets_groups_ab_groups}
    Let $S$ be a base scheme.
\begin{enumerate}
    \item A \CrefAndHyperrefIfExist{definition:presheaf_on_a_category}{presheaf} $P$ of sets on $\Sm/S$ is said to be \hldef{$\bbA^1$-invariant} if for any $X \in \Sm/S$, the map 
    $$P(X) \to P(\bbA^1_S \times_S X)$$
    induced by the projection $\bbA^1_S \times_S X \to X$ is a bijection.

    \item Assume that $S$ is \CrefAndHyperrefIfExist{definition:locally_noetherian_and_noetherian_scheme}{Noetherian} of finite \CrefAndHyperrefIfExist{definition:dimension_of_a_scheme}{dimension} (so that the Nisnevich topology is defined). A \CrefAndHyperrefIfExist{definition:sheaf_on_a_site}{sheaf} $G$ of groups on \CrefAndHyperrefIfExist{definition:nisnevich_topology_on_a_noetherian_scheme_of_finite_dimension}{$\SmSNis$} is said to be \hldef{strongly $\bbA^1$-invariant} if for any $X \in \Sm/S$, the map 
    $$H^i_{\Nis}(X;G) \to H_{\Nis}^i(\bbA^1_S \times_S X;G)$$
    \CrefIfExists{notation:sheaf_cohomology_for_sheaves_of_groups_on_the_nisnevich_site_over_a_noetherian_finite_dimensional_scheme}
    induced by the projection $\bbA^1_S \times_S X \to X$ is a bijection for $i \in \{0,1\}$.

    \item Assume that $S$ is \CrefAndHyperrefIfExist{definition:locally_noetherian_and_noetherian_scheme}{Noetherian} of finite \CrefAndHyperrefIfExist{definition:dimension_of_a_scheme}{dimension}.
    A sheaf $G$ of abelian groups on $\SmSNis$ is said to be \hldef{strictly $\bbA^1$-invariant} if for any $X \in \Sm/S$, the map 
    $$H^i_{\Nis}(X;G) \to H_{\Nis}^i(\bbA^1_S \times_S X;G)$$
    induced by the projection $\bbA^1_S \times_S X \to X$ is a bijection for any $i \geq 0$.
\end{enumerate}
\end{definition}



Here are some facts about these notions of $\bbA^1$-invariance:
\begin{example}
    Let $k$ be a field. 
\begin{enumerate}
    \item \cite[After Theorem 3.2]{morel_a1at_2006} The \CrefAndHyperrefIfExist{definition:constant_sheaf_on_a_site_with_sheafification}{constant sheaf} $\bbZ$ on \CrefAndHyperrefIfExist{definition:nisnevich_topology_on_a_noetherian_scheme_of_finite_dimension}{$\SmkNis$} is \CrefAndHyperrefIfExist{definition:A1_invariant_strongly_A1_invariant_strictly_A1_invariant_sheaves_of_sets_groups_ab_groups}{strictly $\bbA^1$-invariant}. In fact, \CrefAndHyperrefIfExist{definition:sheaf_cohomology_group_of_a_sheaf_of_modules_over_a_sheaf_of_rings_on_a_site}{$H_{\Nis}^i(X;\bbZ) = 0$} for $i > 0$.
    \item \cite[After Theorem 3.2]{morel_a1at_2006} The sheaf on $\SmkNis$ represnted by an abelian variety over $k$ is strictly $\bbA^1$-invariant. In fact, $H_{\Nis}^i(X;\bbZ) = 0$ for $i > 0$.
    \item \cite[After Theorem 3.2]{morel_a1at_2006} The sheaf on $\SmkNis$ represented by the multiplicative group $\bbG_m = \bbA^1 - \{0\}$ is strictly $\bbA^1$-invariant.
\end{enumerate}
\end{example}

\begin{lemma}[cf. {\cite[Remark 1.8]{morel_a1at}}]
    Let $S$ be a \CrefAndHyperrefIfExist{definition:locally_noetherian_and_noetherian_scheme}{Noetherian} base scheme of finite \CrefAndHyperrefIfExist{definition:dimension_of_a_scheme}{dimension}. A sheaf $X$ of sets on $\SmSNis$ is \CrefAndHyperrefIfExist{definition:A1_invariant_strongly_A1_invariant_strictly_A1_invariant_sheaves_of_sets_groups_ab_groups}{$\bbA^1$-invariant} if and only if it is \CrefAndHyperrefIfExist{definition:A1_local_weak_equivalence_fibration_of_simplicial_sheaves_of_sets_on_the_nisnevich_site_over_a_noetherian_finite_dimensional_scheme}{$\bbA^1$-local} \CrefAndHyperrefIfExist{convention:objects_of_a_site_as_sheaves_and_constant_simplicial_object_of_a_category}{as a} \CrefAndHyperrefIfExist{definition:A1_homotopy_theoretic_space_over_a_noetherian_scheme_of_finite_dimension}{space}.
\end{lemma}

By definition, a strictly $\bbA^1$-invariant sheaf of groups is clearly a strongly $\bbA^1$-invariant sheaf. If the sheaf is of abelian groups, then the converse is true:

\begin{theorem}[See {\cite[Theorem 1.16, Theorem 5.46]{morel_a1at}}, {\cite[Theorem 3.2]{morel_a1at_2006}}] \label{theorem:sheaf_on_abelian_groups_is_strongly_A1_invariant_iff_strictly_A1_invariant}
    \TODO{perfect field}
    Let $k$ be a perfect field. A sheaf $G$ of abelian groups on $\SmkNis$ is \CrefAndHyperrefIfExist{definition:A1_invariant_strongly_A1_invariant_strictly_A1_invariant_sheaves_of_sets_groups_ab_groups}{strongly $\bbA^1$-invariant} if and only if it is strictly \CrefAndHyperrefIfExist{definition:A1_invariant_strongly_A1_invariant_strictly_A1_invariant_sheaves_of_sets_groups_ab_groups}{$\bbA^1$-invariant}.
\end{theorem}

\begin{remark}
    Morel's text \cite{morel_a1at} assumes that the base field $k$ is perfect; the only reason for this assumption throughout the text, as explained in \cite[Remark 1.17]{morel_a1at}, is to ensure that Theorem \ref{theorem:sheaf_on_abelian_groups_is_strongly_A1_invariant_iff_strictly_A1_invariant} holds. As such, it may be the case that some statements taken from loc.~cit.~ do not require the perfectness hypothesis.
\end{remark}

Moreover, it is known over perfect fields that $\bbA^1$-homotopy group sheaves are strongly $\bbA^1$-invariant.

\begin{theorem}[{\cite[Theorem 1.9]{morel_a1at}}] \label{theorem:A1_homotopy_groups_are_strongly_A1_invariant}
    Let $k$ be a perfect field, and let $X$ be a pointed space. The sheaf \CrefAndHyperrefIfExist{definition:A1_homotopy_group_sheaf_of_an_object_of_the_pointed_unstable_A1_homotopy_category_over_a_noetherian_finite_dimensional_scheme}{$\pinAone[1](X)$} of groups is \CrefAndHyperrefIfExist{definition:A1_invariant_strongly_A1_invariant_strictly_A1_invariant_sheaves_of_sets_groups_ab_groups}{strongly $\bbA^1$-invariant} and for any $n \geq 2$ the sheaf $\pinAone[n](X)$ is \CrefAndHyperrefIfExist{definition:A1_invariant_strongly_A1_invariant_strictly_A1_invariant_sheaves_of_sets_groups_ab_groups}{strictly $\bbA^1$-invariant}.
\end{theorem}

\begin{remark}
    Ayoub showed that \cite{ayoub_cmcpiA1} $\pinAone[0](X)$ is generally not $\bbA^1$-invariant. On the other hand \cite[Remark 1.13]{morel_a1at}, $\pinAone[0](X)$ is $\bbA^1$-invariant for smooth $k$-schemes of dimension $\leq 1$.
\end{remark}

\begin{proposition}[{\cite[Lemma 1.18]{bachmann_s1isaFm}}]
    Let $k$ be a \CrefAndHyperrefIfExist{definition:field}{field}.  Let $K/k$ be a \CrefAndHyperrefIfExist{definition:separable_field_extension_general}{separable field extension} and write \CrefAndHyperrefIfExist{definition:affine_scheme}{$\eta = \operatorname{Spec} K$}. Note that there is a base change functor \TODO{} $\SmNis[k] \to \SmNis[K]: F \mapsto F|_\eta$. 
    \begin{enumerate}
        \item Let $\{X_\lambda\}$ be a \CrefAndHyperrefIfExist{definition:projective_and_inductive_limits_in_categories}{cofiltered system} of \CrefAndHyperrefIfExist{definition:smooth_at_a_point_for_a_finite_type_scheme_over_a_field}{smooth} \CrefAndHyperrefIfExist{definition:algebraic_variety_over_a_field}{$k$-varieties} such that $\varprojlim_\lambda X_\lambda \in \Sm_\eta$\CrefIfExists{definition:projective_and_inductive_limits_in_categories}. For any sheaf $F$ of sets on $\SmkNis$ we have 
        $$F|_\eta (\varprojlim_\lambda X_\lambda) \cong \varinjlim_\lambda F(X_\lambda).$$

        \item The base change functor
        \begin{enumerate}
            \item preserves \CrefAndHyperrefIfExist{definition:A1_invariant_strongly_A1_invariant_strictly_A1_invariant_sheaves_of_sets_groups_ab_groups}{$\bbA^1$-invariant} sheaves (of sets),

            \item commutes with taking homotopy sheaves \TODO{what does this mean}

            \item preserves \CrefAndHyperrefIfExist{definition:A1_invariant_strongly_A1_invariant_strictly_A1_invariant_sheaves_of_sets_groups_ab_groups}{strongly $\bbA^1$-invariant sheaves} of groups and preserves \CrefAndHyperrefIfExist{definition:A1_invariant_strongly_A1_invariant_strictly_A1_invariant_sheaves_of_sets_groups_ab_groups}{strictly $\bbA^1$-invariant sheaves} of abelian groups
        \end{enumerate}

    \end{enumerate}
\end{proposition}

\TODO{
    My actual goal is something along the following lines: Let $R$ be a Dedekind ring. If $F$ is strictly A1-invariant over $R$, then it is strictly A!-invariant over $R_p$ for all $p$. 
    
I might need a statement for topoi of the following form: 
    Given a Grothendieck topos $\calE$, given a filtered category $I$, a filtered diagram $F_\bullet: I \to \Ab(\calE)$ of abelian sheaves wtih colimit $F = \colim_{i \in I} F_i$, we have
    $$H^q(\calE, F) \cong \colim_{i \in I} H^q(\calE, F_i)$$
    SGA 4 Expose VII, Theorem 5.7 does this for the etale site.
}

\TODO{
    \section*{Proof of Preservation of Strong $\mathbb{A}^1$-invariance under Localization}

    \begin{lemma}
    Let $R$ be a Dedekind ring and let $A$ be a strongly $\mathbb{A}^1$-invariant Nisnevich sheaf on $\mathrm{Sm}/R$. Let $\mathfrak{p} \subset R$ be a prime ideal and let $S = \mathrm{Spec}(R_{\mathfrak{p}})$. Then the restriction $A|_S$ is a strongly $\mathbb{A}^1$-invariant sheaf on $\mathrm{Sm}/S$.
    \end{lemma}

    \begin{proof}
    The localization $S = \mathrm{Spec}(R_{\mathfrak{p}})$ can be written as the cofiltered limit of open affine neighborhoods of $\mathfrak{p}$ in $X = \mathrm{Spec}(R)$:
    \[
    S \cong \varprojlim_{U \ni \mathfrak{p}} U.
    \]
    This limit is a limit of schemes with affine transition maps (open immersions). A standard result in the theory of topoi (SGA 4, Exposé VII, Prop. 5.7) states that the topos of sheaves on such a limit is equivalent to the limit of the topoi. Specifically, for Nisnevich cohomology, we have the continuity property:
    \[
    H^i_{\mathrm{Nis}}(S, \mathcal{F}|_S) \cong \varinjlim_{U \ni \mathfrak{p}} H^i_{\mathrm{Nis}}(U, \mathcal{F}|_U)
    \]
    for any sheaf $\mathcal{F}$ on $\mathrm{Sm}/R$.

    To prove that $A|_S$ is strongly $\mathbb{A}^1$-invariant, we must show that for any smooth scheme $Y \in \mathrm{Sm}/S$ and $i \in \{0, 1\}$, the projection $p: Y \times \mathbb{A}^1_S \to Y$ induces an isomorphism:
    \[
    p^*: H^i_{\mathrm{Nis}}(Y, A|_Y) \xrightarrow{\sim} H^i_{\mathrm{Nis}}(Y \times \mathbb{A}^1_S, A|_{Y \times \mathbb{A}^1_S}).
    \]
    Since $Y$ is of finite presentation over $S$, it descends to a smooth scheme $Y_U$ over some open neighborhood $U$ of $\mathfrak{p}$. By shrinking $U$ if necessary, we may assume $Y = Y_U \times_U S$.

    We compute the cohomology of $Y$ as the limit of the cohomology of its models $Y_V$ over open sets $V \subseteq U$:
    \[
    H^i_{\mathrm{Nis}}(Y, A|_Y) \cong \varinjlim_{V \subseteq U} H^i_{\mathrm{Nis}}(Y_V, A|_{Y_V}).
    \]
    Since $A$ is strongly $\mathbb{A}^1$-invariant on $\mathrm{Sm}/R$, its restriction to any open $V \subset \mathrm{Spec}(R)$ is also strongly $\mathbb{A}^1$-invariant. Therefore, for every $V$, we have an isomorphism:
    \[
    H^i_{\mathrm{Nis}}(Y_V, A|_{Y_V}) \cong H^i_{\mathrm{Nis}}(Y_V \times \mathbb{A}^1_V, A|_{Y_V \times \mathbb{A}^1_V}).
    \]
    Taking the filtered colimit on both sides, and using the continuity of cohomology for $Y \times \mathbb{A}^1_S \cong \varprojlim (Y_V \times \mathbb{A}^1_V)$, we obtain:
    \[
    \varinjlim_{V \subseteq U} H^i_{\mathrm{Nis}}(Y_V, A|_{Y_V}) \cong \varinjlim_{V \subseteq U} H^i_{\mathrm{Nis}}(Y_V \times \mathbb{A}^1_V, A|_{Y_V}) \cong H^i_{\mathrm{Nis}}(Y \times \mathbb{A}^1_S, A|_Y).
    \]
    Thus, the isomorphism holds for $A|_S$, proving it is strongly $\mathbb{A}^1$-invariant.
    \end{proof}

}

\TODO{ And then an argument for strict $\bbA^1$-invariance is probably needed separately, because for strict invariance, we are using nonabelian sheaf cohomology:

\section*{Preservation for Sheaves of Groups (Non-Abelian)}

\begin{lemma}
Let $G$ be a strongly $\mathbb{A}^1$-invariant sheaf of groups on $\mathrm{Sm}/R$. Let $S = \mathrm{Spec}(R_{\mathfrak{p}}) \cong \varprojlim U$. Then the restriction $G|_S$ is strongly $\mathbb{A}^1$-invariant.
\end{lemma}

\begin{proof}
Strong $\mathbb{A}^1$-invariance is defined by the bijectivity of the maps $H^i_{\mathrm{Nis}}(Y, G) \to H^i_{\mathrm{Nis}}(Y \times \mathbb{A}^1, G)$ for $i=0, 1$.

\textbf{Case $i=0$ (Sections):}
By the definition of the inverse image sheaf on a pro-object, the sections on $Y$ are defined as the colimit of sections on the models $Y_U$:
\[
H^0(Y, G|_S) = G(Y) \cong \varinjlim_{U} G(Y_U) = \varinjlim_{U} H^0(Y_U, G).
\]
The colimit of bijections is a bijection, so the property holds for $i=0$.

\textbf{Case $i=1$ (Torsors):}
The set $H^1_{\mathrm{Nis}}(S, G)$ classifies $G$-torsors. Since $G$-torsors are schemes of finite presentation over $S$, Grothendieck's limit theorems (EGA IV$_3$, 8.8.2) apply:
\begin{enumerate}
    \item Every $G$-torsor $P \to Y$ over the limit descends to a $G$-torsor $P_U \to Y_U$ for some $U$ (essential surjectivity).
    \item Any isomorphism between torsors over $Y$ descends to an isomorphism over some $Y_V$ (fullness).
\end{enumerate}
This implies that the functor $H^1_{\mathrm{Nis}}(-, G)$ commutes with filtered limits of schemes:
\[
H^1_{\mathrm{Nis}}(Y, G|_S) \cong \varinjlim_{U} H^1_{\mathrm{Nis}}(Y_U, G).
\]
Since the map $H^1(Y_U) \to H^1(Y_U \times \mathbb{A}^1)$ is a bijection for all $U$ (by the strong invariance of $G$ on $\mathrm{Sm}/R$), the limit map is also a bijection.
\end{proof}

}

\section{\texorpdfstring{$\bbA^1$}{A1}-derived category}

$\bbA^1$-homotopy theory yields an $\bbA^1$-derived category of complexes of Nisnevich sheaves of abelian groups obtained by inverting all the $\bbA^1$-quasi isomorphisms. Spaces can be considered as objects of this derived category and their homology objects are referred to as $\bbA^1$-homology sheaves. 

\begin{notation}
    Given a \CrefAndHyperrefIfExist{definition:locally_noetherian_and_noetherian_scheme}{Noetherian} base scheme $S$ of finite \CrefAndHyperrefIfExist{definition:dimension_of_a_scheme}{dimension}, let \hl{$\calAb(S)$} denote the abelian category of abelian groups for the Nisnevich topology on $\Sm/S$. Let \hl{$C_*(\calAb(S))$} denote the category of chain complexes of objects in $\calAb(S)$. Let \hl{$D(\calAb(S))$} denote the derived category of $\calAb(S)$ in the standard homological algebra sense, i.e. it is the class obtained from $C_*(\calAb(S))$ by inverting the class of quasi-isomorphisms between chain complexes.

    In general, given a site $T$ and a simplicial sheaf $X \in \Deltaop\Shv(T)$, denote by \hl{$\bbZ(X)$} the free abelian sheaf generated by $X$ . More precisely, $\bbZ(X) \in \Deltaop\Shv(T)$ is the simplicial sheaf of abelian groups associated to the presheaf $U \mapsto \bbZ[X(U)]$ where $\bbZ[X(U)]$ is the free abelian simplicial group generated by the simplicial set $X(U)$ (Definition \ref{definition:free_simplicial_abelian_group_of_a_simplicial_set}). 
\end{notation}

\TODO{TODO: figure out what the chain complex $C_*(X)$ associated to a space means}

In particular, given a space $X \in \simpnissheaves$, we can speak of the Nisnevich simplicial sheaf $\bbZ(X)$.

\begin{definition}[cf. {\cite[Definition 6.17]{morel_voevodsky_1999}}]
    Let $S$ be a \CrefAndHyperrefIfExist{definition:locally_noetherian_and_noetherian_scheme}{Noetherian} base scheme of finite \CrefAndHyperrefIfExist{definition:dimension_of_a_scheme}{dimension}.
    \begin{enumerate}
        \item A chain complex $D_* \in C_*(\Ab(S))$ is called \hldef{$\bbA^1$-local} if and only if for any $C_* \in C_*(\Ab(S))$, the projection 
        $$C_* \otimes \bbZ(\bbA^1) \to C_*$$
        induces a bijection
        $$\Hom_{D(\Ab(S))}(C_*, D_*) \to \Hom_{D(\Ab(S))}(C_* \otimes \bbZ(\bbA^1), D_*).$$
        Denote by \hl{$\DAoneloc(\Ab(S))$} the full subcategory of $D(\Ab(S))$ consisting of $\bbA^1$-local complexes.

        \item A morphism $f: C_* \to D_*$ in $C_*(\Ab(S))$ is called an \hldef{$\bbA^1$-quasi isomorphism} if and only if for every $\bbA^1$-local chain complex $E_* \in C_*(\Ab(S))$, the morphism
        $$\Hom_{D(\Ab(S))}(D_*, E_*) \to \Hom_{D(\Ab(S))}(C_*, E_*)$$
        is bijective. Denote by \hl{$\AoneQis$} the class of $\bbA^1$-quasi isomorphisms.

        \item The \hldef{$\bbA^1$-derived category} \hl{$\DAone(\Ab(S))$} is the category obtained from $C_*(\Ab(S))$ by inverting $\AoneQis$.
    \end{enumerate}
\end{definition}

We have the following analogue of Proposition \ref{proposition:A1_model_structure_on_category_of_simplicial_nisnevich_sheaves}

\TODO{TODO: talk about A1-localization of chain complexes}
\begin{theorem}[cf. {\cite[Lemma 6.18, Corollary 6.19]{morel_a1at}}]
    Let $k$ be a field. There exists a functor \hl{$\LAoneab: C_*(\Ab(k)) \to C_*(\Ab(k))$} called the \hldef{(abelian) $\bbA^1$-localization functor} together with a natural transformation
    $$\hlin{\theta: \id \to \LAoneab}$$
    such that for any chain complex $C_* \in C_*(\Ab(k))$, the morphism $\theta_{C_*}: C_* \to \LAoneab(C_*)$ is an $\bbA^1$-quasi isomorphism whose target is an $\bbA^1$-local fibrant chain complex.

    Furthermore, $\LAoneab$ induces a functor
    $$D(\Ab(k)) \to \DAoneloc(\Ab(k))$$
    which is left adjoint to the inclusion $\DAoneloc(\Ab(k)) \subset D(\Ab(k))$, and which induces an equivalence
    $$\DAone(\Ab(k)) \to \DAoneloc(\Ab(k))$$
    of categories.
\end{theorem}

\begin{definition}
    Let $k$ be a field. Denote by \hl{$\CstarAone$} the functor $\LAoneab \circ C_*: \calH(k) \to \DAone(\Ab(k))$. Given a space $X \in \calH(k)$, we call $\CstarAone(X)$ the \hldef{$\bbA^1$-chain complex of $X$}.
    \TODO{TODO: notate $C_*$ }
    Given a space $X \in \calH(k)$, denote by \hl{$\HnAone(X)$} the \hldef{$n$th $\bbA^1$-homology sheaf of $X$}, defined to be the $n$th homology object of the $\CstarAone(X)$. Denote by \hl{$\tildeHnAone(X)$} the \hldef{$n$th $\bbA^1$-reduced homology sheaf of $X$}, defined by
    $$\tildeHnAone(X) = \ker\left(\HnAone(X) \to \HnAone(\pt) \right).$$
\end{definition}

\begin{remark}
    Just as for classical topology, we have that $\HnAone(\pt) = 0$ for $n \neq 0$ and $\bbZ$ for $n = 0$ and hence we have an isomorphism 
    $$\HnAone[*](X) = \bbZ \oplus \tildeHnAone[*](X)$$
    of graded abelian sheaves where $\bbZ$ here is concentrated in degree $1$.
\end{remark}





\section{\texorpdfstring{$\bbA^1$}{A1}-Brouwer degrees}

\subsection{The classical Brouwer degree}
Let \hl{$\Sn$} be the (topological) $n$-sphere. Recall the following classical theorem in topology:
\begin{theorem}
    Let $n > 0$ be an integer. Let $\Sn$ be the $n$-sphere and write $\pi_i$ for the $i$th fundamental group of a (connected) space. For an integer $i > 0$, 
    \begin{align*}
        \pi_i(\Sn) = \begin{cases} 0 &\text{if } i < n \\ \bbZ &\text{if } i = n. \end{cases}
    \end{align*}
\end{theorem}
A consequence of this theorem is that one can speak of the \hldef{Brouwer degree of a continuous map $f: \Sn \to \Sn$} --- it is defined as the unique integer $d$ such that the induced group homomorphism $\pi_n(f) : \pi_n(\Sn) \to \pi_n(\Sn)$, which is identifiable with a group homomorphism $\bbZ \to \bbZ$ is given by $a \mapsto da$. 


Morel's $\bbA^1$-Brouwer degree generalizes this idea in such a way that $(\bbP^1)^{\wedge n}$ (over a field $k$) takes the place of $\Sn$. These ``enriched'' degrees take values in the Grothendieck-Witt ring $\GWring(k)$ \TODO{TODO: ref}, which can be regarded as the degree $0$ part of the Milnor-Witt $K$-theory ring $\MWKtheory$ (Lemma \ref{lemma:grothendieck_witt_ring_is_degree_zero_part_of_milnor_witt_k_theory}).

\subsection{Milnor-Witt \texorpdfstring{$K$}{K}-theory and the Grothendieck-Witt ring of a field}

\begin{definition}
 [See {\cite[Definition 1.21]{morel_voevodsky_1999}}]
Let $F$ be a (commutative) field. The \hldef{Milnor-Witt $K$-theory of $F$} is the graded associated ring \hl{$\MWKtheory(F)$} generated by the degree $1$ symbols \hl{$[u]$} for each $u \in F^\times$ along with one degree $-1$ symbol $\eta$ subject to the following relations:
\begin{enumerate}
    \item (Steinberg relation) $[a] \cdot [1-a] = 0$ for each $a \in F^\times - \{1\}$.
    \item $[ab] = [a] + [b] + \eta \cdot [a] \cdot [b]$ for each $(a,b) \in (F^\times)^2$. 
    \item $[u] \cdot \eta = \eta \cdot [u]$ for each $u \in F^\times$. 
    \item $\eta \cdot h = 0$ where \hldef{$h \coloneq \eta \cdot [-1] + 2$}. 
\end{enumerate}
\end{definition}

\begin{definition}
 Let $F$ be a (commutative) field. The \hldef{Milnor $K$-theory of $F$} is the graded associated ring \hl{$\MKtheory(F)$} generated by the degree $1$ symbols \hl{$[u]$} for each $u \in F^\times$ subject to the Steinberg relations: $[a] \cdot [1-a] = 0$ for each $a \in F^\times - \{1\}$.
\end{definition}

\begin{remark}
Let $F$ be a field. Note that $\MKtheory(F)$ is the quotient of $\MWKtheory(F)$ by $\eta$.
\end{remark}


\begin{definition}
 Let $F$ be a (commutative) field. The \hldef{Grothendieck-Witt ring of $F$} is the commutative ring \hl{$\GWring(F)$} whose underlying group is the group completion of the commutative monoid of isomorphism classes of non-degenerate symmetric bilinear forms under the direct sum, and whose multiplication structure is determined by tensor products of symmetric bilinear forms.

Equivalently (see e.g. \cite[Chapter IV Lemma 1.1]{husemoller_milnor}, cf. \cite[Lemma 3.9]{morel_a1at}), $\GWring(F)$ is generated by elements $\langle u \rangle$ for $u \in F^\times$ subject to the following relations:
\begin{enumerate}
    \item $\langle u(v^2) \rangle = \langle u \rangle$
    \item $\langle u \rangle + \langle -u \rangle  =  \langle 1 \rangle + \langle -1 \rangle$
    \item $\langle u \rangle + \langle v \rangle = \langle u + v  \rangle + \langle (u+v)uv \rangle$ if $u+v \neq 0$.
\end{enumerate}
\end{definition}

\begin{lemma}[{\cite[Lemma 3.10, cf. Definition 1.21]{morel_a1at}}] \label{lemma:grothendieck_witt_ring_is_degree_zero_part_of_milnor_witt_k_theory}
    Let $F$ be a field. There is a ring isomorphism
    $$\GWring(F) \to \MWKtheory[0](F)$$
    given by $\langle u \rangle \mapsto \eta[u] + 1.$
\end{lemma}

The following relates the Milnor-Witt $K$-theory of a field $F$ with that of its function field $F(T)$:

\begin{theorem}(See e.g. \cite[Theorem 3.24]{morel_a1at})
For any field $F$ the following is a (split) short exact sequence of $\MWKtheory(F)$-modules:
$$  0 \rightarrow \MWKtheory[n](F) \rightarrow \MWKtheory[n](F(T)) \stackrel{\Sigma \partial_{(P)}^{P}}{\longrightarrow} \bigoplus_{P} \MWKtheory[n-1](F[T]/P) \rightarrow 0  $$ 
\end{theorem}


\subsection{The \texorpdfstring{$\mathbb{Z}$}{Z}-graded sheaf of unramified Milnor-Witt \texorpdfstring{$K$}{K}-theory of a field}

Let $k$ be a perfect field. There is a sheaf \hl{$\MWKtheorysheaf$} on $\SmkNis$ referred to as the \hldef{$\bbZ$-graded sheaf of unramified MilnorWitt $K$-theory}. See e.g. \cite[The discussions surrounding Lemma 3.32, Theorem 2.46]{morel_a1at} for an establishment of this sheaf. For any field extension $F$ of $k$ of finite transcendence degree, $\MWKtheorysheaf(F) = \MWKtheory(F)$. Moreover, the following is a consequence of \cite[Theorem 3.22]{morel_a1at}: 

\begin{theorem}
Let $k$ be a perfect field. Let $F$ be a field extension of $k$ of finite transcendence degree. Let $v$ be a discrete valuation on $F$. The ring $\MWKtheorysheaf(\calO_v)$ is the subring of $\MWKtheory(F)$ generated by the elements $\eta$ and $[u] \in \MWKtheory[1](F)$ for $u \in \calO_v^\times$ where $\calO_v$ is the ring of integers of $F$ with respect to $v$.
\end{theorem}

\subsection{}

Analogously to classical topology, we have Hurewicz theorems. To state them, we first need the definition of $\bbA^1$-connectedness:

\begin{definition}
    Let $S$ be a \CrefAndHyperrefIfExist{definition:locally_noetherian_and_noetherian_scheme}{Noetherian} base scheme of finite \CrefAndHyperrefIfExist{definition:dimension_of_a_scheme}{dimension}.
    A pointed space $X \in \simpnissheaves_\bullet$ is said to be \hldef{$n$-$\bbA^1$-connected or $\bbA^1$-$n$-connected}  for $n \geq 0$ if the sheaves $\pinAone[i](X)$ are isomorphic to $\pt$ for $i \leq n$. It is said to be \hldef{$\bbA^1$-connected} if it is $0$-$\bbA^1$-connected.
    An unpointed space $X \in \simpnissheaves$ is said to be $\bbA^1$-connected if $\pinAone[0](X)$ is isomorphic to $\pt$. 
\end{definition}
\begin{lemma}
    Let $S$ be a \CrefAndHyperrefIfExist{definition:locally_noetherian_and_noetherian_scheme}{Noetherian} base scheme of finite \CrefAndHyperrefIfExist{definition:dimension_of_a_scheme}{dimension}. A pointed space $X \in \simpnissheaves_\bullet$ is $\bbA^1$-connected if and only if its underlying space is $\bbA^1$-connected.
\end{lemma}
\begin{proof}
    This follows by Lemma \ref{lemma:A1_connected_components_of_pointed_space_equals_that_of_underlying_space}
\end{proof}


\begin{theorem}[Hurewciz theorems for $\bbA^1$-homotopy theory]
    \TODO{TODO; state the Hurewicz theorems}
    Let $k$ be a perfect field. Let $X \in \calH_\bullet(k)$ be a pointed $\bbA^1$-connected space. 
    \begin{enumerate}
        \item \cite[Theorem 6.35]{morel_a1at} The Hurewicz morphism
        $$\pinAone[1](X) \to \HnAone[1](X)$$
        is the initial morphism from $\pinAone[1](X)$ to a strictly $\bbA^1$-invariant sheaf of groups. More precisely, any morphism
        $$\pinAone[1](X) \to M$$
        to a strictly $\bbA^1$-invariant sheaf $M$ of groups factors uniquely through $\HnAone[1](X)$.

        \item  \cite[Theorem 6.37]{morel_a1at}  Let $n \geq 2$ be an integer and let $X \in \calH_\bullet(k)$ be a pointed $(n-1)$-$\bbA^1$-connected space. For each $i \in \{0,\ldots,n-1\}$, we have
        $$\tildeHnAone[i](X) = 0$$
        and the Hurewicz morphism
        $$\pinAone(X) \to \HnAone(X)$$
        is an isomorphism between strictly $\bbA^1$-invariant sheaves.
    \end{enumerate}
\end{theorem}

\begin{theorem}[{\cite[Theorem 1.22]{morel_a1at}}]
    Let $k$ be a perfect field. For $n \geq 1$, there is a morphism
    $$(\bbG_m)^{\wedge n} \to \MWKtheorysheaf[n], \quad (U_1,\ldots,U_n) \mapsto [U_1,\ldots,U_n]$$
    of sheaves on $\SmkNis$ and this morphism is the universal one to a strongly $\bbA^1$-invariant sheaf of abelian groups. More precisely, any morphism $(\bbG_m)^{\wedge n} \to M$ where $M$ is a strongly $\bbA^1$-invariant sheaf of abelian groups factors uniquely through $\MWKtheorysheaf[n]$.
\end{theorem}

The following is a consequence:

\TODO{TODO: find the relationship between the n-fold wedge of multiplicative group and punctured affine spcae, and n-fold wedge of projective line}
\begin{theorem}[{\cite[Theorem 1.23]{morel_a1at}}]
    Let $k$ be a perfect field. For $n \geq 2$, there is an isomorphism 
    $$\pinAone[n-1](\bbA^n - \{0\}) \cong \pinAone((\bbP^1)^{\wedge n}) \cong \MWKtheorysheaf[n]$$ 
    of sheaves on $\SmkNis$.
\end{theorem}



\begin{theorem}[{\cite[Corollary 1.24]{morel_a1at}}]
    Let $k$ be a perfect field. The canonical morphism 
    $$[\bbA^n - \{0\}, \bbA^n - \{0\}]_\bullet \cong [(\bbP^1)^{\wedge n}, (\bbP^1)^{\wedge n}] \to \MWKtheory[0](k) = \GWring(k)$$
    is an epimorphism for $n = 1$ and an isomorphism for $n \geq 2$.
\end{theorem}



\section{\texorpdfstring{$\bbA^1$}{A1}-homotopy groups of spaces}

\section{Stable \texorpdfstring{$\bbA^1$}{A1}-categories}

\TODO{}
\begin{definition}
The \hldef{unstable $\mathbb{A}^1$-homotopy category}, denoted \hl{$\mathcal{H}(S)$}, is the homotopy category of $\text{Spc}(S)$ with respect to the \hldef{$\mathbb{A}^1$-model structure}. This model structure is the left Bousfield localization of the Nisnevich-local model structure on $\text{Spc}(S)$ with respect to the set of projection maps $\{X \times \mathbb{A}^1 \to X \mid X \in \text{Sm}/S\}$. The pointed homotopy category $\mathcal{H}_{\bullet}(S)$ is constructed similarly from $\text{Spc}_{\bullet}(S)$.
\end{definition}

\begin{definition}
Let $S$ be a \CrefAndHyperrefIfExist{definition:locally_noetherian_and_noetherian_scheme}{Noetherian scheme} of finite \CrefAndHyperrefIfExist{definition:dimension_of_a_scheme}{dimension} and let $T$ be the \CrefAndHyperrefIfExist{definition:tate_sphere_over_a_noetherian_finite_dimensional_scheme}{Tate sphere}. 

\begin{enumerate}
    \item A \hldef{motivic spectrum} (or \hldef{$T$-spectrum}) $E$ consists of a sequence \hl{$(E_n)_{n \geq 0}$} of pointed motivic spaces $E_n \in \text{Spc}_{\bullet}(S)$ together with structure morphisms $\sigma_n$:
    $$\hlin{ \sigma_n: T \wedge E_n \to E_{n+1} }$$
    (\Cref{definition:wedge_and_smash_products_of_pointed_simplicial_sheaves_of_sets_on_a_site})

    \item A \hldef{morphism of motivic spectra} $f: E \to F$ is a sequence of maps $f_n: E_n \to F_n$ in $\text{Spc}_{\bullet}(S)$ compatible with the structure morphisms, i.e., $f_{n+1} \circ \sigma_n^E = \sigma_n^F \circ (\text{id}_T \wedge f_n)$. 

    \item We denote the category of motivic spectra over $S$ by \hl{$\text{Spt}_{T}(S)$}.
\end{enumerate}
\end{definition}

\TODO{distingiush between the ``fully stable category'' = $T$-spectra category = $\bbP^1$-spectra category and the $S^1$-spectra category, stabilized only with respect to the simplicial circle $S^1$}
\begin{definition} \label{definition:stable_A1_homotopy_category_over_a_noetherian_scheme_of_finite_dimension}
    \TODO{separate out the stable model structure}
    The \hldef{stable $\mathbb{A}^1$-homotopy category}, denoted by notations such as \hl{$\mathcal{SH}(S)$}, or \hl{$\mathbf{SH}(S)$}, is the homotopy category of the category of motivic spectra $\text{Spt}_{T}(S)$ with respect to the \hldef{stable model structure}. This model structure is the stabilization of the pointed \CrefAndHyperrefIfExist{definition:A1_model_structure_on_the_category_of_simp_sheaves_of_sets_on_the_nis_site_over_a_noeth_scheme_of_fin_dim_and_the_unstable_A1_homotopy_category}{$\mathbb{A}^1$-model structure} on $\text{Spc}_{\bullet}(S)$ with respect to the endofunctor $- \wedge T$, meaning the weak equivalences are the stable $\mathbb{A}^1$-weak equivalences (morphisms inducing isomorphisms on all stable homotopy sheaves).
\end{definition}



\section{\texorpdfstring{$\bbA^1$}{A1}-connectivity theorems}

\TODO{state unstable version}
\TODO{state stable version of Schmidt-Strunk over Dedekind schemes}
\TODO{state Gabber's geometric presentation lemma over fields}

\TODO{State Gabber's geometric presentation lemma by Schmidt-Strunk (over dedekind schemes with infinite residue fields), then by Deshmukh, Hogadi, Kulkarni, and Yadav (DHKY) (over Noetherian domains with infinite residue fields) and Kulkarni-Hogadi's over finite fields}
\TODO{state Gabber's geometric presentation lemma by Druzhinin for henselian local essentially smooth schemes, see Remark 3 of loc.cit.}


\begin{theorem}[Stable $\bbA^1$ connectivity theorm {\cite[Theorems 2, 7]{druzhinin_sa1cob}}]
    \TODO{essentially smooth}
    Let $S$ be a scheme of \CrefAndHyperrefIfExist{definition:dimension_of_a_scheme}{Krull dimension} $d$. Let $U$ be an essentially smooth local henselian scheme over a base scheme $S$. Let $F \in \mathbf{SH}_{S^1}(S)$ be a $0$-connective spectrum. Then
    $$[U, F \wedge S^i]_{\mathbf{SH}_{S^1}(S)} = 0, \quad [U, \Sigma_{\bbG_m}^\infty F \wedge S^i]_{\mathbf{SH}_{S^1}(S)} = 0 \text{ for all } i > d.$$
    \TODO{henselian scheme}
    \TODO{stable homotopy group}
    \TODO{connective spectrum}
    \TODO{$\mathbf{SH}_{S^1}(S)$, $S^1$, $[,]$, $\Sigma_{\bbG_m}^\infty$}
\end{theorem}





\appendix
\section{Category Theory}


\begin{definition}[Category] \label{definition:category}
    A 
    \defin{category}{category}{
        name={Category},
        description={A nice enough collection of objects and morphisms (\Cref{definition:category})},
    }
    \hldef{category} $\mathcal{C}$ consists of the following data:
    \begin{itemize}
        \item A class of \defin{objects}{object_of_a_category}{
            name={Object of a category},
            description={\Cref{definition:category}},
        }
        denoted \notat{\operatorname{Ob}(\mathcal{C})}{class_of_objects_of_a_category}{
            name={$\operatorname{Ob}(\mathcal{C})$},
            description={Class of objects of a category $\calC$ \Cref{definition:category}},
            sort={Ob},
        }.
        % \hl{$\operatorname{Ob}(\mathcal{C})$}.
        \item For each pair of objects $X, Y \in \operatorname{Ob}(\mathcal{C})$, a class
        \notatin{\operatorname{Hom}_{\mathcal{C}}(X,Y)}{class_of_morphisms_between_two_objects_of_a_category}
        {
            name={$\operatorname{Hom}_{\mathcal{C}}(X,Y)$},
            description={Class of morphisms between objects $X$ and $Y$ of the category $\calC$ (\Cref{definition:category})},
            sort={Hom},
        }
        % $$\hlin{\operatorname{Hom}_{\mathcal{C}}(X,Y)}$$
        of \defin{morphisms}{morphism_between_objects_of_a_category}{
            name={Morphism between objects of a category},
            description={(\Cref{definition:category})},
        }
        (also called 
        \defin{arrows}{arrow_between_objects_of_a_category}{
            name={Arrow between objects of a category},
            description={Synonym for morphism (\Cref{definition:category})},
        }
        or
        \defin{homs}{hom_between_objects_of_a_category}{
            name={Hom between objects of a category},
            description={Synonym for morphism (\Cref{definition:category})},
        }). If the category $\calC$ is clear, then this \hldef{hom-class} is also denoted by \hl{$\operatorname{Hom}(X,Y)$}. It may also be denoted by \hl{$\operatorname{hom}_{\mathcal{C}}(X,Y)$} or \hl{$\operatorname{hom}(X,Y)$}, especially to distinguish from other types of hom's (e.g. \hyperrefIfExists{definition:internal_hom_object_in_a_category}{internal hom's})
        \item For each triple of objects $X,Y,Z$, a composition law
        $$ \circ : \operatorname{Hom}_{\mathcal{C}}(Y,Z) \times \operatorname{Hom}_{\mathcal{C}}(X,Y) \to \operatorname{Hom}_{\mathcal{C}}(X,Z), $$
        denoted \hl{$(g,f) \mapsto g \circ f$}.
        \item For each object $X$, an \hldef{identity morphism}
        $$\hlin{\operatorname{id}_X \in \operatorname{Hom}_{\mathcal{C}}(X,X).}$$
    \end{itemize}
    These data satisfy the following axioms:
    \begin{itemize}
        \item (Associativity) For all morphisms $f \in \operatorname{Hom}_{\mathcal{C}}(X,Y)$, $g \in \operatorname{Hom}_{\mathcal{C}}(Y,Z)$, and $h \in \operatorname{Hom}_{\mathcal{C}}(Z,W)$, 
        $$
        h \circ (g \circ f) = (h \circ g) \circ f.
        $$
        \item (Identity) For all $f \in \operatorname{Hom}_{\mathcal{C}}(X,Y)$,
        $$
        \operatorname{id}_Y \circ f = f = f \circ \operatorname{id}_X.
        $$
    \end{itemize}
    One often writes \hl{$X \in \calC$} synonymously with $X \in \Ob(\calC)$, i.e. to denote that $X$ is an object of of $\calC$. 

    We may call a category as above an \hldef{ordinary category} to distinguish this notion from the notions of \hyperrefIfExists{definition:category_enriched_in_a_monoidal_category}{\emph{categories enriched in monoidal categories}} or higher/$n$-categories.
    \TODO{TODO: define $n$-categories}

    A category as defined above may be called called a \hldef{large category} or a \hldef{class category} to emphasize that the hom-classes may be proper classes rather than sets (note, however, that the possibility that hom-classes are sets is not excluded for large categories). Accordingly, a \hldef{category} may often refer to a \hyperrefIfExists{definition:locally_small_category}{locally small category}\CrefIfExists{definition:locally_small_category}, which is a category whose hom-classes are all sets.
\end{definition}

% Later on, we refer to the \gls{category} again.

\begin{definition} \label{definition:functor_between_categories}
Let $\mathcal{C}$ and $\mathcal{D}$ be \CrefAndHyperrefIfExist{definition:category}{(large) categories}. 
\begin{enumerate}
  \item A \hldef{functor $F: \calC \to \calD$ (from $\mathcal{C}$ to $\mathcal{D}$)} consists of :
  \begin{itemize}
    \item For each object $X$ in $\mathcal{C}$, an object $F(X)$ in $\mathcal{D}$.
    \item For each morphism $f: X \to Y$ in $\mathcal{C}$, a morphism $F(f): F(X) \to F(Y)$ in $\mathcal{D}$,
  \end{itemize}
  such that:
  \begin{align*}
    F(\mathrm{id}_X) &= \mathrm{id}_{F(X)} \quad \text{for all objects } X \text{ in } \mathcal{C}, \\
    F(g \circ f) &= F(g) \circ F(f) \quad \text{for all } X,Y,Z \in \Ob(\calC) \text{ and all } f: X \to Y, g: Y \to Z \text{ in } \mathcal{C}.
  \end{align*}

  Functors as defined above are also referred to as \hldef{covariant functors} to distinguish them from contravariant functors

  \item A \hldef{contravariant functor from $\calC$ to $\calD$} refers to a covariant functor $F:\calC^{\op} \to \calD$. Equivalently, such a functor consists of 
  \begin{itemize}
    \item For each object $X$ in $\mathcal{C}$, an object $F(X)$ in $\mathcal{D}$.
    \item For each morphism $f: X \to Y$ in $\mathcal{C}$, a morphism $F(f): F(Y) \to F(X)$ in $\mathcal{D}$,
  \end{itemize}
  such that:
  \begin{align*}
    F(\mathrm{id}_X) &= \mathrm{id}_{F(X)} \quad \text{for all objects } X \text{ in } \mathcal{C}, \\
    F(g \circ f) &= F(f) \circ F(g) \quad \text{for all } X,Y,Z \in \Ob(\calC) \text{ and all } f: X \to Y, g: Y \to Z \text{ in } \mathcal{C}.
  \end{align*}
  \TextIfExists{definition:presheaf_on_a_category}{A synonym for a ``contravariant functor from $\calC$ to $\calD$'' is a ``\CrefAndHyperrefIfExist{definition:presheaf_on_a_category}{presheaf on $\calC$ with values in $\calD$}''.}
  
\end{enumerate}
Note that declarations such as ``Let $F: \calC^{\op} \to \calD$ be a contravariant functor'' can be common; such declarations usually mean ``Let $F$ be a contravariant functor from $\calC$ to $\calD$'' as opposed to ``Let $F$ be a contravariant functor from $\calC^{\op}$ to $\calD$''. further note that a contravariant functor from $\calC$ to $\calD$ is equivalent to a covariant functor from $\calC^{\op}$ to $\calD$.
\end{definition}


\begin{definition} \label{definition:natural_transformation_between_functors_between_categories}
Let $\mathcal{C}$ and $\mathcal{D}$ be \CrefAndHyperrefIfExist{definition:category}{(large) categories}. 
Let $F, G : \mathcal{C} \to \mathcal{D}$ be \CrefAndHyperrefIfExist{definition:functor_between_categories}{functors}.

A \hldef{natural transformation $\eta$ between $F$ and $G$} is a family of morphisms $\eta_X: F(X) \to G(X)$ in $\mathcal{D}$, one for each object $X$ in $\mathcal{C}$, such that for every morphism $f: X \to Y$ in $\mathcal{C}$,
\begin{align*}
G(f) \circ \eta_X = \eta_Y \circ F(f)
\end{align*}
in $\mathcal{D}$. In other words, the following diagram commutes:
\begin{center}
\begin{tikzcd}
    F(X) \arrow[r, "F(f)"] \arrow[d, "\eta_X"']
    & F(Y) \arrow[d, "\eta_Y"] \\
    G(X) \arrow[r, "G(f)"']
    & G(Y)
\end{tikzcd}
\end{center}

We write such a natural transformation by \hl{$\eta: F \Rightarrow G$}.

If $\eta_X$ is an \CrefAndHyperrefIfExist{definition:isomorphism_in_a_category}{isomorphism} for all objects $X$ of $\calC$, then $\eta$ is said to be a \hldef{natural isomorphism}.
\end{definition}



\begin{definition} \label{definition:initial_final_zero_objects_of_a_category}
Let $\mathcal{C}$ be a \CrefAndHyperrefIfExist{definition:category}{(large) category}.

\begin{enumerate}
    \item An object $I \in \mathcal{C}$ is called an \hldef{initial object} if for every object $X \in \mathcal{C}$ there exists a unique morphism
    $$I \to X.$$
    Equivalently, an initial object is a \CrefAndHyperrefIfExist{definition:limit_and_colimit_of_a_diagram_in_a_category}{limit} of the empty \CrefAndHyperrefIfExist{definition:diagram_in_a_category_indexed_by_a_small_category}{diagram}, if such a limit exists.

    \item An object $F \in \mathcal{C}$ is called a \hldef{final object} (or \hldef{terminal object}) if for every object $X \in \mathcal{C}$ there exists a unique morphism
    $$X \to F.$$
    Equivalently, a final object is a \CrefAndHyperrefIfExist{definition:limit_and_colimit_of_a_diagram_in_a_category}{colimit} of the empty \CrefAndHyperrefIfExist{definition:diagram_in_a_category_indexed_by_a_small_category}{diagram}, if such a colimit exists.

    \item An object $Z \in \mathcal{C}$ is called a \hldef{zero object} if $Z$ is both initial and final in $\mathcal{C}$. In particular, for every object $X \in \mathcal{C}$ there exist unique morphisms
    $$Z \to X \quad \text{and} \quad X \to Z.$$
\end{enumerate}
In particular, if initial/final/zero objects exist in a cateogry, then they are unique up to unique isomorphism.
\end{definition}


\begin{definition} \label{definition:equivalence_of_categories}
An \hldef{equivalence of categories} between two \CrefAndHyperrefIfExist{definition:category}{(large) categories} $\mathcal{C}$ and $\mathcal{D}$ consists of a pair of \CrefAndHyperrefIfExist{definition:functor_between_categories}{functors}
$$F : \mathcal{C} \to \mathcal{D} \quad \text{and} \quad G : \mathcal{D} \to \mathcal{C}$$
together with \CrefAndHyperrefIfExist{definition:natural_transformation_between_functors_between_categories}{natural isomorphisms}
$$\eta : \mathrm{Id}_{\mathcal{C}} \xrightarrow{\sim} G \circ F \quad \text{and} \quad \epsilon : F \circ G \xrightarrow{\sim} \mathrm{Id}_{\mathcal{D}}.$$
\CrefIfExists{definition:identity_functor_on_a_category} Such functors $F$ and $G$ may be called \hldef{(natural) inverses of each other}.

When $\calC$ and $\calD$ are \CrefAndHyperrefIfExist{definition:locally_small_category}{locally small categories}, $F$ is an equivalence of categories if and only if $F$ is \CrefAndHyperrefIfExist{definition:full_and_faithful_functor_between_locally_small_categories}{fully faithful} and \CrefAndHyperrefIfExist{definition:essentially_surjective_functor_between_categories}{essentially surjective}
\end{definition}
\begin{definition}[Locally small category] \label{definition:locally_small_category}
A \hyperrefIfExists{definition:category}{(large) category}\CrefIfExists{definition:category} $\mathcal{C}$ is called a \hldef{locally small category} if for every pair of objects $X, Y \in \operatorname{Ob}(\mathcal{C})$, the collection $\operatorname{Hom}_{\mathcal{C}}(X,Y)$ of morphisms between them is a (\CrefAndHyperrefIfExist{definition:small_set}{small}) \emph{set} (as opposed to a proper class). In other words, each hom-class is a set and may even be called a \hldef{hom-set}.

In some contexts, a locally small category may simply be called a \hldef{category}, especially when genuinely large categories are not considered.

A category $\mathcal{C}$ is called a \hldef{small category} if it is a locally small category and the class $\operatorname{Ob}(\mathcal{C})$ of objects is a set.

\TextIfExists{definition:grothendieck_universe}{
Given a \hyperrefIfExists{definition:grothendieck_universe}{universe}\CrefIfExists{definition:grothendieck_universe} $U$, we can define the notion of a \hldef{$U$-locally small category} and of a \hldef{$U$-small category} similarly. More explicitly, 
\begin{enumerate}
    \item a $U$-locally small category is a category such that for every pair of objects $X, Y \in \operatorname{Ob}(\mathcal{C})$, the collection $\operatorname{Hom}_{\mathcal{C}}(X,Y)$ of morphisms between them is a $U$-set.
    \item a $U$-small category is a category such that $\operatorname{Ob}(\mathcal{C})$ is a $U$-set and for every pair of objects $X, Y \in \operatorname{Ob}(\mathcal{C})$, the collection $\operatorname{Hom}_{\mathcal{C}}(X,Y)$ of morphisms between them is a $U$-set; in particular the collection of all objects and morhpisms in a $U$-small category is a $U$-set.
\end{enumerate}
}
\end{definition}

\begin{remark}
    Many ``concrete'' categories considered in ``classical mathematics'' or outside of more ``abstract'' category theory tend to be locally small. For example, the categories of sets, groups, $R$-modules, vector spaces, topological spaces, schemes, manifolds, sheaves on ``small enough'' sites are all locally small.
\end{remark}
\begin{definition}[Presheaf on a category] \label{definition:presheaf_on_a_category}
    Let $C$ and $\mathcal{A}$ be \hyperrefIfExists{definition:category}{(large) categories}\CrefIfExists{definition:category}. 
    \begin{enumerate}
        \item A \hldef{presheaf $\mathcal{F}$ on $C$ with values in $\mathcal{A}$} is a functor
        \[
        \mathcal{F}: C^{\mathrm{op}} \to \mathcal{A}.
        \]
        In other words, a presheaf $\calF$ on $C$ with values in $\calA$ is simply a \CrefAndHyperrefIfExist{definition:functor_between_categories}{contravariant functor} from $C$ to $\calA$. 
        Explicitly, for every object $U$ in $C$, one has an object $\mathcal{F}(U)$ in $\mathcal{A}$ (called the \hldef{$U$-valued sections/sections evaluated at $U$ of $\calF$}\TextIfExists{definition:sections_of_a_presheaf_on_a_category_valued_in_a_category}{, cf. \Cref{definition:sections_of_a_presheaf_on_a_category_valued_in_a_category}}), and for every morphism $f: V \to U$ in $C$, one has a morphism (called the \hldef{restriction map})
        \[
        \mathcal{F}(f): \mathcal{F}(U) \to \mathcal{F}(V)
        \]
        in $\mathcal{A}$, such that for all composable morphisms $W \xrightarrow{g} V \xrightarrow{f} U$ in $C$, the following diagram in $\mathcal{A}$ commutes:
        \[
        \begin{tikzcd}
        \mathcal{F}(U) \arrow[r, "\mathcal{F}(f)"] \arrow[rr, bend left, "\mathcal{F}(f \circ g)"] & \mathcal{F}(V) \arrow[r, "\mathcal{F}(g)"] & \mathcal{F}(W)
        \end{tikzcd}
        \]
        That is,
        \[
        \mathcal{F}(g) \circ \mathcal{F}(f) = \mathcal{F}(f \circ g),
        \]
        and for every object $U$ in $C$, $\mathcal{F}(\mathrm{id}_U) = \mathrm{id}_{\mathcal{F}(U)}$.


        \item 
        Let $\mathcal{F},\mathcal{G}: C^{\mathrm{op}} \to \mathcal{A}$ be two presheaves on $C$ with values in $\mathcal{A}$. A \hldef{morphism of presheaves}
        \[
        \varphi: \mathcal{F} \to \mathcal{G}
        \]
        is a \hyperrefIfExists{definition:natural_transformation_between_functors_between_categories}{natural transformation of functors}\CrefIfExists{definition:natural_transformation_between_functors_between_categories}: for each object $U$ of $C$, one has a morphism
        \[
        \varphi_U: \mathcal{F}(U) \to \mathcal{G}(U)
        \]
        in $\mathcal{A}$, such that for every morphism $f: V \to U$ in $C$, the diagram
        \[
        \begin{tikzcd}
        \mathcal{F}(U) \arrow[r, "\mathcal{F}(f)"] \arrow[d, "\varphi_U"'] & \mathcal{F}(V) \arrow[d, "\varphi_V"] \\
        \mathcal{G}(U) \arrow[r, "\mathcal{G}(f)"'] & \mathcal{G}(V)
        \end{tikzcd}
        \]
        commutes, i.e.,
        \[
        \varphi_V \circ \mathcal{F}(f) = \mathcal{G}(f) \circ \varphi_U
        \]
        for all objects and morphisms in $C$.

        \item Given a \hyperrefIfExists{definition:grothendieck_universe}{universe}\CrefIfExists{definition:grothendieck_universe} $U$, a \hldef{$U$-presheaf on $\calC$} typically refers to a presheaf of $U$-sets on $C$.

        \item The \hldef{presheaf category/category of $\calA$-valued presheaves on $\calC$} is the (large) category whose objects are the presheaves on $C$ with values in $\calA$ and whose morphisms are the presheaf morphisms. Common notations for the presheaf category include, but are not limited to: \hl{$\calA^{\calC^{\op}}$}, \hl{$\PreShv(\calC, \calA)$}, \hl{$[\calC^{\op}, \calA]$}. If the value category $\calA$ is clear from context, then notations such as \hl{$\PreShv(\calC)$} are also common. \TextIfExists{definition:diagram_in_a_category_indexed_by_a_small_category}{Note that the presheaf category $\PreShv(\calC, \calA)$ is equivalent to the \CrefAndHyperrefIfExist{definition:diagram_in_a_category_indexed_by_a_small_category}{category of functors} $\calC^{\op} \to \calA$ and hence notations for the functor categories are applicable as notations for presheaf categories.}

    \end{enumerate}
\end{definition}

% \begin{definition}[Grothendieck topology] \label{definition:grothendieck_topology_on_a_category_site_covering_sieve_topologically_generating_family}
%     Let $\mathscr{U}$ be a \hyperrefIfExists{definition:grothendieck_universe}{universe}\CrefIfExists{definition:grothendieck_universe} and let $\calC$ be a \hyperrefIfExists{definition:locally_small_category}{locally small category}\CrefIfExists{definition:locally_small_category}.

%     \begin{enumerate}
%         \item \textbf{(Grothendieck Topology via Sieves)}
%         A \hldef{Grothendieck topology} $J$ on $\calC$ is an assignment to each object $U \in \calC$ of a collection $J(U)$ of \CrefAndHyperrefIfExist{definition:sieve_on_an_object_in_a_category}{sieves} on $U$, called \hldef{covering sieves}, satisfying:
%         \begin{enumerate}
%             \item (Maximality) The maximal \CrefAndHyperrefIfExist{definition:sieve_on_an_object_in_a_category}{sieve} $\{ f : V \to U \mid V \in \calC \}$ is in $J(U)$.
%             \item (Stability) If $S \in J(U)$ and $f : V \to U$ is any morphism, then the \CrefAndHyperrefIfExist{definition:pullback_sieve_of_an_object_in_a_category_via_a_morphism_to_the_object}{pullback sieve} $f^{*}S$ is in $J(V)$.
%             \item (Transitivity/Local Character) If $S$ is a sieve on $U$ and there exists a covering sieve $R \in J(U)$ such that for every morphism $f : V \to U$ in $R$, the pullback sieve $f^{*}S$ is in $J(V)$, then $S \in J(U)$.
%         \end{enumerate}

%         % \item \textbf{(Grothendieck Pretopology / Basis)}
%         % If $\calC$ admits fiber products, one can define a topology via \hldef{covering families}. A \hldef{Grothendieck pretopology} (or basis) is a collection $K(U)$ of families $\{U_i \to U\}_{i \in I}$ for each object $U$, satisfying:
%         % \begin{itemize}
%         %     \item (Isomorphism) $\{U' \xrightarrow{\sim} U\} \in K(U)$ for any isomorphism.
%         %     \item (Stability) If $\{U_i \to U\} \in K(U)$ and $V \to U$ is a morphism, then $\{U_i \times_U V \to V\} \in K(V)$.
%         %     \item (Composition) If $\{U_i \to U\} \in K(U)$ and for each $i$, $\{V_{ij} \to U_i\} \in K(U_i)$, then the composite family $\{V_{ij} \to U\} \in K(U)$.
%         % \end{itemize}
%         % Every pretopology generates a unique Grothendieck topology $J$, where $S \in J(U)$ iff $S$ contains a covering family from the pretopology.

%         \item A \hldef{site} is a pair $(\calC, J)$ consisting of a category $\calC$ and a Grothendieck topology $J$.

%         \item A family of objects $\mathcal{G} = \{G_\alpha\}$ in a site $(\calC, J)$ is called a \hldef{topologically generating family} if for every object $X \in \calC$, there exists a covering sieve $S \in J(X)$ \CrefAndHyperrefIfExist{definition:sieve_on_an_object_of_a_category_generated_by_a_family_of_morphisms}{generated by} morphisms with domains in $\mathcal{G}$. Equivalently, every object $X$ admits a cover $\{U_i \to X\}$ where each $U_i \in \mathcal{G}$.

%         \item A \hldef{$\mathscr{U}$-site} is a site whose underlying category is $\mathscr{U}$-locally small and which admits a $\mathscr{U}$-small topologically generating family.
%     \end{enumerate}
% \end{definition}

\begin{definition}[Grothendieck topology] \label{definition:grothendieck_topology_on_a_category_site_covering_sieve_topologically_generating_family}
    Let $\mathscr{U}$ be a \hyperrefIfExists{definition:grothendieck_universe}{universe}\CrefIfExists{definition:grothendieck_universe}.
    \begin{enumerate}
        % \item Let $C$ be a \hyperrefIfExists{definition:locally_small_category}{locally small category}\CrefIfExists{definition:locally_small_category}. A \hldef{Grothendieck topology on $C$} assigns to each object $U$ of $C$ a collection of families of morphisms $\{U_i \to U\}_{i \in I}$, called \hldef{coverings of $U$}, satisfying:
        % \begin{itemize}
        %     \item (Isomorphism) If $f: V \to U$ is an isomorphism in $C$, then $\{f: V \to U\}$ is a covering of $U$.
        %     \item (Stability under base change) If $\{U_i \to U\}_{i \in I}$ is a covering of $U$ and $V \to U$ is any morphism, then the family $\{ U_i \times_U V \to V \}_{i \in I}$ is a covering of $V$.
        %     \item (Transitivity) If $\{U_i \to U\}_{i \in I}$ is a covering of $U$ and for each $i$, $\{V_{ij} \to U_i\}_{j \in J_i}$ is a covering of $U_i$, then the family $\{ V_{ij} \to U \}_{i \in I,\, j \in J_i}$ is a covering of $U$.
        % \end{itemize}

        \item (See \cite[Expos\'e II, D\'efinition 1.1]{SGA4_I}) Let $\calC$ be a \CrefAndHyperrefIfExist{definition:category}{category}. A \hldef{Grothendieck topology on $\calC$} assigns to each object $U$ of $\calC$ a collection \hl{$J(U)$} of \CrefAndHyperrefIfExist{definition:sieve_on_an_object_in_a_category}{sieves} $\{U_i \to U\}_{i \in I}$, each called a \hldef{covering sieve of $U$}, satisfying:
        \begin{enumerate}
            \item (Stability under ``base change''): If $S \in J(U)$ is a covering sieve of an object $U$, and $f: V \to U$ is any morphism in $\calC$, then the \CrefAndHyperrefIfExist{definition:pullback_sieve_of_an_object_in_a_category_via_a_morphism_to_the_object}{pullback sieve} $f^* S$ is a covering sieve of $U$.
            % \item (Local character condition) If $F$ is a sieve on $U$ such that the sieve $\bigcup_...$ \TODO{}
            \item (Local character condition) If $S$ is a sieve on $U$, and if there exists a covering sieve $R \in J(U)$ such that for all $f: V \to U$ in $R$ the \CrefAndHyperrefIfExist{definition:pullback_sieve_of_an_object_in_a_category_via_a_morphism_to_the_object}{pullback sieve} $f^* S$ is in $J(V)$, then $S \in J(U)$. 
            
            \item The \CrefAndHyperrefIfExist{definition:maximal_sieve_on_an_object_in_a_category}{maximal sieve} is a covering sieve.
        \end{enumerate}


        % Equivalently, a Grothendieck topology $J$ on a category $C$ is an assignment of a collection $J(U)$ of \CrefAndHyperrefIfExist{definition:sieve_on_an_object_in_a_category}{sieves} on each object $U \in \operatorname{Ob}(C)$ such that:
        % \begin{enumerate}
        %     \item the maximal \CrefAndHyperrefIfExist{definition:sieve_on_an_object_in_a_category}{sieve} $\{ f : V \to U \mid f \in \operatorname{Mor}(C) \}$ belongs to $J(U)$,
        %     \item if $S \in J(U)$ and $f : V \to U$, then the \CrefAndHyperrefIfExist{definition:pullback_sieve_of_an_object_in_a_category_via_a_morphism_to_the_object}{pullback sieve $f^{*}S$} on $V$ belongs to $J(V)$,
        %     \item if $S$ is a sieve on $U$, and if there exists $R \in J(U)$ such that for all $f : V \to U$ in $R$ the \CrefAndHyperrefIfExist{definition:pullback_sieve_of_an_object_in_a_category_via_a_morphism_to_the_object}{pullback sieve $f^{*}S$} is in $J(V)$, then $S \in J(U)$.
        % \end{enumerate}

        Some will refer to a Grothendieck topology as simply a \hldef{topology}, not to be confused with the related, but less general, notion of a \CrefAndHyperrefIfExist{definition:topological_space}{topology on a set}.


        \item (See \cite[Expos\'e II, 1.1.5]{SGA4_I}) A \hldef{site} is a category $\calC$ equipped with a Grothendieck topology.

        When we are working with a \CrefAndHyperref{definition:basis_and_grothendieck_pretopology_for_a_grothendieck_topology_on_a_category}{Grothendieck pretopology} $K$ on a category $\calC$, we may regard $\calC$ as a site by equipping it with the \CrefAndHyperref{definition:grothendieck_topology_generated_by_a_pretopology}{Grothendieck topology generated by} $K$. 

        \item (See \cite[Expos\'e II, D\'efinition 1.2]{SGA4_I}) Let $(\calC, J)$ be a site. A family of morphisms $(U_i \to U)_{i \in I}$ is called a \hldef{covering family of $U$ (with respect to the site/topology)} or a \hldef{cover of $U$ (with respect to the site/topology)} if the \CrefAndHyperrefIfExist{definition:sieve_on_an_object_of_a_category_generated_by_a_family_of_morphisms}{sieve generated by} the family is a covering sieve of $U$. 

        \item (See \cite[Expos\'e II, D\'efinition 3.0.1]{SGA4_I}) Let $(\calC, J)$ be a \CrefAndHyperrefIfExist{definition:grothendieck_topology_on_a_category_site_covering_sieve_topologically_generating_family}{site}, where $J$ is a Grothendieck topology on $\calC$.

        A family $G$ of objects $\calC$ is called a \hldef{topologically generating family of the site/topology} or a \hldef{generating family/collection of the site/topology} if for every object $X \in \calC$, there is a covering family $\{X_\alpha \to X\}_{\alpha \in A}$ of $X$ such that every $X_\alpha$ is a member of $G$.  

        Equivalently, the Grothendieck topology $J$ is the smallest Grothendieck topology containing all covers of the $U_i$. Also equivalently, for any $S \in J(X)$, the sieve $S$ contains a covering family $\{V_i \to X\}$ such that each morphism $V_i \to X$ factors through some member of $G$. \TODO{Verify that these claimed equivalences are indeed equivalences}
        
        % A family of objects $\{U_i\}_{i \in I}$ in $\calC$ is called a \hldef{topologically generating family} if for every object $X \in \calC$ and every covering sieve $S \in J(X)$, the sieve $S$ is \CrefAndHyperrefIfExist{definition:sieve_on_an_object_of_a_category_generated_by_a_family_of_morphisms}{generated by} pullbacks of covering families from the family $\{U_i\}$.

        % More precisely, this means that for any $S \in J(X)$, the sieve $S$ contains a covering family $\{V_j \to X\}$ such that each morphism $V_j \to X$ factors through some $U_i$, and the covering families of the $U_i$ generate the topology $J$. 
        % Equivalently, the Grothendieck topology $J$ is the smallest Grothendieck topology containing all coverings of the $U_i$.

        % When one speaks of a \hldef{generating family/collection} of a site, one usually refers to the above notion of a topologically generating family.

        \item (See \cite[Expos\'e II, D\'efinition 3.0.2]{SGA4_I}) A \hldef {$\mathscr{U}$-site} is a site whose underlying category $\calC$ is \hyperrefIfExists{definition:locally_small_category}{$\mathscr{U}$-locally small}\CrefIfExists{definition:locally_small_category} and which has a $\mathscr{U}$-small topologically generating family. A $\mathscr{U}$-site is called \hldef{$\mathscr{U}$-small} if its underlying category is $\mathscr{U}$-small. Similarly, a \hldef{small site} is a site whose underlying category is a set and a \hldef{locally small site} is a site whose underlying category is \CrefAndHyperrefIfExist{definition:locally_small_category}{locally small}.
    \end{enumerate}
\end{definition}

\begin{definition}[Sheaf on a site] \label{definition:sheaf_on_a_site}

% \TODO{There might be some need to say that $\calA$ is a category for which sheaves on the site ``can be defined''}
% \TODO{go through statements using the notion of sheaves and make sure that the value categories have small products and that the categories have small generating families.}

Let $(\calC, J)$ be a \CrefAndHyperrefIfExist{definition:grothendieck_topology_on_a_category_site_covering_sieve_topologically_generating_family}{site}. Let $\calA$ be a \CrefAndHyperrefIfExist{definition:category}{(large) category}.
\begin{enumerate}
    \item A \CrefAndHyperrefIfExist{definition:presheaf_on_a_category}{presheaf} $\calF: \calC^{\op} \to \calA$\CrefIfExists{definition:opposite_category_of_a_category} is called a \hldef{sheaf on the site $(\calC, J)$ valued in $\calA$} if, for every object $U$ of $\calC$ and every \CrefAndHyperrefIfExist{definition:grothendieck_topology_on_a_category_site_covering_sieve_topologically_generating_family}{covering sieve} $S \in J(U)$, the \CrefAndHyperrefIfExist{definition:limit_and_colimit_of_a_diagram_in_a_category}{limit}
    $$\varprojlim_{(V \to U) \in (\calD_S)^{\op}} \calF|_{\calD_S}(V),$$
    exists and the canonical natural morphism
    $$\calF(U) \to \varprojlim_{(V \to U) \in (\calD_S)^{\op}} \calF|_{\calD_S}(V)$$
    is an isomorphism. Here, $\calD_S \hookrightarrow \calC/U$\CrefIfExists{definition:category_of_objects_over_under_a_fixed_object_in_a_category} is the full \CrefAndHyperrefIfExist{definition:downward_upward_closed_subcategory_of_a_category}{downward-closed subcategory} such that $\operatorname{Ob}(\calD_S) = \{(f: V \to U): f \in S(V)\}$,

    In particular, when we are working with a \CrefAndHyperref{definition:basis_and_grothendieck_pretopology_for_a_grothendieck_topology_on_a_category}{Grothendieck pretopology} $K$ on a category $\calC$, we may speak of sheaves on the site whose Grothendieck topology is the \CrefAndHyperref{definition:grothendieck_topology_generated_by_a_pretopology}{one generated by} $K$.

    \item Given sheaves $\calF, \calG: \calC^{\op} \to \calA$ on the site $(\calC, J)$, a \hldef{morphism between the sheaves} is a \CrefAndHyperrefIfExist{definition:presheaf_on_a_category}{morphism} between $\calF$ and $\calG$ as presheaves.


    \item Let $U$ be a \hyperrefIfExists{definition:grothendieck_universe}{universe}\CrefIfExists{definition:grothendieck_universe}. A \hldef{$U$-sheaf} typically refers to a $U$-presheaf that is a sheaf for a $U$-site. In other words, a $U$-sheaf is a sheaf on a site whose underlying category is \hyperrefIfExists{definition:locally_small_category}{$U$-locally small}\CrefIfExists{definition:locally_small_category} and which has a $U$-small topologically generating family such that the sheaf is valued in $U$-sets.

    \item The \hldef{sheaf category/category of $\calA$-valued sheaves on $\calC$} is the (large) category defined as the full subcategory of $\PreShv(\calC, \calA)$ whose objects are the sheaves on $\calC$ with values in $\calA$. Common notations for the sheaf category include \hl{$\Shv(\calC, \calA)$}, \hl{$\Shv(\calC, J, \calA)$}, \hl{$\Sh(\calC, \calA)$}, \hl{$\Sh(\calC, J, \calA)$}. If the value category $\calA$ is clear from context, then notations such as \hl{$\Shv(\calC)$}, \hl{$\Shv(\calC, J)$}, \hl{$\Sh(\calC)$}, \hl{$\Sh(\calC, J)$} are also common.

\end{enumerate}

% Let $(\calC, J)$ be a \CrefAndHyperrefIfExist{definition:grothendieck_topology_on_a_category_site_covering_sieve_topologically_generating_family}{site} with a small \CrefAndHyperrefIfExist{definition:grothendieck_topology_on_a_category_site_covering_sieve_topologically_generating_family}{topological generating family} (or a $U$-small topologically generating family if a \CrefAndHyperrefIfExist{definition:grothendieck_universe}{universe} $U$ is available) and let $\mathcal{A}$ be a \CrefAndHyperrefIfExist{definition:category}{(large) category} that has all \CrefAndHyperrefIfExist{definition:locally_small_category}{small} \CrefAndHyperrefIfExist{definition:product_and_coproduct_of_objects_in_a_category}{products} (Some common examples of categories that have small products and thus play the role of $\calA$ here include $\mathcal{A} = \text{Set}$, $\text{Ab}$, $R\mathbf{-mod}$ for a fixed ring $R$, $\text{rings}$). 
% \begin{enumerate}

%     \item For any object $U$ of $\calC$ and every covering $\{U_i \to U\}_{i \in I}$ in $J$, note that there are morphisms $U_i \times_U U_j \to U_i$ for every $i,j \in I$. 
%     % Consider the subcategory of $C$ consisting of the objects $U_i$ and $U_i \times_U U_j$, together with these morphisms.
%     Given any presheaf $\calF: C^{\op} \to \calA$, there is a \CrefAndHyperrefIfExist{definition:diagram_in_a_category_indexed_by_a_small_category}{diagram} in $\calA$ consisting of objects $\calF(U_i)$ and $\calF(U_i \times_U U_j)$ and morphisms $\calF(U_i) \to \calF(U_i \times_U U_j)$. The presheaf $\calF$ is called a \hldef{sheaf on the site $(\calC, J)$ valued in $\calA$} if, for every object $U$ of $\calC$ and every covering $\{U_i \to U\}_{i \in I}$ in $J$, the sections object $\calF(U)$ is the \CrefAndHyperrefIfExist{definition:limit_and_colimit_of_a_diagram_in_a_category}{limit} of the aforementioned diagram:
    
%     % A \hyperrefIfExists{definition:presheaf_on_a_category}{presheaf}\CrefIfExists{definition:presheaf_on_a_category} $\mathcal{F}: C^{\mathrm{op}} \to \mathcal{A}$ is a \hldef{sheaf on the site $(\calC,J)$ valued in $\calA$} if, for every object $U$ of $\calC$ and every covering $\{U_i \to U\}_{i \in I}$ in $J$, the sections object $\calF(U)$ is the \CrefAndHyperrefIfExist{definition:limit_and_colimit_of_a_diagram_in_a_category}{limit} of the sections objects $\calF(U_i)$:
%     % $$\calF(U) \cong \varprojlim_{}$$
    
%     % following sequence is an \CrefAndHyperrefIfExist{definition:equalizer_and_coequalizer_of_morphisms_in_a_category}{equalizer} in $\mathcal{A}$:
%     % \[
%     % \mathcal{F}(U) \to \prod_{i} \mathcal{F}(U_i) \rightrightarrows \prod_{i, j} \mathcal{F}(U_i \times_U U_j)
%     % \]
%     % where the first map sends $s$ to $(\mathcal{F}(U_i \to U)(s))_i$ and the arrows to $(\mathcal{F}(U_i \times_U U_j \to U_i)(s_i))_{i,j}$ and $(\mathcal{F}(U_i \times_U U_j \to U_j)(s_j))_{i,j}$, respectively.

%     % \item A \hyperrefIfExists{definition:presheaf_on_a_category}{presheaf}\CrefIfExists{definition:presheaf_on_a_category} $\mathcal{F}: C^{\mathrm{op}} \to \mathcal{A}$ is a \hldef{sheaf on the site $(\calC,J)$ valued in $\calA$} if, for every object $U$ of $\calC$ and every covering $\{U_i \to U\}_{i \in I}$ in $J$, the following sequence is an \CrefAndHyperrefIfExist{definition:equalizer_and_coequalizer_of_morphisms_in_a_category}{equalizer} in $\mathcal{A}$:
%     % \[
%     % \mathcal{F}(U) \to \prod_{i} \mathcal{F}(U_i) \rightrightarrows \prod_{i, j} \mathcal{F}(U_i \times_U U_j)
%     % \]
%     % where the first map sends $s$ to $(\mathcal{F}(U_i \to U)(s))_i$ and the arrows to $(\mathcal{F}(U_i \times_U U_j \to U_i)(s_i))_{i,j}$ and $(\mathcal{F}(U_i \times_U U_j \to U_j)(s_j))_{i,j}$, respectively.

%     \item A \hldef{morphism of sheaves} $\calF: \calC^{\op} \to \calA$ is a \hyperrefIfExists{definition:presheaf_on_a_category}{morphism as presheaves}\CrefIfExists{definition:presheaf_on_a_category}. 


%     \item Let $U$ be a \hyperrefIfExists{definition:grothendieck_universe}{universe}\CrefIfExists{definition:grothendieck_universe}. A \hldef{$U$-sheaf} typically refers to a $U$-presheaf that is a sheaf for a $U$-site. In other words, a $U$-sheaf is a sheaf on a site whose underlying category is \hyperrefIfExists{definition:locally_small_category}{$U$-locally small}\CrefIfExists{definition:locally_small_category} and which has a $U$-small topologically generating family such that the sheaf is valued in $U$-sets.

%     \item The \hldef{sheaf category/category of $\calA$-valued sheaves on $\calC$} is the (large) category defined as the full subcategory of $\PreShv(\calC, \calA)$ whose objects are the sheaves on $C$ with values in $\calA$. Common notations for the sheaf category include \hl{$\Shv(\calC, \calA)$}, \hl{$\Shv(\calC, J, \calA)$}, \hl{$\Sh(\calC, \calA)$}, \hl{$\Sh(\calC, J, \calA)$}. If the value category $\calA$ is clear from context, then notations such as \hl{$\Shv(\calC)$}, \hl{$\Shv(\calC, J)$}, \hl{$\Sh(\calC)$}, \hl{$\Sh(\calC, J)$} are also common.

% \end{enumerate}
\end{definition}

\begin{definition} \label{definition:sheafification_functor_on_a_site}
    Let $\calC$ be a \CrefAndHyperrefIfExist{definition:grothendieck_topology_on_a_category_site_covering_sieve_topologically_generating_family}{site} and let $\calA$ be a \CrefAndHyperrefIfExist{definition:category}{(large) category}.

    Assuming that the \CrefAndHyperrefIfExist{definition:presheaf_on_a_category}{presheaf} category $\PreShv(\calC, \calA)$ (and hence the \CrefAndHyperrefIfExist{definition:sheaf_on_a_site}{sheaf} category $\Shv(\calC, \calA)$) is \CrefAndHyperrefIfExist{definition:locally_small_category}{locally small} (or $U$-locally small if a \CrefAndHyperrefIfExist{definition:grothendieck_universe}{Grothendieck universe} $U$ is available), a \hldef{sheafification functor} refers to a functor
    $$a: \PreShv(\calC, \calA) \to \Shv(\calC, \calA) $$
    that is \CrefAndHyperrefIfExist{definition:adjoint_functors_between_categories_unit_counit_of_adjoint_functors}{left adjoint} to the inclusion functor 
    $$i:\Shv(\calC, \calA) \hookrightarrow \PreShv(\calC, \calA)  .$$
    If such a sheafification functor exists, then it is unique up to unique natural isomorphism. Given a presheaf $P$, the sheafification $a(P)$ is also sometimes called the \hldef{sheaf associated to $P$}.
    \TextIfExists{theorem:sheafification_of_a_presheaf_of_sets_on_a_small_enough_site}{See \Cref{theorem:sheafification_of_a_presheaf_of_sets_on_a_small_enough_site} for common conditions under which sheafification exists.} 
\end{definition}

% See Also
%theorem:sheafification_of_a_presheaf_of_sets_on_a_small_enough_site

\begin{definition}[Diagram in a category and category of diagrams] \label{definition:diagram_in_a_category_indexed_by_a_small_category}
Let $\mathcal{C}$ be a \hyperrefIfExists{definition:category}{(large) category}\CrefIfExists{definition:category}, and let $I$ be a \CrefAndHyperrefIfExist{definition:category}{(large) category}. 
    \begin{enumerate}
        \item 
        A \hldef{diagram of shape $I$ in $\mathcal{C}$} is a \hyperrefIfExists{definition:functor_between_categories}{functor}\CrefIfExists{definition:functor_between_categories} $D: I \to \mathcal{C}$.
        We often denote such a diagram by the family \hl{$\{ D(i) \}_{i \in \mathrm{Ob}(I)}$} with transition maps given by the functorial image of morphisms in $I$. 
        
        A diagram is also synonymously called a \hldef{system}. Moreover, the category $I$ is called the \hldef{index category} or the \hldef{indexing category of the diagram $D$}.

        \item Given two diagrams $D,E: I \to \mathcal{C}$, a \hldef{morphism of diagrams} is a simply a \hyperrefIfExists{definition:natural_transformation_between_functors_between_categories}{natural transformation}\CrefIfExists{definition:natural_transformation_between_functors_between_categories} $D \Rightarrow E$ of the functors $D$ and $E$. 

        \item The \hldef{category of $I$-shaped diagrams in $\mathcal{C}$} or simply \hldef{diagram category (of $I$-shaped diagrams in $\calC$)}, often denoted \hl{$\mathcal{C}^I$}, \hl{$[I, \calC]$}, or \hl{$\operatorname{Fun}(I, \calC)$},
        is the (large) category whose objects are functors $I \to \mathcal{C}$ (that is, diagrams of shape $I$ in $\mathcal{C}$) and whose morphisms are \CrefAndHyperrefIfExist{definition:natural_transformation_between_functors_between_categories}{natural transformations} between such functors. The category $\calC^I$ is also called the \hldef{functor category of functors $I \to \calC$}. \TextIfExists{definition:presheaf_on_a_category}{Equivalently, the functor category $\calC^I$ is the category \CrefAndHyperrefIfExist{definition:presheaf_on_a_category}{$\PreShv(I^{\op}, \calC)$ of presheaves} on $I^{\op}$ with values in $\calC$ and hence notations for presheaf categories are applicable as notations for functor categories.}

        If $\calC$ is \hyperrefIfExists{definition:locally_small_category}{locally small}\CrefIfExists{definition:locally_small_category} and $I$ is small, then $\calC^I$ is locally small by Lemma \ref{lemma:category_of_presheaves_on_a_small_category_of_locally_small_value_is_locally_small}.
    \end{enumerate}
\end{definition}

\begin{definition}[Cones, limits and colimits in a category] \label{definition:limit_and_colimit_of_a_diagram_in_a_category}
Let $\mathcal{C}$ be a \CrefAndHyperrefIfExist{definition:category}{(large) category}, let $I$ be a (large) category, and let $D: I \to \mathcal{C}$ be a \CrefAndHyperrefIfExist{definition:diagram_in_a_category_indexed_by_a_small_category}{diagram}\CrefIfExists{definition:diagram_in_a_category_indexed_by_a_small_category}.

\begin{enumerate}
    \item A \hldef{cone to the diagram $D$} is an object $L \in \mathcal{C}$ together with a family of morphisms
    \[
    \{\pi_i: L \to D(i)\}_{i \in I}
    \]
    such that for every morphism $f: i \to j$ in $I$, the diagram
    \begin{center}
    \begin{tikzcd}[row sep=large, column sep=large]
        & L \arrow[dl, "\pi_i"'] \arrow[dr, "\pi_j"] & \\
        D(i) \arrow[rr, "D(f)"] & & D(j)
    \end{tikzcd}
    \end{center}
    commutes, i.e.  $D(f) \circ \pi_i = \pi_j$.
    


    \item A cone $(L, \{\pi_i\})$ is called a \hldef{limit of $D$} if it satisfies the following ``universal property'':
    for any cone $(C, \{ f_i \})$ over $D$, there exists a \textit{unique} morphism $u: C \to L$ such that
    \[
    \pi_i \circ u = f_i \quad \text{for all } i \in I.
    \]
    Visually, the following diagrams commute every morphism $f: i \to j$ in $I$:
    \begin{center}
    \begin{tikzcd}[row sep=large, column sep=large]
        & C \arrow[d, "\exists ! u", dashed] \arrow[ddl, "f_i"', bend right=20] \arrow[ddr, "f_j", bend left=20] & \\
        & L \arrow[dl, "\pi_i"] \arrow[dr, "\pi_j"'] & \\
        D(i) \arrow[rr, "D(f)"] & & D(j).
    \end{tikzcd}
    \end{center}
    If such a cone exists, then the object $L$ is necessarily unique up to unique isomorphism by the universal property. In this case, $L$ is denoted by \hl{$\lim_{i \in I} D$} or \hl{$\lim D$}.



    
    \item A \hldef{cocone from the diagram $D$} is an object $C \in \mathcal{C}$ together with a family of morphisms
    \[
    \{\iota_i: D(i) \to C\}_{i \in I}
    \]
    such that for every morphism $f: i \to j$ in $I$, the diagram
    \begin{center}
    \begin{tikzcd}[row sep=large, column sep=large]
        D(i) \arrow[rr, "D(f)"] \arrow[dr, "\iota_i"'] & & D(j) \arrow[dl, "\iota_j"] \\
        & C & 
    \end{tikzcd}
    \end{center}
    commutes, i.e. $\iota_j \circ D(f) = \iota_i$.

    \item A cocone $(L, \{\iota_i\})$ is called a \hldef{colimit of $D$} if it satisfies the following ``universal property'':
    for any cocone $(C, \{ g_i \})$ under $D$, there exists a \textit{unique} morphism $u: L \to C$ such that
    \[
    u \circ \iota_i = g_i \quad \text{for all } i \in I.
    \]
    Visually, the following diagrams commute every morphism $f: i \to j$ in $I$:
    \begin{center}
    \begin{tikzcd}[row sep=large, column sep=large]
        D(i) \arrow[rr, "D(f)"] \arrow[dr, "\iota_i"] \arrow[ddr, "g_i"', bend right=20] & & D(j) \arrow[dl, "\iota_j"'] \arrow[ddl, "g_j", bend left=20] \\
        & L \arrow[d, "\exists ! u", dashed] & \\
        & C &. 
    \end{tikzcd}
    \end{center}
    If such a cocone exists, then the object $L$ is necessarily unique up to unique isomorphism by the universal property. In this case, $L$ is denoted by \hl{$\colim_{i \in I} D$} or \hl{$\colim D$}.

\end{enumerate}

A limit/colimit is called \hldef{finite} (resp. \hldef{small}) if the diagram category $I$ is finite (resp. small).

Some authors use the terms \hldef{projective limit} or \hldef{inverse limit} to refer to what is defined here as a limit, Similarly, the terms \hldef{inductive limit} or \hldef{direct limit} are sometimes used to mean a colimit. However, these phrases can have more specific meanings to other authors: a \emph{projective} or \emph{inverse limit} may refer to a limit over a diagram indexed by a \hyperrefIfExists{definition:partially_ordered_set}{codirected poset}\CrefIfExists{definition:partially_ordered_set}. Likewise, an \emph{inductive} or \emph{direct limit} may refer to a colimit over a \hyperrefIfExists{definition:partially_ordered_set}{directed poset}\CrefIfExists{definition:partially_ordered_set}\TextIfExists{definition:projective_and_inductive_limits_in_categories}{ (see \Cref{definition:projective_and_inductive_limits_in_categories})}.

Thus, while the terms are sometimes used interchangeably with ``limit'' and ``colimit,'' they may also emphasize particular indexing shapes and directions, distinguishing them from general limits and colimits taken over arbitrary small categories.
\end{definition}

\begin{definition}[Product in a category] \label{definition:product_and_coproduct_of_objects_in_a_category}
Let $\mathcal{C}$ be a category and let $\{X_i\}_{i \in I}$ be a family of objects in $\mathcal{C}$ indexed by a class $I$. 
\begin{enumerate}
    \item A \hldef{product of the family $\{X_i\}$} is an object $P$ of $\mathcal{C}$ together with a ``universal'' family of morphisms
    $$\pi_i : P \to X_i, \quad \text{for each } i \in I. $$
    More precisely, for any object $Y$ and any family of morphisms $\{f_i : Y \to X_i\}_{i \in I}$, there exists a unique morphism
    $$f : Y \to P$$
    making the following diagram commute for all $i \in I$, i.e. $\pi_i \circ f = f_i$:
    \begin{center}
    \begin{tikzcd}[row sep=large, column sep=large]
        Y \arrow[d, "\exists ! f", dashed] \arrow[dr, "f_i"] & \\
        \prod X_i \arrow[r, "\pi_i"'] & X_i
    \end{tikzcd}
    \end{center}
    Such a product is often denoted by \hl{$\prod_{i \in I} X_i$}. If $\prod_{i \in I} X_i$ exists in $\calC$, then it is unique up to unique isomorphism by the universal property described above.
    
    Equivalently, the product $\prod_{i \in I} X_i$ is the \CrefAndHyperrefIfExist{definition:limit_and_colimit_of_a_diagram_in_a_category}{limit} of the \CrefAndHyperrefIfExist{definition:diagram_in_a_category_indexed_by_a_small_category}{diagram} $I \to \calC, i \mapsto X_i$, where $I$ is made into a category whose objects are the members of $I$ and whose morphisms are just the identity morphisms.


    \item A \hldef{coproduct} (or synonymously \hldef{direct sum}) of the family $\{X_i\}$ is an object $C$ of $\mathcal{C}$ together with a ``universal'' family of morphisms
    $$\iota_i : X_i \to C, \quad \text{for each } i \in I.$$
    More precisely, for any object $Y$ and any family of morphisms $\{g_i : X_i \to Y\}_{i \in I}$, there exists a unique morphism
    $$g : C \to Y$$
    making the following diagram commute for all $i \in I$, i.e. $g \circ \iota_i = g_i$:
    \begin{center}
    \begin{tikzcd}[row sep=large, column sep=large]
        X_i \arrow[r, "\iota_i"] \arrow[dr, "g_i"'] & \coprod X_i \arrow[d, "\exists ! g", dashed] \\
        & Y
    \end{tikzcd}
    \end{center}
    Such a coproduct is often denoted by \hl{$\coprod_{i \in I} X_i$} or \hl{$\oplus_{i \in I} X_i$}. If $\coprod_{i \in I} X_i$ exists in $\calC$, then it is unique up to unique isomorphism by the universal property described above.

    Equivalently, the coproduct $\coprod_{i \in I} X_i$ is the \CrefAndHyperrefIfExist{definition:limit_and_colimit_of_a_diagram_in_a_category}{colimit} of the \CrefAndHyperrefIfExist{definition:diagram_in_a_category_indexed_by_a_small_category}{diagram} $I \to \calC, i \mapsto X_i$, where $I$ is made into a category whose objects are the members of $I$ and whose morphisms are just the identity morphisms.
\end{enumerate}
\end{definition}

\begin{definition} \label{definition:adjoint_functors_between_categories_unit_counit_of_adjoint_functors}
Let $\mathcal{C}$ and $\mathcal{D}$ be \CrefAndHyperrefIfExist{definition:category}{categories}. Let $F : \mathcal{C} \to \mathcal{D}$ and $G : \mathcal{D} \to \mathcal{C}$ be functors. 

An \hldef{adjunction between $F$ and $G$} consists of two \CrefAndHyperrefIfExist{definition:natural_transformation_between_functors_between_categories}{natural transformations}: $\eta : \mathrm{Id}_{\mathcal{C}} \implies GF$ (the \hldef{unit}), and  $\varepsilon : FG \implies \mathrm{Id}_{\mathcal{D}}$ (the \hldef{counit})

These must satisfy the triangle identities: For every object $X \in \mathcal{C}$ 
and $Y \in \mathcal{D}$, 
$$\varepsilon_{FX} \circ F(\eta_X) = \text{id}_{FX}$$
$$G(\varepsilon_Y) \circ \eta_{GY} = \text{id}_{GY}.$$
In diagrammatic form, the triangle identities assert that the following are commutative diagrams:
\begin{center}
\begin{tikzcd}
F(X) \arrow[r, "F(\eta_X)"] \arrow[rd, "\text{id}_{F(X)}"'] & FGF(X) \arrow[d, "\varepsilon_{F(X)}"] \\
& F(X)
\end{tikzcd}
\begin{tikzcd}
G(Y) \arrow[r, "\eta_{G(Y)}"] \arrow[rd, "\text{id}_{G(Y)}"'] & GFG(Y) \arrow[d, "G(\varepsilon_Y)"] \\
& G(Y)
\end{tikzcd}
\end{center}

We say that $F$ is a \hldef{left adjoint to $G$} and $G$ is a \hldef{right adjoint to $F$} (written \hl{$F \dashv G$}). 

% for every object $A$ in $\mathcal{C}$ and $B$ in $\mathcal{D}$ there is a \CrefAndHyperrefIfExist{definition:natural_transformation_between_functors_between_categories}{natural isomorphism}
% \begin{align*}
% \operatorname{Hom}_{\mathcal{D}}(F(A), B) \cong \operatorname{Hom}_{\mathcal{C}}(A, G(B))
% \end{align*}
% that is natural in both $A$ and $B$.


In the case that $\mathcal{C}$ and $\mathcal{D}$ are \CrefAndHyperrefIfExist{definition:locally_small_category}{locally small} categories (or $U$-locally small categories if a \CrefAndHyperrefIfExist{definition:grothendieck_universe}{universe} $U$ is available), we have an adjunction $F \dashv G$ if and only if for every object $X$ in $\mathcal{C}$ and $Y$ in $\mathcal{D}$ there is a \CrefAndHyperrefIfExist{definition:natural_transformation_between_functors_between_categories}{natural isomorphism}
\begin{align*}
\operatorname{Hom}_{\mathcal{D}}(F(X), Y) \cong \operatorname{Hom}_{\mathcal{C}}(X, G(Y))
\end{align*}
that is natural in both $X$ and $Y$. In this case, the \hldef{unit of the adjunction} is the natural transformation $\eta : \mathrm{Id}_{\mathcal{C}} \Rightarrow G F$ such that, 
\begin{enumerate}
    \item for every $X \in \calC$, the morphism $\eta_X: X \to GF(X)$ (each called a \hldef{unit morphism}) in $\calC$ is obtained as the image of $\id_{F(X)}$ via the adjoint isomorphism
    $$\Hom_\calD(F(X), F(X)) \cong \Hom_\calC(X, GF(X)). $$

    \item for every $Y \in \calD$, the morphism $\epsilon_Y: FG(Y) \to Y$ (each called a \hldef{counit morphism}) in $\calD$ is obtained as the image of $\id_{G(Y)}$ via the adjoint isomorphism 
    $$\Hom_\calC(G(Y), G(Y)) \cong \Hom_\calD(FG(Y), Y).$$

\end{enumerate}


% Let $F : \mathcal{C} \to \mathcal{D}$ and $G : \mathcal{D} \to \mathcal{C}$ be functors. 
% $F$ is a \hldef{left adjoint to $G$} and $G$ is a \hldef{right adjoint to $F$} (written \hl{$F \dashv G$}) if for every object $A$ in $\mathcal{C}$ and $B$ in $\mathcal{D}$ there is a \CrefAndHyperrefIfExist{definition:natural_transformation_between_functors_between_categories}{natural isomorphism}
% \begin{align*}
% \operatorname{Hom}_{\mathcal{D}}(F(A), B) \cong \operatorname{Hom}_{\mathcal{C}}(A, G(B))
% \end{align*}
% that is natural in both $A$ and $B$.
\end{definition}


\begin{definition}[Monomorphism and Epimorphism in Categories] \label{definition:monomorphism_and_epimorphism_in_categories}
Let $\mathcal{C}$ be a \CrefAndHyperrefIfExist{definition:category}{category}. For objects $A, B \in \mathcal{C}$, let $f: A \to B$ be a morphism in $\mathcal{C}$.  
\begin{itemize}
    \item The morphism $f$ is called a \hldef{monomorphism} (or a \hldef{monic morphism}) if for every object $X$ and every pair of morphisms $g_1, g_2 : X \to A$, the equality $f \circ g_1 = f \circ g_2$ implies $g_1 = g_2$.  
    \item The morphism $f$ is called an \hldef{epimorphism} (or an \hldef{epic morphism}) if for every object $Y$ and every pair of morphisms $h_1, h_2: B \to Y$, the equality $h_1 \circ f = h_2 \circ f$ implies $h_1 = h_2$.  
\end{itemize}
\end{definition}



\begin{definition}[Category of objects over a fixed object] \label{definition:category_of_objects_over_under_a_fixed_object_in_a_category}
Let $\mathcal{C}$ be a \hyperrefIfExists{definition:category}{category}\CrefIfExists{definition:category} and let $X \in \operatorname{Ob}(\mathcal{C})$ be a fixed object.
\begin{enumerate}
    \item 
        The \hldef{category of objects over $X$} (or synonymously the \hldef{slice category of $X$ in $\calC$} or the \hldef{over category of $X$ in $\calC$}), commonly denoted \hl{$\mathcal{C}/X$}, \hl{$\mathcal{C}_{/X}$}, or \hl{$(\mathcal{C} \downarrow X)$} is the category defined as follows:
        \begin{itemize}
            \item An object of $\mathcal{C}/X$ is a morphism $f \colon A \to X$ in $\mathcal{C}$, where $A \in \operatorname{Ob}(\mathcal{C})$.
            \item A morphism from $f \colon A \to X$ to $g \colon B \to X$ in $\mathcal{C}/X$ is a morphism $h \colon A \to B$ in $\mathcal{C}$ such that the following diagram commutes:
            $$
            \begin{aligned}
            \xymatrix{
            A \ar[dr]_f \ar[r]^h & B \ar[d]^g \\
            & X
            }
            \end{aligned}
            $$
            i.e. such that $g \circ h = f$.
            \item The identity morphisms and composition in $\mathcal{C}/X$ are inherited from $\mathcal{C}$.
        \end{itemize}

    \item 
    The \hldef{category of objects under $X$} (or synonymously the \hldef{coslice category of $X$ in $\calC$} or the \hldef{under category of $X$ in $\calC$}), commonly denoted \hl{$X/\mathcal{C}$}, \hl{$X \backslash \calC$}, \hl{$\mathcal{C}_{X/}$}, or \hl{$(X \downarrow \calC)$}, is the category defined as follows:
    \begin{itemize}
        \item An object of $X/\mathcal{C}$ is a morphism $f \colon X \to A$ in $\mathcal{C}$, where $A \in \operatorname{Ob}(\mathcal{C})$.
        \item A morphism from $f \colon X \to A$ to $g \colon X \to B$ in $X/\mathcal{C}$ is a morphism $h \colon A \to B$ in $\mathcal{C}$ such that the following diagram commutes:
        $$
        \begin{aligned}
        \xymatrix{
        X \ar[dr]^g \ar[r]^f & A \ar[d]^h \\
        & B
        }
        \end{aligned}
        $$
        i.e. such that $h \circ f = g$.
        \item The identity morphisms and composition in $X/\mathcal{C}$ are inherited from $\mathcal{C}$.
    \end{itemize}

\end{enumerate} 
\TextIfExists{definition:comma_category_of_two_functors_to_a_category}{Both notions are special cases of \CrefAndHyperrefIfExist{definition:comma_category_of_two_functors_to_a_category}{comma categories}.}
\end{definition}


\begin{definition}[Category enriched in a monoidal category] \label{definition:category_enriched_in_a_monoidal_category}
Let $(\mathcal{V}, \otimes, \mathbf{1})$ be a \CrefAndHyperrefIfExist{definition:monoidal_category}{monoidal category}. A \hldef{category enriched in $\mathcal{V}$} (or a \hldef{$\mathcal{V}$-enriched category} or a \hldef{$\mathcal{V}$-category}) $\mathcal{C}$ consists of the following data:
\begin{itemize}
    \item A class \hl{$\operatorname{Ob}(\mathcal{C})$} of \hldef{objects}. As with \hyperrefIfExists{definition:category}{regular categories}, we may write \hl{$X \in \operatorname{Ob}(\mathcal{C})$} or \hl{$X \in \calC$} to mean that $X$ is an object of $\calC$.  
    \item For each pair of objects $X, Y \in \operatorname{Ob}(\mathcal{C})$, an object \hl{$\underline{\operatorname{Hom}}_{\mathcal{C}}(X,Y) \in \operatorname{Ob}(\mathcal{V})$} of \hldef{morphisms}; it is an object of the monoidal category $\mathcal{V}$. It is also often denoted by notations such as \hl{$\calC(X,Y)$}, \hl{$\Hom(X,Y) = \Hom_\calC(X,Y)$}, or \hl{$\operatorname{Mor}(X,Y) = \operatorname{Mor}_{\calC}(X,Y)$}.
    \item For each triple $X,Y,Z \in \operatorname{Ob}(\mathcal{C})$, a \hldef{composition morphism} 
    $$\mu_{X,Y,Z} : \underline{\operatorname{Hom}}_{\mathcal{C}}(Y,Z) \otimes \underline{\operatorname{Hom}}_{\mathcal{C}}(X,Y) \to \underline{\operatorname{Hom}}_{\mathcal{C}}(X,Z).$$
    It is a morphism in $\mathcal{V}$.
    \item For each object $X$, a \hldef{unit morphism} \hl{$\eta_X : \mathbf{1} \to \underline{\operatorname{Hom}}_{\mathcal{C}}(X,X)$} in $\mathcal{V}$.
\end{itemize}
These data satisfy the following axioms:
\begin{itemize}
    \item (Associativity) For all $W,X,Y,Z \in \operatorname{Ob}(\mathcal{C})$, the following diagram in $\mathcal{V}$ commutes:
    $$
    \begin{tikzcd}[column sep=large,row sep=large]
    \bigl(\underline{\operatorname{Hom}}_{\mathcal{C}}(Z,W) \otimes \underline{\operatorname{Hom}}_{\mathcal{C}}(Y,Z)\bigr) \otimes \underline{\operatorname{Hom}}_{\mathcal{C}}(X,Y) \ar[r,"\alpha"] \ar[d,"\mu \otimes \mathrm{id}"]
    & \underline{\operatorname{Hom}}_{\mathcal{C}}(Z,W) \otimes \bigl(\underline{\operatorname{Hom}}_{\mathcal{C}}(Y,Z) \otimes \underline{\operatorname{Hom}}_{\mathcal{C}}(X,Y)\bigr) \ar[d,"\mathrm{id} \otimes \mu"] \\
    \underline{\operatorname{Hom}}_{\mathcal{C}}(Y,W) \otimes \underline{\operatorname{Hom}}_{\mathcal{C}}(X,Y) \ar[d,"\mu"] 
    & \underline{\operatorname{Hom}}_{\mathcal{C}}(Z,W) \otimes \underline{\operatorname{Hom}}_{\mathcal{C}}(X,Z) \ar[d,"\mu"] \\
    \underline{\operatorname{Hom}}_{\mathcal{C}}(X,W) \ar[r,equal] & \underline{\operatorname{Hom}}_{\mathcal{C}}(X,W)
    \end{tikzcd}
    $$
    where $\alpha$ is the associativity constraint in $\mathcal{V}$.
    \item (Unit) For all $X,Y \in \operatorname{Ob}(\mathcal{C})$, the following diagrams commute:
    \begin{center}
    \begin{tikzcd}[column sep=large]
    \mathbf{1} \otimes \underline{\operatorname{Hom}}_{\mathcal{C}}(X,Y) \ar[r,"\eta_Y \otimes \mathrm{id}"] \ar[dr,"\lambda"']
    & \underline{\operatorname{Hom}}_{\mathcal{C}}(Y,Y) \otimes \underline{\operatorname{Hom}}_{\mathcal{C}}(X,Y) \ar[d,"\mu"] \\
    & \underline{\operatorname{Hom}}_{\mathcal{C}}(X,Y)
    \end{tikzcd}
    \begin{tikzcd}[column sep=large]
    \underline{\operatorname{Hom}}_{\mathcal{C}}(X,Y) \otimes \mathbf{1} \ar[r,"\mathrm{id} \otimes \eta_X"] \ar[dr,"\rho"']
    & \underline{\operatorname{Hom}}_{\mathcal{C}}(X,Y) \otimes \underline{\operatorname{Hom}}_{\mathcal{C}}(X,X) \ar[d,"\mu"] \\
    & \underline{\operatorname{Hom}}_{\mathcal{C}}(X,Y)
    \end{tikzcd}
    \end{center}
    % $$
    % \quad\quad
    % $$
    where $\lambda$ and $\rho$ are the left and right unit constraints in $\mathcal{V}$.
\end{itemize}
\end{definition}
\begin{definition}[n-ary (Multivariable) Functor] \label{definition:n_ary_functor}
Let $I$ be a finite set with $|I| = n$, and let $\{\mathcal{C}_i\}_{i \in I}$ be \CrefAndHyperrefIfExist{definition:category}{(large) categories}, together with another category $\mathcal{D}$. An \hldef{n-ary functor} (also called a \hldef{multivariable functor}, a \hldef{multivariate functor}, or a \hldef{multifunctor} ) from the categories $\{\mathcal{C}_i\}_{i\in I}$ to $\mathcal{D}$ is a \CrefAndHyperrefIfExist{definition:functor_between_categories}{functor}
$$F : \prod_{i \in I} \mathcal{C}_i \to \mathcal{D}.$$ 
\CrefIfExists{definition:product_category_of_a_family_of_categories} That is, $F$ assigns:
\begin{itemize}
    \item to each object $((A_i)_{i \in I})$ in $\prod_{i \in I} \mathcal{C}_i$, an object $F((A_i)_{i \in I})$ in $\mathcal{D}$,
    \item to each morphism $((f_i)_{i \in I}) : (A_i)_i \to (B_i)_i$, a morphism $F((f_i)_i) : F((A_i)_i) \to F((B_i)_i)$ in $\mathcal{D}$,
\end{itemize}
so that $F$ preserves identities and composition componentwise. 
For instance, a \hldef{bifunctor} is an $n$-ary functor when $n = 2$, a \hldef{ternary functor/trifunctor} is an $n$-ary functor when $n = 3$, etc.

\end{definition}


\begin{definition}[Localization by a multiplicative system] \label{definition:localization_of_a_category_by_a_multiplicative_system}
    Given a category $\mathcal{C}$ and a \hyperrefIfExists{definition:multiplicative_system_of_morphisms_in_a_category}{multiplicative system}\CrefIfExists{definition:multiplicative_system_of_morphisms_in_a_category} $S \subseteq \mathrm{Mor}(\mathcal{C})$, the \hldef{localization} \hl{$\mathcal{C}[S^{-1}]$} is a category equipped with a functor
        $$\hlin{Q: \mathcal{C} \to \mathcal{C}[S^{-1}]}$$
    that sends every morphism in $S$ to an isomorphism satisfying the universal property: any functor from $\mathcal{C}$ sending the morphisms in $S$ to isomorphisms factors uniquely through $Q$.
\end{definition}



\begin{proposition} \label{proposition:sections_functors_on_presheaves_vlaued_in_an_abelian_category_are_left_exact}
Let $\calC$ be an \CrefAndHyperrefIfExist{definition:essentially_small_category}{essentially small category} and let $\mathcal{A}$ be an \CrefAndHyperrefIfExist{definition:abelian_category}{abelian category}. 
\begin{enumerate}
    \item Let $U \in \Ob(\calC)$ be some fixed object. The \CrefAndHyperrefIfExist{definition:sections_of_a_presheaf_on_a_category_valued_in_a_category}{sections functor}
    $$\Gamma(U, -) : \mathrm{PSh}(\mathcal{C}, \mathcal{A}) \to \mathcal{A}$$ 
    is \CrefAndHyperrefIfExist{definition:exact_functor_between_abelian_categories}{left exact}. \TODO{state that presheaves and sheaves valued in an abelian category form abelian categories}

    \item Assume that $\Gamma(\calF)$ exists for all $\calF$ in $\mathrm{PSh}(\calC, \calA)$ so that $\Gamma$ is a functor 
    $$\Gamma: \mathrm{PSh}(\mathcal{C}, \calA) \to \calA.$$
    The functor $\Gamma$ is left exact.

\end{enumerate}
\end{proposition}
\begin{proof}
    Recall that $\mathrm{PSh}(\mathcal{C}, \mathcal{D})$ is an abelian category (\Cref{proposition:examples_of_abelian_categories}) since $\calC$ is essentially small.  \TODO{Talk about how limits are left exact and }
\end{proof}
\begin{theorem}[e.g. see {\cite[Tag 01DU]{stacks-project}}] \label{theorem:category_of_modules_over_a_sheaf_of_rings_on_a_site_on_an_essentially_small_category_has_enough_injectives}
    For any \CrefAndHyperrefIfExist{definition:grothendieck_topology_on_a_category_site_covering_sieve_topologically_generating_family}{site} $(\calC, J)$ on an \CrefAndHyperrefIfExist{definition:essentially_small_category}{essentially small category} $\mathcal{C}$ and a \CrefAndHyperrefIfExist{definition:sheaf_on_a_site}{sheaf of rings} $\calO$ on $\calC$, the category $\mathbf{Mod}(\mathcal{O})$ of \CrefAndHyperrefIfExist{definition:module_over_a_sheaf_of_rings_on_a_site}{$\calO$-modules} is an abelian category that \CrefAndHyperrefIfExist{definition:has_enough_injectives_or_projectives_for_an_abelian_category}{has enough injectives}. In fact, there is a functorial injective embedding \TODO{ what does this mean?}
\end{theorem}
\begin{definition} \label{definition:sheaf_cohomology_group_of_a_sheaf_of_modules_over_a_sheaf_of_rings_on_a_site}

Let $(\mathcal{C}, J)$ be a \CrefAndHyperrefIfExist{definition:grothendieck_topology_on_a_category_site_covering_sieve_topologically_generating_family}{site} on a \CrefAndHyperrefIfExist{definition:locally_small_category}{locally small category} or a $U$-site for some \CrefAndHyperrefIfExist{definition:grothendieck_universe}{universe} $U$. Let $\calO$ be a \CrefAndHyperrefIfExist{definition:sheaf_on_a_site}{sheaf of rings} on $\calC$, so that $(\calC, J, \calO)$ is a \CrefAndHyperrefIfExist{definition:ringed_site}{ringed site}. Recall that the category \CrefAndHyperrefIfExist{definition:module_over_a_sheaf_of_rings_on_a_site}{$\mathbf{Mod}(\mathcal{O})$} of $\calO$-modules is abelian and \CrefAndHyperrefIfExist{definition:has_enough_injectives_or_projectives_for_an_abelian_category}{has enough injectives} (\Cref{theorem:category_of_modules_over_a_sheaf_of_rings_on_a_site_on_an_essentially_small_category_has_enough_injectives}).

Assume that \CrefAndHyperrefIfExist{definition:sections_of_a_presheaf_on_a_category_valued_in_a_category}{global sections objects $\Gamma(\calG)$} exist for all objects $\calG$ of $\mathrm{Sh}(\mathcal{C}, \mathbf{Ab})$\footnote{for example, this occurs when $\calC$ is \CrefAndHyperrefIfExist{definition:essentially_small_category}{essentially small}} so that $\Gamma$ is a functor
$$\Sh(\calC, \mathbf{Ab}) \to \mathbf{Ab},$$
which is a \CrefAndHyperrefIfExist{definition:exact_functor_between_abelian_categories}{left exact functor} (\Cref{proposition:sections_functors_on_presheaves_vlaued_in_an_abelian_category_are_left_exact}).  Note that $\Gamma$ restricts to a left exact functor 
$$\mathbf{Mod}(\mathcal{O}) \to \mathbf{Ab}.$$
If $\calC$ has a \CrefAndHyperrefIfExist{definition:initial_final_zero_objects_of_a_category}{final object} $\ast$ as well, then recall that $\Gamma(\calF) = \calF(\ast)$. 

Let $\calF$ be an object of $\mathbf{Mod}(\mathcal{O})$. 
\begin{enumerate}
    \item For each integer $n \geq 0$, the \hldef{$n$-th (abelian) (global) sheaf cohomology group of $\mathcal{F}$} is
    $$\hlin{H^n(\mathcal{C}, J; \mathcal{F}) := R^n \Gamma(\mathcal{F}),}$$
    where $R^n \Gamma$ is the $n$-th \CrefAndHyperrefIfExist{definition:left_right_derived_functors_of_a_right_left_exact_functor_between_abelian_categories_where_source_has_enough_projectives_injectives}{right derived functor} of the \CrefAndHyperrefIfExist{definition:sections_of_a_presheaf_on_a_category_valued_in_a_category}{global sections functor $\Gamma$}.

    In particular, each $H^n$ is a functor 
    $$H^n: \mathbf{Mod}(\mathcal{O})  \to \mathbf{Ab}.$$

    \item Given an object $U \in \calC$ and for each integer $n \geq 0$, the \hldef{$n$-th (abelian) sheaf cohomology group of $\mathcal{F}$ of sections at $U$} is
    $$\hlin{H^n(U, \mathcal{F}) := (R^n \Gamma(U,-))(\mathcal{F}),}$$
    where $R^n \Gamma(U,-)$ is the $n$-th \CrefAndHyperrefIfExist{definition:left_right_derived_functors_of_a_right_left_exact_functor_between_abelian_categories_where_source_has_enough_projectives_injectives}{right derived functor} of the \CrefAndHyperrefIfExist{definition:sections_of_a_presheaf_on_a_category_valued_in_a_category}{sections functor $\Gamma(U,-)$ evaluated at $U$}.

    In particular, $H^n(U,\calF)$ can be regarded as the $n$th global sheaf cohomology group of the \CrefAndHyperrefIfExist{definition:restriction_of_a_sheaf_on_a_site_to_an_object_of_the_underlying_category_of_the_site}{restriction $\calF|_U$} of $\calF$ to $U$.
    % to the 
    % \CrefAndHyperrefIfExist{definition:site_induced_by_a_site_on_an_over_category}{site induced by $(\calC, J)$} on the \CrefAndHyperrefIfExist{definition:category_of_objects_over_under_a_fixed_object_in_a_category}{over category $\calC_{/U}$}.

\end{enumerate}

% In the case that $\calA = R\mathbf{-mod}$, the category of (left/right/two-sided)modules over some fixed (not necessarily commutative) ring $R$, recall that $R\mathbf{-Mod}$ is complete (and cocomplete) \TODO{Is it precisely completeness that I need or cocompleteness? In other words, for the limits defining $\Gamma$, are those limits projective limits or colimtis?} \TODO{talk about how $R$-modules are complete and cocomplete}, so all global sections objects $\Gamma(\calG)$ exist. \TODO{continue talking about this context of modules}

% The \hldef{sheaf cohomology groups of $\mathcal{F}$} on the site $(\mathcal{C}, J)$ are defined as the \CrefAndHyperrefIfExist{definition:left_right_derived_functors_of_a_right_left_exact_functor_between_abelian_categories_where_source_has_enough_projectives_injectives}{right derived functors} of the global sections functor
% $$\Gamma : \mathrm{Sh}(\mathcal{C}, J) \to \mathbf{Ab}$$
% where $\mathrm{Sh}(\mathcal{C}, J)$ \CrefAndHyperrefIfExist{definition:sheaf_on_a_site}{denotes} the category of sheaves of abelian groups on $(\mathcal{C}, J)$, and $\ast$ denotes the \CrefAndHyperrefIfExist{definition:initial_final_zero_objects_of_a_category}{final object} in $\mathcal{C}$ if it exists.

% More precisely, 

% If $\mathcal{C}$ has no final object, $H^n(\mathcal{C}, J; \mathcal{F})$ is defined by choosing an injective resolution of $\mathcal{F}$ and taking cohomology of the resulting complex obtained by applying $\Gamma$.

% These groups measure the extent to which global sections fail to be exact, and generalize classical sheaf cohomology defined on topological spaces to arbitrary sites.
\end{definition}

\begin{theorem}{cf. {\cite[Expos\'e II, Th\'eor\`eme 3.4]{SGA4_I}}} \label{theorem:sheafification_of_a_presheaf_of_sets_on_a_small_enough_site}
    \begin{enumerate}
        \item Let $U$ be a universe. Let $\calC$ be a \hyperrefIfExists{definition:grothendieck_topology_on_a_category_site_covering_sieve_topologically_generating_family}{$U$-site}\CrefIfExists{definition:grothendieck_topology_on_a_category_site_covering_sieve_topologically_generating_family}. A \CrefAndHyperrefIfExist{definition:sheafification_of_a_presheaf_on_a_topological_space_valued_in_a_category_admitting_direct_colimits}{sheafification functor}
        $$a: \Shv(\calC, \USets) \to \PreShv(\calC, \USets).$$
        exists. 
        % The inclusion functor 
        % $$i: \PreShv(\calC, \USets) \hookrightarrow \Shv(\calC, \USets)$$
        % has a \hyperrefIfExists{definition:adjoint_functors_between_categories_unit_counit_of_adjoint_functors}{left adjoint functor}\CrefIfExists{definition:adjoint_functors_between_categories_unit_counit_of_adjoint_functors}

        \item Let $\calC$ be a site whose underlying category is \CrefAndHyperrefIfExist{definition:locally_small_category}{locally small} and which has a \CrefAndHyperrefIfExist{definition:grothendieck_topology_on_a_category_site_covering_sieve_topologically_generating_family}{topologically generating family} that is a set (rather than a proper class). A sheafification functor 
        $$a: \Shv(\calC, \Sets) \to \PreShv(\calC, \Sets)$$
        exists.

        \item (see e.g. {\cite[3]{nlab:sheafification}}) Let $(\calC, J)$ be a \CrefAndHyperrefIfExist{definition:grothendieck_topology_on_a_category_site_covering_sieve_topologically_generating_family}{site} on an \CrefAndHyperrefIfExist{definition:essentially_small_category}{essentially small category} $\calC$. Suppose that the category $\calA$ is \CrefAndHyperrefIfExist{definition:complete_and_cocomplete_category}{complete, cocomplete}, that small \CrefAndHyperrefIfExist{definition:projective_and_inductive_limits_in_categories}{filtered colimits} in $\calA$ are exact, and that $\calA$ satisfies the IPC-property. A \CrefAndHyperrefIfExist{definition:sheafification_functor_on_a_site}{sheafification functor} 
        $$a: \PreShv(\calC, \calA) \to \Shv(\calC, \calA) $$
        exists.
        \TODO{IPC-property, exactess in this context.}

        \TODO{state as a fact that these categories are complete, cocomplete, with small filtered colimits that are exact}
        This is true for instance of $\calA = \mathbf{Set}, \mathbf{Grp}$, $k-\mathbf{Alg}$ for a field $k$, or $\mathbf{Mod}_R$ for a \CrefAndHyperrefIfExist{definition:ring}{(not necessarily commutative unital) ring $R$}.
    \end{enumerate}
\end{theorem}
\begin{remark}
    If the presheaf is valued in nice ``algebraic category'', e.g. groups, abelian groups, rings, modules over a ring, etc., then the sheafification is again valued in that category. \TODO{Make this more precise.}
\end{remark}

\begin{definition}[Constant sheaf on a site] \label{definition:constant_sheaf_on_a_site_with_sheafification}
    Let $\calC$ be a \hyperrefIfExists{definition:category}{(large) category}\CrefIfExists{definition:category}, let $\calA$ be a (large category), and let $A$ be an object of $\calA$. %a set (or more generally, an abelian group, ring, etc.).
    
    \begin{enumerate}
        \item The \hldef{constant presheaf on $\calC$ with value $A$} is the \hyperrefIfExists{definition:presheaf_on_a_category}{presheaf}\CrefIfExists{definition:presheaf_on_a_category} $P$ defined by
        \[
        P(U) = A
        \]
        for every object $U$ of $\calC$ such that every morphism $f: V \to U$ in $\calC$ induces the identity map $A = P(U)\to P(V) = A$. 

        \item Let $\calC$ be a \CrefAndHyperrefIfExist{definition:grothendieck_topology_on_a_category_site_covering_sieve_topologically_generating_family}{site} and assume that a \CrefAndHyperrefIfExist{definition:sheafification_functor_on_a_site}{sheafification functor} 
        $$a: \Shv(\calC, \calA) \to \PreShv(\calC, \calA)$$
        exists\TextIfExists{theorem:sheafification_of_a_presheaf_of_sets_on_a_small_enough_site}{~(e.g. see \Cref{theorem:sheafification_of_a_presheaf_of_sets_on_a_small_enough_site})}.
        The \hldef{constant sheaf on $\calC$ with value $A$}, or the \hldef{constant sheaf on $\calC$ associated to $A$} commonly denoted \hl{$\underline{A}$} or sometimes just \hl{$A$} by abuse of notation, is the \hyperrefIfExists{theorem:sheafification_of_a_presheaf_of_sets_on_a_small_enough_site}{sheaf associated to}\CrefIfExists{theorem:sheafification_of_a_presheaf_of_sets_on_a_small_enough_site} the constant presheaf $P$ with value $A$ above.

        \item Let $\calC$ be a site. Let $\calO$ be a sheaf of (not-necessarily commutative) rings on $\calC$. Assume that the \CrefAndHyperrefIfExist{definition:sections_of_a_presheaf_on_a_category_valued_in_a_category}{global sections ring $\Gamma(\calO)$} exists. A \hldef{constant $\calO$-module} is an \CrefAndHyperrefIfExist{definition:module_over_a_sheaf_of_rings_on_a_site}{$\calO$-module} $\calF$ which is isomorphic as a sheaf to the constant sheaf on $\calC$ with value $M$ where $M$ is a module of the ring $\Gamma(\calO)$. Note that sheafification functors exist for presheaves/sheaves valued in $\Ab$ (\Cref{theorem:sheafification_of_a_presheaf_of_sets_on_a_small_enough_site}).

        In case that $\calO$ is the constant sheaf associated to $A$ for some (not-necessarily commutative) ring $A$, a constant $\calO$-module is simply called a \hldef{constant $A$-module}.
    \end{enumerate}
\end{definition}


\begin{definition}[Nonabelian Sheaf Cohomology: $H^0$ and $H^1$] \label{definition:zeroth_and_first_nonabelian_sheaf_cohomology_of_a_sheaf_of_groups_on_a_site}
   Let $(\mathcal{C},J)$ be a \CrefAndHyperrefIfExist{definition:grothendieck_topology_on_a_category_site_covering_sieve_topologically_generating_family}{site} and $\mathcal{G}$ a \CrefAndHyperrefIfExist{definition:sheaf_on_a_site}{sheaf} of groups on $(\mathcal{C},J)$. For $U \in \mathcal{C}$, we define the \hldef{$0$th and $1$st (nonabelian) sheaf cohomology sets} as follows:

        \begin{itemize}
        \item The zeroth sheaf cohomology group \hl{$\displaystyle H^0(\calC, J, \mathcal{G}) := \Gamma(\calG)$} is the group of \CrefAndHyperrefIfExist{definition:sections_of_a_presheaf_on_a_category_valued_in_a_category}{global sections} of $\calG$, assuming that it exists (which is always the case when $\calC$ is \CrefAndHyperrefIfExist{definition:essentially_small_category}{essentially small} for example since the category of groups is closed under small projective limits \TODO{}).

        \item The first sheaf cohomology set \hl{$\displaystyle H^1(\calC, J, \mathcal{G})$} is the pointed set of isomorphism classes of \CrefAndHyperrefIfExist{definition:torsor_principal_homogeneous_space_of_a_sheaf_of_groups_on_a_site_over_an_object}{$\mathcal{G}$-torsors} on the site $(\mathcal{C}, J)$.
        % , where a \hldef{$\mathcal{G}$-torsor} is a sheaf $\mathcal{P}$ on $(\mathcal{C}/U,J|_{U})$ with a free and transitive right action of $\mathcal{G}|_{U}$ such that there exists a covering $\{V_i \to U\}$ with $\mathcal{P}(V_i) \neq \emptyset$.
        \end{itemize}
\end{definition}
\begin{definition}[Slice site] \label{definition:site_induced_by_a_site_on_an_over_category}
Let $(\mathcal{C}, \tau)$ be a \CrefAndHyperrefIfExist{definition:grothendieck_topology_on_a_category_site_covering_sieve_topologically_generating_family}{site}, where $\tau$ is a Grothendieck topology on the (\CrefAndHyperrefIfExist{definition:locally_small_category}{locally small or $U$-locally small}, if a \CrefAndHyperrefIfExist{definition:grothendieck_universe}{universe} $U$ is available) category $\mathcal{C}$. For a fixed object $X$ in $\mathcal{C}$, the \hldef{slice site} (or the \hldef{over site}, the \hldef{site on the slice category $\mathcal{C}_{/X}$}, the \hldef{site induced on the over category $\mathcal{C}_{/X}$}, the \hldef{localization of the site $\calC$ at the object $X$}, etc.) $(\mathcal{C}_{/X}, \tau_{/X})$ is the site whose underlying category is the \CrefAndHyperrefIfExist{definition:category_of_objects_over_under_a_fixed_object_in_a_category}{slice category $\mathcal{C}_{/X}$}, and whose Grothendieck topology \hl{$\tau_{/X}$} (also denoted by notations such as \hl{$\tau|_{X}$} or \hl{$\tau/X$}) is defined by declaring a family of morphisms $\{f_i : Y_i \to Y\}$ in $\mathcal{C}_{/X}$ to be a covering if and only if the family $\{f_i : Y_i \to Y\}$ is a covering in $(\mathcal{C}, \tau)$.

% The forgetful functor 
% $$\hlin{j_X: \calC/X \to \calC}$$
% is \CrefAndHyperrefIfExist{definition:continuous_cocontinuous_functor_between_categories}{cocontinuous and continuous} 

\end{definition}
\begin{proposition}{See {\cite[Tag 03AJ]{stacks-project}} for a statement\footnote{While the Stacks Project defines a site as a category eqiupped with a Grothendieck \emph{pre}topology, its proof of this statement should be applicable for more general sites}} \label{proposition:zeroth_and_first_nonabelian_and_abelian_sheaf_cohomologies_of_a_sheaf_of_abelian_groups_on_}
    Let $(\calC, J)$ be a \CrefAndHyperrefIfExist{definition:grothendieck_topology_on_a_category_site_covering_sieve_topologically_generating_family}{site}. 
    Assume that \CrefAndHyperrefIfExist{definition:sections_of_a_presheaf_on_a_category_valued_in_a_category}{global sections objects $\Gamma(\calG)$} exist for all objects $\calG$ of $\mathrm{Sh}(\mathcal{C}, \mathbf{Ab})$\footnote{for example, this occurs when $\calC$ is \CrefAndHyperrefIfExist{definition:essentially_small_category}{essentially small}} so that $\Gamma$ is a functor
    $$\Sh(\calC, \mathbf{Ab}) \to \mathbf{Ab}.$$

    Let $\calG$ be a \CrefAndHyperrefIfExist{definition:sheaf_on_a_site}{sheaf} of abelian groups on $(\calC, J)$. For $i = 0,1$, there is a canonical bijection between the set of isomorphism classes of \CrefAndHyperrefIfExist{definition:torsor_principal_homogeneous_space_of_a_sheaf_of_groups_on_a_site_over_an_object}{$\calG$-torsors} and the \CrefAndHyperrefIfExist{definition:sheaf_cohomology_group_of_a_sheaf_of_modules_over_a_sheaf_of_rings_on_a_site}{(abelian) sheaf cohomology group $H^i(\calC, J; \calG)$}. In other words, for $i = 0,1$, there is a canonical bijection between the \CrefAndHyperrefIfExist{definition:zeroth_and_first_nonabelian_sheaf_cohomology_of_a_sheaf_of_groups_on_a_site}{$i$th nonabelian sheaf cohomology} and the \CrefAndHyperrefIfExist{definition:sheaf_cohomology_group_of_a_sheaf_of_modules_over_a_sheaf_of_rings_on_a_site}{$i$th abelian sheaf cohomology} of $\calG$. 
\end{proposition}

\begin{proof}
    In the case of $i = 0$, both are identifiable with the groups of global sections. In the case of $i = 1$, see {\cite[Tag 03AJ]{stacks-project}}. 
\end{proof}


\section{points of topoi}

We work in a fixed universe.

\begin{definition}
    % \begin{definition}[Topos] \label{definition:topos}
%     There are a multitude of notions of topos. Here are some that we consider; more notions may be added later.
%     \begin{enumerate}
%         \item A \hldef{(sheaf/Grothendieck) topos} is a \CrefAndHyperrefIfExist{definition:category}{category} \CrefAndHyperrefIfExist{definition:equivalence_of_categories}{equivalent} to the category of \CrefAndHyperrefIfExist{definition:sheaf_on_a_site}{sheaves} of sets on some \CrefAndHyperrefIfExist{definition:grothendieck_topology_on_a_category_site_covering_sieve_topologically_generating_family}{site}. That is, there exists a site $(C, J)$ such that the category is equivalent to $\operatorname{Sh}(C, J)$, the category of sheaves of sets on $(C, J)$.
%         \item Let $U$ be a universe. A \hldef{$U$-(sheaf )topos} is a category equivalent to the category of \hyperrefIfExists{definition:sheaf_on_a_site}{$U$-sheaves}\CrefIfExists{definition:sheaf_on_a_site} (valued in $U$-sets) \cite[Expos\'e IV D\'efinition 1.1]{SGA4_I}

%         \item An \hldef{elementary topos} is a cateogry which has all finite \CrefAndHyperrefIfExist{definition:limit_and_colimit_of_a_diagram_in_a_category}{limits}, is cartesian closed, and has a subobject classifier \TODO{cartesian closed, subobject classifier}
%     \end{enumerate}
% \end{definition}

\begin{definition}[Topos] \label{definition:topos}
    There are multiple notions of a topos depending on the context (geometric vs. logical).
    \begin{enumerate}
        \item A \hldef{Grothendieck topos} (or \hldef{sheaf topos}) is a \CrefAndHyperrefIfExist{definition:category}{category} \CrefAndHyperrefIfExist{definition:equivalence_of_categories}{equivalent} to the category of \CrefAndHyperrefIfExist{definition:sheaf_on_a_site}{sheaves} of sets on a \hldef{small} \CrefAndHyperrefIfExist{definition:grothendieck_topology_on_a_category_site_covering_sieve_topologically_generating_family}{site}. That is, there exists a small site $(\mathcal{C}, J)$ such that the category is equivalent to $\operatorname{Sh}(\mathcal{C}, J)$.
        
        \item Let $\mathscr{U}$ be a \hyperrefIfExists{definition:grothendieck_universe}{universe}\CrefIfExists{definition:grothendieck_universe}. A \hldef{$\mathscr{U}$-topos} is a category equivalent to the category of sheaves of sets on a $\mathscr{U}$-small site $(\mathcal{C}, J)$, where the sheaves take values in the category of $\mathscr{U}$-sets ($\mathbf{Set}_{\mathscr{U}}$). \cite[Expos\'e IV D\'efinition 1.1]{SGA4_I}

        \item An \hldef{elementary topos} is a category which has all finite \CrefAndHyperrefIfExist{definition:limit_and_colimit_of_a_diagram_in_a_category}{limits}, is \CrefAndHyperrefIfExist{definition:cartesian_closed_category}{cartesian closed}, and has a \CrefAndHyperrefIfExist{definition:subobject_classifier_in_a_category_with_a_final_object}{subobject classifier}.
    \end{enumerate}
    \textit{Remark:} Every Grothendieck topos is an elementary topos, but the converse is not true (e.g., the category of finite sets is an elementary topos but not a Grothendieck topos).
\end{definition}


% {\cite[Expos\'e IV D\'efinition 1.1]{SGA4_I}}
% Let $\scrU$ be a fixed universe. A \hldef{$\scrU$-topos}, or simply \hldef{topos} if there is no confusion, $E$ is a category that is equivalent to the category $\Shv(T)$ of sheaves of sets on a fixed site $T$ in $\scrU$.
\end{definition}

\begin{proposition}[See {\cite[Expos\'e II Proposition 4.8]{SGA4_I}}] \label{proposition:category_of_sheaves_is_closed_under_finite_projective_limits_and_arbitrary_corpoducts}
    Let $\scrU$ be a universe and let $T$ be a $\scrU$-site. The category $\Shv(T)$ of sheaves on $T$ has the following properties:
    \begin{enumerate}
        \item $\Shv(T)$ is closed under finite projective limits.
        \item $\Shv(T)$ is closed under arbitrary coproducts (indexed by $\scrU$-small set).
    \end{enumerate}
\end{proposition}

\begin{lemma} \label{lemma:final_object_is_projective_limit_of_empty_diagram}
    Let $C$ be a \CrefAndHyperrefIfExist{definition:category}{category}. A \CrefAndHyperrefIfExist{definition:initial_final_zero_objects_of_a_category}{final object}, if it exists, of $C$ is the \CrefAndHyperrefIfExist{definition:limit_and_colimit_of_a_diagram_in_a_category}{limit} of the empty diagram. In particular, any category that is \CrefAndHyperrefIfExist{definition:complete_and_cocomplete_category}{closed} under finite limits or \CrefAndHyperrefIfExist{definition:product_and_coproduct_of_objects_in_a_category}{finite products} has a final object.
\end{lemma}
\begin{proof}
    This follows by considering the universal property of the final object.
\end{proof}

\begin{lemma} \label{lemma:topos_has_a_final_object}
    Let $\scrU$ be a universe and let $T$ be a $\scrU$-site. The category $\Shv(T)$ of sheaves on $T$ has a final object.
\end{lemma}
\begin{proof}
    This follows from Lemma \ref{lemma:final_object_is_projective_limit_of_empty_diagram} and Proposition \ref{proposition:category_of_sheaves_is_closed_under_finite_projective_limits_and_arbitrary_corpoducts}.
\end{proof}

\begin{definition}
    {\cite[Expos\'e IV D\'efinition 3.1]{SGA4_I}, see \cite[Definition 2.1]{nlab:geometric_morphism}}
    Let $E$ and $E'$ be topoi (in a universe $\scrU$). A \hldef{(continuous) morphism $f: E \to E'$} is a triple \hl{$f = (f_*, f^*, \varphi)$} consisting of functors
    $$f_*: E \to E', \quad f^*: E' \to E$$
    and an \hldef{adjunction isomorphism}
    $$\varphi: \Hom_E(u^*(X'), Y) \xrightarrow{\sim} \Hom_{E'}(X', u_*(Y))$$
    of \CrefAndHyperrefIfExist{definition:n_ary_functor}{bifunctors} in $X' \in \Ob E'$ and $Y \in \Ob E$ such that $f^*$ commutes with finite (projective) limits.
    The functors $f_*$ and $f^*$ are respectively called the \hldef{direct image functor of $f$} and the \hldef{inverse image functor of $f$}.

    This notion of morphism of topoi is also referred to as \hldef{geometric morphism of topoi}.
    
\end{definition}

Note that $\Sets$ is a topos by virtue of being the category of sheaves on the site consisting of a single point.

\begin{definition}[{\cite[Definition 1.1]{nlab:point_of_a_topos}}] \label{definition:point_of_a_topos}
    A \hldef{point of a topos $E$} is a \CrefAndHyperrefIfExist{definition:geometric_morphism_between_topoi}{geometric morphism} $x: \Sets \to E$\CrefIfExists{definition:category_of_sets}. 

    For an object $A \in E$, its inverse image $x^* A \in \Sets$ under such a point $x$ is the \hldef{stalk of $A$ at $x$ of $A$ at $x$}.
\end{definition}
 



\begin{definition}[{\cite[Definition 1.3]{nlab:point_of_a_topos}}] \label{definition:site_with_enough_points}
    A topos $E$ is said to have \hldef{enough points} if for any morphism $f: A \to B$ in $E$, the following are equivalent:
    \begin{enumerate}
        \item $f$ is an isomorphism.
        \item for every geometric point $p: \Sets \to E$, the morphism of stalks $p^* f: p^*A \to p^* B$ is an isomorphism (of sets)
    \end{enumerate}
    A \CrefAndHyperrefIfExist{definition:grothendieck_topology_on_a_category_site_covering_sieve_topologically_generating_family}{site} $T$ is said to have \hldef{enough points} if $\Shv(T)$ has enough points.
\end{definition}

\begin{lemma}
Let $\scrU$ be a universe. Let $E$ be a $\scrU$-topos. Write $e_E$ for the final object of $E$, which exists by Lemma \ref{lemma:topos_has_a_final_object}. Given a point $x: \Sets \to E$,     
\TODO{TODO}
\end{lemma}

\TODO{TODO: understand the relationship betewen big/small topologies}


\section{Topology}


\begin{definition}[Topology] \label{definition:topological_space}
Let $X$ be a set. A \hldef{topology on $X$} is a collection $\mathcal{T}$ of subsets of $X$ such that:
\begin{enumerate}
    \item $\emptyset \in \mathcal{T}$ and $X \in \mathcal{T}$,
    \item For any collection $\{ U_i \}_{i \in I} \subseteq \mathcal{T}$ (with $I$ arbitrary), the union $\bigcup_{i \in I} U_i \in \mathcal{T}$,
    \item For any finite collection $\{ U_1, \ldots, U_n \} \subseteq \mathcal{T}$, the intersection $U_1 \cap \cdots \cap U_n \in \mathcal{T}$.
\end{enumerate}
If $\mathcal{T}$ is a topology on $X$, the pair $(X, \mathcal{T})$ is called a \hldef{topological space}. Members of $\mathcal{T}$ are called \hldef{open sets}. 

A subset $C \subseteq X$ is \hldef{closed} if its complement $X \setminus C$ is an open set in $\mathcal{T}$

One very often refers to $X$ as a topological spcae, omitting the notation of the topology $\mathcal{T}$. 

The collection of all topologies on a set $X$ may be denoted by notations such as \hl{$\mathrm{Top}(X)$}, \hl{$\mathbf{Top}(X)$}, or \hl{$\mathsf{Top}(X)$}.
\end{definition}






\section{Miscellaneous definitions}
\begin{definition} \label{definition:free_simplicial_abelian_group_of_a_simplicial_set}
    Let $X: \Deltaop \to \Sets$ be a simplicial set. The \hldef{free simplicial abelian group} \hl{$\bbZ[X]$} (which we may also denote by \hl{$\bbZ(X)$}) is the simplicial abelian group $\Deltaop \to \Ab$ given by sending $[n]$ to the free abelian group $\bbZ[X_n]$ generated by the set $X_n$; the morphisms $[n] \to [m]$ naturally induce morphisms $X_{m} \to X_{n}$ which in turn induce natural morphisms $\bbZ[X_{m}] \to \bbZ[X_{n}]$. Note that $\bbZ[X]$ is functorial in $X$.

    % Equivalently, the functor 
    % $$\bbZ[-]: \Deltaop \Sets = \Fun(\Deltaop, \Sets) \to \Fun(\Deltaop, \Ab) = \Deltaop \Ab$$
    % is given by 

\end{definition}



\begin{definition}[Scheme] \label{definition:scheme}
    A \hldef{scheme} is a \CrefAndHyperrefIfExist{definition:locally_ringed_space_on_a_topological_space}{locally ringed space} $(X, \mathcal{O}_X)$ that admits an open cover $\{U_i\}_{i \in I}$ such that each $(U_i, \mathcal{O}_X|_{U_i})$ is \CrefAndHyperrefIfExist{definition:morphism_of_locally_ringed_spaces}{isomorphic (as a locally ringed space)} to an \CrefAndHyperrefIfExist{definition:affine_scheme}{affine scheme $(\mathrm{Spec}(A_i), \mathcal{O}_{\mathrm{Spec}(A_i)})$} for some \CrefAndHyperrefIfExist{ring}{commutative ring} $A_i$.  
    In other words, a scheme is a locally ringed space locally isomorphic to affine schemes.

    
\end{definition}



\begin{definition}[Morphism of schemes] \label{definition:morphism_of_schemes}
    Let $(X, \mathcal{O}_X)$ and $(Y, \mathcal{O}_Y)$ be \CrefAndHyperrefIfExist{definition:scheme}{schemes}.  
    A \hldef{morphism of schemes} is a \CrefAndHyperrefIfExist{definition:morphism_of_locally_ringed_spaces}{morphism as locally ringed spaces}.

    In particular, there is a \CrefAndHyperrefIfExist{definition:category}{category}, often denoted by \hl{$\mathrm{Sch}$}, \hl{$\mathbf{Sch}$} etc., whose objects are schemes and whose morphisms are morphisms of schemes.
    
\end{definition}


\begin{definition}[Locally Noetherian Scheme and Noetherian Scheme] \label{definition:locally_noetherian_and_noetherian_scheme}
Let $X$ be a \CrefAndHyperrefIfExist{definition:scheme}{scheme}.

\begin{itemize}
    \item $X$ is called \hldef{locally Noetherian} if it admits an open cover $\{U_i\}$ such that for each $i$, the ring $\mathcal{O}_X(U_i)$ of regular functions on $U_i$ is a \CrefAndHyperrefIfExist{definition:noetherian_ring}{Noetherian ring}. Equivalently, $X$ is locally Noetherian if it is covered by open affine subschemes $\Spec A_i$ with each $A_i$ a Noetherian ring.

    \item $X$ is called \hldef{Noetherian} if it is locally Noetherian and \CrefAndHyperrefIfExist{definition:quasi_compact_scheme}{quasi-compact}, i.e., $X$ can be covered by finitely many affine opens $\Spec A_i$ where each $A_i$ is Noetherian.
\end{itemize}
\end{definition}

\begin{definition}[Smooth Morphism of Schemes] \label{definition:smooth_morphism_of_schemes}
Let \(f: X \to S\) be a morphism of \CrefAndHyperrefIfExist{definition:scheme}{schemes}.

We say that \(f\) is \hldef{smooth}, and that $X$ is a \hldef{smooth scheme over $S$}, if it satisfies the following conditions:

\TODO{residue field}
\begin{itemize}
    \item \(f\) is \CrefAndHyperrefIfExist{definition:locally_of_finite_presentation_finite_presentation_morphism_of_schemes}{locally of finite presentation}: for every point \(x \in X\), there exists an open neighborhood \(U \subseteq X\) of \(x\) and an open neighborhood \(V \subseteq S\) of \(f(x)\) such that the restriction \(f|_U : U \to V\) corresponds to a morphism of rings \( \mathcal{O}_S(V) \to \mathcal{O}_X(U) \) that is finitely presented.
    
    \item \(f\) is \CrefAndHyperrefIfExist{definition:flat_morphism_of_schemes}{flat}: the induced map on local rings is flat.
    
    \item For every point \(x \in X\), the fiber \(X_{f(x)} = X \times_S \Spec \kappa(f(x))\) is a smooth variety over the residue field \(\kappa(f(x))\), equivalently, the sheaf of relative Kähler differentials \(\Omega_{X/S}\) is locally free of finite rank.

\end{itemize}

Informally, a smooth morphism behaves like a submersion in differential geometry, providing "nice" fiber structures and descent properties.

Given a scheme $S$, the \hldef{category of smooth schemes over $S$} is the following \CrefAndHyperrefIfExist{definition:locally_small_category}{locally small} \CrefAndHyperrefIfExist{definition:category}{category}:
\begin{itemize}
    \item The objects are smooth morphisms $X \to S$. 
    \item The morphisms betewen objects $X_1 \to S$ and $X_2 \to S$ are \CrefAndHyperrefIfExist{definition:scheme_over_a_scheme}{$S$-morphisms} $X_1 \to X_2$ such that the following commutes:
    \begin{center}
        \begin{tikzcd}
        X_1 \ar[rr] \ar[dr] & &  X_2 \ar[dl] \\
        & S &
        \end{tikzcd}
    \end{center}
\end{itemize}
The category of smooth schemes over $S$ is often denoted by notations such as \hl{$\mathrm{Sm}/S$}, \hl{$\mathbf{Sm}/S$}, \hl{$\mathrm{Sm}_S$}, \hl{$\mathbf{Sm}_S$} etc.

\end{definition}
\begin{definition} \label{definition:smooth_at_a_point_for_a_finite_type_scheme_over_a_field}
Let $X$ be a scheme of \CrefAndHyperrefIfExist{definition:finite_type_morphism_of_schemes}{finite type} over a field $k$.
\begin{enumerate}
    \item The scheme $X$ is said to be \hldef{smooth at a point $x \in X$} of \CrefAndHyperrefIfExist{definition:dimension_of_a_scheme}{dimension} $d$ if the local ring $\mathcal{O}_{X,x}$ is geometrically regular, or equivalently, if the rank of the sheaf of differentials $\Omega_{X/k}$ at $x$ is equal to the \CrefAndHyperrefIfExist{definition:dimension_of_a_scheme}{dimension of $X$ at $x$}.

    \item The scheme $X$ is said to be \hldef{smooth} if it is smooth at all of its points. The scheme $X$ is smooth if and only if its \CrefAndHyperrefIfExist{definition:scheme_over_a_scheme}{structure morphism} $X \to \Spec k$ is a \CrefAndHyperrefIfExist{definition:smooth_morphism_of_schemes}{smooth morphism}. 
\end{enumerate}
\TODO{go into this more deeply}

\end{definition}

\begin{definition} \label{definition:etale_morphism_of_schemes}
A \CrefAndHyperrefIfExist{definition:morphism_of_schemes}{morphism of schemes} $f : X \to Y$ is called \hldef{\'etale} if it satisfies the following conditions:
\TODO{sheaf of relative differentials}
\begin{itemize}
  \item $f$ is \CrefAndHyperrefIfExist{definition:locally_of_finite_presentation_finite_presentation_morphism_of_schemes}{locally of finite presentation},
  \item $f$ is \CrefAndHyperrefIfExist{definition:flat_morphism_of_schemes}{flat},
  \item $f$ is \CrefAndHyperrefIfExist{definition:unramified_morphism_of_schemes}{unramified}, i.e., the sheaf of relative differentials $\Omega_{X/Y}$ equals $0$.
\end{itemize}
\TODO{relative dimension}
Equivalently, a morphism of schemes is \'etale if and only if it is \CrefAndHyperrefIfExist{definition:smooth_morphism_of_schemes}{smooth} of relative dimension $0$.
A \CrefAndHyperrefIfExist{definition:finite_morphism_of_schemes}{finite} \'etale morphism is synonymously called a \hldef{finite \'etale cover}.
\end{definition}


\begin{definition}[Scheme over a scheme] \label{definition:scheme_over_a_scheme}
    Let $(S, \mathcal{O}_S)$ be a scheme. A \hldef{scheme over $S$} (or an \hldef{$S$-scheme}) is a scheme $(X, \mathcal{O}_X)$ together with a morphism of schemes
    $$\pi: (X, \mathcal{O}_X) \to (S, \mathcal{O}_S).$$
    This morphism $\pi$ is called the \hldef{structure morphism of the scheme $X$ over $S$}.  

    If $S = \mathrm{Spec}(R)$ is an affine scheme for a commutative ring $R$, then an $S$-scheme is synonymously called an \hldef{$R$-scheme} or a \hldef{scheme over $R$}. 

    Let $(S, \mathcal{O}_S)$ be a scheme, and let $(X, \mathcal{O}_X)$ and $(Y, \mathcal{O}_Y)$ be schemes over $S$ with structure morphisms
    $$\pi_X: X \to S, \quad \pi_Y: Y \to S.$$
    A \hldef{morphism of $S$-schemes} (or synonymously a \hldef{$S$-scheme morphism}) is a \CrefAndHyperrefIfExist{definition:morphism_of_schemes}{morphism of schemes}
    $$(f, f^\#): (X, \mathcal{O}_X) \to (Y, \mathcal{O}_Y)$$
    such that the following diagram commutes:
    $$
    \begin{array}{ccc}
    X & \xrightarrow{f} & Y \\
    {\scriptstyle \pi_X} \downarrow & & \downarrow {\scriptstyle \pi_Y} \\
    S & = & S
    \end{array}
    $$
    In other words,
    $$\pi_Y \circ f = \pi_X.$$

    Given a fixed scheme $S$, there is a category, often denoted by \hl{$\mathrm{Sch}_S$}, \hl{$\mathrm{Sch}_{/S}$}, \hl{$\mathrm{Sch}/S$}, \hl{$\mathbf{Sch}_S$}, \hl{$\mathbf{Sch}_{/S}$}, \hl{$\mathbf{Sch}/S$} etc. whose objects are schemes $T$ over $S$ and whose morphisms $T_1 \to T_2$ are morphisms of schemes over $S$. If $S = \Spec R$ for some commutative ring $R$, then we may instead write \hl{$\mathrm{Sch}_R$} to denote $\mathrm{Sch}_{\Spec R}$, etc. It is noteworthy that $\mathrm{Sch}_\bbZ$ coincides with the category \CrefAndHyperrefIfExist{definition:morphism_of_schemes}{$\mathrm{Sch}$} of all schemes. In other words, a $\bbZ$-scheme can be identified simply with a scheme. 

    \TextIfExists{definition:category_of_objects_over_under_a_fixed_object_in_a_category}{Equivalently, the category $\mathrm{Sch}_{/S}$ is the category of schemes over $S$ in the sense of \Cref{definition:category_of_objects_over_under_a_fixed_object_in_a_category}.}

\end{definition}

\begin{definition}[Dimension of a Scheme] \label{definition:dimension_of_a_scheme}
Let $X$ be a scheme with underlying topological space $|X|$.

    \TODO{krull dimension}
\begin{itemize}
    \item The \hldef{dimension at a point $x \in |X|$}, denoted $\dim_x(X)$, is the Krull dimension of the \CrefAndHyperrefIfExist{definition:locally_ringed_space_on_a_topological_space}{local ring} \CrefAndHyperrefIfExist{definition:stalk_of_a_presheaf_on_a_topological_space_at_a_point}{$\mathcal{O}_{X,x}$}. This is the supremum of the lengths $n$ of chains of prime ideals
    $$ \mathfrak{p}_0 \subsetneq \mathfrak{p}_1 \subsetneq \cdots \subsetneq \mathfrak{p}_n \subseteq \mathcal{O}_{X,x}.  $$

    \item The \hldef{dimension of the scheme $X$} is defined as
    $$ \hlin{\dim(X) := \sup_{x \in |X|} \dim_x(X). }$$
    Equivalently, it is the supremum of the lengths of chains of distinct irreducible closed subsets of $|X|$ ordered by inclusion.
\end{itemize}
\end{definition}


\begin{definition}[Field] \label{definition:field}
A \hldef{field} is commutative \CrefAndHyperrefIfExist{definition:division_ring}{division} \CrefAndHyperrefIfExist{definition:commutative_ring}{ring}. In other words, a field is a commutative ring for which all nonzero elements have a \CrefAndHyperrefIfExist{definition:unit_of_a_ring}{multiplicative inverse}.
\end{definition}


\begin{definition}[Filtered category] \label{definition:filtered_cofiltered_category}
    \begin{enumerate}
        \item 
        A \hldef{filtered category} is a (nonempty, large) category $\mathcal{I}$ satisfying the following conditions:

        \begin{itemize}
            \item For every finite collection of objects $i_1, i_2, \ldots, i_n$ in $\mathcal{I}$, there exists an object $j$ and morphisms
            \[
            \phi_k: i_k \to j, \quad \text{for each } k=1, \ldots, n.
            \]

            \item For every pair of morphisms $f,g: i \to j$ in $\mathcal{I}$, there exists an object $k$ and a morphism 
            \[
            h: j \to k
            \] 
            that satisfies 
            \[
            h \circ f = h \circ g.
            \]
        \end{itemize}

        \begin{figure}[h]
            \centering
            \begin{minipage}{0.45\textwidth}
                \centering
                % Diagram 1: "Joint Limit" (Upper Bound)
                \begin{tikzcd}[row sep=large, column sep=large]
                    i_1 \arrow[dr, "\phi_1", dashed] & \\
                    & j \\
                    i_2 \arrow[ur, "\phi_2"', dashed] & 
                \end{tikzcd}
                \caption*{Condition 1: Upper Bound}
            \end{minipage}
            \hfill
            \begin{minipage}{0.45\textwidth}
                \centering
                \begin{tikzcd}[row sep=large, column sep=large]
                    i \arrow[r, "f", shift left] \arrow[r, "g"', shift right] & 
                    j \arrow[r, "h", dashed] & 
                    k
                \end{tikzcd}
                \caption*{Condition 2: Coequalizing map}
            \end{minipage}
        \end{figure}
        In other words, $\mathcal{I}$ is nonempty, any finite diagram of objects admits a \CrefAndHyperrefIfExist{definition:limit_and_colimit_of_a_diagram_in_a_category}{cocone}, and any pair of parallel morphisms become equal after post-composition with an appropriate morphism.

    \item Dually, a \hldef{Cofiltered category} is a category whose \hyperrefIfExists{definition:opposite_category_of_a_category}{opposite category}\CrefIfExists{definition:opposite_category_of_a_category} is filtered. More explicitly, A cofiltered category is a (nonempty, large) category $\mathcal{I}$ satisfying the following conditions:

    \begin{itemize}
        \item For every finite collection of objects $i_1, i_2, \ldots, i_n$ in $\mathcal{I}$, there exists an object $j$ and morphisms
        \[
        \phi_k: j \to i_k, \quad \text{for each } k=1, \ldots, n.
        \]

        \item For every pair of morphisms $f,g: j \to i$ in $\mathcal{I}$, there exists an object $k$ and a morphism 
        \[
        h: k \to j
        \] 
        that satisfies
        \[
        f \circ h = g \circ h.
        \]
    \end{itemize}

    \begin{figure}[h]
        \centering
        \begin{minipage}{0.45\textwidth}
            \centering
            % Diagram 1: "Joint Limit" (Lower Bound)
            \begin{tikzcd}[row sep=large, column sep=large]
                i_1 & \\
                & j \arrow[ul, "\phi_1", dashed] \arrow[dl, "\phi_2"', dashed] \\
                i_2 & 
            \end{tikzcd}
            \caption*{Condition 1: Lower Bound}
        \end{minipage}
        \hfill
        \begin{minipage}{0.45\textwidth}
            \centering
            \begin{tikzcd}[row sep=large, column sep=large]
                k \arrow[r, "h", dashed] &
                j \arrow[r, "f", shift left] \arrow[r, "g"', shift right] & 
                i
            \end{tikzcd}
            \caption*{Condition 2: Equalizing map}
        \end{minipage}
    \end{figure}

    In other words, $\mathcal{I}$ is nonempty, any finite diagram of objects admits a cone, and any pair of parallel morphisms become equal after pre-composition with an appropriate morphism.

    \end{enumerate}

    
\end{definition}


\begin{definition}[Special cases of limits] \label{definition:projective_and_inductive_limits_in_categories}
Let $\mathcal{C}$ be a (large) category. Let $I$ be a (large) category. Let $I \to \mathcal{C}$ be a diagram/system. 
\begin{itemize}
    \item Suppose that the system is a \hyperrefIfExists{definition:system_in_a_category_indexed_by_a_directed_poset}{cofiltered system}\CrefIfExists{definition:system_in_a_category_indexed_by_a_directed_poset}, i.e. $I$ is a cofiltered category. A \hyperrefIfExists{definition:limit_and_colimit_of_a_diagram_in_a_category}{limit}\CrefIfExists{definition:limit_and_colimit_of_a_diagram_in_a_category} of this diagram is often denoted by 
    $$\hlin{ \varprojlim_{i\in I} D(i) }$$
    and may be called a \hldef{cofiltered (inverse/projective) limit}. In case that the system is more specifically an \hyperrefIfExists{definition:system_in_a_category_indexed_by_a_directed_poset}{inverse/projective system}\CrefIfExists{definition:system_in_a_category_indexed_by_a_directed_poset}, i.e. $I$ is a cofiltered poset, the preferred term for such a limit is \emph{inverse/projective limit}.

    \item Suppose that the system is a filtered system, i.e. $I$ is a filtered category. A colimit of this diagram is often denoted by 
    $$\hlin{ \varinjlim_{i\in I} D(i) }$$
    and may be called a \hldef{filtered colimit} or a \hldef{direct/inductive/injective limit}. In case that the system is more specifically a direct/inductive system, i.e. $I$ is a filtered poset, the preferred term for such a limit is \emph{direct/inductive limit}.

\end{itemize}
\end{definition}

\begin{definition} \label{definition:algebraic_variety_over_a_field}
Let $k$ be a \CrefAndHyperrefIfExist{definition:field}{field}. A \hldef{(algebraic) variety over $k$} is an \CrefAndHyperrefIfExist{definition:integral_scheme}{integral}, \CrefAndHyperrefIfExist{definition:quasi_separated_morphism_of_schemes}{separated} scheme of \CrefAndHyperrefIfExist{definition:finite_type_morphism_of_schemes}{finite type} over $k$. 
\end{definition}

\begin{definition}[Field Extension] \label{definition:extension_of_a_field}
Let $K$ be a field and let $L$ be a field such that $K \subseteq L$ and the operations of $K$ are the restrictions of those of $L$.  
Then $L$ is called a \hldef{extension field (or just an extension) of $K$}. The notation \hl{$L/K$} is often used synonymously; we say that $L/K$ is a \hldef{field extension}. Moreover, $K$ is said to be a \hldef{subfield of $L$}.
\end{definition}

\begin{definition}[Affine scheme] \label{definition:affine_scheme}
Let $A$ be a \CrefAndHyperrefIfExist{definition:commutative_ring}{commutative ring with unity}. Define the set \hl{$\mathrm{Spec}(A)$}
to be the set of all \CrefAndHyperrefIfExist{definition:prime_and_maximal_ideal_of_a_ring}{prime ideals} of $A$. Equip it with the \hldef{Zariski topology}, which is the \CrefAndHyperrefIfExist{definition:topological_space}{topology} whose closed sets are given by \hldef{vanishing loci}
$$\hlin{V(I) = \{\mathfrak{p} \in \mathrm{Spec}(A) : I \subseteq \mathfrak{p}\}}$$
for ideals $I \subseteq A$.  
Define the sheaf \hl{$\mathcal{O}_{\mathrm{Spec}(A)}$}, called the \hldef{structure sheaf of $\Spec A$}, by 
$$\mathcal{O}_{\mathrm{Spec}(A)}(U) = \{ \ \text{locally defined fractions of elements of $A$ on $U$} \ \},$$
for each open set $U \subseteq \mathrm{Spec}(A)$. It is the case that the stalk at $\mathfrak{p} \in \mathrm{Spec}(A)$ is canonically the \CrefAndHyperrefIfExist{definition:localization_of_a_commutative_ring_by_a_multiplicative_subset}{localization $A_{\mathfrak{p}}$}.  
Then $(\mathrm{Spec}(A), \mathcal{O}_{\mathrm{Spec}(A)})$ is a \CrefAndHyperrefIfExist{definition:locally_ringed_space_on_a_topological_space}{locally ringed space}, called the \hldef{affine scheme associated to $A$}.

Moreover, given $f \in A$, we define the locus \hl{$D(f)$} by 
$$D(f) = \Spec A \setminus V((f)) = \{ \mathfrak{p} \in \operatorname{Spec} A : f \notin \mathfrak{p} \}$$
\end{definition}


\begin{definition}[Separable Element, Separable Extension] \label{definition:separable_algebraic_field_extension}
Let $L/K$ be a \CrefAndHyperrefIfExist{definition:extension_of_a_field}{field extension} and let $x \in L$ be \CrefAndHyperrefIfExist{definition:algebraic_element_minimal_polynomial_of_an_algebraic_element_algebraic_field_extension_transcendental_element_field_extension}{algebraic over $K$} with \CrefAndHyperrefIfExist{definition:algebraic_element_minimal_polynomial_of_an_algebraic_element_algebraic_field_extension_transcendental_element_field_extension}{minimal polynomial} $m_{x,K}(t) \in K[t]$.  
\begin{itemize}
    \item The element $x$ is \hldef{separable over $K$} if $m_{x,K}(t)$ has distinct roots in a \CrefAndHyperrefIfExist{definition:splitting_field_of_polynomials_over_a_field_normal_field_extension}{splitting field}.  
    \item The element $x$ is \hldef{inseparable over $K$} otherwise.  
\end{itemize}
An \CrefAndHyperrefIfExist{definition:algebraic_element_minimal_polynomial_of_an_algebraic_element_algebraic_field_extension_transcendental_element_field_extension}{algebraic extension} $L/K$ is called \hldef{separable extension} if every element $x \in L$ is separable over $K$.  

\TextIfExists{definition:separable_field_extension_general}{See also \Cref{definition:separable_field_extension_general}, which defines separable field extensions in greater generality.}
\end{definition}

\begin{definition}[Separable field extension] \label{definition:separable_field_extension_general}
Let $E/F$ be a \CrefAndHyperrefIfExist{definition:extension_of_a_field}{field extension}. 
In general (allowing transcendental extensions), $E/F$ is called \hldef{separable} if there exists a \CrefAndHyperrefIfExist{definition:separating_transcendence_basis_of_a_field_extension}{separating transcendence basis} for $E$ over $F$. Equivalently, $E/F$ is separable if every \CrefAndHyperrefIfExist{definition:finitely_generated_field_extension}{finitely generated} intermediate field $F \subseteq K \subseteq E$ has a separating transcendence basis over $F$.
In particular, a field extension $E/F$ is algebraic and separable in the above sense if and only if it is a separable algebraic extension in the sense of \Cref{definition:separable_algebraic_field_extension}.
\end{definition}