
\begin{definition}[Representation of a Group on a Module] \label{definition:representation_of_a_group_on_a_module_over_a_not_necessarily_commutative_ring}
Let $(G, \cdot)$ be a \CrefAndHyperrefIfExist{definition:group}{group} and let $R$ be a \CrefAndHyperrefIfExist{definition:ring}{(not necessarily commutative) ring with unity}. Let $M$ be a \CrefAndHyperrefIfExist{definition:module_of_a_ring}{left $R$-module}. A \hldef{representation of $G$ on the $R$-module $M$} is a group homomorphism
$$\hlin{\rho : G \to \operatorname{Aut}_R(M)},$$
\TODO{defien the group of automorphisms of a module}
where $\operatorname{Aut}_R(M)$ denotes the group of $R$-module automorphisms of $M$. Equivalently, this means $G$ acts on the left of $M$ via $R$-module automorphisms, i.e.,
\begin{align*}
\rho(e) &= \mathrm{id}_M, \\
\rho(g h) &= \rho(g) \circ \rho(h) \quad \text{for all } g,h \in G,
\end{align*}
and for each $g \in G$, the map $\rho(g) : M \to M$ is an $R$-linear automorphism.


Synonymously, we say that $M$ is a \hldef{$G$-module over $R$}.

Similarly, one could define a \hldef{representation of $G$ on a right $R$-module}.
\end{definition}