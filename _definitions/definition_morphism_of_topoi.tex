{\cite[Expos\'e IV D\'efinition 3.1]{SGA4_I}, see \cite[Definition 2.1]{nlab:geometric_morphism}}
    Let $E$ and $E'$ be topoi (in a universe $\scrU$). A \hldef{(continuous) morphism $f: E \to E'$} is a triple \hl{$f = (f_*, f^*, \varphi)$} consisting of functors
    $$f_*: E \to E', \quad f^*: E' \to E$$
    and an \hldef{adjunction isomorphism}
    $$\varphi: \Hom_E(u^*(X'), Y) \xrightarrow{\sim} \Hom_{E'}(X', u_*(Y))$$
    of \CrefAndHyperrefIfExist{definition:n_ary_functor}{bifunctors} in $X' \in \Ob E'$ and $Y \in \Ob E$ such that $f^*$ commutes with finite (projective) limits.
    The functors $f_*$ and $f^*$ are respectively called the \hldef{direct image functor of $f$} and the \hldef{inverse image functor of $f$}.

    This notion of morphism of topoi is also referred to as \hldef{geometric morphism of topoi}.
    