\begin{definition}[Borel $\sigma$-algebra] \label{definition:borel_sigma_algebra_of_a_topological_space}
Let $(X,\mathcal{T})$ be a \CrefAndHyperrefIfExist{definition:topological_space}{topological space}.
    A \CrefAndHyperrefIfExist{definition:sigma_algebra_on_a_set_and_measure_space}{$\sigma$-algebra} $\mathcal{B}$ on $X$ is called the \hldef{Borel $\sigma$-algebra} (associated with $\mathcal{T}$) if:
    \begin{enumerate}
    \item $\mathcal{T} \subseteq \mathcal{B}$, i.e.\ every open set in $X$ is an element of $\mathcal{B}$,
    \item $\mathcal{B}$ is contained in every $\sigma$-algebra on $X$ that contains $\mathcal{T}$ (i.e.\ it is the smallest $\sigma$-algebra on $X$ containing all open sets).
    \end{enumerate}
    Equivalently, the Borel $\sigma$-algebra on $X$ is the intersection of all $\sigma$-algebras on $X$ that contain the topology $\mathcal{T}$.

    We often denote the Borel $\sigma$-algebra on $X$ by
    $$\hlin{\mathcal{B}(X)}$$
    or, when the topology $\mathcal{T}$ is emphasized or there is potential ambiguity, by
    $$\hlin{\mathcal{B}(X,\mathcal{T})}.$$
\end{definition}