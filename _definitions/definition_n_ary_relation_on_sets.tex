\begin{definition} \label{definition:n_ary_relation_on_sets}
    Let $n$ be a positive integer. Let $X_1, X_2, \dots, X_n$ be \CrefAndHyperrefIfExist{definition:set}{sets}. An \hldef{$n$-ary relation} on these sets is a subset $R \subseteq X_1 \times X_2 \times \dots \times X_n$. The integer $n$ is called the \hldef{arity} (or \hldef{degree}) of the relation. The sets $X_1, \dots, X_n$ are called the \hldef{domains of the relation}.

    If $(x_1, x_2, \dots, x_n) \in R$, we say that the elements $x_1, \dots, x_n$ are \hldef{related} by $R$.

    In the special case where $X_1 = X_2 = \dots = X_n = X$, we say that $R$ is an \hldef{$n$-ary relation on the set $X$}. In this case, $R \subseteq X^n$.

    Specific arities have standard names:
    \begin{itemize}
        \item A \hldef{unary relation} on $X$ is a subset of $X$ (arity $n=1$).
        \item A \hldef{binary relation} from $X$ to $Y$ is a subset of $X \times Y$ (arity $n=2$).
        \item A \hldef{ternary relation} is a subset of $X \times Y \times Z$ (arity $n=3$).
    \end{itemize}
\end{definition}