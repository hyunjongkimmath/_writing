\begin{definition} \label{definition:dirichlet_character_modulo_a_nonnegative_integer}
Let $k \in \mathbb{Z}_{\geq 0}$. A \hldef{Dirichlet character modulo $k$} is a function $\chi : \mathbb{Z} \to \mathbb{C}$ satisfying:
\begin{itemize}
   \item For all $m, n \in \mathbb{Z}$, if $m \equiv n \pmod{k}$ then $\chi(m) = \chi(n)$ (i.e., $\chi$ factors through $\mathbb{Z}/k\mathbb{Z}$).
   \item For all $m, n \in \mathbb{Z}$, $\chi(mn) = \chi(m)\chi(n)$.
   \item For all $n \in \mathbb{Z}$ with $\gcd(n,k) > 1$, $\chi(n) = 0$; otherwise, $\chi(n)$ is a complex number of modulus $1$.
\end{itemize}
Equivalently, a Dirichlet character modulo $k$ can be defined by first taking a \hyperrefIfExists{definition:quasi_character_of_a_locally_compact_hausdorff_group}{character} $\chi: (\bbZ / k\bbZ)^\times \to \bbC$, extending it to $\chi: \bbZ / k \bbZ \to \bbC$, then composing that as $\bbZ \to \bbZ / k \bbZ \xrightarrow{\chi} \bbC$.
\end{definition}
