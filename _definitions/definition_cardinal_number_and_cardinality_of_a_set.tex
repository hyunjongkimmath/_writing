\begin{definition} \label{definition:cardinal_number_and_cardinality_of_a_set}
An \CrefAndHyperrefIfExist{definition:ordinal_number}{ordinal number} $\kappa$ is a \hldef{cardinal number} (or simply a \hldef{cardinal}) if for every ordinal $\alpha < \kappa$, there is no \CrefAndHyperrefIfExist{definition:injective_surjective_bijective_map_of_sets}{bijection} between $\alpha$ and $\kappa$. Equivalently, a cardinal is an initial ordinal—an ordinal that is not equinumerous with any smaller ordinal.

The \hldef{cardinality of an arbitrary set $X$}, denoted by \hl{$|X|$}, \hl{$\operatorname{card}(X)$}, or \hl{$\# X$}, is the unique cardinal number $\kappa$ such that there exists a bijection between $X$ and $\kappa$. (The existence of such a $\kappa$ for every set requires the \CrefAndHyperrefIfExist{definition:zermelo_fraenkel_set_theory}{Axiom of Choice}).
\end{definition}