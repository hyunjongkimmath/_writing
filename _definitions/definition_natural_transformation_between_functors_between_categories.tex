\begin{definition} \label{definition:natural_transformation_between_functors_between_categories}
Let $\mathcal{C}$ and $\mathcal{D}$ be \CrefAndHyperrefIfExist{definition:category}{(large) categories}. 
Let $F, G : \mathcal{C} \to \mathcal{D}$ be \CrefAndHyperrefIfExist{definition:functor_between_categories}{functors}.

A \hldef{natural transformation $\eta$ between $F$ and $G$} is a family of morphisms $\eta_X: F(X) \to G(X)$ in $\mathcal{D}$, one for each object $X$ in $\mathcal{C}$, such that for every morphism $f: X \to Y$ in $\mathcal{C}$,
\begin{align*}
G(f) \circ \eta_X = \eta_Y \circ F(f)
\end{align*}
in $\mathcal{D}$. In other words, the following diagram commutes:
\begin{center}
\begin{tikzcd}
    F(X) \arrow[r, "F(f)"] \arrow[d, "\eta_X"']
    & F(Y) \arrow[d, "\eta_Y"] \\
    G(X) \arrow[r, "G(f)"']
    & G(Y)
\end{tikzcd}
\end{center}

We write such a natural transformation by \hl{$\eta: F \Rightarrow G$}.

If $\eta_X$ is an \CrefAndHyperrefIfExist{definition:isomorphism_in_a_category}{isomorphism} for all objects $X$ of $\calC$, then $\eta$ is said to be a \hldef{natural isomorphism}.
\end{definition}
