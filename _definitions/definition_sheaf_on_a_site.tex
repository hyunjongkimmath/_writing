\begin{definition}[Sheaf on a site] \label{definition:sheaf_on_a_site}

% \TODO{There might be some need to say that $\calA$ is a category for which sheaves on the site ``can be defined''}
% \TODO{go through statements using the notion of sheaves and make sure that the value categories have small products and that the categories have small generating families.}

Let $(\calC, J)$ be a \CrefAndHyperrefIfExist{definition:grothendieck_topology_on_a_category_site_covering_sieve_topologically_generating_family}{site}. Let $\calA$ be a \CrefAndHyperrefIfExist{definition:category}{(large) category}.
\begin{enumerate}
    \item A \CrefAndHyperrefIfExist{definition:presheaf_on_a_category}{presheaf} $\calF: \calC^{\op} \to \calA$\CrefIfExists{definition:opposite_category_of_a_category} is called a \hldef{sheaf on the site $(\calC, J)$ valued in $\calA$} if, for every object $U$ of $\calC$ and every \CrefAndHyperrefIfExist{definition:grothendieck_topology_on_a_category_site_covering_sieve_topologically_generating_family}{covering sieve} $S \in J(U)$, the \CrefAndHyperrefIfExist{definition:limit_and_colimit_of_a_diagram_in_a_category}{limit}
    $$\varprojlim_{(V \to U) \in (\calD_S)^{\op}} \calF|_{\calD_S}(V),$$
    exists and the canonical natural morphism
    $$\calF(U) \to \varprojlim_{(V \to U) \in (\calD_S)^{\op}} \calF|_{\calD_S}(V)$$
    is an isomorphism. Here, $\calD_S \hookrightarrow \calC/U$\CrefIfExists{definition:category_of_objects_over_under_a_fixed_object_in_a_category} is the full \CrefAndHyperrefIfExist{definition:downward_upward_closed_subcategory_of_a_category}{downward-closed subcategory} such that $\operatorname{Ob}(\calD_S) = \{(f: V \to U): f \in S(V)\}$,

    In particular, when we are working with a \CrefAndHyperref{definition:basis_and_grothendieck_pretopology_for_a_grothendieck_topology_on_a_category}{Grothendieck pretopology} $K$ on a category $\calC$, we may speak of sheaves on the site whose Grothendieck topology is the \CrefAndHyperref{definition:grothendieck_topology_generated_by_a_pretopology}{one generated by} $K$.

    \item Given sheaves $\calF, \calG: \calC^{\op} \to \calA$ on the site $(\calC, J)$, a \hldef{morphism between the sheaves} is a \CrefAndHyperrefIfExist{definition:presheaf_on_a_category}{morphism} between $\calF$ and $\calG$ as presheaves.


    \item Let $U$ be a \hyperrefIfExists{definition:grothendieck_universe}{universe}\CrefIfExists{definition:grothendieck_universe}. A \hldef{$U$-sheaf} typically refers to a $U$-presheaf that is a sheaf for a $U$-site. In other words, a $U$-sheaf is a sheaf on a site whose underlying category is \hyperrefIfExists{definition:locally_small_category}{$U$-locally small}\CrefIfExists{definition:locally_small_category} and which has a $U$-small topologically generating family such that the sheaf is valued in $U$-sets.

    \item The \hldef{sheaf category/category of $\calA$-valued sheaves on $\calC$} is the (large) category defined as the full subcategory of $\PreShv(\calC, \calA)$ whose objects are the sheaves on $\calC$ with values in $\calA$. Common notations for the sheaf category include \hl{$\Shv(\calC, \calA)$}, \hl{$\Shv(\calC, J, \calA)$}, \hl{$\Sh(\calC, \calA)$}, \hl{$\Sh(\calC, J, \calA)$}. If the value category $\calA$ is clear from context, then notations such as \hl{$\Shv(\calC)$}, \hl{$\Shv(\calC, J)$}, \hl{$\Sh(\calC)$}, \hl{$\Sh(\calC, J)$} are also common.

\end{enumerate}

% Let $(\calC, J)$ be a \CrefAndHyperrefIfExist{definition:grothendieck_topology_on_a_category_site_covering_sieve_topologically_generating_family}{site} with a small \CrefAndHyperrefIfExist{definition:grothendieck_topology_on_a_category_site_covering_sieve_topologically_generating_family}{topological generating family} (or a $U$-small topologically generating family if a \CrefAndHyperrefIfExist{definition:grothendieck_universe}{universe} $U$ is available) and let $\mathcal{A}$ be a \CrefAndHyperrefIfExist{definition:category}{(large) category} that has all \CrefAndHyperrefIfExist{definition:locally_small_category}{small} \CrefAndHyperrefIfExist{definition:product_and_coproduct_of_objects_in_a_category}{products} (Some common examples of categories that have small products and thus play the role of $\calA$ here include $\mathcal{A} = \text{Set}$, $\text{Ab}$, $R\mathbf{-mod}$ for a fixed ring $R$, $\text{rings}$). 
% \begin{enumerate}

%     \item For any object $U$ of $\calC$ and every covering $\{U_i \to U\}_{i \in I}$ in $J$, note that there are morphisms $U_i \times_U U_j \to U_i$ for every $i,j \in I$. 
%     % Consider the subcategory of $C$ consisting of the objects $U_i$ and $U_i \times_U U_j$, together with these morphisms.
%     Given any presheaf $\calF: C^{\op} \to \calA$, there is a \CrefAndHyperrefIfExist{definition:diagram_in_a_category_indexed_by_a_small_category}{diagram} in $\calA$ consisting of objects $\calF(U_i)$ and $\calF(U_i \times_U U_j)$ and morphisms $\calF(U_i) \to \calF(U_i \times_U U_j)$. The presheaf $\calF$ is called a \hldef{sheaf on the site $(\calC, J)$ valued in $\calA$} if, for every object $U$ of $\calC$ and every covering $\{U_i \to U\}_{i \in I}$ in $J$, the sections object $\calF(U)$ is the \CrefAndHyperrefIfExist{definition:limit_and_colimit_of_a_diagram_in_a_category}{limit} of the aforementioned diagram:
    
%     % A \hyperrefIfExists{definition:presheaf_on_a_category}{presheaf}\CrefIfExists{definition:presheaf_on_a_category} $\mathcal{F}: C^{\mathrm{op}} \to \mathcal{A}$ is a \hldef{sheaf on the site $(\calC,J)$ valued in $\calA$} if, for every object $U$ of $\calC$ and every covering $\{U_i \to U\}_{i \in I}$ in $J$, the sections object $\calF(U)$ is the \CrefAndHyperrefIfExist{definition:limit_and_colimit_of_a_diagram_in_a_category}{limit} of the sections objects $\calF(U_i)$:
%     % $$\calF(U) \cong \varprojlim_{}$$
    
%     % following sequence is an \CrefAndHyperrefIfExist{definition:equalizer_and_coequalizer_of_morphisms_in_a_category}{equalizer} in $\mathcal{A}$:
%     % \[
%     % \mathcal{F}(U) \to \prod_{i} \mathcal{F}(U_i) \rightrightarrows \prod_{i, j} \mathcal{F}(U_i \times_U U_j)
%     % \]
%     % where the first map sends $s$ to $(\mathcal{F}(U_i \to U)(s))_i$ and the arrows to $(\mathcal{F}(U_i \times_U U_j \to U_i)(s_i))_{i,j}$ and $(\mathcal{F}(U_i \times_U U_j \to U_j)(s_j))_{i,j}$, respectively.

%     % \item A \hyperrefIfExists{definition:presheaf_on_a_category}{presheaf}\CrefIfExists{definition:presheaf_on_a_category} $\mathcal{F}: C^{\mathrm{op}} \to \mathcal{A}$ is a \hldef{sheaf on the site $(\calC,J)$ valued in $\calA$} if, for every object $U$ of $\calC$ and every covering $\{U_i \to U\}_{i \in I}$ in $J$, the following sequence is an \CrefAndHyperrefIfExist{definition:equalizer_and_coequalizer_of_morphisms_in_a_category}{equalizer} in $\mathcal{A}$:
%     % \[
%     % \mathcal{F}(U) \to \prod_{i} \mathcal{F}(U_i) \rightrightarrows \prod_{i, j} \mathcal{F}(U_i \times_U U_j)
%     % \]
%     % where the first map sends $s$ to $(\mathcal{F}(U_i \to U)(s))_i$ and the arrows to $(\mathcal{F}(U_i \times_U U_j \to U_i)(s_i))_{i,j}$ and $(\mathcal{F}(U_i \times_U U_j \to U_j)(s_j))_{i,j}$, respectively.

%     \item A \hldef{morphism of sheaves} $\calF: \calC^{\op} \to \calA$ is a \hyperrefIfExists{definition:presheaf_on_a_category}{morphism as presheaves}\CrefIfExists{definition:presheaf_on_a_category}. 


%     \item Let $U$ be a \hyperrefIfExists{definition:grothendieck_universe}{universe}\CrefIfExists{definition:grothendieck_universe}. A \hldef{$U$-sheaf} typically refers to a $U$-presheaf that is a sheaf for a $U$-site. In other words, a $U$-sheaf is a sheaf on a site whose underlying category is \hyperrefIfExists{definition:locally_small_category}{$U$-locally small}\CrefIfExists{definition:locally_small_category} and which has a $U$-small topologically generating family such that the sheaf is valued in $U$-sets.

%     \item The \hldef{sheaf category/category of $\calA$-valued sheaves on $\calC$} is the (large) category defined as the full subcategory of $\PreShv(\calC, \calA)$ whose objects are the sheaves on $C$ with values in $\calA$. Common notations for the sheaf category include \hl{$\Shv(\calC, \calA)$}, \hl{$\Shv(\calC, J, \calA)$}, \hl{$\Sh(\calC, \calA)$}, \hl{$\Sh(\calC, J, \calA)$}. If the value category $\calA$ is clear from context, then notations such as \hl{$\Shv(\calC)$}, \hl{$\Shv(\calC, J)$}, \hl{$\Sh(\calC)$}, \hl{$\Sh(\calC, J)$} are also common.

% \end{enumerate}
\end{definition}
