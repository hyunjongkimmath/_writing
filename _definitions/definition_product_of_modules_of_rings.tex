\begin{definition}[Product of Modules] \label{definition:product_of_modules_of_rings}
    Let $R$ and $S$ be \CrefAndHyperrefIfExist{definition:ring}{(not necessarily commutative) rings}, and let $\{ M_i \}_{i \in I}$ be a (possibly \CrefAndHyperrefIfExist{definition:countable_finite_uncountable_sets}{infinite} but \CrefAndHyperrefIfExist{definition:small_set}{small}) family of \CrefAndHyperrefIfExist{definition:module_of_a_ring}{$(R,S)$-bimodules}.  

\CrefAndHyperrefIfExist{definition:module_of_a_ring}{left $R$-modules, of right $R$-modules, of two-sided $R$-modules}, or of 

    The \hldef{(direct) product of the family $\{M_i\}_{i \in I}$} is defined, as a \CrefAndHyperrefIfExist{definition:group}{group}, as the \CrefAndHyperrefIfExist{definition:product_of_groups}{product of sets}:
    $$ \prod_{i \in I} M_i := \{ (m_i)_{i \in I} \mid m_i \in M_i \text{ for all } i \in I \}.  $$

    $\prod_{i \in I} M_i$ inherits a natural $R$-$S$ module structure defined componentwise by the following rules for all $(m_i)_{i \in I}, (n_i)_{i \in I} \in \prod_{i \in I} M_i$ and all scalars $r \in R$, $s \in S$:
    $$ (m_i)_{i \in I} + (n_i)_{i \in I} := (m_i + n_i)_{i \in I}, \qquad r \cdot (m_i)_{i \in I} \cdot s := (r \cdot m_i \cdot s)_{i \in I}.  $$

    % \begin{itemize}
    % \item If each $M_i$ is a left $R$-module:
    % $$
    % (m_i)_{i \in I} + (n_i)_{i \in I} := (m_i + n_i)_{i \in I}, \qquad
    % r (m_i)_{i \in I} := (r m_i)_{i \in I}.
    % $$

    % \item If each $M_i$ is a right $R$-module:
    % $$
    % (m_i)_{i \in I} + (n_i)_{i \in I} := (m_i + n_i)_{i \in I}, \qquad
    % (m_i)_{i \in I} r := (m_i r)_{i \in I}.
    % $$

    % \item If each $M_i$ is an $(R,S)$-bimodule:
    % $$
    % (r (m_i)_{i \in I}) s := (r m_i s)_{i \in I}, \qquad r \in R,\, s \in S.
    % $$

    The zero element of $\prod_{i \in I} M_i$ is the tuple $(0)_{i \in I}$, and additive inverses are given componentwise:
    $$
    -(m_i)_{i \in I} := (-m_i)_{i \in I}.
    $$
    % Thus, $\prod_{i \in I} M_i$ is a left/right/two-sided $R$-module or an $(R,S)$-bimodule under these operations.
    % \end{itemize}

    Note that we can define the product of a family $\{M_i\}_{i \in I}$ of left/right/two-sided $R$-modules by taking the \CrefAndHyperrefIfExist{definition:module_of_a_ring}{natural bimodule structure} of each module.

    \CrefAndHyperrefIfExist{definition:product_of_groups}{As usual}, $\prod_{i \in I} M_i$ is the \CrefAndHyperrefIfExist{definition:product_and_coproduct_of_objects_in_a_category}{categorical product} of the objects $M_i$ in the appropriate \CrefAndHyperrefIfExist{definition:category_of_modules_and_bimodules_over_rings}{category of modules}. Moreover, the product of finitely many modules $M_1,\ldots,M_n$ is often written as \hl{$M_1 \times \cdots \times M_n$}, which agrees with notation for the \CrefAndHyperrefIfExist{definition:product_of_groups}{product of finitely many groups}. We often write the self-product of a module $M$ indexed by a (small) set $I$ as \hl{$M^I$} or by \hl{$M^{\oplus I}$}. A finite self-product of a module $M$ taken $n$ times is often denoted by \hl{$M^n$} or \hl{$M^{\oplus n}$}; note that these all agree with the notations for abelian groups.
\end{definition}