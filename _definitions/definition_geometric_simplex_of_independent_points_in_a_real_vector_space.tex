\begin{definition} \label{definition:geometric_simplex_of_independent_points_in_a_real_vector_space}
Let $V$ be a real vector space of finite dimension.  
A \hldef{$k$-simplex in topology} (or a \hldef{geometric $k$-simplex}) is the convex hull of $k+1$ affinely independent points $v_0, v_1, \dots, v_k \in V$, and is denoted by
$$\hlin{[v_0, v_1, \dots, v_k] := \left\{ \sum_{i=0}^k t_i v_i \ \middle| \ t_i \ge 0, \ \sum_{i=0}^k t_i = 1 \right\}.}$$


It is also standard to talk of the \hldef{standard topological $n$-simplex} ---  the topological space \hl{$|\Delta^n|$} defined as the subset of Euclidean space $\mathbb{R}^{n+1}$ given by
$$ |\Delta^n| = \Big\{ (t_0, t_1, \ldots, t_n) \in \mathbb{R}^{n+1} : \sum_{i=0}^n t_i = 1, \text{ and } t_i \geq 0 \text{ for all } i \Big\} $$
equipped with the induced topology from the usual Euclidean topology on $\mathbb{R}^{n+1}$.

\TODO{comment on how $|\Delta^n|$ makes sense via a geometric realization}
% Eqiuvalently, $|\Delta^n|$ 

\end{definition}