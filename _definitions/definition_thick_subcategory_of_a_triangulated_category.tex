\begin{definition}[Thick subcategory] \label{definition:thick_subcategory_of_a_triangulated_category}
    Let $\mathcal{T}$ be a \CrefAndHyperrefIfExist{definition:triangulated_category}{triangulated category}. A \CrefAndHyperrefIfExist{definition:full_subcategory_of_a_category}{full} triangulated subcategory $\mathcal{S} \subseteq \mathcal{T}$ is called \hldef{thick} (also \hldef{epaisse}) if it is closed under \CrefAndHyperrefIfExist{definition:additive_category_preadditive_category}{direct summands}. More explictly, for any object $X \in \mathcal{T}$, if $X$ is isomorphic to a \CrefAndHyperrefIfExist{definition:additive_category_preadditive_category}{direct sum}
    $$ X \cong Y \oplus Z, $$
    and $X$ lies in $\mathcal{S}$, then both $Y$ and $Z$ lie in $\mathcal{S}$.
    % Equivalently, a thick subcategory is a triangulated subcategory $\mathcal{S}$ such that whenever an object is a direct summand of an object in $\mathcal{S}$, it also belongs to $\mathcal{S}$.

    The notion of a thick subcategory of a triangulated category should not be confused for the notion of a \CrefAndHyperrefIfExist{definition:serre_subcategory_of_an_abelian_category}{thick subcategory} of an abelian category.
\end{definition}