\begin{definition}[Least upper bound and greatest lower bound] \label{definition:least_upper_bound_and_greatest_lower_bound_of_a_subset_of_an_ordered_set}
Let $(X,\leq)$ be an \CrefAndHyperrefIfExist{definition:partially_ordered_set}{ordered set} and let $S \subseteq X$ be a subset.
\begin{enumerate}
  \item Assume that $S$ has at least one upper bound in $X$, i.e.\ there exists $u \in X$ such that $s \leq u$ for all $s \in S$.
  An element $u_0 \in X$ is called a \hldef{least upper bound of $S$ in $X$} if:
  \begin{enumerate}
    \item $u_0$ is an upper bound of $S$, i.e.\ $s \leq u_0$ for all $s \in S$,
    \item for every upper bound $u \in X$ of $S$ one has $u_0 \leq u$.
  \end{enumerate}
  If such $u_0$ exists, it is unique and is also called the \hldef{supremum of $S$ in $X$}.

  \item Assume that $S$ has at least one lower bound in $X$, i.e.\ there exists $\ell \in X$ such that $\ell \leq s$ for all $s \in S$.
  An element $\ell_0 \in X$ is called a \hldef{greatest lower bound of $S$ in $X$} if:
  \begin{enumerate}
    \item $\ell_0$ is a lower bound of $S$, i.e.\ $\ell_0 \leq s$ for all $s \in S$,
    \item for every lower bound $\ell \in X$ of $S$ one has $\ell \leq \ell_0$.
  \end{enumerate}
  If such $\ell_0$ exists, it is unique and is also called the \hldef{infimum of $S$ in $X$}.
\end{enumerate}
\end{definition}