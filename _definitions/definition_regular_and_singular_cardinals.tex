\begin{definition} \label{definition:regular_and_singular_cardinals}
Let $\kappa$ be a \CrefAndHyperrefIfExist{definition:cardinal_number_and_cardinality_of_a_set}{cardinal number}.
\begin{itemize}
    \item $\kappa$ is a \hldef{singular cardinal} if it can be written as a union of fewer than $\kappa$ sets, each of cardinality less than $\kappa$. Formally, $\kappa$ is singular if its \hldef{cofinality} is strictly less than $\kappa$, i.e., $\operatorname{cf}(\kappa) < \kappa$.
    \item $\kappa$ is a \hldef{regular cardinal} if it is not singular. Formally, $\kappa$ is regular if $\operatorname{cf}(\kappa) = \kappa$.
\end{itemize}
\end{definition}