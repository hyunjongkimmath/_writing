\begin{definition}\label{definition:dirichlet_L_function_of_a_dirichlet_character_modulo_a_nonnegative_integer}
Let $k \in \mathbb{Z}_{\geq 0}$ and let $\chi$ be a \hyperrefIfExists{definition:dirichlet_character_modulo_a_nonnegative_integer}{Dirichlet character modulo $k$}. For $s \in \mathbb{C}$, the \hldef{$L$-function/series of $\chi$} is defined as
$$\hlin{L(s, \chi) = \sum_{n=1}^\infty \frac{\chi(n)}{n^s},}$$
which converges absolute when $\Re(s) > 1$ and in this case, $L(s,\chi)$ can be expressed as an Euler product:
$$L(s,\chi) = \prod_{n=1}^\infty \frac{1}{(1- \chi(p) p^{-s})}.$$
\TODO{TODO: discuss that it extends meromorphically, its trivial zeroes, functional equation}
\TextIfExists{definition:hecke_L_function_of_a_hecke_character_of_a_number_field}{Equivalently, $L(s,\chi)$ may be defined as the Hecke $L$-function \TODO{TODO: discuss how hecke characters on the rationals are equivalent to dirichlet characters}}
\end{definition}
