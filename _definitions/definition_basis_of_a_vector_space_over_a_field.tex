\begin{definition} \label{definition:basis_of_a_vector_space_over_a_field}
Let $F$ be a \CrefAndHyperrefIfExist{definition:field}{field}, and let $V$ be an \CrefAndHyperrefIfExist{definition:vector_space_over_a_field}{$F$-vector space}.
A subset $B \subseteq V$ is called a \hldef{basis of $V$} if: (i) $B$ is \CrefAndHyperrefIfExist{definition:linearly_independent_elements_of_a_module_over_a_ring}{linearly independent} over $F$, and (ii) $B$ \CrefAndHyperrefIfExist{definition:span_a_module_over_a_ring_for_elements_of_the_module}{spans} $V$.

If $B$ is a basis, we define the \hldef{dimension of $V$ over $F$} (or \hldef{rank of $V$ over $F$}), denoted by
$$ \hlin{\dim_F(V)}, $$
\TODO{cardinality}
to be the cardinality of $B$.
This value is uniquely determined by $V$ and $F$.
\end{definition}