
\begin{definition}[Quasi-isomorphism] \label{definition:quasi_isomorphism_of_chain_complexes_of_objects_in_an_abelian_category}
    Let $\mathcal{A}$ be an \hyperrefIfExists{definition:abelian_category}{abelian category}\CrefIfExists{definition:abelian_category}, and let
    \[
    f_\bullet: (C_\bullet, d_\bullet^C) \to (D_\bullet, d_\bullet^D)
    \]
    be a \hyperrefIfExists{definition:chain_complex_of_objects_in_an_additive_category}{chain map between complexes}\CrefIfExists{definition:chain_complex_of_objects_in_an_additive_category} in $\mathcal{A}$.

    The morphism $f_\bullet$ is called a \hldef{quasi-isomorphism} if it induces isomorphisms on all cohomology objects, i.e., for every integer $n$, the induced morphism on \hyperrefIfExists{definition:homology_and_cohomology_objects_for_a_chain_complex_in_an_additive_category}{homology}\CrefIfExists{definition:homology_and_cohomology_objects_for_a_chain_complex_in_an_additive_category} (or cohomology, depending on the convention)
    \[
    H^n(f_\bullet): H^n(C_\bullet) \to H^n(D_\bullet)
    \]
    is an isomorphism in $\mathcal{A}$.

    Note that all of these notions are applicable to the cohomological convention as well\CrefIfExists{remark:cohomological_vs_homological_conventions}.
\end{definition}
