% \begin{definition}[Grothendieck topology] \label{definition:grothendieck_topology_on_a_category_site_covering_sieve_topologically_generating_family}
%     Let $\mathscr{U}$ be a \hyperrefIfExists{definition:grothendieck_universe}{universe}\CrefIfExists{definition:grothendieck_universe} and let $\calC$ be a \hyperrefIfExists{definition:locally_small_category}{locally small category}\CrefIfExists{definition:locally_small_category}.

%     \begin{enumerate}
%         \item \textbf{(Grothendieck Topology via Sieves)}
%         A \hldef{Grothendieck topology} $J$ on $\calC$ is an assignment to each object $U \in \calC$ of a collection $J(U)$ of \CrefAndHyperrefIfExist{definition:sieve_on_an_object_in_a_category}{sieves} on $U$, called \hldef{covering sieves}, satisfying:
%         \begin{enumerate}
%             \item (Maximality) The maximal \CrefAndHyperrefIfExist{definition:sieve_on_an_object_in_a_category}{sieve} $\{ f : V \to U \mid V \in \calC \}$ is in $J(U)$.
%             \item (Stability) If $S \in J(U)$ and $f : V \to U$ is any morphism, then the \CrefAndHyperrefIfExist{definition:pullback_sieve_of_an_object_in_a_category_via_a_morphism_to_the_object}{pullback sieve} $f^{*}S$ is in $J(V)$.
%             \item (Transitivity/Local Character) If $S$ is a sieve on $U$ and there exists a covering sieve $R \in J(U)$ such that for every morphism $f : V \to U$ in $R$, the pullback sieve $f^{*}S$ is in $J(V)$, then $S \in J(U)$.
%         \end{enumerate}

%         % \item \textbf{(Grothendieck Pretopology / Basis)}
%         % If $\calC$ admits fiber products, one can define a topology via \hldef{covering families}. A \hldef{Grothendieck pretopology} (or basis) is a collection $K(U)$ of families $\{U_i \to U\}_{i \in I}$ for each object $U$, satisfying:
%         % \begin{itemize}
%         %     \item (Isomorphism) $\{U' \xrightarrow{\sim} U\} \in K(U)$ for any isomorphism.
%         %     \item (Stability) If $\{U_i \to U\} \in K(U)$ and $V \to U$ is a morphism, then $\{U_i \times_U V \to V\} \in K(V)$.
%         %     \item (Composition) If $\{U_i \to U\} \in K(U)$ and for each $i$, $\{V_{ij} \to U_i\} \in K(U_i)$, then the composite family $\{V_{ij} \to U\} \in K(U)$.
%         % \end{itemize}
%         % Every pretopology generates a unique Grothendieck topology $J$, where $S \in J(U)$ iff $S$ contains a covering family from the pretopology.

%         \item A \hldef{site} is a pair $(\calC, J)$ consisting of a category $\calC$ and a Grothendieck topology $J$.

%         \item A family of objects $\mathcal{G} = \{G_\alpha\}$ in a site $(\calC, J)$ is called a \hldef{topologically generating family} if for every object $X \in \calC$, there exists a covering sieve $S \in J(X)$ \CrefAndHyperrefIfExist{definition:sieve_on_an_object_of_a_category_generated_by_a_family_of_morphisms}{generated by} morphisms with domains in $\mathcal{G}$. Equivalently, every object $X$ admits a cover $\{U_i \to X\}$ where each $U_i \in \mathcal{G}$.

%         \item A \hldef{$\mathscr{U}$-site} is a site whose underlying category is $\mathscr{U}$-locally small and which admits a $\mathscr{U}$-small topologically generating family.
%     \end{enumerate}
% \end{definition}

\begin{definition}[Grothendieck topology] \label{definition:grothendieck_topology_on_a_category_site_covering_sieve_topologically_generating_family}
    Let $\mathscr{U}$ be a \hyperrefIfExists{definition:grothendieck_universe}{universe}\CrefIfExists{definition:grothendieck_universe}.
    \begin{enumerate}
        % \item Let $C$ be a \hyperrefIfExists{definition:locally_small_category}{locally small category}\CrefIfExists{definition:locally_small_category}. A \hldef{Grothendieck topology on $C$} assigns to each object $U$ of $C$ a collection of families of morphisms $\{U_i \to U\}_{i \in I}$, called \hldef{coverings of $U$}, satisfying:
        % \begin{itemize}
        %     \item (Isomorphism) If $f: V \to U$ is an isomorphism in $C$, then $\{f: V \to U\}$ is a covering of $U$.
        %     \item (Stability under base change) If $\{U_i \to U\}_{i \in I}$ is a covering of $U$ and $V \to U$ is any morphism, then the family $\{ U_i \times_U V \to V \}_{i \in I}$ is a covering of $V$.
        %     \item (Transitivity) If $\{U_i \to U\}_{i \in I}$ is a covering of $U$ and for each $i$, $\{V_{ij} \to U_i\}_{j \in J_i}$ is a covering of $U_i$, then the family $\{ V_{ij} \to U \}_{i \in I,\, j \in J_i}$ is a covering of $U$.
        % \end{itemize}

        \item (See \cite[Expos\'e II, D\'efinition 1.1]{SGA4_I}) Let $\calC$ be a \CrefAndHyperrefIfExist{definition:category}{category}. A \hldef{Grothendieck topology on $\calC$} assigns to each object $U$ of $\calC$ a collection \hl{$J(U)$} of \CrefAndHyperrefIfExist{definition:sieve_on_an_object_in_a_category}{sieves} $\{U_i \to U\}_{i \in I}$, each called a \hldef{covering sieve of $U$}, satisfying:
        \begin{enumerate}
            \item (Stability under ``base change''): If $S \in J(U)$ is a covering sieve of an object $U$, and $f: V \to U$ is any morphism in $\calC$, then the \CrefAndHyperrefIfExist{definition:pullback_sieve_of_an_object_in_a_category_via_a_morphism_to_the_object}{pullback sieve} $f^* S$ is a covering sieve of $U$.
            % \item (Local character condition) If $F$ is a sieve on $U$ such that the sieve $\bigcup_...$ \TODO{}
            \item (Local character condition) If $S$ is a sieve on $U$, and if there exists a covering sieve $R \in J(U)$ such that for all $f: V \to U$ in $R$ the \CrefAndHyperrefIfExist{definition:pullback_sieve_of_an_object_in_a_category_via_a_morphism_to_the_object}{pullback sieve} $f^* S$ is in $J(V)$, then $S \in J(U)$. 
            
            \item The \CrefAndHyperrefIfExist{definition:maximal_sieve_on_an_object_in_a_category}{maximal sieve} is a covering sieve.
        \end{enumerate}


        % Equivalently, a Grothendieck topology $J$ on a category $C$ is an assignment of a collection $J(U)$ of \CrefAndHyperrefIfExist{definition:sieve_on_an_object_in_a_category}{sieves} on each object $U \in \operatorname{Ob}(C)$ such that:
        % \begin{enumerate}
        %     \item the maximal \CrefAndHyperrefIfExist{definition:sieve_on_an_object_in_a_category}{sieve} $\{ f : V \to U \mid f \in \operatorname{Mor}(C) \}$ belongs to $J(U)$,
        %     \item if $S \in J(U)$ and $f : V \to U$, then the \CrefAndHyperrefIfExist{definition:pullback_sieve_of_an_object_in_a_category_via_a_morphism_to_the_object}{pullback sieve $f^{*}S$} on $V$ belongs to $J(V)$,
        %     \item if $S$ is a sieve on $U$, and if there exists $R \in J(U)$ such that for all $f : V \to U$ in $R$ the \CrefAndHyperrefIfExist{definition:pullback_sieve_of_an_object_in_a_category_via_a_morphism_to_the_object}{pullback sieve $f^{*}S$} is in $J(V)$, then $S \in J(U)$.
        % \end{enumerate}

        Some will refer to a Grothendieck topology as simply a \hldef{topology}, not to be confused with the related, but less general, notion of a \CrefAndHyperrefIfExist{definition:topological_space}{topology on a set}.


        \item (See \cite[Expos\'e II, 1.1.5]{SGA4_I}) A \hldef{site} is a category $\calC$ equipped with a Grothendieck topology.

        When we are working with a \CrefAndHyperref{definition:basis_and_grothendieck_pretopology_for_a_grothendieck_topology_on_a_category}{Grothendieck pretopology} $K$ on a category $\calC$, we may regard $\calC$ as a site by equipping it with the \CrefAndHyperref{definition:grothendieck_topology_generated_by_a_pretopology}{Grothendieck topology generated by} $K$. 

        \item (See \cite[Expos\'e II, D\'efinition 1.2]{SGA4_I}) Let $(\calC, J)$ be a site. A family of morphisms $(U_i \to U)_{i \in I}$ is called a \hldef{covering family of $U$ (with respect to the site/topology)} or a \hldef{cover of $U$ (with respect to the site/topology)} if the \CrefAndHyperrefIfExist{definition:sieve_on_an_object_of_a_category_generated_by_a_family_of_morphisms}{sieve generated by} the family is a covering sieve of $U$. 

        \item (See \cite[Expos\'e II, D\'efinition 3.0.1]{SGA4_I}) Let $(\calC, J)$ be a \CrefAndHyperrefIfExist{definition:grothendieck_topology_on_a_category_site_covering_sieve_topologically_generating_family}{site}, where $J$ is a Grothendieck topology on $\calC$.

        A family $G$ of objects $\calC$ is called a \hldef{topologically generating family of the site/topology} or a \hldef{generating family/collection of the site/topology} if for every object $X \in \calC$, there is a covering family $\{X_\alpha \to X\}_{\alpha \in A}$ of $X$ such that every $X_\alpha$ is a member of $G$.  

        Equivalently, the Grothendieck topology $J$ is the smallest Grothendieck topology containing all covers of the $U_i$. Also equivalently, for any $S \in J(X)$, the sieve $S$ contains a covering family $\{V_i \to X\}$ such that each morphism $V_i \to X$ factors through some member of $G$. \TODO{Verify that these claimed equivalences are indeed equivalences}
        
        % A family of objects $\{U_i\}_{i \in I}$ in $\calC$ is called a \hldef{topologically generating family} if for every object $X \in \calC$ and every covering sieve $S \in J(X)$, the sieve $S$ is \CrefAndHyperrefIfExist{definition:sieve_on_an_object_of_a_category_generated_by_a_family_of_morphisms}{generated by} pullbacks of covering families from the family $\{U_i\}$.

        % More precisely, this means that for any $S \in J(X)$, the sieve $S$ contains a covering family $\{V_j \to X\}$ such that each morphism $V_j \to X$ factors through some $U_i$, and the covering families of the $U_i$ generate the topology $J$. 
        % Equivalently, the Grothendieck topology $J$ is the smallest Grothendieck topology containing all coverings of the $U_i$.

        % When one speaks of a \hldef{generating family/collection} of a site, one usually refers to the above notion of a topologically generating family.

        \item (See \cite[Expos\'e II, D\'efinition 3.0.2]{SGA4_I}) A \hldef {$\mathscr{U}$-site} is a site whose underlying category $\calC$ is \hyperrefIfExists{definition:locally_small_category}{$\mathscr{U}$-locally small}\CrefIfExists{definition:locally_small_category} and which has a $\mathscr{U}$-small topologically generating family. A $\mathscr{U}$-site is called \hldef{$\mathscr{U}$-small} if its underlying category is $\mathscr{U}$-small. Similarly, a \hldef{small site} is a site whose underlying category is a set and a \hldef{locally small site} is a site whose underlying category is \CrefAndHyperrefIfExist{definition:locally_small_category}{locally small}.
    \end{enumerate}
\end{definition}
