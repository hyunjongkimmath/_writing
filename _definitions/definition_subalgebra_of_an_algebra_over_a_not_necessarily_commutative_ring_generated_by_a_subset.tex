\begin{definition}[Subalgebra generated by a subset] \label{definition:subalgebra_of_an_algebra_over_a_not_necessarily_commutative_ring_generated_by_a_subset}
    Let $R$ be a \CrefAndHyperrefIfExist{definition:ring}{(not necessarily commutative) ring} and let $A$ be an \CrefAndHyperrefIfExist{definition:algebra_of_a_ring}{$R$-algebra}. Given a subset $S \subseteq A$, the \hldef{subalgebra of $A$ generated by $S$}, denoted by notations such as \hl{$\langle S \rangle = \langle S \rangle_R$}, is the smallest \CrefAndHyperrefIfExist{definition:subalgebra_of_an_algebra_over_a_not_necessarily_commutative_ring}{$R$-subalgebra} of $A$ that contains $S$.

    Explicitly, the \hldef{subalgebra generated by $S$} is the intersection 
    $$\langle S \rangle_R = \bigcap_{T \subseteq S \text{ subalgebra}} T$$
    of all $R$-subalgebras of $A$ containing $S$. Equivalently, it consists of all $R$-linear combinations of finite products of elements from $S$ and from the multiplicative identity element $1_A$.

    In case that $R$ and $A$ are both commutative, the subalgebra generated by $S$ may be denoted by notations such as \hl{$R[S]$}.
\end{definition}