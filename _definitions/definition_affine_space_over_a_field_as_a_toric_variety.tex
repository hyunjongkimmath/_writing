\begin{definition} \label{definition:affine_space_over_a_field_as_a_toric_variety}
The \CrefAndHyperrefIfExist{definition:affine_space_of_dimension_n_over_a_scheme}{affine space} $\mathbb{A}^n_k$ over a field $k$ is the simplest example of a toric variety. The torus $T = (\mathbb{G}_m)^n$ acts on $\mathbb{A}^n$ by coordinate-wise multiplication:
$$(t_1, \dots, t_n) \cdot (x_1, \dots, x_n) = (t_1 x_1, \dots, t_n x_n).$$
The \CrefAndHyperrefIfExist{theorem:fans_of_a_real_fin_dim_vector_space_correspond_to_normal_separated_toric_varieties_over_an_algebraicaly_closed_field}{corresponding} \CrefAndHyperrefIfExist{definition:fan_in_a_finite_dimensional_real_vector_space}{fan} $\Sigma_{\mathbb{A}^n}$ in $\mathbb{R}^n$ consists of the single maximal cone $\sigma = \operatorname{cone}(e_1, \dots, e_n)$ generated by the standard basis vectors, along with all its faces.
\end{definition}