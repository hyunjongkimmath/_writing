\begin{definition} \label{definition:hecke_character_of_a_number_field}
\TODO{TODO: can this be more generally defined for function fields?}

Let $K$ be a global field. A \hldef{Hecke character} (or \hldef{Größencharacter}) of $K$ is a \hyperrefIfExists{definition:quasi_character_of_a_locally_compact_hausdorff_group}{character}/continuous group homomorphism
$$ \chi : \mathbb{A}_K^\times \to \mathbb{C}^\times $$
from the \hyperrefIfExists{definition:adeles_and_ideles_of_a_global_field}{id\`ele group $\mathbb{A}_K^\times$} such that there exists an integer $m \geq 0$ (the \hl{$\infty$-type}) and a finite character $\chi_f$ of the \hyperrefIfExists{definition:adeles_and_ideles_of_a_global_field}{finite idèles} satisfying
$$ \chi(x) = \chi_f(x) |x|^m $$
for all $x \in K^\times$ embedded diagonally in $\mathbb{A}_K^\times$, or more generally: $\chi$ is trivial on $K^\times$ (viewed as a subgroup of $\mathbb{A}_K^\times$ by the diagonal embedding). If $|\chi(a)| = 1$ for all $a \in \bbA_K^\times$, then $\chi$ is called a \hldef{unitary Hecke character}.
\TODO{TODO: define the weight of a hecke character}
\end{definition}
