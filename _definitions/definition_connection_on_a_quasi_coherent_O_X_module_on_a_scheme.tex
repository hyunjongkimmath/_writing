
\begin{definition} \label{definition:connection_on_a_quasi_coherent_O_X_module_on_a_scheme_over_a_ring}
Let \(X\) be a scheme over a ring \(k\) and let \(\mathcal{E}\), \(\mathcal{F}\) be quasi-coherent \(\mathcal{O}_X\)-modules. 

\begin{enumerate}
  \item A \hldef{connection on $\mathcal{E}$} is a \(k\)-linear map
  \[
    \nabla : \mathcal{E} \to \mathcal{E} \otimes_{\mathcal{O}_X} \Omega^1_{X/k}
  \]
  satisfying the Leibniz rule,
  \[
    \nabla(f s) = s \otimes df + f \nabla(s)
  \]
  for all sections \(f \in \mathcal{O}_X\) and \(s \in \mathcal{E}\).
  
  \item The \hldef{curvature of the connection $\nabla$} is the sheaf morphism
  $$\hlin{\nabla^2 : \mathcal{E} \to \mathcal{E} \otimes_{\mathcal{O}_X} \Omega^2_{X/k}}$$
  defined by the composition
  \[
    \nabla^2 := (\nabla \otimes \mathrm{id}) \circ \nabla.
  \]
  \item A connection \(\nabla\) on $\mathcal{E}$ is called \hldef{flat} (or \hldef{integrable}) if \(\nabla^2 = 0\).
  
  \item Given two connections \((\mathcal{E}, \nabla_{\mathcal{E}})\) and \((\mathcal{F}, \nabla_{\mathcal{F}})\), a \hldef{morphism of connections} is an \(\mathcal{O}_X\)-module homomorphism
  \[
    \varphi : \mathcal{E} \to \mathcal{F}
  \]
  such that the following diagram commutes:
  \[
  \begin{tikzcd}
    \mathcal{E} \arrow[r, "\varphi"] \arrow[d, "\nabla_{\mathcal{E}}"'] & \mathcal{F} \arrow[d, "\nabla_{\mathcal{F}}"] \\
    \mathcal{E} \otimes_{\mathcal{O}_X} \Omega^1_{X/k} \arrow[r, "\varphi \otimes \mathrm{id}"] & \mathcal{F} \otimes_{\mathcal{O}_X} \Omega^1_{X/k}
  \end{tikzcd}
  \]
  that is,
  \[
    \nabla_{\mathcal{F}} \circ \varphi = (\varphi \otimes \mathrm{id}) \circ \nabla_{\mathcal{E}}.
  \]
\end{enumerate}
\TextIfExists{definition:relative_connection_on_a_quasi_coherent_O_X_module}{Equivalently, these notions may be regarded as special cases of the notions of \Cref{definition:relative_connection_on_a_quasi_coherent_O_X_module} in case that the base scheme is (the spectrum of) a ring.}
\end{definition}
