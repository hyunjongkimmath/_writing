\begin{definition}[Sheafified Tensor Product] \label{definition:sheafified_tensor_product_of_sheaves_of_modules_of_sheaves_of_rings_on_a_site_with_a_small_topologically_generating_family}
    Let $(\calD, K)$ be a \CrefAndHyperrefIfExist{definition:grothendieck_topology_on_a_category_site_covering_sieve_topologically_generating_family}{site} that admits a \CrefAndHyperrefIfExist{definition:grothendieck_topology_on_a_category_site_covering_sieve_topologically_generating_family}{small topologically generating family} (e.g. by being \CrefAndHyperrefIfExist{definition:essentially_small_site}{essentially small}) and let $\calR, \calS, \calT$ be \CrefAndHyperrefIfExist{definition:sheaf_on_a_site}{sheaves of rings} on $(\calD, K)$. Let $\calM$ be a \CrefAndHyperrefIfExist{definition:sheaf_on_a_site}{sheaf} of \CrefAndHyperrefIfExist{definition:module_over_a_sheaf_of_rings_on_a_site}{$\calR$-$\calS$ bimodules} and $\calN$ be a sheaf of $\calS$-$\calT$ bimodules. 
    The \hldef{sheafified tensor product}, denoted \hl{$\calM \otimes_{\calS} \calN$}, is the sheaf of $\calR$-$\calT$ bimodules defined as the \CrefAndHyperrefIfExist{definition:sheafification_functor_on_a_site}{sheafification} of the section-wise tensor product presheaf:
    \[ U \mapsto \calM(U) \otimes_{\calS(U)} \calN(U). \]
    \CrefIfExists{definition:tensor_product_of_bimodules_of_rings}
    Explicitly, $\calM \otimes_{\calS} \calN = a(U \mapsto \calM(U) \otimes_{\calS(U)} \calN(U))$, where $a$ is the sheafification functor, which exists by \Cref{theorem:sheafification_of_a_presheaf_of_sets_on_a_small_enough_site}. \TODO{to apply the theorem, it seems necessary to show that the category of sheaves of $\calR$-$\calT$ bimodules is complete, cocomplete (probably need sheafification of pointwise construction), has exact filtered colimits, and the IPC property}

    In particular, we have a \CrefAndHyperrefIfExist{definition:n_ary_additive_functor_between_additive_categories}{bi-additive functor}
    $$\otimes: {}_{\calR} \mathsf{Mod}_{\calS} \times {}_{\calS} \mathsf{Mod}_{\calT} \to  {}_{\calR} \mathsf{Mod}_{\calT}$$


    \TextIfExists{definition:sheafy_functor_of_a_functor_between_two_categories_for_the_categories_of_sheaves_on_a_site}{
    In the framework of \Cref{definition:sheafy_functor_of_a_functor_between_two_categories_for_the_categories_of_sheaves_on_a_site}, the sheafified tensor product is the evaluation of the sheafy functor $\mathscr{F}$ induced by the \CrefAndHyperref{definition:tensor_product_functor_on_the_category_of_tensor_composable_bimodules_over_all_rings}{tensor product functor} $F = \otimes: \mathcal{B}imod \times_{\mathcal{R}ing} \mathcal{B}imod \to \mathcal{B}imod$ (\Cref{definition:category_of_all_bimodules_over_all_rings}) (\Cref{definition:category_of_tensor_composable_bimodules_over_all_rings}); \TODO{must show that the category $\mathcal{B}imod$ is complete, cocomplete, has exact essentially small filtered limits, and satisfies the IPC property to argue that there is a sheafification functor by \Cref{theorem:sheafification_of_a_presheaf_of_sets_on_a_small_enough_site}} \TODO{show that the ``composable'' subcategory of sheaves of bimodules over all sheaves of rings on the site is equivalence to the category of sheaves valued in the caetgory of all composable bimodules}
    Specifically, if $\mathbb{M} := (\calR, \calS, \calM) \in \Sh(\calD, K; \mathcal{B}imod)$ and $\mathbb{N} := (\calS, \calT, \calN) \in \Sh(\calD, K; \mathcal{B}imod)$, then the sheafified tensor product is the module component of the sheaf $\mathscr{F}(\mathbb{M}, \mathbb{N}) \in \Sh(\calD, K; \mathcal{B}imod)$, which is the triple $(\calR, \calT, \calM \otimes_{\calS} \calN)$.
    }

    % In the framework of \Cref{definition:sheafy_functor_of_a_functor_between_two_categories_for_the_categories_of_sheaves_on_a_site}, the sheafified tensor product is the sheafy functor $\mathscr{F}$ induced by the algebraic tensor product functor $F: \calA \times \calB \to \calC$ where $\calA, \calB, \calC$ are the categories of bimodules over rings. It satisfies assumption (1) of said definition, as the section-wise tensor product generally fails to be a sheaf and requires the sheafification functor $a$.
\end{definition}