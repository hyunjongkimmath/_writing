
\begin{definition} \label{definition:ideal_of_a_ring}
Let $R$ be a (not necessarily commutative, possibly nonunital) \CrefAndHyperrefIfExist{definition:ring}{ring}.  
A \hldef{left ideal of $R$} is a subset $I \subseteq R$ such that
\begin{itemize}
    \item $(I,+)$ is an additive \CrefAndHyperrefIfExist{definition:subgroup_of_a_group}{subgroup} of $(R,+)$,
    \item $RI \subseteq I$, i.e., for all $r \in R$ and $x \in I$, one has $rx \in I$.
\end{itemize}
Similarly, a \hldef{right ideal of $R$} is a subset $I \subseteq R$ such that
\begin{itemize}
    \item $(I,+)$ is an additive subgroup of $(R,+)$,
    \item $IR \subseteq I$, i.e., for all $r \in R$ and $x \in I$, one has $xr \in I$.
\end{itemize}
A \hldef{two-sided ideal} (or simply an \hldef{ideal}) of $R$ is a subset $I \subseteq R$ which is both a left ideal and a right ideal of $R$. We denote by \hl{$I \unlhd R$} the relation expressing that $I$ is a two-sided ideal of $R$.

\TextIfExists{definition:module_of_a_ring}{Equivalently, an left/right/two-sided ideal of $R$ is a \CrefAndHyperrefIfExist{definition:submodule_of_a_module_over_a_ring}{submodule} of $R$ as an \CrefAndHyperrefIfExist{definition:module_of_a_ring}{$R$-module}.}

A left/right/two-sided ideal is said to be \hldef{proper} if it is strictly contained in $R$.

Note that every left or right ideal of a commutative ring is a two-sided ideal.
\end{definition}
