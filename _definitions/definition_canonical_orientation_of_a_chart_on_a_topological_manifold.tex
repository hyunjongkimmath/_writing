\begin{definition} \label{definition:canonical_orientation_of_a_chart_on_a_topological_manifold}
% Let $M$ be an $n$-dimensional topological manifold, and let $(U, \varphi)$ be a \CrefAndHyperrefIfExist{definition:chart_on_a_topological_manifold}{chart on $M$}.

% % The \hldef{orientation of the chart $(U,\varhpi)$} is the orientation on $U$ obtained by pulling back the standard

% An \hldef{orientation for the chart $(U, \varphi)$} is a choice of equivalence class of ordered bases of $\mathbb{R}^n$ at each point of $V$ consistent under orientation-preserving homeomorphisms of $V$.

% More concretely, such an orientation for $(U, \varphi)$ is an equivalence class determined by the \hyperrefIfExists{definition:standard_orientation_on_R_n}{standard orientation of $\mathbb{R}^n$}. The chart $(U, \varphi)$ is called \hldef{oriented} if $\varphi$ is orientation-preserving, i.e., its transition maps with other charts respect the chosen orientation.
% \end{definition}