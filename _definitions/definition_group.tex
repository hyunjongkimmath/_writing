
\begin{definition}[Groups] \label{definition:group}
A \hldef{group} is a pair $(G,\cdot)$ where $G$ is a set and $\cdot : G \times G \to G$ is a binary operation, subject to the following conditions:

1. (Associativity) For all $g,h,k \in G$ one has 
$$ (g \cdot h) \cdot k = g \cdot (h \cdot k). $$

2. (Identity element) There exists an element \hl{$e \in G$} such that for all $g \in G$, 
$$ e \cdot g = g \cdot e = g. $$

3. (Inverse element) For all $g \in G$ there exists an element \hl{$g^{-1} \in G$} such that 
$$ g \cdot g^{-1} = g^{-1} \cdot g = e. $$

The element $e$ is called the \hldef{identity element of $G$}, and $g^{-1}$ is called the \hldef{inverse of $g$}.

Equivalently, a group is a \CrefAndHyperrefIfExist{definition:monoid}{monoid} with inverse elements.

\TextIfExists{definition:group_object_in_a_category_with_a_final_object}{Equivalently, a group is a \CrefAndHyperrefIfExist{definition:group_object_in_a_category_with_a_final_object}{group object} in the \CrefAndHyperrefIfExist{definition:category_of_sets}{category of sets}.}

A group $(G, \cdot)$ is often simply written as $G$, when the notation for the binary operation $\cdot$ is clear. 

An \hldef{abelian group} or synonymously, a \hldef{commutative group}, is a group $(G,\cdot)$ whose binary operation $\cdot$ is \CrefAndHyperrefIfExist{definition:commutative_binary_operation}{\hldef{abelian} or \hldef{commutative}}, i.e. satisfies 
$$g \cdot h = h \cdot g$$
for all $g,h \in G$. 


\TextIfExists{definition:module_of_a_ring}{An abelian group is equivalent to a $\bbZ$-module.}

\end{definition}
