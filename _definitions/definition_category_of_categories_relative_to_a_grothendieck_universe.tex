\begin{definition} \label{definition:category_of_categories_relative_to_a_grothendieck_universe}
Let $\mathcal{U}$ be a fixed \CrefAndHyperrefIfExist{definition:grothendieck_universe}{Grothendieck universe}. The \hldef{category of categories (relative to $\mathcal{U}$)} or the \hldef{category of $U$-small categories} is the category defined by:
\begin{itemize}
    \item The objects are all \CrefAndHyperrefIfExist{definition:locally_small_category}{$U$-small} \CrefAndHyperrefIfExist{definition:category}{categories} $\mathcal{C}$.
    \item The morphisms are \CrefAndHyperrefIfExist{definition:functor_between_categories}{functors} between such categories.
\end{itemize}
This category is denoted by \hl{$U-\mathbf{Cat}$} or by \hl{$\mathbf{Cat}$}, if $U$ is understood.

In the case that \CrefAndHyperrefIfExist{definition:tarski_grothendieck_set_theory}{Tarski-Grothendieck} set theory is assumed (so in particular, the axiom of universes is assumed), one often adopts the convention of denoting $U_n-\mathbf{Cat}$ (\Cref{definition:hierarchy_of_grothendieck_universes_assuming_tarski_grothendieck_set_theory}) by \hl{$\mathbf{Cat}$} and $U_{n+1}-\mathbf{Cat}$ (\Cref{definition:hierarchy_of_grothendieck_universes_assuming_tarski_grothendieck_set_theory}) by \hl{$\mathbf{CAT}$}. Thus, for $n = 1$, $\mathbf{Cat}$ would be the category of small categories, whereas $\mathbf{CAT}$ would serve as the ``category of large categories'' which consists of many common categories that are too large to be $U_1$-categories such as $\mathbf{Set}$, $\mathbf{Grp}$, $\mathbf{Top}$ or even $\mathbf{Cat}$.

% denotes $U_2-\mathbf{Cat}$ (\Cref{definition:hierarchy_of_grothendieck_universes_assuming_tarski_grothendieck_set_theory}) by \hl{$\mathbf{CAT}$}.
% In this context, $\mathbf{Cat}$ (the category of categories belonging to $\mathcal{U}$) is an object of $\mathbf{CAT}$.
\end{definition}