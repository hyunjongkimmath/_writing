% \begin{definition}[Topos] \label{definition:topos}
%     There are a multitude of notions of topos. Here are some that we consider; more notions may be added later.
%     \begin{enumerate}
%         \item A \hldef{(sheaf/Grothendieck) topos} is a \CrefAndHyperrefIfExist{definition:category}{category} \CrefAndHyperrefIfExist{definition:equivalence_of_categories}{equivalent} to the category of \CrefAndHyperrefIfExist{definition:sheaf_on_a_site}{sheaves} of sets on some \CrefAndHyperrefIfExist{definition:grothendieck_topology_on_a_category_site_covering_sieve_topologically_generating_family}{site}. That is, there exists a site $(C, J)$ such that the category is equivalent to $\operatorname{Sh}(C, J)$, the category of sheaves of sets on $(C, J)$.
%         \item Let $U$ be a universe. A \hldef{$U$-(sheaf )topos} is a category equivalent to the category of \hyperrefIfExists{definition:sheaf_on_a_site}{$U$-sheaves}\CrefIfExists{definition:sheaf_on_a_site} (valued in $U$-sets) \cite[Expos\'e IV D\'efinition 1.1]{SGA4_I}

%         \item An \hldef{elementary topos} is a cateogry which has all finite \CrefAndHyperrefIfExist{definition:limit_and_colimit_of_a_diagram_in_a_category}{limits}, is cartesian closed, and has a subobject classifier \TODO{cartesian closed, subobject classifier}
%     \end{enumerate}
% \end{definition}

\begin{definition}[Topos] \label{definition:topos}
    There are multiple notions of a topos depending on the context (geometric vs. logical).
    \begin{enumerate}
        \item A \hldef{Grothendieck topos} (or \hldef{sheaf topos}) is a \CrefAndHyperrefIfExist{definition:category}{category} \CrefAndHyperrefIfExist{definition:equivalence_of_categories}{equivalent} to the category of \CrefAndHyperrefIfExist{definition:sheaf_on_a_site}{sheaves} of sets on a \hldef{small} \CrefAndHyperrefIfExist{definition:grothendieck_topology_on_a_category_site_covering_sieve_topologically_generating_family}{site}. That is, there exists a small site $(\mathcal{C}, J)$ such that the category is equivalent to $\operatorname{Sh}(\mathcal{C}, J)$.
        
        \item Let $\mathscr{U}$ be a \hyperrefIfExists{definition:grothendieck_universe}{universe}\CrefIfExists{definition:grothendieck_universe}. A \hldef{$\mathscr{U}$-topos} is a category equivalent to the category of sheaves of sets on a $\mathscr{U}$-small site $(\mathcal{C}, J)$, where the sheaves take values in the category of $\mathscr{U}$-sets ($\mathbf{Set}_{\mathscr{U}}$). \cite[Expos\'e IV D\'efinition 1.1]{SGA4_I}

        \item An \hldef{elementary topos} is a category which has all finite \CrefAndHyperrefIfExist{definition:limit_and_colimit_of_a_diagram_in_a_category}{limits}, is \CrefAndHyperrefIfExist{definition:cartesian_closed_category}{cartesian closed}, and has a \CrefAndHyperrefIfExist{definition:subobject_classifier_in_a_category_with_a_final_object}{subobject classifier}.
    \end{enumerate}
    \textit{Remark:} Every Grothendieck topos is an elementary topos, but the converse is not true (e.g., the category of finite sets is an elementary topos but not a Grothendieck topos).
\end{definition}


% {\cite[Expos\'e IV D\'efinition 1.1]{SGA4_I}}
% Let $\scrU$ be a fixed universe. A \hldef{$\scrU$-topos}, or simply \hldef{topos} if there is no confusion, $E$ is a category that is equivalent to the category $\Shv(T)$ of sheaves of sets on a fixed site $T$ in $\scrU$.