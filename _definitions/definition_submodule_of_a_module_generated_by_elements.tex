\begin{definition}[Submodule generated by elements in an $(R,S)$-bimodule] \label{definition:submodule_of_a_module_generated_by_elements}
    Let $R$ and $S$ be \CrefAndHyperrefIfExist{definition:ring}{(not necessarily commutative) rings}. 
    \begin{enumerate}
        \item 
        Let $M$ be an \CrefAndHyperrefIfExist{definition:module_of_a_ring}{$(R,S)$-bimodule}.

        Given a subset $X \subseteq M$, the \hldef{sub-bimodule of $M$ generated by $X$} is the smallest $(R,S)$-sub-bimodule of $M$ containing $X$. It is often denoted by notations such as \hl{$\langle X \rangle = \langle X \rangle_{R,S}$} and is more explicitly the intersection
        $$\langle X \rangle_{R,S} = \bigcap_{X \subseteq T \subseteq M, T \text{ is a } (R,S)\text{-submodule of } M} T$$
        of al $(R,S)$-submodules of $M$ containing $X$. 
        
        Equivalently, $\langle X \rangle_{R,S}$ consists of all \CrefAndHyperrefIfExist{definition:linear_combination_of_elements_in_a_module}{linear combinations} of $X$. 

        \item If $M$ is a left/right/two-sided $R$-module and given a subset $X \subseteq M$, the \hldef{submodule of $M$ generated by $X$} is the submodule of the \CrefAndHyperrefIfExist{definition:module_of_a_ring}{natural bimodule} of $M$ generated by $X$. It is denoted by notations such as $\langle X \rangle = \langle X \rangle_R$. 
    \end{enumerate}
    % Explicitly, this sub-bimodule consists of all finite sums of elements of the form
    % \[
    % \hl{$r \cdot x \cdot s$}
    % \]
    % where $r \in R$, $s \in S$, and $x \in X$. That is,
    % \[
    % \hl{$\langle X \rangle_{R,S} = \left\{ \sum_{i=1}^n r_i x_i s_i \mid n \in \mathbb{N}, r_i \in R, s_i \in S, x_i \in X \right\}$}.
    % \]
    % This sub-bimodule is the smallest $(R,S)$-bimodule containing $X$, closed under the actions of $R$ and $S$ and addition.
\end{definition}