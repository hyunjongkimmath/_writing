\begin{definition} \label{definition:weak_n_category_tamsamani_simpson}
A \hldef{weak $n$-category} (in the sense of Tamsamani and Simpson) is defined inductively on $n \ge 0$.

\textbf{Base Case ($n=0$):} A weak 0-category is a set.

\textbf{Inductive Step:} For $n \ge 1$, a \hldef{weak $n$-category} is a simplicial object $X_\bullet$ in the category of weak $(n-1)$-categories, denoted $X: \Delta^{op} \to \text{Weak-}(n-1)\text{-Cat}$, satisfying the Segal condition (composition) and the discreteness condition (identities).

Explicitly, $X$ consists of:
\begin{itemize}
    \item A sequence of weak $(n-1)$-categories $X_k$ for each integer $k \ge 0$ (where $X_0$ represents the objects, $X_1$ the morphisms, $X_2$ composable pairs, etc.);
    \item Face maps $d_i: X_k \to X_{k-1}$ and degeneracy maps $s_i: X_k \to X_{k+1}$ in the category of weak $(n-1)$-categories, satisfying the simplicial identities.
\end{itemize}
These data must satisfy the following properties:
\begin{enumerate}
    \item \textbf{Discreteness (Units):} The object of 0-simplices $X_0$ is "discrete" in the sense that the underlying simplicial set of $X_0$ is constant (homotopically equivalent to a set).
    
    \item \textbf{Segal Condition (Composition):} For every $k \ge 2$, the map induced by the simplicial segment inclusions
    $$
    \mu_k: X_k \xrightarrow{\approx} X_1 \times_{X_0} X_1 \times_{X_0} \dots \times_{X_0} X_1 \quad (k \text{ times})
    $$
    is an \hldef{equivalence} of weak $(n-1)$-categories. 
    
    Here, the fiber product is formed using the source and target maps ($d_1, d_0: X_1 \to X_0$). The "equivalence" is the notion of equivalence appropriate for $(n-1)$-categories (defined inductively as maps inducing weak homotopy equivalences on all hom-spaces).
\end{enumerate}
\end{definition}