\begin{definition} \label{definition:canonical_truncation_of_chain_complexes_of_objects_in_an_abelian_category}

    Let $\calA$ be an \CrefAndHyperrefIfExist{definition:abelian_category}{abelian category}.

    \begin{enumerate}
        \item  
    The \hldef{canonical truncations} of a 
    \CrefAndHyperrefIfExist{definition:chain_complex_of_objects_in_an_additive_category}{chain complex} 
    $A_\bullet$ in 
    \CrefAndHyperrefIfExist{definition:chain_complex_of_objects_in_an_additive_category}{$\mathbf{Ch}(\mathcal{A})$} 
    are defined by

    \hlalign{
    \begin{align*}
    (\tau_{\ge n} A)_i &=
    \begin{cases}
        A_i, & i > n, \\
        \ker(d_n : A_n \to A_{n-1}), & i = n, \\
        0, & i < n,
    \end{cases}
    &
    (\tau_{\le n} A)_i &=
    \begin{cases}
        0, & i > n, \\
        \mathrm{coker}(d_{n+1} : A_{n+1} \to A_n), & i = n, \\
        A_i, & i < n.
    \end{cases}
    \end{align*}
    }

    The differentials are the restrictions and/or quotient maps induced from $A_\bullet$. 
    In particular,
    $$
    H_i(\tau_{\ge n} A_\bullet) = 
    \begin{cases}
        H_i(A_\bullet), & i \ge n, \\
        0, & i < n,
    \end{cases}
    \quad\text{and}\quad
    H_i(\tau_{\le n} A_\bullet) = 
    \begin{cases}
        H_i(A_\bullet), & i \le n, \\
        0, & i > n.
    \end{cases}
    $$
    The assignments $A_\bullet \mapsto \tau_{\ge n} A_\bullet$ and $A_\bullet \mapsto \tau_{\le n} A_\bullet$ extend to endofunctors

    \hlalign{
    \begin{align*}
    \tau_{\ge n},\, \tau_{\le n} : \mathbf{Ch}(\mathcal{A}) \to \mathbf{Ch}(\mathcal{A}),
    \end{align*}
    }

    called the \hldef{truncation functors}. They are natural in both $A_\bullet$ and $n$, and fit into canonical morphisms of complexes
    $$
    \tau_{\ge n} A_\bullet \longrightarrow A_\bullet \longrightarrow \tau_{\le n} A_\bullet.
    $$

    \item 
    \noindent
    Similarly, let $A^\bullet$ be a 
    \CrefAndHyperrefIfExist{definition:cochain_complex_of_objects_in_an_additive_category}{cochain complex} 
    in $\mathbf{Ch}(\mathcal{A})$, i.e.
    $$
    \cdots \xrightarrow{d^{n-2}} A^{n-1} \xrightarrow{d^{n-1}} A^n \xrightarrow{d^n} A^{n+1} \xrightarrow{d^{n+1}} \cdots
    $$
    with $d^{n+1} \circ d^n = 0$ for all $n \in \mathbb{Z}$. 
    The \hldef{canonical truncations of $A^\bullet$} are defined by

    \hlalign{
    \begin{align*}
    (\tau_{\le n} A)^i &=
    \begin{cases}
        A^i, & i < n, \\
        \ker(d^n : A^n \to A^{n+1}), & i = n, \\
        0, & i > n,
    \end{cases}
    &
    (\tau_{\ge n} A)^i &=
    \begin{cases}
        0, & i < n, \\
        \mathrm{coker}(d^{n-1} : A^{n-1} \to A^n), & i = n, \\
        A^i, & i > n.
    \end{cases}
    \end{align*}
    }

    The differentials are the restrictions or quotient maps induced by those of $A^\bullet$. 
    These truncations satisfy
    $$
    H^i(\tau_{\le n} A^\bullet) = 
    \begin{cases}
        H^i(A^\bullet), & i \le n, \\
        0, & i > n,
    \end{cases}
    \quad\text{and}\quad
    H^i(\tau_{\ge n} A^\bullet) = 
    \begin{cases}
        0, & i < n, \\
        H^i(A^\bullet), & i \ge n.
    \end{cases}
    $$

    They also extend to endofunctors

    \hlalign{
    \begin{align*}
    \tau_{\le n},\, \tau_{\ge n} : \mathbf{Ch}(\mathcal{A}) \to \mathbf{Ch}(\mathcal{A}),
    \end{align*}
    }

    natural in both $A^\bullet$ and $n$, fitting into canonical morphisms of cochain complexes
    $$
    \tau_{\le n} A^\bullet \longrightarrow A^\bullet \longrightarrow \tau_{\ge n} A^\bullet.
    $$

    \end{enumerate}

%     Let $\calA$ be an \CrefAndHyperrefIfExist{definition:abelian_category}{abelian category}.
% The \hldef{canonical truncations} of a \CrefAndHyperrefIfExist{definition:chain_complex_of_objects_in_an_additive_category}{chain complex} $A_\bullet$ in \CrefAndHyperrefIfExist{definition:chain_complex_of_objects_in_an_additive_category}{$\mathbf{Ch}(\mathcal{A})$} are defined by

% \hlalign{
% \begin{align*}
% (\tau_{\ge n} A)_i &=
%   \begin{cases}
%     A_i, & i > n, \\
%     \ker(d_n : A_n \to A_{n-1}), & i = n, \\
%     0, & i < n,
%   \end{cases}
% &
% (\tau_{\le n} A)_i &=
%   \begin{cases}
%     0, & i > n, \\
%     \mathrm{coker}(d_{n+1} : A_{n+1} \to A_n), & i = n, \\
%     A_i, & i < n.
%   \end{cases}
% \end{align*}
% }

% The differentials are the restrictions and/or quotient maps induced from $A_\bullet$. 
% In particular,
% $$
% H_i(\tau_{\ge n} A_\bullet) = 
%   \begin{cases}
%     H_i(A_\bullet), & i \ge n, \\
%     0, & i < n,
%   \end{cases}
% \quad\text{and}\quad
% H_i(\tau_{\le n} A_\bullet) = 
%   \begin{cases}
%     H_i(A_\bullet), & i \le n, \\
%     0, & i > n.
%   \end{cases}
% $$

% The assignments $A_\bullet \mapsto \tau_{\ge n} A_\bullet$ and $A_\bullet \mapsto \tau_{\le n} A_\bullet$ extend to endofunctors

% \hlalign{
% \begin{align*}
% \tau_{\ge n},\, \tau_{\le n} : \mathbf{Ch}(\mathcal{A}) \to \mathbf{Ch}(\mathcal{A}),
% \end{align*}
% }

% called the \hldef{truncation functors}. They are natural in both $A_\bullet$ and $n$, and fit into canonical morphisms of complexes
% $$
% \tau_{\ge n} A_\bullet \longrightarrow A_\bullet \longrightarrow \tau_{\le n} A_\bullet.
% $$
\end{definition}