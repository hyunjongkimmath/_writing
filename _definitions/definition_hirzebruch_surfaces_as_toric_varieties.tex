\begin{definition} \label{definition:hirzebruch_surfaces_as_toric_varieties}
Let $n \ge 0$. The \hldef{Hirzebruch surface} $\mathbb{F}_n$ is a smooth complete toric surface. It is the toric variety \CrefAndHyperrefIfExist{theorem:fans_of_a_real_fin_dim_vector_space_correspond_to_normal_separated_toric_varieties_over_an_algebraicaly_closed_field}{associated} to the \CrefAndHyperrefIfExist{definition:fan_in_a_finite_dimensional_real_vector_space}{fan} $\Sigma_n$ in $\mathbb{R}^2$ generated by the ray generators:
$$v_1 = (1, 0), \quad v_2 = (0, 1), \quad v_3 = (-1, n), \quad v_4 = (0, -1).$$
The maximal cones are $\operatorname{cone}(v_1, v_2)$, $\operatorname{cone}(v_2, v_3)$, $\operatorname{cone}(v_3, v_4)$, and $\operatorname{cone}(v_4, v_1)$. Topologically, $\mathbb{F}_n$ is a $\mathbb{P}^1$-bundle over $\mathbb{P}^1$.
\end{definition}