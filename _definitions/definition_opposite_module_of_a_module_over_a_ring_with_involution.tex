\begin{definition} \label{definition:opposite_module_of_a_module_over_a_ring_with_involution}
Let $R$ be a \CrefAndHyperref{definition:involution_on_a_ring}{ring with involution} $r \mapsto \bar{r}$.
\begin{enumerate}
    \item Let $M$ be a \CrefAndHyperref{definition:module_of_a_ring}{left/right $R$-module}. The \hldef{opposite module}, also called the \hldef{conjugate module} or the module by restriction of scalars along the involution (not to be confused with the notion of \CrefAndHyperref{definition:opposite_module_of_a_general_bimodule_over_rings}{opposite modules} for modules over general rings), is the right/left $R$-module, denoted by notations such as \hl{$M^{op}$} or \hl{$\overline{M}$}, with the same underlying abelian group as $M$, where the right $R$-action is defined as follows
    \begin{itemize}
        \item $$m \cdot r = \bar{r} m$$ for all $m \in M$ and $r \in R$ if $M$ is a left $R$-module.

        $$r \cdot m = m \bar{r}$$
        for all $m \in M$ and $r \in R$ if $M$ is a right $R$-module.
    \end{itemize}
\end{enumerate}
\end{definition}