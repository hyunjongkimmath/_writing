\begin{definition} \label{definition:quasi_character_of_a_locally_compact_hausdorff_group}
    Let \( G \) be a locally compact Hausdorff group.
    \begin{enumerate}
        \item A \hldef{quasicharacter of $G$} is a continuous group homomorphism $\chi : G \to \mathbb{C}^\times$.
        \item A quasicharacter \(\chi\) is \hldef{unitary} if its image lies in the unit circle $S^1 \subset \bbC^\times$, i.e. $|\chi(g)| = 1$ for all $g \in G$. Such a quasicharacter is also simply called a \hldef{character}.
        \item A (quasi)character is \hldef{finite} if its image is finite, i.e. its kernel has finite index in its domain.
    \end{enumerate}

    When $G$ is finite, $G$ is usually equipped with the discrete topology --- a character (and even a quasicharacter) is thus simply a group homomorphism $G \to \bbC^\times$ such that $|\chi(g)| = 1$ for all $g \in G$.

    \TextIfExists{definition:additive_and_multiplicative_characters_of_a_ring}{
    Compare against \Cref{definition:additive_and_multiplicative_characters_of_a_ring}}
\end{definition}