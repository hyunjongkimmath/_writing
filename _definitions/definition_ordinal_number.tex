\begin{definition} \label{definition:ordinal_number}
A set $\alpha$ is an \hldef{ordinal number} (or simply an \hldef{ordinal}) if it satisfies the following two conditions:
\begin{enumerate}
    \item $\alpha$ is \hldef{transitive}, meaning that every element of $\alpha$ is also a subset of $\alpha$. That is, if $x \in \alpha$, then $x \subseteq \alpha$.
    \item $\alpha$ is \hldef{strictly well-ordered} by the membership relation $\in$. That is, the relation $<$ on $\alpha$ defined by $x < y \iff x \in y$ is a strict total ordering, and every non-empty subset of $\alpha$ has a least element under this ordering.
\end{enumerate}
The class of all ordinal numbers is denoted by \hl{$\operatorname{Ord}$} or \hl{$\mathbf{ON}$}. For ordinals $\alpha$ and $\beta$, we write \hl{$\alpha < \beta$} when $\alpha \in \beta$, and we write and \hl{$\alpha \le \beta$}  when $\alpha \in \beta$ or $\alpha = \beta$.
\end{definition}