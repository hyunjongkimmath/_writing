
\begin{definition} \label{definition:ideal_generated_by_a_subset_of_a_ring}
    Let $R$ be a \CrefAndHyperrefIfExist{definition:ring}{(not necessarily commutative) ring}, and let $X \subseteq R$ be a \CrefAndHyperrefIfExist{definition:subset_of_a_set}{subset}.

    The \hldef{left ideal generated by $X$} is the smallest \CrefAndHyperref{definition:ideal_of_a_ring}{left ideal} of $R$ containing $X$; it equals the set of all finite sums of elements of the form $r x$ with $r \in R$ and $x \in X$.

    Similarly, the \hldef{right ideal generated by $X$} is the smallest right ideal of $R$ containing $X$; it equals the set of all finite sums of elements of the form $x r$ with $x \in X$ and $r \in R$.

    The \hldef{two-sided ideal generated by $X$} is the smallest two-sided ideal of $R$ containing $X$; it equals the set of all finite sums of elements of the form $r x s$ with $r, s \in R$ and $x \in X$.

    A left/right/two-sided ideal $I$ of $R$ is said to be \hldef{finitely generated} if there exists some finite subset $X$ of $R$ such that $I$ equals the left/right/two-sided ideal generated by $X$. Moreover, $I$ is said to be \hldef{principal} if there exists some subset $X$ of $R$ of cardinality $1$ such that $I$ equals the left/right/two-sided ideal generated by $X$. 

\end{definition}
