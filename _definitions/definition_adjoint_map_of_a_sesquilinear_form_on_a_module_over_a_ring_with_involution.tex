% \begin{definition} \label{definition:adjoint_map_of_a_sesquilinear_form_on_a_module_over_a_ring_with_involution}
% Let $(A, \sigma)$ be a ring with involution and $M$ a right $A$-module equipped with a sesquilinear form $b: M \times M \to A$. The \hldef{adjoint map of $b$} is the morphism of $A$-modules
% $$ \hat{b}: M \to M^*, \quad u \mapsto (v \mapsto b(u, v)) $$
% where $M^* = \operatorname{Hom}_A(M, A)_\sigma$ is the \CrefAndHyperrefIfExist{definition:twisted_dual_module_of_a_module_over_a_ring_with_involution}{twisted dual module}. 
% \end{definition}

\begin{definition} \label{definition:adjoint_map_of_a_sesquilinear_form_on_a_module_over_a_ring_with_involution}
Let $(A, \sigma)$ be a \CrefAndHyperrefIfExist{definition:involution_on_a_ring}{ring with involution} and $M$ a \CrefAndHyperrefIfExist{definition:module_of_a_ring}{right $A$-module} equipped with a \CrefAndHyperrefIfExist{definition:sesquilinear_form_on_a_module_over_a_ring_with_involution}{sesquilinear form} $b: M \times M \to A$. We define two $A$-linear mappings associated with $b$:
\begin{enumerate}
    \item The \hldef{left adjoint map} is the morphism of $A$-modules:
    $$ \hat{b}_l: M \to M^*, \quad u \mapsto (v \mapsto b(u, v)) $$
    where $M^* = \operatorname{Hom}_A(M, A)_\sigma$ is the \CrefAndHyperrefIfExist{definition:twisted_dual_module_of_a_module_over_a_ring_with_involution}{twisted dual module}.
    \item The \hldef{right adjoint map} is the morphism of $A$-modules:
    $$ \hat{b}_r: M \to M^*, \quad v \mapsto (u \mapsto \sigma(b(u, v))) $$
\end{enumerate}
If $b$ is an \CrefAndHyperrefIfExist{definition:varepsilon_hermitian_form_on_a_module_over_a_ring_with_involution}{$\varepsilon$-hermitian form}, these maps are related by $\hat{b}_l = \hat{b}_r \cdot \varepsilon$ (under the standard identification of the module and its dual).

When we simply talk about the \hldef{adjoint map} of $b$ on the right $A$-module $M$, we will mean the left adjoint and denote it by \hl{$\hat{b}$}.
\end{definition}