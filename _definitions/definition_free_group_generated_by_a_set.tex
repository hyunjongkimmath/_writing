\begin{definition} \label{definition:free_group_generated_by_a_set}
Let $S$ be a set. The \hldef{free group generated by $S$} is a pair $(F(S), \iota)$ consisting of a \CrefAndHyperrefIfExist{definition:group}{group} $F(S)$ and a function $\iota: S \to F(S)$, satisfying the following universal property: for any group $G$ and any function $f: S \to G$, there exists a unique group homomorphism $\varphi: F(S) \to G$ such that the diagram commutes (i.e., $\varphi \circ \iota = f$).
The standard notation for the free group on $S$ is
\hl{$F(S)$}
or sometimes
\hl{$\langle S \rangle$}
Elements of $F(S)$ are uniquely represented as reduced words in the alphabet $S \cup \{s^{-1} \mid s \in S\}$.
\end{definition}