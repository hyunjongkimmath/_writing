\begin{definition} \label{definition:kummer_sheaf_of_a_character_on_the_tensor_product_of_commutative_group_schemes_over_a_finite_field}
Assume that $G_1$ and $G_2$ are connected commutative group schemes over a finite field $\Fq$ such that the \hyperrefIfExists{definition:tensor_product_of_group_objects_in_a_category}{tensor product $G_1 \otimes G_2$} exists. Continue letting $\ell$ be a prime number distinct from $\Char \Fq$. Given a group homomorphism $b: (G_1 \otimes G_2)(\Fq) \to \Qellbar^\times$, which we intuitively think of as taking the role of a bilinear pairing, we may take its \TODO{TODO: define a Kummer sheaf of a character} \hyperrefIfExists{definition:kummer_sheaf_associated_to_a_character_on_points_of_a_connected_commutative_algebraic_group_over_a_finite_field}{Kummer sheaf}\CrefIfExists{definition:kummer_sheaf_associated_to_a_character_on_points_of_a_connected_commutative_algebraic_group_over_a_finite_field}$\mathcal{L}_b$ on $G_1 \otimes G_2$, which is the rank $1$ local system corresponding to the composition
$$\pioneet(G_1 \otimes G_2, e) \to (G_1 \otimes G_2)(\Fq) \xrightarrow{b} \Qellbar^\times$$  
where the first homomorphism corresponds to the Lang torsor for $(G_1 \otimes G_2)(\Fq)$. Pulling $\calL_b$ to $G_1 \times G_2$ yields a rank $1$ local system on $G_1 \times G_2$, which we again denote by \hl{$\calL_b$} by abuse of notation. 
\end{definition}