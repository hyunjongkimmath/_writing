\begin{definition}[Sheafified Hom] \label{definition:sheaf_hom_for_sheaves_of_modules_over_sheaves_of_rings_on_a_site_with_a_small_topologically_generating_family}
    Let $(\calD, K)$ be a \CrefAndHyperrefIfExist{definition:grothendieck_topology_on_a_category_site_covering_sieve_topologically_generating_family}{site} that admits a \CrefAndHyperrefIfExist{definition:grothendieck_topology_on_a_category_site_covering_sieve_topologically_generating_family}{small topologically generating family} (e.g. by being \CrefAndHyperrefIfExist{definition:essentially_small_site}{essentially small}) and let $\calR, \calS, \calT$ be sheaves of rings on $(\calD, K)$. Let $\calM$ be a \CrefAndHyperrefIfExist{definition:sheaf_on_a_site}{sheaf} of \CrefAndHyperrefIfExist{definition:module_over_a_sheaf_of_rings_on_a_site}{$\calR$-$\calS$ bimodules} and $\calN$ be a sheaf of $\calR$-$\calT$ bimodules.

    The \hldef{sheafified Hom}, denoted by notations such as \hl{$\mathcal{H}om_{\calR}(\calM, \calN)$}, \hl{$\mathscr{H}om_{\calR}(\calM, \calN)$}, or \hl{$\underline{\mathrm{Hom}}_{\calR}(\calM, \calN)$}, is the sheaf of $\calS$-$\calT$ bimodules defined section-wise by:
    \[ \mathcal{H}om_{\calR}(\calM, \calN)(U) = \operatorname{Hom}_{\Sh(\calD/U, K|_U; \operatorname{Mod}(\calR|_U))}(\calM|_U, \calN|_U). \]
    (\Cref{definition:category_of_objects_over_under_a_fixed_object_in_a_category}) (\Cref{definition:site_induced_by_a_site_on_an_over_category}) (\Cref{definition:restriction_of_a_sheaf_on_a_site_to_an_object_of_the_underlying_category_of_the_site}) 

    Equivalently, it is the unique sheaf which satisfies the right adjoint property relative to the sheafified tensor product (\Cref{theorem:tensor_hom_adjucntion_for_sheaves_of_modules_over_sheaves_of_rings_on_a_site_with_a_small_topologically_generating_family}). 

    \TextIfExists{definition:sheafy_functor_of_a_functor_between_two_categories_for_the_categories_of_sheaves_on_a_site}{
    In the framework of \Cref{definition:sheafy_functor_of_a_functor_between_two_categories_for_the_categories_of_sheaves_on_a_site}, the sheafified Hom corresponds to the right adjoint $\mathscr{G}$ to the sheafified tensor product $\mathscr{F}$. While the general framework defines sheafy functors via section-wise application, the internal Hom for sheaves of modules requires the restriction-based definition (using $\calD/U$) to correctly represent the right adjoint in the category of sheaves, ensuring compatibility with the varying ring structure $\calR$.

    \TODO{the following would actually need to be modified to have a notion of ``Hom-composable'' category of functors}
    Also in the framework of \Cref{definition:sheafy_functor_of_a_functor_between_two_categories_for_the_categories_of_sheaves_on_a_site}, the sheafified Hom is the evaluation of the sheafy functor $\mathscr{G}$ induced by the algebraic Hom functor $G: \calA^{\op} \times \calC \to \calB$ under assumption (2). Given input sheaves $\mathbb{M} := (\calR, \calS, \calM)$ and $\mathbb{N} := (\calR, \calT, \calN)$, the result is the sheaf $\mathscr{G}(\mathbb{M}, \mathbb{N}) = (\calS, \calT, \mathcal{H}om_{\calR}(\calM, \calN))$. Because $G$ is a right adjoint in the base categories, it is a continuous functor (i.e., it preserves limits); thus, $\mathscr{G}$ can be defined section-wise as in Case (2) of the framework. This section-wise construction $U \mapsto \operatorname{Hom}_{\calR(U)}(\calM(U), \calN(U))$ is naturally isomorphic to the restriction-based internal Hom defined above, as both satisfy the same local-to-global universal property.

    }

    % \TextIfExists{definition:sheafy_functor_of_a_functor_between_two_categories_for_the_categories_of_sheaves_on_a_site}{
    % In the framework of \Cref{definition:sheafy_functor_of_a_functor_between_two_categories_for_the_categories_of_sheaves_on_a_site}, the sheafified Hom is the evaluation of the sheafy functor $\mathscr{G}$ induced by the algebraic Hom functor $G: \calA^{\op} \times \calC \to \calB$ under assumption (2). Given input sheaves $\mathbb{M} := (\calR, \calS, \calM)$ and $\mathbb{P} := (\calR, \calT, \calP)$, the result is the sheaf $\mathscr{G}(\mathbb{M}, \mathbb{P}) = (\calS, \calT, \mathcal{H}om_{\calR}(\calM, \calP))$. Because $G$ is a right adjoint in the base categories, it is a continuous functor; thus, $\mathscr{G}$ is defined section-wise as in Case (2) of the framework, which is naturally isomorphic to the restriction-based internal Hom defined above.
    % }
\end{definition}