\begin{definition} \label{definition:span_a_module_over_a_ring_for_elements_of_the_module}

        Let $R$ and $S$ be \CrefAndHyperrefIfExist{definition:ring}{(not necessarily commutative) rings}. 
    \begin{enumerate}
        \item If $M$ is an \CrefAndHyperrefIfExist{definition:module_of_a_ring}{$R$-$S$-bidmodule}, then a subset $\{m_i\}_{i \in I} \subseteq M$ is said to \hldef{span $M$ (as an $R$-$S$-bimodule)} if every element $m \in M$ is a \CrefAndHyperrefIfExist{definition:linear_combination_of_elements_in_a_module}{linear combination} of $\{m_i\}_{i \in I}$.
        
        \item If $M$ is a left/right/two-sided $R$-module, then a subset $\{m_i \}_{i \in I} \subseteq M$ is said to \hldef{span $M$ (as a left/right/two-sided $R$-module)} if $\{m_i\}_{i \in }$ spans its \CrefAndHyperrefIfExist{definition:module_of_a_ring}{natural bimodule structure}.
    \end{enumerate}
    In each case, such a set $\{m_i\}_{i \in I}$ is called a \hldef{generating set} or \hldef{spanning set of $M$ over $R$}.

    % Let $R$ be a \CrefAndHyperrefIfExist{definition:ring}{(not necessarily commutative) ring} and let $M$ be an \CrefAndHyperrefIfExist{definition:module_of_a_ring}{$R$-module}.
    % \begin{enumerate}
    %     \item If $M$ is a \CrefAndHyperrefIfExist{definition:module_of_a_ring}{left $R$-module}, a subset $S = \{x_i\}_{i \in I} \subseteq M$ is said to \hldef{span $M$ over $R$} if every element $m \in M$ can be expressed as a finite linear combination of the form
    %     $$m = r_1 x_{i_1} + r_2 x_{i_2} + \cdots + r_k x_{i_k},$$
    %     where $x_{i_1}, \dots, x_{i_k} \in S$ and $r_1, \dots, r_k \in R$.
        
    %     \item If $M$ is a \CrefAndHyperrefIfExist{definition:module_of_a_ring}{right $R$-module}, a subset $S = \{x_i\}_{i \in I} \subseteq M$ is said to \hldef{span $M$ over $R$} if every $m \in M$ can be expressed as a finite linear combination
    %     $$m = x_{i_1} r_1 + x_{i_2} r_2 + \cdots + x_{i_k} r_k,$$
    %     for some $x_{i_1}, \dots, x_{i_k} \in S$ and $r_1, \dots, r_k \in R$.

    %     \item If $M$ is a two-sided $R$-module (that is, $M$ carries compatible left and right $R$-actions), a subset $S = \{x_i\}_{i \in I} \subseteq M$ is said to \hldef{span $M$ over $R$} if every $m \in M$ can be expressed as a finite linear combination of the form
    %     $$m = r_1 x_{i_1} s_1 + r_2 x_{i_2} s_2 + \cdots + r_k x_{i_k} s_k,$$
    %     where $x_{i_1}, \dots, x_{i_k} \in S$ and $r_1, \dots, r_k, s_1, \dots, s_k \in R$.
    % \end{enumerate}
    In each case, such a set $S$ is called a \hldef{generating set} or \hldef{spanning set of $M$ over $R$}.
\end{definition}