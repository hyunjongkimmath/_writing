\begin{definition} \label{definition:bounded_double_complex_of_objects_in_an_additive_category}
Let $(A^{p,q},d'_A,d''_A)$ be a double complex in an additive category $\mathcal{A}$.
\begin{itemize}
  \item The double complex is called \hldef{bounded above} if there exist integers $p_0,q_0$ such that $A^{p,q}=0$ whenever $p>p_0$ or $q>q_0$.
  \item The double complex is called \hldef{bounded below} if there exist integers $p_0,q_0$ such that $A^{p,q}=0$ whenever $p<p_0$ or $q<q_0$.

  \item The double complex is called \hldef{bounded} if it is both bounded above and below. 
  \item The double complex is said to be in the \hldef{first quadrant} (also called \emph{first-quadrant double complex}) if $A^{p,q}=0$ whenever $p<0$ or $q<0$. In particular, any first quadrant double complex is bounded below.
  \item The double complex is said to be in the \hldef{third quadrant} (also called \emph{third-quadrant double complex}) if $A^{p,q}=0$ whenever $p>0$ or $q>0$. In particular, any third quadrant double complex is bounded above.

  \item Let us say that the double complex is \hldef{locally finite along diagonals} or \hldef{locally bounded along diagonals}\footnote{These do not seem to be standard teminology.} if for each integer $n$, there exist at most finitely many pairs $(p,q)$ with $p+q = n$ such thta $A^{p,q} \neq 0$. 
  
  \item Let us say that the double complex is \hldef{bounded in total degree}\footnote{This does not seem to be standard teminology.} if there exist integers $m$ and $M$ such that $A^{p,q} = 0$ whenever $m \leq p+q \leq M$. 
\end{itemize}
\end{definition}