\begin{definition} \label{definition:basis_for_a_topology}
Let $X$ be a set and let $\mathcal{B}$ be a collection of subsets of $X$. The collection $\mathcal{B}$ is called a \hldef{basis} (or \hldef{base}) for a \CrefAndHyperrefIfExist{definition:topological_space}{topology} on $X$ if the following two conditions hold:
\begin{enumerate}
  \item For every $x \in X$, there exists at least one $B \in \mathcal{B}$ such that $x \in B$.
  \item For any $B_1, B_2 \in \mathcal{B}$ and any $x \in B_1 \cap B_2$, there exists $B_3 \in \mathcal{B}$ such that $x \in B_3 \subseteq B_1 \cap B_2$.
\end{enumerate}
Given such a collection $\mathcal{B}$, the collection $\mathcal{T}$ of all unions of elements of $\mathcal{B}$ defines a topology on $X$, and it coincides with $\calT_\calB$, the \CrefAndHyperref{definition:topology_on_a_set_generated_by_a_collection_of_subsets}{topology generated by $\calB$}. In other words, 
$$\mathcal{T}_\mathcal{B} = \{ U \subseteq X : \text{for every } x \in U, \text{ there exists } B \in \mathcal{B} \text{ with } x \in B \subseteq U \}.$$ 
\end{definition}