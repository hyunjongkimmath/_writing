
\begin{definition}[Special cases of limits] \label{definition:projective_and_inductive_limits_in_categories}
Let $\mathcal{C}$ be a (large) category. Let $I$ be a (large) category. Let $I \to \mathcal{C}$ be a diagram/system. 
\begin{itemize}
    \item Suppose that the system is a \hyperrefIfExists{definition:system_in_a_category_indexed_by_a_directed_poset}{cofiltered system}\CrefIfExists{definition:system_in_a_category_indexed_by_a_directed_poset}, i.e. $I$ is a cofiltered category. A \hyperrefIfExists{definition:limit_and_colimit_of_a_diagram_in_a_category}{limit}\CrefIfExists{definition:limit_and_colimit_of_a_diagram_in_a_category} of this diagram is often denoted by 
    $$\hlin{ \varprojlim_{i\in I} D(i) }$$
    and may be called a \hldef{cofiltered (inverse/projective) limit}. In case that the system is more specifically an \hyperrefIfExists{definition:system_in_a_category_indexed_by_a_directed_poset}{inverse/projective system}\CrefIfExists{definition:system_in_a_category_indexed_by_a_directed_poset}, i.e. $I$ is a cofiltered poset, the preferred term for such a limit is \emph{inverse/projective limit}.

    \item Suppose that the system is a filtered system, i.e. $I$ is a filtered category. A colimit of this diagram is often denoted by 
    $$\hlin{ \varinjlim_{i\in I} D(i) }$$
    and may be called a \hldef{filtered colimit} or a \hldef{direct/inductive/injective limit}. In case that the system is more specifically a direct/inductive system, i.e. $I$ is a filtered poset, the preferred term for such a limit is \emph{direct/inductive limit}.

\end{itemize}
\end{definition}
