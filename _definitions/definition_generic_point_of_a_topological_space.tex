\begin{definition} \label{definition:generic_point_of_a_topological_space}
Let $X$ be a \CrefAndHyperrefIfExist{definition:topological_space}{topological space}. A point $\eta \in X$ is called a \hldef{generic point of $X$} if the \CrefAndHyperrefIfExist{definition:closure_of_a_subspace_of_a_topological_space}{closure} of the singleton set $\{\eta\}$ is the entire space $X$, i.e., $\overline{\{\eta\}} = X$. More generally, if $Z$ is an irreducible closed subset of $X$, a point $\eta \in Z$ is called a \hldef{generic point of $Z$} if $\overline{\{\eta\}} = Z$. In the context of schemes, every irreducible closed subset has a unique generic point.
\end{definition}