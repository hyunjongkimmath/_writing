% \begin{definition}[Chain complex in an additive category] \label{definition:chain_complex_of_objects_in_an_additive_category}
% Let $\mathcal{A}$ be an \hyperrefIfExists{definition:additive_category_preadditive_category}{preadditive category} and let $I$ be a totally ordered set (typically $\mathbb{Z}$, but $I \subseteq \mathbb{Z}$ is also allowed). 
% \begin{enumerate}
%     \item A \hldef{chain complex} $(K^\bullet, d^\bullet)$ in $\mathcal{A}$ indexed by $I$ consists of:
%     \begin{itemize}
%         \item Objects $\{ K^i \}_{i \in I}$ in $\mathcal{A}$, called the \hldef{terms in degree $i$},
%         \item Morphisms $d^i: K^i \to K^{i+1}$ in $\mathcal{A}$, called the \hldef{differentials in degree $i$},
%     \end{itemize}
%     such that for every $i \in I$, $d^{i+1} \circ d^i = 0$. That is,
%     $$ K^\bullet: \cdots \xrightarrow{d^{i-2}} K^{i-1} \xrightarrow{d^{i-1}} K^i \xrightarrow{d^i} K^{i+1} \xrightarrow{d^{i+1}} \cdots $$
%     with $d^{i+1}d^i = 0$ for all $i$. We might typically use notation such as \hl{$K^\bullet = (K^i, d^i)_{i \in I}$} to denote a chain complex in $\mathcal{A}$.

%     A cochain complex can be defined similarly/dually.

%     \item Let $K^\bullet = (K^i, d_K^i)$ and $L^\bullet = (L^i, d_L^i)$ be \CrefAndHyperrefIfExist{definition:chain_complex_of_objects_in_an_additive_category}{chain complexes} in $\mathcal{A}$ indexed by the same set $I$. 
%     A \hldef{morphism of chain complexes} (or \hldef{chain map})
%     $$ f^\bullet: K^\bullet \to L^\bullet $$
%     consists of morphisms $f^i: K^i \to L^i$ for all $i \in I$, such that for every $i \in I$,
%     $$ d_L^i \circ f^i = f^{i+1} \circ d_K^i, $$
%     i.e., the following diagram commutes for all $i$:

%     $$ \begin{array}{ccc} K^i & \xrightarrow{d_K^i} & K^{i+1} \\ \downarrow{f^i} && \downarrow{f^{i+1}} \\ L^i & \xrightarrow{d_L^i} & L^{i+1} \end{array}.$$

% \end{enumerate}

% There is then a category, often denoted by \hl{$\mathrm{Ch}(\mathcal{A})$} or \hl{$\mathbf{Ch}(\mathcal{A})$}, whose objects are chain complexes in $\calA$ and whose morphisms are morphisms of chain complexes. In particular, we may denote by 
% $$\hlin{\operatorname{Hom}(K^\bullet, L^\bullet)=  \operatorname{Hom}_{\mathrm{Ch}(\mathcal{A})}(K^\bullet, L^\bullet)}$$
% the set of chain maps $K^\bullet \to L^\bullet$; it is in fact an abelian group.

% A \hldef{morphism of cochain complexes} is defined similarly, and we similarly denote by \hl{$\mathrm{Ch}(\mathcal{A})$} or \hl{$\mathbf{Ch}(\mathcal{A})$} the caetgory of cochain complexes in $\calA$. 


% \TextIfExists{definition:dg_category_over_a_ring}{
% If $k$ is a \CrefAndHyperrefIfExist{definition:commutative_ring}{commutative ring} such that $\Hom_\calA(X,Y)$ is \CrefAndHyperrefIfExist{definition:category_enriched_in_a_monoidal_category}{enriched in} the category of \CrefAndHyperrefIfExist{definition:module_of_a_ring}{$k$-modules}, then $\mathrm{Ch}(\calA)$ \CrefAndHyperref{definition:category_of_chain_complexes_of_objects_in_an_additive_category_as_a_dg_category}{can be equipped with} the structure of a \CrefAndHyperrefIfExist{definition:dg_category_over_a_ring}{dg-category over $k$}.
% }


% \end{definition}

\begin{definition}[Chain complex in a preadditive category] \label{definition:chain_complex_of_objects_in_an_additive_category}
Let $\mathcal{A}$ be a \hyperrefIfExists{definition:additive_category_preadditive_category}{preadditive category} and let $I$ be a totally ordered set (typically $\mathbb{Z}$, but $I \subseteq \mathbb{Z}$ is also allowed). 
\begin{enumerate}
    \item A \hldef{chain complex} $(K_\bullet, d_\bullet)$ in $\mathcal{A}$ indexed by $I$ is the homological convention for sequences with decreasing degrees. It consists of:
    \begin{itemize}
        \item Objects $\{ K_i \}_{i \in I}$ in $\mathcal{A}$, called the \hldef{terms in degree $i$},
        \item Morphisms $d_i: K_i \to K_{i-1}$ in $\mathcal{A}$, called the \hldef{boundary maps} or \hldef{differentials in degree $i$},
    \end{itemize}
    such that for every $i \in I$, $d_{i-1} \circ d_i = 0$. That is,
    $$ K_\bullet: \cdots \xrightarrow{d_{i+1}} K_i \xrightarrow{d_i} K_{i-1} \xrightarrow{d_{i-1}} K_{i-2} \xrightarrow{} \cdots $$
    with $d_{i-1}d_i = 0$ for all $i$. We typically use the notation \hl{$K_\bullet = (K_i, d_i)_{i \in I}$}.



    \item Dually, a \hldef{cochain complex} $(K^\bullet, d^\bullet)$ in $\mathcal{A}$ follows the \hldef{cohomological convention} with increasing degrees. It consists of objects $\{ K^i \}_{i \in I}$ and \hldef{coboundary maps} $d^i: K^i \to K^{i+1}$ such that $d^{i+1} \circ d^i = 0$:
    $$ K^\bullet: \cdots \xrightarrow{d^{i-1}} K^i \xrightarrow{d^i} K^{i+1} \xrightarrow{d^{i+1}} K^{i+2} \xrightarrow{} \cdots $$
    We typically use the notation \hl{$K^\bullet = (K^i, d^i)_{i \in I}$}.

    \item Let $K_\bullet = (K_i, d_i^K)$ and $L_\bullet = (L_i, d_i^L)$ be \CrefAndHyperrefIfExist{definition:chain_complex_of_objects_in_an_additive_category}{chain complexes} in $\mathcal{A}$ indexed by the same set $I$. A \hldef{morphism of chain complexes} (or \hldef{chain map})
    $$ f_\bullet: K_\bullet \to L_\bullet $$
    consists of morphisms $f_i: K_i \to L_i$ for all $i \in I$, such that for every $i \in I$, the following diagram commutes:
    $$ \begin{array}{ccc} K_i & \xrightarrow{d_i^K} & K_{i-1} \\ \downarrow{f_i} && \downarrow{f_{i-1}} \\ L_i & \xrightarrow{d_i^L} & L_{i-1} \end{array} $$
    i.e., $d_i^L \circ f_i = f_{i-1} \circ d_i^K$. 



    A \hldef{morphism of cochain complexes} $f^\bullet: K^\bullet \to L^\bullet$ is defined similarly, satisfying the commutativity condition $d_L^i \circ f^i = f^{i+1} \circ d_K^i$.
\end{enumerate}

The collection of these objects and morphisms forms a category. Notation for these categories is as follows:
\begin{itemize}
    \item \hl{$\mathrm{Ch}(\mathcal{A})$} or \hl{$\mathbf{Ch}(\mathcal{A})$} is often used as a general term.
    \item To be explicit about the indexing convention, one uses \hl{$\mathrm{Ch}_\bullet(\mathcal{A})$} for chain complexes and \hl{$\mathrm{Ch}^\bullet(\mathcal{A})$} (or sometimes $\mathrm{CoCh}(\mathcal{A})$) for cochain complexes.
    \item The set of chain maps between two complexes is denoted by $\hlin{\operatorname{Hom}_{\mathrm{Ch}(\mathcal{A})}(K_\bullet, L_\bullet)}$; it is an abelian group under pointwise addition $(f+g)_i = f_i + g_i$.
\end{itemize}

\TextIfExists{definition:dg_category_over_a_ring}{
If $k$ is a \CrefAndHyperrefIfExist{definition:commutative_ring}{commutative ring} such that $\Hom_\calA(X,Y)$ is \CrefAndHyperrefIfExist{definition:category_enriched_in_a_monoidal_category}{enriched in} the category of \CrefAndHyperrefIfExist{definition:module_of_a_ring}{$k$-modules}, then $\mathrm{Ch}(\calA)$ \CrefAndHyperref{definition:category_of_chain_complexes_of_objects_in_an_additive_category_as_a_dg_category}{can be equipped with} the structure of a \CrefAndHyperrefIfExist{definition:dg_category_over_a_ring}{dg-category over $k$}.
}
\end{definition}

% 
\begin{remark} \label{remark:cohomological_vs_homological_conventions}
    The convention used to define chain complexes in \Cref{definition:chain_complex_of_objects_in_an_additive_category} is a \emph{cohomological one} --- note that indices are written as superscripts and increase when ``following the arrows''. Such a chain complex may also be referred to as a \hldef{cochain complex} or a \hldef{cohomological chain complex} to emphasize an adoption of a cohomological convention. 

    The dual convention would be a \emph{homological one}, in which indices are written as subscripts and decrease when ``following the arrow''. As such, one may speak of a \hldef{(homological) chain complex} $(K_\bullet, d_\bullet)$ indexed by $I$ as consisting of:

    \begin{itemize}
    \item Objects $\{ K_i \}_{i \in I}$ in $\mathcal{A}$, called the \hldef{terms in degree $i$},
    \item Morphisms $d_i: K_i \to K_{i-1}$ in $\mathcal{A}$, called the \hldef{differentials in degree $i$},
    \end{itemize}
    such that for every $i \in I$, $d_{i-1} \circ d_i = 0$. That is,
    $$ 
    K_\bullet: \cdots \xrightarrow{d_{i+1}} K_i \xrightarrow{d_i} K_{i-1} \xrightarrow{d_{i-1}} K_{i-2} \xrightarrow{d_{i-2}} \cdots
    $$
    with $d_{i-1} d_i = 0$ for all $i$. We might typically use notation such as \hl{$K_\bullet = (K_i, d_i)_{i \in I}$} to denote a chain complex in $\mathcal{A}$.

    The differences between the conventions persist --- for example, cohomological objects are usually written with superscript indicees whereas homological objects are usually written with subscript indicees.
\end{remark}
%\begin{convention} \label{convention:homological_algebra_is_discussed_in_cohomological_terms}
    When discussing homological algebra in abstract terms, we may often adopt the homological convention in some discussions and the cohomological convention in others \CrefIfExists{remark:cohomological_vs_homological_conventions}.
    %; for instance, indices are written as superscripts and increase along the direction of the arrows in chain complexes. 
\end{convention}

% \begin{definition}[Morphisms of chain complexes] \label{definition:chain_complex_of_objects_in_an_additive_category}
Let $\mathcal{A}$ be an \CrefAndHyperrefIfExist{definition:additive_category}{additive category}, and let $K^\bullet = (K^i, d_K^i)$ and $L^\bullet = (L^i, d_L^i)$ be \CrefAndHyperrefIfExist{definition:chain_complex_of_objects_in_an_additive_category}{chain complexes} in $\mathcal{A}$ indexed by the same set $I$. 
A \hldef{morphism of chain complexes} (or \hldef{chain map})
$$ f^\bullet: K^\bullet \to L^\bullet $$
consists of morphisms $f^i: K^i \to L^i$ for all $i \in I$, such that for every $i \in I$,
$$ d_L^i \circ f^i = f^{i+1} \circ d_K^i, $$
i.e., the following diagram commutes for all $i$:

$$ \begin{array}{ccc} K^i & \xrightarrow{d_K^i} & K^{i+1} \\ \downarrow{f^i} && \downarrow{f^{i+1}} \\ L^i & \xrightarrow{d_L^i} & L^{i+1} \end{array}.$$

There is then a category, often denoted by \hl{$\mathrm{Ch}(\mathcal{A})$} or \hl{$\mathbf{Ch}(\mathcal{A})$}, whose objects are chain complexes in $\calA$ and whose morphisms are morphisms of chain complexes. In particular, we may denote by 
$$\hlin{\operatorname{Hom}(K^\bullet, L^\bullet)=  \operatorname{Hom}_{\mathrm{Ch}(\mathcal{A})}(K^\bullet, L^\bullet)}$$
the set of chain maps $K^\bullet \to L^\bullet$; it is in fact an abelian group.

A \hldef{morphism of cochain complexes} is defined similarly, and we similarly denote by \hl{$\mathrm{Ch}(\mathcal{A})$} or \hl{$\mathbf{Ch}(\mathcal{A})$} the caetgory of cochain complexes in $\calA$. 
\end{definition}

% See Also
% 
\begin{proposition} \label{proposition:category_of_chain_complexes_in_an_additive_category_is_additive}
Let $\mathcal{A}$ be an \hyperrefIfExists{definition:additive_category}{additive category}. 
\begin{enumerate}
    \item The category \hyperrefIfExists{definition:chain_complex_of_objects_in_an_additive_category}{$\mathrm{Ch}(\calA)$} of chain complexes is itself and additive category.

    \item If $\calA$ is an \hyperrefIfExists{definition:abelian_category}{abelian category}, then $\mathrm{Ch}(\calA)$ is an abelian category.

    \item If $\calA$ is an \hyperrefIfExists{definition:abelian_category}{abelian category} satisfying Grothendieck's axiom \CrefAndHyperrefIfExist{definition:grothendiecks_additional_axioms_for_abelian_categories}{AB$n$ (resp. AB$n^*$)} for $n \in \{3,4,5,6\}$, then $\mathrm{Ch}(\calA)$ also satisfies AB$n$ (resp. AB$n^*$). If $\calA$ is a \CrefAndHyperrefIfExist{definition:grothendiecks_additional_axioms_for_abelian_categories}{Grothendieck abelian category}, then so is $\mathrm{Ch}(\calA)$
\end{enumerate}
\end{proposition}
\begin{proof}
    Combine \Cref{proposition:category_of_chain_complexes_of_objects_in_a_preadditive_category_is_equivalent_to_the_category_of_additive_functors_from_the_walking_chain_complex_category} and \Cref{lemma:additive_functor_category_from_small_preadditive_categories_preserves}.
\end{proof}

% \begin{corollary}
% Let $\calB$ be a \CrefAndHyperrefIfExist{definition:additive_category}{preadditive category}.
% \begin{enumerate}
%     \item The \CrefAndHyperrefIfExist{definition:chain_complex_of_objects_in_an_additive_category}{(co)chain complex category} $\text{Ch}(\calB)$ is preadditive. If $\calB$ is additionally \CrefAndHyperrefIfExist{definition:additive_category}{additive}/\CrefAndHyperrefIfExist{definition:abelian_category}{abelian}, then so is $\text{Ch}(\calB)$.

%     \item If $\calB$ is an abelian category with property \CrefAndHyperrefIfExist{definition:grothendiecks_additional_axioms_for_abelian_categories}{$ABn$ for $n = 3,4,5,6$ or $ABn^*$ for $n = 3,4,5$}, then $\text{Add}(\calA, \calB)$ possesses the same property.
% \end{enumerate}

% \end{corollary}