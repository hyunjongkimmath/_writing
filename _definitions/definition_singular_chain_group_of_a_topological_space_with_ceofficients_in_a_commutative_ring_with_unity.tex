\begin{definition}[Singular chain group with coefficients] \label{definition:singular_chain_group_of_a_topological_space_with_ceofficients_in_a_commutative_ring_with_unity}
Let $X$ be a \CrefAndHyperrefIfExist{definition:topological_space}{topological space}, let $S_n(X)$ be the set of \CrefAndHyperrefIfExist{definition:singular_simplex_of_a_topological_space}{singular $n$-simplices in $X$}, and let $R$ be a \CrefAndHyperrefIfExist{definition:commutative_ring}{commutative ring} with unity.  

\begin{enumerate}
    \item The \hldef{singular $n$-chain group of $X$ with coefficients in $R$} is the free $R$-module \hl{$C_n(X;R)$} whose elements are finite formal linear combinations
    $$\sum_i r_i\, \sigma_i, \quad \text{with } \sigma_i \in S_n(X), \ r_i \in R.$$
    Elements of $C_n(X;R)$ are called \hldef{singular $n$-chains in $X$ with coefficients in $R$}.

    \item If $A \subseteq X$ is a subspace, the quotient groups
    $$\hlin{C_n(X,A;R) = C_n(X;R)/C_n(A;R)}$$
    may be referred to as the \hldef{relative singular $n$-chain groups of $X$ in $A$ with coefficients in $R$} and elements of this group may be referred to as \hldef{relative singular $n$-chains in $X$ relative to $A$ with coefficients in $R$}.
\end{enumerate}
In either case, when $R = \bbZ$, the ring $R$ may be suppressed from notation, so we may write \hl{$C_n(X)$} and \hl{$C_n(X,A)$} for $C_n(X,\bbZ)$ and $C_n(X,A,\bbZ)$ respectively.
\end{definition}