The purpose of these notes is to concisely define stable homotopy categories and to state their relationship to stable homotopy groups.

\section{Definitions}

All spaces are assumed to be compactly generated. 

\hl{$X, Y$}: (Compactly generated) spaces. Unless otherwise specified, they will be based spaces. By default, \hl{$*$} denotes the base point of a based space.


\begin{definition} \label{definition:fibration_of_topological_spaces}
Let $X$ and $B$ be \CrefAndHyperrefIfExist{definition:topological_space}{topological spaces} and let $p:X \to B$ be a \CrefAndHyperrefIfExist{definition:continuous_map_of_topological_spaces}{continuous map}. 

The map $p$ is called a \hldef{fibration} or a \hldef{Serre fibration} if it satisfies the \hldef{Homotopy Lifting Property (HLP)}: for every space $Y$, every map $f:Y \to X$, every homotopy $H:Y \times I \to B$ with $H(y,0) = p(f(y))$ for all $y \in Y$, and whenever a lift $\tilde f_0:Y \to X$ of $H(\cdot,0)$ is given, there exists a homotopy $\tilde H:Y \times I \to X$ such that
$$
\tilde H(y,0) = \tilde f_0(y) \quad \text{for all } y \in Y,\\
p \circ \tilde H = H \quad \text{on } Y \times I.
$$
Equivalently, for all $Y$, the map
\[
\operatorname{Map}(Y,X) \to \operatorname{Map}(Y,B),\quad f \mapsto p \circ f
\]
has the right lifting property with respect to the inclusion $\operatorname{Map}(Y,\ast) \hookrightarrow \operatorname{Map}(Y \times I, B)$ encoded by the HLP.
\end{definition}


\begin{definition} \label{definition:cofibration_of_topological_spaces}
Let $X$ be a topological space and let $A \subseteq X$ be a subspace. The inclusion map $i:A \hookrightarrow X$ is called a \hldef{cofibration} if it satisfies the \hldef{Homotopy Extension Property (HEP)}: for every space $Y$, every map $f:A \to Y$ and every homotopy $H:X \times I \to Y$ with $H(a, t) = f(a)$ for all $a \in A$ and all $t \in I$, there exists a homotopy $\tilde H:X \times I \to Y$ extending $H$ such that $\tilde H|_{A \times I} = H|_{A \times I}$ and $\tilde H(x,0) = f(x)$ for all $x \in A$.
Equivalently, the inclusion $i:A \hookrightarrow X$ has the left lifting property with respect to every map that is a fibration.
\end{definition}




\begin{definition} \label{definition:nondegenerately_based_well_pointed_topological_space}
    A \CrefAndHyperrefIfExist{definition:pointed_topological_space}{pointed topological space} $X$ is said to be \hldef{nondegenerately based} or \hldef{well pointed} if the inclusion $* \hookrightarrow X$ is a \CrefAndHyperrefIfExist{definition:cofibration_of_topological_spaces}{cofibration} in the unbased sense, i.e. the inclusion satisfies the homotopy extension property.
\end{definition}


% \hl{$[X,Y]$}: the set of based homotopy classes of based maps $X \to Y$. It has a natural basepoint, namely the homotopy class of the constant map from $X$ to the basepoint of $Y$. 






\begin{definition} \label{definition:compact_open_topology_on_the_set_of_continuous_maps_between_topological_spaces_and_function_space}
    Let $X$ and $Y$ be \CrefAndHyperrefIfExist{definition:topological_space}{topological spaces}. 
    \begin{enumerate}
        \item Write $\mathrm{Open}(Y)$ for the \CrefAndHyperrefIfExist{definition:topological_space}{topology} of $Y$ and $\mathrm{Comp}(X)$ for the family of \CrefAndHyperrefIfExist{compact}{compact subsets} of $X$. 
        For $K \in \mathrm{Comp}(X)$ and $U \in \mathrm{Open}(Y)$, define the subset
        $$\hlin{\langle K,U\rangle \coloneqq \{\, f \in C(X,Y) \mid f(K) \subseteq U \,\} \subseteq C(X,Y)}.$$
        where \CrefAndHyperrefIfExist{definition:continuous_map_of_topological_spaces}{$C(X,Y)$} is the set of all \CrefAndHyperrefIfExist{definition:continuous_map_of_topological_spaces}{continuous maps} $X \to Y$. 
        The collection of all such subsets is denoted by \hl{$\mathcal{S}_{\mathrm{co}}(X,Y)$}.

        \item The \hldef{compact-open topology on $C(X,Y)$} \CrefIfExists{definition:continuous_map_of_topological_spaces} is the \CrefAndHyperrefIfExist{definition:topological_space}{topology} \CrefAndHyperrefIfExist{definition:topology_on_a_set_generated_by_a_collection_of_subsets}{generated by} the \CrefAndHyperrefIfExist{definition:subasis_on_a_set_and_topology_generated_by_a_subbasis}{subbasis} $\mathcal{S}_{\mathrm{co}}(X,Y)$. The resulting topological space is called the \hldef{function space from $X$ to $Y$} and is commonly denoted by notations such as \hl{$\operatorname{Map}(X,Y)$}, \hl{$Y^X$}, \hl{$\operatorname{Fun}(X,Y)$}, or \hl{$\operatorname{F}(X,Y)$}. As a set, it equals $C(X,Y)$.

        \item Let $(X,x_0)$ and $(Y,y_0)$ be \CrefAndHyperrefIfExist{definition:pointed_topological_space}{based topological spaces}. The \hldef{based function space from $(X,x_0)$ to $(Y,y_0)$} is the subspace of $\operatorname{Map}(X,Y)$ consisting of all based maps $f$ with $f(x_0)=y_0$, equivalently $C_*((X,x_0),(Y,y_0))$ endowed with the subspace topology inherited from the compact-open function space $\operatorname{Map}(X,Y)$. It itself is a based topological space whose base point is given by the constant map $X \to Y$ with value $y_0$. Common notations for the based function space include those which include a star or bullet such as \hl{$\operatorname{Map}_*((X,x_0),(Y,y_0))$}, \hl{$\operatorname{Map}_*(X,Y)$}, or \hl{$Y^X_*$} to emphasize that the spaces involved are pointed, or those which omit a star or bullet, such as \hl{$\operatorname{Map}(X,Y)$}, and \hl{$Y^X$}. 
    \end{enumerate}
%  Common notations for this function space are \hl{$\operatorname{Map}(X,Y)$} and \hl{$Y^X$}.
\end{definition}



\begin{definition}
Let $X$ be a set and $\mathcal{A} \subseteq \mathsf{Top}(X)$. The \hldef{greatest lower bound} (coarsest topology below every member of $\mathcal{A}$) is $\bigcap \mathcal{A}$, which lies in $\mathsf{Top}(X)$. The \hldef{least upper bound} (finest topology above every member of $\mathcal{A}$) is the topology generated by $\bigcup \mathcal{A}$, namely $\tau(\bigcup \mathcal{A})$.
\end{definition}

\begin{definition}
Let $X$ be a set. The \hldef{indiscrete topology} on $X$ is $\{\varnothing,X\}$ and is the coarsest topology on $X$. The \hldef{discrete topology} on $X$ is $\mathcal{P}(X)$ and is the finest topology on $X$. When the underlying set is clear, one may write \hl{$\tau_{\mathrm{ind}}$} and \hl{$\tau_{\mathrm{disc}}$} for these topologies.
\end{definition}




% \hl{$Y^X = \Map(X,Y)$}: the function space of spaces $X$ and $Y$ , i.e. the set of continuous maps $X \to Y$ with the $k$-ification of the standard compact-open topolo
% y.
% maps $X \to Y$ with the $k$-ification of the standard compact-open topology.

\begin{proposition}{\cite[Proposition Page 41]{may}} \label{proposition:function_space_of_maps_from_product_space_of_compactly_generated_spaces_is_naturally_isomorphic_to_function_space_of_function_space}
    For \CrefAndHyperrefIfExist{definition:compactly_generated_topological_space}{compactly generated} \CrefAndHyperrefIfExist{definition:topological_space}{topological spaces} $X,Y,Z$, the canonical bijection
    $$Z^{(X \times Y)} \cong (Z^Y)^X$$
    of \CrefAndHyperrefIfExist{definition:compact_open_topology_on_the_set_of_continuous_maps_between_topological_spaces_and_function_space}{function spaces} is a \CrefAndHyperrefIfExist{definition:homeomorphism_of_topological_spaces}{homeomorphism}.
\end{proposition}

\TODO{continue going through here}
\hl{$F(X,Y)$}: The subspace of $Y^X$ consisting of the based maps, with the constant based map as basepoint. We have a natural homeomorphism
$$F(X \wedge Y, Z) \cong F(X, F(Y,Z))$$
of based spaces for any based spaces $X,Y$, and $Z$. 

\begin{definition}[Homotopy groups] \label{definition:homotopy_groups_of_a_pointed_topological_space}
For any \CrefAndHyperrefIfExist{definition:pointed_topological_space}{pointed topological space} $(X,x_0)$ and integer $n \ge 0$, the \hldef{$n$-th homotopy group of $X$ at $x_0$}, denoted \hl{$\pi_n(X,x_0)$}, is defined as the set of all \CrefAndHyperrefIfExist{definition:homotopy_class_of_maps_of_topological_spaces_relative_to_a_subset}{homotopy classes (rel.\ $\partial I^n$)} of based maps
$$f : (I^n, \partial I^n) \to (X,x_0),$$
where $I^n = [0,1]^n$. For $n \ge 1$, $\pi_n(X,x_0)$ is a group under concatenation of based maps, and for $n \ge 2$, it is abelian.

The \hldef{fundamental group of $(X,x_0)$} refers to $\pi_1(X,x_0)$. Equivalently, it is the group of homotopy classes (rel.\ endpoints) of \CrefAndHyperrefIfExist{definition:path_and_loop_in_a_topological_space}{loops} $\gamma : [0,1] \to X$ satisfying $\gamma(0)=\gamma(1)=x_0$. 
\end{definition}

\hl{$\pi_n(X) = \pi_n(X,*)$}: The \hldef{$n$th homotopy group of $X$}, defined by 
$$\pi_n(X) = \pi_n(X,*) = [\Sn, X].$$

\begin{claim}
    $[X,Y]$ may be identified with $\pi_0(F(X,Y))$.
\end{claim}

\begin{definition}
    A space $X$ is said to be \hldef{$n$-connected} if $\pi_q(X,x) = 0$ for $0 \leq q \leq n$ for all $x$. 
\end{definition}

\hl{$\Sigma X$}: The \hldef{(reduced) suspension of $X$}, defined as 
$$\Sigma = X \wedge \Sone = X \times \Sone /\left(\{*\} \times \S1 \cup X \times\{1\}\right).$$

\hl{$\Sigma^n X$}: The \hldef{$n$-fold suspension of $X$}.

\hl{$\Omega X$}: The \hldef{loop space of $X$}, defined as $F(\Sone, X)$; the points are the loops in $X$ at the basepoint. 

\hl{$\Omega^n X$}: The \hldef{$n$-fold loop space of $X$}. 

\begin{claim}
    There is an adjunction
    $$F(\Sigma X, Y) \cong F(X, \Omega Y).$$
    Applying $\pi_0$, we have an adjunction
    $$[\Sigma X, Y] \cong [X, \Omega Y].$$
\end{claim}

\subsection{Stable homotopy groups and the stable homotopy theory}

\hfill\\

\hl{$\Sigma: \pi_q(X) \to \pi_{q+1}(\Sigma X)$}: The suspension homomorphism of $X$, defined by setting 
$$\Sigma f = f \wedge \id : \sphere{q+1} \cong \sphere{q} \wedge \Sone \to X \wedge \Sone.$$


\begin{theorem}[Freudenthal suspension, see e.g. {\cite[Chapter 11]{may}}] \label{theorem:freudenthal_suspension}
If $X$ is nondegenerately based and $(n-1)$-connected where $n \geq 1$, then $\Sigma: \pi_q(X) \to \pi_{q+1}(\Sigma X)$ is a bijection if $q < 2n-1$ and a surjection if $q = 2n-1$. 
\end{theorem}

\begin{definition}
\begin{enumerate}
    \item A \hldef{prespectrum} $T$ is a sequence of based space \hldef{$T_n$} and based maps \hldef{$\sigma = \sigma_n^T: \Sigma T_n \to T_{n+1}$}. The space $T_n$ may be referred to as the \hldef{degree $n$ space of the prespectrum} and the maps $\sigma: \Sigma T_n \to T_{n+1}$ may be referred to as the \hldef{structure maps}. A \hldef{morphism $f: T \to T'$ of prespectra} is a sequence of maps $f_n: T_n \to T_n'$ such thta $\sigma_n' \circ \Sigma f_n = f_{n+1} \circ \sigma_n$ for all $n$.

    \item A \hldef{spectrum} $E$ is a prespectrum such that the adjoints $\tilde{\sigma}: E_n \to \Omega E_{n+1}$ to the structure maps $\sigma: \Sigma T_n \to T_{n+1}$ are homeomorphisms. A \hldef{morphism $f: E \to E'$ of spectra} is a morphism of prespectra.

    \item The \hldef{suspension spectrum of a based space $X$} is the spectrum $\Sigma^\infty X = \Sigma^\infty(X)$ whose degree $n$ space is $\Sigma^n X$. 
\end{enumerate}
\end{definition}

\hl{$\pi_q^s(X) = \pi_q(\Sigma^\infty X)$}: the \hldef{$q$th stable homotopy group of $X$}, defined by 
$$\pi_q^s(X) = \pi_q(\Sigma^\infty X) \coloneq \colim \pi_{q+n}(\Sigma^n X).$$
By the Freudenthal suspension Theorem \ref{theorem:freudenthal_suspension}, $\Sigma^n: \pi_q(X) \to \pi_{q+n}(\Sigma^n X)$ is an isomorphsim for $q < n-1$, so 
$$\pi_q(\Sigma^\infty X) \cong \pi_{q+n}(\Sigma^n X)$$
for any $n > q+1$. In particular, note that $\pi_q(\Sigma^\infty X)$ is abelian. 

\begin{theorem}{\cite[Theorem Chapter 22 Page 176]{may}}
    Let $\{T_n\}$ be a prespectrum such that $T_n$ is $(n-1)$-connected and of the homotopy type of a CW complex for each $n$. Define
    $$\hlin{\tilde{E}_q(X) = \colim_n \pi_{q+n}(X \wedge T_n)}$$
    where the colimit is taken over the maps 
    $$ \pi_{q+n}\left(X \wedge T_{n}\right) \stackrel{\Sigma}{\rightarrow} \pi_{q+n+1}\left(\Sigma\left(X \wedge T_{n}\right)\right) \cong \pi_{q+n+1}\left(X \wedge \Sigma T_{n}\right) \stackrel{\text { id } \wedge \sigma}{\longrightarrow} \pi_{q+n+1}\left(X \wedge T_{n+1}\right).$$
    Then the functors $\tilde{E}_q$ define a reduced homology theory on based CW complexes.
\end{theorem}

\begin{corollary}
    The stable homotopy groups $\pi_q(\Sigma^\infty X)$ give a reduced homology theory. 
\end{corollary}

\begin{definition}
    The reduced homology theory given by the functors $\{X \mapsto \pi_q(\Sigma^\infty X)\}_{q}$ is called \hldef{stable homotopy theory}.
\end{definition}

More generally, we can define the stable homotopy group of a spectrum.
\begin{definition}[{\cite[Definition 2.1]{nlab:homotopy_group_of_a_spectrum}}] \label{definition:homotopy_group_of_a_spectrum}
    Let $E$ be a spectrum. For $q \in \mathbb{Z}$, the \hldef{$q$th (stable) homotopy group of $E$} is the colimit
    $$\hlin{\pi_q(E) := \colim_n \pi_{q+n}(E_n)}$$
    where the colimit is taken over the maps
    $$\pi_{q+n}(E_n) \xrightarrow{\Sigma} \pi_{q+n+1}(\Sigma E_n) \xrightarrow{\pi_{q+n+1}(\sigma_n^E)} \pi_{q+n+1}(E_{n+1})$$
\end{definition}
Recall that we notated $\pi_q(\Sigma^\infty X)$ as the $q$th stable homotopy group $\pi_q^s(X)$ of the based space $X$ and this indeed coincides with the $q$th homotopy group of the suspension spectrum $\Sigma^\infty X$. 

\subsection{Stable homotopy category}


\begin{definition}[{\cite[Definition 0.38]{nlab:model_structure_on_topological_sequential_spectra}}] \label{definition:model_category_classes_for_spectra_of_topological_spaces}
Let $f_\bullet: X_\bullet \to Y_\bullet$ be a morphism of spectra. 
\begin{enumerate}
    \item We say that $f$ is a \hl{strict weak equivalence} if each component $f_n: X_n \to Y_n$ is a weak homotopy equivalence (i.e. a weak equivalence for the classical model category structure for topological spaces).

    \item We say that $f$ is a \hl{strict fibration} if each component $f_n: X_n \to Y_n$ is a Serre fibration (i.e. a fibration for the classical model category structure for topological spaces).

    \item We say that $f$ is a \hl{strict cofibration} if $f_0: X_0 \to Y_0$ and the maps 
    $$(f_{n+1}, \sigma_n^Y): X_{n+1} \coprod_{S^1 \wedge X_n} (S^1 \wedge Y_n) \to Y_{n+1}$$
    for $n > 1$ are retracts of relative cell complexes (i.e. cofibrations for the classical model category structure for topological spaces).

\end{enumerate}
\end{definition}

\begin{theorem}[{\cite[Theorem 0.40]{nlab:model_structure_on_topological_sequential_spectra}}]
    The classes (i.e. strict weak equivalences, strict fibrations, and strict cofibrations) of morphisms in Definition \ref{definition:model_category_classes_for_spectra_of_topological_spaces} give the category of spectra the structure of a (closed) model category called the {\em strict/level model structure on topological spectra}, denoted by \hl{$\SeqSpec(\Topcg)_\strict$}.
\end{theorem}

\begin{definition}[{\cite[Definition 0.14]{nlab:model_structure_on_topological_sequential_spectra}}]
    A morphism $f: X \to Y$ of spectra is called a \hldef{stable weak homotopy equivalence} if its image 
    $$\pi_\bullet(f): \pi_\bullet(X) \xrightarrow{\simeq} \pi_\bullet(Y)$$
    under the stable homotopy group functor (Definition \ref{definition:homotopy_group_of_a_spectrum}) is an isomorphism.
\end{definition}

\begin{theorem}[{\cite[Theorem 0.70]{nlab:model_structure_on_topological_sequential_spectra}}] \label{theorem:stable_model_structure_topological_spectra}
    The left Bousfield localization of the strict model structure on spectra at the clsas of stable weak homotopy equivalences exists.
\end{theorem}

\begin{definition}
The left Bousfield localization in Theorem \ref{theorem:stable_model_structure_topological_spectra} is a model category called the \hldef{stable model structure on topological spectra}, denoted by \hl{$\SeqSpec(\Topcg)_\stable$}:
    $$\SeqSpec(\Topcg)_{\stable} \rightleftarrows \SeqSpec(\Topcg)_{\strict}.$$
The \hldef{stable homotopy category}, often denoted by \hl{$\SH$} or \hl{$\hospectra$}, is the homotopy category of the model category $\SeqSpec(\Topcg)_\stable$. 
\end{definition}

\begin{claim}[{\cite[Properties]{nlab:stable_homotopy_category}}, {\cite[Proposition 4.14]{nlab:introduction_to_stable_homotopy_theory_--_1-1}}]
    \begin{enumerate}
        \item The smash product of spectra makes $\SH$ into a symmetric monoidal category.
        \item $SH$ has the structure of a triangulated category, where the translation functor is the canonical suspension functor $\Sigma: \SH \to \SH$ and the distinguished triangles are the closures under isomorphisms of triangles of the images (under localization $\SeqSpec(\Topcg)_\stable \to \SH$) of the canonical long homotopy cofiber seqeunces
        $$A \xrightarrow{f} B \to \hocofib(f) \to \Sigma A.$$
        \item $\SH$ is an additive category --- there is an abelian group structure on the pointed hom-sets $[X,Y]$ for $X,Y$ in $\SH$. 
    \end{enumerate}
\end{claim}

\appendix

\section{Miscellaneous definitions}


\begin{definition}[Topology] \label{definition:topological_space}
Let $X$ be a set. A \hldef{topology on $X$} is a collection $\mathcal{T}$ of subsets of $X$ such that:
\begin{enumerate}
    \item $\emptyset \in \mathcal{T}$ and $X \in \mathcal{T}$,
    \item For any collection $\{ U_i \}_{i \in I} \subseteq \mathcal{T}$ (with $I$ arbitrary), the union $\bigcup_{i \in I} U_i \in \mathcal{T}$,
    \item For any finite collection $\{ U_1, \ldots, U_n \} \subseteq \mathcal{T}$, the intersection $U_1 \cap \cdots \cap U_n \in \mathcal{T}$.
\end{enumerate}
If $\mathcal{T}$ is a topology on $X$, the pair $(X, \mathcal{T})$ is called a \hldef{topological space}. Members of $\mathcal{T}$ are called \hldef{open sets}. 

A subset $C \subseteq X$ is \hldef{closed} if its complement $X \setminus C$ is an open set in $\mathcal{T}$

One very often refers to $X$ as a topological spcae, omitting the notation of the topology $\mathcal{T}$. 

The collection of all topologies on a set $X$ may be denoted by notations such as \hl{$\mathrm{Top}(X)$}, \hl{$\mathbf{Top}(X)$}, or \hl{$\mathsf{Top}(X)$}.
\end{definition}






\begin{definition} \label{definition:continuous_map_of_topological_spaces}
Let $(X,\mathcal{T}_X)$ and $(Y,\mathcal{T}_Y)$ be \CrefAndHyperrefIfExist{definition:topological_space}{topological spaces}. A map $f : X \to Y$ is called \hldef{continuous} if for every open set $V \in \mathcal{T}_Y$, the preimage $f^{-1}(V)$ is an open set in $X$, that is,
$$\hlin{\forall V \in \mathcal{T}_Y, \; f^{-1}(V) \in \mathcal{T}_X.}$$
Equivalently, $f$ is continuous if and only if for every closed set $C \subseteq Y$, the preimage $f^{-1}(C)$ is closed in $X$. 


A \hldef{map of topological spaces} usually refers to a continuous map between the topological spaces.

The set of continuous maps from $X$ to $Y$ is sometimes denoted by \hl{$C(X,Y)$}. Other standard notation include \hl{$\operatorname{Hom}_{\mathrm{Top}}(X,Y)$} or \hl{$\operatorname{Top}(X,Y)$} coming from more general notation for morphisms between objects in a \CrefAndHyperrefIfExist{definition:category}{category}.

% The collection of topological spaces along with continuous maps form a \CrefAndHyperrefIfExist{definition:locally_small_category}{locally small} \CrefAndHyperrefIfExist{definition:category}{category}, usually called the \hldef{category of topological spaces} and often denoted by notations such as $\mathrm{Top}$, $\mathbf{Top}$, etc. 

% The set of continuous maps from $X$ to $Y$ is sometimes denoted by \hl{$C(X,Y)$}. Other standard notation include \hl{$\operatorname{Hom}_{\mathrm{Top}}(X,Y)$} or \hl{$\operatorname{Top}(X,Y)$} coming from more general notation for morphisms between objects in a category.
\end{definition}

\begin{definition} \label{definition:coarser_smaller_finer_larger_for_topologies_on_a_set}
    Let $X$ be a set. 
    \begin{enumerate}
        \item Let $\tau_1,\tau_2$ be \CrefAndHyperrefIfExist{definition:topological_space}{topologies on} $X$. Say that $\tau_1$ is \hldef{coarser than} (equivalently, \hldef{smaller than}) $\tau_2$ if $\tau_1 \subseteq \tau_2$, and that $\tau_1$ is \hldef{finer than} (equivalently, \hldef{larger than}) $\tau_2$ if $\tau_2 \subseteq \tau_1$. These relations are denoted by \hl{$\tau_1 \preceq \tau_2$} for “$\tau_1$ coarser than $\tau_2$” and \hl{$\tau_1 \succeq \tau_2$} for “$\tau_1$ finer than $\tau_2$”; their strict versions are \hl{$\tau_1 \prec \tau_2$} and \hl{$\tau_1 \succ \tau_2$}, meaning proper inclusion.

        \item Let $\mathcal{C}$ be some family of topologies on $X$. A topology $\tau \in \mathcal{C}$ is the \hldef{coarsest} (or \hldef{smallest}) element of $\mathcal{C}$ if for every $\sigma \in \mathcal{C}$ one has $\tau \subseteq \sigma$ (equivalently, $\tau \preceq \sigma$ for all $\sigma \in \mathcal{C}$). Dually, $\tau \in \mathcal{C}$ is the \hldef{finest} (or \hldef{largest}) element of $\mathcal{C}$ if for every $\sigma \in \mathcal{C}$ one has $\sigma \subseteq \tau$ (equivalently, $\sigma \preceq \tau$ for all $\sigma \in \mathcal{C}$).
    \end{enumerate}
\end{definition}

\begin{definition} \label{definition:topology_on_a_set_generated_by_a_collection_of_subsets}
Let $X$ be a set and $S \subseteq \mathcal{P}(X)$. The \hldef{topology generated by $S$}, often denoted by notations such as \hl{$\tau(S)$} and \hl{$\calT_S$}, is
$$\tau(S) \coloneqq \bigcap \{\, \mathcal{T} \in \mathsf{Top}(X) \mid S \subseteq \mathcal{T} \,\},$$
\CrefIfExists{definition:topological_space} which is the \CrefAndHyperrefIfExist{definition:coarser_smaller_finer_larger_for_topologies_on_a_set}{coarsest} topology on $X$ that contains $S$.
\end{definition}

\begin{definition} \label{definition:subasis_on_a_set_and_topology_generated_by_a_subbasis}
    Let $X$ be a set and let $S \subseteq \mathcal{P}(X)$ be a family of subsets of $X$. 
    \begin{enumerate}
        \item The family $S$ is a \hldef{subbasis (on $X$)} if $\bigcup S = X$. 

        \item If $\tau$ is a topology on $X$, and $S$ is a subbasis on $X$, then we say that $S$ is a \hldef{subbasis for $\tau$} if and only if \CrefAndHyperrefIfExist{definition:topology_on_a_set_generated_by_a_collection_of_subsets}{$\tau = \tau(S)$}; in this case, members of $S$ are called \hldef{subbasic open sets} of $(X,\tau)$.
        % \item The \hldef{topology generated by the subbasis $S$} is the smallest topology, often denoted by \hl{$\tau(S)$}, on $X$ that contains $S$. More precisely,
        % $$\tau(S) \coloneqq \bigcap\{\, \mathcal{T} \subseteq \mathcal{P}(X) \mid \mathcal{T} \text{ is a topology on } X \text{ and } S \subseteq \mathcal{T} \,\}.$$
        % Equivalently, $\tau(S)$ is the family of all unions of members of $\mathcal{F}(S)$:
        % $$\tau(S) = \{\, \bigcup \mathcal{U} \mid \mathcal{U} \subseteq \mathcal{F}(S) \,\}.$$
    \end{enumerate}
\end{definition}

\begin{definition} \label{definition:basis_for_a_topology}
Let $X$ be a set and let $\mathcal{B}$ be a collection of subsets of $X$. The collection $\mathcal{B}$ is called a \hldef{basis} (or \hldef{base}) for a \CrefAndHyperrefIfExist{definition:topological_space}{topology} on $X$ if the following two conditions hold:
\begin{enumerate}
  \item For every $x \in X$, there exists at least one $B \in \mathcal{B}$ such that $x \in B$.
  \item For any $B_1, B_2 \in \mathcal{B}$ and any $x \in B_1 \cap B_2$, there exists $B_3 \in \mathcal{B}$ such that $x \in B_3 \subseteq B_1 \cap B_2$.
\end{enumerate}
Given such a collection $\mathcal{B}$, the collection $\mathcal{T}$ of all unions of elements of $\mathcal{B}$ defines a topology on $X$, and it coincides with $\calT_\calB$, the \CrefAndHyperref{definition:topology_on_a_set_generated_by_a_collection_of_subsets}{topology generated by $\calB$}. In other words, 
$$\mathcal{T}_\mathcal{B} = \{ U \subseteq X : \text{for every } x \in U, \text{ there exists } B \in \mathcal{B} \text{ with } x \in B \subseteq U \}.$$ 
\end{definition}

\begin{definition}[Pointed topological space] \label{definition:pointed_topological_space}
Let $X$ be a \CrefAndHyperrefIfExist{definition:topological_space}{topological space} and let $x_0 \in X$ be a chosen element of $X$.  
A \hldef{pointed/based (topological) space} is a pair $(X, x_0)$ consisting of the space $X$ together with the distinguished point $x_0$, called the \hldef{base point of $X$}. If the base point of a pointed space $(X,x_0)$ is understood, then it may be suppressed from notation; in particular, $X$ may be written as a pointed space as opposed to the full notation of $(X,x_0)$. 

A \hldef{morphism of pointed spaces} (or \hldef{based map}) or \hldef{continuous map} between pointed spaces $(X, x_0)$ and $(Y, y_0)$ is a \CrefAndHyperrefIfExist{definition:continuous_map_of_topological_spaces}{continuous map} 
$$f : X \to Y$$ 
such that $f(x_0) = y_0$.

The collection of pointed spaces with their morphisms form a \CrefAndHyperrefIfExist{definition:locally_small_category}{locally small} \CrefAndHyperrefIfExist{definition:category}{category}, often called the \hldef{category of pointed spaces}. This category is often denoted by notations such as \hl{$\mathrm{Top}_*$}, \hl{$\mathrm{Top}_\bullet$}, \hl{$\mathbf{Top}_*$}, \hl{$\mathbf{Top}_\bullet$}, etc. The set of continuous maps from pointed spaces $X$ to $Y$ may denoted by notations such as \hl{$C_*(X,Y)$}, \hl{$C_\bullet(X,Y)$}, \hl{$\mathrm{Top}_*(X,Y)$}, \hl{$\mathrm{Top}_\bullet(X,Y)$}, \hl{$\Hom_{\mathrm{Top}_\bullet}(X,Y)$}, etc. 
\end{definition}

\begin{definition}[Homotopy class of maps relative to a subset] \label{definition:homotopy_class_of_maps_of_topological_spaces_relative_to_a_subset}
    Let $X$ and $Y$ be \CrefAndHyperrefIfExist{definition:topological_space}{topological spaces} and let $K \subseteq X$. Let $C(X,Y)$ denote the set of all \CrefAndHyperrefIfExist{definition:continuous_map_of_topological_spaces}{continuous maps} $f : X \to Y$.  

    \begin{enumerate}
        \item Two maps $f, g \in C(X,Y)$ are said to be in the same \hldef{homotopy class relative to $K$} if there exists a \CrefAndHyperrefIfExist{definition:homotopy_of_maps_of_topological_spaces_relative_to_a_subset}{homotopy relative to $K$}
        $$H : X \times [0,1] \to Y$$
        such that
        $$H(x,0) = f(x), \quad H(x,1) = g(x),$$
        and
        $$H(k,t) = f(k) = g(k) \quad \text{for all } k \in K, t \in [0,1].$$

        The \hldef{homotopy class of maps relative to $K$} containing a map $f: X \to Y$ is denoted by \hl{$[f]_K$}.

        Two maps $f, g \in C(X,Y)$ are said to be in the same \hldef{homotopy class} if they are in the same homotopy class relative to $\emptyset$.

        The \hldef{homotopy class of maps} containing a map $f: X \to Y$ is denoted by \hl{$[f]$}.

        The set of homotopy classes of maps may often be denoted by \hl{$[X,Y]$}.

        \item 
        Let $(X, x_0)$ and $(Y, y_0)$ be \CrefAndHyperrefIfExist{definition:pointed_topological_space}{pointed topological spaces} and let $K \subseteq X$ be a subset containing $x_0$. Let $C_*(X,Y)$ denote the set of all continuous based maps $f : X \to Y$ with $f(x_0) = y_0$.

        Two based maps $f, g \in C_*(X,Y)$ are said to be in the same \hldef{homotopy class relative to $K$} if there exists a homotopy of based maps relative to $K$
        $$H : X \times [0,1] \to Y$$
        such that for all $x \in X$,
        $$H(x,0) = f(x), \quad H(x,1) = g(x),$$
        and for all $k \in K$ and $t \in [0,1]$,
        $$H(k,t) = f(k) = g(k),$$
        particularly ensuring the basepoint is fixed throughout,
        $$H(x_0,t) = y_0 \quad \text{for all } t \in [0,1].$$

        The \hldef{homotopy class relative to $K$} containing $f: (X,x_0) \to (Y,y_0)$ is denoted by \hl{$[f]_K$}.

        Two based maps $f, g \in C_*(X,Y)$ are said to be in the same \hldef{homotopy class} if they are in the same homotopy class relative to $\{x_0\}$.

        The \hldef{homotopy class} containing a map $f:(X,x_0) \to (Y,y_0)$ is denoted by \hl{$[f]$}.

        The set of homotopy classes of pointed maps $(X,x_0) \to (Y,y_0)$ may often be denoted by \hl{$[(X,x_0),(Y,y_0)]$} or by \hl{$[X,Y]$} if the base points are clear.
    \end{enumerate}
\end{definition}
\begin{definition} \label{definition:homeomorphism_of_topological_spaces}
Let $(X,\mathcal{T}_X)$ and $(Y,\mathcal{T}_Y)$ be \CrefAndHyperrefIfExist{definition:topological_space}{topological spaces}. A function $f : X \to Y$ is called a \hldef{homeomorphism} if it satisfies all of the following:
\begin{enumerate}
  \item $f$ is \CrefAndHyperrefIfExist{definition:injective_surjective_bijective_map_of_sets}{bijective};
  \item $f$ is \CrefAndHyperrefIfExist{definition:continuous_map_of_topological_spaces}{continuous} with respect to $\mathcal{T}_X$ and $\mathcal{T}_Y$; and
  \item the \CrefAndHyperrefIfExist{definition:injective_surjective_bijective_map_of_sets}{inverse map $f^{-1} : Y \to X$} is also continuous.
\end{enumerate}
If such a function exists, the spaces $X$ and $Y$ are said to be \hldef{homeomorphic}.
\end{definition}


\begin{definition}[Compactness]  \label{definition:compact_subset_of_a_topological_space_alternate_definition}
Let $(X, \mathcal{T})$ be a \hyperrefIfExists{definition:topological_space}{topological space}. A subset $K \subseteq X$ is \hldef{compact} if for every collection $\{ U_i \}_{i \in I}$ of open sets such that $K \subseteq \bigcup_{i \in I} U_i$, there exists a finite subcollection $\{ U_{i_j} \}_{j=1}^n$ with $K \subseteq \bigcup_{j=1}^n U_{i_j}$.
\end{definition}
\begin{definition} \label{definition:compactly_generated_topological_space}
    \TODO{final  topology}
A \CrefAndHyperrefIfExist{definition:topological_space}{topological space} $X$ is said to be \hldef{compactly generated} (or a \hldef{k-space}) if a subset $U \subseteq X$ is open whenever for every \CrefAndHyperrefIfExist{definition:compact_topological_space}{compact} subset $K \subseteq X$, the intersection $U \cap K$ is open in the subspace $K$.  
Equivalently, $X$ is compactly generated if and only if the topology of $X$ is the final topology with respect to the collection of inclusions $K \hookrightarrow X$ for compact $K \subseteq X$.
\end{definition}