%% Delete this \nocite command invocation to make the references section only list out the bibitems that are actually cited.

\section{Definitions}

\subsection{Algebraic groups and linear algebraic groups}

\begin{definition} \label{definition:algebraic_group_scheme_over_a_scheme}
Let $S$ be a scheme. An \hl{algebraic group scheme over $S$} (or an \hl{$S$-group scheme}) is a group object $G$ in the category of schemes over $S$; that is, $G$ is an $S$-scheme equipped with $S$-morphisms:
\hl{$m: G \times_S G \to G$} (\hldef{multiplication}), \hl{$i: G \to G$} (\hldef{inverse}), and \hl{$e: S \to G$} (\hldef{identity}),
satisfying the group axioms expressed by the commutativity of the following diagrams:

\begin{enumerate}
    \item \textbf{Associativity}\quad
    $
    \begin{tikzcd}[column sep=small]
      G \times_S G \times_S G \arrow{r}{m \times \mathrm{id}} \arrow{d}[swap]{\mathrm{id} \times m} & G \times_S G \arrow{d}{m} \\
      G \times_S G \arrow{r}{m} & G
    \end{tikzcd}
    $
    \item \textbf{Identity}\quad
    $
    \begin{tikzcd}[column sep=small]
      G \times_S S \arrow{r}{\mathrm{id} \times e} \arrow{dr}[swap]{\simeq} & G \times_S G \arrow{d}{m} \\
      & G
    \end{tikzcd}
    \qquad
    \begin{tikzcd}[column sep=small]
      S \times_S G \arrow{r}{e \times \mathrm{id}} \arrow{dr}[swap]{\simeq} & G \times_S G \arrow{d}{m} \\
      & G
    \end{tikzcd}
    $
    \item \textbf{Inverse}\quad
    $
    \begin{tikzcd}[column sep=small]
      G \arrow{r}{(\mathrm{id}, i)} \arrow{d}[swap]{\mathrm{id}} & G \times_S G \arrow{d}{m} \\
      G \arrow{r}{e \circ \pi} & G
    \end{tikzcd}
    $
    where $\pi: G \to S$ is the structure morphism and $e \circ \pi$ sends $g$ to the identity section.
\end{enumerate}

\TextIfExists{definition:group_object_in_a_category_with_a_final_object}{Equivalently, a group scheme over $S$ is a \CrefAndHyperrefIfExist{definition:group_object_in_a_category_with_a_final_object}{group object} in the \CrefAndHyperrefIfExist{definition:scheme_over_a_scheme}{category of $S$-schemes}}

If $G$ is \CrefAndHyperrefIfExist{definition:affine_morphism_of_schemes}{affine over} $S$, we call it an \hldef{affine group scheme over $S$}.

If the base scheme $S$ is the spectrum of a field $k$, then we call $G$ a \hldef{$k$-algebraic group} or an \hldef{algebraic group (scheme) over $k$}. If $G$ is additionally a $k$-variety, then we call $G$ a \hldef{$k$-group variety}.
\end{definition}

% \begin{definition} \label{definition:algebraic_group_over_a_field}
%     \TODO{TODO: define variety, group object}
% Let $k$ be a field. An \hl{$k$-algebraic group} (or an \hl{algebraic group over $k$}) is a group object $G$ in the category of $k$-schemes; that is, $G$ is a scheme over $k$ equipped with morphisms $m: G \times G \to G$ (multiplication), $i: G \to G$ (inverse), and $e: \operatorname{Spec} k \to G$ (identity), satisfying the group axioms expressed by the commutativity of the following diagrams:

% \begin{itemize}
%     \item[(Associativity)]
%     \begin{center}
%     \begin{tikzcd}
%     G \times G \times G \arrow{r}{m \times \mathrm{id}} \arrow{d}[swap]{\mathrm{id} \times m} & G \times G \arrow{d}{m} \\
%     G \times G \arrow{r}{m} & G
%     \end{tikzcd}
%     \end{center}

%     \item[(Identity)] 
%     \begin{center}
%     \begin{tikzcd}
%     G \times \operatorname{Spec} k \arrow{r}{\mathrm{id} \times e} \arrow{dr}[swap]{\simeq} & G \times G \arrow{d}{m} \\
%     & G
%     \end{tikzcd}
%     \qquad
%     \begin{tikzcd}
%     \operatorname{Spec} k \times G \arrow{r}{e \times \mathrm{id}} \arrow{dr}[swap]{\simeq} & G \times G \arrow{d}{m} \\
%     & G
%     \end{tikzcd}
%     \end{center}

%     \item[(Inverse)] 
%     \begin{center}
%         \begin{tikzcd}
%         G \arrow{r}{(\mathrm{id}, i)} \arrow{d}[swap]{\mathrm{id}} & G \times G \arrow{d}{m} \\
%         G \arrow{r}{e \circ \pi} & G
%         \end{tikzcd}
%     \end{center}
%     where $\pi: G \to \operatorname{Spec} k$ is the structure morphism and $e \circ \pi$ sends $g$ to the identity.
% \end{itemize}
% If $G$ is a $k$-variety, when we call it a \hldef{$k$-group variety}. If $G$ is an affine $k$-scheme, then we call it an \hldef{affine algebraic $k$-group}.
% \end{definition}

\begin{theorem}[Cartier; see {\cite[Theorem 3.23]{milne_aggs}}] \label{theorem:every_algebraic_group_scheme_over_a_field_of_characteristic_zero_is_smooth}
    Every \CrefAndHyperrefIfExist{definition:algebraic_group_scheme_over_a_scheme}{affine algebraic group} over a field of characteristic zero is smooth.
\end{theorem}


\begin{definition} \label{definition:subgroup_scheme_of_an_algebraic_group_scheme_over_a_scheme}
Let $S$ be a scheme, and let $G$ be an algebraic group scheme over $S$. A \hldef{subgroup scheme} (or synonymously an \hldef{algebraic subgroup}) $H$ of $G$ is an $S$-subscheme
$$ H \subseteq G $$
such that for every $S$-scheme $T$, the subset $H(T) \subseteq G(T)$ is a subgroup of the group $G(T)$. Equivalently, $H$ is a group object in the category of $S$-schemes equipped with a monomorphism of group schemes $H \to G$.
\end{definition}


\TODO{ARe all group schemes over a characteric zero field reduced?}
\TODO{ARe all subgroups of group schemes over a characteric zero field closed?}

Over positive characteristic fields, not all subgroup schemes are closed:

\begin{example}[A subgroup scheme that is not closed]
Let $k$ be an algebraically closed field of characteristic $p > 0$. Consider the additive group scheme $\mathbf{G}_a = \mathrm{Spec}(k[t])$ over $k$. Define the subgroup functor $H$ by
$$ H(R) = \{ x \in \mathbf{G}_a(R) \mid x^{p} = 0 \} $$
for any $k$-algebra $R$. Then $H$ is a subgroup scheme of $\mathbf{G}_a$ known as the \hldef{infinitesimal subgroup of order $p$}.

However, $H$ is not a closed subgroup scheme of $\mathbf{G}_a$ because it corresponds to the ideal $(t^p)$ in $k[t]$, which is not radical and hence does not define a closed subscheme.
\end{example}



\begin{definition} \label{definition:homomorphism_of_algebraic_groups_over_a_scheme}
Let $S$ be a scheme, and let $G$ and $H$ be \hyperrefIfExists{definition:algebraic_group_scheme_over_a_scheme}{$S$-algebraic groups}. A morphism of $S$-schemes $f: G \to H$ is a \hldef{homomorphism of algebraic groups} if $f$ is a group homomorphism, i.e.,
\begin{itemize}
    \item $f(m_G(x, y)) = m_H(f(x), f(y)) \quad \text{for all } x, y \in G,$
    \item $f(i_G(x)) = i_H(f(x))$ for all $x \in G$, and 
    \item $f(e_G) = e_H$.
\end{itemize}
It is called an \hldef{isomorphism of algebraic groups (over $S$)} if it is additionally an isomorphism of $S$-schemes. If there exists an isomorphism $f: G \to H$ of algebraic groups over $S$, then $G$ and $H$ are said to be \hldef{isomorphic $S$-algebraic groups}.
\end{definition}



\begin{definition}[General linear group] \label{definition:general_linear_group_over_a_scheme}
Let \(S\) be a base scheme and \(n \geq 1\) an integer.  
The \hldef{general linear group over $S$}, denoted \hl{$\operatorname{GL}_{n,S}$} or \hldef{$\operatorname{GL}_n/S$} or simply by \hldef{$\operatorname{GL}_n$} if the base $S$ is clear, is the group scheme over \(S\) defined by
$$ \operatorname{GL}_{n,S}(T) = \{\, \text{invertible } n \times n \text{ matrices with entries in } \mathcal{O}_T(T) \,\} $$
for any \(S\)-scheme \(T\), where \(\mathcal{O}_T\) is the structure sheaf of \(T\).

\TextIfExists{definition:general_linear_group_scheme_of_a_locally_free_O_S_module_on_a_scheme}{
    Equivalently, $\operatorname{GL}_{n,S}$ may be defined as the \CrefAndHyperref{definition:general_linear_group_scheme_of_a_locally_free_O_S_module_on_a_scheme}{general linear group scheme $\mathrm{GL}(\mathscr{O}_S^{\oplus n})$ associated to the free $\mathscr{O}_S$-module of rank $n$}.
}
\end{definition}


\begin{definition}[Linear algebraic group over a scheme] \label{definition:linear_algebraic_group_over_a_scheme}
Let \(S\) be a base scheme.  
A \hldef{linear algebraic group over $S$} is an \CrefAndHyperrefIfExist{definition:algebraic_group_scheme_over_a_scheme}{affine group scheme} \(G\) over \(S\) that is finitely presented and smooth over \(S\),  
and such that for some integer \(n \geq 1\), there exists a closed immersion of \(S\)-group schemes
$$ G \hookrightarrow \operatorname{GL}_{n, S}.$$
\CrefIfExists{definition:general_linear_group_over_a_scheme}
\end{definition}


\subsection{Normal subgroup schemes}


\begin{definition} \label{definition:normal_subgroup_and_characteristic_subgroup_of_an_algebraic_group_over_a_scheme}
Let $S$ be a scheme, let $G$ be an \CrefAndHyperref{definition:algebraic_group_scheme_over_a_scheme}{algebraic group over $S$}, and let $H \subseteq G$ be a closed \CrefAndHyperref{definition:subgroup_scheme_of_an_algebraic_group_scheme_over_a_scheme}{subgroup scheme} over $S$. 
\begin{enumerate}
    \item The subgroup scheme $H$ is called a \hldef{normal subgroup scheme of $G$} if for every $S$-scheme $T$, the subgroup $H(T) \subseteq G(T)$ is normalized by $G(T)$, i.e.
    $$ g h g^{-1} \in H(T), \quad \forall \, g \in G(T), \, h \in H(T).  $$
    \item The subgroup scheme $H$ is called a \hldef{characteristic subgroup scheme of $G$} if for every $S$-scheme $T$ and automorphism $\varphi: G_T \to G_T$ of $T$-group schemes, one has
    $$ \varphi(H_T) = H_T.  $$
\end{enumerate}
In particular, all characteristic subgroup schemes are normal subgroup schemes.
\end{definition}



\begin{definition} \label{definition:identity_component_of_a_group_scheme_over_a_scheme}

    \begin{enumerate}
        \item Let $k$ be a field, and let $G$ be a \CrefAndHyperref{definition:algebraic_group_scheme_over_a_scheme}{group scheme} over \CrefAndHyperref{definition:affine_scheme}{$\mathrm{Spec}(k)$}. The \hldef{identity component or neutral component of $G$}, denoted \hl{$G^0$}, is the unique connected component of $G$ containing the identity point $e \in G(k)$. 

        \item Let $S$ be a scheme, and let $G$ be an algebraic group scheme over $S$. The \hldef{identity component of $G$}, denoted \hl{$G^0$}, is the unique open and closed subgroup scheme of $G$ such that for every geometric point $\bar{s} \to S$, the fiber $(G^0)_{\bar{s}}$ is the connected component of the identity element in the group scheme $G_{\bar{s}}$.

        Note that $G^0/S$ might not be connected if $S$ is not connected
    \end{enumerate}
\end{definition}


\begin{lemma}[cf. {\cite[Proposition 1.52]{milne_aggs}}]
    \TODO{Does this hold for more general groups $G/S$?}
    Let $G/k$ be an \CrefAndHyperref{definition:algebraic_group_scheme_over_a_scheme}{algebraic group} over a field. The \CrefAndHyperref{definition:identity_component_of_a_group_scheme_over_a_scheme}{identity component $G^0$} is a \CrefAndHyperref{definition:normal_subgroup_and_characteristic_subgroup_of_an_algebraic_group_over_a_scheme}{characteristic subgroup} and hence a normal subgroup of $G$.
    
    % Let $H/k$ be an \CrefAndHyperref{definition:subgroup_scheme_of_an_algebraic_group_scheme_over_a_scheme}{algebraic subgroup} of $G$. The group $H$ is a \CrefAndHyperref{definition:normal_subgroup_and_characteristic_subgroup_of_an_algebraic_group_over_a_scheme}{normal subgroup} of $G$ if and only if $H(R)$ is 
\end{lemma}


\begin{definition}[Finite index subgroup scheme of an algebraic group scheme over a scheme] \label{definition:finite_index_subgroup_scheme_of_an_algebraic_group_scheme_over_a_scheme}
Let $S$ be a scheme and $G$ an \CrefAndHyperrefIfExist{definition:algebraic_group_scheme_over_a_scheme}{algebraic group scheme over $S$}. A (not necessarily normal) closed subgroup scheme $H \subseteq G$ (i.e., a monomorphism of group schemes $H \to G$ over $S$) is called a \hldef{finite index subgroup scheme} if the fppf (or equivalently, the fpqc) \CrefAndHyperrefIfExist{definition:quotient_of_an_algebraic_group_scheme_by_an_algebraic_subgroup_scheme}{quotient sheaf $G/H$} is \CrefAndHyperrefIfExist{definition:representable_functor_on_a_category_enriched_in_a_monoidal_category}{representable} by a finite $S$-scheme.

\noindent
Equivalently, $H$ is of finite index in $G$ if $H$ is a closed subgroup scheme and the morphism $G \to G/H$ exhibits $G$ as a finite fppf cover of $G/H$.
\end{definition}

\begin{lemma}[see e.g. {\cite[Between Definitions 6.5 and 6.6]{milne_aggs}}] \label{lemma:subgroup_of_an_algebraic_group_over_a_field_has_finite_index_if_and_only_if_dimensions_coincide}
    Let $G/k$ be an \CrefAndHyperrefIfExist{definition:algebraic_group_scheme_over_a_scheme}{algebraic group scheme} and let $H$ be a \CrefAndHyperrefIfExist{definition:subgroup_scheme_of_an_algebraic_group_scheme_over_a_scheme}{subgroup scheme}. 
    \begin{enumerate}
        \item The subgroup $H$ has \CrefAndHyperrefIfExist{definition:finite_index_subgroup_scheme_of_an_algebraic_group_scheme_over_a_scheme}{finite index} if and only if $\dim H = \dim G$. 
        \item If $G$ is smooth, then $H$ has finite index if and only if $H$ contains \CrefAndHyperrefIfExist{definition:identity_component_of_a_group_scheme_over_a_scheme}{$G^0$}. 
    \end{enumerate}
\end{lemma}

\begin{definition}[Quotient of an algebraic group scheme by an algebraic subgroup] \label{definition:quotient_of_an_algebraic_group_scheme_by_an_algebraic_subgroup_scheme}
Let $S$ be a scheme. Let $G$ be an \CrefAndHyperrefIfExist{definition:algebraic_group_scheme_over_a_scheme}{algebraic group scheme over $S$}, and let $H \subseteq G$ be a closed subgroup scheme over $S$, i.e., a monomorphism of group schemes $H \to G$ over $S$.

The \hldef{quotient sheaf} \hl{$G/H$} is the fppf sheafification of the presheaf
$$ T \mapsto G(T)/H(T) $$
on the category of $S$-schemes, where $T$ is any $S$-scheme.

If $G/H$ is \CrefAndHyperrefIfExist{definition:representable_by_schemes_for_a_morphism_of_presheaves_on_Sch_S}{representable by a scheme} (or \CrefAndHyperrefIfExist{definition:representable_by_algebraic_spaces_for_a_1_morphism_of_stacks_in_groupoids_over_a_site_on_S_schemes}{by an algebraic space}) $Q$ over $S$, then $Q$ is called the \hldef{quotient of $G$ by $H$}. In general, if $H$ is \CrefAndHyperrefIfExist{definition:normal_subgroup_and_characteristic_subgroup_of_an_algebraic_group_over_a_scheme}{normal}, then $G/H$ inherits the structure of a group scheme over $S$.

Otherwise, $G/H$ is just a sheaf of sets (or \CrefAndHyperrefIfExist{definition:algebraic_space_on_a_site_on_Sch_S_that_preserves_under_base_change_and_has_tau_local_surjectivity_and_etaleness}{algebraic space on $(\Sch/S)_{fppf}$}) over $S$ without a natural group structure.
\end{definition}

\begin{proposition}[{e.g. \cite[Proposition 5.23]{milne_aggs}}] \label{proposition:dimension_of_group_scheme_equals_sums_of_dimensions_of_subgroup_and_quotient_if_quotient_is_a_scheme}
    Let $G$ be an \CrefAndHyperrefIfExist{definition:algebraic_group_scheme_over_a_scheme}{algebraic group} over a field and let $H$ be an algebraic subgroup such that the \CrefAndHyperrefIfExist{definition:quotient_of_an_algebraic_group_scheme_by_an_algebraic_subgroup_scheme}{quotient $G/H$} is \CrefAndHyperrefIfExist{definition:representable_by_schemes_for_a_morphism_of_presheaves_on_Sch_S}{representable by a scheme}. We have
    $\dim G = \dim H + \dim G/H$.
\end{proposition}

\begin{definition}[Strongly connected algebraic group scheme over a scheme] \label{definition:strongly_connected_algebraic_group_scheme_over_a_scheme}
Let $S$ be a scheme. A \CrefAndHyperrefIfExist{definition:algebraic_group_scheme_over_a_scheme}{group scheme $G$ over $S$} is called a \hldef{strongly connected algebraic group scheme} if it has no proper algebraic gsubgroup over $S$ of finite index.
\end{definition}

\begin{definition} \label{definition:strong_identity_component_of_an_algebraic_group_scheme_over_a_scheme}
Let $S$ be a scheme and $G$ an \CrefAndHyperrefIfExist{definition:algebraic_group_scheme_over_a_scheme}{algebraic group scheme over $S$}. The \hldef{strong identity component} \hl{$G^{\mathrm{so}}$} of $G$ is the intersection of the algebraic subgroups over $S$ of finite index.
\end{definition}

\begin{proposition}[see {\cite[Between Definition 6.9 and Proposition 6.10]{milne_aggs}}] \label{proposition:strong_identity_component_of_a_smooth_algebraic_group_scheme_over_a_field_equals_the_identity_component}
    Let $G/k$ be a smooth algebraic group scheme over a field. The \CrefAndHyperrefIfExist{definition:strong_identity_component_of_an_algebraic_group_scheme_over_a_scheme}{strong identity component $G^{\mathrm{so}}$} coincides with the \CrefAndHyperrefIfExist{definition:identity_component_of_a_group_scheme_over_a_scheme}{identity component $G^{0}$}.
\end{proposition}



\subsection{Major examples of linear algebraic groups}


\begin{definition}[Special linear group] \label{definition:special_linear_group_over_a_scheme}
Let \(S\) be a base scheme and \(n \geq 1\) an integer.  
The \hldef{special linear group} \hl{$\operatorname{SL}_{n,S}$} is the subgroup scheme of \CrefAndHyperrefIfExist{definition:general_linear_group_over_a_scheme}{$\operatorname{GL}_{n,S}$} defined by the condition
$$ \operatorname{SL}_{n,S}(T) = \{\, g \in \operatorname{GL}_{n,S}(T) : \det(g) = 1 \,\}.  $$
\TextIfExists{definition:general_linear_group_scheme_of_a_locally_free_O_S_module_on_a_scheme}{
    Equivalently, a linear algebraic group over $S$ may be regarded as an affine group scheme over $S$ equipped with a \CrefAndHyperref{definition:representation_of_an_affine_algebraic_group_scheme_over_a_scheme}{representation on} the free $\mathscr{O}_S$-module of rank $n$.
}
\end{definition}



\begin{definition}[Orthogonal group] \label{definition:orthogonal_group_over_a_scheme_for_a_quadratic_form}
\TODO{define quadratic form}
Let \(S\) be a base scheme, \(n \geq 1\) an integer, and let \(q\) be a quadratic form on the rank \(n\) free \(\mathcal{O}_S\)-module \(\mathcal{O}_S^n\), i.e., a global section
$$ q : \mathcal{O}_S^n \to \mathcal{O}_S, $$
satisfying the usual properties of a quadratic form.  
The \hldef{orthogonal group} \hl{$\operatorname{O}(q)$} over \(S\) is the subgroup scheme of \CrefAndHyperrefIfExist{definition:general_linear_group_over_a_scheme}{$\operatorname{GL}_{n,S}$} given by
$$ \operatorname{O}(q)(T) = \{\, g \in \operatorname{GL}_{n,S}(T) : q_T(g(v)) = q_T(v) \text{ for all } v \in \mathcal{O}_T^n \,\}, $$
where \(q_T\) is the pullback of \(q\) to \(T\).
\end{definition}


\begin{definition}[Special orthogonal group] \label{definition:special_orthogonal_group_over_a_scheme_for_a_quadratic_form}
    \TODO{quadratic form}
Let $S$ be a base scheme, $n \geq 1$ an integer, and let $q$ be a quadratic form on the rank $n$ free $\mathcal{O}_S$-module $\mathcal{O}_S^n$, i.e., a global section
$$ q : \mathcal{O}_S^n \to \mathcal{O}_S, $$
satisfying the usual properties of a quadratic form.  
The \hldef{special orthogonal group} \hl{$\operatorname{SO}(q)$} is the subgroup scheme of $\operatorname{O}(q)$ consisting of elements with determinant 1:
$$ \operatorname{SO}(q)(T) = \{\, g \in \operatorname{O}(q)(T) : \det(g) = 1 \,\}.  $$
\end{definition}



\TODO{check the definition of unitary group}
\begin{definition}[Unitary group] \label{definition:unitary_group_over_a_scheme}
Let $S$ be a \CrefAndHyperrefIfExist{definition:scheme}{base scheme}. Suppose $A$ is an \CrefAndHyperrefIfExist{definition:algebra_over_a_sheaf_of_rings_on_a_site}{$\mathcal{O}_S$-algebra} equipped with an involution $\ast: A \to A$, and let $h$ be a \CrefAndHyperrefIfExist{definition:hermitian_form_on_a_sheaf_of_rings_with_involution_on_a_site}{hermitian form} on a \CrefAndHyperrefIfExist{definition:free_and_locally_free_modules_on_a_ringed_site}{locally free} $A$-module $M$ of finite rank over $S$.  
The \hldef{unitary group over $S$} is the subgroup scheme \hl{$\operatorname{U}(h)$} of \TODO{general linear group of sheaf algebra} $\operatorname{GL}_{A}(M)$ defined by
$$
\operatorname{U}(h)(T) = \{\,g \in \operatorname{GL}_{A}(M)(T) : h_T(g(x), g(y)) = h_T(x, y) \text{ for all } x,y \in M_T \,\},
$$
where $h_T$ and $M_T$ are the base changes of $h$ and $M$ to $T$.
\end{definition}


\begin{definition}[Special unitary group]  \label{definition:special_unitary_group_over_a_scheme}
Let $S$ be a \CrefAndHyperrefIfExist{definition:scheme}{base scheme}. Suppose $A$ is an \CrefAndHyperrefIfExist{definition:algebra_over_a_sheaf_of_rings_on_a_site}{$\mathcal{O}_S$-algebra} equipped with an involution $\ast: A \to A$, and let $h$ be a \CrefAndHyperrefIfExist{definition:hermitian_form_on_a_sheaf_of_rings_with_involution_on_a_site}{hermitian form} on a \CrefAndHyperrefIfExist{definition:free_and_locally_free_modules_on_a_ringed_site}{locally free} $A$-module $M$ of finite rank over $S$.  
The \hldef{special unitary group} \hl{$\operatorname{SU}(h)$} is the subgroup scheme of \CrefAndHyperrefIfExist{definition:unitary_group_over_a_scheme}{$\operatorname{U}(h)$} consisting of elements of reduced norm 1, i.e.,
\TODO{reduced norm}
$$
\operatorname{SU}(h)(T) = \{\,g \in \operatorname{U}(h)(T) : \operatorname{Nrd}(g) = 1 \,\},
$$
where $\operatorname{Nrd}$ is the reduced norm.
\end{definition}



\begin{definition}[Multiplicative group scheme] \label{definition:multiplicative_group_scheme_over_a_scheme}
Let $S$ be a \CrefAndHyperrefIfExist{definition:scheme}{scheme}.
The \hldef{multiplicative group scheme over $S$}, denoted \hl{$\mathbb{G}_{m,S}$}, is the group scheme over $S$ defined as the open subscheme
$$\mathbb{G}_{m,S} := \mathbb{A}^1_S \setminus \{0\}_S$$
of the \CrefAndHyperrefIfExist{definition:affine_space_of_dimension_n_over_a_scheme}{affine line} $\mathbb{A}^1_S$, equipped with the group law given by multiplication of functions:
$$m: \mathbb{G}_{m,S} \times_S \mathbb{G}_{m,S} \to \mathbb{G}_{m,S}, \quad (x,y) \mapsto xy.$$
The identity section is the morphism
$$e: S \to \mathbb{G}_{m,S}, \quad s \mapsto 1,$$
and the inversion morphism is given by
$$i: \mathbb{G}_{m,S} \to \mathbb{G}_{m,S}, \quad x \mapsto x^{-1}.$$
\end{definition}


\subsection{Unipotent groups, reductive groups, tori, and Borel subgroups}


\begin{definition}[Unipotent element in a linear algebraic group] \label{definition:unipotent_element_in_a_linear_algebraic_group}
Let $k$ be a field, and let $G \subseteq \mathrm{GL}_n(k)$ be a linear algebraic group given by a faithful representation.  
An element $g \in G(k)$ is called a \hldef{unipotent element} if its image in $\mathrm{GL}_n(k)$ is a unipotent matrix, i.e. if all eigenvalues of $g$ in an algebraic closure $\overline{k}$ of $k$ are equal to $1$.  
Equivalently, $g$ is unipotent if $g - I_n$ is nilpotent.
\end{definition}



\begin{definition} \label{definition:unit_upper_triangular_matrix_over_a_ring}
    Let $R$ be a ring. The group of $n \times n$ square matrices over $R$ that are upper triangular and whose diagonal entries are all $1$ is called the \hldef{unitriangular group} or the \hldef{group of unit upper triangular matrices}. It is often denoted by \hl{$U_n$}\footnote{Note that this notation can be confused for or conflicts with notation for the \CrefAndHyperrefIfExist{definition:unitary_group_over_a_scheme}{unitary group}.} or \hl{$UT_n$}.
\end{definition}



\begin{definition} \label{definition:linear_algebraic_group_of_unit_upper_triangular_matrices_over_a_scheme}
    Let $S$ be a scheme and let $n \geq 1$ be an integer. The \hldef{linear algebraic group of unit upper triangular $n \times n$ matrices over $S$} \CrefIfExists{definition:linear_algebraic_group_over_a_scheme} is the subgroup scheme \hl{$U_{n,S} = UT_{n,S}$} (or \hl{$U_n/S = UT_n/S$} or \hl{$U_n = UT_n$} if the base $S$ is clear) of \CrefAndHyperrefIfExist{definition:general_linear_group_over_a_scheme}{$\operatorname{GL}_{n,S}$} defined by 
    $$UT_{n,S}(T) = \{A \in \mathrm{GL}_{n,S}(T): A \text{ is upper triangular with diagonal entries } 1\}$$
    for any $S$-scheme $T$.
\end{definition}



\begin{definition}[Unipotent group] \label{definition:unipotent_algebraic_group_over_a_scheme}
    \begin{enumerate}
        \item An \CrefAndHyperrefIfExist{definition:algebraic_group_scheme_over_a_scheme}{algebraic group scheme} $U/k$ over a field is said to be \hldef{unipotent} if the following equivalent conditions hold:
        \TODO{state some of these conditions more precisely}
        \begin{itemize}
            \item every nonzero representation of the group has a nonzero fixed vector. 
            \item there exists a \CrefAndHyperrefIfExist{definition:faithful_representation_of_an_affine_algebraic_group_scheme_over_a_scheme}{faithful representation} $U \hookrightarrow \mathrm{GL}_n$ whose image consists entirely of \CrefAndHyperrefIfExist{definition:unipotent_element_in_a_linear_algebraic_group}{unipotent matrices}. 
            \item $U$ is \CrefAndHyperrefIfExist{definition:homomorphism_of_algebraic_groups_over_a_scheme}{isomorphic} to a closed subgroup scheme of $UT_{n}/\Spec k$ for some integer $n \geq 1$.
            \item $U$ is a \CrefAndHyperrefIfExist{definition:linear_algebraic_group_over_a_scheme}{linear algebraic group} and for any $k$-algebra $R$, the matrix group $U(R)$ consists entirely of unipotent matrices.
        \end{itemize}

        \item Let \(U\) be an algebraic group scheme over a base scheme \(S\). We say that \(U\) is \hldef{unipotent} if it satisfies one of the following equivalent conditions:
        \begin{itemize}
            \item For every geometric point \(s \in S\), the fiber \(U_s\) is a unipotent algebraic group over the residue field \(\kappa(s)\).
            
            \item There exists a \CrefAndHyperrefIfExist{definition:faithful_representation_of_an_affine_algebraic_group_scheme_over_a_scheme}{faithful representation}
            \[
            \rho: U \hookrightarrow \mathrm{GL}_n
            \]
            over \(S\) such that for every \(S\)-algebra \(R\), the image \(\rho(U(R)) \subseteq \mathrm{GL}_n(R)\) consists entirely of unipotent matrices (i.e., matrices whose eigenvalues are all equal to 1).
            
            \item \(U\) admits a finite central composition series
            \[
            1 = U_0 \triangleleft U_1 \triangleleft \cdots \triangleleft U_r = U,
            \]
            where each successive quotient \(U_{i}/U_{i-1}\) is isomorphic to a subgroup scheme of the additive group scheme \(\mathbb{G}_{a, S}\).
            
            \item \(U\) is isomorphic to a closed subgroup scheme of the group scheme \(UT_n\) of upper unitriangular \(n \times n\) matrices over \(S\) for some integer \(n\).
        \end{itemize}
    \end{enumerate}



\end{definition}


\begin{example}
    Let $k$ be a field.

    \textbf{(1) Characteristic zero case:} Consider the additive group scheme $\mathbb{G}_a$ over $k$. Over a field of characteristic zero, $\mathbb{G}_a$ is a unipotent group scheme.

    \textbf{(2) Positive characteristic case:} For $k$ a field of characteristic $p > 0$, consider the finite group scheme $\alpha_p$ defined as the kernel of the Frobenius morphism on $\mathbb{G}_a$. Then $\alpha_p$ is a unipotent group scheme over $k$.

    These examples illustrate fundamental instances of unipotent group schemes in different characteristic settings.
\end{example}


\begin{definition} \label{definition:unipotent_radical_of_a_group_scheme_that_is_smooth_finitely_presented_and_with_connected_geometric_fibers_over_a_scheme}
    \TODO{define smooth scheme, finitely presented, geometric fibers}
    Let $G$ be a \CrefAndHyperrefIfExist{definition:algebraic_group_scheme_over_a_scheme}{group scheme} that is smooth, finitely presented, with connected geometric fibers over a scheme $S$. The \hldef{unipotent radical of $G$}, denoted \hl{$\mathrm{Rad}_u(G)$}, is the largest smooth, finitely presented, \CrefAndHyperrefIfExist{definition:normal_subgroup_and_characteristic_subgroup_of_an_algebraic_group_over_a_scheme}{normal}, unipotent subgroup of $G$ with connected geometric fibers.
\end{definition}
\TODO{show why the unipotent radical exists}



\begin{definition}[Reductive group] \label{definition:reductive_linear_algebraic_group_over_a_scheme}
Let $G$ be a \CrefAndHyperrefIfExist{definition:linear_algebraic_group_over_a_scheme}{linear algebraic group} that is smooth, finitely presented, and with connected geometric fibers over a scheme $S$.
\begin{itemize}
    % \item The \hldef{unipotent radical of $G$}, denoted \hl{$\mathrm{Rad}_u(G)$}, is the largest smooth connected normal unipotent subgroup of $G$.
    \item In the case that $S$ is $\Spec k$ for a field $k$, the group $G$ is called \hldef{reductive} if $\mathrm{Rad}_u(G)$ is trivial, i.e. $\mathrm{Rad}_u(G) = \{e\}$, and $G$ is smooth, connected, and affine over $S$.
    \item For general $S$, the group $G$ is called \hldef{reductive} if it is a smooth affine group scheme $G/S$ whose geometric fibers are connected reductive algebraic groups.
\end{itemize}
\end{definition}


\begin{lemma} \label{lemma:normal_subgroup_of_a_reductive_group_is_reductive}
    Let $G/k$ be a \CrefAndHyperrefIfExist{definition:reductive_linear_algebraic_group_over_a_scheme}{reductive group} over a field. Any \CrefAndHyperrefIfExist{definition:normal_subgroup_and_characteristic_subgroup_of_an_algebraic_group_over_a_scheme}{normal subgroup scheme} of $G$ over $k$ is a reductive group.
\end{lemma}

\begin{proof}
    Let $N$ be a normal subgroup scheme of $G$. The \CrefAndHyperrefIfExist{lemma:unipotent_radical_of_a_smooth_finitely_presented_scheme_connected_is_a_characteristic_subgroup}{unipotent radical $\mathrm{Rad}_u(N)$} is a characteristic subgroup of $N$ (\Cref{lemma:unipotent_radical_of_a_smooth_finitely_presented_scheme_connected_is_a_characteristic_subgroup}) and hence it is a normal subgroup of $G$. In particular, $\mathrm{Rad}_u(N) \leq \mathrm{Rad}_u(G) = 1$ and hence $\mathrm{Rad}_u(N)$ is trivial, i.e. $N$ is reductive.
\end{proof}

\begin{lemma} \label{lemma:unipotent_radical_of_a_smooth_finitely_presented_scheme_connected_is_a_characteristic_subgroup}
    Let $G/k$ be a smooth, finitely presented scheme over a field. The \CrefAndHyperrefIfExist{definition:unipotent_radical_of_a_group_scheme_that_is_smooth_finitely_presented_and_with_connected_geometric_fibers_over_a_scheme}{unipotent radical $\mathrm{Rad}_u(G)$} is a \CrefAndHyperrefIfExist{definition:normal_subgroup_and_characteristic_subgroup_of_an_algebraic_group_over_a_scheme}{characteristic subgroup} of $G$.
    %\CrefAndHyperrefIfExist{}{reductive group} over a field.  
\end{lemma}
\begin{proof}
    For any $k$-scheme $T$, given a $T$-automorphism $\varphi: G \times_k T \to G \times_k T$, \TODO{continue; }
\end{proof}

\begin{theorem} \label{theorem:category_of_finite_dimensional_representations_of_a_reductive_group_over_a_field_of_characteristic_0_is_semisimple}
    \TODO{define the characteristic of a ring}
    The category of finite-dimensional representations of a \CrefAndHyperrefIfExist{definition:reductive_linear_algebraic_group_over_a_scheme}{reductive group} over a field of characteristic $0$ is \CrefAndHyperrefIfExist{definition:semisimple_additive_category}{semisimple}
\end{theorem}




\begin{definition}[Borel subgroup scheme] \label{definition:borel_subgroup_scheme_of_a_connected_reductive_group_scheme_over_a_scheme}

    \begin{enumerate}
        \item Let $G$ be a connected reductive group over an algebraically closed field $k$. A \hldef{Borel subgroup} $B \subseteq G$ is a maximal connected closed solvable subgroup of $G$. Equivalently, a Borel subgroup is a minimal parabolic subgroup of $G$.

        \item Let $G \to S$ be a connected \CrefAndHyperrefIfExist{definition:reductive_linear_algebraic_group_over_a_scheme}{reductive group scheme} over a base scheme $S$. 
        A \hldef{Borel subgroup scheme of $G$} is a closed subgroup scheme $B \subseteq G$ such that:
        \begin{enumerate}
            \item $B \to S$ is smooth and affine,
            \item for every geometric point $s \to S$, the fiber $B_s \subseteq G_s$ 
            is connected and is a Borel subgroup of the connected reductive algebraic group $G_s$.
        \end{enumerate}
    \end{enumerate}
    One also says \hldef{maximal Borel subgroup} to emphasize the maximality. 
\end{definition}


\begin{proposition}
    Let $G$ be a connected \CrefAndHyperrefIfExist{definition:reductive_linear_algebraic_group_over_a_scheme}{reductive group} over an algebraically closed field $k$. For any two Borel subgroups $B_1,B_2 \subseteq G$, there exists some $g \in G(k)$ such that $B_2 = gB_1g^{-1}$.
\end{proposition}

In certain contexts, maximal compact subgroups and maximal tori of reductive linear algebraic groups may be of interest; in nice enough cases, maximal compact subgroups of reductive groups are unique up to conjugation.

\begin{proposition}
    \begin{enumerate}
        \item Let $G$ be a connected \CrefAndHyperrefIfExist{definition:reductive_linear_algebraic_group_over_a_scheme}{reductive group} over $k = \bbR$ or $\bbC$. For any two maximal compact subgropus $K_1,K_2 \subseteq G$, there exists some $g \in G(k)$ such that $K_2 = gK_1 g^{-1}$.

        \TODO{define a p-adic field}
        \item Let $F$ be a $p$-adic field and let $\calO_F$ be its ring of integers. For any maximal compact subgroup $K \subseteq \GL_n/F$, there exists some $g \in \GL_n(F)$ such that $\GL_n(\calO_F) = g K g^{-1}$.

    \end{enumerate}
\end{proposition}

\begin{proposition}
Let $G$ be a connected \CrefAndHyperrefIfExist{definition:reductive_linear_algebraic_group_over_a_scheme}{reductive group} over an algebraically closed field $k$. for any two maximal tori $T_1,T_2 \subseteq G$, there exists some $g \in G(k)$ such that $T_2 = gT_1 g^{-1}$.
\end{proposition}

\subsubsection{Examples}

\begin{example}[Examples of reductive linear algebraic groups]
    Some standard examples of reductive linear algebraic groups over fields include:
    \begin{enumerate}
        \item The \CrefAndHyperrefIfExist{definition:general_linear_group_over_a_scheme}{general linear group $\mathrm{GL}_n$}, the group of invertible $n \times n$ matrices.
        \item The \CrefAndHyperrefIfExist{definition:special_linear_group_over_a_scheme}{special linear group $\mathrm{SL}_n$}, consisting of matrices with determinant 1, which is connected and semisimple.
        \item The \CrefAndHyperrefIfExist{definition:multiplicative_group_scheme_over_a_scheme}{multiplicative group $\mathbb{G}_m \cong \mathrm{GL}_1$}, a one-dimensional torus.
        \item Products of multiplicative groups, i.e., \emph{algebraic tori} $\mathbb{G}_m^r$.
        \item The \CrefAndHyperrefIfExist{definition:special_orthogonal_group_over_a_scheme_for_a_quadratic_form}{special orthogonal groups $\mathrm{SO}_n$}, preserving nondegenerate quadratic forms (connected for $n \geq 3$).
        \TODO{define the symplectic groups}
        \item The \emph{symplectic groups} $\mathrm{Sp}_{2n}$, preserving nondegenerate alternating bilinear forms.
    \end{enumerate}
    Non-examples include the additive group $\mathbb{G}_a$, since it is unipotent, and Borel subgroups that have nontrivial unipotent radical.
\end{example}

\begin{example}[Borel subgroups of standard reductive groups]
For each of the following reductive groups over an algebraically closed field $k$, a Borel subgroup can be described as follows:
\begin{enumerate}
    \item For the \emph{general linear group} $\mathrm{GL}_n(k)$, a Borel subgroup is the subgroup of \emph{invertible upper triangular matrices}.
    \item For the \emph{special linear group} $\mathrm{SL}_n(k)$, the Borel subgroup is given by the subgroup of upper triangular matrices with determinant 1.
    \item For the \emph{multiplicative group} $\mathbb{G}_m$, the group itself is a torus and hence a Borel subgroup.
    \item For products of multiplicative groups $\mathbb{G}_m^r$, Borel subgroups are the groups themselves since they are tori.
    \item For the \emph{special orthogonal group} $\mathrm{SO}_n(k)$, a Borel subgroup can be realized as the stabilizer of a suitable isotropic flag of subspaces, often realized by certain block upper-triangular matrices preserving the quadratic form.
    \item For the \emph{symplectic group} $\mathrm{Sp}_{2n}(k)$, a Borel subgroup is typically the subgroup preserving a full isotropic flag with respect to the symplectic form, formed by upper-triangular block matrices in a suitable basis.
\end{enumerate}
\end{example}

\begin{example}[Maximal compact subgroups of standard reductive groups]
Some standard examples of maximal compact subgroups of reductive linear algebraic groups over \(\mathbb{R}\) and \(\mathbb{C}\) include:

\begin{enumerate}
    \item \textbf{General linear group \(\mathrm{GL}_n\)}:
    \begin{itemize}
        \item Over \(\mathbb{R}\), a maximal compact subgroup is the orthogonal group \(\mathrm{O}(n)\).
        \item Over \(\mathbb{C}\), the maximal compact subgroup is the unitary group \(\mathrm{U}(n)\).
    \end{itemize}

    \item \textbf{Special linear group \(\mathrm{SL}_n\)}:
    \begin{itemize}
        \item Over \(\mathbb{R}\), a maximal compact subgroup is the special orthogonal group \(\mathrm{SO}(n)\).
        \item Over \(\mathbb{C}\), the maximal compact subgroup is the special unitary group \(\mathrm{SU}(n)\).
    \end{itemize}

    \item \textbf{Multiplicative group \(\mathbb{G}_m \cong \mathrm{GL}_1\)}:
    \begin{itemize}
        \item Over \(\mathbb{R}\), the maximal compact subgroup is \(\{\pm 1\}\), the unit circle in \(\mathbb{R}^*\).
        \item Over \(\mathbb{C}\), the maximal compact subgroup is the unit circle \(S^1 = \{ z \in \mathbb{C} : |z| = 1 \}\).
    \end{itemize}

    \item \textbf{Algebraic tori \(\mathbb{G}_m^r\)}:
    \begin{itemize}
        \item Over \(\mathbb{R}\), a product of copies of \(\{\pm 1\}\) and \(S^1\).
        \item Over \(\mathbb{C}\), the maximal compact subgroup is \((S^1)^r\).
    \end{itemize}

    \item \textbf{Special orthogonal groups \(\mathrm{SO}_n\)}:
    \begin{itemize}
        \item Over \(\mathbb{R}\), \(\mathrm{SO}_n(\mathbb{R})\) itself is compact (for \(n \geq 2\)) and forms its own maximal compact subgroup.
        \item Over \(\mathbb{C}\), the maximal compact subgroup corresponds to the compact real form \(\mathrm{SO}_n(\mathbb{R})\).
    \end{itemize}

    \item \textbf{Symplectic groups \(\mathrm{Sp}_{2n}\)}:
    \begin{itemize}
        \item Over \(\mathbb{R}\), the maximal compact subgroup is \(\mathrm{Sp}(n)\), the compact symplectic group (quaternionic unitary group).
        \item Over \(\mathbb{C}\), the maximal compact subgroup corresponds to the compact real form isomorphic to \(\mathrm{Sp}(n)\).
    \end{itemize}
\end{enumerate}
\end{example}

\section{Representations of linear algebraic groups}

% \begin{definition}
% Let $S$ be a scheme. An \hldef{affine group scheme over $S$} is a group object $G$ in the category of affine $S$-schemes. Equivalently, $G$ is an affine $S$-scheme such that for every $S$-scheme $T$, the set $G(T)$ of $T$-points forms a group, and these group operations are functorial in $T$.
% \end{definition}


\begin{definition} \label{definition:general_linear_group_scheme_of_a_locally_free_O_S_module_on_a_scheme}
    \TODO{define locally free module on a scheme}
Let $S$ be a \CrefAndHyperrefIfExist{definition:scheme}{scheme} and let $\mathcal{E}$ be a locally free $\mathcal{O}_S$-module of finite rank $n$. The \hldef{general linear group scheme associated to $\mathcal{E}$}, denoted \hl{$\mathrm{GL}(\mathcal{E})$}, is the affine group scheme over $S$ representing the functor
$$ \mathrm{GL}(\mathcal{E})(T) = \mathrm{Aut}_{\mathcal{O}_T}(\mathcal{E}_T) $$
for every $S$-scheme $T$, where $\mathcal{E}_T := \mathcal{E} \otimes_{\mathcal{O}_S} \mathcal{O}_T$ and $\mathrm{Aut}_{\mathcal{O}_T}(\mathcal{E}_T)$ denotes the group of $\mathcal{O}_T$-linear automorphisms of $\mathcal{E}_T$.
\end{definition}


\begin{definition} \label{definition:representation_of_an_affine_algebraic_group_scheme_over_a_scheme}
    Let $S$ be a \CrefAndHyperref{definition:scheme}{scheme}, let $G$ be an \CrefAndHyperref{definition:algebraic_group_scheme_over_a_scheme}{affine group scheme} over $S$, and let $\mathcal{E}$ be a locally free $\mathcal{O}_S$-module of finite rank. A \hldef{representation of $G$ on $\mathcal{E}$} is a \CrefAndHyperref{definition:homomorphism_of_algebraic_groups_over_a_scheme}{morphism of $S$-group schemes}
    $$ \rho : G \longrightarrow \mathrm{GL}(\mathcal{E}), $$
    where $\mathrm{GL}(\mathcal{E})$ is the \CrefAndHyperref{definition:general_linear_group_scheme_of_a_locally_free_O_S_module_on_a_scheme}{general linear group scheme associated to $\mathcal{E}$}.

    In the case that $S = \Spec k$ for a field $k$, note that a representation of $G$ is necessarily a morphism of $k$-group schemes
    $$\rho: G \to \GL_{n, \Spec k}$$
    \CrefIfExists{definition:general_linear_group_over_a_scheme} for some $n \geq 1$. 
\end{definition}


\begin{definition} \label{definition:faithful_representation_of_an_affine_algebraic_group_scheme_over_a_scheme}
Let $S$ be a scheme, and let $G$ be an \CrefAndHyperref{definition:algebraic_group_scheme_over_a_scheme}{affine group scheme} over $S$. Let $\mathcal{E}$ be a locally free $\mathcal{O}_S$-module of finite rank. Let
$$ \rho : G \longrightarrow \mathrm{GL}(\mathcal{E}), $$
\TODO{define monomorphism of group schemes}
be a \CrefAndHyperref{definition:representation_of_an_affine_algebraic_group_scheme_over_a_scheme}{representation of $G$}. Such a representation $\rho$ is called \hldef{faithful} if $\rho$ is a monomorphism of group schemes, i.e., if for every $S$-scheme $T$, the induced group homomorphism
$$ \rho_T : G(T) \longrightarrow \mathrm{GL}(\mathcal{E})(T) $$
is injective.
\end{definition}


\section{Algebraic groupoids}


\TODO{groupoid object}


\begin{definition} \label{definition:groupoid}
A \hldef{groupoid} can be defined equivalently in categorical or set-theoretic terms:

\begin{enumerate}
    \item \textbf{Categorical Definition}: A groupoid is a \CrefAndHyperrefIfExist{definition:locally_small_category}{small category} $\mathcal{G}$ in which every morphism is an isomorphism. That is, for every morphism $f: x \to y$ in $\mathcal{G}$, there exists a morphism $g: y \to x$ such that $g \circ f = \operatorname{id}_x$ and $f \circ g = \operatorname{id}_y$.

    \item \textbf{Set-Theoretic Definition}: A groupoid consists of a pair of sets $(G_0, G_1)$, called the \hldef{set of objects} and the \hldef{set of arrows} respectively, equipped with the following structure maps:
    \begin{itemize}
        \item \hldef{Source} and \hldef{Target}: $s, t: G_1 \to G_0$,
        \item \hldef{Identity}: $e: G_0 \to G_1$, assigning to each object $x \in G_0$ an identity arrow $e(x)$,
        \item \hldef{Composition}: A partial map $m: G_1 \times_{s, G_0, t} G_1 \to G_1$, defined on the set of composable pairs
        $$ \hlin{ G_1 \times_{s, G_0, t} G_1 := \{ (g, h) \in G_1 \times G_1 \mid s(g) = t(h) \} } $$
        and denoted by $m(g, h) = g \circ h$,
        \item \hldef{Inverse}: $i: G_1 \to G_1$, denoted by $i(g) = g^{-1}$.
    \end{itemize}
    These structure maps must satisfy the following axioms for all $g, h, k \in G_1$ and $x \in G_0$ where the operations are defined:
    \begin{enumerate}
        \item \textbf{Source and Target Compatibility}:

        \begin{align*}
        s(g \circ h) = s(h), \quad t(g \circ h) = t(g).
        \end{align*}

        \item \textbf{Associativity}: If $s(g) = t(h)$ and $s(h) = t(k)$, then

        $$ (g \circ h) \circ k = g \circ (h \circ k). $$

        \item \textbf{Identity}:
        \begin{align*}
        s(e(x)) = x, \quad t(e(x)) = x, \\
        g \circ e(s(g)) = g, \quad e(t(g)) \circ g = g.
        \end{align*}

        \item \textbf{Inverse}:
        \begin{align*}
        s(g^{-1}) = t(g), \quad t(g^{-1}) = s(g), \\
        g \circ g^{-1} = e(t(g)), \quad g^{-1} \circ g = e(s(g)).
        \end{align*}
    \end{enumerate}
\end{enumerate}
\end{definition}


\begin{definition} \label{definition:groupoid_object_in_a_category}
Let $\mathcal{C}$ be a \CrefAndHyperrefIfExist{definition:category}{(large) category}.

A \hldef{groupoid object in $\mathcal{C}$} consists of two objects $X_0$ (the "object of objects") and $X_1$ (the "object of morphisms"), together with five structure morphisms:
\begin{itemize}
    \item \hldef{Source} and \hldef{Target}: $s, t: X_1 \to X_0$, such that the \CrefAndHyperrefIfExist{definition:cartesian_product_of_two_objects_in_a_category_over_an_object}{fiber product} $X_1 \times_{s,X_0,t} X_1$ of the morphisms $s$ and $t$ exists in $\calC$,
    \item \hldef{Identity}: $e: X_0 \to X_1$,
    \item \hldef{Composition}: $m: X_1 \times_{s, X_0, t} X_1 \to X_1$,
    \item \hldef{Inverse}: $i: X_1 \to X_1$,
\end{itemize}
such that the following conditions hold (expressing the axioms of a category where every morphism is invertible):
\begin{enumerate}
    \item \textbf{Source/Target identities}:
    $$ s \circ e = \operatorname{id}_{X_0}, \quad t \circ e = \operatorname{id}_{X_0} $$
    $$ s \circ m = s \circ \pi_2, \quad t \circ m = t \circ \pi_1 $$
    \item \textbf{Associativity}: The following diagram of composition commutes:
    $$ m \circ (m \times \operatorname{id}_{X_1}) = m \circ (\operatorname{id}_{X_1} \times m) $$
    \item \textbf{Unitality}:
    $$ m \circ (e \circ s, \operatorname{id}_{X_1}) = \operatorname{id}_{X_1}, \quad m \circ (\operatorname{id}_{X_1}, e \circ t) = \operatorname{id}_{X_1} $$
    \item \textbf{Invertibility}:
    $$ m \circ (i, \operatorname{id}_{X_1}) = e \circ s, \quad m \circ (\operatorname{id}_{X_1}, i) = e \circ t $$
\end{enumerate}
\end{definition}

\begin{definition} \label{definition:hopf_algebra_over_a_commutative_ring}
Let $R$ be a \CrefAndHyperrefIfExist{definition:commutative_ring}{commutative ring}.

A \hldef{Hopf algebra over $R$} is a $R$-module $H$ equipped with the structure of a unital associative algebra $(H, \mu, \eta)$ and a counital coassociative coalgebra $(H, \Delta, \varepsilon)$, along with a $R$-linear map $S: H \to H$ called the \hldef{antipode}, satisfying the following compatibility axioms:
\begin{enumerate}
    \item $\Delta$ and $\varepsilon$ are algebra homomorphisms.
    \item The antipode condition:
    $$ \mu \circ (S \otimes \operatorname{id}_H) \circ \Delta = \eta \circ \varepsilon = \mu \circ (\operatorname{id}_H \otimes S) \circ \Delta $$
\end{enumerate}
In Sweedler notation, if $\Delta(h) = \sum h_{(1)} \otimes h_{(2)}$, the antipode condition is expressed as:
$$ \sum S(h_{(1)}) h_{(2)} = \varepsilon(h) 1_H = \sum h_{(1)} S(h_{(2)}) $$
\end{definition}

\begin{definition} \label{definition:hopf_algebroid_over_a_commutative_ring}
Let $R$ be a \CrefAndHyperrefIfExist{definition:commutative_ring}{commutative ring}.

A \hldef{Hopf algebroid over $R$} is a pair of commutative $R$-algebras $(A, \Gamma)$ together with structure maps that form a cogroupoid object in the category of commutative $R$-algebras. Specifically, it consists of:
\begin{itemize}
    \item \hldef{Left and Right Units} (dual to source/target): $\eta_L: A \to \Gamma$ and $\eta_R: A \to \Gamma$. These induce a left $A$-module structure on $\Gamma$ via $\eta_L$ and a right $A$-module structure on $\Gamma$ via $\eta_R$.
    \item \hldef{Counit} (dual to identity): $\varepsilon: \Gamma \to A$, such that $\varepsilon \circ \eta_L = \operatorname{id}_A = \varepsilon \circ \eta_R$.
    \item \hldef{Comultiplication} (dual to composition): $\Delta: \Gamma \to \Gamma \otimes_A \Gamma$, where the tensor product is formed using the right $A$-action on the left factor and the left $A$-action on the right factor.
    \item \hldef{Antipode} (dual to inverse): $S: \Gamma \to \Gamma$, which is an algebra anti-homomorphism (or homomorphism if $\Gamma$ is commutative) satisfying the appropriate compatibility diagrams dual to the groupoid axioms.
\end{itemize}
The pair $(A, \Gamma)$ represents the affine groupoid scheme with objects $\operatorname{Spec}(A)$ and morphisms $\operatorname{Spec}(\Gamma)$.
\end{definition}



\begin{definition} \label{definition:affine_groupoid_scheme_over_a_scheme}
Let $S$ be a scheme.

An \hldef{affine groupoid scheme over $S$} (or simply an \hldef{affine groupoid over $S$}) is a \CrefAndHyperrefIfExist{definition:groupoid_object_in_a_category}{groupoid object} in the category of schemes \CrefAndHyperrefIfExist{definition:affine_morphism_of_schemes}{affine} \CrefAndHyperrefIfExist{definition:scheme_over_a_scheme}{over} $S$. Explicitly, it consists of a pair of $S$-schemes $(G_0, G_1)$ equipped with structural morphisms:
\begin{itemize}
    \item \hldef{Source} and \hldef{Target}: $s, t: G_1 \to G_0$,
    \item \hldef{Identity}: $e: G_0 \to G_1$,
    \item \hldef{Multiplication} (or composition): $m: G_1 \times_{s, G_0, t} G_1 \to G_1$,
    \item \hldef{Inverse}: $i: G_1 \to G_1$,
\end{itemize}
such that:
\begin{enumerate}
    \item The structure morphisms $G_1 \to S$ and $G_0 \to S$ are \CrefAndHyperrefIfExist{definition:affine_morphism_of_schemes}{affine morphisms}.
    \item The morphisms satisfy the standard axioms of a groupoid (associativity of $m$, unitality of $e$, invertibility via $i$).
\end{enumerate}
If $S = \operatorname{Spec}(R)$ for a \CrefAndHyperrefIfExist{definition:commutative_ring}{commutative ring} $k$, then $G_0 = \operatorname{Spec}(A)$ and $G_1 = \operatorname{Spec}(H)$ for some commutative \CrefAndHyperrefIfExist{definition:algebra_of_a_ring}{$R$-algebras} $A$ and $H$, and the data is equivalent to a \CrefAndHyperrefIfExist{definition:hopf_algebroid_over_a_commutative_ring}{Hopf algebroid structure} on the pair $(A, H)$.
\end{definition}




\appendix

\section{Sesquilinear and Hermitian forms}

\begin{definition} \label{definition:involution_on_an_object_in_a_category}
Let \hl{$\mathcal{C}$} be a category. An \hldef{involution on an object $X \in \operatorname{Ob}(\mathcal{C})$} is an endomorphism $i: X \to X$ such that $i \circ i = \operatorname{id}_X$.
\end{definition}
\begin{definition} \label{definition:involution_on_a_ring}
Let $R$ be a unital associative ring. An \hldef{involution on $R$} is a \CrefAndHyperrefIfExist{definition:ring_homomorphism}{ring homomorphism} that is an \CrefAndHyperrefIfExist{definition:involution_on_an_object_in_a_category}{involution}, i.e. a map $\sigma: R \to R$ satisfying the following conditions for all $a, b \in R$:
\begin{enumerate}
    \item $\sigma(a+b) = \sigma(a) + \sigma(b)$;
    \item $\sigma(ab) = \sigma(b)\sigma(a)$ (anti-homomorphism property);
    \item $\sigma(1) = 1$;
    \item $\sigma(\sigma(a)) = a$.
\end{enumerate}
In many contexts, an involution may simply be notated by \hl{$\cdot \mapsto \overline{\cdot}$}
A \hldef{ring with involution} is a ring eqiupped with an involution.
\end{definition}
% \begin{definition} \label{definition:sesquilinear_form_on_a_module_over_a_ring_with_involution}
% Let $(A, \sigma)$ be a \CrefAndHyperrefIfExist{definition:involution_on_a_ring}{ring with involution}. Let $M$ be a left $A$-module.
% A \hldef{$\sigma$-sesquilinear form} (or simply \hldef{sesquilinear form}) on $M$ is a map $b: M \times M \to A$ such that for all $u, v, w \in M$ and $a \in A$:
% \begin{enumerate}
%     \item $b(u + v, w) = b(u, w) + b(v, w)$ and $b(u, v + w) = b(u, v) + b(u, w)$;
%     \item $b(au, v) = ab(u, v)$ (linear in the first variable);
%     \item $b(u, av) = b(u, v)\sigma(a)$ (conjugate-linear in the second variable).
% \end{enumerate}
% Alternatively, one can impose the convention that the module $M$ be a right module and that the sesquilinear form is so that the first variable is conjugate linear and the second variable is linear; for example, \cite{knus} adopts this latter convention.

% A module equipped with a sesquilinear form may be called a \hldef{sesquilinear module}.

% We may denote the set of sesquilinear forms on $M$ by \hl{$\operatorname{Sesq}_A(M)$}. It is a module over the \CrefAndHyperref{definition:center_of_a_ring}{center of $A$}.
% \end{definition}

\begin{definition} \label{definition:sesquilinear_form_on_a_module_over_a_ring_with_involution}
Let $(A, \sigma)$ be a \CrefAndHyperrefIfExist{definition:involution_on_a_ring}{ring with involution}. Let $M$ be a \CrefAndHyperref{definition:module_of_a_ring}{right $A$-module}.
A \hldef{$\sigma$-sesquilinear form} (or simply \hldef{sesquilinear form}) on $M$ is a map $b: M \times M \to A$ such that for all $u, v, w \in M$ and $a \in A$:
\begin{enumerate}
    \item $b(u + v, w) = b(u, w) + b(v, w)$ and $b(u, v + w) = b(u, v) + b(u, w)$;
    \item $b(ua, v) = \sigma(a)b(u, v)$ (conjugate-linear in the first variable);
    \item $b(u, va) = b(u, v)a$ (linear in the second variable).
\end{enumerate}
Alternatively, one can impose the convention that the module $M$ be a left module and that the sesquilinear form is so that the first variable is linear and the second variable is conjugate-linear. The above convention is used in \cite{knus_qhfr}.

We may denote the set of sesquilinear forms on $M$ by \hl{$\operatorname{Sesq}_A(M)$}. It is a module over the \CrefAndHyperref{definition:center_of_a_ring}{center of $A$}.
\end{definition}
\begin{definition} \label{definition:hermitian_form_on_a_module_over_a_ring_with_involution}
Let $(R, \sigma)$ be a \CrefAndHyperrefIfExist{definition:involution_on_a_ring}{ring with involution} and $M$ a \CrefAndHyperrefIfExist{definition:module_of_a_ring}{left $R$-module}.
A \hldef{hermitian form on $M$} is a \CrefAndHyperrefIfExist{definition:sesquilinear_form_on_a_module_over_a_ring_with_involution}{$\sigma$-sesquilinear form} $\phi: M \times M \to R$ satisfying the symmetry condition:
$$ \phi(v, u) = \sigma(\phi(u, v)) $$
for all $u, v \in M$. The pair $(M, \phi)$ is often called a \hldef{hermitian module}.
\end{definition}
\begin{definition} \label{definition:twisted_dual_module_of_a_module_over_a_ring_with_involution}
Let $(R, \sigma)$ be a \CrefAndHyperrefIfExist{definition:involution_on_a_ring}{ring with involution}. Let $N$ be a \CrefAndHyperrefIfExist{definition:module_of_a_ring}{left $R$-module}.
The \hldef{twisted dual module}, denoted \hl{$N^\vee$} or \hl{$\operatorname{Hom}_R(N, R)_\sigma$}, is the set of all \CrefAndHyperrefIfExist{definition:homomorphism_of_modules_over_a_ring}{$R$-linear maps} $f: N \to R$, equipped with the left $R$-module structure given by
$$ (r \cdot f)(n) = f(n) \sigma(r) $$
for all $r \in R$, $f \in N^\vee$, $n \in N$.
\end{definition}
\begin{definition} \label{definition:nondegenerate_sesquilinear_form_on_a_module_over_an_involution_ring}
Let $(R, \sigma)$ be a \CrefAndHyperrefIfExist{definition:involution_on_a_ring}{ring with involution} and $M$ a \CrefAndHyperrefIfExist{definition:module_of_a_ring}{left $R$-module} equipped with a \CrefAndHyperrefIfExist{definition:sesquilinear_form_on_a_module_over_a_ring_with_involution}{sesquilinear form} $\phi: M \times M \to R$.
The form $\phi$ is \hldef{nondegenerate} if the induced \CrefAndHyperrefIfExist{defintiion:adjoint_map_of_a_sesquilinear_form_on_a_module_over_a_ring_with_involution}{adjoint map}
$$\hat{\phi}: M \to M^\vee, \quad u \mapsto (v \mapsto \phi(v, u))$$
is injective.
\end{definition}
% \begin{definition} \label{definition:adjoint_map_of_a_sesquilinear_form_on_a_module_over_a_ring_with_involution}
% Let $(A, \sigma)$ be a ring with involution and $M$ a right $A$-module equipped with a sesquilinear form $b: M \times M \to A$. The \hldef{adjoint map of $b$} is the morphism of $A$-modules
% $$ \hat{b}: M \to M^*, \quad u \mapsto (v \mapsto b(u, v)) $$
% where $M^* = \operatorname{Hom}_A(M, A)_\sigma$ is the \CrefAndHyperrefIfExist{definition:twisted_dual_module_of_a_module_over_a_ring_with_involution}{twisted dual module}. 
% \end{definition}

\begin{definition} \label{definition:adjoint_map_of_a_sesquilinear_form_on_a_module_over_a_ring_with_involution}
Let $(A, \sigma)$ be a \CrefAndHyperrefIfExist{definition:involution_on_a_ring}{ring with involution} and $M$ a \CrefAndHyperrefIfExist{definition:module_of_a_ring}{right $A$-module} equipped with a \CrefAndHyperrefIfExist{definition:sesquilinear_form_on_a_module_over_a_ring_with_involution}{sesquilinear form} $b: M \times M \to A$. We define two $A$-linear mappings associated with $b$:
\begin{enumerate}
    \item The \hldef{left adjoint map} is the morphism of $A$-modules:
    $$ \hat{b}_l: M \to M^*, \quad u \mapsto (v \mapsto b(u, v)) $$
    where $M^* = \operatorname{Hom}_A(M, A)_\sigma$ is the \CrefAndHyperrefIfExist{definition:twisted_dual_module_of_a_module_over_a_ring_with_involution}{twisted dual module}.
    \item The \hldef{right adjoint map} is the morphism of $A$-modules:
    $$ \hat{b}_r: M \to M^*, \quad v \mapsto (u \mapsto \sigma(b(u, v))) $$
\end{enumerate}
If $b$ is an \CrefAndHyperrefIfExist{definition:varepsilon_hermitian_form_on_a_module_over_a_ring_with_involution}{$\varepsilon$-hermitian form}, these maps are related by $\hat{b}_l = \hat{b}_r \cdot \varepsilon$ (under the standard identification of the module and its dual).

When we simply talk about the \hldef{adjoint map} of $b$ on the right $A$-module $M$, we will mean the left adjoint and denote it by \hl{$\hat{b}$}.
\end{definition}
\begin{proposition} \label{proposition:sesquilinear_form_on_a_finitely_generated_projective_module_over_an_involution_ring_is_nondegenerate_if_and_only_if_adjoint_map_is_isomorphism}
Let $(R, \sigma)$ be a \CrefAndHyperrefIfExist{definition:involution_on_a_ring}{ring with involution} and $\phi:M \times M \to R$ a \CrefAndHyperrefIfExist{definition:sesquilinear_form_on_a_module_over_a_ring_with_involution}{sesquilinear form}.
If $M$ is a \CrefAndHyperrefIfExist{definition:finitely_generated_modules_over_rings}{finitely generated} \CrefAndHyperrefIfExist{definition:projective_bimodule_over_rings}{projective} $R$-module, then the \CrefAndHyperrefIfExist{defintiion:adjoint_map_of_a_sesquilinear_form_on_a_module_over_a_ring_with_involution}{adjoint map} $\hat{\phi}: M \to \operatorname{Hom}_R(M, R)_\sigma$ is an isomorphism if and only if $\phi$ is \CrefAndHyperrefIfExist{definition:nondegenerate_sesquilinear_form_on_a_module_over_an_involution_ring}{non-degenerate}.
\end{proposition}
\begin{definition} \label{definition:involution_on_a_sheaf_of_rings_on_a_site}
Let \hl{$(\mathcal{S}, \tau)$} be a \CrefAndHyperrefIfExist{definition:grothendieck_topology_on_a_category_site_covering_sieve_topologically_generating_family}{site}. Let $\mathcal{O}$ be a \CrefAndHyperrefIfExist{definition:sheaf_on_a_site}{sheaf} of rings on $\mathcal{S}$.
An \hldef{involution on $\mathcal{O}$} is a morphism of sheaves of rings \hl{$\sigma: \mathcal{O} \to \mathcal{O}^{\operatorname{op}}$} such that $\sigma \circ \sigma = \operatorname{id}_{\mathcal{O}}$, i.e. an \CrefAndHyperrefIfExist{definition:involution_on_an_object_in_a_category}{involution} in the category of sheaves of rings on $\mathcal{S}$.
\end{definition}
\begin{definition} \label{definition:sesquilinear_form_on_a_sheaf_of_modules_over_a_sheaf_of_rings_on_a_site}
Let $(\mathcal{S}, \tau)$ be a \CrefAndHyperrefIfExist{definition:grothendieck_topology_on_a_category_site_covering_sieve_topologically_generating_family}{site}. Let $(\mathcal{O}, \sigma)$ be a \CrefAndHyperrefIfExist{definition:sheaf_on_a_site}{sheaf of rings} on $\mathcal{S}$ with \CrefAndHyperrefIfExist{definition:involution_on_a_sheaf_of_rings_on_a_site}{involution} $\sigma: \calO \to \calO$. Let $\mathcal{E}$ be a \CrefAndHyperrefIfExist{definition:module_over_a_sheaf_of_rings_on_a_site}{sheaf of left $\mathcal{O}$-modules} on $\mathcal{S}$.

A \hldef{sesquilinear form on $\mathcal{E}$} is a \CrefAndHyperrefIfExist{definition:sheaf_on_a_site}{morphism of sheaves} of abelian groups
$$\phi: \mathcal{E} \times \mathcal{E} \to \mathcal{O}$$
such that for every object $U \in \mathcal{S}$, the induced map on sections
$$ \phi_U: \mathcal{E}(U) \times \mathcal{E}(U) \to \mathcal{O}(U) $$
is a \CrefAndHyperrefIfExist{definition:sesquilinear_form_on_a_module_over_a_ring_with_involution}{$\sigma_U$-sesquilinear form} on the $\mathcal{O}(U)$-module $\mathcal{E}(U)$, where $\sigma_U: \mathcal{O}(U) \to \mathcal{O}(U)$ is the \CrefAndHyperrefIfExist{definition:involution_on_a_ring}{involution} on global sections.
\end{definition}
\begin{definition} \label{definition:hermitian_form_on_a_sheaf_of_rings_with_involution_on_a_site}
Let $(\mathcal{O}, \sigma)$ be a \CrefAndHyperrefIfExist{definition:sheaf_on_a_site}{sheaf} of rings with \CrefAndHyperrefIfExist{definition:involution_on_a_sheaf_of_rings_on_a_site}{involution} on a \CrefAndHyperrefIfExist{definition:grothendieck_topology_on_a_category_site_covering_sieve_topologically_generating_family}{site} $\mathcal{S}$. Let $\mathcal{E}$ be a \CrefAndHyperrefIfExist{definition:module_over_a_sheaf_of_rings_on_a_site}{sheaf of left $\mathcal{O}$-modules}.

A \hldef{hermitian form on $\mathcal{E}$} is a morphism of sheaves of sets
$$ \phi: \mathcal{E} \times \mathcal{E} \to \mathcal{O} $$
such that for every object $U$ in $\mathcal{S}$, the map on sections
$$ \phi_U: \mathcal{E}(U) \times \mathcal{E}(U) \to \mathcal{O}(U) $$
is a \CrefAndHyperrefIfExist{definition:hermitian_form_on_a_module_over_a_ring_with_involution}{hermitian form} on the $\mathcal{O}(U)$-module $\mathcal{E}(U)$ with respect to the \CrefAndHyperrefIfExist{definition:involution_on_a_ring}{involution} $\sigma_U: \mathcal{O}(U) \to \mathcal{O}(U)$.
\end{definition}


\section{Miscellaneous definitions}


\begin{definition}[Affine scheme] \label{definition:affine_scheme}
Let $A$ be a \CrefAndHyperrefIfExist{definition:commutative_ring}{commutative ring with unity}. Define the set \hl{$\mathrm{Spec}(A)$}
to be the set of all \CrefAndHyperrefIfExist{definition:prime_and_maximal_ideal_of_a_ring}{prime ideals} of $A$. Equip it with the \hldef{Zariski topology}, which is the \CrefAndHyperrefIfExist{definition:topological_space}{topology} whose closed sets are given by \hldef{vanishing loci}
$$\hlin{V(I) = \{\mathfrak{p} \in \mathrm{Spec}(A) : I \subseteq \mathfrak{p}\}}$$
for ideals $I \subseteq A$.  
Define the sheaf \hl{$\mathcal{O}_{\mathrm{Spec}(A)}$}, called the \hldef{structure sheaf of $\Spec A$}, by 
$$\mathcal{O}_{\mathrm{Spec}(A)}(U) = \{ \ \text{locally defined fractions of elements of $A$ on $U$} \ \},$$
for each open set $U \subseteq \mathrm{Spec}(A)$. It is the case that the stalk at $\mathfrak{p} \in \mathrm{Spec}(A)$ is canonically the \CrefAndHyperrefIfExist{definition:localization_of_a_commutative_ring_by_a_multiplicative_subset}{localization $A_{\mathfrak{p}}$}.  
Then $(\mathrm{Spec}(A), \mathcal{O}_{\mathrm{Spec}(A)})$ is a \CrefAndHyperrefIfExist{definition:locally_ringed_space_on_a_topological_space}{locally ringed space}, called the \hldef{affine scheme associated to $A$}.

Moreover, given $f \in A$, we define the locus \hl{$D(f)$} by 
$$D(f) = \Spec A \setminus V((f)) = \{ \mathfrak{p} \in \operatorname{Spec} A : f \notin \mathfrak{p} \}$$
\end{definition}


\begin{definition}[Scheme] \label{definition:scheme}
    A \hldef{scheme} is a \CrefAndHyperrefIfExist{definition:locally_ringed_space_on_a_topological_space}{locally ringed space} $(X, \mathcal{O}_X)$ that admits an open cover $\{U_i\}_{i \in I}$ such that each $(U_i, \mathcal{O}_X|_{U_i})$ is \CrefAndHyperrefIfExist{definition:morphism_of_locally_ringed_spaces}{isomorphic (as a locally ringed space)} to an \CrefAndHyperrefIfExist{definition:affine_scheme}{affine scheme $(\mathrm{Spec}(A_i), \mathcal{O}_{\mathrm{Spec}(A_i)})$} for some \CrefAndHyperrefIfExist{ring}{commutative ring} $A_i$.  
    In other words, a scheme is a locally ringed space locally isomorphic to affine schemes.

    
\end{definition}



\begin{definition}[Additive category] \label{definition:additive_category}
Let $\mathcal{A}$ be a \CrefAndHyperrefIfExist{definition:locally_small_category}{locally small category}. 
\begin{enumerate}
    \item $\calA$ is said to be a \hldef{preadditive category} if the following hold:
    \begin{itemize}
        \item For any two objects $A, B$ in $\mathcal{A}$, the set $\operatorname{Hom}_{\mathcal{A}}(A, B)$ is an \CrefAndHyperrefIfExist{definition:group}{abelian group}, and composition of morphisms is bilinear.
        \item There is a \CrefAndHyperrefIfExist{definition:initial_final_zero_objects_of_a_category}{zero object} $0$ in $\mathcal{A}$.
    \end{itemize}
    \TextIfExists{definition:category_enriched_in_a_monoidal_category}{Equvialently, a preadditive cateogry $\calA$ is a (necessarily locally small) category \CrefAndHyperrefIfExist{definition:category_enriched_in_a_monoidal_category}{enriched in} the \CrefAndHyperrefIfExist{definition:monoidal_category}{monoidal category} $\Ab$ that also possesses a zero object.}

    \item
    If $\calA$ is preadditive, then it is called \hldef{additive} if it additionally satisfies the following:
    \begin{itemize}
        \item For any two objects $A, B$ in $\mathcal{A}$, there exists a \CrefAndHyperrefIfExist{definition:product_and_coproduct_of_objects_in_a_category}{product object $A \times B$}, often written \hl{$A \oplus B$}, called the \hldef{direct sum of $A$ and $B$}. In fact, $A \oplus B$ is not only a product but also a \CrefAndHyperrefIfExist{definition:coproduct_of_modules_of_rings}{coproduct} of $A$ and $B$\CrefIfExists{lemma:finite_products_and_finite_coproducts_coincide_in_preadditive_categories}.
    \end{itemize}

    Given a finite collection $\{A_i\}_i$ of objects $A_i$ in an additive category $\calA$, we may more generally speak of the \hldef{direct sum} \hl{$\bigoplus_i A_i$}; it has canonical injections from and projections to each $A_i$.


\end{enumerate}
\end{definition}


\begin{definition} \label{definition:semisimple_object_of_an_additive_category}
Let $\mathcal{C}$ be an \CrefAndHyperrefIfExist{definition:additive_category_preadditive_category}{additive category}. An object $X \in \mathrm{Ob}(\mathcal{C})$ is called \hldef{semisimple} if it is isomorphic to a finite \CrefAndHyperrefIfExist{definition:additive_category_preadditive_category}{direct sum} of \CrefAndHyperrefIfExist{definition:simple_object_of_an_additive_category}{simple objects} in $\mathcal{C}$.
\end{definition}


\begin{definition} \label{definition:semisimple_additive_category}
An \CrefAndHyperrefIfExist{definition:additive_category_preadditive_category}{additive category} $\mathcal{C}$ is called a \hldef{semisimple category} if every object of $\mathcal{C}$ is \CrefAndHyperrefIfExist{definition:semisimple_object_of_an_additive_category}{semisimple}.
\end{definition}


\begin{definition} \label{definition:representable_functor_on_a_category_enriched_in_a_monoidal_category}
    Let $C$ be a \CrefAndHyperrefIfExist{definition:category_enriched_in_a_monoidal_category}{category enriched in a monidal category} $\mathcal{V}$. Given an object $X$ of $C$, the \hldef{functor of points} \hl{$h_X$} is the \CrefAndHyperrefIfExist{definition:functor_between_categories}{functor}/\CrefAndHyperrefIfExist{definition:presheaf_on_a_category}{presheaf} $C^{\op} \to \mathcal{V}$ given by $T \mapsto \Hom_C(T, X)$. A functor $C^{\op} \to \mathcal{V}$ (or equivalently, a presheaf on $C$ valued in $\mathcal{V}$) is said to be \hldef{representable} if it is \CrefAndHyperrefIfExist{definition:natural_transformation_between_functors_between_categories}{naturally isomorphic} to some functor $h_X$ of points for an object $X$ of $C$.

    Dually, a functor $C \to \calV$ is called \hldef{co-representable} if it is naturally isomorphic to a functor $T \mapsto \Hom_C(X, T)$ for an object $X$ in $C$. 

    For instance, we may speak of these notions when $\calV$ is the monoidal category $\Sets$, i.e. $C$ is a \CrefAndHyperrefIfExist{definition:locally_small_category}{locally small category}.
\end{definition}
\begin{definition} \label{definition:representable_by_schemes_for_a_morphism_of_presheaves_on_Sch_S}
    \TODO{make this definition be for more general presheaves valued in more general caetgories}
    Let $S$ be a scheme and let $f: F \to G$ be a \CrefAndHyperrefIfExist{definition:presheaf_on_a_category}{morphism of presheaves} valued in sets on $\mathit{Sch}/S$.  We say that $f$ is \hldef{representable by schemes} if for every scheme $T$ over $S$ and every morphism $h_T \to G$ of sheaves (i.e., every $T$-point of $G$), the fiber product
    \[ h_T \times_G F \]
    is \CrefAndHyperrefIfExist{definition:representable_functor_on_a_category_enriched_in_a_monoidal_category}{representable} by a scheme over $T$, i.e. there exists a scheme $X$ over $T$ such that $h_T \times_G F \cong h_X$ as presheaves.
\end{definition}
\begin{definition} \label{definition:representable_by_algebraic_spaces_for_a_1_morphism_of_stacks_in_groupoids_over_a_site_on_S_schemes}

\TODO{TODO: read }
\TODO{TODO: state that a sheaf gives a stack}
\TODO{TODO: find out if it is equivalent to have $U$ be an algebraic space instead.}
Let $S$ be a \CrefAndHyperrefIfExist{definition:scheme}{scheme}. Let $\tau$ be a \CrefAndHyperrefIfExist{definition:grothendieck_topology_on_a_category_site_covering_sieve_topologically_generating_family}{Grothendieck topology} on $\mathit{Sch}/S$. Let $f: \mathcal{X} \to \mathcal{Y}$ be a $1$-morphism of \CrefAndHyperrefIfExist{definition:stack_in_groupoids_over_a_site}{stacks in groupoids} over the site $(\mathit{Sch}/S)_{\tau}$. We say that $f$ is \hldef{representable by algebraic spaces} if for every scheme $U$ over $S$ and every $1$-morphism $y: U \to \mathcal{Y}$ (i.e., every object $y \in \mathcal{Y}(U)$), the \CrefAndHyperrefIfExist{definition:2_fiber_product_of_1_morphisms_in_a_2_category}{$2$-fiber product}
\[
\mathcal{X} \times_{\mathcal{Y}} U
\]
is equivalent (as a stack over $(\mathit{Sch}/U)_{\tau}$) to an \CrefAndHyperrefIfExist{definition:algebraic_space_on_a_site_on_Sch_S_that_preserves_under_base_change_and_has_tau_local_surjectivity_and_etaleness}{algebraic space over $U$}.
\end{definition}
\begin{definition} \label{definition:algebraic_space_on_a_site_on_Sch_S_that_preserves_under_base_change_and_has_tau_local_surjectivity_and_etaleness}
Let $S$ be a scheme. Let $\tau$ be a \CrefAndHyperrefIfExist{definition:grothendieck_topology_on_a_category_site_covering_sieve_topologically_generating_family}{Grothendieck topology} on \CrefAndHyperrefIfExist{definition:representable_by_schemes_for_a_morphism_of_presheaves_on_Sch_S}{$\mathit{Sch}/S$} for which surjectivity and \'etaleness are preserved under base change and \CrefAndHyperrefIfExist{definition:tau_local_property_of_a_morphism_of_schemes_over_a_scheme}{$\tau$-local on the base}\TextIfExists{proposition:list_of_properties_of_morphisms_of_schemes_over_a_base_preserved_under_base_change_and_fpqc_local_on_the_base}{\footnote{For example, $\tau$ can be any Grothendieck topology coarser than the fpqc topology thanks to \CrefIfExists{proposition:list_of_properties_of_morphisms_of_schemes_over_a_base_preserved_under_base_change_and_fpqc_local_on_the_base}.}}. An \hldef{algebraic space over $S$ for the topology $\tau$} is a sheaf $F$ of sets for $(\mathit{Sch}/S)_{\tau}$ such that 
\begin{enumerate}
    \item The \CrefAndHyperrefIfExist{definition:diagonal_morphism_of_a_morphism_of_schemes}{diagonal morphism} $F \to F \times F$ is \CrefAndHyperrefIfExist{definition:representable_by_schemes_for_a_morphism_of_presheaves_on_Sch_S}{representable by schemes};
    \item There exists a scheme $U$ over $S$ and a morphism of sheaves $h_U \to F$\TextIfExists{lemma:diagonal_morphism_for_presheaf_is_representable_by_schemes_iff_morphism_from_a_scheme_to_the_presheaf_is_representable_by_schemes}{\footnote{Recall by Lemma \ref{lemma:diagonal_morphism_for_presheaf_is_representable_by_schemes_iff_morphism_from_a_scheme_to_the_presheaf_is_representable_by_schemes} that such a morphism of sheaves is necessarily representable by schemes.}} which is \CrefAndHyperrefIfExist{definition:morphism_of_sheaves_with_a_property_of_morphisms_of_S_schemes_preserved_under_base_change_and_tau_local_on_the_base}{surjective and \'etale}.
\end{enumerate}
\end{definition}



\begin{definition} \label{definition:affine_morphism_of_schemes}
Let $f : X \to Y$ be a \CrefAndHyperrefIfExist{definition:morphism_of_schemes}{morphism of schemes}. We say that $f$ is an \hldef{affine morphism} if for every \CrefAndHyperrefIfExist{definition:affine_open_subscheme_of_a_scheme}{affine open} $V = \operatorname{Spec} B \subseteq Y$, the \CrefAndHyperrefIfExist{definition:preimage_of_an_open_subset_of_a_scheme_under_a_morphism_of_schemes}{preimage $U = f^{-1}(V)$} is an \CrefAndHyperrefIfExist{definition:affine_scheme}{affine scheme}.
\end{definition}


\begin{definition}[Scheme over a scheme] \label{definition:scheme_over_a_scheme}
    Let $(S, \mathcal{O}_S)$ be a scheme. A \hldef{scheme over $S$} (or an \hldef{$S$-scheme}) is a scheme $(X, \mathcal{O}_X)$ together with a morphism of schemes
    $$\pi: (X, \mathcal{O}_X) \to (S, \mathcal{O}_S).$$
    This morphism $\pi$ is called the \hldef{structure morphism of the scheme $X$ over $S$}.  

    If $S = \mathrm{Spec}(R)$ is an affine scheme for a commutative ring $R$, then an $S$-scheme is synonymously called an \hldef{$R$-scheme} or a \hldef{scheme over $R$}. 

    Let $(S, \mathcal{O}_S)$ be a scheme, and let $(X, \mathcal{O}_X)$ and $(Y, \mathcal{O}_Y)$ be schemes over $S$ with structure morphisms
    $$\pi_X: X \to S, \quad \pi_Y: Y \to S.$$
    A \hldef{morphism of $S$-schemes} (or synonymously a \hldef{$S$-scheme morphism}) is a \CrefAndHyperrefIfExist{definition:morphism_of_schemes}{morphism of schemes}
    $$(f, f^\#): (X, \mathcal{O}_X) \to (Y, \mathcal{O}_Y)$$
    such that the following diagram commutes:
    $$
    \begin{array}{ccc}
    X & \xrightarrow{f} & Y \\
    {\scriptstyle \pi_X} \downarrow & & \downarrow {\scriptstyle \pi_Y} \\
    S & = & S
    \end{array}
    $$
    In other words,
    $$\pi_Y \circ f = \pi_X.$$

    Given a fixed scheme $S$, there is a category, often denoted by \hl{$\mathrm{Sch}_S$}, \hl{$\mathrm{Sch}_{/S}$}, \hl{$\mathrm{Sch}/S$}, \hl{$\mathbf{Sch}_S$}, \hl{$\mathbf{Sch}_{/S}$}, \hl{$\mathbf{Sch}/S$} etc. whose objects are schemes $T$ over $S$ and whose morphisms $T_1 \to T_2$ are morphisms of schemes over $S$. If $S = \Spec R$ for some commutative ring $R$, then we may instead write \hl{$\mathrm{Sch}_R$} to denote $\mathrm{Sch}_{\Spec R}$, etc. It is noteworthy that $\mathrm{Sch}_\bbZ$ coincides with the category \CrefAndHyperrefIfExist{definition:morphism_of_schemes}{$\mathrm{Sch}$} of all schemes. In other words, a $\bbZ$-scheme can be identified simply with a scheme. 

    \TextIfExists{definition:category_of_objects_over_under_a_fixed_object_in_a_category}{Equivalently, the category $\mathrm{Sch}_{/S}$ is the category of schemes over $S$ in the sense of \Cref{definition:category_of_objects_over_under_a_fixed_object_in_a_category}.}

\end{definition}


\begin{definition} \label{definition:commutative_ring}
   A \hldef{commutative (unital) ring} is a \CrefAndHyperrefIfExist{definition:ring}{ring} $(R, +, \cdot)$ such that $\cdot$ is a \CrefAndHyperrefIfExist{definition:commutative_binary_operation}{commutative operation}, i.e. $a \cdot b = b \cdot a$. 

   For many writers (e.g. ``commutative'' algebraists or number theorists), a \hldef{ring} refers to a commutative ring as above.
\end{definition}


\begin{definition} \label{definition:algebra_of_a_ring}
Let $R$ be a \CrefAndHyperrefIfExist{definition:ring}{(not-necessarily commutative) ring with unity}. An \hldef{$R$-algebra} is a ring $A$ together with a \CrefAndHyperrefIfExist{definition:ring_homomorphism}{ring homomorphism} 
$$\varphi : R \to A$$ 
into the \CrefAndHyperrefIfExist{definition:center_of_a_ring}{center $Z(A)$} of $A$ (so that $\varphi(r)$ commutes with every element of $A$ for all $r \in R$), such that $\varphi(1_R) = 1_A$. The ring homomorphism $\varphi$ is called the \hldef{structure map} of the algebra.

Equivalently, an $R$-algebra consists of a ring $A$ endowed with a \CrefAndHyperrefIfExist{definition:module_of_a_ring}{two-sided $R$-module} structure for which the scalar multiplication satisfies
$$ r \cdot (ab) = (r \cdot a) b = a (r \cdot b) \quad \text{for all } r \in R, \, a,b \in A. $$

In particular, any ring homomorphism between \CrefAndHyperrefIfExist{definition:commutative_ring}{commutative rings} specifies an algebra structure.
\end{definition}

\begin{definition} \label{definition:cartesian_product_of_two_objects_in_a_category_over_an_object}
    Let $\mathcal{C}$ be a \CrefAndHyperrefIfExist{definition:category}{category}, let $Z$ be an object, and let $X, Y$ be objects of $\mathcal{C}$ \CrefAndHyperrefIfExist{definition:category_of_objects_over_under_a_fixed_object_in_a_category}{over} $Z$, i.e. morphisms $X \to Z$ and $Y \to Z$ are fixed. A \hldef{cartesian product of $X$ and $Y$ over $Z$ in $\mathcal{C}$} (or \hldef{fiber product} or \hldef{pullback diagram}) is an object, often denoted by \hl{$X \times_Z Y$}, with \hldef{projection morphisms} $X \times_Z Y \to X$ and $X \times_Z Y \to Y$ that are universal. 
    More precisely, for any object $T$ of $\mathcal{C}$ and morphisms $f_X : T \to X$, $f_Y : T \to Y$, there exists a unique morphism $u : T \to X \times_Z Y$ such that the following diagram commutes:
        \begin{center}
        \begin{tikzcd}
            T \ar[rd, dotted, "u" ] \ar[rrd, "f_X", bend left] \ar[ddr, "f_Y", bend right] & & \\
            & X \times_Z Y \ar[r] \ar[d] &  \ar[d] X \\
            & Y \ar[r] & Z
        \end{tikzcd}
        \end{center}
        Equivalently, $X \times_Z Y$ is the \CrefAndHyperrefIfExist{definition:limit_and_colimit_of_a_diagram_in_a_category}{limit} of the \CrefAndHyperrefIfExist{definition:diagram_in_a_category_indexed_by_a_small_category}{diagram}
        \begin{center}
            \begin{tikzcd}
            & X \ar[d] \\
            Y \ar[r] & Z
            \end{tikzcd}
        \end{center}
        in $\calC$. 

        The commutative diagram 
        \begin{center}
        \begin{tikzcd}
        X \times_Z Y \ar[r] \ar[d] & X \ar[d] \\
        Y \ar[r] & Z
        \end{tikzcd} 
        \end{center}
        may be referred to as a \hldef{cartesian square}.

\end{definition}
\begin{definition}[Category] \label{definition:category}
    A 
    \defin{category}{category}{
        name={Category},
        description={A nice enough collection of objects and morphisms (\Cref{definition:category})},
    }
    \hldef{category} $\mathcal{C}$ consists of the following data:
    \begin{itemize}
        \item A class of \defin{objects}{object_of_a_category}{
            name={Object of a category},
            description={\Cref{definition:category}},
        }
        denoted \notat{\operatorname{Ob}(\mathcal{C})}{class_of_objects_of_a_category}{
            name={$\operatorname{Ob}(\mathcal{C})$},
            description={Class of objects of a category $\calC$ \Cref{definition:category}},
            sort={Ob},
        }.
        % \hl{$\operatorname{Ob}(\mathcal{C})$}.
        \item For each pair of objects $X, Y \in \operatorname{Ob}(\mathcal{C})$, a class
        \notatin{\operatorname{Hom}_{\mathcal{C}}(X,Y)}{class_of_morphisms_between_two_objects_of_a_category}
        {
            name={$\operatorname{Hom}_{\mathcal{C}}(X,Y)$},
            description={Class of morphisms between objects $X$ and $Y$ of the category $\calC$ (\Cref{definition:category})},
            sort={Hom},
        }
        % $$\hlin{\operatorname{Hom}_{\mathcal{C}}(X,Y)}$$
        of \defin{morphisms}{morphism_between_objects_of_a_category}{
            name={Morphism between objects of a category},
            description={(\Cref{definition:category})},
        }
        (also called 
        \defin{arrows}{arrow_between_objects_of_a_category}{
            name={Arrow between objects of a category},
            description={Synonym for morphism (\Cref{definition:category})},
        }
        or
        \defin{homs}{hom_between_objects_of_a_category}{
            name={Hom between objects of a category},
            description={Synonym for morphism (\Cref{definition:category})},
        }). If the category $\calC$ is clear, then this \hldef{hom-class} is also denoted by \hl{$\operatorname{Hom}(X,Y)$}. It may also be denoted by \hl{$\operatorname{hom}_{\mathcal{C}}(X,Y)$} or \hl{$\operatorname{hom}(X,Y)$}, especially to distinguish from other types of hom's (e.g. \hyperrefIfExists{definition:internal_hom_object_in_a_category}{internal hom's})
        \item For each triple of objects $X,Y,Z$, a composition law
        $$ \circ : \operatorname{Hom}_{\mathcal{C}}(Y,Z) \times \operatorname{Hom}_{\mathcal{C}}(X,Y) \to \operatorname{Hom}_{\mathcal{C}}(X,Z), $$
        denoted \hl{$(g,f) \mapsto g \circ f$}.
        \item For each object $X$, an \hldef{identity morphism}
        $$\hlin{\operatorname{id}_X \in \operatorname{Hom}_{\mathcal{C}}(X,X).}$$
    \end{itemize}
    These data satisfy the following axioms:
    \begin{itemize}
        \item (Associativity) For all morphisms $f \in \operatorname{Hom}_{\mathcal{C}}(X,Y)$, $g \in \operatorname{Hom}_{\mathcal{C}}(Y,Z)$, and $h \in \operatorname{Hom}_{\mathcal{C}}(Z,W)$, 
        $$
        h \circ (g \circ f) = (h \circ g) \circ f.
        $$
        \item (Identity) For all $f \in \operatorname{Hom}_{\mathcal{C}}(X,Y)$,
        $$
        \operatorname{id}_Y \circ f = f = f \circ \operatorname{id}_X.
        $$
    \end{itemize}
    One often writes \hl{$X \in \calC$} synonymously with $X \in \Ob(\calC)$, i.e. to denote that $X$ is an object of of $\calC$. 

    We may call a category as above an \hldef{ordinary category} to distinguish this notion from the notions of \hyperrefIfExists{definition:category_enriched_in_a_monoidal_category}{\emph{categories enriched in monoidal categories}} or higher/$n$-categories.
    \TODO{TODO: define $n$-categories}

    A category as defined above may be called called a \hldef{large category} or a \hldef{class category} to emphasize that the hom-classes may be proper classes rather than sets (note, however, that the possibility that hom-classes are sets is not excluded for large categories). Accordingly, a \hldef{category} may often refer to a \hyperrefIfExists{definition:locally_small_category}{locally small category}\CrefIfExists{definition:locally_small_category}, which is a category whose hom-classes are all sets.
\end{definition}

% Later on, we refer to the \gls{category} again.

\begin{definition}[Locally small category] \label{definition:locally_small_category}
A \hyperrefIfExists{definition:category}{(large) category}\CrefIfExists{definition:category} $\mathcal{C}$ is called a \hldef{locally small category} if for every pair of objects $X, Y \in \operatorname{Ob}(\mathcal{C})$, the collection $\operatorname{Hom}_{\mathcal{C}}(X,Y)$ of morphisms between them is a (\CrefAndHyperrefIfExist{definition:small_set}{small}) \emph{set} (as opposed to a proper class). In other words, each hom-class is a set and may even be called a \hldef{hom-set}.

In some contexts, a locally small category may simply be called a \hldef{category}, especially when genuinely large categories are not considered.

A category $\mathcal{C}$ is called a \hldef{small category} if it is a locally small category and the class $\operatorname{Ob}(\mathcal{C})$ of objects is a set.

\TextIfExists{definition:grothendieck_universe}{
Given a \hyperrefIfExists{definition:grothendieck_universe}{universe}\CrefIfExists{definition:grothendieck_universe} $U$, we can define the notion of a \hldef{$U$-locally small category} and of a \hldef{$U$-small category} similarly. More explicitly, 
\begin{enumerate}
    \item a $U$-locally small category is a category such that for every pair of objects $X, Y \in \operatorname{Ob}(\mathcal{C})$, the collection $\operatorname{Hom}_{\mathcal{C}}(X,Y)$ of morphisms between them is a $U$-set.
    \item a $U$-small category is a category such that $\operatorname{Ob}(\mathcal{C})$ is a $U$-set and for every pair of objects $X, Y \in \operatorname{Ob}(\mathcal{C})$, the collection $\operatorname{Hom}_{\mathcal{C}}(X,Y)$ of morphisms between them is a $U$-set; in particular the collection of all objects and morhpisms in a $U$-small category is a $U$-set.
\end{enumerate}
}
\end{definition}

\begin{remark}
    Many ``concrete'' categories considered in ``classical mathematics'' or outside of more ``abstract'' category theory tend to be locally small. For example, the categories of sets, groups, $R$-modules, vector spaces, topological spaces, schemes, manifolds, sheaves on ``small enough'' sites are all locally small.
\end{remark}
\begin{definition} \label{definition:algebra_over_a_sheaf_of_rings_on_a_site}
Let $(\mathcal{C}, J)$ be a \CrefAndHyperrefIfExist{definition:grothendieck_topology_on_a_category_site_covering_sieve_topologically_generating_family}{site}. Let $\mathcal{O}$ be a \CrefAndHyperrefIfExist{definition:sheaf_on_a_site}{sheaf of commutative rings on $(\mathcal{C}, J)$}, i.e., $((\mathcal{C}, J), \mathcal{O})$ is a \CrefAndHyperrefIfExist{definition:ringed_site}{ringed site}.  

\begin{enumerate}
    \item An \hldef{$\mathcal{O}$-algebra} consists of the following data:
    \begin{itemize}
        \item A sheaf $\mathcal{A}$ of (not necessarily commutative) rings on $(\mathcal{C}, J)$,
        \item A morphism of sheaves of rings $\eta: \mathcal{O} \to \mathcal{A}$ such that for every object $U \in \mathcal{C}$, the image of $\eta_U: \mathcal{O}(U) \to \mathcal{A}(U)$ is contained in the center of $\mathcal{A}(U)$.
    \end{itemize}
    This makes $\mathcal{A}(U)$ an $\mathcal{O}(U)$-algebra for every $U \in \mathcal{C}$, such that for every morphism $f: V \to U$ in $\mathcal{C}$, the restriction map 
    $$\rho_{U,V}: \mathcal{A}(U) \to \mathcal{A}(V)$$ 
    is a homomorphism of $\mathcal{O}(U)$-algebras (where the $\mathcal{O}(U)$-algebra structure on $\mathcal{A}(V)$ is induced via restriction $\mathcal{O}(U) \to \mathcal{O}(V)$).

    \item Let $\mathcal{A}$ and $\mathcal{B}$ be \CrefAndHyperrefIfExist{definition:algebra_over_a_sheaf_of_rings_on_a_site}{$\mathcal{O}$-algebras}.

    A \hldef{morphism of $\mathcal{O}$-algebras} $\varphi: \mathcal{A} \to \mathcal{B}$ is a \CrefAndHyperrefIfExist{definition:sheaf_on_a_site}{morphism of sheaves} of rings such that, for every object $U \in \mathcal{C}$, the component map
    $$\varphi_U : \mathcal{A}(U) \to \mathcal{B}(U)$$
    is a homomorphism of $\mathcal{O}(U)$-algebras, i.e., it is a ring homomorphism that commutes with the structure maps $\eta_{\mathcal{A}}$ and $\eta_{\mathcal{B}}$:
    $$\varphi_U(\eta_{\mathcal{A}, U}(r)) = \eta_{\mathcal{B}, U}(r) \quad \text{for all } r \in \mathcal{O}(U).$$
\end{enumerate}

The collection of all $\mathcal{O}$-algebras together with their morphisms forms the \hldef{category of $\mathcal{O}$-algebras}, denoted by notations such as \hl{$\mathbf{Alg}(\mathcal{O})$}.
\end{definition}
\begin{definition} \label{definition:free_and_locally_free_modules_on_a_ringed_site}
Let $((\mathcal{C}, J), \mathcal{O})$ be a \CrefAndHyperrefIfExist{definition:ringed_site}{ringed site}, where $\mathcal{O}$ is a \CrefAndHyperrefIfExist{definition:sheaf_on_a_site}{sheaf} of rings on the site $(\mathcal{C}, J)$.
\TODO{This needs some genuine fixing to be definable on a general site, not just one given by a pretopology}
\begin{enumerate}
    \item Let $I$ be an indexing set. The \hldef{free sheaf of $\mathcal{O}$-modules of rank $I$} (or simply a \hldef{free sheaf}), denoted by $\mathcal{O}^{\oplus I}$, is the sheaf associated to the presheaf $U \mapsto \mathcal{O}(U)^{\oplus I}$. If $I$ is finite with cardinality $n$, we usually write \hl{$\mathcal{O}^{\oplus n}$}.

    \item 
    
    Let $\calA$ be a (large) category, and let $A$ be an object of $\calA$. %a set (or more generally, an abelian group, ring, etc.).
    Assume that a \CrefAndHyperrefIfExist{definition:sheafification_functor_on_a_site}{sheafification functor} 
    \TODO{If such a sheafification functor exist, does a sheafification functor exist when restricted to an object $U$?}
    $$a: \PreShv(\calC, J, \calA) \to \Shv(\calC,J, \calA)$$
    \CrefIfExists{definition:sheaf_on_a_site}
    \CrefIfExists{definition:presheaf_on_a_category}
    exists\TextIfExists{theorem:sheafification_of_a_presheaf_of_sets_on_a_small_enough_site}{~(e.g. see \Cref{theorem:sheafification_of_a_presheaf_of_sets_on_a_small_enough_site})}. Let $U$ be an object of $\calC$.


    An \CrefAndHyperrefIfExist{definition:module_over_a_sheaf_of_rings_on_a_site}{$\mathcal{O}$-module} $\mathcal{F}$ is called \hldef{locally free of rank $n$ on $U$} (for an integer $n \ge 0$) if there exists a \CrefAndHyperrefIfExist{definition:grothendieck_topology_on_a_category_site_covering_sieve_topologically_generating_family}{covering sieve} $\{U_i \to U\}_{i \in I}$ such that for each $i$, the \CrefAndHyperrefIfExist{definition:restriction_of_a_sheaf_on_a_site_to_an_object_of_the_underlying_category_of_the_site}{restriction $\mathcal{F}|_{U_i}$} is isomorphic to the free sheaf $(\mathcal{O}|_{U_i})^{\oplus n}$ as an $\mathcal{O}|_{U_i}$-module.
\end{enumerate}
We might call a locally free $\calO$-module of rank $n$ an \hldef{algebraic vector bundle of rank $n$}. A \hldef{(algebraic) line bundle} or \hldef{invertible sheaf} or is then an algebraic vector bundle of rank $1$.
\end{definition}

\begin{definition} \label{definition:module_of_a_ring}
Let $R$ be a \CrefAndHyperrefIfExist{definition:ring}{not-necessarily commutative ring}. 
\begin{enumerate}
    \item A \hldef{left $R$-module} is an abelian group $(M,+)$ together with an operation $R \times M \to M$, denoted $(r,m) \mapsto rm$, such that for all $r,s \in R$ and $m,n \in M$:
    \begin{itemize}
        \item $r(m+n) = rm + rn$,
        \item $(r+s)m = rm + sm$,
        \item $(rs)m = r(sm)$,
        \item $1_R m = m$ where $1_R$ is the multiplicative identity of $R$.
    \end{itemize}

    \item A \hldef{right $R$-module} is defined similarly as an abelian group $(M,+)$ with an operation $M \times R \to M$, denoted $(m,r) \mapsto mr$, such that for all $r,s \in R$ and $m,n \in M$:
    \begin{itemize}
        \item $(m+n)r = mr + nr$,
        \item $m(r+s) = mr + ms$,
        \item $m(rs) = (mr)s$,
        \item $m 1_R = m$.
    \end{itemize}

    \item Let $R$ and $S$ be  (not necessarily commutative) \CrefAndHyperrefIfExist{definition:ring}{rings}.

    An \hldef{$R$-$S$-bimodule} (or an \hldef{$R$-$S$-module} or an $(R,S)$-module, etc.)is an \CrefAndHyperrefIfExist{definition:group}{abelian group} $(M,+)$ equipped with
    \begin{enumerate}
        \item a left action of $R$:
        $$\hlin{R \times M \to M, \quad (r,m) \mapsto r \cdot m},$$
        making $M$ a \CrefAndHyperrefIfExist{definition:module_of_a_ring}{left $R$-module},
        \item a right action of $S$:
        $$\hlin{M \times S \to M, \quad (m,s) \mapsto m \cdot s},$$
        making $M$ a right $S$-module,
    \end{enumerate}
    such that the left and right actions commute; that is, for all $r \in R$, $s \in S$, and $m \in M$,
    $$ r \cdot (m \cdot s) = (r \cdot m) \cdot s.  $$

    \item A \hldef{two-sided $R$-module} (or \hldef{$R$-bimodule}) is an $R$-$R$-bimodule.
    
    % an abelian group $(M,+)$ which is simultaneously a left $R$-module and a right $R$-module, such that $(rm)s = r(ms)$ for all $r,s \in R$, $m \in M$. Equivalently, a two-sided $R$-module is an \hldef{$R$-$R$-bimodule}\CrefIfExists{definition:module_of_a_ring}


\end{enumerate}
If $R$ is a \CrefAndHyperrefIfExist{definition:commutative_ring}{commutative ring}, then a left/right $R$-module can automatically be regarded as a two-sided $R$-module. As such, we simply talk about \hldef{$R$-modules} in this case. 

Any abelian group is equivalent to a two-sided $\bbZ$-module. Moreover, any left $R$-module is equivalent to an \CrefAndHyperrefIfExist{definition:module_of_a_ring}{$R-\bbZ$-bimodule} and any right $R$-module is equivalent to an \CrefAndHyperrefIfExist{definition:module_of_a_ring}{$\bbZ-R$-bimodule}. Given a left/right/two-sided $R$-module, its \hldef{natural bimodule structure} will refer to its structure as a $R$-$\bbZ$/$\bbZ$-$R$/$R$-$R$ bimodule. In this way, many definitions associated with the notions of left/right/two-sided $R$-modules can be defined as special cases for definitions for $R$-$S$-bimodules.
\end{definition}

% \begin{definition}[Grothendieck topology] \label{definition:grothendieck_topology_on_a_category_site_covering_sieve_topologically_generating_family}
%     Let $\mathscr{U}$ be a \hyperrefIfExists{definition:grothendieck_universe}{universe}\CrefIfExists{definition:grothendieck_universe} and let $\calC$ be a \hyperrefIfExists{definition:locally_small_category}{locally small category}\CrefIfExists{definition:locally_small_category}.

%     \begin{enumerate}
%         \item \textbf{(Grothendieck Topology via Sieves)}
%         A \hldef{Grothendieck topology} $J$ on $\calC$ is an assignment to each object $U \in \calC$ of a collection $J(U)$ of \CrefAndHyperrefIfExist{definition:sieve_on_an_object_in_a_category}{sieves} on $U$, called \hldef{covering sieves}, satisfying:
%         \begin{enumerate}
%             \item (Maximality) The maximal \CrefAndHyperrefIfExist{definition:sieve_on_an_object_in_a_category}{sieve} $\{ f : V \to U \mid V \in \calC \}$ is in $J(U)$.
%             \item (Stability) If $S \in J(U)$ and $f : V \to U$ is any morphism, then the \CrefAndHyperrefIfExist{definition:pullback_sieve_of_an_object_in_a_category_via_a_morphism_to_the_object}{pullback sieve} $f^{*}S$ is in $J(V)$.
%             \item (Transitivity/Local Character) If $S$ is a sieve on $U$ and there exists a covering sieve $R \in J(U)$ such that for every morphism $f : V \to U$ in $R$, the pullback sieve $f^{*}S$ is in $J(V)$, then $S \in J(U)$.
%         \end{enumerate}

%         % \item \textbf{(Grothendieck Pretopology / Basis)}
%         % If $\calC$ admits fiber products, one can define a topology via \hldef{covering families}. A \hldef{Grothendieck pretopology} (or basis) is a collection $K(U)$ of families $\{U_i \to U\}_{i \in I}$ for each object $U$, satisfying:
%         % \begin{itemize}
%         %     \item (Isomorphism) $\{U' \xrightarrow{\sim} U\} \in K(U)$ for any isomorphism.
%         %     \item (Stability) If $\{U_i \to U\} \in K(U)$ and $V \to U$ is a morphism, then $\{U_i \times_U V \to V\} \in K(V)$.
%         %     \item (Composition) If $\{U_i \to U\} \in K(U)$ and for each $i$, $\{V_{ij} \to U_i\} \in K(U_i)$, then the composite family $\{V_{ij} \to U\} \in K(U)$.
%         % \end{itemize}
%         % Every pretopology generates a unique Grothendieck topology $J$, where $S \in J(U)$ iff $S$ contains a covering family from the pretopology.

%         \item A \hldef{site} is a pair $(\calC, J)$ consisting of a category $\calC$ and a Grothendieck topology $J$.

%         \item A family of objects $\mathcal{G} = \{G_\alpha\}$ in a site $(\calC, J)$ is called a \hldef{topologically generating family} if for every object $X \in \calC$, there exists a covering sieve $S \in J(X)$ \CrefAndHyperrefIfExist{definition:sieve_on_an_object_of_a_category_generated_by_a_family_of_morphisms}{generated by} morphisms with domains in $\mathcal{G}$. Equivalently, every object $X$ admits a cover $\{U_i \to X\}$ where each $U_i \in \mathcal{G}$.

%         \item A \hldef{$\mathscr{U}$-site} is a site whose underlying category is $\mathscr{U}$-locally small and which admits a $\mathscr{U}$-small topologically generating family.
%     \end{enumerate}
% \end{definition}

\begin{definition}[Grothendieck topology] \label{definition:grothendieck_topology_on_a_category_site_covering_sieve_topologically_generating_family}
    Let $\mathscr{U}$ be a \hyperrefIfExists{definition:grothendieck_universe}{universe}\CrefIfExists{definition:grothendieck_universe}.
    \begin{enumerate}
        % \item Let $C$ be a \hyperrefIfExists{definition:locally_small_category}{locally small category}\CrefIfExists{definition:locally_small_category}. A \hldef{Grothendieck topology on $C$} assigns to each object $U$ of $C$ a collection of families of morphisms $\{U_i \to U\}_{i \in I}$, called \hldef{coverings of $U$}, satisfying:
        % \begin{itemize}
        %     \item (Isomorphism) If $f: V \to U$ is an isomorphism in $C$, then $\{f: V \to U\}$ is a covering of $U$.
        %     \item (Stability under base change) If $\{U_i \to U\}_{i \in I}$ is a covering of $U$ and $V \to U$ is any morphism, then the family $\{ U_i \times_U V \to V \}_{i \in I}$ is a covering of $V$.
        %     \item (Transitivity) If $\{U_i \to U\}_{i \in I}$ is a covering of $U$ and for each $i$, $\{V_{ij} \to U_i\}_{j \in J_i}$ is a covering of $U_i$, then the family $\{ V_{ij} \to U \}_{i \in I,\, j \in J_i}$ is a covering of $U$.
        % \end{itemize}

        \item (See \cite[Expos\'e II, D\'efinition 1.1]{SGA4_I}) Let $\calC$ be a \CrefAndHyperrefIfExist{definition:category}{category}. A \hldef{Grothendieck topology on $\calC$} assigns to each object $U$ of $\calC$ a collection \hl{$J(U)$} of \CrefAndHyperrefIfExist{definition:sieve_on_an_object_in_a_category}{sieves} $\{U_i \to U\}_{i \in I}$, each called a \hldef{covering sieve of $U$}, satisfying:
        \begin{enumerate}
            \item (Stability under ``base change''): If $S \in J(U)$ is a covering sieve of an object $U$, and $f: V \to U$ is any morphism in $\calC$, then the \CrefAndHyperrefIfExist{definition:pullback_sieve_of_an_object_in_a_category_via_a_morphism_to_the_object}{pullback sieve} $f^* S$ is a covering sieve of $U$.
            % \item (Local character condition) If $F$ is a sieve on $U$ such that the sieve $\bigcup_...$ \TODO{}
            \item (Local character condition) If $S$ is a sieve on $U$, and if there exists a covering sieve $R \in J(U)$ such that for all $f: V \to U$ in $R$ the \CrefAndHyperrefIfExist{definition:pullback_sieve_of_an_object_in_a_category_via_a_morphism_to_the_object}{pullback sieve} $f^* S$ is in $J(V)$, then $S \in J(U)$. 
            
            \item The \CrefAndHyperrefIfExist{definition:maximal_sieve_on_an_object_in_a_category}{maximal sieve} is a covering sieve.
        \end{enumerate}


        % Equivalently, a Grothendieck topology $J$ on a category $C$ is an assignment of a collection $J(U)$ of \CrefAndHyperrefIfExist{definition:sieve_on_an_object_in_a_category}{sieves} on each object $U \in \operatorname{Ob}(C)$ such that:
        % \begin{enumerate}
        %     \item the maximal \CrefAndHyperrefIfExist{definition:sieve_on_an_object_in_a_category}{sieve} $\{ f : V \to U \mid f \in \operatorname{Mor}(C) \}$ belongs to $J(U)$,
        %     \item if $S \in J(U)$ and $f : V \to U$, then the \CrefAndHyperrefIfExist{definition:pullback_sieve_of_an_object_in_a_category_via_a_morphism_to_the_object}{pullback sieve $f^{*}S$} on $V$ belongs to $J(V)$,
        %     \item if $S$ is a sieve on $U$, and if there exists $R \in J(U)$ such that for all $f : V \to U$ in $R$ the \CrefAndHyperrefIfExist{definition:pullback_sieve_of_an_object_in_a_category_via_a_morphism_to_the_object}{pullback sieve $f^{*}S$} is in $J(V)$, then $S \in J(U)$.
        % \end{enumerate}

        Some will refer to a Grothendieck topology as simply a \hldef{topology}, not to be confused with the related, but less general, notion of a \CrefAndHyperrefIfExist{definition:topological_space}{topology on a set}.


        \item (See \cite[Expos\'e II, 1.1.5]{SGA4_I}) A \hldef{site} is a category $\calC$ equipped with a Grothendieck topology.

        When we are working with a \CrefAndHyperref{definition:basis_and_grothendieck_pretopology_for_a_grothendieck_topology_on_a_category}{Grothendieck pretopology} $K$ on a category $\calC$, we may regard $\calC$ as a site by equipping it with the \CrefAndHyperref{definition:grothendieck_topology_generated_by_a_pretopology}{Grothendieck topology generated by} $K$. 

        \item (See \cite[Expos\'e II, D\'efinition 1.2]{SGA4_I}) Let $(\calC, J)$ be a site. A family of morphisms $(U_i \to U)_{i \in I}$ is called a \hldef{covering family of $U$ (with respect to the site/topology)} or a \hldef{cover of $U$ (with respect to the site/topology)} if the \CrefAndHyperrefIfExist{definition:sieve_on_an_object_of_a_category_generated_by_a_family_of_morphisms}{sieve generated by} the family is a covering sieve of $U$. 

        \item (See \cite[Expos\'e II, D\'efinition 3.0.1]{SGA4_I}) Let $(\calC, J)$ be a \CrefAndHyperrefIfExist{definition:grothendieck_topology_on_a_category_site_covering_sieve_topologically_generating_family}{site}, where $J$ is a Grothendieck topology on $\calC$.

        A family $G$ of objects $\calC$ is called a \hldef{topologically generating family of the site/topology} or a \hldef{generating family/collection of the site/topology} if for every object $X \in \calC$, there is a covering family $\{X_\alpha \to X\}_{\alpha \in A}$ of $X$ such that every $X_\alpha$ is a member of $G$.  

        Equivalently, the Grothendieck topology $J$ is the smallest Grothendieck topology containing all covers of the $U_i$. Also equivalently, for any $S \in J(X)$, the sieve $S$ contains a covering family $\{V_i \to X\}$ such that each morphism $V_i \to X$ factors through some member of $G$. \TODO{Verify that these claimed equivalences are indeed equivalences}
        
        % A family of objects $\{U_i\}_{i \in I}$ in $\calC$ is called a \hldef{topologically generating family} if for every object $X \in \calC$ and every covering sieve $S \in J(X)$, the sieve $S$ is \CrefAndHyperrefIfExist{definition:sieve_on_an_object_of_a_category_generated_by_a_family_of_morphisms}{generated by} pullbacks of covering families from the family $\{U_i\}$.

        % More precisely, this means that for any $S \in J(X)$, the sieve $S$ contains a covering family $\{V_j \to X\}$ such that each morphism $V_j \to X$ factors through some $U_i$, and the covering families of the $U_i$ generate the topology $J$. 
        % Equivalently, the Grothendieck topology $J$ is the smallest Grothendieck topology containing all coverings of the $U_i$.

        % When one speaks of a \hldef{generating family/collection} of a site, one usually refers to the above notion of a topologically generating family.

        \item (See \cite[Expos\'e II, D\'efinition 3.0.2]{SGA4_I}) A \hldef {$\mathscr{U}$-site} is a site whose underlying category $\calC$ is \hyperrefIfExists{definition:locally_small_category}{$\mathscr{U}$-locally small}\CrefIfExists{definition:locally_small_category} and which has a $\mathscr{U}$-small topologically generating family. A $\mathscr{U}$-site is called \hldef{$\mathscr{U}$-small} if its underlying category is $\mathscr{U}$-small. Similarly, a \hldef{small site} is a site whose underlying category is a set and a \hldef{locally small site} is a site whose underlying category is \CrefAndHyperrefIfExist{definition:locally_small_category}{locally small}.
    \end{enumerate}
\end{definition}

\begin{definition}[Sheaf on a site] \label{definition:sheaf_on_a_site}

% \TODO{There might be some need to say that $\calA$ is a category for which sheaves on the site ``can be defined''}
% \TODO{go through statements using the notion of sheaves and make sure that the value categories have small products and that the categories have small generating families.}

Let $(\calC, J)$ be a \CrefAndHyperrefIfExist{definition:grothendieck_topology_on_a_category_site_covering_sieve_topologically_generating_family}{site}. Let $\calA$ be a \CrefAndHyperrefIfExist{definition:category}{(large) category}.
\begin{enumerate}
    \item A \CrefAndHyperrefIfExist{definition:presheaf_on_a_category}{presheaf} $\calF: \calC^{\op} \to \calA$\CrefIfExists{definition:opposite_category_of_a_category} is called a \hldef{sheaf on the site $(\calC, J)$ valued in $\calA$} if, for every object $U$ of $\calC$ and every \CrefAndHyperrefIfExist{definition:grothendieck_topology_on_a_category_site_covering_sieve_topologically_generating_family}{covering sieve} $S \in J(U)$, the \CrefAndHyperrefIfExist{definition:limit_and_colimit_of_a_diagram_in_a_category}{limit}
    $$\varprojlim_{(V \to U) \in (\calD_S)^{\op}} \calF|_{\calD_S}(V),$$
    exists and the canonical natural morphism
    $$\calF(U) \to \varprojlim_{(V \to U) \in (\calD_S)^{\op}} \calF|_{\calD_S}(V)$$
    is an isomorphism. Here, $\calD_S \hookrightarrow \calC/U$\CrefIfExists{definition:category_of_objects_over_under_a_fixed_object_in_a_category} is the full \CrefAndHyperrefIfExist{definition:downward_upward_closed_subcategory_of_a_category}{downward-closed subcategory} such that $\operatorname{Ob}(\calD_S) = \{(f: V \to U): f \in S(V)\}$,

    In particular, when we are working with a \CrefAndHyperref{definition:basis_and_grothendieck_pretopology_for_a_grothendieck_topology_on_a_category}{Grothendieck pretopology} $K$ on a category $\calC$, we may speak of sheaves on the site whose Grothendieck topology is the \CrefAndHyperref{definition:grothendieck_topology_generated_by_a_pretopology}{one generated by} $K$.

    \item Given sheaves $\calF, \calG: \calC^{\op} \to \calA$ on the site $(\calC, J)$, a \hldef{morphism between the sheaves} is a \CrefAndHyperrefIfExist{definition:presheaf_on_a_category}{morphism} between $\calF$ and $\calG$ as presheaves.


    \item Let $U$ be a \hyperrefIfExists{definition:grothendieck_universe}{universe}\CrefIfExists{definition:grothendieck_universe}. A \hldef{$U$-sheaf} typically refers to a $U$-presheaf that is a sheaf for a $U$-site. In other words, a $U$-sheaf is a sheaf on a site whose underlying category is \hyperrefIfExists{definition:locally_small_category}{$U$-locally small}\CrefIfExists{definition:locally_small_category} and which has a $U$-small topologically generating family such that the sheaf is valued in $U$-sets.

    \item The \hldef{sheaf category/category of $\calA$-valued sheaves on $\calC$} is the (large) category defined as the full subcategory of $\PreShv(\calC, \calA)$ whose objects are the sheaves on $\calC$ with values in $\calA$. Common notations for the sheaf category include \hl{$\Shv(\calC, \calA)$}, \hl{$\Shv(\calC, J, \calA)$}, \hl{$\Sh(\calC, \calA)$}, \hl{$\Sh(\calC, J, \calA)$}. If the value category $\calA$ is clear from context, then notations such as \hl{$\Shv(\calC)$}, \hl{$\Shv(\calC, J)$}, \hl{$\Sh(\calC)$}, \hl{$\Sh(\calC, J)$} are also common.

\end{enumerate}

% Let $(\calC, J)$ be a \CrefAndHyperrefIfExist{definition:grothendieck_topology_on_a_category_site_covering_sieve_topologically_generating_family}{site} with a small \CrefAndHyperrefIfExist{definition:grothendieck_topology_on_a_category_site_covering_sieve_topologically_generating_family}{topological generating family} (or a $U$-small topologically generating family if a \CrefAndHyperrefIfExist{definition:grothendieck_universe}{universe} $U$ is available) and let $\mathcal{A}$ be a \CrefAndHyperrefIfExist{definition:category}{(large) category} that has all \CrefAndHyperrefIfExist{definition:locally_small_category}{small} \CrefAndHyperrefIfExist{definition:product_and_coproduct_of_objects_in_a_category}{products} (Some common examples of categories that have small products and thus play the role of $\calA$ here include $\mathcal{A} = \text{Set}$, $\text{Ab}$, $R\mathbf{-mod}$ for a fixed ring $R$, $\text{rings}$). 
% \begin{enumerate}

%     \item For any object $U$ of $\calC$ and every covering $\{U_i \to U\}_{i \in I}$ in $J$, note that there are morphisms $U_i \times_U U_j \to U_i$ for every $i,j \in I$. 
%     % Consider the subcategory of $C$ consisting of the objects $U_i$ and $U_i \times_U U_j$, together with these morphisms.
%     Given any presheaf $\calF: C^{\op} \to \calA$, there is a \CrefAndHyperrefIfExist{definition:diagram_in_a_category_indexed_by_a_small_category}{diagram} in $\calA$ consisting of objects $\calF(U_i)$ and $\calF(U_i \times_U U_j)$ and morphisms $\calF(U_i) \to \calF(U_i \times_U U_j)$. The presheaf $\calF$ is called a \hldef{sheaf on the site $(\calC, J)$ valued in $\calA$} if, for every object $U$ of $\calC$ and every covering $\{U_i \to U\}_{i \in I}$ in $J$, the sections object $\calF(U)$ is the \CrefAndHyperrefIfExist{definition:limit_and_colimit_of_a_diagram_in_a_category}{limit} of the aforementioned diagram:
    
%     % A \hyperrefIfExists{definition:presheaf_on_a_category}{presheaf}\CrefIfExists{definition:presheaf_on_a_category} $\mathcal{F}: C^{\mathrm{op}} \to \mathcal{A}$ is a \hldef{sheaf on the site $(\calC,J)$ valued in $\calA$} if, for every object $U$ of $\calC$ and every covering $\{U_i \to U\}_{i \in I}$ in $J$, the sections object $\calF(U)$ is the \CrefAndHyperrefIfExist{definition:limit_and_colimit_of_a_diagram_in_a_category}{limit} of the sections objects $\calF(U_i)$:
%     % $$\calF(U) \cong \varprojlim_{}$$
    
%     % following sequence is an \CrefAndHyperrefIfExist{definition:equalizer_and_coequalizer_of_morphisms_in_a_category}{equalizer} in $\mathcal{A}$:
%     % \[
%     % \mathcal{F}(U) \to \prod_{i} \mathcal{F}(U_i) \rightrightarrows \prod_{i, j} \mathcal{F}(U_i \times_U U_j)
%     % \]
%     % where the first map sends $s$ to $(\mathcal{F}(U_i \to U)(s))_i$ and the arrows to $(\mathcal{F}(U_i \times_U U_j \to U_i)(s_i))_{i,j}$ and $(\mathcal{F}(U_i \times_U U_j \to U_j)(s_j))_{i,j}$, respectively.

%     % \item A \hyperrefIfExists{definition:presheaf_on_a_category}{presheaf}\CrefIfExists{definition:presheaf_on_a_category} $\mathcal{F}: C^{\mathrm{op}} \to \mathcal{A}$ is a \hldef{sheaf on the site $(\calC,J)$ valued in $\calA$} if, for every object $U$ of $\calC$ and every covering $\{U_i \to U\}_{i \in I}$ in $J$, the following sequence is an \CrefAndHyperrefIfExist{definition:equalizer_and_coequalizer_of_morphisms_in_a_category}{equalizer} in $\mathcal{A}$:
%     % \[
%     % \mathcal{F}(U) \to \prod_{i} \mathcal{F}(U_i) \rightrightarrows \prod_{i, j} \mathcal{F}(U_i \times_U U_j)
%     % \]
%     % where the first map sends $s$ to $(\mathcal{F}(U_i \to U)(s))_i$ and the arrows to $(\mathcal{F}(U_i \times_U U_j \to U_i)(s_i))_{i,j}$ and $(\mathcal{F}(U_i \times_U U_j \to U_j)(s_j))_{i,j}$, respectively.

%     \item A \hldef{morphism of sheaves} $\calF: \calC^{\op} \to \calA$ is a \hyperrefIfExists{definition:presheaf_on_a_category}{morphism as presheaves}\CrefIfExists{definition:presheaf_on_a_category}. 


%     \item Let $U$ be a \hyperrefIfExists{definition:grothendieck_universe}{universe}\CrefIfExists{definition:grothendieck_universe}. A \hldef{$U$-sheaf} typically refers to a $U$-presheaf that is a sheaf for a $U$-site. In other words, a $U$-sheaf is a sheaf on a site whose underlying category is \hyperrefIfExists{definition:locally_small_category}{$U$-locally small}\CrefIfExists{definition:locally_small_category} and which has a $U$-small topologically generating family such that the sheaf is valued in $U$-sets.

%     \item The \hldef{sheaf category/category of $\calA$-valued sheaves on $\calC$} is the (large) category defined as the full subcategory of $\PreShv(\calC, \calA)$ whose objects are the sheaves on $C$ with values in $\calA$. Common notations for the sheaf category include \hl{$\Shv(\calC, \calA)$}, \hl{$\Shv(\calC, J, \calA)$}, \hl{$\Sh(\calC, \calA)$}, \hl{$\Sh(\calC, J, \calA)$}. If the value category $\calA$ is clear from context, then notations such as \hl{$\Shv(\calC)$}, \hl{$\Shv(\calC, J)$}, \hl{$\Sh(\calC)$}, \hl{$\Sh(\calC, J)$} are also common.

% \end{enumerate}
\end{definition}

\begin{definition}[Finitely generated modules and bimodules]  \label{definition:finitely_generated_modules_over_rings}

    Let $R$ and $S$ be \CrefAndHyperrefIfExist{definition:ring}{(not necessarily commutative) rings}. 
    \begin{enumerate}
        \item An $R$-$S$-bimodule $M$ is \hldef{finitely generated} if it has a \CrefAndHyperrefIfExist{definition:span_a_module_over_a_ring_for_elements_of_the_module}{finite spanning set}. 
        
        % there exists a finite set $\{m_1,\ldots,m_n\} \subseteq M$ such that the \CrefAndHyperrefIfExist{definition:submodule_of_a_module_generated_by_elements}{submodule of $M$ generated by} this set is $M$ itself or equivalently every element $m \in M$ is a linear combination of $m_1,\ldots,m_n$.

        \item A left/right/two-sided $R$-module is \hldef{finitely generated} if has a \CrefAndHyperrefIfExist{definition:span_a_module_over_a_ring_for_elements_of_the_module}{finite spanning set}, or equivalently if its \CrefAndHyperrefIfExist{definition:module_of_a_ring}{natural bimodule structure} is finitely generated.
    \end{enumerate}
\end{definition}
\begin{definition} \label{definition:projective_bimodule_over_rings}
    Let $R$ and $S$ be \CrefAndHyperrefIfExist{definition:ring}{(not necessarily commutative) rings}.  A \hldef{projective $R$-$S$-bimodule} is an \CrefAndHyperrefIfExist{definition:module_of_a_ring}{$(R,S)$-bimodule} $P$ that satisfies any of the following equivalent conditions:
    \begin{enumerate}
        \item The functor
        \[
            \operatorname{Hom}_{{}_R\mathsf{Mod}_S}(P, -): {}_R\mathsf{Mod}_S \to \mathsf{Ab}
        \]
        is an \CrefAndHyperrefIfExist{definition:exact_functor_between_abelian_categories}{exact functor} between the abelian categories \CrefAndHyperrefIfExist{definition:category_of_modules_and_bimodules_over_rings}{${}_R\mathsf{Mod}_S$} and \CrefAndHyperrefIfExist{definition:category_of_groups_of_abelian_groups}{$\mathsf{Ab}$}.

        \item $P$ is a projective left module over the ring $R \otimes_{\mathbb{Z}} S^{\operatorname{op}}$\CrefIfExists{definition:tensor_product_of_a_ring_and_an_algebra_over_a_ring}\CrefIfExists{definition:opposite_ring_of_a_ring}.

        \item $P$ is a direct summand of a free $(R,S)$-bimodule. (A free $(R,S)$-bimodule is a direct sum of copies of the tensor product $R \otimes_{\mathbb{Z}} S$, equipped with the natural left $R$-action and right $S$-action).

        \item $P$ is a \CrefAndHyperrefIfExist{definition:injective_and_projective_objects_in_a_category}{projective object} in the category ${}_R\mathsf{Mod}_S$. That is, for every surjective homomorphism of $(R,S)$-bimodules $f: M \to N$ and every homomorphism $g: P \to N$, there exists a homomorphism $h: P \to M$ such that $f \circ h = g$.
    \end{enumerate}

    Being a projective bimodule is a strictly stronger condition than being projective as a left or right module. 
    \begin{itemize}
        \item A bimodule ${}_R P_S$ may be projective as a left $R$-module (i.e., projective in ${}_R\mathsf{Mod}$) without being a projective bimodule.
        \item Similarly, it may be projective as a right $S$-module (i.e., projective in $\mathsf{Mod}_S$) without being a projective bimodule.
        \item A bimodule that is projective on both sides is sometimes called \hldef{biprojective}, but this does not imply it is a projective object in ${}_R\mathsf{Mod}_S$. For example, if $R=S=\mathbb{Z}$, the bimodule $\mathbb{Z}$ is free (hence projective) on both sides, but it is \textit{not} a projective $(\mathbb{Z}, \mathbb{Z})$-bimodule because $\mathbb{Z}$ is not a projective $\mathbb{Z}[\mathbb{Z}]$-module (the augmentation ideal is not projective).
    \end{itemize}
\end{definition}

\begin{definition} \label{definition:ring_homomorphism}
Let $(R,+,\cdot)$ and $(S,+,\cdot)$ be \CrefAndHyperrefIfExist{definition:ring}{rings}, not assumed to be commutative. A function $f: R \to S$ is called a \hldef{ring homomorphism} if for all $r_1,r_2 \in R$ the following properties hold:
\begin{enumerate}
    \item $f(r_1 + r_2) = f(r_1) + f(r_2)$,
    \item $f(r_1 r_2) = f(r_1) f(r_2)$,
    \item $f(1_R) = 1_S$ where $1_R$ and $1_S$ denote the multiplicative identities in $R$ and $S$, respectively.
\end{enumerate}
A ring homomorphism is said to be a \hldef{ring isomorphism} if it is invertible as a map of sets.

An \hldef{$R$-ring} refers to a ring $S$ equipped with a ring homomorphism $f: R \to S$. 

We note that a ring homomorphism $f: R \to S$ yields a natural \CrefAndHyperrefIfExist{definition:module_of_a_ring}{left $R$-module} structure on $S$ and a natural right $R$-module structure on $S$ respectively as follows for $r \in R$ and $s \in S$:
$$r \cdot s = f(r) \cdot s$$
$$s \cdot r = s \cdot f(r).$$
However, these left and right module structures need not yield a two-sided $R$--module structure.
% A ring $S$ equipped with a ring homomorphism $f: R \to S$ is called an \hldef{$R$-algebra}.
\end{definition}

\begin{definition} \label{definition:homomorphism_of_modules_over_a_ring}
Let $R,S$ be \CrefAndHyperrefIfExist{definition:ring}{(not-necessarily commutative) rings}. 
\begin{enumerate}
    \item Let $M$ and $N$ be \CrefAndHyperrefIfExist{definition:module_of_a_ring}{$R$-$S$-bimodules}. A function $\varphi: M \to N$ is called an \hldef{$R$-$S$-bimodule homomorphism} or \hldef{$R$-$S$-linear} if it is a \CrefAndHyperrefIfExist{definition:group_homomorphism}{group homomorphism} of the underlying abelian groups of $M$ and $N$ and respects the scalar actions as follows: 
    for all $m_1,m_2 \in M$, $r \in R$, and $s \in S$,
        \begin{align*}
        % \varphi(m_1 + m_2) &= \varphi(m_1) + \varphi(m_2), \\
        \varphi(r \cdot m_1) &= r \cdot \varphi(m_1), \\
        \varphi(m_1 \cdot s) &= \varphi(m_1) \cdot s.
        \end{align*}

    \item Let $M$ and $N$ be \CrefAndHyperrefIfExist{definition:module_of_a_ring}{left/right/two-sided $R$-modules}. A function $\varphi: M \to N$ is called a \hldef{left/right/two-sided $R$-module homomorphism} if it is an bimodule homomorphism on the \CrefAndHyperrefIfExist{definition:module_of_a_ring}{natural bimodule structures} of $M$ and $N$.
    %  $R$-$\bbZ$/$\bbZ$-$R$/$R$-$R$-bimodule homomorphism. 
     Such a function is also called \hldef{$R$-linear}.

\end{enumerate}

Modules and homomorphisms of a fixed type (i.e. $R$-$S$-bimodules or left/righ/two-sided $R$-modules) form a \CrefAndHyperrefIfExist{definition:locally_small_category}{locally small} \CrefAndHyperrefIfExist{definition:category}{category}.

% Let $M$ and $N$ be \CrefAndHyperrefIfExist{definition:module_of_a_ring}{left/right/two-sided $R$-modules or $R$-$S$-bidmodules}. 

% \begin{enumerate}
%     \item A function $\varphi : M \to N$ is called a \hldef{left/right/two-sided module homomorphism} or \hldef{$R$-linear} if it is additive (more precisely, a \CrefAndHyperrefIfExist{definition:group_homomorphism}{group homomorphism} of \CrefAndHyperrefIfExist{definition:group}{abelian groups}) and respects the scalar action(s) as follows: for all $m_1,m_2 \in M$, $r \in R$, and $s \in S$,
%     \begin{align*}
%     % \varphi(m_1 + m_2) &= \varphi(m_1) + \varphi(m_2), \\
%     \varphi(r \cdot m_1) &= r \cdot \varphi(m_1), \\
%     \varphi(m_1 \cdot s) &= \varphi(m_1) \cdot s.
%     \end{align*}

%     \item 
% \end{enumerate}

% \begin{enumerate}
%     \item If $M$ and $N$ are left $R$-modules, then for all $m_1,m_2 \in M$ and $r \in R$,
%     \begin{align*}
%     \varphi(m_1 + m_2) &= \varphi(m_1) + \varphi(m_2), \\
%     \varphi(r \cdot m_1) &= r \cdot \varphi(m_1).
%     \end{align*}

%     \item If $M$ and $N$ are right $R$-modules, then for all $m_1,m_2 \in M$ and $r \in R$,
%     \begin{align*}
%     \varphi(m_1 + m_2) &= \varphi(m_1) + \varphi(m_2), \\
%     \varphi(m_1 \cdot r) &= \varphi(m_1) \cdot r.
%     \end{align*}

%     \item If $M$ and $N$ are two-sided $R$-modules, then for all $m_1,m_2 \in M$, and $r_1,r_2 \in R$,
%     \begin{align*}
%     \varphi(m_1 + m_2) &= \varphi(m_1) + \varphi(m_2), \\
%     \varphi(r_1 \cdot m_1) &= r_1 \cdot \varphi(m_1), \\
%     \varphi(m_1 \cdot r_2) &= \varphi(m_1) \cdot r_2.
%     \end{align*}

%     \item If $M$ and $N$ are $(R,S)$-bimodules, then for all $m_1,m_2 \in M$, $r \in R$, and $s \in S$,
%     \begin{align*}
%     \varphi(m_1 + m_2) &= \varphi(m_1) + \varphi(m_2), \\
%     \varphi(r \cdot m_1) &= r \cdot \varphi(m_1), \\
%     \varphi(m_1 \cdot s) &= \varphi(m_1) \cdot s.
%     \end{align*}
% \end{enumerate}
\end{definition}
\begin{definition} \label{definition:module_over_a_sheaf_of_rings_on_a_site}

    \begin{enumerate}
        \item 
        Let $\mathcal{C}$ be a \CrefAndHyperrefIfExist{definition:grothendieck_topology_on_a_category_site_covering_sieve_topologically_generating_family}{site}, and let $\mathcal{A}$ and $\mathcal{B}$ be \CrefAndHyperrefIfExist{definition:sheaf_on_a_site}{sheaves} of (not necessarily commutative) \CrefAndHyperrefIfExist{definition:ring}{rings} on $\mathcal{C}$. 
        
        \begin{enumerate}
            \item 
            An \hldef{$(\mathcal{A}, \mathcal{B})$-bimodule} (or a \hldef{bimodule over $(\mathcal{A}, \mathcal{B})$}) is a \CrefAndHyperrefIfExist{definition:sheaf_on_a_site}{sheaf} $\mathcal{M}$ of abelian groups on $\mathcal{C}$ equipped with a left $\mathcal{A}$-module structure given by a \CrefAndHyperrefIfExist{definition:sheaf_on_a_site}{morphism of sheaves} of sets
            $$ \lambda: \mathcal{A} \times \mathcal{M} \longrightarrow \mathcal{M}, $$
            and a right $\mathcal{B}$-module structure given by a morphism of sheaves of sets
            $$ \rho: \mathcal{M} \times \mathcal{B} \longrightarrow \mathcal{M}, $$
            such that the actions are compatible. Specifically, for every object $U$ in $\mathcal{C}$, every section $m \in \mathcal{M}(U)$, every $a \in \mathcal{A}(U)$, and every $b \in \mathcal{B}(U)$, the equality
            $$ \lambda_U(a, \rho_U(m, b)) = \rho_U(\lambda_U(a, m), b) $$
            holds in $\mathcal{M}(U)$. In standard multiplicative notation where $\lambda(a,m)$ is denoted $a \cdot m$ and $\rho(m,b)$ is denoted $m \cdot b$, this condition is the associativity axiom
            $$ (a \cdot m) \cdot b = a \cdot (m \cdot b). $$

            In particular, for every object $U \in \calC$, the abelian group $\calM(U)$ has the structure of an \CrefAndHyperrefIfExist{definition:module_of_a_ring}{$\calA(U)-\calB(U)$-bimodule}.

            \item Let $\mathcal{M}$ and $\mathcal{N}$ be $(\mathcal{A}, \mathcal{B})$-bimodules. A \hldef{homomorphism of $(\mathcal{A}, \mathcal{B})$-bimodules} (or an \hldef{$(\mathcal{A}, \mathcal{B})$-linear morphism}) is a morphism of sheaves of abelian groups $f: \mathcal{M} \to \mathcal{N}$ such that for every object $U$ of $\mathcal{C}$, every section $m \in \mathcal{M}(U)$, every $a \in \mathcal{A}(U)$, and every $b \in \mathcal{B}(U)$, the following compatibility conditions hold:
            $$ f_U(a \cdot m) = a \cdot f_U(m) \quad \text{and} \quad f_U(m \cdot b) = f_U(m) \cdot b. $$


        \end{enumerate}

        \noindent We denote the category of $(\mathcal{A}, \mathcal{B})$-bimodules, with morphisms being morphisms of sheaves of abelian groups compatible with both the left $\mathcal{A}$-action and the right $\mathcal{B}$-action, by
        \hl{$ \mathcal{A}\text{-}\mathcal{B}\text{-}\mathsf{Mod} $}
        or sometimes by
        \hl{$ {}_{\mathcal{A}}\mathsf{Mod}_{\mathcal{B}} $}
        \TODO{talk about how bimodules can be identifies with left/right modules}

        \item 

        Let $(\mathcal{C}, J)$ be a \CrefAndHyperrefIfExist{definition:grothendieck_topology_on_a_category_site_covering_sieve_topologically_generating_family}{site}. Let $\mathcal{O}$ be a \CrefAndHyperrefIfExist{definition:sheaf_on_a_site}{sheaf of (not necessarily commutative) rings on $(\mathcal{C}, J)$}, i.e. $((\calC, J), \calO)$ is a \CrefAndHyperrefIfExist{definition:ringed_site}{ringed site}.  

        \begin{enumerate}
            \item An \hldef{(left/right/two-sided) $\mathcal{O}$-module} consists of the following data:
            \begin{itemize}
                \item A sheaf $\mathcal{F}$ of abelian groups on $(\mathcal{C}, J)$,
            \item for every object $U \in \mathcal{C}$, the structure of an (left/right/two-sided) $\mathcal{O}(U)$-module on $\mathcal{F}(U)$,
            \end{itemize}
            such that for every morphism $f: V \to U$ in $\mathcal{C}$, the restriction map 
            $$\rho_{U,V}: \mathcal{F}(U) \to \mathcal{F}(V)$$ 
            is $\mathcal{O}(U)$-linear when the $\mathcal{O}(U)$-action on $\mathcal{F}(V)$ is defined via the natural ring homomorphism 
            $$\mathcal{O}(U) \to \mathcal{O}(V)$$
            induced by $f$.


            \item Let $\mathcal{F}$ and $\mathcal{G}$ be \CrefAndHyperrefIfExist{definition:module_over_a_sheaf_of_rings_on_a_site}{$\mathcal{O}$-modules}.

            A \hldef{morphism of $\mathcal{O}$-modules} $\varphi: \mathcal{F} \to \mathcal{G}$ is a \CrefAndHyperrefIfExist{definition:sheaf_on_a_site}{morphism of sheaves} of abelian groups such that, for every object $U \in \mathcal{C}$, the component map
            $$\varphi_U : \mathcal{F}(U) \to \mathcal{G}(U)$$
            is $\mathcal{O}(U)$-linear, i.e. it satisfies
            $$\varphi_U(r \cdot s) = r \cdot \varphi_U(s) \quad \text{for all } r \in \mathcal{O}(U), \, s \in \mathcal{F}(U).$$

            The collection of all $\mathcal{O}$-modules together with their morphisms of $\mathcal{O}$-modules forms the \hldef{category of $\mathcal{O}$-modules}, denoted \hl{$\mathbf{Mod}(\mathcal{O})$}.

            \TextIfExists{definition:algebra_over_a_sheaf_of_rings_on_a_site}{See also \Cref{definition:algebra_over_a_sheaf_of_rings_on_a_site}.}
        \end{enumerate}

        \noindent In case that a \CrefAndHyperrefIfExist{definition:sheafification_functor_on_a_site}{sheafification functor} 
        $$\PreShv(\calC, \mathbf{Rings}) \to \Shv(\calC, \mathbf{Rings})$$ 
        exists, a left, right, two-sided $\calO$-module (and morphisms thereof) is equivalent to a $(\calO,\bbZ)$-bimodule, $(\bbZ,\calO)$-bimodule, and $(\calO, \calO)$-bimodule (and morphisms thereof) respectively, where $\bbZ$ is the \CrefAndHyperrefIfExist{definition:constant_sheaf_on_a_site_with_sheafification}{constant sheaf} of the integer ring $\bbZ$.

\end{enumerate}


\end{definition}


% See Also
% theorem:category_of_modules_over_a_sheaf_of_rings_on_a_site_on_an_essentially_small_category_has_enough_injectives

\begin{definition} \label{definition:affine_space_of_dimension_n_over_a_scheme}
Let $S$ be a \CrefAndHyperrefIfExist{definition:scheme}{scheme} and let $n \geq 0$ be an integer. We define the \hldef{affine space of dimension $n$ over $S$}, denoted by \hl{$\mathbb{A}^n_S$}, as follows:
\begin{enumerate}
    \item If $S = \operatorname{Spec} A$ is an \CrefAndHyperrefIfExist{definition:affine_scheme}{affine scheme}, then $\mathbb{A}^n_S$ is the affine scheme defined by the polynomial ring in $n$ variables over $A$:
    $$\mathbb{A}^n_{\operatorname{Spec} A} = \operatorname{Spec}(A[T_1, \dots, T_n]).$$
    \item For a general scheme $S$, let $\{U_i = \operatorname{Spec} A_i\}_{i \in I}$ be an affine open covering of $S$. For each $i$, let $X_i = \mathbb{A}^n_{U_i} = \operatorname{Spec}(A_i[T_1, \dots, T_n])$. Since polynomial rings behave well under localization, for any open immersion $U_{ij} = U_i \cap U_j \hookrightarrow U_i$, there is a canonical isomorphism on the overlaps:
    $$\phi_{ij}: X_i|_{U_{ij}} \xrightarrow{\sim} X_j|_{U_{ij}}.$$
    The scheme $\mathbb{A}^n_S$ is obtained by gluing the family $\{X_i\}_{i \in I}$ along these isomorphisms.
\end{enumerate}
\TextIfExists{definition:relative_spectrum_of_a_quasi_coherent_sheaf_of_O_algebras_on_a_scheme}{
Alternatively, $\mathbb{A}^n_S$ can be defined globally as the \CrefAndHyperrefIfExist{definition:relative_spectrum_of_a_quasi_coherent_sheaf_of_O_algebras_on_a_scheme}{relative spectrum} of the sheaf of polynomial algebras over $\mathcal{O}_S$:
$$\mathbb{A}^n_S = \mathbf{Spec}(\mathcal{O}_S[T_1, \dots, T_n]).$$
}
\end{definition}