%% Delete this \nocite command invocation to make the references section only list out the bibitems that are actually cited.

\section{Elliptic curves}

\subsection{General abelian varieties}

\begin{definition}[Abelian variety over a field] \label{definition:abelian_variety_over_a_field}
    Let $k$ be a field. An \hldef{abelian variety over $k$} is a complete, connected algebraic group variety defined over $k$, i.e.
    \begin{itemize}
        \item $A$ is a smooth, proper, geometrically connected algebraic variety over $k$,
        \item $A$ is endowed with a group structure defined by morphisms of varieties over $k$ (multiplication $m: A \times_k A \to A$ and inverse $i: A \to A$),
        \item the group law satisfies the group axioms scheme-theoretically.
    \end{itemize}
    In particular, an abelian variety is a projective \CrefAndHyperrefIfExist{definition:algebraic_group_scheme_over_a_scheme}{algebraic group variety} over $k$.
\end{definition}



\begin{definition}[Abelian scheme] \label{definition:abelian_scheme_over_a_scheme}
Let $S$ be a \CrefAndHyperrefIfExist{definition:scheme}{scheme}. An \hldef{abelian scheme over $S$} is a proper and smooth \CrefAndHyperrefIfExist{definition:algebraic_group_scheme_over_a_scheme}{group scheme} 
$$\pi: A \to S$$ 
\TODO{define geometric fiber, geometrically connected, abelia nvariety}
with geometrically connected fibers. Each geometric fiber $A_{\overline{s}}$ (over a geometric point $\overline{s} \to S$) is then an \CrefAndHyperrefIfExist{definition:abelian_variety_over_a_field}{abelian variety}.

In particular, over a field, an abelian scheme is precisely an abelian variety.
\end{definition}


\begin{definition}[Elliptic curve over a scheme] \label{definition:elliptic_curve_over_a_scheme}
Let $S$ be a scheme. An \hldef{elliptic curve over $S$} is a pair $(E, \pi)$ where
\begin{itemize}
    \item $E$ is a scheme together with a morphism $\pi: E \to S$,
    \item $(E, \pi)$ is an abelian scheme of relative dimension $1$ over $S$, i.e. $\pi$ is proper, smooth, of relative dimension $1$, with geometrically connected fibers, and 
    \item a chosen section $e: S \to E$, called the \hldef{zero section}, endowing $(E, \pi)$ with the structure of a commutative group scheme over $S$.
\end{itemize}
Equivalently, an elliptic curve over $S$ is a smooth proper curve of genus $1$ over $S$ together with a marked $S$-point that plays the role of the identity.
\end{definition}



\subsection{Weierstrass equations}

\TODO{polynomial rings}

% 
\begin{definition} \label{definition:commutative_ring}
   A \hldef{commutative (unital) ring} is a \CrefAndHyperrefIfExist{definition:ring}{ring} $(R, +, \cdot)$ such that $\cdot$ is a \CrefAndHyperrefIfExist{definition:commutative_binary_operation}{commutative operation}, i.e. $a \cdot b = b \cdot a$. 

   For many writers (e.g. ``commutative'' algebraists or number theorists), a \hldef{ring} refers to a commutative ring as above.
\end{definition}

% 
\begin{definition} \label{definition:unit_of_a_ring}
    Let $(R,+,\cdot)$ be a \CrefAndHyperrefIfExist{definition:ring}{not-necessarily commutative ring}. A \hldef{unit} or \hldef{invertible element of $R$} is an element $u \in R$ such that there exist an element $v \in R$ such that 
    $$uv = 1 = vu.$$ 
    Such an element $v$ is called the \hldef{multiplicative inverse of $u$} and is often denoted by \hl{$u^{-1}$}. If it exists, then it is unique.
    
    The set of units of $R$ forms a \CrefAndHyperrefIfExist{definition:group}{group}, often denoted by \hl{$R^\times$} or \hl{$R^*$}, under the multiplication operation $\cdot$. It is called the \hldef{group of units} or \hldef{unit group} of $R$.
\end{definition}

\subsubsection{Algebraic facts entirely about Weierstrass equations}


% \begin{definition}[General Weierstrass Equation] \label{definition:weierstrass_equation_over_a_ring}
% Let $R$ be a \CrefAndHyperrefIfExist{definition:commutative_ring}{(commutative unital) ring}, and let $a_1,a_2,a_3,a_4,a_6 \in R$. A \hldef{general Weierstrass equation over $R$} is an equation of the form 
% $$y^2 + a_1 x y + a_3 y = x^3 + a_2 x^2 + a_4 x + a_6.$$
% for variables $x$ and $y$ and where $a_1,a_3,a_2,a_4,a_6 \in R$. A \hldef{(short) Weierstrass equation over $R$} is a Weierstrass equation for which $a_1,a_3,a_2 = 0$.
% \end{definition}

\begin{definition}[General Weierstrass Equation over a Ring] \label{definition:weierstrass_equation_over_a_ring}
    \TODO{define homogenization of a polynomial}
Let $R$ be a \CrefAndHyperrefIfExist{definition:commutative_ring}{(commutative unital) ring}, and let $a_1,a_2,a_3,a_4,a_6 \in R$.

A \hldef{general Weierstrass equation over $R$} is either of the following equivalent descriptions:

\begin{itemize}
    \item The affine equation in variables $x,y$ over $R$
    $$ y^2 + a_1 x y + a_3 y = x^3 + a_2 x^2 + a_4 x + a_6.  $$
    
    \item The projective cubic equation 
    $$
    Y^2 Z + a_1 X Y Z + a_3 Y Z^2 = X^3 + a_2 X^2 Z + a_4 X Z^2 + a_6 Z^3
    $$
    in homogeneous coordinates $X,Y,Z$. Note that this defines a closed subscheme
    inside $\mathbf{P}^2_R = \operatorname{Proj} R[X,Y,Z]$.
\end{itemize}

A \hldef{short Weierstrass equation over $R$} is a Weierstrass equation for which $a_1 = a_3 = a_2 = 0$, having the form
$$ y^2 = x^3 + a_4 x + a_6, $$
or equivalently one which has projective homogenization
$$ Y^2 Z = X^3 + a_4 X Z^2 + a_6 Z^3.  $$
\end{definition}



\begin{definition}[Admissible Change of Variables for a Weierstrass Equation] \label{definition:admissible_change_of_variables_for_a_weierstrass_equation_over_a_commutative_ring_with_unity}
Let $R$ be a commutative ring with unity and let
$$ y^2 + a_1 x y + a_3 y = x^3 + a_2 x^2 + a_4 x + a_6 $$
be a \CrefAndHyperrefIfExist{lemma:general_weierstrass_equation_can_be_simplified_if_2_or_3_is_invertible}{general Weierstrass equation over $R$} with coefficients $a_1,a_2,a_3,a_4,a_6 \in R$. 

An \hldef{admissible change of variables} (or \hldef{isomorphism of Weierstrass equations}) over $R$ is a change of variables of the form

\begin{align*}
x &= u^2 x' + r, \\
y &= u^3 y' + u^2 s x' + t,
\end{align*}
where $u \in R^\times$ (a unit of $R$) and $r,s,t \in R$. This transformation maps the given Weierstrass equation to another of the same form with possibly different coefficients, preserving its structure and properties.
\end{definition}


\begin{proposition}[Admissible Changes of Variables Preserving Short Weierstrass Form] \label{proposition:admissible_changes_of_variables_of_a_short_weierstrass_equation_preserving_short_weierstrass_form}
    Let $R$ be a \CrefAndHyperrefIfExist{definition:commutative_ring}{commutative ring with unity}. Suppose
    $$ y^2 = x^3 + A x + B $$
    is a \CrefAndHyperrefIfExist{definition:weierstrass_equation_over_a_ring}{short Weierstrass equation over $R$} with coefficients $A,B \in R$.

    An \CrefAndHyperrefIfExist{definition:admissible_change_of_variables_for_a_weierstrass_equation_over_a_commutative_ring_with_unity}{admissible change of variables}
    $$ x = u^2 x' + r, \quad y = u^3 y' + u^2 s x' + t, $$
    with $u \in R^\times$ and $r,s,t \in R$, transforms this equation into another short Weierstrass equation
    $$ {y'}^2 = {x'}^3 + A' x' + B', $$
    if and only if the translation parameters satisfy $r = t = s = 0$. In this case, the coefficients transform as
    $$ A' = u^4 A, \quad B' = u^6 B.  $$

    Thus, the subgroup of admissible changes preserving the short Weierstrass form consists precisely of scaling transformations
    $$ x = u^2 x', \quad y = u^3 y'.  $$
\end{proposition}


\begin{lemma} \label{lemma:general_weierstrass_equation_can_be_simplified_if_2_or_3_is_invertible}
    Let $R$ be a \CrefAndHyperrefIfExist{definition:commutative_ring}{(commutative unital) ring}. A \CrefAndHyperrefIfExist{weierstrass_equation_over_a_ring}{general Weierstrass equation}
    \begin{equation} 
    y^2 + a_1 x y + a_3 y = x^3 + a_2 x^2 + a_4 x + a_6 
    \end{equation}
    
    is isomorphic (via an \CrefAndHyperrefIfExist{definition:admissible_change_of_variables_for_a_weierstrass_equation_over_a_commutative_ring_with_unity}{admissible change of variables}) to the following Weierstrass equations under the following conditions:
    \begin{itemize}
        \item If $2,3 \in R^\times$, then to
        a short Weierstrass equation
        $$ y^2 = x^3 + A x + B, $$
        with $A,B \in R$. 
        \item If $2 \in R^\times$, then to a simplified Weierstrass equation of the form
        $$ y^2 = x^3 + a_2 x^2 + a_4 x + a_6, $$
        with $a_2,a_4,a_6 \in R$.

        \item If $3 \in R^\times$, then to a simplified Weierstrass equation of the form
        $$ y^2 + a_1 x y + a_3 y = x^3 + b_4 x + b_6, $$
        with $a_1,a_3,b_4,b_6 \in R$.

    \end{itemize} 
\end{lemma}



\begin{notation}[Auxiliary Invariants for a General Weierstrass Equation] \label{notation:auxiliary_invariants_for_a_general_weierstrass_equation}
    Let $R$ be a \CrefAndHyperrefIfExist{definition:commutative_ring}{(commutative unital) ring}. 
    For the \CrefAndHyperrefIfExist{lemma:general_weierstrass_equation_can_be_simplified_if_2_or_3_is_invertible}{general Weierstrass equation}
    $$ y^2 + a_1 x y + a_3 y = x^3 + a_2 x^2 + a_4 x + a_6.  $$
    with coefficients $a_1,a_2,a_3,a_4,a_6 \in R$, the following are standard notation:

    \hlalign{
    \begin{align*}
    b_2 &= a_1^2 + 4a_2, \\
    b_4 &= 2a_4 + a_1 a_3, \\
    b_6 &= a_3^2 + 4a_6, \\
    b_8 &= a_1^2 a_6 + 4 a_2 a_6 - a_1 a_3 a_4 + a_2 a_3^2 - a_4^2. \\
    c_{4}&=b_{2}^{2}-24 b_{4} \\
    c_{6}&=-b_{2}^{3}+36 b_{2} b_{4}-216 b_{6}
    \end{align*} }

    For a \CrefAndHyperrefIfExist{definition:weierstrass_equation_over_a_ring}{short Weierstrass equation}
    $$ y^2 = x^3 + A x + B, $$
    with $A,B \in R$, these invariants simplify as follows:

    \begin{align*}
    b_{2} &= 0, \\
    b_{4} &= 2 A, \\
    b_{6} &= 4 B, \\
    b_{8} &= -A^2, \\
    c_{4} &= -24 b_{4} = -48 A, \\
    c_{6} &= -216 b_{6} = -864 B.
    \end{align*}



\end{notation}


\begin{definition}[Discriminant of a General Weierstrass Equation] \label{definition:discriminant_of_a_general_weierstrass_equation}
    Let $R$ be a \CrefAndHyperrefIfExist{definition:commutative_ring}{(commutative unital) ring}. 
    For the \CrefAndHyperrefIfExist{definition:weierstrass_equation_over_a_ring}{general Weierstrass equation}
    $$ y^2 + a_1 x y + a_3 y = x^3 + a_2 x^2 + a_4 x + a_6.  $$
    with coefficients $a_1,a_2,a_3,a_4,a_6 \in R$, the following are standard notation:
    Let $R$ and $a_1,a_2,a_3,a_4,a_6 \in R$ be as above. The \hldef{discriminant of the general Weierstrass equation}
    $$ y^2 + a_1 x y + a_3 y = x^3 + a_2 x^2 + a_4 x + a_6 $$
    is defined as the element \hl{$\Delta \in R$} given by
    $$ \Delta = -b_2^2 b_8 - 8 b_4^3 - 27 b_6^2 + 9 b_2 b_4 b_6, $$
    (\Cref{notation:auxiliary_invariants_for_a_general_weierstrass_equation})

    The discriminant for a \CrefAndHyperrefIfExist{definition:weierstrass_equation_over_a_ring}{short Weierstrass equation}
    $$ y^2 = x^3 + A x + B, $$
    with $A,B \in R$ is thus
    $$ \Delta = -16(4A^3 + 27B^2).  $$
\end{definition}




\begin{definition}[\hldef{j-invariant} of a General Weierstrass Equation] \label{definition:j_invariant_of_a_general_weierstrass_equation_with_invertible_discriminant}
    Let $R$ be a \CrefAndHyperrefIfExist{definition:commutative_ring}{(commutative unital) ring}. 
    For the \CrefAndHyperrefIfExist{definition:weierstrass_equation_over_a_ring}{general Weierstrass equation}
    $$ y^2 + a_1 x y + a_3 y = x^3 + a_2 x^2 + a_4 x + a_6, $$
    with coefficients $a_1,a_2,a_3,a_4,a_6 \in R$, the following are standard notation:
    
    The \hldef{j-invariant} associated to the Weierstrass equation (assuming $\Delta$ is \CrefAndHyperrefIfExist{definition:unit_of_a_ring}{invertible in $R$}) is the element \hl{$j \in R$} given by
    $$ j = \frac{c_4^3}{\Delta}, $$
    (\Cref{notation:auxiliary_invariants_for_a_general_weierstrass_equation})
    
    For a \CrefAndHyperrefIfExist{definition:weierstrass_equation_over_a_ring}{short Weierstrass equation}
    $$
    y^2 = x^3 + A x + B,
    $$
    with $A,B \in R$ the $j$-invariant becomes
    $$ j = 1728 \frac{4 A^3}{4 A^3 + 27 B^2} = 1728 \frac{4 A^3}{-\frac{\Delta}{16}} = -1728 \frac{4 A^3}{\Delta / 16}.  $$
\end{definition}




\begin{theorem}[Effect of an Admissible Change of Variables on the Discriminant and j-invariant] \label{theorem:effects_of_admissible_change_of_variables_on_a_gneeral_weierstass_equation_on_discriminant_and_j_invariant}
    Let $R$ be a \CrefAndHyperrefIfExist{definition:commutative_ring}{commutative ring with unity}. Consider a \CrefAndHyperrefIfExist{lemma:general_weierstrass_equation_can_be_simplified_if_2_or_3_is_invertible}{general Weierstrass equation over $R$},
    $$ y^2 + a_1 x y + a_3 y = x^3 + a_2 x^2 + a_4 x + a_6, $$
    with coefficients $a_1,a_2,a_3,a_4,a_6 \in R$, and let
    $\Delta \in R$ and $j \in R$ be its \CrefAndHyperrefIfExist{definition:discriminant_of_a_general_weierstrass_equation}{discriminant} and \CrefAndHyperrefIfExist{definition:j_invariant_of_a_general_weierstrass_equation_with_invertible_discriminant}{$j$-invariant}, respectively.

    Suppose this equation is transformed by an \CrefAndHyperrefIfExist{definition:admissible_change_of_variables_for_a_weierstrass_equation_over_a_commutative_ring_with_unity}{admissible change of variables}
    $$ x = u^2 x' + r, \quad y = u^3 y' + u^2 s x' + t, $$
    where $u \in R^\times$ (a unit), and $r,s,t \in R$ into the Weierstrass equation
    $$ (y')^2 + a_1' x' y' + a_3' y' = (x')^3 + a'_2 (x')^2 + a'_4 x' + a'_6. $$
    Write $b_i'$ and $c_j'$ for the \CrefAndHyperref{notation:auxiliary_invariants_for_a_general_weierstrass_equation}{standard invariants} for this Weierstrass equation and write $\Delta'$ and $j'$ for the discriminant and $j$-invariant of the transformed Weierstrass equation.

    Then the invariants transform according to the following:
    
    $$\begin{aligned} \hline
    u a_{1}^{\prime} &=a_{1}+2 s \\ u^{2} a_{2}^{\prime} &=a_{2}-s a_{1}+3 r-s^{2} \\ u^{3} a_{3}^{\prime} &=a_{3}+r a_{1}+2 t \\ u^{4} a_{4}^{\prime} &=a_{4}-s a_{3}+2 r a_{2}-(t+r s) a_{1}+3 r^{2}-2 s t \\ u^{6} a_{6}^{\prime} &=a_{6}+r a_{4}+r^{2} a_{2}+r^{3}-t a_{3}-t^{2}-r t a_{1} \\
    \hline
    u^{2} b_{2}^{\prime} &=b_{2}+12 r \\ u^{4} b_{4}^{\prime} &=b_{4}+r b_{2}+6 r^{2} \\ u^{6} b_{6}^{\prime} &=b_{6}+2 r b_{4}+r^{2} b_{2}+4 r^{3} \\ u^{8} b_{8}^{\prime} &=b_{8}+3 r b_{6}+3 r^{2} b_{4}+r^{3} b_{2}+3 r^{4} \\
    \hline
    u^{4} c_{4}^{\prime} &=c_{4} \\ u^{6} c_{6}^{\prime} &=c_{6} \\ u^{12} \Delta^{\prime} &=\Delta \\ j^{\prime} &=j  \\ \hline \end{aligned}$$

    % :
    % $$ \Delta' = u^{-12} \Delta, $$
    % and
    % $$ j' = j.  $$

    In particular, the discriminant changes by a factor of the twelfth power of the inverse of the unit $u$, while the $j$-invariant remains invariant under all admissible changes of variables.
\end{theorem}



\subsubsection{Elliptic curves as plane curves given by Weierstrass equations}


\begin{definition}[Elliptic Curve over a Ring Given by a Projective Weierstrass Equation] \label{definition:elliptic_curve_over_a_ring_given_by_a_projective_weierstrass_equation_with_invertible_discriminant}
    Let $R$ be a commutative ring with unity.
    % An \hldef{elliptic curve over $R$ given by a projective Weierstrass equation} is 

    Let 
    $$ Y^2 Z + a_1 X Y Z + a_3 Y Z^2 = X^3 + a_2 X^2 Z + a_4 X Z^2 + a_6 Z^3, $$
    be a \CrefAndHyperrefIfExist{definition:weierstrass_equation_over_a_ring}{homogeneous Weierstrass equation} where $a_1,a_2,a_3,a_4,a_6 \in R$.  
    Let
    $$ \mathcal{E} \subseteq \mathbf{P}^2_R = \operatorname{Proj} R[X,Y,Z] $$
    be the $R$-scheme defined by the homogeneous Weierstrass equation.

    Equip this scheme $\mathcal{E}$ with the structural morphism $\pi: \mathcal{E} \to \operatorname{Spec} R$ induced by the projection, and the distinguished $R$-rational \hldef{point at infinity} defined by the section
    $$ O : \operatorname{Spec} R \to \mathcal{E} $$
    given by the projective point $(X:Y:Z) = (0:1:0)$. 
    
    Assuming that the \CrefAndHyperrefIfExist{definition:j_invariant_of_a_general_weierstrass_equation_with_invertible_discriminant}{discriminant $\Delta$} (of the affinization of the Weierstrass equation) is invertible on $\Spec R$, i.e. is an element of \CrefAndHyperrefIfExist{definition:unit_of_a_ring}{$R^\times$}, the pair $(\mathcal{E}, O)$ is an \CrefAndHyperrefIfExist{definition:j_invariant_of_a_general_weierstrass_equation_with_invertible_discriminant}{elliptic curve over $R$} and is called the \hldef{elliptic curve over $R$ defined by the projective Weierstrass equation} with coefficients $a_1,a_2,a_3,a_4,a_6$.

    The 
    %\hldef{discriminant} \hl{$\Delta = \Delta_{\calE}$}
    \hldef{$j$-invariant} \hl{$j = j_{\calE}$} of $\mathcal{E}$ refers to the 
    %\CrefAndHyperrefIfExist{definition:discriminant_of_a_general_weierstrass_equation}{discriminant} (resp. 
    \CrefAndHyperrefIfExist{definition:j_invariant_of_a_general_weierstrass_equation_with_invertible_discriminant}{$j$-invariant} of the Weierstrass equation.
    
    % We thus call $\mathcal{E}$ the \hldef{elliptic curve over $R$ given by the projective Weierstrass equation}.

    \TODO{define an $R$-rational point on a scheme, projective space, projective points}
\end{definition}




\begin{theorem}[Existence of a Weierstrass Equation for an Elliptic Curve over a Scheme] \label{theorem:elliptic_curve_over_a_scheme_is_zariski_locally_given_by_weierstrass_equation}
    Let $S$ be a \CrefAndHyperrefIfExist{scheme}{scheme}. Let $\mathcal{E} \to S$ be an \CrefAndHyperrefIfExist{definition:elliptic_curve_over_a_scheme}{elliptic curve over $S$}.

    \TODO{make precise the notion of Zariski local}
    Then, Zariski locally on $S$, there exists a \CrefAndHyperrefIfExist{definition:weierstrass_equation_over_a_ring}{Weierstrass equation}
    $$ y^2 + a_1 x y + a_3 y = x^3 + a_2 x^2 + a_4 x + a_6, $$
    with coefficients $a_1,a_2,a_3,a_4,a_6 \in \Gamma(U, \mathcal{O}_S)$ for some open subscheme $U \subseteq S$, and an isomorphism of elliptic curves over $U$ identifying $\mathcal{E}|_U$ with the elliptic curve \CrefAndHyperrefIfExist{definition:elliptic_curve_over_a_ring_given_by_a_projective_weierstrass_equation_with_invertible_discriminant}{defined by (the projective homogenization of) this Weierstrass equation}.

    In other words, every elliptic curve over a scheme admits a Weierstrass equation locally in the Zariski topology on the base.
\end{theorem}



\begin{definition} \label{definition:weierstrass_model_of_an_elliptic_curve_over_the_fraction_field_of_a_dedekind_domain}
Let $R$ be a \CrefAndHyperrefIfExist{definition:dedekind_domain}{Dedekind domain} with field of fractions $K$. Given an elliptic curve $E/K$, a \hldef{Weierstrass model of $E$ over $R$} is a closed subscheme $W \subseteq \mathbb{P}^2_R$ defined by a \CrefAndHyperrefIfExist{definition:weierstrass_equation_over_a_ring}{(projective) Weierstrass equation} such that $W$ is flat over $R$, whose generic fiber $W_K = W \times_{\Spec R} \Spec K$ is isomorphic to $E$ as a curve over $K$.
\end{definition}




\begin{definition} \label{definition:minimal_weierstrass_model_of_elliptic_curve_over_fraction_field_of_dedekind_domain}
    Let $R$ be a \CrefAndHyperrefIfExist{definition:dedekind_domain}{Dedekind domain} with field of fractions $K$ and let $E/K$ be an \CrefAndHyperrefIfExist{definition:elliptic_curve_over_a_scheme}{elliptic curve}. A \hldef{minimal Weierstrass model of $E$ over $R$} is a \CrefAndHyperrefIfExist{definition:weierstrass_model_of_an_elliptic_curve_over_the_fraction_field_of_a_dedekind_domain}{Weierstrass model} $W$ of $E$ over $R$ such that for every nonzero prime ideal $\mathfrak{p} \subset R$, the $R_\mathfrak{p}$-model $W_{R_\mathfrak{p}} = W \times_{\Spec R} \Spec R_{\mathfrak{p}}$ is a Weierstrass model whose \CrefAndHyperrefIfExist{definition:discriminant_of_a_general_weierstrass_equation}{discriminant} $\Delta(W_{R_\mathfrak{p}})$ has minimal possible $v_\mathfrak{p}$-adic valuation among all Weierstrass models of $E$ over $R_\mathfrak{p}$. 
    
    A (either non-homogeneous/affine or homogeneous/projective) \CrefAndHyperrefIfExist{definition:weierstrass_equation_over_a_ring}{Weierstrass equation} yielding a minimal Weierstrass model of $E$ over $R$ would be called a \hldef{minimal Weierstrass equation of $E/R$}.

    \TODO{define the ring of $S$-integers }
    In the case that $R$ is the ring of integers of a local field, or more generally a \CrefAndHyperrefIfExist{definition:discrete_valuation_ring}{DVR}, we might call a minimal Weierstrass model a \hldef{local minimal Weierstrass model} and the equation a \hldef{local minimal Weierstrass equation}. In the case that $R$ has infinitely many prime ideals (e.g. $R$ is the ring of integers or some ring of $S$-integers of a global field), or more generally more than one nonzero prime ideal, we might call a minimal Weierstrass model a \hldef{global minimal Weierstrass model} and the equation a \hldef{global minimal Weierstrass equation}.
\end{definition}


\begin{lemma}[cf. {\cite[Remark VII.1.1, Exercise 7.1]{silverman}} for a discussion over local fields]
    Let $R$ be a \CrefAndHyperrefIfExist{definition:discrete_valuation_ring}{DVR} with \CrefAndHyperrefIfExist{definition:field_of_fractions_of_an_integral_domain}{field of fractions} $K$. Let $E/K$ be an \CrefAndHyperrefIfExist{definition:elliptic_curve_over_a_scheme}{elliptic curve}.
    Writing a general Weierstrass equation in the form
    \begin{equation} \label{eq:general_weierstrass_equation_over_fraction_field_of_DVR}
    y^2 + a_1 xy + a_3 y = x^3 + a_2 x^2 + a_4 x + a_6
    \end{equation}
    where the coefficients $a_i$ are in $K$.

    \begin{enumerate}
        \item There exists a \CrefAndHyperrefIfExist{definition:minimal_weierstrass_model_of_elliptic_curve_over_fraction_field_of_dedekind_domain}{(local) minimal Weierstrass model $\calE$ of $E$ over $R$}. 

        \item Writing the minimal Weierstrass equation in the form \eqref{eq:general_weierstrass_equation_over_fraction_field_of_DVR} where the coefficients $a_i$ are in $R$, we must have $v(a_i) < i$ for at least one of the $i$. 

        \item Given a Weierstrass equation in the form \eqref{eq:general_weierstrass_equation_over_fraction_field_of_DVR} over $R$, if $v(\Delta) < 12$, $v(c_4) < 4$ or $v(c_6) < 6$ (\Cref{notation:auxiliary_invariants_for_a_general_weierstrass_equation}, \Cref{definition:discriminant_of_a_general_weierstrass_equation}), then the equation is minimal over $R$.

        \item If $2$ and $3$ are invertible in $R$ and if the equation is minimal over $R$, then $v(\Delta) < 12$ or $v(c_4) < 4$. \TODO{carefully verify this}
    \end{enumerate}

% In fact, the \CrefAndHyperrefIfExist{definition:discriminant_of_a_general_weierstrass_equation}{discriminant} $\Delta_{\calE}$ of the Weierstrass equation defining $\mathcal{E}$ satisfies
% $$0 \leq v(\Delta_{\calE}) < 12.$$
\end{lemma}

\begin{proof}
    \begin{enumerate}
        \item This is obvious, say due to the well-ordering principle of the integers.

        \item If $v(a_i) < i$ was false for all of the $i$, then the formula for the \CrefAndHyperrefIfExist{definition:discriminant_of_a_general_weierstrass_equation}{discriminant} shows that $v(\Delta) \geq 12$. Let $\pi$ be a \CrefAndHyperrefIfExist{definition:discrete_valuation_ring}{uniformizer of $R$}. The \CrefAndHyperrefIfExist{definition:admissible_change_of_variables_for_a_weierstrass_equation_over_a_commutative_ring_with_unity}{admissible change of variables}
        \begin{align*}
        x &= \pi^2 x' \\
        y &= \pi^3 y'
        \end{align*}
        yields a Weierstrass equation in the variables $x'$ and $y'$ whose discriminant $\Delta'$ satisfies
        $$\Delta' = \pi^{-12} \Delta.$$
        by \Cref{theorem:effects_of_admissible_change_of_variables_on_a_gneeral_weierstass_equation_on_discriminant_and_j_invariant} In fact, the resulting Weierstrass equation is still over $R$, and hence the original Weierstrass equation over $R$ could not have been minimal. 

        \item This holds because a change of variables affects the valuation of $\Delta$, $c_4$, and $c_6$ by a multiple of $12$, $4$, and $6$ respectively.

        \item \TODO{}

    \end{enumerate}
\end{proof}

For general local fields $K$, Tate's algorithm determines whether a given equation is minimal for the ring of integers \TODO{cite}


\TODO{talk about when global minimal weierstrass equations exist, say over global fields, cf. Silverman proposition VIII 8.2, Corollray 8.3}


\subsection{Isogenies of abelian varieties}

\begin{definition}[Isogeny of abelian schemes] \label{definition:isogeny_of_abelian_schemes_over_a_scheme}
    Let $S$ be a \CrefAndHyperrefIfExist{definition:scheme}{scheme}, and let $A$ and $B$ be \CrefAndHyperrefIfExist{definition:abelian_scheme_over_a_scheme}{abelian schemes over $S$}.  
    An \hldef{isogeny of abelian schemes over $S$} or \hldef{$S$-isogeny of abelian schemes} is a \CrefAndHyperrefIfExist{definition:homomorphism_of_algebraic_groups_over_a_scheme}{morphism of group schemes}
    $$\varphi: A \to B$$
    \TODO{define finite, faithfully flat, and sujrective scheme morpshisms}
    over $S$ which is finite, faithfully flat, and surjective.  

    Equivalently, $\varphi$ is a morphism of abelian schemes whose geometric fibers are \hldef{isogenies of abelian varieties}, i.e., surjective homomorphisms with finite kernel.  

    \TextIfExists{definition:isogeny_of_algebraic_group_schemes_over_a_scheme}{Equivalently, some might define an isogeny of abelian schemes over $S$ to simply be an \CrefAndHyperrefIfExist{definition:isogeny_of_algebraic_group_schemes_over_a_scheme}{isogeny of the abelian varieties} as \CrefAndHyperrefIfExist{definition:algebraic_group_scheme_over_a_scheme}{algebraic groups over $S$}, depending on what they mean by an ``isogeny of algebraic group schemes''.}

    The kernel of $\varphi$ is often denoted by \hl{$\ker \varphi$} or \hl{$A[\varphi]$}.
\end{definition}


\begin{definition} \label{definition:isogenous_abelian_varieties_over_a_field}
    Let $A$ and $B$ be abelian varieties over a field $k$. We say that $A$ and $B$ are \hldef{isogenous}, often written \hl{$A \sim B$}, if there exists an \CrefAndHyperrefIfExist{definition:isogeny_of_abelian_schemes_over_a_scheme}{$k$-isogeny} $A \to B$. The relation $\sim$ turns out to be an equivalence relation on the set of abelian varieties over a fixed field $k$.
\end{definition}



\section{N\'eron models of abelian varieties}


\TODO{define smooth, separated, finite type morphism of schemes}
\begin{definition}[Néron mapping property]
Let $R$ be a \CrefAndHyperrefIfExist{definition:dedekind_domain}{Dedekind domain} with \CrefAndHyperrefIfExist{definition:field_of_fractions_of_an_integral_domain}{fraction field $K$}, and let $A/K$ be an abelian variety.  
A smooth separated \CrefAndHyperrefIfExist{definition:algebraic_group_scheme_over_a_scheme}{group scheme} $\mathcal{A}/R$ of finite type extending $A$ satisfies the \hldef{Néron mapping property} if for every smooth $R$-scheme $S$ and every $K$-morphism $f_K: S_K \to A$, there exists a unique $R$-morphism $f: S \to \mathcal{A}$ extending $f_K$.
\end{definition}

\begin{definition}[Néron model of an abelian variety]
Let $K$ be a field that is the fraction field of a \CrefAndHyperrefIfExist{definition:dedekind_domain}{Dedekind domain} $R$, and let $A/K$ be an \CrefAndHyperrefIfExist{definition:abelian_variety_over_a_field}{abelian variety}.  
A \hldef{Néron model of $A$ over $R$} is a smooth separated group scheme $\mathcal{A}/R$ of finite type such that:
\begin{itemize}
    \item The generic fiber of $\mathcal{A}$ is $A$.
    \item $\mathcal{A}$ satisfies the Néron mapping property.
\end{itemize}

If such a model exists, it is unique up to unique isomorphism.
\end{definition}


\section{Reduction}

% \begin{definition}[Non-archimedean place of a global field]
% Let $K$ be a global field. A \hldef{non-archimedean place} $v$ of $K$ is an equivalence class of non-trivial non-archimedean absolute values on $K$.  

% Associated to such a place $v$ are the following:
% \begin{itemize}
%     \item The completion $K_v$ of $K$ at $v$.
%     \item The valuation ring $\mathcal{O}_v = \{ x \in K_v : |x|_v \leq 1 \}$.
%     \item The maximal ideal $\mathfrak{m}_v = \{ x \in K_v : |x|_v < 1 \}$.
%     \item The residue field $k_v = \mathcal{O}_v / \mathfrak{m}_v$.
% \end{itemize}
% \end{definition}



\TODO{read the following definitions}
\TODO{define neron model of an abelian variety}
\TODO{what kind of field is $K$ here?}
\begin{notation}[Reduction of abelian varieties at a non-archimedean place]
Let $A/K$ be an abelian variety and let $v$ be a non-archimedean place of $K$.  
Denote by $\mathcal{A}/\mathcal{O}_v$ the Néron model of $A$ over $\mathcal{O}_v$, which is a smooth, separated, finite type group scheme over $\mathcal{O}_v$ with generic fiber $A$.

The special fiber $\mathcal{A}_v := \mathcal{A} \otimes_{\mathcal{O}_v} k_v$ is called the \hl{$\text{reduction of $A$ at $v$}$}.  
\end{notation}

\begin{definition}[Reduction type of an abelian variety over a non-archimedean place]
Let $A/K$ be an abelian variety, $v$ a non-archimedean place of $K$, and $\mathcal{A}/\mathcal{O}_v$ its Néron model.  
The reduction type of $A$ at $v$ is defined as follows:

\begin{itemize}
    \item $A$ has \hldef{good reduction at $v$} if $\mathcal{A}_v$ is an abelian variety (i.e., smooth, connected, complete, of dimension equal to $\dim A$).
    \item $A$ has \hldef{bad reduction at $v$} if $\mathcal{A}_v$ is not an abelian variety.
        \begin{itemize}
            \item Within bad reduction, $A$ has \hldef{multiplicative reduction at $v$} if the connected component $\mathcal{A}_v^0$ of $\mathcal{A}_v$ is an extension of an abelian variety by a torus of positive dimension.
            \item $A$ has \hldef{additive reduction at $v$} if $\mathcal{A}_v^0$ has a non-trivial unipotent subgroup.
        \end{itemize}
\end{itemize}
\end{definition}

\begin{definition}[Reduction type of an abelian variety over a global field]
Let $A/K$ be an abelian variety over a global field $K$.  
The \hldef{reduction type of $A$ at a place $v$ of $K$} is the classification of the reduction of $A$ over $v$ as good, multiplicative, or additive, according to the preceding definition at all non-archimedean places. At archimedean places, reduction type is not defined.
\end{definition}



\section{Mordell-Weil theorem for abelian varieties over global fields}


\begin{theorem}[Mordell--Weil Theorem] \label{theorem:mordell_weil_theorem_for_abelian_varieties_over_global_fields}
    \TODO{for function fields, I think I need to say that the curve is not isotypical}
    \TODO{Try to find a mordell-weil statement over $\bbQ(T)$}
    Let $K$ be a \CrefAndHyperrefIfExist{definition:global_field}{global field}, and let $E$ be an \CrefAndHyperrefIfExist{definition:abelian_variety_over_a_field}{abelian variety} defined over $K$.  
    Then the group $E(K)$ of $K$-rational points on $E$ is a finitely generated abelian group; that is, there exist integers $r \geq 0$ and a finite abelian group $T$ such that
    $$E(K) \cong \mathbb{Z}^r \times T.$$
    Here, $r$ is called the \hldef{Mordell--Weil rank of $E$ over $K$} or the \hldef{algebraic rank of $E$ over $K$}, and $T$ is the torsion subgroup of $E(K)$.
\end{theorem}


\section{Selmer group for an isogeny of abelian varieties over global fields}



\begin{definition}[Principal homogeneous space (torsor) for a group scheme over a base scheme] \label{definition:principal_homogeneous_space_for_a_group_scheme_over_a_base_scheme}
    Let $S$ be a scheme, and let $G$ be a group scheme over $S$.  
    A \hldef{principal homogeneous space} (or \hldef{$G$-torsor}) over $S$ is an $S$-scheme $X$ equipped with an action
    $$a: G \times_S X \to X$$
    satisfying:
    \begin{itemize}
        \item The action $a$ is simply transitive fpqc-locally on $S$, i.e. there exists an fpqc covering $\{U_i \to S\}$ such that for each $i$, the base-changed scheme $X_{U_i} \cong G_{U_i}$ as $G_{U_i}$-schemes.
        \item The morphism 
        $$G \times_S X \to X \times_S X, \quad (g,x) \mapsto (g \cdot x, x)$$
        is an isomorphism of $S$-schemes, expressing the free and transitive nature of the action.
    \end{itemize}
    \TODO{explain what is meant by local triviality}
    Such a torsor is \'etale locally trivial or locally trivial in the fpqc topology.  
\end{definition}




\begin{definition}[Weil–Châtelet group of an abelian variety over a field] \label{definition:weil_chatelet_group_of_an_abelian_variety_over_a_field}
    Let $k$ be a field, and let $A$ be an abelian variety defined over $k$.  
    Denote by $G_k = \mathrm{Gal}(\overline{k}/k)$ the absolute Galois group of $k$.  

    The \hldef{Weil–Châtelet group of $A$ over $k$}, denoted \hl{$\mathrm{WC}(A/k)$}, is defined as the collection equivalence classes of \CrefAndHyperrefIfExist{definition:principal_homogeneous_space_for_a_group_scheme_over_a_base_scheme}{homogeneous spaces for $A/k$}.
    % In fact, $\mathrm{WC}(A/k)$ classifies the isomorphism classes of principal homogeneous spaces (torsors) under $A$ defined over $k$\CrefIfExists{theorem:the_weil_chatelet_group_of_an_abelian_variety_over_a_field_is_in_bijection_with_the_first_galois_cohomology_group_of_the_abelian_variety}.  
\end{definition}



\begin{theorem} \label{theorem:the_weil_chatelet_group_of_an_abelian_variety_over_a_field_is_in_bijection_with_the_first_galois_cohomology_group_of_the_abelian_variety}
    Let $k$ be a field, and let $A$ be an abelian variety defined over $k$.  
    Denote by $G_k = \mathrm{Gal}(\overline{k}/k)$ the absolute Galois group of $k$.  

    The Weil–Châtelet group $\mathrm{WC}(A/k)$ is in natural bijection with the Galois cohomology group $H^1(G_k, A(\overline{k}))$ where $A(\overline{k})$ denotes the group of $\overline{k}$-rational points of $A$ with its natural continuous $G_k$-action. The bijection can be given by the map
    \begin{align*}
    \mathrm{WC}(A/k) &\to H^1(G_k, A(\overline{k})) \\
    \{C/k\} &\mapsto \{\sigma \mapsto p_0^\sigma - p_0 \text{ for any point } p_0 \in C(\bark)\}.
    \end{align*}
    \TODO{define Galois cohomology}

\end{theorem}



\begin{definition}[Kummer map associated to an isogeny of an abelian variety] \label{definition:kummer_map_associated_to_an_isogeny_of_an_abelian_variety}
    Let $k$ be a field, and let $A$ and $B$ be abelian varieties defined over $k$.  
    Suppose $\varphi: A \to B$ is an \CrefAndHyperrefIfExist{definition:isogeny_of_abelian_schemes_over_a_scheme}{isogeny defined over $k$}.  
    Denote by $G_k = \mathrm{Gal}(\overline{k}/k)$ the absolute Galois group of $k$.  

    The short exact sequence 
    $$0 \to \ker \varphi \to A(\overline{k}) \xrightarrow{\varphi} B(\overline{k}) \to 0.$$
    yields the following long exact sequence in Galois cohomology:
    $$
    \begin{aligned}
    0 \to \ker \varphi(k) \to &A(k) \xrightarrow{\varphi} B(k) \\
    \xrightarrow{\delta_\varphi} H^1(G_k, \ker \varphi) \to &H^1(G_k, A(\overline{k})) \to H^1(G_k, B(\overline{k})) \to \cdots.
    \end{aligned}
    $$

    The \hldef{Kummer map associated to the isogeny $\varphi$} is the Galois cohomological connecting homomorphism
    $$\delta_\varphi: B(k) \to H^1(G_k, \ker \varphi),$$
    above. In fact, there is a short exact sequence
    $$0 \to B(k) / \varphi(A(k)) \xrightarrow{\delta'_{\varphi}} H^1(G_{k}, \ker \varphi) \to H^1(G_{k}, A)[\varphi] \to 0$$
    where the map $\delta'_{\varphi}$ is induced by $\delta_{\varphi}$. We may also let the \hldef{Kummer map associated to $\varphi$} refer to this map $\delta'_{\varphi}$. We may also use the abuse of notation \hl{$\delta_\varphi$} to denote $\delta'_{\varphi}$.
    % The map $\delta_\varphi$ measures the obstruction to lifting points of $B(k)$ to points of $A(k)$ through the isogeny $\varphi$, fundamental in descent theory.

\end{definition}



\begin{definition}[Selmer group for an isogeny of abelian varieties over a global field] \label{definition:selmer_group_for_an_isogeny_of_abelian_varieties_over_a_global_field}
    \TODO{It should be possible to define this for more general group schemes}
    Let $K$ be a \CrefAndHyperrefIfExist{definition:global_field}{global field}, and let $A$ and $B$ be \CrefAndHyperrefIfExist{definition:abelian_variety_over_a_field}{abelian varieties} defined over $K$.  
    Suppose $\varphi: A \to B$ is an \CrefAndHyperrefIfExist{definition:isogeny_of_abelian_schemes_over_a_scheme}{isogeny defined over $K$}.  

    Denote by $G_K = \mathrm{Gal}(\overline{K}/K)$ the absolute Galois group of $K$, and by
    \hl{$\mathrm{Sel}^\varphi(A/K)$} (or \hl{$\mathrm{Sel}^{(\varphi)}(A/K)$})
    the \hldef{Selmer group of $\varphi$ over $K$}, defined as the subgroup of the Galois cohomology group $H^1(K, \ker \varphi)$ given by
    $$\mathrm{Sel}^\varphi(A/K) := \ker \left( H^1(K, \ker \varphi) \to \prod_v H^1(K_v, A(\barK))[\varphi] \right),$$
    \TODO{define the kummer map associated to $\varphi$}
    where the product runs over all \CrefAndHyperrefIfExist{definition:place_of_a_global_field}{places} $v$ of $K$, and $H^1(K_v, A)[\varphi]$ denotes the image of the local Kummer map associated to $\varphi$.  

    We describe the map 
    \begin{equation} \label{eq:map_whose_kernel_is_selmer_group}
    H^1(K, \ker \varphi) \to \prod_v H^1(K_v, A(\barK))[\varphi]
    \end{equation}
    \TODO{define a decomposition group}
    used to define the kernel above: for each place $v$ of $K$, fix an extension $v$ to $\barK$, which yields an embedding $\barK \subset \barK_v$ and a decomposition group $G_v \subset G_K$. Note that $G_v$ acts on $A(\barK_v)$ and $B(\barK_v)$, and the base change $\varphi_v$ of $\varphi$ to $K_v$ induces a \CrefAndHyperrefIfExist{definition:kummer_map_associated_to_an_isogeny_of_an_abelian_variety}{Kummer short exact sequence}
    $$0 \to B(K_v) / \varphi(A(K_v)) \xrightarrow{\delta_{\varphi_v}} H^1(G_v, A[\varphi]) \to H^1(G_v, A(\barK_v))[\varphi] \to 0.$$
    The natural inclusions $G_v \subset G_{K}$ and $E(K) \subset E(K_v)$ give restriction maps on cohomology, and we have a commutative diagram
    \begin{center}
        \begin{tikzcd}
        0 \ar[r] & B'(K) / \varphi(A(K))  \ar[r, "\delta_{\varphi}"] \ar[d] & H^1(G_K, A[\varphi])\ar[r] \ar[d] & H^1(G_K,A(\barK))[\phi] \ar[r] \ar[d] & 0 \\
        0 \ar[r] & \prod_v B'(K_v) / \varphi(A(K_v)) \ar[r, "\delta_{\varphi_v}"]  & \prod_v H^1(G_v,A[\varphi])  \ar[r] & \prod_v H^1(G_v,A(\barK_v))[\phi] \ar[r] & 0
        \end{tikzcd}
    \end{center}
    The map \eqref{eq:map_whose_kernel_is_selmer_group} is the one given in the above commutative diagram.
\end{definition}



The Selmer group is a finite, computable abelian group. %which plays a central role in the arithmetic of $A$ over $K$.

\begin{remark}
    \TODO{discuss variants of selmer groups}
There are many variants of selmer groups. 
\end{remark}

\TODO{define the shafarevich group}
\TODO{read the following statements}


% \begin{theorem}[Basic properties of Selmer groups for isogenies of abelian varieties over global fields]
% Let $K$ be a global field, and let $A$ and $B$ be abelian varieties defined over $K$.  
% Suppose $\varphi: A \to B$ is an isogeny defined over $K$.  
% Denote by $G_K = \mathrm{Gal}(\overline{K}/K)$ the absolute Galois group of $K$.  

% Then the \hldef{Selmer group} $\mathrm{Sel}^\varphi(A/K) \subset H^1(G_K, \ker \varphi)$ defined by
% $$\mathrm{Sel}^\varphi(A/K) := \ker \left( H^1(G_K, \ker \varphi) \to \prod_v H^1(G_v, A)[\varphi] \right),$$
% where the product runs over all places $v$ of $K$, and $G_v \subset G_K$ is a decomposition group at $v$, satisfies the following fundamental properties:
% \begin{itemize}
%     \item $\mathrm{Sel}^\varphi(A/K)$ is a finite abelian group.
%     \item There is an exact sequence
%     $$0 \to B(K)/\varphi(A(K)) \xrightarrow{\delta_\varphi} \mathrm{Sel}^\varphi(A/K) \to \Sha(A/K)[\varphi] \to 0,$$
%     where $\delta_\varphi$ is the global Kummer map associated to $\varphi$, and $\Sha(A/K)$ is the Tate–Shafarevich group of $A/K$.
%     \item The Selmer group fits into the commutative diagram relating global and local Kummer maps:
%     \[
%     \begin{tikzcd}
%     0 \ar[r] & B(K)/\varphi(A(K)) \ar[r, "\delta_\varphi"] \ar[d] & \mathrm{Sel}^\varphi(A/K) \ar[r] \ar[d] & \Sha(A/K)[\varphi] \ar[r] \ar[d] & 0 \\
%     0 \ar[r] & \prod_v B(K_v)/\varphi(A(K_v)) \ar[r, "\delta_{\varphi_v}"] & \prod_v H^1(G_v, A[\varphi]) \ar[r] & \prod_v H^1(G_v, A)(\varphi) \ar[r] & 0,
%     \end{tikzcd}
%     \]
%     where $K_v$ are completions of $K$ at places $v$, and $\delta_{\varphi_v}$ are the local Kummer maps.
% \end{itemize}
% These properties establish the Selmer group as a finite intermediate subgroup controlling the descent obstruction and linking the arithmetic of $A$ and $B$ with local-global principles.
% \end{theorem}

\begin{proposition}[Equivalent characterizations of the Selmer group]
With notation as above, the Selmer group $\mathrm{Sel}^\varphi(A/K)$ admits the following equivalent descriptions:
\begin{itemize}
    \item It is the subgroup of $H^1(G_K, \ker \varphi)$ consisting of classes that are locally in the image of the local Kummer maps $\delta_{\varphi_v}: B(K_v)/\varphi(A(K_v)) \to H^1(G_v, A[\varphi])$ for every place $v$ of $K$.
    \item It consists of Galois cohomology classes of $\ker \varphi$ that come from global torsors under $A$ which become trivial locally everywhere.
    \item It can be viewed as the preimage of the local images under the restriction maps:
    $$\mathrm{Sel}^\varphi(A/K) = \left\{ \xi \in H^1(G_K, \ker \varphi) \ \middle|\ \mathrm{res}_v(\xi) \in \mathrm{Im}(\delta_{\varphi_v}) \ \forall v \right\}.$$
\end{itemize}
\end{proposition}

\begin{corollary}[Functors of Selmer groups under isogenies]
Let $\varphi: A \to B$ and $\psi: B \to C$ be isogenies of abelian varieties over a global field $K$.  
There is a natural map of Selmer groups
$$\mathrm{Sel}^\varphi(A/K) \to \mathrm{Sel}^{\psi \circ \varphi}(A/K)$$
induced by composition of isogenies, compatible with the connecting Kummer maps and functorial in the category of isogenies and abelian varieties.
\end{corollary}




\section{Heights of elliptic curves}

\subsection{Heights of points in projective varieties over number fields}

\begin{definition}[Absolute value and height on projective space] \label{definition:height_of_points_on_projective_space_over_a_global_field}
    \TODO{define the product formula}
    Let $K$ be a global field equipped with a set of normalized absolute values $M_K$ satisfying the product formula. For each $v \in M_K$, let $|\cdot|_v$ denote the corresponding absolute value on $K$.

    Consider a point 
    $$P = (x_0 : x_1 : \cdots : x_n) \in \mathbb{P}^n(K).$$

    The \hldef{height} \hl{$H(P)$} of $P$ is defined as
    $$H(P) = \prod_{v \in M_K} \max \{ |x_0|_v, |x_1|_v, \ldots, |x_n|_v \}.$$

    The \hldef{logarithmic height} (or \hldef{log height}) of $P$ is defined as
    $$\hlin{h(P) = \log H(P) = \sum_{v \in M_K} \log \max \{ |x_0|_v, |x_1|_v, \ldots, |x_n|_v \}.}$$
\end{definition}


\subsection{Heights of elliptic curves over number fields}


\begin{definition}[naive height of an elliptic curve over a number field, cf. {\cite{silverman_ecbdh}}] \label{definition:naive_height_of_an_elliptic_curve_over_a_number_field}
    \TODO{can this definition be aplicable for a global function field}
    Let $K$ be a number field. For $a,b \in K$ with $4a^3 + 27b^2 \neq 0$, let $E(a,b)$ be the elliptic curve given by \CrefAndHyperrefIfExist{definition:weierstrass_equation_over_a_ring}{(affine) Weierstrass equation}
    $$E(a,b): y^2 = x^3 + ax + b$$
    \begin{enumerate}
        \item Define the \hldef{(naive multiplicative) height of an elliptic curve} $E/K$ to be 
        $$\hlin{H(E) = \inf_{\substack{a,b \in K \\ E \cong_K E(a,b)}} H([a^3,b^2,1])}$$
        where $H$ is the \CrefAndHyperrefIfExist{definition:height_of_points_on_projective_space_over_a_global_field}{height function} of points in $\bbP^2(K)$.

        \item Define the \hldef{(naive logarithmic) height of an elliptic curve} $E/K$ to be 
        $$\hlin{h(E) = \inf_{\substack{a,b \in K \\ E \cong_K E(a,b)}} h([a^3,b^2,1])}$$
        where $h$ is the \CrefAndHyperrefIfExist{definition:height_of_points_on_projective_space_over_a_global_field}{logarithmic height function} of points in $\bbP^2(K)$.
    \end{enumerate}
\end{definition}


    \TODO{read tjhe following}
\begin{definition}[Height of an elliptic curve over a number field or function field]
    \TODO{define non-logarithmic height}
    Let $K$ be a global field, i.e., either a number field or a function field. Fix a set of normalized absolute values $M_K$ on $K$ satisfying the product formula.

    Let $E/K$ be an elliptic curve given by a Weierstrass equation with coefficients in $K$. The \hldef{(logarithmic) height of $E$} is defined by
    $$h(E) = \frac{1}{[K:\mathbb{Q}]} \sum_{v \in M_K} n_v \, \log \max \big\{ |4a_4|_v^3, |27a_6|_v^2 \big\},$$
    where $a_4, a_6$ are the coefficients of the minimal Weierstrass equation of $E$, $|\cdot|_v$ are the normalized absolute values on $K$, and $n_v$ are the local degrees.

    This height measures the arithmetic complexity of the elliptic curve and generalizes the classical height notion on points to the curve itself.
\end{definition}


\subsection{Heights functions of elliptic curves over number fields}

    \TODO{read tjhe following}

\begin{definition}[Weil height function on elliptic curves]
Let $K$ be a global field with a set of absolute values $M_K$ normalized so that the product formula holds. Let $E/K$ be an elliptic curve defined by a Weierstrass equation with coefficients in $K$.

The \hldef{(logarithmic) height} of a point $P = (x:y:z) \in E(\overline{K})$ (projective coordinates) is defined by
$$h(P) = \frac{1}{[L : K]} \sum_{v \in M_L} \log \max\{ |x|_v, |y|_v, |z|_v \},$$
where $L$ is a finite extension of $K$ over which $P$ is defined, and the sum is taken over all places $v$ of $L$, with $|\cdot|_v$ suitably normalized absolute values.

The \hldef{height of the elliptic curve} $E$ itself is defined by
$$h(E) = h(j(E)),$$
where $j(E)$ is the $j$-invariant of $E$.
\end{definition}

\begin{definition}[Naive height and canonical height]
Given an elliptic curve $E/K$ and a point $P \in E(\overline{K})$, the \hldef{naive height} $h(P)$ is as above. The \hldef{canonical height} $\hat{h}(P)$ is a quadratic form on $E(\overline{K})$ defined by the Néron-Tate construction, satisfying
\[
\hat{h}(nP) = n^2 \hat{h}(P), \quad \text{and} \quad |\hat{h}(P) - h(P)| < C
\]
for some constant $C$ depending on $E$ and $K$.
\end{definition}



\section{Serre's open image theorem}

Serre's celebrated open image theorem was originally proved for elliptic curves over number fields without complex multiplication. 
\TODO{complex multiplication}
\begin{theorem}[Serre's open image theorem, {\cite[Th\'eor\`eme 2]{serre_pgpofce}}] \label{theorem:serres_original_open_image_theorem_for_elliptic_curves_over_number_fields_without_geometric_complex_multiplication}
    Let $E/K$ be an \CrefAndHyperrefIfExist{definition:elliptic_curve_over_a_scheme}{elliptic curve} over a \CrefAndHyperrefIfExist{global_field}{number field} such that $E$ does not have complex multiplication over $\overline{K}$. \TODO{complex multiplication, algebraically closed field extension, algebraic closure}. For all but finitely many prime numbers $\ell$, the Galois representation
    $$\Gal(\overline{K} / K) \to \Aut(T_\ell E)$$
    on the $\ell$-adic Tate module $T_\ell E$ of $E$ is surjective.
    \TODO{$\ell$-adic tate module}
\end{theorem}

Serre later proved a generalization for abelian varieties:
\begin{theorem}[{\cite[Corollaire au Th\'eor\`eme 3]{serre_lettre_vigneras}}] \label{theorem:serres_generalized_open_image_theorem_for_abelian_varieties_over_number_fields_without_geometric_complex_multiplication}
    Let $A/K$ be an \CrefAndHyperrefIfExist{definition:abelian_variety_over_a_field}{abelian variety} of dimension $n$ over a \CrefAndHyperrefIfExist{definition:global_field}{number field} such that $\End(A_{\overline{K}}) = \bbZ$. \TODO{complete}
    \begin{enumerate}
        \item For all but finitely many primes $\ell$, the ``mod-$\ell$'' Galois representation
        $$\Gal(\overline{K} / K) \to \Aut(A[\ell](\overline{K})) \cong \mathrm{GSp}_{2n}(\bbF_\ell)$$
        is surjective, where $\Aut(A[\ell](\overline{K}))$ is the group of automorphisms on the $\bbF_\ell$-vector space $A[\ell](\overline{K})$ preserving the Weil pairing \TODO{Weil pairing}

        \item For all but finitely many primes $\ell$, the ``$\ell$-adic'' Galois representation
        $$\Gal(\overline{K} / K) \to \Aut(T_\ell A) \cong \mathrm{GSp}_{2n}(\bbZ_\ell)$$
        is surjective, where $\Aut(T_\ell A)$ is the group of automorphisms on the $\bbZ_\ell$-module $T_\ell A$ preserving the Weil pairing 
        \TODO{Weil pairing, Tate module}

        \item The image of the ``ad\'elic'' Galois representation
        $$\Gal(\overline{K} / K) \to \prod_{\ell}' \Aut(V_\ell A) \cong \prod_{\ell}' \mathrm{GSp}_{2n}(\bbQ_\ell)$$
        is open (with respect to the ad\'elic topology), where $\prod_{\ell}' \Aut(V_\ell A)$ is the \CrefAndHyperrefIfExist{definition:restricted_product_of_a_family_of_topological_spaces_with_respect_to_subspaces}{restricted product} of the groups $\Aut(V_\ell A)$ of automorphisms on the $\bbQ_\ell$-vectors spaces $V_\ell A = V_\ell A \otimes_{\bbZ_\ell} \bbQ_\ell$ preserving the Weil pairing.
    \end{enumerate}
\end{theorem}





\appendix

\section{Miscellaneous definitions}


\begin{definition} \label{definition:unit_of_a_ring}
    Let $(R,+,\cdot)$ be a \CrefAndHyperrefIfExist{definition:ring}{not-necessarily commutative ring}. A \hldef{unit} or \hldef{invertible element of $R$} is an element $u \in R$ such that there exist an element $v \in R$ such that 
    $$uv = 1 = vu.$$ 
    Such an element $v$ is called the \hldef{multiplicative inverse of $u$} and is often denoted by \hl{$u^{-1}$}. If it exists, then it is unique.
    
    The set of units of $R$ forms a \CrefAndHyperrefIfExist{definition:group}{group}, often denoted by \hl{$R^\times$} or \hl{$R^*$}, under the multiplication operation $\cdot$. It is called the \hldef{group of units} or \hldef{unit group} of $R$.
\end{definition}


\begin{definition} \label{definition:field_of_fractions_of_an_integral_domain}
Let $R$ be an \CrefAndHyperrefIfExist{definition:zero_divisor_of_a_ring}{integral domain}, and consider the set $R \times (R \setminus \{0\})$ as above. Define a relation $\sim$ on $R \times (R \setminus \{0\})$ by declaring that
$$(a,b) \sim (c,d) \quad \text{if and only if} \quad ad = bc,$$
for $a,c \in R$ and $b,d \in R \setminus \{0\}$.  
This relation is an equivalence relation. Its equivalence classes are denoted by
\hl{$\tfrac{a}{b}$}.

The set of equivalence classes
$$\left\{\, \tfrac{a}{b} \,\middle|\, a \in R, \, b \in R \setminus \{0\} \,\right\}$$
under the relation $\sim$ defined above is called the \hldef{field of fractions of $R$}, and is denoted by \hl{$\operatorname{Frac}(R)$}.  

The operations on $\operatorname{Frac}(R)$ are defined by
\begin{align*}
\tfrac{a}{b} + \tfrac{c}{d} &= \tfrac{ad+bc}{bd}, \\
\tfrac{a}{b} \cdot \tfrac{c}{d} &= \tfrac{ac}{bd},
\end{align*}
for $a,c \in R$ and $b,d \in R \setminus \{0\}$.
With these operations, $\operatorname{Frac}(R)$ is a \CrefAndHyperrefIfExist{definition:field}{field}.

\TextIfExists{definition:localization_of_a_commutative_ring_by_a_multiplicative_subset}{
Equivalently, $\operatorname{Frac}(R)$ may be defined as the \CrefAndHyperrefIfExist{definition:localization_of_a_commutative_ring_by_a_multiplicative_subset}{localization of $R$} by the \CrefAndHyperrefIfExist{definition:multiplicative_subset_of_a_ring}{multiplicative subset $R \setminus \{0\}$}.
}
\end{definition}



\begin{definition}[Discrete valuation ring] \label{definition:discrete_valuation_ring}
\TODO{define principal ideal}
    A local \CrefAndHyperrefIfExist{definition:zero_divisor_of_a_ring}{integral domain} $(R, \mathfrak{m})$ with \CrefAndHyperrefIfExist{definition:prime_and_maximal_ideal_of_a_ring}{maximal ideal} $\mathfrak{m}$ is called a \hldef{discrete valuation ring (DVR)} if $\mathfrak{m}$ is principal and nonzero, and every nonzero ideal of $R$ is of the form $\mathfrak{m}^n$ for some integer $n \geq 0$.

    A \hldef{uniformizer of $R$} refers to any generator of $\mathfrak{m}$.

    The \hldef{(normalized) discrete valuation} $v: R^\times \to \mathbb{Z}_{\geq 1}$ is given by
    $$v(x) = \text{minimal } n \text{ such that } x \in \mathfrak{m}^n.$$
    Alternatively, $v$ may be extended to a map $v: R \to \mathbb{Z}_{\geq 1} \cup \{\infty\}$ by letting $v(0) = \infty$. 

    In fact, $v$ extends to a \CrefAndHyperrefIfExist{definition:discrete_valuation_on_a_field}{discrete valuation} on the \CrefAndHyperrefIfExist{definition:field_of_fractions_of_an_integral_domain}{fraction field of $R$} by defining
    $$v\left( \frac{a}{b} \right) = v(a) - v(b)$$
    for $a \in R$ and $b \in R^\times$. This is a well defined map
    $$v: K \to \bbZ \cup \{\infty\}.$$
\end{definition}



\begin{definition}[Dedekind domain] \label{definition:dedekind_domain}
An \CrefAndHyperrefIfExist{definition:zero_divisor_of_a_ring}{integral domain} $R$ is called a \hldef{Dedekind domain} if it satisfies the following equivalent conditions:
\TODO{define field of fractions}
\begin{itemize}
    \item $R$ is \CrefAndHyperrefIfExist{definition:noetherian_ring}{Noetherian}, \CrefAndHyperrefIfExist{definition:integral_element_over_a_ring}{integrally closed} in its field of fractions, and every nonzero \CrefAndHyperrefIfExist{definition:prime_and_maximal_ideal_of_a_ring}{prime ideal} of $R$ is \CrefAndHyperrefIfExist{definition:prime_and_maximal_ideal_of_a_ring}{maximal}.
    \item Equivalently: for every nonzero prime ideal $\mathfrak{p}$ of $R$, the \CrefAndHyperrefIfExist{definition:localization_of_a_commutative_ring_by_a_multiplicative_subset}{localization} $R_{\mathfrak{p}}$ is a \CrefAndHyperrefIfExist{definition:discrete_valuation_ring}{discrete valuation ring}.
\end{itemize}
\end{definition}


\begin{definition} \label{definition:algebraic_group_scheme_over_a_scheme}
Let $S$ be a scheme. An \hl{algebraic group scheme over $S$} (or an \hl{$S$-group scheme}) is a group object $G$ in the category of schemes over $S$; that is, $G$ is an $S$-scheme equipped with $S$-morphisms:
\hl{$m: G \times_S G \to G$} (\hldef{multiplication}), \hl{$i: G \to G$} (\hldef{inverse}), and \hl{$e: S \to G$} (\hldef{identity}),
satisfying the group axioms expressed by the commutativity of the following diagrams:

\begin{enumerate}
    \item \textbf{Associativity}\quad
    $
    \begin{tikzcd}[column sep=small]
      G \times_S G \times_S G \arrow{r}{m \times \mathrm{id}} \arrow{d}[swap]{\mathrm{id} \times m} & G \times_S G \arrow{d}{m} \\
      G \times_S G \arrow{r}{m} & G
    \end{tikzcd}
    $
    \item \textbf{Identity}\quad
    $
    \begin{tikzcd}[column sep=small]
      G \times_S S \arrow{r}{\mathrm{id} \times e} \arrow{dr}[swap]{\simeq} & G \times_S G \arrow{d}{m} \\
      & G
    \end{tikzcd}
    \qquad
    \begin{tikzcd}[column sep=small]
      S \times_S G \arrow{r}{e \times \mathrm{id}} \arrow{dr}[swap]{\simeq} & G \times_S G \arrow{d}{m} \\
      & G
    \end{tikzcd}
    $
    \item \textbf{Inverse}\quad
    $
    \begin{tikzcd}[column sep=small]
      G \arrow{r}{(\mathrm{id}, i)} \arrow{d}[swap]{\mathrm{id}} & G \times_S G \arrow{d}{m} \\
      G \arrow{r}{e \circ \pi} & G
    \end{tikzcd}
    $
    where $\pi: G \to S$ is the structure morphism and $e \circ \pi$ sends $g$ to the identity section.
\end{enumerate}

\TextIfExists{definition:group_object_in_a_category_with_a_final_object}{Equivalently, a group scheme over $S$ is a \CrefAndHyperrefIfExist{definition:group_object_in_a_category_with_a_final_object}{group object} in the \CrefAndHyperrefIfExist{definition:scheme_over_a_scheme}{category of $S$-schemes}}

If $G$ is \CrefAndHyperrefIfExist{definition:affine_morphism_of_schemes}{affine over} $S$, we call it an \hldef{affine group scheme over $S$}.

If the base scheme $S$ is the spectrum of a field $k$, then we call $G$ a \hldef{$k$-algebraic group} or an \hldef{algebraic group (scheme) over $k$}. If $G$ is additionally a $k$-variety, then we call $G$ a \hldef{$k$-group variety}.
\end{definition}

% \begin{definition} \label{definition:algebraic_group_over_a_field}
%     \TODO{TODO: define variety, group object}
% Let $k$ be a field. An \hl{$k$-algebraic group} (or an \hl{algebraic group over $k$}) is a group object $G$ in the category of $k$-schemes; that is, $G$ is a scheme over $k$ equipped with morphisms $m: G \times G \to G$ (multiplication), $i: G \to G$ (inverse), and $e: \operatorname{Spec} k \to G$ (identity), satisfying the group axioms expressed by the commutativity of the following diagrams:

% \begin{itemize}
%     \item[(Associativity)]
%     \begin{center}
%     \begin{tikzcd}
%     G \times G \times G \arrow{r}{m \times \mathrm{id}} \arrow{d}[swap]{\mathrm{id} \times m} & G \times G \arrow{d}{m} \\
%     G \times G \arrow{r}{m} & G
%     \end{tikzcd}
%     \end{center}

%     \item[(Identity)] 
%     \begin{center}
%     \begin{tikzcd}
%     G \times \operatorname{Spec} k \arrow{r}{\mathrm{id} \times e} \arrow{dr}[swap]{\simeq} & G \times G \arrow{d}{m} \\
%     & G
%     \end{tikzcd}
%     \qquad
%     \begin{tikzcd}
%     \operatorname{Spec} k \times G \arrow{r}{e \times \mathrm{id}} \arrow{dr}[swap]{\simeq} & G \times G \arrow{d}{m} \\
%     & G
%     \end{tikzcd}
%     \end{center}

%     \item[(Inverse)] 
%     \begin{center}
%         \begin{tikzcd}
%         G \arrow{r}{(\mathrm{id}, i)} \arrow{d}[swap]{\mathrm{id}} & G \times G \arrow{d}{m} \\
%         G \arrow{r}{e \circ \pi} & G
%         \end{tikzcd}
%     \end{center}
%     where $\pi: G \to \operatorname{Spec} k$ is the structure morphism and $e \circ \pi$ sends $g$ to the identity.
% \end{itemize}
% If $G$ is a $k$-variety, when we call it a \hldef{$k$-group variety}. If $G$ is an affine $k$-scheme, then we call it an \hldef{affine algebraic $k$-group}.
% \end{definition}
\begin{definition} \label{definition:homomorphism_of_algebraic_groups_over_a_scheme}
Let $S$ be a scheme, and let $G$ and $H$ be \hyperrefIfExists{definition:algebraic_group_scheme_over_a_scheme}{$S$-algebraic groups}. A morphism of $S$-schemes $f: G \to H$ is a \hldef{homomorphism of algebraic groups} if $f$ is a group homomorphism, i.e.,
\begin{itemize}
    \item $f(m_G(x, y)) = m_H(f(x), f(y)) \quad \text{for all } x, y \in G,$
    \item $f(i_G(x)) = i_H(f(x))$ for all $x \in G$, and 
    \item $f(e_G) = e_H$.
\end{itemize}
It is called an \hldef{isomorphism of algebraic groups (over $S$)} if it is additionally an isomorphism of $S$-schemes. If there exists an isomorphism $f: G \to H$ of algebraic groups over $S$, then $G$ and $H$ are said to be \hldef{isomorphic $S$-algebraic groups}.
\end{definition}


\begin{definition} \label{definition:isogeny_of_algebraic_group_schemes_over_a_scheme}
    \TODO{TODO: define kernel,image of a group homomorphism of algebraic groups, }
    Let $S$ be a scheme and let $f: G \to H$ be a \hyperrefIfExists{definition:algebraic_group_scheme_over_a_scheme}{homomorphism} of \hyperrefIfExists{definition:algebraic_group_scheme_over_a_scheme}{$S$-algebraic groups}. 

    There are related, but conflicting definitions for what it means for $f$ to be an \hldef{isogeny of $S$-algebraic groups}:
    \begin{enumerate}
        \item We commonly say that $f$ is an isogeny of algebraic groups if it is surjective and its kernel $\ker f$ is a finite flat group scheme over $S$.
        \item Inequivalently, we might alternatively say that $f$ is an isogeny of algebraic groups if $\ker f$ is a finite flat group scheme over $S$ and its image has finite index in $H$.
    \end{enumerate}
    Unless otherwise specified, the first definition will generally be used.
\end{definition}


\begin{definition}[Scheme] \label{definition:scheme}
    A \hldef{scheme} is a \CrefAndHyperrefIfExist{definition:locally_ringed_space_on_a_topological_space}{locally ringed space} $(X, \mathcal{O}_X)$ that admits an open cover $\{U_i\}_{i \in I}$ such that each $(U_i, \mathcal{O}_X|_{U_i})$ is \CrefAndHyperrefIfExist{definition:morphism_of_locally_ringed_spaces}{isomorphic (as a locally ringed space)} to an \CrefAndHyperrefIfExist{definition:affine_scheme}{affine scheme $(\mathrm{Spec}(A_i), \mathcal{O}_{\mathrm{Spec}(A_i)})$} for some \CrefAndHyperrefIfExist{ring}{commutative ring} $A_i$.  
    In other words, a scheme is a locally ringed space locally isomorphic to affine schemes.

    
\end{definition}

\begin{definition} \label{definition:global_field}
A \hldef{global field} is a field $K$ that is either:
\begin{itemize}
    \item a finite extension of the field of rational numbers $\mathbb{Q}$ (i.e., a \hl{number field}), or
    \item a finite extension of the field of rational functions $\mathbb{F}_q(t)$ in one variable over a finite field $\mathbb{F}_q$ (i.e., a \hl{global function field}).
\end{itemize}
\end{definition}


\begin{definition}[Place of a global field] \label{definition:place_of_a_global_field}
Let $F$ be a global field. A \hldef{place of $F$} is an \hyperrefIfExists{definition:equivalent_absolute_values_on_a_field}{equivalence} class\CrefIfExists{definition:equivalent_absolute_values_on_a_field} of \hyperrefIfExists{definition:absolute_value_on_a_field}{absolute values}\CrefIfExists{definition:absolute_value_on_a_field} on $F$. 

If any (equivalently all) representatives of a place $v$ of $F$ is an \hyperrefIfExists{definition:archimedean_absolute_value_on_a_field}{archimedean absolute value}\CrefIfExists{definition:archimedean_absolute_value_on_a_field} (resp. non-archimedean absolute value), then we say that $v$ is an \hldef{archimedean place} (resp. \hldef{non-archimedean place}). A representative of a place $v$ is often denoted by \hldef{$|\cdot|_v$}.
\end{definition}
\begin{definition}\label{definition:restricted_product_of_a_family_of_topological_spaces_with_respect_to_subspaces}
Let $\{X_i\}_{i \in I}$ be a family of topological spaces indexed by a set $I$. For each $i \in I$, let $K_i \subseteq X_i$ be a topological subspace.

The \hldef{restricted product topology} on the \hyperrefIfExists{definition:restricted_product_of_a_family_of_sets_with_respect_to_a_family_of_distinguished_subsets}{restricted product}
$$ \prod_{i \in I}' X_i := \left\{ (x_i)_{i \in I} \in \prod_{i \in I} X_i \;\middle|\; x_i \in K_i \text{ for all but finitely many } i \in I \right\}, $$
with respect to the subsets $\{K_i\}_{i \in I}$, is the coarsest topology such that:
\begin{itemize}
    \item The natural inclusion maps $X_j \to \prod_{i \in I}' X_i$, defined by $x_j \mapsto (y_i)$ where $y_j = x_j$ and $y_i = k_i$ (a fixed element in $K_i$) for all $i \neq j$, are continuous for all $j \in I$.
    \item The subspace topology on the product $\prod_{i \in F} X_i$ for any finite subset $F \subseteq I$ (where coordinates outside $F$ are fixed in $K_i$) coincides with the product topology on finitely many factors.
\end{itemize}
Equivalently, the restricted product topology is generated by the base consisting of sets of the form
$$ \prod_{i \in F} U_i \times \prod_{i \notin F} K_i, $$
where $F \subseteq I$ is finite, $U_i$ are open sets in $X_i$, and outside $F$ the coordinates lie in $K_i$.
\end{definition}


\begin{definition} \label{definition:adeles_and_ideles_of_a_global_field}
Let $K$ be a \hyperrefIfExists{definition:global_field}{global field}. Write $M_K$ for the set of all \hyperrefIfExists{definition:place_of_a_global_field}{places}\CrefIfExists{definition:place_of_a_global_field} of $K$ and write $M_K^\infty$ for the set of \hyperrefIfExists{definition:place_of_a_global_field}{archimedean places} of $K$. Let $S \subseteq M_K$ be some subset of places of $K$ (typically, $S$ is a finite set). For each $v \in M_K$, write $\calO_v$ for the \CrefAndHyperrefIfExist{definition:ring_of_integers_of_a_global_or_local_ring}{ring of integers} in \CrefAndHyperrefIfExist{definition:completion_of_a_global_field_at_a_place}{the completion $K_v$}\CrefIfExists{theorem:completion_of_a_global_field_at_a_place_is_a_local_field}
\begin{itemize}
    \item The \hldef{adèle ring of $K$}, denoted \hl{$\mathbb{A}_K$}, is the \hyperrefIfExists{definition:restricted_product_of_a_family_of_topological_spaces_with_respect_to_subspaces}{restricted direct product} of the $K_v$ (over all places $v$ of $K$), with respect to the $\mathcal{O}_v$ at \hyperrefIfExists{definition:archimedean_absolute_value_on_a_field}{non-archimedean} $v$:
    $$ \hlin{\mathbb{A}_K = \left\{ (x_v)_v \in \prod_{v \in M_K} K_v \;\middle|\; x_v \in \mathcal{O}_v \text{ for all but finitely many non\text{-}archimedean $v$} \right\}.}  $$

    \item The \hldef{idèle group of $K$}, commonly denoted \hl{$\mathbb{A}_K^\times$} or \hl{$\bbI_K$}, is the group of invertible elements of $\mathbb{A}_K$: 
    $$ \hlin{\bbI_K = \mathbb{A}_K^\times = \left\{ (x_v)_v \in \prod_{v \in M_K} K_v^\times \;\middle|\; x_v \in \mathcal{O}_v^\times \text{ for all but finitely many non\text{-}archimedean $v$} \right\},} $$
    where $\mathcal{O}_v^\times$ denotes the group of units of $\mathcal{O}_v$ for non-archimedean $v$.

    \item The \hldef{adèle ring outside $S$ of $K$}, commonly denoted \hl{$\mathbb{A}_K^S$} or \hl{$\mathbb{A}_{K,S}$}, is the restricted product of the completions $K_v$ over all places $v \in M_K \setminus S$, with respect to the rings of integers $\mathcal{O}_v$ at non-archimedean places:
    $$\hlin{\mathbb{A}_{K,S} = \mathbb{A}_K^S = \left\{ (x_v)_v \in \prod_{v \in M_K \setminus S} K_v \;\middle|\; x_v \in \mathcal{O}_v \text{ for all but finitely many non-archimedean } v \right\}.}$$
    \item The \hldef{idèle group outside $S$ of $K$}, commonly denoted \hl{$(\mathbb{A}_K^\times)^S$}, \hl{$(\mathbb{A}_{K,S}^\times)$}, \hl{$\bbI_K^S$}, or \hl{$\bbI_{K,S}$} is the group of invertible elements of $\mathbb{A}_K^S$:
    $$\hlin{(\mathbb{A}_K^\times)^S = \left\{ (x_v)_v \in \prod_{v \in M_K \setminus S} K_v^\times \;\middle|\; x_v \in \mathcal{O}_v^\times \text{ for all but finitely many non-archimedean } v \right\}.}$$

    \item The \hldef{ring of finite adèles of $K$}, commonly denoted \hl{$\mathbb{A}_{K, \mathrm{fin}}$}, \hl{$\mathbb{A}_{K}^{\mathrm{fin}}$}, \hl{$\mathbb{A}_{K, \mathrm{f}}$}, \hl{$\mathbb{A}_{K}^{\mathrm{f}}$}, is the adèle ring outside $S = M_K^\infty$, the set of archimedean places of $K$:
    $$\hlin{\mathbb{A}_{K, \mathrm{fin}} := \mathbb{A}_K^{M_K^\infty} = \left\{ (x_v)_v \in \prod_{v \notin M_K^\infty} K_v \;\middle|\; x_v \in \mathcal{O}_v \text{ for all but finitely many non-archimedean } v \right\}.} $$

    \item The \hldef{finite idèle group of $K$}, commonly denoted \hl{$\mathbb{A}_{K, \mathrm{fin}}^\times$}, \hl{$\mathbb{I}_{K, \mathrm{fin}}$}, \hl{$\mathbb{I}_{K}^{\mathrm{fin}}$}, \hl{$\mathbb{I}_{K, \mathrm{f}}$}, \hl{$\mathbb{I}_{K}^{\mathrm{f}}$} etc.  is the group of units of the ring of finite adèles:
    $$\hlin{\mathbb{A}_{K, \mathrm{fin}}^\times := (\mathbb{A}_K^\times)^{M_K^\infty} = \left\{ (x_v)_v \in \prod_{v \notin M_K^\infty} K_v^\times \;\middle|\; x_v \in \mathcal{O}_v^\times \text{ for all but finitely many non-archimedean } v \right\}.}$$

\end{itemize}
All of these are equipped with the \hyperrefIfExists{definition:restricted_product_of_a_family_of_topological_spaces_with_respect_to_subspaces}{restricted product topology} induced by the topologies of the \hyperrefIfExists{definition:local_field}{local fields}\CrefIfExists{definition:local_field} $K_v$ and the subspace topologies thereof.
\end{definition}
