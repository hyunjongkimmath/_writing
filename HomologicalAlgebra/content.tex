
\section{Categories}
\begin{definition}[Category] \label{definition:category}
    A 
    \defin{category}{category}{
        name={Category},
        description={A nice enough collection of objects and morphisms (\Cref{definition:category})},
    }
    \hldef{category} $\mathcal{C}$ consists of the following data:
    \begin{itemize}
        \item A class of \defin{objects}{object_of_a_category}{
            name={Object of a category},
            description={\Cref{definition:category}},
        }
        denoted \notat{\operatorname{Ob}(\mathcal{C})}{class_of_objects_of_a_category}{
            name={$\operatorname{Ob}(\mathcal{C})$},
            description={Class of objects of a category $\calC$ \Cref{definition:category}},
            sort={Ob},
        }.
        % \hl{$\operatorname{Ob}(\mathcal{C})$}.
        \item For each pair of objects $X, Y \in \operatorname{Ob}(\mathcal{C})$, a class
        \notatin{\operatorname{Hom}_{\mathcal{C}}(X,Y)}{class_of_morphisms_between_two_objects_of_a_category}
        {
            name={$\operatorname{Hom}_{\mathcal{C}}(X,Y)$},
            description={Class of morphisms between objects $X$ and $Y$ of the category $\calC$ (\Cref{definition:category})},
            sort={Hom},
        }
        % $$\hlin{\operatorname{Hom}_{\mathcal{C}}(X,Y)}$$
        of \defin{morphisms}{morphism_between_objects_of_a_category}{
            name={Morphism between objects of a category},
            description={(\Cref{definition:category})},
        }
        (also called 
        \defin{arrows}{arrow_between_objects_of_a_category}{
            name={Arrow between objects of a category},
            description={Synonym for morphism (\Cref{definition:category})},
        }
        or
        \defin{homs}{hom_between_objects_of_a_category}{
            name={Hom between objects of a category},
            description={Synonym for morphism (\Cref{definition:category})},
        }). If the category $\calC$ is clear, then this \hldef{hom-class} is also denoted by \hl{$\operatorname{Hom}(X,Y)$}. It may also be denoted by \hl{$\operatorname{hom}_{\mathcal{C}}(X,Y)$} or \hl{$\operatorname{hom}(X,Y)$}, especially to distinguish from other types of hom's (e.g. \hyperrefIfExists{definition:internal_hom_object_in_a_category}{internal hom's})
        \item For each triple of objects $X,Y,Z$, a composition law
        $$ \circ : \operatorname{Hom}_{\mathcal{C}}(Y,Z) \times \operatorname{Hom}_{\mathcal{C}}(X,Y) \to \operatorname{Hom}_{\mathcal{C}}(X,Z), $$
        denoted \hl{$(g,f) \mapsto g \circ f$}.
        \item For each object $X$, an \hldef{identity morphism}
        $$\hlin{\operatorname{id}_X \in \operatorname{Hom}_{\mathcal{C}}(X,X).}$$
    \end{itemize}
    These data satisfy the following axioms:
    \begin{itemize}
        \item (Associativity) For all morphisms $f \in \operatorname{Hom}_{\mathcal{C}}(X,Y)$, $g \in \operatorname{Hom}_{\mathcal{C}}(Y,Z)$, and $h \in \operatorname{Hom}_{\mathcal{C}}(Z,W)$, 
        $$
        h \circ (g \circ f) = (h \circ g) \circ f.
        $$
        \item (Identity) For all $f \in \operatorname{Hom}_{\mathcal{C}}(X,Y)$,
        $$
        \operatorname{id}_Y \circ f = f = f \circ \operatorname{id}_X.
        $$
    \end{itemize}
    One often writes \hl{$X \in \calC$} synonymously with $X \in \Ob(\calC)$, i.e. to denote that $X$ is an object of of $\calC$. 

    We may call a category as above an \hldef{ordinary category} to distinguish this notion from the notions of \hyperrefIfExists{definition:category_enriched_in_a_monoidal_category}{\emph{categories enriched in monoidal categories}} or higher/$n$-categories.
    \TODO{TODO: define $n$-categories}

    A category as defined above may be called called a \hldef{large category} or a \hldef{class category} to emphasize that the hom-classes may be proper classes rather than sets (note, however, that the possibility that hom-classes are sets is not excluded for large categories). Accordingly, a \hldef{category} may often refer to a \hyperrefIfExists{definition:locally_small_category}{locally small category}\CrefIfExists{definition:locally_small_category}, which is a category whose hom-classes are all sets.
\end{definition}

% Later on, we refer to the \gls{category} again.

\begin{definition}[Opposite category] \label{definition:opposite_category_of_a_category}

    Let $\mathcal{C}$ be a \hyperrefIfExists{definition:category}{(large) category}\CrefIfExists{definition:category}. The \hldef{opposite category} of $\mathcal{C}$, denoted \hl{$\mathcal{C}^{\mathrm{op}}$}, is defined as follows:
    \begin{itemize}
        \item The objects of $\mathcal{C}^{\mathrm{op}}$ are the same as those of $\mathcal{C}$.
        \item For any pair of objects $X,Y \in \mathcal{C}$, the morphisms from $X$ to $Y$ in $\mathcal{C}^{\mathrm{op}}$ are given by the morphisms from $Y$ to $X$ in $\mathcal{C}$:
        \[
        \mathrm{Hom}_{\mathcal{C}^{\mathrm{op}}}(X,Y) := \mathrm{Hom}_{\mathcal{C}}(Y,X).
        \]
        \item Composition in $\mathcal{C}^{\mathrm{op}}$ is defined by reversing the order of composition in $\mathcal{C}$. That is, for morphisms $f \in \mathrm{Hom}_{\mathcal{C}^{\mathrm{op}}}(X,Y)$ and $g \in \mathrm{Hom}_{\mathcal{C}^{\mathrm{op}}}(Y,Z)$, their composition is
        \[
        g \circ_{\mathcal{C}^{\mathrm{op}}} f := f \circ_{\mathcal{C}} g.
        \]
    \end{itemize}
    Intuitively, the category $\mathcal{C}^{\mathrm{op}}$ thus "reverses" the direction of all morphisms in $\mathcal{C}$.

\end{definition}

\begin{definition} \label{definition:functor_between_categories}
Let $\mathcal{C}$ and $\mathcal{D}$ be \CrefAndHyperrefIfExist{definition:category}{(large) categories}. 
\begin{enumerate}
  \item A \hldef{functor $F: \calC \to \calD$ (from $\mathcal{C}$ to $\mathcal{D}$)} consists of :
  \begin{itemize}
    \item For each object $X$ in $\mathcal{C}$, an object $F(X)$ in $\mathcal{D}$.
    \item For each morphism $f: X \to Y$ in $\mathcal{C}$, a morphism $F(f): F(X) \to F(Y)$ in $\mathcal{D}$,
  \end{itemize}
  such that:
  \begin{align*}
    F(\mathrm{id}_X) &= \mathrm{id}_{F(X)} \quad \text{for all objects } X \text{ in } \mathcal{C}, \\
    F(g \circ f) &= F(g) \circ F(f) \quad \text{for all } X,Y,Z \in \Ob(\calC) \text{ and all } f: X \to Y, g: Y \to Z \text{ in } \mathcal{C}.
  \end{align*}

  Functors as defined above are also referred to as \hldef{covariant functors} to distinguish them from contravariant functors

  \item A \hldef{contravariant functor from $\calC$ to $\calD$} refers to a covariant functor $F:\calC^{\op} \to \calD$. Equivalently, such a functor consists of 
  \begin{itemize}
    \item For each object $X$ in $\mathcal{C}$, an object $F(X)$ in $\mathcal{D}$.
    \item For each morphism $f: X \to Y$ in $\mathcal{C}$, a morphism $F(f): F(Y) \to F(X)$ in $\mathcal{D}$,
  \end{itemize}
  such that:
  \begin{align*}
    F(\mathrm{id}_X) &= \mathrm{id}_{F(X)} \quad \text{for all objects } X \text{ in } \mathcal{C}, \\
    F(g \circ f) &= F(f) \circ F(g) \quad \text{for all } X,Y,Z \in \Ob(\calC) \text{ and all } f: X \to Y, g: Y \to Z \text{ in } \mathcal{C}.
  \end{align*}
  \TextIfExists{definition:presheaf_on_a_category}{A synonym for a ``contravariant functor from $\calC$ to $\calD$'' is a ``\CrefAndHyperrefIfExist{definition:presheaf_on_a_category}{presheaf on $\calC$ with values in $\calD$}''.}
  
\end{enumerate}
Note that declarations such as ``Let $F: \calC^{\op} \to \calD$ be a contravariant functor'' can be common; such declarations usually mean ``Let $F$ be a contravariant functor from $\calC$ to $\calD$'' as opposed to ``Let $F$ be a contravariant functor from $\calC^{\op}$ to $\calD$''. further note that a contravariant functor from $\calC$ to $\calD$ is equivalent to a covariant functor from $\calC^{\op}$ to $\calD$.
\end{definition}


\begin{definition} \label{definition:natural_transformation_between_functors_between_categories}
Let $\mathcal{C}$ and $\mathcal{D}$ be \CrefAndHyperrefIfExist{definition:category}{(large) categories}. 
Let $F, G : \mathcal{C} \to \mathcal{D}$ be \CrefAndHyperrefIfExist{definition:functor_between_categories}{functors}.

A \hldef{natural transformation $\eta$ between $F$ and $G$} is a family of morphisms $\eta_X: F(X) \to G(X)$ in $\mathcal{D}$, one for each object $X$ in $\mathcal{C}$, such that for every morphism $f: X \to Y$ in $\mathcal{C}$,
\begin{align*}
G(f) \circ \eta_X = \eta_Y \circ F(f)
\end{align*}
in $\mathcal{D}$. In other words, the following diagram commutes:
\begin{center}
\begin{tikzcd}
    F(X) \arrow[r, "F(f)"] \arrow[d, "\eta_X"']
    & F(Y) \arrow[d, "\eta_Y"] \\
    G(X) \arrow[r, "G(f)"']
    & G(Y)
\end{tikzcd}
\end{center}

We write such a natural transformation by \hl{$\eta: F \Rightarrow G$}.

If $\eta_X$ is an \CrefAndHyperrefIfExist{definition:isomorphism_in_a_category}{isomorphism} for all objects $X$ of $\calC$, then $\eta$ is said to be a \hldef{natural isomorphism}.
\end{definition}

\begin{definition}[Locally small category] \label{definition:locally_small_category}
A \hyperrefIfExists{definition:category}{(large) category}\CrefIfExists{definition:category} $\mathcal{C}$ is called a \hldef{locally small category} if for every pair of objects $X, Y \in \operatorname{Ob}(\mathcal{C})$, the collection $\operatorname{Hom}_{\mathcal{C}}(X,Y)$ of morphisms between them is a (\CrefAndHyperrefIfExist{definition:small_set}{small}) \emph{set} (as opposed to a proper class). In other words, each hom-class is a set and may even be called a \hldef{hom-set}.

In some contexts, a locally small category may simply be called a \hldef{category}, especially when genuinely large categories are not considered.

A category $\mathcal{C}$ is called a \hldef{small category} if it is a locally small category and the class $\operatorname{Ob}(\mathcal{C})$ of objects is a set.

\TextIfExists{definition:grothendieck_universe}{
Given a \hyperrefIfExists{definition:grothendieck_universe}{universe}\CrefIfExists{definition:grothendieck_universe} $U$, we can define the notion of a \hldef{$U$-locally small category} and of a \hldef{$U$-small category} similarly. More explicitly, 
\begin{enumerate}
    \item a $U$-locally small category is a category such that for every pair of objects $X, Y \in \operatorname{Ob}(\mathcal{C})$, the collection $\operatorname{Hom}_{\mathcal{C}}(X,Y)$ of morphisms between them is a $U$-set.
    \item a $U$-small category is a category such that $\operatorname{Ob}(\mathcal{C})$ is a $U$-set and for every pair of objects $X, Y \in \operatorname{Ob}(\mathcal{C})$, the collection $\operatorname{Hom}_{\mathcal{C}}(X,Y)$ of morphisms between them is a $U$-set; in particular the collection of all objects and morhpisms in a $U$-small category is a $U$-set.
\end{enumerate}
}
\end{definition}

\begin{remark}
    Many ``concrete'' categories considered in ``classical mathematics'' or outside of more ``abstract'' category theory tend to be locally small. For example, the categories of sets, groups, $R$-modules, vector spaces, topological spaces, schemes, manifolds, sheaves on ``small enough'' sites are all locally small.
\end{remark}
\begin{lemma} \label{lemma:category_of_presheaves_on_a_small_category_of_locally_small_value_is_locally_small}
    Let $\calC$ be a \hyperrefIfExists{definition:locally_small_category}{small category}\CrefIfExists{definition:locally_small_category} (resp. $U$-small category where $U$ is some \hyperrefIfExists{definition:grothendieck_universe}{universe}\CrefIfExists{definition:grothendieck_universe}) and let $\calA$ be a \CrefAndHyperrefIfExist{definition:locally_small_category}{locally small} category (resp. $U$-locally small category). The \hyperrefIfExists{definition:presheaf_on_a_category}{presheaf category $\PreShv(\calC, \calA)$}\CrefIfExists{definition:presheaf_on_a_category} is locally small (resp. $U$-locally small).
\end{lemma}
\begin{proof}
    A morphism $\calF \to \calG$ in $\PreShv(\calC, \calA)$ is a \hyperrefIfExists{definition:natural_transformation_between_functors_between_categories}{natural transformation}\CrefIfExists{definition:natural_transformation_between_functors_between_categories} of the functors $\calF, \calG: \calC^{\op} \to \calA$. Such a natural transformation is encoded by a family $(\eta_C)_C$ of morphisms (satisfying certain conditions) $\eta_C: \calF(C) \to \calG(C)$ in $\calA$ over objects $C$ of $\calC^{\op}$. The product $\prod_{C \in \Ob \calC^{\op}} \Hom_{\calA}(\calF(C), \calG(C))$ is a product of ($U$-small) sets indexed by a ($U$-small) set, and the collection of natural transformations is a subset of this set. Therefore, $\Hom_{\PreShv(\calC, \calA)}(\calF, \calG)$ is a ($U$-small) set.  
\end{proof}



\begin{definition}[Full subcategory] \label{definition:full_subcategory_of_a_category}
    Let $\mathcal{C}$ be a \CrefAndHyperrefIfExist{definition:category}{(large) category}. A \hldef{full subcategory} $\mathcal{D}$ of $\mathcal{C}$ is a \CrefAndHyperrefIfExist{definition:subcategory_of_a_category}{subcategory} such that for every pair of objects $X, Y \in \mathrm{Ob}(\mathcal{D})$, the morphism classes coincide:
    $$\mathrm{Hom}_{\mathcal{D}}(X,Y) = \mathrm{Hom}_{\mathcal{C}}(X,Y).$$
    In other words, a full subcategory includes all morphisms between its objects that exist in the ambient category $\mathcal{C}$.
\end{definition}


\begin{definition} \label{definition:full_and_faithful_functor_between_locally_small_categories}

Let $\mathcal{C}$ and $\mathcal{D}$ be \CrefAndHyperrefIfExist{definition:category}{(large)) categories}. Let $F : \mathcal{C} \to \mathcal{D}$ be a \CrefAndHyperrefIfExist{definition:functor_between_categories}{functor}. 
\begin{enumerate}
    \item $F$ is called \hldef{full} if for every pair of objects $x,y \in \mathrm{Ob}(\mathcal{C})$, the induced rule/assignment/class function
    $$ F_{x,y} : \Hom_\mathcal{C}(x,y) \to \Hom_\mathcal{D}(F(x), F(y)) $$
    on Hom-collections is ``surjective'', i.e. for all morphisms $g:F(x) \to F(y)$, there exists some morphism $f: x \to y$ such that $F(f) = g$. 

    \item $F$ is called \hldef{faithful} if for every pair of objects $x,y \in \mathrm{Ob}(\mathcal{C})$, 
    the induced class function (assignment)
    $$ F_{x,y} : \mathrm{Hom}_\mathcal{C}(x,y) \to \mathrm{Hom}_\mathcal{D}(F(x), F(y)) $$
    on Hom-collections is ``injective'', i.e., for any morphisms $f_1, f_2 \in \mathrm{Hom}_\mathcal{C}(x,y)$, 
    if $F(f_1) = F(f_2)$ in $\mathrm{Hom}_\mathcal{D}(F(x), F(y))$, then $f_1 = f_2$.

    \item $F$ is called \hldef{fully faithful} if it is both full and faithful.
\end{enumerate}

\end{definition}

\begin{definition} \label{definition:equivalence_of_categories}
An \hldef{equivalence of categories} between two \CrefAndHyperrefIfExist{definition:category}{(large) categories} $\mathcal{C}$ and $\mathcal{D}$ consists of a pair of \CrefAndHyperrefIfExist{definition:functor_between_categories}{functors}
$$F : \mathcal{C} \to \mathcal{D} \quad \text{and} \quad G : \mathcal{D} \to \mathcal{C}$$
together with \CrefAndHyperrefIfExist{definition:natural_transformation_between_functors_between_categories}{natural isomorphisms}
$$\eta : \mathrm{Id}_{\mathcal{C}} \xrightarrow{\sim} G \circ F \quad \text{and} \quad \epsilon : F \circ G \xrightarrow{\sim} \mathrm{Id}_{\mathcal{D}}.$$
\CrefIfExists{definition:identity_functor_on_a_category} Such functors $F$ and $G$ may be called \hldef{(natural) inverses of each other}.

When $\calC$ and $\calD$ are \CrefAndHyperrefIfExist{definition:locally_small_category}{locally small categories}, $F$ is an equivalence of categories if and only if $F$ is \CrefAndHyperrefIfExist{definition:full_and_faithful_functor_between_locally_small_categories}{fully faithful} and \CrefAndHyperrefIfExist{definition:essentially_surjective_functor_between_categories}{essentially surjective}
\end{definition}
\begin{definition} \label{definition:essentially_small_category}
A category $\mathcal{C}$ is called \hldef{essentially small} if it is \CrefAndHyperrefIfExist{definition:equivalence_of_categories}{equivalent} to a \CrefAndHyperrefIfExist{definition:locally_small_category}{small category}, i.e., there exists a small category $\mathcal{D}$ and an equivalence of categories
$$F : \mathcal{D} \to \mathcal{C}.$$
Note that an essentially small category is necessarily \CrefAndHyperrefIfExist{definition:locally_small_category}{locally small}.
\end{definition}
\begin{definition}[Product Category of a Family of Categories] \label{definition:product_category_of_a_family_of_categories}

    Let $\{\mathcal{C}_i\}_{i \in I}$ be a family of \CrefAndHyperrefIfExist{definition:category}{(large) categories} indexed by a class $I$. The \hldef{product category of the family}, denoted
        $$\hlin{\prod_{i\in I} \mathcal{C}_i},$$
        is the very large category \TODO{define very large categories} defined as follows:
        \begin{itemize}
            \item The class of objects is
            $$\mathrm{Ob}\Big(\prod_{i\in I} \mathcal{C}_i\Big) = \prod_{i\in I} \mathrm{Ob}(\mathcal{C}_i),$$
            i.e., an object is a family $(A_i)_{i\in I}$ with $A_i \in \mathrm{Ob}(\mathcal{C}_i)$.

            \item For two objects $(A_i)_i$ and $(B_i)_i$, the morphism class is
            $$\mathrm{Hom}_{\prod_{i\in I} \mathcal{C}_i}((A_i)_i,(B_i)_i) = \prod_{i\in I} \mathrm{Hom}_{\mathcal{C}_i}(A_i,B_i).$$
            In other words, a morphism $(f_i)_i : (A_i)_i \to (B_i)_i$ consists of morphisms $f_i: A_i \to B_i$ in each $\mathcal{C}_i$.

            \item For morphisms $(f_i)_i : (A_i)_i \to (B_i)_i$ and $(g_i)_i : (B_i)_i \to (C_i)_i$, composition is defined componentwise:
            $$(g_i)_i \circ (f_i)_i = (g_i \circ_i f_i)_i.$$

            \item For each object $(A_i)_i$, the identity morphism is given by the family $$(\mathrm{id}_{A_i})_i.$$
        \end{itemize}
        If $I$ is a set, then $\prod_{i \in I} \calC_i$ is a large category. If $I$ is a set and if each $\calC_i$ is \CrefAndHyperrefIfExist{definition:locally_small_category}{locally small}, then $\prod_{i \in I} \calC_i$ is locally small. 

        In case that $I$ is finite, the notation of \hl{$\times$} may be used for product categories, e.g. \hl{$\calC_i \times \calC_j$} denotes the product of two categories $\calC_i \times \calC_j$.

        \TODO{ordinal, $U_\alpha$}
        If $\alpha$ is an ordinal such that $\calC_i$ and $I$ are $U_\alpha$-large (i.e. they live in $U_{\alpha+1}$), then $\prod_{i \in I} \calC_i$ is $U_{\alpha+1}$-large.

\end{definition}
\begin{definition}[n-ary (Multivariable) Functor] \label{definition:n_ary_functor}
Let $I$ be a finite set with $|I| = n$, and let $\{\mathcal{C}_i\}_{i \in I}$ be \CrefAndHyperrefIfExist{definition:category}{(large) categories}, together with another category $\mathcal{D}$. An \hldef{n-ary functor} (also called a \hldef{multivariable functor}, a \hldef{multivariate functor}, or a \hldef{multifunctor} ) from the categories $\{\mathcal{C}_i\}_{i\in I}$ to $\mathcal{D}$ is a \CrefAndHyperrefIfExist{definition:functor_between_categories}{functor}
$$F : \prod_{i \in I} \mathcal{C}_i \to \mathcal{D}.$$ 
\CrefIfExists{definition:product_category_of_a_family_of_categories} That is, $F$ assigns:
\begin{itemize}
    \item to each object $((A_i)_{i \in I})$ in $\prod_{i \in I} \mathcal{C}_i$, an object $F((A_i)_{i \in I})$ in $\mathcal{D}$,
    \item to each morphism $((f_i)_{i \in I}) : (A_i)_i \to (B_i)_i$, a morphism $F((f_i)_i) : F((A_i)_i) \to F((B_i)_i)$ in $\mathcal{D}$,
\end{itemize}
so that $F$ preserves identities and composition componentwise. 
For instance, a \hldef{bifunctor} is an $n$-ary functor when $n = 2$, a \hldef{ternary functor/trifunctor} is an $n$-ary functor when $n = 3$, etc.

\end{definition}

\begin{lemma} \label{lemma:Hom_functor_on_an_enriched_category_is_contravariant_in_the_first_argument_and_covariant_in_the_second}
    Let $\calC$ be a \CrefAndHyperrefIfExist{definition:category_enriched_in_a_monoidal_category}{category enriched} in a \CrefAndHyperrefIfExist{definition:monoidal_category}{monoidal category} $(\calV, \otimes, 1)$ and assume that $\calV$ is closed under finite \CrefAndHyperrefIfExist{definition:product_and_coproduct_of_objects_in_a_category}{products}. The objects $\underline{\Hom}_{\calC}(X,Y) \in \Ob(\calV)$ describe an \CrefAndHyperrefIfExist{definition:n_ary_functor_of_categories_enriched_in_a_monoidal_category}{enriched bifunctor}
    $$\underline{\Hom}_{\calC}(-,-): \calC^{\op} \times \calC \to \calV, \quad (A,B) \mapsto \underline{\Hom}_{\calC}(A,B)$$
    \CrefIfExists{definition:opposite_of_a_category_enriched_in_a_monoidal_category} of enriched categories.

    In particular, when $\calV = \Sets$, we have a \CrefAndHyperrefIfExist{definition:n_ary_functor}{bifunctor}
    $$\Hom_{\calC}(-,-): \calC^{\op} \times \calC \to \Sets, \quad (A,B) \mapsto \Hom_{\calC}(A,B).$$
    \CrefIfExists{definition:opposite_category_of_a_category}.

    In other words, $\underline{\Hom}$ and $\Hom$ are \CrefAndHyperrefIfExist{definition:functor_between_categories}{contravariant} in the first variable and covariant in the second.
\end{lemma}


\subsection{Sites and sheaves}


\begin{definition}[Ringed space] \label{definition:ringed_space}
    A \hldef{ringed space} is a pair $(X, \mathcal{O}_X)$ where
    \begin{itemize}
        \item $X$ is a \CrefAndHyperrefIfExist{definition:topological_space}{topological space}, and
        \item $\mathcal{O}_X$ is a \CrefAndHyperrefIfExist{definition:sheaf_on_a_topological_space_valued_in_a_category_with_a_terminal_object}{sheaf of} \CrefAndHyperrefIfExist{definition:commutative_ring}{commutative rings} on $X$.
    \end{itemize}
    Equivalently, a ringed space is a \CrefAndHyperrefIfExist{definition:ringed_site}{ringed site} where the site is the \CrefAndHyperrefIfExist{definition:site_of_opens_on_a_topological_space}{site of opens} of the topological space $X$
    The sheaf $\calO_X$ may be suppressed from the notation and only $X$ may be used to denote a ringed space. The sheaf \hl{$\mathcal{O}_X$}, also commonly denoted by $\mathscr{O}_X$, is called the \hldef{structure sheaf of $X$}.
\end{definition}

\begin{definition} \label{definition:module_over_a_sheaf_of_rings_on_a_site}

    \begin{enumerate}
        \item 
        Let $\mathcal{C}$ be a \CrefAndHyperrefIfExist{definition:grothendieck_topology_on_a_category_site_covering_sieve_topologically_generating_family}{site}, and let $\mathcal{A}$ and $\mathcal{B}$ be \CrefAndHyperrefIfExist{definition:sheaf_on_a_site}{sheaves} of (not necessarily commutative) \CrefAndHyperrefIfExist{definition:ring}{rings} on $\mathcal{C}$. 
        
        \begin{enumerate}
            \item 
            An \hldef{$(\mathcal{A}, \mathcal{B})$-bimodule} (or a \hldef{bimodule over $(\mathcal{A}, \mathcal{B})$}) is a \CrefAndHyperrefIfExist{definition:sheaf_on_a_site}{sheaf} $\mathcal{M}$ of abelian groups on $\mathcal{C}$ equipped with a left $\mathcal{A}$-module structure given by a \CrefAndHyperrefIfExist{definition:sheaf_on_a_site}{morphism of sheaves} of sets
            $$ \lambda: \mathcal{A} \times \mathcal{M} \longrightarrow \mathcal{M}, $$
            and a right $\mathcal{B}$-module structure given by a morphism of sheaves of sets
            $$ \rho: \mathcal{M} \times \mathcal{B} \longrightarrow \mathcal{M}, $$
            such that the actions are compatible. Specifically, for every object $U$ in $\mathcal{C}$, every section $m \in \mathcal{M}(U)$, every $a \in \mathcal{A}(U)$, and every $b \in \mathcal{B}(U)$, the equality
            $$ \lambda_U(a, \rho_U(m, b)) = \rho_U(\lambda_U(a, m), b) $$
            holds in $\mathcal{M}(U)$. In standard multiplicative notation where $\lambda(a,m)$ is denoted $a \cdot m$ and $\rho(m,b)$ is denoted $m \cdot b$, this condition is the associativity axiom
            $$ (a \cdot m) \cdot b = a \cdot (m \cdot b). $$

            In particular, for every object $U \in \calC$, the abelian group $\calM(U)$ has the structure of an \CrefAndHyperrefIfExist{definition:module_of_a_ring}{$\calA(U)-\calB(U)$-bimodule}.

            \item Let $\mathcal{M}$ and $\mathcal{N}$ be $(\mathcal{A}, \mathcal{B})$-bimodules. A \hldef{homomorphism of $(\mathcal{A}, \mathcal{B})$-bimodules} (or an \hldef{$(\mathcal{A}, \mathcal{B})$-linear morphism}) is a morphism of sheaves of abelian groups $f: \mathcal{M} \to \mathcal{N}$ such that for every object $U$ of $\mathcal{C}$, every section $m \in \mathcal{M}(U)$, every $a \in \mathcal{A}(U)$, and every $b \in \mathcal{B}(U)$, the following compatibility conditions hold:
            $$ f_U(a \cdot m) = a \cdot f_U(m) \quad \text{and} \quad f_U(m \cdot b) = f_U(m) \cdot b. $$


        \end{enumerate}

        \noindent We denote the category of $(\mathcal{A}, \mathcal{B})$-bimodules, with morphisms being morphisms of sheaves of abelian groups compatible with both the left $\mathcal{A}$-action and the right $\mathcal{B}$-action, by
        \hl{$ \mathcal{A}\text{-}\mathcal{B}\text{-}\mathsf{Mod} $}
        or sometimes by
        \hl{$ {}_{\mathcal{A}}\mathsf{Mod}_{\mathcal{B}} $}
        \TODO{talk about how bimodules can be identifies with left/right modules}

        \item 

        Let $(\mathcal{C}, J)$ be a \CrefAndHyperrefIfExist{definition:grothendieck_topology_on_a_category_site_covering_sieve_topologically_generating_family}{site}. Let $\mathcal{O}$ be a \CrefAndHyperrefIfExist{definition:sheaf_on_a_site}{sheaf of (not necessarily commutative) rings on $(\mathcal{C}, J)$}, i.e. $((\calC, J), \calO)$ is a \CrefAndHyperrefIfExist{definition:ringed_site}{ringed site}.  

        \begin{enumerate}
            \item An \hldef{(left/right/two-sided) $\mathcal{O}$-module} consists of the following data:
            \begin{itemize}
                \item A sheaf $\mathcal{F}$ of abelian groups on $(\mathcal{C}, J)$,
            \item for every object $U \in \mathcal{C}$, the structure of an (left/right/two-sided) $\mathcal{O}(U)$-module on $\mathcal{F}(U)$,
            \end{itemize}
            such that for every morphism $f: V \to U$ in $\mathcal{C}$, the restriction map 
            $$\rho_{U,V}: \mathcal{F}(U) \to \mathcal{F}(V)$$ 
            is $\mathcal{O}(U)$-linear when the $\mathcal{O}(U)$-action on $\mathcal{F}(V)$ is defined via the natural ring homomorphism 
            $$\mathcal{O}(U) \to \mathcal{O}(V)$$
            induced by $f$.


            \item Let $\mathcal{F}$ and $\mathcal{G}$ be \CrefAndHyperrefIfExist{definition:module_over_a_sheaf_of_rings_on_a_site}{$\mathcal{O}$-modules}.

            A \hldef{morphism of $\mathcal{O}$-modules} $\varphi: \mathcal{F} \to \mathcal{G}$ is a \CrefAndHyperrefIfExist{definition:sheaf_on_a_site}{morphism of sheaves} of abelian groups such that, for every object $U \in \mathcal{C}$, the component map
            $$\varphi_U : \mathcal{F}(U) \to \mathcal{G}(U)$$
            is $\mathcal{O}(U)$-linear, i.e. it satisfies
            $$\varphi_U(r \cdot s) = r \cdot \varphi_U(s) \quad \text{for all } r \in \mathcal{O}(U), \, s \in \mathcal{F}(U).$$

            The collection of all $\mathcal{O}$-modules together with their morphisms of $\mathcal{O}$-modules forms the \hldef{category of $\mathcal{O}$-modules}, denoted \hl{$\mathbf{Mod}(\mathcal{O})$}.

            \TextIfExists{definition:algebra_over_a_sheaf_of_rings_on_a_site}{See also \Cref{definition:algebra_over_a_sheaf_of_rings_on_a_site}.}
        \end{enumerate}

        \noindent In case that a \CrefAndHyperrefIfExist{definition:sheafification_functor_on_a_site}{sheafification functor} 
        $$\PreShv(\calC, \mathbf{Rings}) \to \Shv(\calC, \mathbf{Rings})$$ 
        exists, a left, right, two-sided $\calO$-module (and morphisms thereof) is equivalent to a $(\calO,\bbZ)$-bimodule, $(\bbZ,\calO)$-bimodule, and $(\calO, \calO)$-bimodule (and morphisms thereof) respectively, where $\bbZ$ is the \CrefAndHyperrefIfExist{definition:constant_sheaf_on_a_site_with_sheafification}{constant sheaf} of the integer ring $\bbZ$.

\end{enumerate}


\end{definition}


% See Also
% theorem:category_of_modules_over_a_sheaf_of_rings_on_a_site_on_an_essentially_small_category_has_enough_injectives

\begin{definition}[Sheaf associated to a presheaf] \label{definition:sheafification_of_a_presheaf_on_a_topological_space_valued_in_a_category_admitting_direct_colimits}
    Let $X$ be a topological space, and let $\mathcal{D}$ be a \CrefAndHyperrefIfExist{definition:category}{category} admitting \CrefAndHyperrefIfExist{definition:projective_and_inductive_limits_in_categories}{direct colimits} (e.g. the category of sets, groups, abelian groups, modules over rings, or vector spaces over fields). Let $\mathcal{P}$ be a \CrefAndHyperrefIfExist{definition:presheaf_on_a_topological_space}{presheaf on $X$ with values in $\mathcal{D}$}.  

    The \hldef{sheaf associated to the presheaf $\mathcal{P}$} or the \hldef{sheaffification of the presheaf $\calP$}, denoted \hl{$\mathcal{P}^+$} or sometimes by \hl{$a\calP$}, is a sheaf on $X$ together with a morphism of presheaves
    $$\eta: \mathcal{P} \to \mathcal{P}^+,$$
    satisfying the following universal property:  
    for every sheaf $\mathcal{F}$ on $X$ (valued in $\mathcal{D}$), any morphism of presheaves
    $$\varphi: \mathcal{P} \to \mathcal{F}$$
    factors uniquely through $\eta$, i.e., there exists a unique morphism of sheaves
    $$\widetilde{\varphi}: \mathcal{P}^+ \to \mathcal{F}$$
    such that
    $$\varphi = \widetilde{\varphi} \circ \eta.$$

    Concretely, $\mathcal{P}^+$ can be constructed by assigning to each open set $U \subseteq X$ the set (or object in $\mathcal{D}$)
        \[
        \mathcal{P}^+(U) := \left\{ 
        s = (s_x)_{x \in U} \in \prod_{x \in U} \mathcal{P}_x \ 
        \middle|\ 
        \begin{aligned}
        &\forall x \in U, \\
        &\exists \text{ an open } V \subseteq U \text{ with } x \in V, \\
        &\exists t \in \mathcal{P}(V) \text{ such that } \\
        &\quad \forall y \in V, s_y = t_y
        \end{aligned}
        \right\}.
        \]
    where $\mathcal{P}_x$ is the \CrefAndHyperrefIfExist{definition:stalk_of_a_presheaf_on_a_topological_space_at_a_point}{stalk} of $\mathcal{P}$ at $x$, and $t_y$ is the \CrefAndHyperrefIfExist{definition:stalk_of_a_presheaf_on_a_topological_space_at_a_point}{germ} of $t$ at $y$. In particular, $\calP^+$ exists.

    It is noteworthy that the assignment $\calP \mapsto \calP^+$ is a functor
    $$\PreShv(X, \calD) \to \Shv(X, \calD).$$
    (\Cref{definition:categories_of_presheaves_and_sheaves_on_a_topological_space_valued_in_a_category}) and that this functor is left adjoint to the inclusion functor
    $$\Shv(X, \calD) \hookrightarrow \PreShv(X, \calD)$$
    \TextIfExists{definition:sheafification_functor_on_a_site}{Equivalently, the assignment $\calP \mapsto \calP^+$ is the sheafification functor as defined in \Cref{definition:sheafification_functor_on_a_site}.}
\end{definition}

% See Also
% theorem:sheafification_of_a_presheaf_of_sets_on_a_small_enough_site 
\begin{definition} \label{definition:sheafification_functor_on_a_site}
    Let $\calC$ be a \CrefAndHyperrefIfExist{definition:grothendieck_topology_on_a_category_site_covering_sieve_topologically_generating_family}{site} and let $\calA$ be a \CrefAndHyperrefIfExist{definition:category}{(large) category}.

    Assuming that the \CrefAndHyperrefIfExist{definition:presheaf_on_a_category}{presheaf} category $\PreShv(\calC, \calA)$ (and hence the \CrefAndHyperrefIfExist{definition:sheaf_on_a_site}{sheaf} category $\Shv(\calC, \calA)$) is \CrefAndHyperrefIfExist{definition:locally_small_category}{locally small} (or $U$-locally small if a \CrefAndHyperrefIfExist{definition:grothendieck_universe}{Grothendieck universe} $U$ is available), a \hldef{sheafification functor} refers to a functor
    $$a: \PreShv(\calC, \calA) \to \Shv(\calC, \calA) $$
    that is \CrefAndHyperrefIfExist{definition:adjoint_functors_between_categories_unit_counit_of_adjoint_functors}{left adjoint} to the inclusion functor 
    $$i:\Shv(\calC, \calA) \hookrightarrow \PreShv(\calC, \calA)  .$$
    If such a sheafification functor exists, then it is unique up to unique natural isomorphism. Given a presheaf $P$, the sheafification $a(P)$ is also sometimes called the \hldef{sheaf associated to $P$}.
    \TextIfExists{theorem:sheafification_of_a_presheaf_of_sets_on_a_small_enough_site}{See \Cref{theorem:sheafification_of_a_presheaf_of_sets_on_a_small_enough_site} for common conditions under which sheafification exists.} 
\end{definition}

% See Also
%theorem:sheafification_of_a_presheaf_of_sets_on_a_small_enough_site
\begin{theorem}{cf. {\cite[Expos\'e II, Th\'eor\`eme 3.4]{SGA4_I}}} \label{theorem:sheafification_of_a_presheaf_of_sets_on_a_small_enough_site}
    \begin{enumerate}
        \item Let $U$ be a universe. Let $\calC$ be a \hyperrefIfExists{definition:grothendieck_topology_on_a_category_site_covering_sieve_topologically_generating_family}{$U$-site}\CrefIfExists{definition:grothendieck_topology_on_a_category_site_covering_sieve_topologically_generating_family}. A \CrefAndHyperrefIfExist{definition:sheafification_of_a_presheaf_on_a_topological_space_valued_in_a_category_admitting_direct_colimits}{sheafification functor}
        $$a: \Shv(\calC, \USets) \to \PreShv(\calC, \USets).$$
        exists. 
        % The inclusion functor 
        % $$i: \PreShv(\calC, \USets) \hookrightarrow \Shv(\calC, \USets)$$
        % has a \hyperrefIfExists{definition:adjoint_functors_between_categories_unit_counit_of_adjoint_functors}{left adjoint functor}\CrefIfExists{definition:adjoint_functors_between_categories_unit_counit_of_adjoint_functors}

        \item Let $\calC$ be a site whose underlying category is \CrefAndHyperrefIfExist{definition:locally_small_category}{locally small} and which has a \CrefAndHyperrefIfExist{definition:grothendieck_topology_on_a_category_site_covering_sieve_topologically_generating_family}{topologically generating family} that is a set (rather than a proper class). A sheafification functor 
        $$a: \Shv(\calC, \Sets) \to \PreShv(\calC, \Sets)$$
        exists.

        \item (see e.g. {\cite[3]{nlab:sheafification}}) Let $(\calC, J)$ be a \CrefAndHyperrefIfExist{definition:grothendieck_topology_on_a_category_site_covering_sieve_topologically_generating_family}{site} on an \CrefAndHyperrefIfExist{definition:essentially_small_category}{essentially small category} $\calC$. Suppose that the category $\calA$ is \CrefAndHyperrefIfExist{definition:complete_and_cocomplete_category}{complete, cocomplete}, that small \CrefAndHyperrefIfExist{definition:projective_and_inductive_limits_in_categories}{filtered colimits} in $\calA$ are exact, and that $\calA$ satisfies the IPC-property. A \CrefAndHyperrefIfExist{definition:sheafification_functor_on_a_site}{sheafification functor} 
        $$a: \PreShv(\calC, \calA) \to \Shv(\calC, \calA) $$
        exists.
        \TODO{IPC-property, exactess in this context.}

        \TODO{state as a fact that these categories are complete, cocomplete, with small filtered colimits that are exact}
        This is true for instance of $\calA = \mathbf{Set}, \mathbf{Grp}$, $k-\mathbf{Alg}$ for a field $k$, or $\mathbf{Mod}_R$ for a \CrefAndHyperrefIfExist{definition:ring}{(not necessarily commutative unital) ring $R$}.
    \end{enumerate}
\end{theorem}
\begin{remark}
    If the presheaf is valued in nice ``algebraic category'', e.g. groups, abelian groups, rings, modules over a ring, etc., then the sheafification is again valued in that category. \TODO{Make this more precise.}
\end{remark}
\begin{lemma} \label{lemma:final_object_is_projective_limit_of_empty_diagram}
    Let $C$ be a \CrefAndHyperrefIfExist{definition:category}{category}. A \CrefAndHyperrefIfExist{definition:initial_final_zero_objects_of_a_category}{final object}, if it exists, of $C$ is the \CrefAndHyperrefIfExist{definition:limit_and_colimit_of_a_diagram_in_a_category}{limit} of the empty diagram. In particular, any category that is \CrefAndHyperrefIfExist{definition:complete_and_cocomplete_category}{closed} under finite limits or \CrefAndHyperrefIfExist{definition:product_and_coproduct_of_objects_in_a_category}{finite products} has a final object.
\end{lemma}
% \begin{definition}[Topos] \label{definition:topos}
%     There are a multitude of notions of topos. Here are some that we consider; more notions may be added later.
%     \begin{enumerate}
%         \item A \hldef{(sheaf/Grothendieck) topos} is a \CrefAndHyperrefIfExist{definition:category}{category} \CrefAndHyperrefIfExist{definition:equivalence_of_categories}{equivalent} to the category of \CrefAndHyperrefIfExist{definition:sheaf_on_a_site}{sheaves} of sets on some \CrefAndHyperrefIfExist{definition:grothendieck_topology_on_a_category_site_covering_sieve_topologically_generating_family}{site}. That is, there exists a site $(C, J)$ such that the category is equivalent to $\operatorname{Sh}(C, J)$, the category of sheaves of sets on $(C, J)$.
%         \item Let $U$ be a universe. A \hldef{$U$-(sheaf )topos} is a category equivalent to the category of \hyperrefIfExists{definition:sheaf_on_a_site}{$U$-sheaves}\CrefIfExists{definition:sheaf_on_a_site} (valued in $U$-sets) \cite[Expos\'e IV D\'efinition 1.1]{SGA4_I}

%         \item An \hldef{elementary topos} is a cateogry which has all finite \CrefAndHyperrefIfExist{definition:limit_and_colimit_of_a_diagram_in_a_category}{limits}, is cartesian closed, and has a subobject classifier \TODO{cartesian closed, subobject classifier}
%     \end{enumerate}
% \end{definition}

\begin{definition}[Topos] \label{definition:topos}
    There are multiple notions of a topos depending on the context (geometric vs. logical).
    \begin{enumerate}
        \item A \hldef{Grothendieck topos} (or \hldef{sheaf topos}) is a \CrefAndHyperrefIfExist{definition:category}{category} \CrefAndHyperrefIfExist{definition:equivalence_of_categories}{equivalent} to the category of \CrefAndHyperrefIfExist{definition:sheaf_on_a_site}{sheaves} of sets on a \hldef{small} \CrefAndHyperrefIfExist{definition:grothendieck_topology_on_a_category_site_covering_sieve_topologically_generating_family}{site}. That is, there exists a small site $(\mathcal{C}, J)$ such that the category is equivalent to $\operatorname{Sh}(\mathcal{C}, J)$.
        
        \item Let $\mathscr{U}$ be a \hyperrefIfExists{definition:grothendieck_universe}{universe}\CrefIfExists{definition:grothendieck_universe}. A \hldef{$\mathscr{U}$-topos} is a category equivalent to the category of sheaves of sets on a $\mathscr{U}$-small site $(\mathcal{C}, J)$, where the sheaves take values in the category of $\mathscr{U}$-sets ($\mathbf{Set}_{\mathscr{U}}$). \cite[Expos\'e IV D\'efinition 1.1]{SGA4_I}

        \item An \hldef{elementary topos} is a category which has all finite \CrefAndHyperrefIfExist{definition:limit_and_colimit_of_a_diagram_in_a_category}{limits}, is \CrefAndHyperrefIfExist{definition:cartesian_closed_category}{cartesian closed}, and has a \CrefAndHyperrefIfExist{definition:subobject_classifier_in_a_category_with_a_final_object}{subobject classifier}.
    \end{enumerate}
    \textit{Remark:} Every Grothendieck topos is an elementary topos, but the converse is not true (e.g., the category of finite sets is an elementary topos but not a Grothendieck topos).
\end{definition}


% {\cite[Expos\'e IV D\'efinition 1.1]{SGA4_I}}
% Let $\scrU$ be a fixed universe. A \hldef{$\scrU$-topos}, or simply \hldef{topos} if there is no confusion, $E$ is a category that is equivalent to the category $\Shv(T)$ of sheaves of sets on a fixed site $T$ in $\scrU$.
\begin{lemma} \label{lemma:topos_has_a_final_object}
    Let $\scrU$ be a universe and let $T$ be a $\scrU$-site. The category $\Shv(T)$ of sheaves on $T$ has a final object.
\end{lemma}



\subsection{Miscellaneous categorical constructions and definitions}


\begin{definition}[Monomorphism and Epimorphism in Categories] \label{definition:monomorphism_and_epimorphism_in_categories}
Let $\mathcal{C}$ be a \CrefAndHyperrefIfExist{definition:category}{category}. For objects $A, B \in \mathcal{C}$, let $f: A \to B$ be a morphism in $\mathcal{C}$.  
\begin{itemize}
    \item The morphism $f$ is called a \hldef{monomorphism} (or a \hldef{monic morphism}) if for every object $X$ and every pair of morphisms $g_1, g_2 : X \to A$, the equality $f \circ g_1 = f \circ g_2$ implies $g_1 = g_2$.  
    \item The morphism $f$ is called an \hldef{epimorphism} (or an \hldef{epic morphism}) if for every object $Y$ and every pair of morphisms $h_1, h_2: B \to Y$, the equality $h_1 \circ f = h_2 \circ f$ implies $h_1 = h_2$.  
\end{itemize}
\end{definition}



\begin{definition} \label{definition:kernel_and_cokernel_of_a_morphism_in_a_category}
Let $\mathcal{C}$ be a \CrefAndHyperrefIfExist{definition:category}{(large)} \CrefAndHyperrefIfExist{definition:pointed_category}{pointed category}, i.e. a category with a \CrefAndHyperrefIfExist{definition:initial_final_zero_objects_of_a_category}{zero object} $0$. Let $X,Y \in \mathrm{Ob}(\mathcal{C})$ be an object and let $f: X \to Y$ be a morphism. 

\begin{enumerate}
    \item A morphism $i: K \to X$ is called the \hldef{kernel of $f$} if:
    \begin{enumerate}
        \item $f \circ i = 0$, where $0$ is the \CrefAndHyperrefIfExist{definition:zero_morphism_in_a_pointed_category}{zero morphism} $K \to Y$,
        \item for any morphism $g: Z \to X$ such that $f \circ g = 0$, there exists a unique morphism $u: Z \to K$ such that $g = i \circ u$.
    \end{enumerate}
    The kernel, if it exists, is unique up to unique \CrefAndHyperrefIfExist{definition:isomorphism_in_a_category}{isomorphism}. \hl{$\ker(f)$} denotes the object $K$ determined (up to isomorphism) by a kernel of $f$.

    \TextIfExists{definition:equalizer_and_coequalizer_of_morphisms_in_a_category}{
        Equivalently, $\ker(f)$ is the \CrefAndHyperref{definition:equalizer_and_coequalizer_of_morphisms_in_a_category}{equalizer} of $f$ and the $0$ morphism $X \to Y$.
    }

    \item a morphism $p: Y \to Q$ is called the \hldef{cokernel of $f$} if:
    \begin{enumerate}
        \item $p \circ f = 0$, where $0$ is the \CrefAndHyperrefIfExist{definition:initial_final_zero_objects_of_a_category}{zero morphism} $X \to Q$,
        \item for any morphism $g: Y \to Z$ such that $g \circ f = 0$, there exists a unique morphism $v: Q \to Z$ such that $g = v \circ p$.
    \end{enumerate}
    The cokernel, if it exists, is unique up to unique isomorphism. \hl{$\operatorname{coker}(f)$} denotes the object $Q$ determined (up to isomorphism) by a cokernel of $f$.

    \TextIfExists{definition:equalizer_and_coequalizer_of_morphisms_in_a_category}{
        Equivalently, $\coker(f)$ is the \CrefAndHyperref{definition:equalizer_and_coequalizer_of_morphisms_in_a_category}{coequalizer} of $f$ and the $0$ morphism $X \to Y$.
    }

\end{enumerate}

\end{definition}


\begin{definition} \label{definition:subobject_of_an_object_of_an_additive_category}
Let $\mathcal{C}$ be an \CrefAndHyperrefIfExist{definition:additive_category}{additive category}. Let $X \in \Ob(\calC)$ be an object. 
A \hldef{subobject of $X$} refers to a \CrefAndHyperrefIfExist{definition:monomorphism_and_epimorphism_in_categories}{monomorphism} $i: Y \hookrightarrow X$ in $\mathcal{C}$. We regard two subobjects $(Y,i)$ and $(Y',i')$ of $X$ as isomorphic if there exists an isomorphism $f: Y \to Y'$ such that $i = i' \circ f$. One often leaves the monomorphism $i$ implicit, suprressing it from the notation.
\end{definition}


\begin{definition} \label{definition:quotient_object_of_an_object_of_an_abelian_category_by_a_subobject}
Let $\mathcal{C}$ be an \CrefAndHyperrefIfExist{definition:abelian_category}{abelian category}. Let $X \in \Ob(\calC)$ be an object. Let $i: A \hookrightarrow X$ be a \CrefAndHyperrefIfExist{definition:subobject_of_an_object_of_an_additive_category}{subobject}. The \CrefAndHyperrefIfExist{definition:kernel_and_cokernel_of_a_morphism_in_a_category}{cokernel} $\pi: X \twoheadrightarrow X/A := \operatorname{coker}(i)$ is called the \hldef{quotient object of $X$ by $A$}. The object $X/A$ is determined up to canonical isomorphism.
\end{definition}


\begin{definition} \label{definition:image_coimage_of_a_morphism_in_a_category}
Let $\mathcal{C}$ be a \CrefAndHyperrefIfExist{definition:category}{category}, and let $f: A \to B$ be a morphism in $\mathcal{C}$. 
\begin{enumerate}
    \item 
    An \hldef{image of $f$} consists of an object $I \in \mathrm{Ob}(\mathcal{C})$ together with a factorization of $f$ into two morphisms
        \begin{align*}
        A \xrightarrow{e} I \xrightarrow{m} B,
        \end{align*}
    where $e$ is an \CrefAndHyperrefIfExist{definition:monomorphism_and_epimorphism_in_categories}{epimorphism} and $m$ is a \CrefAndHyperrefIfExist{definition:monomorphism_and_epimorphism_in_categories}{monomorphism}, such that for any other factorization
        \begin{align*}
        A \xrightarrow{e'} I' \xrightarrow{m'} B
        \end{align*}
    with $e'$ epi and $m'$ mono, there exists a unique isomorphism $\varphi: I \simeq I'$ satisfying $m = m'\varphi$ and $\varphi e = e'$.
    \begin{center}
    \begin{tikzcd}
        & I \arrow[dr, "m", hook] \arrow[dd, "\exists ! \varphi", dashed, "\sim"' {sloped} ] & \\
        A \arrow[ur, "e", two heads] \arrow[dr, "e'"', two heads] & & B \\
        & I' \arrow[ur, "m'"', hook] &
    \end{tikzcd}
    \end{center}
    The monomorphism $m: I \to B$ (or equivalently its subobject class) is called the \hldef{image of $f$ in $\mathcal{C}$}.

    \item
    Let $\mathcal{C}$ be a \CrefAndHyperrefIfExist{definition:category}{category}, and let $f: A \to B$ be a morphism in $\mathcal{C}$. A \hldef{coimage of $f$} consists of an object $C \in \mathrm{Ob}(\mathcal{C})$ together with a factorization of $f$ into two morphisms
        \begin{align*}
        A \xrightarrow{e} C \xrightarrow{m} B,
        \end{align*}
    where $e$ is an epimorphism and $m$ is a monomorphism, such that for any other factorization
        \begin{align*}
        A \xrightarrow{e'} C' \xrightarrow{m'} B
        \end{align*}
    with $e'$ epi and $m'$ mono, there exists a unique isomorphism $\varphi: C \simeq C'$ satisfying $m = m'\varphi$ and $\varphi e = e'$.
    \begin{center}
    \begin{tikzcd}
        & C \arrow[dr, "m", hook] \arrow[dd, "\exists ! \varphi", dashed, "\sim"' {sloped}] & \\
        A \arrow[ur, "e", two heads] \arrow[dr, "e'"', two heads] & & B \\
        & C' \arrow[ur, "m'"', hook] &
    \end{tikzcd}
    \end{center}
    The epimorphism $e: A \to C$ (or equivalently its quotient class) is called the \hldef{coimage of $f$ in $\mathcal{C}$}.


\end{enumerate}
\end{definition}


\subsection{Diagrams, systems, and limits in categories}

\import{../_excerpts}{excerpts_diagrams_systems_and_limits_in_categories.tex}

\begin{lemma} \label{lemma:bifunctors_induce_functors_to_a_functor_category_and_natural_transformations}
    Let $\calA,\calB,\calC$ be \CrefAndHyperrefIfExist{definition:category}{(large) categories} and let $F: \calA \times \calB \to \calC$\CrefIfExists{definition:product_category_of_a_family_of_categories} be a \CrefAndHyperrefIfExist{definition:n_ary_functor}{functor}.
    \begin{enumerate}
        \item There is a functor $A \to \Fun(\calB, \calC)$\CrefIfExists{definition:diagram_in_a_category_indexed_by_a_small_category} given by $A \mapsto (F(A,-): \calB \to \calC)$.

        \item Given a morphism $f: A \to A'$ in $\calA$, there is a \CrefAndHyperrefIfExist{definition:natural_transformation_between_functors_between_categories}{natural transformation} $F(f,-): F(A,-) \to F(A',-)$.
    \end{enumerate}
\end{lemma}

\Cref{theorem:right_left_adjoints_commute_with_limits_colimits} states the right adjoint functors commute with limits and left adjoint functors commute with colimits.

\begin{definition} \label{definition:comparison_morphism_of_limits_and_colimits_under_functors}
    Let $\mathcal{C}$ and $\mathcal{D}$ be \CrefAndHyperrefIfExist{definition:category}{(large) categories}, $F: \mathcal{C} \to \mathcal{D}$ a \CrefAndHyperrefIfExist{definition:functor_between_categories}{functor}, and $D: \mathcal{J} \to \mathcal{C}$ a \CrefAndHyperrefIfExist{definition:diagram_in_a_category_indexed_by_a_small_category}{diagram} indexed by a small category $\mathcal{J}$.

    \begin{enumerate}
        \item Assume that the \CrefAndHyperrefIfExist{definition:limit_and_colimit_of_a_diagram_in_a_category}{colimit} of $D$ exists in $\mathcal{C}$, denoted by $\operatorname{colim}_{\mathcal{J}} D$, with universal \CrefAndHyperrefIfExist{definition:limit_and_colimit_of_a_diagram_in_a_category}{cocone} $\eta: D \Rightarrow \Delta_{\operatorname{colim} D}$. The functor $F$ maps this to a cocone $F(\eta): F \circ D \Rightarrow \Delta_{F(\operatorname{colim} D)}$ in $\mathcal{D}$.
        
        If the colimit of the composite diagram $F \circ D$ exists in $\mathcal{D}$, denoted by $\operatorname{colim}_{\mathcal{J}} (F \circ D)$, then by the universal property of the colimit, there exists a unique morphism 
        $$\phi: \operatorname{colim}_{j \in \mathcal{J}} F(D(j)) \longrightarrow F\left(\operatorname{colim}_{j \in \mathcal{J}} D(j)\right)$$
        mediating between the cocone of the composite colimit and the image cocone $F(\eta)$. This unique morphism is called the \hldef{canonical colimit comparison morphism}.
        
        \item Dually, assume that the limit of $D$ exists in $\mathcal{C}$, denoted by $\lim_{\mathcal{J}} D$, with universal cone $\varepsilon: \Delta_{\lim D} \Rightarrow D$. The functor $F$ maps this to a cone $F(\varepsilon): \Delta_{F(\lim D)} \Rightarrow F \circ D$ in $\mathcal{D}$.
        
        If the limit of the composite diagram $F \circ D$ exists in $\mathcal{D}$, denoted by $\lim_{\mathcal{J}} (F \circ D)$, then by the universal property of the limit, there exists a unique morphism 
        $$\psi: F\left(\lim_{j \in \mathcal{J}} D(j)\right) \longrightarrow \lim_{j \in \mathcal{J}} F(D(j))$$
        mediating between the image cone $F(\varepsilon)$ and the cone of the composite limit. This unique morphism is called the \hldef{canonical limit comparison morphism}:
    \end{enumerate}
\end{definition}

\begin{definition} \label{definition:functor_preserving_limit_colimit_of_a_diagram}
    Let $F: \mathcal{C} \to \mathcal{D}$ be a \CrefAndHyperrefIfExist{definition:functor_between_categories}{functor} between \CrefAndHyperrefIfExist{definition:category}{(large) categories} $\mathcal{C}$ and $\mathcal{D}$.
    
    Let $D: \mathcal{J} \to \mathcal{C}$ be a diagram in $\mathcal{C}$ such that the limit $\lim_{\mathcal{J}} D$ exists in $\mathcal{C}$, with limiting cone $\lambda: \Delta_{\lim D} \Rightarrow D$. The functor $F$ \hldef{preserves the limit of $D$} if the image cone
    $$
    F(\lambda): \Delta_{F(\lim D)} \cong F \circ \Delta_{\lim D} \Rightarrow F \circ D
    $$
    exhibits $F(\lim_{\mathcal{J}} D)$ as a limit of the composite diagram $F \circ D$ in $\mathcal{D}$. Explicitly, the \CrefAndHyperrefIfExist{definition:comparison_morphism_of_limits_and_colimits_under_functors}{canonical comparison morphism}
    $$F(\lim_{\mathcal{J}} D) \xrightarrow{\cong} \lim_{\mathcal{J}} (F \circ D)$$
    must be an isomorphism.
    
    Dually, let $D: \mathcal{J} \to \mathcal{C}$ be a diagram such that the colimit $\operatorname{colim}_{\mathcal{J}} D$ exists in $\mathcal{C}$. The functor $F$ \hldef{preserves the colimit of $D$} if the \CrefAndHyperrefIfExist{definition:comparison_morphism_of_limits_and_colimits_under_functors}{canonical comparison morphism}
    $$\operatorname{colim}_{\mathcal{J}} (F \circ D) \xrightarrow{\cong} F(\operatorname{colim}_{\mathcal{J}} D)$$
    is an isomorphism.
\end{definition}

\begin{definition} \label{definition:continuous_cocontinuous_functor_between_categories}
    Let $F: \mathcal{C} \to \mathcal{D}$ be a \CrefAndHyperrefIfExist{definition:functor_between_categories}{functor} between \CrefAndHyperrefIfExist{definition:category}{(large) categories} $\mathcal{C}$ and $\mathcal{D}$.
    \begin{itemize}
        \item The functor $F$ is called \hldef{continuous} if it preserves all small limits that exist in $\mathcal{C}$. That is, for every small category $\mathcal{J}$ and every diagram $D: \mathcal{J} \to \mathcal{C}$ having a limit in $\mathcal{C}$, $F$ \CrefAndHyperrefIfExist{definition:functor_preserving_limit_colimit_of_a_diagram}{preserves the limit} of $D$.
        \item The functor $F$ is called \hldef{cocontinuous} if it preserves all small colimits that exist in $\mathcal{C}$. That is, for every small category $\mathcal{J}$ and every diagram $D: \mathcal{J} \to \mathcal{C}$ having a colimit in $\mathcal{C}$, $F$ \CrefAndHyperrefIfExist{definition:functor_preserving_limit_colimit_of_a_diagram}{preserves the colimit} of $D$.
    \end{itemize}
\end{definition}

\begin{theorem}[Continuity of Adjoint Functors] \label{theorem:right_left_adjoints_commute_with_limits_colimits}
    Let $\mathcal{C}$ and $\mathcal{D}$ be \CrefAndHyperrefIfExist{definition:category}{(large) categories}, and let $F: \mathcal{C} \to \mathcal{D}$ and $G: \mathcal{D} \to \mathcal{C}$ be functors forming an \CrefAndHyperrefIfExist{definition:adjoint_functors_between_categories_unit_counit_of_adjoint_functors}{adjunction} $F \dashv G$, i.e. $F$ is the left adjoint and $G$ is the right adjoint. Let $\mathcal{J}$ be a small index category.

    \begin{enumerate}
        \item \textbf{Right Adjoints Preserve Limits:}
        Let $D: \mathcal{J} \to \mathcal{D}$ be a \CrefAndHyperrefIfExist{definition:diagram_in_a_category_indexed_by_a_small_category}{diagram} such that the \CrefAndHyperrefIfExist{definition:limit_and_colimit_of_a_diagram_in_a_category}{limit} $\lim_{\mathcal{J}} D$ exists in $\mathcal{D}$. Then the limit of the composite diagram $G \circ D: \mathcal{J} \to \mathcal{C}$ exists in $\mathcal{C}$, and $G$ \CrefAndHyperrefIfExist{definition:functor_preserving_limit_colimit_of_a_diagram}{preserves the limit}, i.e. the \CrefAndHyperrefIfExist{definition:comparison_morphism_of_limits_and_colimits_under_functors}{comparison morphism} induced by the universal property of the limit,
        $$
        \vartheta: G\left(\lim_{j \in \mathcal{J}} D(j)\right) \xrightarrow{\cong} \lim_{j \in \mathcal{J}} G(D(j)),
        $$
        is an isomorphism.

        \item \textbf{Left Adjoints Preserve Colimits:}
        Let $D: \mathcal{J} \to \mathcal{C}$ be a diagram such that the colimit $\operatorname{colim}_{\mathcal{J}} D$ exists in $\mathcal{C}$. Then the colimit of the composite diagram $F \circ D: \mathcal{J} \to \mathcal{D}$ exists in $\mathcal{D}$, and $F$ \CrefAndHyperrefIfExist{definition:functor_preserving_limit_colimit_of_a_diagram}{preserves the colimits}, i.e. the \CrefAndHyperrefIfExist{definition:comparison_morphism_of_limits_and_colimits_under_functors}{comparison morphism} induced by the universal property of the colimit,
        $$
        \varphi: \operatorname{colim}_{j \in \mathcal{J}} F(D(j)) \xrightarrow{\cong} F\left(\operatorname{colim}_{j \in \mathcal{J}} D(j)\right),
        $$
        is an isomorphism.
    \end{enumerate}
\end{theorem}

\Cref{theorem:colimits_commute_with_colimits_and_limits_commute_with_limits} states the colimits commutes with colimits and limits commute with limits.

\begin{theorem}[Fubini Theorem for Colimits and Limtis] \label{theorem:colimits_commute_with_colimits_and_limits_commute_with_limits}
    Let $\mathcal{C}$ be a category and let $I$ and $J$ be small categories. Let $D: I \times J \to \mathcal{C}$ be a bifunctor. 
    
    \begin{enumerate}
        \item Assume that the following \CrefAndHyperrefIfExist{definition:limit_and_colimit_of_a_diagram_in_a_category}{colimits} exist in $\calC$:
        \begin{itemize}
            \item for every object $i \in I$, $\varinjlim_{j \in J} D(i, j)$
            \item $\varinjlim_{i \in I} \left( \varinjlim_{j \in J} D(i, j) \right)$
            \item for every object $j \in J$, $\varinjlim_{i \in I} D(i, j)$
            \item $\varinjlim_{j \in J} \left( \varinjlim_{i \in I} D(i, j) \right)$.
        \end{itemize}
        Then there is a canonical isomorphism:
        $$
        \varinjlim_{i \in I} \left( \varinjlim_{j \in J} D(i, j) \right) 
        \cong \varinjlim_{(i, j) \in I \times J} D(i, j) 
        \cong \varinjlim_{j \in J} \left( \varinjlim_{i \in I} D(i, j) \right).
        $$

                \item Assume that the following \CrefAndHyperrefIfExist{definition:limit_and_colimit_of_a_diagram_in_a_category}{limits} exist in $\calC$:
        \begin{itemize}
            \item for every object $i \in I$, $\varprojlim_{j \in J} D(i, j)$
            \item $\varprojlim_{i \in I} \left( \varprojlim_{j \in J} D(i, j) \right)$
            \item for every object $j \in J$, $\varprojlim_{i \in I} D(i, j)$
            \item $\varprojlim_{j \in J} \left( \varprojlim_{i \in I} D(i, j) \right)$.
        \end{itemize}
        Then there is a canonical isomorphism:
        $$
        \varprojlim_{i \in I} \left( \varprojlim_{j \in J} D(i, j) \right) 
        \cong \varprojlim_{(i, j) \in I \times J} D(i, j) 
        \cong \varprojlim_{j \in J} \left( \varprojlim_{i \in I} D(i, j) \right).
        $$

    \end{enumerate}
\end{theorem}
\begin{remark}
    These results assert that the colimit functor $\operatorname{colim}: \mathcal{C}^{I \times J} \to \mathcal{C}$ (assuming $\mathcal{C}$ is cocomplete) is isomorphic to the composition of partial colimit functors $\operatorname{colim}_I \circ \operatorname{colim}_J$ and $\operatorname{colim}_J \circ \operatorname{colim}_I$.
\end{remark}


\begin{remark}
    These results assert that the colimit functor $\operatorname{colim}: \mathcal{C}^{I \times J} \to \mathcal{C}$ (assuming $\mathcal{C}$ is cocomplete) is isomorphic to the composition of partial colimit functors $\operatorname{colim}_I \circ \operatorname{colim}_J$ and $\operatorname{colim}_J \circ \operatorname{colim}_I$.
\end{remark}

At times, we may want to consider the limit/colimit of a system of functors. Such limits/colimits can be computed ``pointwise'', whenever the pointwise limits/colimits exist.

\begin{proposition}[Pointwise Computation of Limits and Colimits in Functor Categories] \label{proposition:limits_and_colimits_in_functor_categories_may_be_computed_pointwise}
    Let $\mathcal{C}$ and $\mathcal{D}$ be categories. Let $\operatorname{Fun}(\mathcal{C}, \mathcal{D})$ denote the category of functors from $\mathcal{C}$ to $\mathcal{D}$ (also denoted $\mathcal{D}^{\mathcal{C}}$). Let $\mathcal{J}$ be a small category and let $F: \mathcal{J} \to \operatorname{Fun}(\mathcal{C}, \mathcal{D})$ be a diagram of functors, denoted by $j \mapsto F_j$.
    
    \begin{enumerate}
        \item \textbf{Limits are computed pointwise:}
        Suppose that for every object $C \in \mathcal{C}$, the limit of the diagram $j \mapsto F_j(C)$ exists in $\mathcal{D}$. Then the limit of the diagram $F$ exists in $\operatorname{Fun}(\mathcal{C}, \mathcal{D})$ and is computed pointwise. That is, there is an isomorphism in $\operatorname{Fun}(\mathcal{C}, \mathcal{D})$:
        \[
        \left( \lim_{j \in \mathcal{J}} F_j \right)(C) \cong \lim_{j \in \mathcal{J}} (F_j(C)).
        \]
        The action of this limit functor on a morphism $f: C \to C'$ in $\mathcal{C}$ is the unique morphism induced by the family $\{ F_j(f) \}_{j \in \mathcal{J}}$ via the universal property of limits in $\mathcal{D}$.
        
        \item \textbf{Colimits are computed pointwise:}
        Suppose that for every object $C \in \mathcal{C}$, the colimit of the diagram $j \mapsto F_j(C)$ exists in $\mathcal{D}$. Then the colimit of the diagram $F$ exists in $\operatorname{Fun}(\mathcal{C}, \mathcal{D})$ and is computed pointwise. That is, there is an isomorphism in $\operatorname{Fun}(\mathcal{C}, \mathcal{D})$:
        \[
        \left( \operatorname{colim}_{j \in \mathcal{J}} F_j \right)(C) \cong \operatorname{colim}_{j \in \mathcal{J}} (F_j(C)).
        \]
        The action of this colimit functor on a morphism $f: C \to C'$ in $\mathcal{C}$ is the unique morphism induced by the family $\{ F_j(f) \}_{j \in \mathcal{J}}$ via the universal property of colimits in $\mathcal{D}$.
    \end{enumerate}
\end{proposition}



\begin{definition} \label{definition:diagonal_functor_from_a_category_to_a_diagram_category_of_the_category}
    Let $\mathcal{C}$ and $\mathcal{J}$ be \CrefAndHyperrefIfExist{definition:category}{categories}.
    The \hldef{diagonal functor}
    $$\hlin{\Delta: \mathcal{C} \to \mathcal{C}^{\mathcal{J}}}$$
    \CrefIfExists{definition:diagram_in_a_category_indexed_by_a_small_category}
    \TODO{constant functor}
    is the functor sending an object $X \in \mathcal{C}$ to the constant functor $\Delta(X): \mathcal{J} \to \mathcal{C}$, which maps every object in $\mathcal{J}$ to $X$ and every morphism in $\mathcal{J}$ to the identity morphism $1_X$.
    For a morphism $f: X \to Y$ in $\mathcal{C}$, $\Delta(f)$ is the \CrefAndHyperrefIfExist{definition:natural_transformation_between_functors_between_categories}{natural transformation} whose component at every $j \in \mathcal{J}$ is $f$.
\end{definition}

\begin{theorem} \label{theorem:limit_and_colimit_are_left_right_adjoint_to_diagonal_functor_for_locally_small_base_and_small_index}
    Let $\mathcal{C}$ be a \CrefAndHyperrefIfExist{definition:locally_small_category}{locally small} category and $\mathcal{J}$ be a \CrefAndHyperrefIfExist{definition:locally_small_category}{small index category}.
    
    \begin{enumerate}
        \item If $\mathcal{C}$ admits all \CrefAndHyperrefIfExist{definition:limit_and_colimit_of_a_diagram_in_a_category}{colimits} of \CrefAndHyperrefIfExist{definition:diagram_in_a_category_indexed_by_a_small_category}{shape} $\mathcal{J}$, then the \CrefAndHyperrefIfExist{theorem:limits_and_colimits_as_functors_from_functor_category_to_value_category}{colimit functor}
        $$\operatorname{colim}: \mathcal{C}^{\mathcal{J}} \to \mathcal{C}$$
        is \CrefAndHyperrefIfExist{definition:adjoint_functors_between_categories_unit_counit_of_adjoint_functors}{left adjoint} to the \CrefAndHyperrefIfExist{definition:diagonal_functor}{diagonal functor} $\Delta$. That is, for any functor $F: \mathcal{J} \to \mathcal{C}$ and any object $X \in \mathcal{C}$, there is a natural bijection:
        $$\operatorname{Hom}_{\mathcal{C}}(\operatorname{colim} F, X) \cong \operatorname{Hom}_{\mathcal{C}^{\mathcal{J}}}(F, \Delta(X)).$$
        
        \item If $\mathcal{C}$ admits all limits of shape $\mathcal{J}$, then the \CrefAndHyperrefIfExist{theorem:limits_and_colimits_as_functors_from_functor_category_to_value_category}{limit functor}
        $$\lim: \mathcal{C}^{\mathcal{J}} \to \mathcal{C}$$
        is \CrefAndHyperrefIfExist{definition:adjoint_functors_between_categories_unit_counit_of_adjoint_functors}{right adjoint} to the \CrefAndHyperrefIfExist{definition:diagonal_functor_from_a_category_to_a_diagram_category_of_the_category}{diagonal functor} $\Delta$. That is, for any object $X \in \mathcal{C}$ and any functor $F: \mathcal{J} \to \mathcal{C}$, there is a natural bijection:
        $$\operatorname{Hom}_{\mathcal{C}}(X, \lim F) \cong \operatorname{Hom}_{\mathcal{C}^{\mathcal{J}}}(\Delta(X), F).$$
    \end{enumerate}
    These adjunctions characterize limits and colimits via their universal properties.
\end{theorem}


\section{Additive and abelian categories}


\begin{definition} \label{definition:initial_final_zero_objects_of_a_category}
Let $\mathcal{C}$ be a \CrefAndHyperrefIfExist{definition:category}{(large) category}.

\begin{enumerate}
    \item An object $I \in \mathcal{C}$ is called an \hldef{initial object} if for every object $X \in \mathcal{C}$ there exists a unique morphism
    $$I \to X.$$
    Equivalently, an initial object is a \CrefAndHyperrefIfExist{definition:limit_and_colimit_of_a_diagram_in_a_category}{limit} of the empty \CrefAndHyperrefIfExist{definition:diagram_in_a_category_indexed_by_a_small_category}{diagram}, if such a limit exists.

    \item An object $F \in \mathcal{C}$ is called a \hldef{final object} (or \hldef{terminal object}) if for every object $X \in \mathcal{C}$ there exists a unique morphism
    $$X \to F.$$
    Equivalently, a final object is a \CrefAndHyperrefIfExist{definition:limit_and_colimit_of_a_diagram_in_a_category}{colimit} of the empty \CrefAndHyperrefIfExist{definition:diagram_in_a_category_indexed_by_a_small_category}{diagram}, if such a colimit exists.

    \item An object $Z \in \mathcal{C}$ is called a \hldef{zero object} if $Z$ is both initial and final in $\mathcal{C}$. In particular, for every object $X \in \mathcal{C}$ there exist unique morphisms
    $$Z \to X \quad \text{and} \quad X \to Z.$$
\end{enumerate}
In particular, if initial/final/zero objects exist in a cateogry, then they are unique up to unique isomorphism.
\end{definition}

\begin{lemma} \label{lemma:initial_or_final_object_in_a_category_that_is_also_in_a_full_subcategory_is_initial_or_final_in_the_subcategory}
    Let $\calC$ be a \CrefAndHyperrefIfExist{definition:full_subcategory_of_a_category}{full subcategory} of a \CrefAndHyperrefIfExist{definition:category}{(large) category} $\calD$. 
    Suppose that $\calD$ hsa an \CrefAndHyperrefIfExist{definition:initial_final_zero_objects_of_a_category}{initial object} $I$ (resp. a \CrefAndHyperrefIfExist{definition:initial_final_zero_objects_of_a_category}{final object} $F$) and that this object also belongs to $\calC$. The object is initial (resp. final) in $\calC$. 
\end{lemma}

\begin{definition}[Product in a category] \label{definition:product_and_coproduct_of_objects_in_a_category}
Let $\mathcal{C}$ be a category and let $\{X_i\}_{i \in I}$ be a family of objects in $\mathcal{C}$ indexed by a class $I$. 
\begin{enumerate}
    \item A \hldef{product of the family $\{X_i\}$} is an object $P$ of $\mathcal{C}$ together with a ``universal'' family of morphisms
    $$\pi_i : P \to X_i, \quad \text{for each } i \in I. $$
    More precisely, for any object $Y$ and any family of morphisms $\{f_i : Y \to X_i\}_{i \in I}$, there exists a unique morphism
    $$f : Y \to P$$
    making the following diagram commute for all $i \in I$, i.e. $\pi_i \circ f = f_i$:
    \begin{center}
    \begin{tikzcd}[row sep=large, column sep=large]
        Y \arrow[d, "\exists ! f", dashed] \arrow[dr, "f_i"] & \\
        \prod X_i \arrow[r, "\pi_i"'] & X_i
    \end{tikzcd}
    \end{center}
    Such a product is often denoted by \hl{$\prod_{i \in I} X_i$}. If $\prod_{i \in I} X_i$ exists in $\calC$, then it is unique up to unique isomorphism by the universal property described above.
    
    Equivalently, the product $\prod_{i \in I} X_i$ is the \CrefAndHyperrefIfExist{definition:limit_and_colimit_of_a_diagram_in_a_category}{limit} of the \CrefAndHyperrefIfExist{definition:diagram_in_a_category_indexed_by_a_small_category}{diagram} $I \to \calC, i \mapsto X_i$, where $I$ is made into a category whose objects are the members of $I$ and whose morphisms are just the identity morphisms.


    \item A \hldef{coproduct} (or synonymously \hldef{direct sum}) of the family $\{X_i\}$ is an object $C$ of $\mathcal{C}$ together with a ``universal'' family of morphisms
    $$\iota_i : X_i \to C, \quad \text{for each } i \in I.$$
    More precisely, for any object $Y$ and any family of morphisms $\{g_i : X_i \to Y\}_{i \in I}$, there exists a unique morphism
    $$g : C \to Y$$
    making the following diagram commute for all $i \in I$, i.e. $g \circ \iota_i = g_i$:
    \begin{center}
    \begin{tikzcd}[row sep=large, column sep=large]
        X_i \arrow[r, "\iota_i"] \arrow[dr, "g_i"'] & \coprod X_i \arrow[d, "\exists ! g", dashed] \\
        & Y
    \end{tikzcd}
    \end{center}
    Such a coproduct is often denoted by \hl{$\coprod_{i \in I} X_i$} or \hl{$\oplus_{i \in I} X_i$}. If $\coprod_{i \in I} X_i$ exists in $\calC$, then it is unique up to unique isomorphism by the universal property described above.

    Equivalently, the coproduct $\coprod_{i \in I} X_i$ is the \CrefAndHyperrefIfExist{definition:limit_and_colimit_of_a_diagram_in_a_category}{colimit} of the \CrefAndHyperrefIfExist{definition:diagram_in_a_category_indexed_by_a_small_category}{diagram} $I \to \calC, i \mapsto X_i$, where $I$ is made into a category whose objects are the members of $I$ and whose morphisms are just the identity morphisms.
\end{enumerate}
\end{definition}

% \begin{definition} \label{definition:biproduct_of_objects_in_a_category_with_a_zero_object}
% Let $\mathcal{C}$ be a \CrefAndHyperrefIfExist{definition:category}{(large) category}, and let $c_1, c_2 \in \mathrm{Ob}(\mathcal{C})$ be objects.

% A \hldef{biproduct of $c_1$ and $c_2$} in $\mathcal{C}$ is a tuple
% $$\hlin{\bigl(c_1 \oplus c_2, p_1: c_1 \oplus c_2 \to c_1, p_2: c_1 \oplus c_2 \to c_2, i_1: c_1 \to c_1 \oplus c_2, i_2: c_2 \to c_1 \oplus c_2 \bigr)}$$
% such that:
% \begin{enumerate}
%     \item $(c_1 \oplus c_2, p_1, p_2)$ is a \CrefAndHyperrefIfExist{definition:product_and_coproduct_of_objects_in_a_category}{product} of $c_1$ and $c_2$ in $\mathcal{C}$,
%     \item $(c_1 \oplus c_2, i_1, i_2)$ is a \CrefAndHyperrefIfExist{definition:product_and_coproduct_of_objects_in_a_category}{coproduct} of $c_1$ and $c_2$ in $\mathcal{C}$,
%     \item the following identities hold:
%     $$
%     p_1 \circ i_1 = \mathrm{id}_{c_1}, \quad p_2 \circ i_2 = \mathrm{id}_{c_2},
%     $$
%     $$
%     i_1 \circ p_1 \circ i_2 \circ p_2 = i_2 \circ p_2 \circ i_1 \circ p_1.
%     $$
% \end{enumerate}
% More generally, we may speak of a \hldef{biproduct of a finite family of objects of $\calC$}. 
% \end{definition}



\begin{definition} \label{definition:product_of_finitely_many_objects_in_a_category}
Let $\mathcal{C}$ be a \CrefAndHyperrefIfExist{definition:category}{(large) category}, and let $c_1, c_2, \dots, c_n \in \mathrm{Ob}(\mathcal{C})$ be objects.

A \hldef{product of $c_1, c_2, \dots, c_n$} in $\mathcal{C}$ is a tuple
$$
\hlin{\bigl( c_1 \times c_2 \times \cdots \times c_n,\ 
p_1 : c_1 \times \cdots \times c_n \to c_1,\ 
\dots,\ 
p_n : c_1 \times \cdots \times c_n \to c_n \bigr)}
$$
such that for every object $d \in \mathrm{Ob}(\mathcal{C})$ and every family of morphisms
$$
(f_1 : d \to c_1,\ f_2 : d \to c_2,\ \dots,\ f_n : d \to c_n),
$$
there exists a unique morphism
$$
\langle f_1, f_2, \dots, f_n \rangle : d \to c_1 \times c_2 \times \cdots \times c_n
$$
satisfying
$$
p_i \circ \langle f_1, f_2, \dots, f_n \rangle = f_i \quad \text{for all } 1 \le i \le n.
$$
The morphisms $p_i$ are called the \hldef{projection morphisms}, and the morphism
$\langle f_1, f_2, \dots, f_n \rangle$ is called the \hldef{product morphism}. The product $c_1 \times \cdots \times c_n$ is also denoted by \hl{$\prod_{i=1}^n c_i$}, \hl{$c_1 \oplus \cdots \oplus c_n$}, or \hl{$\oplus_{i=1}^n c_i$}.

When $n = 0$, the product is defined to be a \CrefAndHyperrefIfExist{definition:initial_final_zero_objects_of_a_category}{terminal object} of $\mathcal{C}$, if one exists.  
When $n = 2$, we simply speak of the \hldef{binary product} or \hldef{biproduct} of $c_1$ and $c_2$.
\end{definition}




\begin{definition}[Complete and Cocomplete Category] \label{definition:complete_and_cocomplete_category}
Let $\mathcal{C}$ be a \CrefAndHyperrefIfExist{definition:category}{category}.  
\begin{itemize}
    \item The category $\mathcal{C}$ is called \hldef{complete} (resp. \hldef{finitely complete}) if all \CrefAndHyperrefIfExist{definition:small_and_finite_limits_and_colimits_in_a_category}{small limits} (resp. finite limits) exist in $\mathcal{C}$; that is, for every small diagram $D : J \to \mathcal{C}$ (with $J$ a small category), the limit $\lim D$ exists and is an object of $\mathcal{C}$.
    \item The category $\mathcal{C}$ is called \hldef{cocomplete} (resp. \hldef{finitely cocomplete}) if all \CrefAndHyperrefIfExist{definition:small_and_finite_limits_and_colimits_in_a_category}{small colimits} (resp. finite colimits) exist in $\mathcal{C}$; that is, for every small diagram $D : J \to \mathcal{C}$, the colimit $\mathrm{colim}\ D$ exists and is an object of $\mathcal{C}$.
\end{itemize}
\end{definition}

\begin{definition}[Additive category] \label{definition:additive_category}
Let $\mathcal{A}$ be a \CrefAndHyperrefIfExist{definition:locally_small_category}{locally small category}. 
\begin{enumerate}
    \item $\calA$ is said to be a \hldef{preadditive category} if the following hold:
    \begin{itemize}
        \item For any two objects $A, B$ in $\mathcal{A}$, the set $\operatorname{Hom}_{\mathcal{A}}(A, B)$ is an \CrefAndHyperrefIfExist{definition:group}{abelian group}, and composition of morphisms is bilinear.
        \item There is a \CrefAndHyperrefIfExist{definition:initial_final_zero_objects_of_a_category}{zero object} $0$ in $\mathcal{A}$.
    \end{itemize}
    \TextIfExists{definition:category_enriched_in_a_monoidal_category}{Equvialently, a preadditive cateogry $\calA$ is a (necessarily locally small) category \CrefAndHyperrefIfExist{definition:category_enriched_in_a_monoidal_category}{enriched in} the \CrefAndHyperrefIfExist{definition:monoidal_category}{monoidal category} $\Ab$ that also possesses a zero object.}

    \item
    If $\calA$ is preadditive, then it is called \hldef{additive} if it additionally satisfies the following:
    \begin{itemize}
        \item For any two objects $A, B$ in $\mathcal{A}$, there exists a \CrefAndHyperrefIfExist{definition:product_and_coproduct_of_objects_in_a_category}{product object $A \times B$}, often written \hl{$A \oplus B$}, called the \hldef{direct sum of $A$ and $B$}. In fact, $A \oplus B$ is not only a product but also a \CrefAndHyperrefIfExist{definition:coproduct_of_modules_of_rings}{coproduct} of $A$ and $B$\CrefIfExists{lemma:finite_products_and_finite_coproducts_coincide_in_preadditive_categories}.
    \end{itemize}

    Given a finite collection $\{A_i\}_i$ of objects $A_i$ in an additive category $\calA$, we may more generally speak of the \hldef{direct sum} \hl{$\bigoplus_i A_i$}; it has canonical injections from and projections to each $A_i$.


\end{enumerate}
\end{definition}


\begin{lemma} \label{lemma:finite_products_and_finite_coproducts_coincide_in_preadditive_categories}
    Let $\calA$ be a \CrefAndHyperrefIfExist{definition:additive_category}{preadditive category}. Finite \CrefAndHyperrefIfExist{definition:product_and_coproduct_of_objects_in_a_category}{products} in $\calA$ coincide with finite \CrefAndHyperrefIfExist{definition:product_and_coproduct_of_objects_in_a_category}{coproducts}. More precisely, if $\{A_i\}_{i = 1}^n$ is a finite collection of objects of $\calA$, then 
    \begin{enumerate}
        \item if $\prod_{i = 1}^n A_i$ exists, then so does $\coprod_{i=1}^n A_i$ and these are naturally isomorphic. 

        \item if $\coprod_{i=1}^n A_i$, then so does $\prod_{i=1}^n A_i$ and these are naturally isomorphic. 
    \end{enumerate}
\end{lemma}

\begin{proof}
    \TODO{}
\end{proof}

\begin{definition}[Additive functor] \label{definition:additive_functor_between_additive_categories}

    \begin{enumerate}
        \item Let $\mathcal{A}$ and $\mathcal{B}$ be \hyperrefIfExists{definition:additive_category_preadditive_category}{pre-additive categories}. A functor
        $$ F: \mathcal{A} \to \mathcal{B} $$
        is an \hldef{additive functor} if for every pair of objects $A, A' \in \mathcal{A}$, the induced map
        $$ F_{A,A'}: \operatorname{Hom}_{\mathcal{A}}(A, A') \to \operatorname{Hom}_{\mathcal{B}}(F(A), F(A')) $$
        is a group homomorphism of abelian groups, or equvialently if it is \CrefAndHyperrefIfExist{definition:category_enriched_in_a_monoidal_category}{enriched over the category $\Ab$ of abelian groups}.
        
        \item Let $\mathcal{A}$ and $\mathcal{B}$ be \hyperrefIfExists{definition:additive_category_preadditive_category}{additive categories}. A functor
        $$ F: \mathcal{A} \to \mathcal{B} $$
        is an \hldef{additive functor} if it an additive functor of pre-additive categories and satisfies the following:
        \begin{itemize}
            \item $F$ sends the zero object $0_{\mathcal{A}}$ of $\mathcal{A}$ to the zero object $0_{\mathcal{B}}$ of $\mathcal{B}$, i.e.,
            $$ F(0_{\mathcal{A}}) = 0_{\mathcal{B}}.  $$
            \item $F$ preserves finite direct sums: For any finite family of objects $\{A_i\}_{i=1}^n$ in $\mathcal{A}$,
            $$ F\left(\bigoplus_{i=1}^n A_i\right) \cong \bigoplus_{i=1}^n F(A_i) $$
            via the canonical isomorphism induced by $F$ applied to the canonical injections and projections.
        \end{itemize}
        In other words, $F$ is a functor that is compatible with the additive structures on $\mathcal{A}$ and $\mathcal{B}$.
    \end{enumerate}
\end{definition}


\begin{definition}[Abelian category] \label{definition:abelian_category}
Let $\mathcal{A}$ be a category. The category $\mathcal{A}$ is an \hldef{abelian category} if:
\begin{itemize}
    \item $\mathcal{A}$ is an \CrefAndHyperrefIfExist{definition:additive_category_preadditive_category}{additive category}.

    \item Every morphism $f: A \to B$ has a \CrefAndHyperrefIfExist{definition:kernel_and_cokernel_of_a_morphism_in_a_category}{kernel $\ker(f)$ and a cokernel $\operatorname{coker}(f)$}.

    \item For every morphism $f: A \to B$, the canonical morphism $\operatorname{coim}(f) \to \operatorname{im}(f)$ is an isomorphism, where
    $$
    \operatorname{coim}(f) = \operatorname{coker}(\ker(f) \to A),\quad \operatorname{im}(f) = \ker(B \to \operatorname{coker}(f)).
    $$
    \TODO{I think I need to re-check this defintion}
    \TODO{coimage}
\end{itemize}

\TextIfExists{definition:pre_abelian_category}{In particular, every abelian category is \Cref{definition:pre_abelian_category}{pre-abelian}}.

It is also worth considering Grothendieck's additional axioms for abelian categories\CrefIfExists{definition:grothendiecks_additional_axioms_for_abelian_categories}.

\end{definition}

\begin{proposition} \label{proposition:diagram_category_of_a_preadditive_additive_abelian_category_indexed_by_a_small_category_is_preadditive_additive_abelian}
    Let $\calA$ be a \CrefAndHyperrefIfExist{definition:additive_category}{preadditive} (resp. \CrefAndHyperrefIfExist{definition:additive_category}{additive}, \CrefAndHyperrefIfExist{definition:abelian_category}{abelian}) category and let $J$ be a \CrefAndHyperrefIfExist{definition:locally_small_category}{small} category. The \CrefAndHyperrefIfExist{definition:diagram_in_a_category_indexed_by_a_small_category}{diagram category} $\calA^J$ is preadditive (resp. additive, abelian).
\end{proposition}

\begin{definition}[Equalizer in a category] \label{definition:equalizer_and_coequalizer_of_morphisms_in_a_category}
Let $\mathcal{C}$ be a \CrefAndHyperrefIfExist{definition:category}{(large) category} and let $f, g : X \to Y$ be morphisms in $\mathcal{C}$. 
\begin{enumerate}
    \item An \hldef{equalizer of $f$ and $g$} is an object $E$ together with a morphism
    $$e : E \to X$$
    such that
    $$f \circ e = g \circ e$$
    and for any object $Z$ with morphism $z : Z \to X$ satisfying
    $$f \circ z = g \circ z,$$
    there exists a unique morphism $u : Z \to E$ making the diagram commute:
    $$e \circ u = z.$$

    \begin{center}
        \begin{tikzcd}[column sep=large, row sep=large]
        Z \arrow[d, dashed, "\exists! u"] \arrow[dr, "z"] & & \\
        E \arrow[r, "e"] & X \arrow[r, shift left, "f"] \arrow[r, shift right, "g"'] & Y
        \end{tikzcd}
    \end{center}
    If such an equalizer of $f$ and $g$ exists, then we say that the following \hldef{equalizer diagram is exact}:
    \begin{center}
    \begin{tikzcd}[column sep=large, row sep=large]
    E \arrow[r, "e"] & X \arrow[r, shift left, "f"] \arrow[r, shift right, "g"'] & Y
    \end{tikzcd}
    \end{center}

    \item A \hldef{coequalizer of $f$ and $g$} is an object $Q$ together with a morphism
    $$q : Y \to Q$$
    such that
    $$q \circ f = q \circ g$$
    and for any object $Z$ with morphism $w : Y \to Z$ satisfying
    $$w \circ f = w \circ g,$$
    there exists a unique morphism $v : Q \to Z$ making the diagram commute:
    $$v \circ q = w.$$

    \begin{center}
        \begin{tikzcd}[column sep=large]
        X \arrow[r, shift left, "f"] \arrow[r, shift right, "g"'] & Y \arrow[r, "q"] \arrow[dr, "w"'] & Q \arrow[d, dashed, "\exists! v"] \\
        & & Z
        \end{tikzcd}
    \end{center}
    If such a coequalizer of $f$ and $g$ exists, then we say that the following \hldef{coequalizer diagram is exact}:
    \begin{center}
    \begin{tikzcd}[column sep=large, row sep=large]
        X \arrow[r, shift left, "f"] \arrow[r, shift right, "g"'] & Y \arrow[r, "q"] & Q 
    \end{tikzcd}
    \end{center}



\end{enumerate}
\end{definition}

\begin{lemma} \label{lemma:equalizer_coequalizer_in_an_additive_category_are_given_by_kernel_and_cokernel}
    Let $f,g: X \to Y$ be morphisms in an \CrefAndHyperrefIfExist{definition:additive_category}{additive category}.
    \begin{enumerate}
        \item The \CrefAndHyperrefIfExist{definition:equalizer_and_coequalizer_of_morphisms_in_a_category}{equalizer} of $f$ and $g$ is given by $\ker(f-g)$. 
        \item The \CrefAndHyperrefIfExist{definition:equalizer_and_coequalizer_of_morphisms_in_a_category}{coequalizer} of $f$ and $g$ is given by $\operatorname{coker}(f-g)$. 
    \end{enumerate}
\end{lemma}
\begin{lemma} \label{lemma:abelian_categories_are_finitely_complete_and_finitely_cocomplete}
    \CrefAndHyperrefIfExist{definition:abelian_category}{Abelian categories} are \CrefAndHyperrefIfExist{definition:complete_and_cocomplete_category}{finitely complete and finitely cocomplete}.
\end{lemma}
\begin{proof}
    Abelian categories have finite products and finite coproducts (in the form of direct sums), empty products and coproducts (in the form of the zero object), and \CrefAndHyperrefIfExist{definition:equalizer_and_coequalizer_of_morphisms_in_a_category}{equalizers and coequalizers} (in the form of kernels and cokernels of morphisms), so Theorem \ref{theorem:limits_colimits_are_equalizers_coequaulizers_of_products_coproducts} applies.
\end{proof}


\begin{proposition} \label{proposition:examples_of_abelian_categories}
The following are examples of \CrefAndHyperrefIfExist{definition:abelian_category}{abelian categories}:

\begin{enumerate}
    \item The category of $R$-$S$ bimodules where $R$,$S$ are \CrefAndHyperrefIfExist{definition:ring}{(not necessarily commutative) rings} (\Cref{theorem:the_category_of_R_S_bimodules_is_a_grothendieck_abelian_category_and_AB4_star}).

    \item The category $\mathbf{Ab}$ of abelian groups and group homomorphisms is abelian.

    \item The category $\text{Vect}_k$ of vector spaces over a field $k$ and $k$-linear maps is abelian.

    \item More generally, if $R$ is a \CrefAndHyperrefIfExist{definition:noetherian_ring}{noetherian ring}, then the category of \CrefAndHyperrefIfExist{definition:finitely_generated_modules_over_rings}{finitely generated} $R$-modules is abelian.

    \item For a \CrefAndHyperrefIfExist{definition:ringed_space}{ringed space} $(X, \mathcal{O}_X)$, the category of \CrefAndHyperrefIfExist{definition:module_over_a_sheaf_of_rings_on_a_site}{$\mathcal{O}_X$-modules} is abelian.
    \TODO{a quasi-coherent sheaf on a locally ringed space}
    \item If $X$ is a \CrefAndHyperrefIfExist{definition:scheme}{scheme} (or more generally a \CrefAndHyperrefIfExist{definition:locally_ringed_space_on_a_topological_space}{locally ringed space}), the category of \CrefAndHyperrefIfExist{definition:quasi_coherent_sheaf_on_a_general_scheme}{quasi-coherent sheaves on $X$} is abelian.
    \item For any \CrefAndHyperrefIfExist{definition:essentially_small_category}{essentially small category} $\mathcal{C}$ and any abelian category $\mathcal{A}$, the \CrefAndHyperrefIfExist{definition:diagram_in_a_category_indexed_by_a_small_category}{functor category $[\mathcal{C}, \mathcal{A}]$} and the category $\PreShv(\calC, \calA)$ of \CrefAndHyperrefIfExist{definition:presheaf_on_a_category}{presheaves} are abelian.
    \TODO{apparently, the essentially smallness condition is removable, provided that the sheafification functor exists. However, the essentially small assumption is needed to show that the category of sheaves of $O$-modules is a Grothendieck abelian caetgory. Verify all this. Moreover, when working with a big site of a scheme, one typically fixes a unvierse or work relative to a cardinal cutoff to treat it as essentially small}

    \item For any \CrefAndHyperrefIfExist{definition:grothendieck_topology_on_a_category_site_covering_sieve_topologically_generating_family}{site} $(\calC, J)$ on an \CrefAndHyperrefIfExist{definition:essentially_small_category}{essentially small category} $\mathcal{C}$ and any abelian category $\mathcal{A}$, the category $\Shv(\calC, \calA)$ of \CrefAndHyperrefIfExist{definition:sheaf_on_a_site}{sheaves} is abelian.

    \item For any \CrefAndHyperrefIfExist{definition:grothendieck_topology_on_a_category_site_covering_sieve_topologically_generating_family}{site} $(\calC, J)$ on an \CrefAndHyperrefIfExist{definition:essentially_small_category}{essentially small category} $\mathcal{C}$ and a \CrefAndHyperrefIfExist{definition:sheaf_on_a_site}{sheaf of rings} $\calO$ on $\calC$, the category $\mathbf{Mod}(\mathcal{O})$ of \CrefAndHyperrefIfExist{definition:module_over_a_sheaf_of_rings_on_a_site}{$\calO$-modules} is an abelian category.

\end{enumerate}
\end{proposition}



\begin{proposition}[Monomorphisms and kernels, epimorphisms and cokernels in abelian categories] \label{proposition:monomorphisms_are_the_same_as_kernels_and_epimorphisms_are_the_same_as_cokernels_in_abelian_categories}
Let $\mathcal{A}$ be an \CrefAndHyperrefIfExist{definition:abelian_category}{abelian category}.

\begin{itemize}
    \item[(1)] A morphism $f : X \to Y$ in $\mathcal{A}$ is a \CrefAndHyperrefIfExist{definition:monomorphism_and_epimorphism_in_categories}{monomorphism} if and only if $f$ is the \CrefAndHyperrefIfExist{definition:kernel_and_cokernel_of_a_morphism_in_a_category}{kernel} of some morphism.
    \item[(2)] Dually, $f : X \to Y$ is an \CrefAndHyperrefIfExist{definition:monomorphism_and_epimorphism_in_categories}{epimorphism} if and only if $f$ is the \CrefAndHyperrefIfExist{definition:kernel_and_cokernel_of_a_morphism_in_a_category}{cokernel} of some morphism.
\end{itemize}
\end{proposition}



\begin{definition} \label{definition:exact_functor_between_abelian_categories}
    Let $F: \mathcal{A} \to \mathcal{B}$ be an \hyperrefIfExists{definition:additive_functor_between_additive_categories}{additive functor}\CrefIfExists{definition:additive_functor_between_additive_categories} between \hyperrefIfExists{definition:abelian_category}{abelian categories}\CrefIfExists{definition:abelian_category}.
    \begin{enumerate}

        \item $F$ is called \hldef{left exact} if it preserves all \CrefAndHyperrefIfExist{definition:limit_and_colimit_of_a_diagram_in_a_category}{finite limits}, or equivalently it preserves \CrefAndHyperrefIfExist{definition:kernel_and_cokernel_of_a_morphism_in_a_category}{kernels} and any finite limit diagrams. Equivalently, for every left exact sequence in $\mathcal{A}$
        \[
        0 \to A' \xrightarrow{f} A \xrightarrow{g} A''
        \]
        the sequence
        \[
        0 \to F(A') \xrightarrow{F(f)} F(A) \xrightarrow{F(g)} F(A'')
        \]
        is exact at $F(A')$ and $F(A)$ (i.e., $F$ preserves \CrefAndHyperrefIfExist{definition:monomorphism_and_epimorphism_in_categories}{monomorphisms} and exactness at the first two terms).

        \item Dually, $F$ is called \hldef{right exact} if it preserves all \CrefAndHyperrefIfExist{definition:limit_and_colimit_of_a_diagram_in_a_category}{finite colimits}, or equivalently it preserves \CrefAndHyperrefIfExist{definition:kernel_and_cokernel_of_a_morphism_in_a_category}{cokernels} and any finite colimit diagrams. Equivalently, for every right exact sequence in $\mathcal{A}$
        \[
        A' \xrightarrow{f} A \xrightarrow{g} A'' \to 0,
        \]
        the sequence
        \[
        F(A') \xrightarrow{F(f)} F(A) \xrightarrow{F(g)} F(A'') \to 0
        \]
        is exact at $F(A)$ and $F(A'')$ (i.e., $F$ preserves \CrefAndHyperrefIfExist{definition:monomorphism_and_epimorphism_in_categories}{epimorphisms} and exactness at the last two terms).

        \item $F$ is called \hldef{exact} if it is both left and right exact.
    \end{enumerate}

    \TextIfExists{definition:left_right_exact_functor_between_categories}{
        The additive functor $F$ is left/right exact if and only if it is \CrefAndHyperref{definition:left_right_exact_functor_between_categories}{left/right exact} in the more general sense, i.e. if it preserves all \CrefAndHyperrefIfExist{definition:small_and_finite_limits_and_colimits_in_a_category}{finite} \CrefAndHyperrefIfExist{definition:limit_and_colimit_of_a_diagram_in_a_category}{limits/colimits}

    }
\end{definition}


\begin{proposition} \label{proposition:left_right_adjoint_is_right_left_exact}
    Let $\calA, \calB$ be \CrefAndHyperrefIfExist{definition:abelian_category}{abelian categories} and let $F: \calA \to \calB$ and $G: \calB \to \calA$ be \CrefAndHyperrefIfExist{definition:adjoint_functors_between_categories_unit_counit_of_adjoint_functors}{adjoint} \CrefAndHyperrefIfExist{definition:additive_functor_between_additive_categories}{additive functors}, say with $F \dashv G$ (i.e. $F$ is left adjoint to $G$). The left adjoint functor $F$ is \CrefAndHyperrefIfExist{definition:exact_functor_between_abelian_categories}{right exact} and and the right adjoint functor $G$ is \CrefAndHyperrefIfExist{definition:exact_functor_between_abelian_categories}{left exact}
\end{proposition}

\begin{definition} \label{definition:short_exact_sequence_in_an_additive_category}
Let $\mathcal{A}$ be an \CrefAndHyperrefIfExist{definition:additive_category_preadditive_category}{additive category}. A sequence 
$$0 \to A \xrightarrow{f} B \xrightarrow{g} C \to 0$$ 
of moprhisms in $\mathcal{A}$ is called a \hldef{short exact sequence} if the morphisms satisfy:
\begin{itemize}
  \item $f : A \to B$ is a \CrefAndHyperrefIfExist{definition:monomorphism_and_epimorphism_in_categories}{monomorphism} and is the \CrefAndHyperrefIfExist{definition:kernel_and_cokernel_of_a_morphism_in_a_category}{kernel of $g$},
  \item $g : B \to C$ is an \CrefAndHyperrefIfExist{definition:monomorphism_and_epimorphism_in_categories}{epimorphism} and is the \CrefAndHyperrefIfExist{definition:kernel_and_cokernel_of_a_morphism_in_a_category}{cokernel of $f$},
  \item the sequence is \CrefAndHyperrefIfExist{definition:acyclic_complex_of_objects_in_an_abelian_category}{exact at} $B$, meaning \CrefAndHyperrefIfExist{definition:image_coimage_of_a_morphism_in_a_category}{$\mathrm{Im}(f) = \mathrm{Ker}(g)$}.
\end{itemize}
This means the sequence starts and ends with the zero object and is exact everywhere.
\end{definition}

\begin{proposition} \label{proposition:Hom_functors_on_abelian_categories_are_left_exact}
Let $\mathcal{A}$ be an \CrefAndHyperrefIfExist{definition:abelian_category}{abelian category} and $A \in \mathrm{Ob}(\mathcal{A})$. Then the covariant Hom functor
$$\operatorname{Hom}_{\mathcal{A}}(A,-): \mathcal{A} \to \mathbf{Ab}$$
is \CrefAndHyperrefIfExist{definition:exact_functor_between_abelian_categories}{left exact}, and the contravariant Hom functor
$$\operatorname{Hom}_{\mathcal{A}}(-,A): \mathcal{A}^{\mathrm{op}} \to \mathbf{Ab}$$
\CrefIfExists{definition:opposite_category_of_a_category} is also left exact. Explicitly, given a left exact sequence
$$0 \to X' \to X \to X'', $$
we have a left exact sequence
$$0 \to \Hom_\calA(A, X') \to \Hom_{\calA}(A, X) \to \Hom_{\calA}(A, X''), $$
and given a right exact sequence
$$X' \to X \to X'' \to 0$$
we have a left exact sequence
$$0 \to \Hom_\calA(X'', A) \to \Hom_{\calA}(X, A) \to \Hom_{\calA}(X', A).$$
\end{proposition}

\begin{theorem} \label{theorem:functor_category_of_an_abelian_category_indexed_by_a_small_category_is_abelian}
    Let $\mathcal{A}$ be an \CrefAndHyperrefIfExist{definition:abelian_category}{abelian category} and $\mathcal{J}$ be a \CrefAndHyperrefIfExist{definition:locally_small_category}{small category}. The \CrefAndHyperrefIfExist{definition:diagram_in_a_category_indexed_by_a_small_category}{functor category} $\mathcal{A}^{\mathcal{J}}$ inherits the structure of an abelian category from $\mathcal{A}$. Specifically:
    \begin{itemize}
        \item The \CrefAndHyperrefIfExist{definition:initial_final_zero_objects_of_a_category}{zero object} in $\mathcal{A}^{\mathcal{J}}$ is the constant functor at the zero object of $\mathcal{A}$.
        \item \CrefAndHyperrefIfExist{definition:kernel_and_cokernel_of_a_morphism_in_a_category}{Kernels}, \CrefAndHyperrefIfExist{definition:kernel_and_cokernel_of_a_morphism_in_a_category}{cokernels}, \CrefAndHyperrefIfExist{definition:product_and_coproduct_of_objects_in_a_category}{products}, and \CrefAndHyperrefIfExist{definition:product_and_coproduct_of_objects_in_a_category}{coproducts} are computed pointwise in $\mathcal{A}$. For example, if $\eta: F \to G$ is a \CrefAndHyperrefIfExist{definition:natural_transformation_between_functors_between_categories}{natural transformation}, the kernel is the functor $K: \mathcal{J} \to \mathcal{A}$ defined by $K(j) = \ker(\eta_j)$ for each $j \in \mathcal{J}$.
        \item A sequence of functors $0 \to F \to G \to H \to 0$ is \CrefAndHyperrefIfExist{definition:short_exact_sequence_in_an_additive_category}{exact} in $\mathcal{A}^{\mathcal{J}}$ if and only if for every object $j \in \mathcal{J}$, the sequence $0 \to F(j) \to G(j) \to H(j) \to 0$ is exact in $\mathcal{A}$.
    \end{itemize}
    Moreover, if $\mathcal{A}$ admits arbitrary limits (resp. colimits), then $\mathcal{A}^{\mathcal{J}}$ also admits arbitrary limits (resp. colimits), computed pointwise.
\end{theorem}


\begin{proposition} \label{proposition:limits_colimits_in_abelian_category_are_left_right_exact}
    Let $\mathcal{A}$ be an \CrefAndHyperrefIfExist{definition:abelian_category}{abelian category} and $\mathcal{J}$ be a \CrefAndHyperrefIfExist{definition:small_category}{small index category}.
    
        \TODO{}
    \begin{enumerate}
        \item The \CrefAndHyperrefIfExist{theorem:limits_and_colimits_as_functors_from_functor_category_to_value_category}{limit functor} $\lim: \mathcal{A}^\mathcal{J} \to \mathcal{A}$ is \CrefAndHyperrefIfExist{definition:exact_functor_between_abelian_categories}{left exact}. That is, if
        $$0 \to F \to G \to H \to 0$$
        is a \CrefAndHyperrefIfExist{definition:short_exact_sequence_in_an_additive_category}{short exact sequence} of functors in $\mathcal{A}^\mathcal{J}$ (\CrefAndHyperrefIfExist{theorem:functor_category_of_an_abelian_category_indexed_by_a_small_category_is_abelian}{equivalently} it is exact object-wise), then the induced sequence
        $$0 \to \lim F \to \lim G \to \lim H$$
        is \CrefAndHyperrefIfExist{definition:exact_functor_between_abelian_categories}{exact} in $\mathcal{A}$.
        
        \item Limits commute with kernels (and indeed with all limits). Since the kernel is a finite limit, strictly speaking, $\lim (\ker \Phi) \cong \ker (\lim \Phi)$ for any natural transformation $\Phi$.
        
        \item The \CrefAndHyperrefIfExist{theorem:limits_and_colimits_as_functors_from_functor_category_to_value_category}{colimit functor} $\operatorname{colim}: \mathcal{A}^\mathcal{J} \to \mathcal{A}$ is right exact. That is, if
        $$0 \to F \to G \to H \to 0$$
        is a short exact sequence of functors, then the induced sequence
        $$\operatorname{colim} F \to \operatorname{colim} G \to \operatorname{colim} H \to 0$$
        is exact in $\mathcal{A}$.
        
        \item Colimits commute with cokernels (and indeed with all colimits). Strictly speaking, $\operatorname{colim} (\operatorname{coker} \Phi) \cong \operatorname{coker} (\operatorname{colim} \Phi)$.
    \end{enumerate}
\end{proposition}
\begin{proof}
    \TODO{}
\end{proof}



% Intuitively, Hom functors preserve the algebraic structure encoded by morphisms but do not necessarily preserve cokernels or epimorphisms, which explains their partial (left) exactness. The failure of exactness on the right motivates the construction of Ext functors.


\begin{proposition} \label{proposition:full_and_faithful_additive_functor_between_abelian_categories_reflects_exactness}
Let $F: \mathcal{A} \to \mathcal{B}$ be a \CrefAndHyperrefIfExist{definition:full_and_faithful_functor_between_locally_small_categories}{full and faithful} \CrefAndHyperrefIfExist{definition:additive_functor_between_additive_categories}{additive functor} between \CrefAndHyperrefIfExist{definition:abelian_category}{abelian categories}. Then $F$ \CrefAndHyperrefIfExist{definition:reflects_a_type_of_morphism_for_a_functor_between_categories}{reflects exactness}, i.e., given a sequence
\[
0 \to A' \xrightarrow{f} A \xrightarrow{g} A'' \to 0
\]
in $\mathcal{A}$, if the sequence
\[
0 \to F(A') \xrightarrow{F(f)} F(A) \xrightarrow{F(g)} F(A'') \to 0
\]
is exact in $\mathcal{B}$, then the original sequence is exact in $\mathcal{A}$. More generally, $F$ reflects any \CrefAndHyperrefIfExist{definition:acyclic_complex_of_objects_in_an_abelian_category}{exact sequence} and all limits and colimits that exist in $\calB$ for diagrams coming from $\calA$.
\end{proposition}

\begin{theorem}[Freyd-Mitchell Embedding Theorem] \label{theorem:freyd_mitchell_embedding_theorem_for_small_abelian_categories}
    Let $\mathcal{A}$ be a \CrefAndHyperrefIfExist{definition:locally_small_category}{small} \CrefAndHyperrefIfExist{definition:abelian_category}{abelian category}.
    There exists a \CrefAndHyperrefIfExist{definition:ring}{ring} $R$ (which may not be commutative) and a functor $F: \mathcal{A} \to \operatorname{Mod}_R$\CrefIfExists{theorem:category_of_modules_over_a_sheaf_of_rings_on_a_site_on_an_essentially_small_category_has_enough_injectives} such that:
    \TODO{Show thta exact functors preserve finite limits and colimits}
    \begin{enumerate}
        \item $F$ is \CrefAndHyperrefIfExist{definition:exact_functor_between_abelian_categories}{exact}, meaning it preserves all finite limits and colimits (in particular, kernels, cokernels, and exact sequences).
        \item $F$ is \CrefAndHyperrefIfExist{definition:full_and_faithful_functor_between_locally_small_categories}{fully faithful}.
    \end{enumerate}
    Consequently, any diagram-chasing argument valid for modules over a ring is also valid in any small abelian category, and by extension (using the fact that any exact diagram involves only a set of objects), in any abelian category.
\end{theorem}

% \begin{theorem}[Freyd-Mitchell Embedding Theorem] \label{theorem:freyd_mitchell_embedding_theorem}
%     Let $\calA$ be a \CrefAndHyperrefIfExist{definition:locally_small_category}{small} \CrefAndHyperrefIfExist{definition:abelian_category}{abelian category}.
%     There exists a \CrefAndHyperrefIfExist{definition:ring}{ring} $R$ and an \CrefAndHyperrefIfExist{definition:exact_functor_between_abelian_categories}{exact}, \CrefAndHyperrefIfExist{definition:full_and_faithful_functor_between_locally_small_categories}{fully faithful} functor from $\calA$ into the category \CrefAndHyperrefIfExist{definition:category_of_modules_and_bimodules_over_rings}{$R\mathrm{-mod}$} of \CrefAndHyperrefIfExist{definition:module_of_a_ring}{left $R$-modules}.
% \end{theorem}



\subsection{Grothendieck's additional axioms for abelian categories}

\begin{definition} \label{definition:representable_functor_on_a_category_enriched_in_a_monoidal_category}
    Let $C$ be a \CrefAndHyperrefIfExist{definition:category_enriched_in_a_monoidal_category}{category enriched in a monidal category} $\mathcal{V}$. Given an object $X$ of $C$, the \hldef{functor of points} \hl{$h_X$} is the \CrefAndHyperrefIfExist{definition:functor_between_categories}{functor}/\CrefAndHyperrefIfExist{definition:presheaf_on_a_category}{presheaf} $C^{\op} \to \mathcal{V}$ given by $T \mapsto \Hom_C(T, X)$. A functor $C^{\op} \to \mathcal{V}$ (or equivalently, a presheaf on $C$ valued in $\mathcal{V}$) is said to be \hldef{representable} if it is \CrefAndHyperrefIfExist{definition:natural_transformation_between_functors_between_categories}{naturally isomorphic} to some functor $h_X$ of points for an object $X$ of $C$.

    Dually, a functor $C \to \calV$ is called \hldef{co-representable} if it is naturally isomorphic to a functor $T \mapsto \Hom_C(X, T)$ for an object $X$ in $C$. 

    For instance, we may speak of these notions when $\calV$ is the monoidal category $\Sets$, i.e. $C$ is a \CrefAndHyperrefIfExist{definition:locally_small_category}{locally small category}.
\end{definition}
\begin{theorem}[Yoneda Lemma] \label{theorem:yoneda_lemma_on_a_locally_small_category}
Let $\mathcal{C}$ be a \CrefAndHyperrefIfExist{definition:locally_small_category}{locally small category}. Let $A$ be an object of $\mathcal{C}$, and let $F: \mathcal{C} \to \mathbf{Set}$ be a \CrefAndHyperrefIfExist{definition:functor_between_categories}{covariant functor} to the \CrefAndHyperrefIfExist{definition:category_of_sets}{category of sets}. Let $h^A: \mathcal{C} \to \mathbf{Set}$ denote the covariant \CrefAndHyperrefIfExist{definition:representable_functor_on_a_locally_small_category}{representable functor} defined by $h^A(X) = \operatorname{Hom}_{\mathcal{C}}(A, X)$.

There exists a bijection
$$y_{A, F} : \operatorname{Nat}(h^A, F) \xrightarrow{\cong} F(A)$$
between the set of \CrefAndHyperrefIfExist{definition:natural_transformation_between_functors_between_categories}{natural transformations} from $h^A$ to $F$ and the set $F(A)$. This bijection is given by the mapping
$$\alpha \mapsto \alpha_A(\operatorname{id}_A),$$
where $\alpha: h^A \to F$ is a natural transformation, $\alpha_A: h^A(A) \to F(A)$ is its component at $A$, and $\operatorname{id}_A \in h^A(A) = \operatorname{Hom}_{\mathcal{C}}(A, A)$ is the identity morphism.

Furthermore, this isomorphism is natural in both $A$ and $F$. Explicitly:
\begin{enumerate}
    \item For any morphism $f: A \to B$ in $\mathcal{C}$, the following diagram commutes:
    \begin{center}
    \begin{tikzcd}[row sep=large, column sep=large]
        \operatorname{Nat}(h^B, F) \arrow[r, "y_{B,F}"] \arrow[d, "-\circ h^f"'] & F(B) \arrow[d, "F(f)"] \\
        \operatorname{Nat}(h^A, F) \arrow[r, "y_{A,F}"] & F(A)
    \end{tikzcd}
    \end{center}
    where $h^f: h^B \to h^A$ is the natural transformation induced by pre-composition with $f$.

    \item For any natural transformation $\eta: F \to G$, the following diagram commutes:
    \begin{center}
    \begin{tikzcd}[row sep=large, column sep=large]
        \operatorname{Nat}(h^A, F) \arrow[r, "y_{A,F}"] \arrow[d, "\eta \circ -"'] & F(A) \arrow[d, "\eta_A"] \\
        \operatorname{Nat}(h^A, G) \arrow[r, "y_{A,G}"] & G(A)
    \end{tikzcd}
    \end{center}
\end{enumerate}
\end{theorem}
\begin{corollary}[Yoneda Embedding] \label{corollary:yoneda_embedding_on_a_locally_small_category}
Let $\mathcal{C}$ be a \CrefAndHyperrefIfExist{definition:locally_small_category}{locally small category}. The functor
$$h^\bullet: \mathcal{C}^{\operatorname{op}} \to \mathbf{Set}^{\mathcal{C}}$$
\CrefIfExists{definition:opposite_category_of_a_category} \CrefIfExists{definition:diagram_in_a_category_indexed_by_a_small_category}
defined on objects by $A \mapsto h^A = \operatorname{Hom}_{\mathcal{C}}(A, -)$\CrefIfExists{definition:representable_functor_on_a_locally_small_category} and on morphisms by $f \mapsto h^f = (-\circ f)$ is \CrefAndHyperrefIfExist{definition:full_and_faithful_functor_between_locally_small_categories}{fully faithful}. That is, for any objects $A, B$ in $\mathcal{C}$, the map
$$\operatorname{Hom}_{\mathcal{C}}(A, B) \to \operatorname{Nat}(h^B, h^A)$$
given by sending a morphism $f: A \to B$ to the \CrefAndHyperrefIfExist{definition:natural_transformation_between_functors_between_categories}{natural transformation} $h^f: h^B \to h^A$ (pre-composition by $f$) is a bijection.

Consequently, $\mathcal{C}^{\operatorname{op}}$ embeds as a \CrefAndHyperrefIfExist{definition:full_subcategory_of_a_category}{full subcategory} of the functor category $\mathbf{Set}^{\mathcal{C}}$.
\end{corollary}

\begin{definition}[Generator of a category] \label{definition:generator_of_a_category}
Let \(\mathcal{C}\) be a \CrefAndHyperrefIfExist{definition:category}{category}. 
\begin{enumerate}
    \item  An object \(G \in \mathcal{C}\) is called a \hldef{generator} (or \hldef{separator}) if for every pair of distinct morphisms \(f, g : X \to Y\) in \(\mathcal{C}\), there exists a morphism \(h : G \to X\) such that
    \[
    f \circ h \neq g \circ h.
    \]
    In case that $\calC$ is \CrefAndHyperrefIfExist{definition:locally_small_category}{locally small}, this is equivalent to the condition that the \CrefAndHyperrefIfExist{definition:representable_functor_on_a_category_enriched_in_a_monoidal_category}{representable functor}
    \[
    \mathrm{Hom}_{\mathcal{C}}(G, -) : \mathcal{C} \to \mathbf{Set}
    \]
    is \CrefAndHyperrefIfExist{definition:full_and_faithful_functor_between_locally_small_categories}{faithful}, 
    %
    % In other words, for every pair of distinct morphisms \(f, g : X \to Y\) in \(\mathcal{C}\), there exists a morphism \(h : G \to X\) such that
    % \[
    % f \circ h \neq g \circ h.
    % \]
    %
    which in turn is equivalent to the condition that for every object \(X \in \mathcal{C}\), there exists an epimorphism
    \[
    \bigoplus_{i \in I} G \twoheadrightarrow X
    \]
    for some indexing set \(I\), where \(\bigoplus\) denotes the \CrefAndHyperrefIfExist{definition:product_and_coproduct_of_objects_in_a_category}{coproduct} in \(\mathcal{C}\).

    \item A family \(\{G_i\}_{i \in I}\) is called a \hldef{generating family} if for every pair of distinct morphisms \(f, g : X \to Y\) in \(\mathcal{C}\), there exists some index \(i \in I\) and a morphism \(h : G_i \to X\) such that
    \[
    f \circ h \neq g \circ h.
    \]
    In case $\calC$ is locally small, this is equivalent to the condition that the collection of representable functors
    \[
    \{\mathrm{Hom}_{\mathcal{C}}(G_i, -) : \mathcal{C} \to \mathbf{Set}\}_{i \in I}
    \]
    is jointly faithful, which in turn is equivalent to the condition that for every object \(X \in \mathcal{C}\), there exists a family of objects \(\{G_i\}_{i \in J}\) from the generating set indexed by some set \(J\), and an epimorphism
    \[
    \bigoplus_{i \in J} G_i \twoheadrightarrow X.
    \]

\end{enumerate}
\end{definition}


\begin{example}
\begin{enumerate}
  \item In the category of abelian groups \(\mathbf{Ab}\), the group \(\mathbb{Z}\) is a generator, as homomorphisms out of \(\mathbb{Z}\) detect morphisms, and every abelian group is a quotient of a direct sum of copies of \(\mathbb{Z}\).

  \item In the category of sets \(\mathbf{Set}\), the one-point set \(\{*\}\) is a generator.

  \item In the category of left modules over a ring \(R\), denoted \(R\text{-}\mathbf{Mod}\), the free module \(R\) is a generator.

  \item In the category of sheaves of \(\mathcal{O}\)-modules on a ringed space or site, stalkwise generators often induce global generators, e.g., the sheaf \(\mathcal{O}\) itself is a generator in \(\mathrm{Mod}(\mathcal{O})\).

\end{enumerate}
\end{example}



\begin{definition}[Grothendieck's axioms for abelian categories (Ab1--Ab5)] \label{definition:grothendiecks_additional_axioms_for_abelian_categories}
Let $\mathcal{A}$ be an \CrefAndHyperrefIfExist{definition:abelian_category}{abelian category}.

Grothendieck introduced the following hierarchy of additional axioms to express stronger completeness and exactness properties in $\mathcal{A}$ --- we note that Ab1, Ab2, and Ab2\textsuperscript{*} are already satisfied for any abelian category:

\begin{itemize}

  \item \hldef{Ab1}: Every morphism in $\calA$ has a \CrefAndHyperrefIfExist{definition:kernel_and_cokernel_of_a_morphism_in_a_category}{kernel and a cokernel}.
  \item \hldef{Ab2}: Every \CrefAndHyperrefIfExist{definition:monomorphism_and_epimorphism_in_categories}{monic} in $\calA$ is the kernel of its cokernel. 
  \item \hldef{Ab2\textsuperscript{*}}: Every epi in $\calA$ is the cokernel of its kernel. 

  \item \hldef{AB3}: The category $\mathcal{A}$ is \CrefAndHyperrefIfExist{definition:complete_and_cocomplete_category}{cocomplete}.
  \begin{itemize}
    \item Since $\mathcal{A}$ is abelian (and hence \CrefAndHyperrefIfExist{lemma:equalizer_coequalizer_in_an_additive_category_are_given_by_kernel_and_cokernel}{admits} \CrefAndHyperrefIfExist{definition:equalizer_and_coequalizer_of_morphisms_in_a_category}{equalizers} as \CrefAndHyperrefIfExist{definition:kernel_image_cokernel_coimage_of_a_module_homomorphism}{kernels}), this is equivalent to requiring that $\mathcal{A}$ has all small \CrefAndHyperrefIfExist{definition:product_and_coproduct_of_objects_in_a_category}{coproducts} (direct sums).
  \end{itemize}

  \item \hldef{AB4}: The category $\mathcal{A}$ satisfies AB3, and coproducts are \emph{exact}.
  \begin{itemize}
    \item That is, the coproduct of a family of short exact sequences is a short exact sequence. Explicitly, for any family of short exact sequences $0 \to A_i \to B_i \to C_i \to 0$ indexed by a set $I$, the sequence
    \[ 0 \to \bigoplus_{i \in I} A_i \to \bigoplus_{i \in I} B_i \to \bigoplus_{i \in I} C_i \to 0 \]
    is exact in $\mathcal{A}$.
  \end{itemize}

  \item \hldef{AB5}: The category $\mathcal{A}$ satisfies AB3, and \CrefAndHyperrefIfExist{definition:projective_and_inductive_limits_in_categories}{filtered colimits} are \emph{exact}.
  \begin{itemize}
    \item Equivalently, for any \CrefAndHyperrefIfExist{definition:filtered_cofiltered_category}{filtered} index category $J$ and any \CrefAndHyperrefIfExist{definition:system_in_a_category_indexed_by_a_directed_poset}{directed system} of short exact sequences $0 \to A_j \to B_j \to C_j \to 0$, the colimit sequence
    \[ 0 \to \varinjlim A_j \to \varinjlim B_j \to \varinjlim C_j \to 0 \]
    is exact.
    \item Note: AB5 implies AB4. An abelian category satisfying AB5 and having a \CrefAndHyperrefIfExist{definition:generator_of_a_category}{generator} is called a \hldef{Grothendieck category}.
  \end{itemize}
  
  \item \hldef{AB6}: The category $\mathcal{A}$ satisfies AB3, and for any object $X$ and any family of filtered subobjects $\{F_i\}_{i \in I}$ of $X$ (where each $F_i$ is a filter of subobjects), the intersection commutes with the limit:
  \[ \bigcap_{i \in I} (\varinjlim_{j \in F_i} U_{i,j}) = \varinjlim_{(j_i) \in \prod F_i} (\bigcap_{i \in I} U_{i, j_i}). \]
  (This axiom is less commonly cited but appears in Grothendieck's Tohoku paper).

  \item \hldef{AB3\textsuperscript{*}}: The category \(\mathcal{A}\) is \CrefAndHyperrefIfExist{definition:complete_and_cocomplete_category}{complete} (i.e., has all small products).

  \item \hldef{AB4\textsuperscript{*}}: The category \(\mathcal{A}\) satisfies AB3\textsuperscript{*} and products are exact.
  \begin{itemize}
    \item Note: This is rarely satisfied for module categories (e.g., it fails for Abelian groups), but it is satisfied for the category of sheaves on a space.
  \end{itemize}

  \item \hldef{AB5\textsuperscript{*}}: The category \(\mathcal{A}\) satisfies AB3\textsuperscript{*} and filtered limits (inverse limits) are exact.
\end{itemize}

\textbf{Notes:}
\begin{itemize}
  \item AB5 implies AB4, and AB4 implies AB3.
  \item AB5\textsuperscript{*} implies AB4\textsuperscript{*}, and AB4\textsuperscript{*} implies AB3\textsuperscript{*}.
\end{itemize}
\end{definition}


% \begin{definition}[Grothendieck's axioms AB3--AB6] \label{definition:grothendiecks_ab_axioms}
% Let $\mathcal{A}$ be an \CrefAndHyperrefIfExist{definition:abelian_category}{abelian category}. (Recall that the axioms AB1 and AB2 refer to the existence of kernels/cokernels and the isomorphism between coimage and image, which are part of the definition of an abelian category).

% Grothendieck introduced the following hierarchy of additional axioms to express stronger completeness and exactness properties in $\mathcal{A}$:

% \begin{itemize}
%   \item \hldef{AB3}: The category $\mathcal{A}$ is \CrefAndHyperrefIfExist{definition:complete_and_cocomplete_category}{cocomplete}.
%   \begin{itemize}
%     \item Since $\mathcal{A}$ is abelian (hence has finite colimits), this is equivalent to requiring that $\mathcal{A}$ has all small \CrefAndHyperrefIfExist{definition:product_and_coproduct_of_objects_in_a_category}{coproducts} (direct sums).
%   \end{itemize}

%   \item \hldef{AB4}: The category $\mathcal{A}$ satisfies AB3, and coproducts are \emph{exact}.
%   \begin{itemize}
%     \item That is, the coproduct of a family of short exact sequences is a short exact sequence. Explicitly, for any family of short exact sequences $0 \to A_i \to B_i \to C_i \to 0$ indexed by a set $I$, the sequence
%     \[ 0 \to \bigoplus_{i \in I} A_i \to \bigoplus_{i \in I} B_i \to \bigoplus_{i \in I} C_i \to 0 \]
%     is exact in $\mathcal{A}$.
%   \end{itemize}

%   \item \hldef{AB5}: The category $\mathcal{A}$ satisfies AB3, and \CrefAndHyperrefIfExist{definition:projective_and_inductive_limits_in_categories}{filtered colimits} are \emph{exact}.
%   \begin{itemize}
%     \item Equivalently, for any \CrefAndHyperrefIfExist{definition:filtered_cofiltered_category}{filtered} index category $J$ and any \CrefAndHyperrefIfExist{definition:system_in_a_category_indexed_by_a_directed_poset}{directed system} of short exact sequences $0 \to A_j \to B_j \to C_j \to 0$, the colimit sequence
%     \[ 0 \to \varinjlim A_j \to \varinjlim B_j \to \varinjlim C_j \to 0 \]
%     is exact.
%     \item Note: AB5 implies AB4. An abelian category satisfying AB5 and having a \CrefAndHyperrefIfExist{definition:generator_of_a_category}{generator} is called a \hldef{Grothendieck category}.
%   \end{itemize}
  
%   \item \hldef{AB6}: The category $\mathcal{A}$ satisfies AB3, and for any object $X$ and any family of filtered subobjects $\{F_i\}_{i \in I}$ of $X$ (where each $F_i$ is a filter of subobjects), the intersection commutes with the limit:
%   \[ \bigcap_{i \in I} (\varinjlim_{j \in F_i} U_{i,j}) = \varinjlim_{(j_i) \in \prod F_i} (\bigcap_{i \in I} U_{i, j_i}). \]
%   (This axiom is less commonly cited but appears in Grothendieck's Tohoku paper).

%   \item \hldef{AB3\textsuperscript{*}}: The category \(\mathcal{A}\) is \CrefAndHyperrefIfExist{definition:complete_and_cocomplete_category}{complete} (i.e., has all small products).

%   \item \hldef{AB4\textsuperscript{*}}: The category \(\mathcal{A}\) satisfies AB3\textsuperscript{*} and products are exact.
%   \begin{itemize}
%     \item Note: This is rarely satisfied for module categories (e.g., it fails for Abelian groups), but it is satisfied for the category of sheaves on a space.
%   \end{itemize}

%   \item \hldef{AB5\textsuperscript{*}}: The category \(\mathcal{A}\) satisfies AB3\textsuperscript{*} and filtered limits (inverse limits) are exact.
% \end{itemize}

% \textbf{Notes:}
% \begin{itemize}
%   \item AB5 implies AB4, and AB4 implies AB3.
%   \item AB5\textsuperscript{*} implies AB4\textsuperscript{*}, and AB4\textsuperscript{*} implies AB3\textsuperscript{*}.
%   \item The condition you originally listed as "Ab2" (disjoint/universal sums) characterizes \emph{extensive categories} or \emph{toposes}, not abelian categories. In an abelian category, the coproduct is a biproduct and is never "disjoint" in the sense of set theory (unless $0=1$).
% \end{itemize}
% \end{definition}


\begin{lemma} \label{lemma:Ab5_category_with_a_family_of_generators_is_grothendieck}
    Let $\calA$ be an \CrefAndHyperrefIfExist{definition:abelian_category}{abelian category} satisfying \CrefAndHyperrefIfExist{definition:grothendiecks_additional_axioms_for_abelian_categories}{Ab5} and which admits a \CrefAndHyperrefIfExist{definition:locally_small_category}{small} family $\{U_i\}_{i \in I}$ of \CrefAndHyperrefIfExist{definition:generator_of_a_category}{generators}. The coproduct $\bigoplus_{i \in I} U_i$ is in fact a generator and hence $\calA$ is a \CrefAndHyperrefIfExist{definition:grothendiecks_additional_axioms_for_abelian_categories}{Grothendieck category}.
\end{lemma}

\begin{theorem}[Examples of Grothendieck Categories] \label{theorem:examples_of_grothendieck_categories}
Examples of \CrefAndHyperrefIfExist{definition:grothendiecks_additional_axioms_for_abelian_categories}{Grothendieck categories} include:
\begin{itemize}
    \item The category of abelian groups,
    \item The category of $R$-$S$ bimodules where $R$,$S$ are \CrefAndHyperrefIfExist{definition:ring}{(not necessarily commutative) rings} (\Cref{theorem:the_category_of_R_S_bimodules_is_a_grothendieck_abelian_category_and_AB4_star})
    \item The category of \CrefAndHyperrefIfExist{definition:sheaf_on_a_site}{sheaves} of abelian groups on a \CrefAndHyperrefIfExist{definition:grothendieck_topology_on_a_category_site_covering_sieve_topologically_generating_family}{site} with a small \CrefAndHyperrefIfExist{definition:grothendieck_topology_on_a_category_site_covering_sieve_topologically_generating_family}{topologically generating family},
    \item The category of sheaves of \CrefAndHyperrefIfExist{definition:module_over_a_sheaf_of_rings_on_a_site}{$O_X$-modules} for any \CrefAndHyperrefIfExist{definition:ringed_space}{ringed space} $(X, O_X)$.
    \item The category of quasi-coherent sheaves on a scheme or algebraic stack. \TODO{quasi-coherent sheaves}
    \TODO{I need to figure out if for sheaves of abelian groups/sheaves of $\calO$-modules whether essentially smallness of the site is really necessary}
    \item The category of \CrefAndHyperrefIfExist{definition:sheaf_on_a_site}{sheaves} of abelian groups on an  \CrefAndHyperrefIfExist{definition:essentially_small_category}{essentially small} \CrefAndHyperrefIfExist{definition:grothendieck_topology_on_a_category_site_covering_sieve_topologically_generating_family}{site} $(C,J)$.

    \item (\cite[Expos\'e II, Proposition 6.7]{SGA4_I}) The category of sheaves of $\calO$-modules on an essentially small site (or an essentially $\mathscr{U}$-small site if a universe $\mathscr{U}$ is available) $(C,J)$. 
\end{itemize}
\end{theorem}

\begin{theorem} \label{theorem:the_category_of_R_S_bimodules_is_a_grothendieck_abelian_category_and_AB4_star}
    Let $R,S$ be \CrefAndHyperrefIfExist{definition:ring}{(not necessarily commutative) rings}.
    The \CrefAndHyperrefIfExist{definition:category_of_modules_and_bimodules_over_rings}{category of} \CrefAndHyperrefIfExist{definition:module_of_a_ring}{$R$-$S$-bimodules} is a an \CrefAndHyperrefIfExist{definition:grothendiecks_additional_axioms_for_abelian_categories}{Grothendieck} category and an \CrefAndHyperrefIfExist{definition:grothendiecks_additional_axioms_for_abelian_categories}{$AB4*$} category.
\end{theorem}

\begin{proof}
    We handwave details.

    Given $R$-$S$-bimodules $M$ and $N$, the set $\operatorname{Hom}_{{}_R \mathrm{Mod}_S}(M,N)$ is an abelian group. Moreover, there is a $0$-object, namely the zero module in ${}_R \mathrm{Mod}_S$. Therefore, ${}_R \mathrm{Mod}_S$ is \CrefAndHyperrefIfExist{definition:additive_category}{preadditive}. The direct sum of finitely many $R$-$S$-bimodules is their \CrefAndHyperrefIfExist{definition:coproduct_of_modules_of_rings}{coproduct}. Therefore, ${}_R \mathrm{Mod}_S$ is \CrefAndHyperrefIfExist{definition:additive_category}{additive}.

    Given a \CrefAndHyperrefIfExist{definition:homomorphism_of_modules_over_a_ring}{morphism} $f: M \to N$ be $R$-$S$-bimodules, \CrefAndHyperrefIfExist{definition:kernel_image_cokernel_coimage_of_a_module_homomorphism}{$\ker f$ and $\operatorname{coker} f$} are the \CrefAndHyperrefIfExist{definition:kernel_and_cokernel_of_a_morphism_in_a_category}{categorical kernel and cokernel} of $f$ in \CrefAndHyperrefIfExist{definition:category_of_modules_and_bimodules_over_rings}{${}_R \mathrm{Mod}_S$}. Moreover, the \CrefAndHyperrefIfExist{definition:monomorphism_and_epimorphism_in_categories}{monomorphisms} in ${}_R \mathrm{Mod}_S$ are the injective module homomorphisms $f: M \to N$; such an $f$ is the kernel of its cokernel. In other words, ${}_R \mathrm{Mod}_S$ satisfies $AB1$ and $AB2$ and hence is an \CrefAndHyperrefIfExist{definition:abelian_category}{abelian category}.

    Moreover, small \CrefAndHyperrefIfExist{definition:product_and_coproduct_of_objects_in_a_category}{coproducts} exist in ${}_R \mathrm{Mod}_S$ (\Cref{definition:coproduct_of_modules_of_rings}), and it is easy to see that they are in fact exact, so ${}_R \mathrm{Mod}_S$ satisfies $AB3$ and $AB4$. To show that filtered colimits in ${}_R \mathrm{Mod}_S$ are exact, we first note that \CrefAndHyperrefIfExist{definition:small_and_finite_limits_and_colimits_in_a_category}{small} \CrefAndHyperrefIfExist{definition:projective_and_inductive_limits_in_categories}{colimits} \CrefAndHyperrefIfExist{theorem:limit_and_colimit_are_left_right_adjoint_to_diagonal_functor_for_locally_small_base_and_small_index}{are} \CrefAndHyperrefIfExist{definition:adjoint_functors_between_categories_unit_counit_of_adjoint_functors}{left adjoint} and \CrefAndHyperrefIfExist{proposition:left_right_adjoint_is_right_left_exact}{hence} is \CrefAndHyperrefIfExist{definition:exact_functor_between_abelian_categories}{right exact}; for any small index category $J$ and any system of short exact sequences $0 \to A_j \to B_j \to C_j \to 0$, the sequence $0 \to A \to B \to C \to 0$ of \CrefAndHyperrefIfExist{definition:diagram_in_a_category_indexed_by_a_small_category}{diagrams} is exact by \Cref{proposition:exact_sequence_in_a_diagram_category_of_an_abelian_category_indexed_by_a_small_category_is_given_by_componentwise_exact_sequences}, so applying the \CrefAndHyperrefIfExist{theorem:limits_and_colimits_as_functors_from_functor_category_to_value_category}{colimit functor} yields a right exact sequence 
    $$\colim_{j \in J} A_j \to \colim_{j \in J} B_j \to \colim_{j \in J} C_j \to 0.$$ 
    If $J$ is additionally \CrefAndHyperrefIfExist{definition:filtered_cofiltered_category}{filtered} so that the system is a \CrefAndHyperrefIfExist{definition:system_in_a_category_indexed_by_a_directed_poset}{directed} one, then we show that the sequence
    \begin{equation} \label{eq:right_exact_sequence_of_direct_system_of_exact_ring_modules}
    \varinjlim_{j \in J} A_j \to \varinjlim_{j \in J} B_j \to \varinjlim_{j \in J} C_j \to 0.
    \end{equation}
    is left exact as well. Take some element of the kernel of $\varinjlim_{j \in J} A_j \to \varinjlim_{j \in J} B_j$; such an element is represented by some element $a_j \in A_j$ for some $j \in J$. Since its image is zero in $\varinjlim{j \in J} B_j$, it must be zero as an element of $B_k$ for some $k \in J$. Since $J$ is filtered, there exists some $k' \in J$ so that there are arrows $j \to k'$ and $k \to k'$ and so that the image of $a_j$ in $B_{k'}$ is $0$. The image of $a_j$ in $A_{k'}$ is then $0$ due to the assumption that 
    $$0 \to A_{k'} \to B_{k'} \to C_{k'} \to 0$$
    is exact. Therefore, $a_j$ is $0$ in $\varinjlim_{j \in J} A_j$, so \eqref{eq:right_exact_sequence_of_direct_system_of_exact_ring_modules} is left exact as claimed and ${}_R \mathrm{Mod}_S$ is $AB5$.

    Similarly as how we argued that ${}_R \mathrm{Mod}_S$ is $AB3$ and $AB4$, one can argue that ${}_R \mathrm{Mod}_S$ is $AB3^*$ and $AB4^*$. Moreover, one can show that the $R$-$S$-bimodule $R \otimes_{\bbZ} S^{\op}$\CrefIfExists{definition:tensor_product_of_a_ring_and_an_algebra_over_a_ring}\CrefIfExists{definition:opposite_ring_of_a_ring} is a generator for ${}_R \mathrm{Mod}_S$.
\end{proof}


\begin{definition}[Topological groups]  \label{definition:topological_group}
    \TODO{Product topology}
    A \hldef{topological group} is a \CrefAndHyperrefIfExist{definition:group}{group} $(G,\cdot)$ together with a \CrefAndHyperrefIfExist{definition:topological_space}{topology} $\mathcal{T}$ on $G$ such that the maps
    \begin{align*}
    \mu : G \times G &\to G, & (g,h) &\mapsto g \cdot h, \\
    \iota : G &\to G, & g &\mapsto g^{-1},
    \end{align*}
    are \CrefAndHyperrefIfExist{definition:continuous_map_between_open_subsets_of_euclidean_spaces}{continuous} with respect to the product topology on $G \times G$ and the topology $\mathcal{T}$ on $G$.
\end{definition}


\begin{definition}[Compact topological space] \label{definition:compact_topological_space}
A topological space $(X, \mathcal{T})$ is \hldef{compact} if every open cover of $X$ admits a finite subcover; that is, for every collection $\{ U_i \}_{i \in I}$ of open sets in $\mathcal{T}$ such that $X = \bigcup_{i \in I} U_i$, there exists a finite subcollection $\{ U_{i_j} \}_{j=1}^n$ such that $X = \bigcup_{j=1}^n U_{i_j}$.

Some mathematicians, e.g. algebraic geometers, would refer to this property as \hldef{quasi-compactness}.
\end{definition}
\begin{definition}[Separation axioms] \label{definition:separation_axioms_of_topology}
Let $(X,\mathcal{T})$ be a \CrefAndHyperrefIfExist{definition:topological_space}{topological space}.
\begin{itemize}
  \item $(X,\mathcal{T})$ is \hldef{T$_0$} (\hldef{Kolmogorov}) if for every pair of distinct points $x, y \in X$, there exists an open set $U \in \mathcal{T}$ such that, without loss of generality, $x \in U$ and $y \notin U$.
  \item $(X,\mathcal{T})$ is \hldef{T$_1$} (\hldef{Fréchet}) if for every pair of distinct points $x, y \in X$, there exist open sets $U, V \in \mathcal{T}$ such that $x \in U$, $y \notin U$, and $y \in V$, $x \notin V$.
  \item $(X,\mathcal{T})$ is \hldef{T$_2$} or \hldef{Hausdorff} if for every pair of distinct points $x, y \in X$, there exist disjoint open sets $U, V \in \mathcal{T}$ such that $x \in U$ and $y \in V$.
  \item $(X,\mathcal{T})$ is \hldef{regular} if it is T$_1$ and for each point $x \in X$ and closed set $F \subseteq X$ with $x \notin F$, there exist disjoint open sets $U, V \in \mathcal{T}$ such that $x \in U$ and $F \subseteq V$.
  \item $(X,\mathcal{T})$ is \hldef{T$_3$} (regular Hausdorff) if it is T$_1$ and regular.
  \item $(X,\mathcal{T})$ is \hldef{completely regular} if for each closed set $F \subseteq X$ and $x \notin F$, there exists a continuous function $f : X \to [0,1]$ such that $f(x) = 0$ and $f|_F = 1$.
  \item $(X,\mathcal{T})$ is \hldef{T$_{3\frac{1}{2}}$} (completely regular Hausdorff) if it is T$_1$ and completely regular.
  \item $(X,\mathcal{T})$ is \hldef{normal} if it is T$_1$ and for each pair of disjoint closed sets $A, B \subseteq X$, there exist disjoint open sets $U, V \in \mathcal{T}$ such that $A \subseteq U$ and $B \subseteq V$.
  \item $(X,\mathcal{T})$ is \hldef{T$_4$} (normal Hausdorff) if it is T$_1$ and normal.
   \item $(X,\mathcal{T})$ is \hldef{T$_5$} (completely normal Hausdorff) if it is T$1$ and completely normal.
    \item $(X,\mathcal{T})$ is \hldef{perfectly normal} if every closed set is a \CrefAndHyperrefIfExist{definition:G_delta_set_and_F_sigma_set_of_a_topological_space}{$G\delta$} (\CrefAndHyperrefIfExist{definition:countable_finite_uncountable_sets}{countable} intersection of open sets) and the space is normal.
    \item $(X,\mathcal{T})$ is \hldef{T$_6$} (perfectly normal Hausdorff) if it is T$_1$ and perfectly normal.
\end{itemize}
\end{definition}
\begin{definition} \label{definition:locally_compact_group}
    A \hldef{locally compact group} is an topological group that is locally compact Hausdorff. It is called a \hldef{locally compact abelian group} if it is abelian as well.
\end{definition}


\begin{definition}[Field] \label{definition:field}
A \hldef{field} is commutative \CrefAndHyperrefIfExist{definition:division_ring}{division} \CrefAndHyperrefIfExist{definition:commutative_ring}{ring}. In other words, a field is a commutative ring for which all nonzero elements have a \CrefAndHyperrefIfExist{definition:unit_of_a_ring}{multiplicative inverse}.
\end{definition}


\begin{definition}[Vector space over a field] \label{definition:vector_space_over_a_field}
    Let $(k,+,\cdot)$ be a \CrefAndHyperrefIfExist{definition:field}{field}. A \hldef{vector space over $k$} or a \hldef{$k$-vector space} is a triple $(V,+,\cdot)$\footnote{Note that $+$ and $\cdot$ are abuse of notation here as these are already used for the addition and multiplication of $\cdot$} where 
    \begin{enumerate}
        \item $(V,+)$ is an abelian group, and
        \item $\cdot$ is a map $k \times V \to V$, called \hldef{scalar multiplication}
    \end{enumerate}
    such that the following axioms hold for all $a,b \in k$ and all $u,v \in V$:

    1. (Compatibility with field multiplication)  
    $$ (ab)\cdot v = a \cdot (b \cdot v). $$

    2. (Identity scalar)  
    $$ 1 \cdot v = v. $$

    3. (Distributivity over vector addition)  
    $$ a \cdot (u+v) = a \cdot u + a \cdot v. $$

    4. (Distributivity over scalar addition)  
    $$ (a+b) \cdot v = a \cdot v + b \cdot v. $$
\end{definition}


\begin{theorem} \label{theorem:category_of_compact_hausdorff_abelian_groups_is_AB5_star}
    The category of \CrefAndHyperrefIfExist{definition:compact_topological_space}{compact} \CrefAndHyperrefIfExist{definition:separation_axioms_of_topology}{Hausdorff} abelian \CrefAndHyperrefIfExist{definition:topological_group}{topological groups} is \CrefAndHyperrefIfExist{definition:grothendiecks_additional_axioms_for_abelian_categories}{AB$5^*$}.
\end{theorem}
\begin{theorem} \label{theorem:category_of_k_vector_spaces_is_AB_5_star}
    Let $k$ be a \CrefAndHyperrefIfExist{definition:field}{field}. The category of \CrefAndHyperrefIfExist{definition:vector_space_over_a_field}{$k$-vector spaces} is \CrefAndHyperrefIfExist{definition:grothendiecks_additional_axioms_for_abelian_categories}{AB$5^*$} and AB$5$.
\end{theorem}
\begin{theorem} \label{theorem:category_of_R_modules_is_AB_4_star}
    Let $R$ be a \CrefAndHyperrefIfExist{definition:ring}{(not necessarily commutative) ring}. The category of \CrefAndHyperrefIfExist{definition:vector_space_over_a_field}{(left or right) $R$-modules} is \CrefAndHyperrefIfExist{definition:grothendiecks_additional_axioms_for_abelian_categories}{AB$4^*$}.
\end{theorem}
\begin{remark}
    In general, the category of $R$-modules is not {AB$5^*$}. In general, the category of $\calO$-modules for a ringed site $(\calC, \calO)$ is not AB$5^*$.
\end{remark}

\begin{example}
    The following example shows that for general rings $R$, the category of $R$-modules is not generally \CrefAndHyperrefIfExist{definition:grothendiecks_additional_axioms_for_abelian_categories}{AB$5^*$}: let $R = \bbZ$, fix a prime number $p$, and let
    $$0 \to A_n \to B_n \to C_n \to 0$$
    be the following system of short exact sequences indexed by $n \geq 0$: 
    \begin{itemize}
        \item $A_n = \bbZ$ with transition maps $f: A_{n+1} \to A_n$ given by multiplication-by-$p$,
        \item $B_n = \bbZ$ with transition maps given by the identity, and
        \item $C_n = \bbZ/p^n \bbZ$ with transition maps given by the canonical projections.
    \end{itemize}
    Taking limits, 
    \begin{itemize}
        \item $\varprojlim A_n = 0$,
        \item $\varprojlim B_n = \bbZ$, and 
        \item $\varprojlim C_n = \bbZ_p$,
    \end{itemize}
    so
    $$0 \to \varprojlim A_n \to \varprojlim B_n \to \varprojlim C_n \to 0$$
    fails to be exact.
\end{example}


\begin{proposition}[Stability and constructions of Grothendieck categories] \ \label{proposition:stability_and_constructions_of_grothendieck_categories}

\begin{itemize}
    \item Any category that is \emph{equivalent} to a \CrefAndHyperrefIfExist{definition:grothendiecks_additional_axioms_for_abelian_categories}{Grothendieck category} is itself a Grothendieck category.
    \item Given Grothendieck categories \( \mathcal{A}_1, \ldots, \mathcal{A}_n \), the \CrefAndHyperrefIfExist{definition:product_category_of_a_family_of_categories}{product category}
    \[
    \mathcal{A}_1 \times \cdots \times \mathcal{A}_n
    \]
    is a Grothendieck category.
    \item Given a small category \(\mathcal{C}\) and a Grothendieck category \(\mathcal{A}\), the functor category
    \[
    \mathrm{Funct}(\mathcal{C}, \mathcal{A})
    \]
    consisting of all covariant functors from \(\mathcal{C}\) to \(\mathcal{A}\), is a Grothendieck category.
    \item Given a small preadditive category \(\mathcal{C}\) and a Grothendieck category \(\mathcal{A}\), the category of additive covariant functors
    \[
    \mathrm{Add}(\mathcal{C}, \mathcal{A})
    \]
    is a Grothendieck category.
    \item If \(\mathcal{A}\) is a Grothendieck category and \(\mathcal{C}\) is a localizing subcategory of \(\mathcal{A}\), then both \(\mathcal{C}\) and the Serre quotient category \(\mathcal{A} / \mathcal{C}\) are Grothendieck categories.
\end{itemize}
\end{proposition}






\subsection{Miscellaneous definitions}


\begin{definition} \label{definition:strict_abelian_subcategory_of_an_abelian_category}
Let $\mathcal{A}$ be an \CrefAndHyperrefIfExist{definition:abelian_category}{abelian category}. A \CrefAndHyperrefIfExist{definition:full_subcategory_of_a_category}{full subcategory} $\mathcal{B} \subseteq \mathcal{A}$ is called a \hldef{strict abelian subcategory of $\mathcal{A}$} if the following hold:

\begin{enumerate}
    \item $\mathcal{B}$ is itself an abelian category.
    \item The inclusion functor $i: \mathcal{B} \hookrightarrow \mathcal{A}$ is \CrefAndHyperrefIfExist{definition:exact_functor_between_abelian_categories}{exact}.
    \item For any morphism $f \in \mathrm{Hom}_{\mathcal{B}}(X,Y)$ with $X,Y \in \mathrm{Ob}(\mathcal{B})$, the \CrefAndHyperrefIfExist{definition:kernel_and_cokernel_of_a_morphism_in_a_category}{kernel and cokernel} of $f$ computed in $\mathcal{A}$ coincide with the kernel and cokernel of $f$ computed in $\mathcal{B}$.
\end{enumerate}
\end{definition}


\section{Chain complexes}


\subsection{The category of chain complexes of objects in an additive category}

% \begin{definition}[Chain complex in an additive category] \label{definition:chain_complex_of_objects_in_an_additive_category}
% Let $\mathcal{A}$ be an \hyperrefIfExists{definition:additive_category_preadditive_category}{preadditive category} and let $I$ be a totally ordered set (typically $\mathbb{Z}$, but $I \subseteq \mathbb{Z}$ is also allowed). 
% \begin{enumerate}
%     \item A \hldef{chain complex} $(K^\bullet, d^\bullet)$ in $\mathcal{A}$ indexed by $I$ consists of:
%     \begin{itemize}
%         \item Objects $\{ K^i \}_{i \in I}$ in $\mathcal{A}$, called the \hldef{terms in degree $i$},
%         \item Morphisms $d^i: K^i \to K^{i+1}$ in $\mathcal{A}$, called the \hldef{differentials in degree $i$},
%     \end{itemize}
%     such that for every $i \in I$, $d^{i+1} \circ d^i = 0$. That is,
%     $$ K^\bullet: \cdots \xrightarrow{d^{i-2}} K^{i-1} \xrightarrow{d^{i-1}} K^i \xrightarrow{d^i} K^{i+1} \xrightarrow{d^{i+1}} \cdots $$
%     with $d^{i+1}d^i = 0$ for all $i$. We might typically use notation such as \hl{$K^\bullet = (K^i, d^i)_{i \in I}$} to denote a chain complex in $\mathcal{A}$.

%     A cochain complex can be defined similarly/dually.

%     \item Let $K^\bullet = (K^i, d_K^i)$ and $L^\bullet = (L^i, d_L^i)$ be \CrefAndHyperrefIfExist{definition:chain_complex_of_objects_in_an_additive_category}{chain complexes} in $\mathcal{A}$ indexed by the same set $I$. 
%     A \hldef{morphism of chain complexes} (or \hldef{chain map})
%     $$ f^\bullet: K^\bullet \to L^\bullet $$
%     consists of morphisms $f^i: K^i \to L^i$ for all $i \in I$, such that for every $i \in I$,
%     $$ d_L^i \circ f^i = f^{i+1} \circ d_K^i, $$
%     i.e., the following diagram commutes for all $i$:

%     $$ \begin{array}{ccc} K^i & \xrightarrow{d_K^i} & K^{i+1} \\ \downarrow{f^i} && \downarrow{f^{i+1}} \\ L^i & \xrightarrow{d_L^i} & L^{i+1} \end{array}.$$

% \end{enumerate}

% There is then a category, often denoted by \hl{$\mathrm{Ch}(\mathcal{A})$} or \hl{$\mathbf{Ch}(\mathcal{A})$}, whose objects are chain complexes in $\calA$ and whose morphisms are morphisms of chain complexes. In particular, we may denote by 
% $$\hlin{\operatorname{Hom}(K^\bullet, L^\bullet)=  \operatorname{Hom}_{\mathrm{Ch}(\mathcal{A})}(K^\bullet, L^\bullet)}$$
% the set of chain maps $K^\bullet \to L^\bullet$; it is in fact an abelian group.

% A \hldef{morphism of cochain complexes} is defined similarly, and we similarly denote by \hl{$\mathrm{Ch}(\mathcal{A})$} or \hl{$\mathbf{Ch}(\mathcal{A})$} the caetgory of cochain complexes in $\calA$. 


% \TextIfExists{definition:dg_category_over_a_ring}{
% If $k$ is a \CrefAndHyperrefIfExist{definition:commutative_ring}{commutative ring} such that $\Hom_\calA(X,Y)$ is \CrefAndHyperrefIfExist{definition:category_enriched_in_a_monoidal_category}{enriched in} the category of \CrefAndHyperrefIfExist{definition:module_of_a_ring}{$k$-modules}, then $\mathrm{Ch}(\calA)$ \CrefAndHyperref{definition:category_of_chain_complexes_of_objects_in_an_additive_category_as_a_dg_category}{can be equipped with} the structure of a \CrefAndHyperrefIfExist{definition:dg_category_over_a_ring}{dg-category over $k$}.
% }


% \end{definition}

\begin{definition}[Chain complex in a preadditive category] \label{definition:chain_complex_of_objects_in_an_additive_category}
Let $\mathcal{A}$ be a \hyperrefIfExists{definition:additive_category_preadditive_category}{preadditive category} and let $I$ be a totally ordered set (typically $\mathbb{Z}$, but $I \subseteq \mathbb{Z}$ is also allowed). 
\begin{enumerate}
    \item A \hldef{chain complex} $(K_\bullet, d_\bullet)$ in $\mathcal{A}$ indexed by $I$ is the homological convention for sequences with decreasing degrees. It consists of:
    \begin{itemize}
        \item Objects $\{ K_i \}_{i \in I}$ in $\mathcal{A}$, called the \hldef{terms in degree $i$},
        \item Morphisms $d_i: K_i \to K_{i-1}$ in $\mathcal{A}$, called the \hldef{boundary maps} or \hldef{differentials in degree $i$},
    \end{itemize}
    such that for every $i \in I$, $d_{i-1} \circ d_i = 0$. That is,
    $$ K_\bullet: \cdots \xrightarrow{d_{i+1}} K_i \xrightarrow{d_i} K_{i-1} \xrightarrow{d_{i-1}} K_{i-2} \xrightarrow{} \cdots $$
    with $d_{i-1}d_i = 0$ for all $i$. We typically use the notation \hl{$K_\bullet = (K_i, d_i)_{i \in I}$}.



    \item Dually, a \hldef{cochain complex} $(K^\bullet, d^\bullet)$ in $\mathcal{A}$ follows the \hldef{cohomological convention} with increasing degrees. It consists of objects $\{ K^i \}_{i \in I}$ and \hldef{coboundary maps} $d^i: K^i \to K^{i+1}$ such that $d^{i+1} \circ d^i = 0$:
    $$ K^\bullet: \cdots \xrightarrow{d^{i-1}} K^i \xrightarrow{d^i} K^{i+1} \xrightarrow{d^{i+1}} K^{i+2} \xrightarrow{} \cdots $$
    We typically use the notation \hl{$K^\bullet = (K^i, d^i)_{i \in I}$}.

    \item Let $K_\bullet = (K_i, d_i^K)$ and $L_\bullet = (L_i, d_i^L)$ be \CrefAndHyperrefIfExist{definition:chain_complex_of_objects_in_an_additive_category}{chain complexes} in $\mathcal{A}$ indexed by the same set $I$. A \hldef{morphism of chain complexes} (or \hldef{chain map})
    $$ f_\bullet: K_\bullet \to L_\bullet $$
    consists of morphisms $f_i: K_i \to L_i$ for all $i \in I$, such that for every $i \in I$, the following diagram commutes:
    $$ \begin{array}{ccc} K_i & \xrightarrow{d_i^K} & K_{i-1} \\ \downarrow{f_i} && \downarrow{f_{i-1}} \\ L_i & \xrightarrow{d_i^L} & L_{i-1} \end{array} $$
    i.e., $d_i^L \circ f_i = f_{i-1} \circ d_i^K$. 



    A \hldef{morphism of cochain complexes} $f^\bullet: K^\bullet \to L^\bullet$ is defined similarly, satisfying the commutativity condition $d_L^i \circ f^i = f^{i+1} \circ d_K^i$.
\end{enumerate}

The collection of these objects and morphisms forms a category. Notation for these categories is as follows:
\begin{itemize}
    \item \hl{$\mathrm{Ch}(\mathcal{A})$} or \hl{$\mathbf{Ch}(\mathcal{A})$} is often used as a general term.
    \item To be explicit about the indexing convention, one uses \hl{$\mathrm{Ch}_\bullet(\mathcal{A})$} for chain complexes and \hl{$\mathrm{Ch}^\bullet(\mathcal{A})$} (or sometimes $\mathrm{CoCh}(\mathcal{A})$) for cochain complexes.
    \item The set of chain maps between two complexes is denoted by $\hlin{\operatorname{Hom}_{\mathrm{Ch}(\mathcal{A})}(K_\bullet, L_\bullet)}$; it is an abelian group under pointwise addition $(f+g)_i = f_i + g_i$.
\end{itemize}

\TextIfExists{definition:dg_category_over_a_ring}{
If $k$ is a \CrefAndHyperrefIfExist{definition:commutative_ring}{commutative ring} such that $\Hom_\calA(X,Y)$ is \CrefAndHyperrefIfExist{definition:category_enriched_in_a_monoidal_category}{enriched in} the category of \CrefAndHyperrefIfExist{definition:module_of_a_ring}{$k$-modules}, then $\mathrm{Ch}(\calA)$ \CrefAndHyperref{definition:category_of_chain_complexes_of_objects_in_an_additive_category_as_a_dg_category}{can be equipped with} the structure of a \CrefAndHyperrefIfExist{definition:dg_category_over_a_ring}{dg-category over $k$}.
}
\end{definition}

% 
\begin{remark} \label{remark:cohomological_vs_homological_conventions}
    The convention used to define chain complexes in \Cref{definition:chain_complex_of_objects_in_an_additive_category} is a \emph{cohomological one} --- note that indices are written as superscripts and increase when ``following the arrows''. Such a chain complex may also be referred to as a \hldef{cochain complex} or a \hldef{cohomological chain complex} to emphasize an adoption of a cohomological convention. 

    The dual convention would be a \emph{homological one}, in which indices are written as subscripts and decrease when ``following the arrow''. As such, one may speak of a \hldef{(homological) chain complex} $(K_\bullet, d_\bullet)$ indexed by $I$ as consisting of:

    \begin{itemize}
    \item Objects $\{ K_i \}_{i \in I}$ in $\mathcal{A}$, called the \hldef{terms in degree $i$},
    \item Morphisms $d_i: K_i \to K_{i-1}$ in $\mathcal{A}$, called the \hldef{differentials in degree $i$},
    \end{itemize}
    such that for every $i \in I$, $d_{i-1} \circ d_i = 0$. That is,
    $$ 
    K_\bullet: \cdots \xrightarrow{d_{i+1}} K_i \xrightarrow{d_i} K_{i-1} \xrightarrow{d_{i-1}} K_{i-2} \xrightarrow{d_{i-2}} \cdots
    $$
    with $d_{i-1} d_i = 0$ for all $i$. We might typically use notation such as \hl{$K_\bullet = (K_i, d_i)_{i \in I}$} to denote a chain complex in $\mathcal{A}$.

    The differences between the conventions persist --- for example, cohomological objects are usually written with superscript indicees whereas homological objects are usually written with subscript indicees.
\end{remark}
%\begin{convention} \label{convention:homological_algebra_is_discussed_in_cohomological_terms}
    When discussing homological algebra in abstract terms, we may often adopt the homological convention in some discussions and the cohomological convention in others \CrefIfExists{remark:cohomological_vs_homological_conventions}.
    %; for instance, indices are written as superscripts and increase along the direction of the arrows in chain complexes. 
\end{convention}

% \begin{definition}[Morphisms of chain complexes] \label{definition:chain_complex_of_objects_in_an_additive_category}
Let $\mathcal{A}$ be an \CrefAndHyperrefIfExist{definition:additive_category}{additive category}, and let $K^\bullet = (K^i, d_K^i)$ and $L^\bullet = (L^i, d_L^i)$ be \CrefAndHyperrefIfExist{definition:chain_complex_of_objects_in_an_additive_category}{chain complexes} in $\mathcal{A}$ indexed by the same set $I$. 
A \hldef{morphism of chain complexes} (or \hldef{chain map})
$$ f^\bullet: K^\bullet \to L^\bullet $$
consists of morphisms $f^i: K^i \to L^i$ for all $i \in I$, such that for every $i \in I$,
$$ d_L^i \circ f^i = f^{i+1} \circ d_K^i, $$
i.e., the following diagram commutes for all $i$:

$$ \begin{array}{ccc} K^i & \xrightarrow{d_K^i} & K^{i+1} \\ \downarrow{f^i} && \downarrow{f^{i+1}} \\ L^i & \xrightarrow{d_L^i} & L^{i+1} \end{array}.$$

There is then a category, often denoted by \hl{$\mathrm{Ch}(\mathcal{A})$} or \hl{$\mathbf{Ch}(\mathcal{A})$}, whose objects are chain complexes in $\calA$ and whose morphisms are morphisms of chain complexes. In particular, we may denote by 
$$\hlin{\operatorname{Hom}(K^\bullet, L^\bullet)=  \operatorname{Hom}_{\mathrm{Ch}(\mathcal{A})}(K^\bullet, L^\bullet)}$$
the set of chain maps $K^\bullet \to L^\bullet$; it is in fact an abelian group.

A \hldef{morphism of cochain complexes} is defined similarly, and we similarly denote by \hl{$\mathrm{Ch}(\mathcal{A})$} or \hl{$\mathbf{Ch}(\mathcal{A})$} the caetgory of cochain complexes in $\calA$. 
\end{definition}

% See Also
% 
\begin{proposition} \label{proposition:category_of_chain_complexes_in_an_additive_category_is_additive}
Let $\mathcal{A}$ be an \hyperrefIfExists{definition:additive_category}{additive category}. 
\begin{enumerate}
    \item The category \hyperrefIfExists{definition:chain_complex_of_objects_in_an_additive_category}{$\mathrm{Ch}(\calA)$} of chain complexes is itself and additive category.

    \item If $\calA$ is an \hyperrefIfExists{definition:abelian_category}{abelian category}, then $\mathrm{Ch}(\calA)$ is an abelian category.

    \item If $\calA$ is an \hyperrefIfExists{definition:abelian_category}{abelian category} satisfying Grothendieck's axiom \CrefAndHyperrefIfExist{definition:grothendiecks_additional_axioms_for_abelian_categories}{AB$n$ (resp. AB$n^*$)} for $n \in \{3,4,5,6\}$, then $\mathrm{Ch}(\calA)$ also satisfies AB$n$ (resp. AB$n^*$). If $\calA$ is a \CrefAndHyperrefIfExist{definition:grothendiecks_additional_axioms_for_abelian_categories}{Grothendieck abelian category}, then so is $\mathrm{Ch}(\calA)$
\end{enumerate}
\end{proposition}
\begin{proof}
    Combine \Cref{proposition:category_of_chain_complexes_of_objects_in_a_preadditive_category_is_equivalent_to_the_category_of_additive_functors_from_the_walking_chain_complex_category} and \Cref{lemma:additive_functor_category_from_small_preadditive_categories_preserves}.
\end{proof}

% \begin{corollary}
% Let $\calB$ be a \CrefAndHyperrefIfExist{definition:additive_category}{preadditive category}.
% \begin{enumerate}
%     \item The \CrefAndHyperrefIfExist{definition:chain_complex_of_objects_in_an_additive_category}{(co)chain complex category} $\text{Ch}(\calB)$ is preadditive. If $\calB$ is additionally \CrefAndHyperrefIfExist{definition:additive_category}{additive}/\CrefAndHyperrefIfExist{definition:abelian_category}{abelian}, then so is $\text{Ch}(\calB)$.

%     \item If $\calB$ is an abelian category with property \CrefAndHyperrefIfExist{definition:grothendiecks_additional_axioms_for_abelian_categories}{$ABn$ for $n = 3,4,5,6$ or $ABn^*$ for $n = 3,4,5$}, then $\text{Add}(\calA, \calB)$ possesses the same property.
% \end{enumerate}

% \end{corollary}

\begin{remark} \label{remark:cohomological_vs_homological_conventions}
    The convention used to define chain complexes in \Cref{definition:chain_complex_of_objects_in_an_additive_category} is a \emph{cohomological one} --- note that indices are written as superscripts and increase when ``following the arrows''. Such a chain complex may also be referred to as a \hldef{cochain complex} or a \hldef{cohomological chain complex} to emphasize an adoption of a cohomological convention. 

    The dual convention would be a \emph{homological one}, in which indices are written as subscripts and decrease when ``following the arrow''. As such, one may speak of a \hldef{(homological) chain complex} $(K_\bullet, d_\bullet)$ indexed by $I$ as consisting of:

    \begin{itemize}
    \item Objects $\{ K_i \}_{i \in I}$ in $\mathcal{A}$, called the \hldef{terms in degree $i$},
    \item Morphisms $d_i: K_i \to K_{i-1}$ in $\mathcal{A}$, called the \hldef{differentials in degree $i$},
    \end{itemize}
    such that for every $i \in I$, $d_{i-1} \circ d_i = 0$. That is,
    $$ 
    K_\bullet: \cdots \xrightarrow{d_{i+1}} K_i \xrightarrow{d_i} K_{i-1} \xrightarrow{d_{i-1}} K_{i-2} \xrightarrow{d_{i-2}} \cdots
    $$
    with $d_{i-1} d_i = 0$ for all $i$. We might typically use notation such as \hl{$K_\bullet = (K_i, d_i)_{i \in I}$} to denote a chain complex in $\mathcal{A}$.

    The differences between the conventions persist --- for example, cohomological objects are usually written with superscript indicees whereas homological objects are usually written with subscript indicees.
\end{remark}
\begin{convention} \label{convention:homological_algebra_is_discussed_in_cohomological_terms}
    When discussing homological algebra in abstract terms, we may often adopt the homological convention in some discussions and the cohomological convention in others \CrefIfExists{remark:cohomological_vs_homological_conventions}.
    %; for instance, indices are written as superscripts and increase along the direction of the arrows in chain complexes. 
\end{convention}

\begin{definition} \label{definition:quiver}
A \hldef{quiver} is a quadruple $Q = (Q_0, Q_1, s, t)$, where:
\begin{itemize}
    \item $Q_0$ is a collection of \hldef{vertices}.
    \item $Q_1$ is a collection of \hldef{arrows}.
    \item $s, t: Q_1 \to Q_0$ are functions assigning to each arrow $\alpha \in Q_1$ its \hldef{source} $s(\alpha)$ and its \hldef{target} $t(\alpha)$.
\end{itemize}
\end{definition}
\begin{definition} \label{definition:path_category_generated_by_a_quiver}
Let $Q$ be a \CrefAndHyperrefIfExist{definition:quiver}{quiver}.
The \hldef{path category generated by $Q$}, denoted \hl{$\mathcal{F}(Q)$}, is the \CrefAndHyperrefIfExist{definition:category}{category} defined as follows:
\begin{itemize}
    \item The objects of $\mathcal{F}(Q)$ are the vertices $Q_0$.
    \item For any two objects $x, y \in Q_0$, the set of morphisms $\text{Hom}_{\mathcal{F}(Q)}(x, y)$ consists of all paths from $x$ to $y$ --- A \hldef{path of length $n \ge 1$ from $x$ to $y$} is a sequence of arrows $\alpha_n \dots \alpha_1$ such that $s(\alpha_1) = x$, $t(\alpha_n) = y$, and $s(\alpha_{i+1}) = t(\alpha_i)$ for all $1 \le i < n$. Additionally, for each vertex $x$, there is a path $e_x$ of length $0$, which serves as the identity morphism.
    \item Composition of morphisms is defined by the concatenation of paths.
\end{itemize}
\end{definition}
\begin{definition} \label{definition:preadditive_category_generated_by_a_quiver}
Let $Q$ be a \CrefAndHyperrefIfExist{definition:quiver}{quiver} whose collection of arrows is small.

The \hldef{preadditive category generated by $Q$}, denoted \hl{$\mathbb{Z}Q$}, is the \CrefAndHyperrefIfExist{definition:additive_category_preadditive_category}{preadditive category}, i.e. the \CrefAndHyperrefIfExist{definition:category_enriched_in_a_monoidal_category}{category enriched over} the \CrefAndHyperrefIfExist{definition:category_of_groups_of_abelian_groups}{category of abelian groups} defined as follows:
\begin{itemize}
    \item The objects of $\mathbb{Z}Q$ are the vertices $Q_0$.
    \item For any objects $x, y \in Q_0$, the morphism set $\text{Hom}_{\mathbb{Z}Q}(x, y)$ is the free abelian group generated by the set of all paths from $x$ to $y$ in $Q$.
    \item Composition is the unique bilinear extension of the path concatenation in $\mathcal{F}(Q)$. That is, for paths $u, v, w$ where concatenation is defined, composition satisfies $(u + v) \circ w = u \circ w + v \circ w$ and $w \circ (u + v) = w \circ u + w \circ v$.
\end{itemize}
\end{definition}
\begin{definition} \label{definition:walking_chain_complex_category}
Let $Q_{\text{chain}}$ be the \CrefAndHyperrefIfExist{definition:quiver}{quiver} with vertex set $Q_0 = \mathbb{Z}$ and arrow set $Q_1 = \{ d_n : n \to n-1 \mid n \in \mathbb{Z} \}$. 
\TODO{quotient of a category}
The \hldef{walking chain complex category}, denoted \hl{$\mathcal{I}$}, is the quotient of the \CrefAndHyperrefIfExist{definition:additive_category}{preadditive category} $\mathbb{Z}Q_{\text{chain}}$ by the ideal generated by the relations $d_{n-1} \circ d_n = 0$ for all $n \in \mathbb{Z}$. Explicitly:
\begin{itemize}
    \item Objects are the integers $\mathbb{Z}$.
    \item Morphisms are $\mathbb{Z}$-linear combinations of paths, subject to the relation that any path containing a subsegment $d_{n-1}d_n$ is identified with the zero morphism.
\end{itemize}
\end{definition}
\begin{definition} \label{definition:additive_functor_category_between_preadditive_categories}
Let $\mathcal{A}$ and $\mathcal{B}$ be \CrefAndHyperrefIfExist{definition:additive_category_preadditive_category}{preadditive categories} (\CrefAndHyperrefIfExist{definition:category_enriched_in_a_monoidal_category}{categories enriched over} the \CrefAndHyperrefIfExist{definition:category_of_groups_of_abelian_groups}{category of abelian groups}). The \hldef{additive functor category} \hl{$\text{Add}(\mathcal{A}, \mathcal{B})$} is the functor category where:
\begin{itemize}
    \item Objects are {additive functors} $F: \mathcal{A} \to \mathcal{B}$. An \CrefAndHyperrefIfExist{definition:additive_functor_between_additive_categories}{additive functor} is a functor such that for any $x, y \in \text{Ob}(\mathcal{A})$, the map $F: \text{Hom}_{\mathcal{A}}(x, y) \to \text{Hom}_{\mathcal{B}}(F(x), F(y))$ is a group homomorphism.
    \item Morphisms are \CrefAndHyperrefIfExist{definition:natural_transformation_between_functors_between_categories}{natural transformations} between \CrefAndHyperrefIfExist{definition:additive_functor_between_additive_categories}{additive functors}.
\end{itemize}
\end{definition}
\input{../_concepts/proposition_category_of_chain_complexes_of_objects_in_a_preadditive_category_is_equivalent_to_the_category_of_additive_functors_from_the_walking_chain_complex_category.tex}
\begin{lemma} \label{lemma:additive_functor_category_from_small_preadditive_categories_preserves}
Let $\calA, \calB$ be \CrefAndHyperrefIfExist{definition:additive_category}{preadditive categories} with $\calA$ \CrefAndHyperrefIfExist{definition:locally_small_category}{small}.
\begin{enumerate}
    \item The \CrefAndHyperrefIfExist{definition:additive_functor_category_between_preadditive_categories}{additive functor category} $\text{Add}(\calA, \calB)$ is preadditive. If $\calB$ is additionally \CrefAndHyperrefIfExist{definition:additive_category}{additive}/\CrefAndHyperrefIfExist{definition:abelian_category}{abelian}, then so is $\text{Add}(\calA, \calB)$.

    \item If $\calB$ is an abelian category with property \CrefAndHyperrefIfExist{definition:grothendiecks_additional_axioms_for_abelian_categories}{$ABn$ for $n = 3,4,5,6$ or $ABn^*$ for $n = 3,4,5$}, then $\text{Add}(\calA, \calB)$ possesses the same property.
\end{enumerate}
\end{lemma}

\begin{proposition} \label{proposition:category_of_chain_complexes_in_an_additive_category_is_additive}
Let $\mathcal{A}$ be an \hyperrefIfExists{definition:additive_category}{additive category}. 
\begin{enumerate}
    \item The category \hyperrefIfExists{definition:chain_complex_of_objects_in_an_additive_category}{$\mathrm{Ch}(\calA)$} of chain complexes is itself and additive category.

    \item If $\calA$ is an \hyperrefIfExists{definition:abelian_category}{abelian category}, then $\mathrm{Ch}(\calA)$ is an abelian category.

    \item If $\calA$ is an \hyperrefIfExists{definition:abelian_category}{abelian category} satisfying Grothendieck's axiom \CrefAndHyperrefIfExist{definition:grothendiecks_additional_axioms_for_abelian_categories}{AB$n$ (resp. AB$n^*$)} for $n \in \{3,4,5,6\}$, then $\mathrm{Ch}(\calA)$ also satisfies AB$n$ (resp. AB$n^*$). If $\calA$ is a \CrefAndHyperrefIfExist{definition:grothendiecks_additional_axioms_for_abelian_categories}{Grothendieck abelian category}, then so is $\mathrm{Ch}(\calA)$
\end{enumerate}
\end{proposition}
\begin{proof}
    Combine \Cref{proposition:category_of_chain_complexes_of_objects_in_a_preadditive_category_is_equivalent_to_the_category_of_additive_functors_from_the_walking_chain_complex_category} and \Cref{lemma:additive_functor_category_from_small_preadditive_categories_preserves}.
\end{proof}

% \begin{corollary}
% Let $\calB$ be a \CrefAndHyperrefIfExist{definition:additive_category}{preadditive category}.
% \begin{enumerate}
%     \item The \CrefAndHyperrefIfExist{definition:chain_complex_of_objects_in_an_additive_category}{(co)chain complex category} $\text{Ch}(\calB)$ is preadditive. If $\calB$ is additionally \CrefAndHyperrefIfExist{definition:additive_category}{additive}/\CrefAndHyperrefIfExist{definition:abelian_category}{abelian}, then so is $\text{Ch}(\calB)$.

%     \item If $\calB$ is an abelian category with property \CrefAndHyperrefIfExist{definition:grothendiecks_additional_axioms_for_abelian_categories}{$ABn$ for $n = 3,4,5,6$ or $ABn^*$ for $n = 3,4,5$}, then $\text{Add}(\calA, \calB)$ possesses the same property.
% \end{enumerate}

% \end{corollary}
\begin{proposition} \label{proposition:limit_and_colimit_or_diagram_of_chain_complexes_is_computed_pointwise}
Let $\mathcal{A}$ be an \CrefAndHyperrefIfExist{definition:additive_category_preadditive_category}{additive category}. Let $\mathcal{J}$ be a \CrefAndHyperrefIfExist{definition:category}{category} and let $M^\bullet: \mathcal{J} \to \operatorname{Ch}(\mathcal{A})$ be a \CrefAndHyperrefIfExist{definition:diagram_in_a_category_indexed_by_a_small_category}{diagram} of \CrefAndHyperrefIfExist{definition:chain_complex_of_objects_in_an_additive_category}{chain complexes}, denoted by $j \mapsto M^\bullet_{(j)}$.

\begin{enumerate}
    \item If the \CrefAndHyperrefIfExist{definition:limit_and_colimit_of_a_diagram_in_a_category}{limit} of the diagram of objects $j \mapsto M^n_{(j)}$ exists in $\mathcal{A}$ for every degree $n \in \mathbb{Z}$, then the limit of the diagram $M^\bullet$ exists in $\operatorname{Ch}(\mathcal{A})$. It is computed termwise:
    \[
    \left( \lim_{j \in \mathcal{J}} M^\bullet_{(j)} \right)^n \cong \lim_{j \in \mathcal{J}} \left( M^n_{(j)} \right).
    \]
    The differential $d^n: (\lim M^\bullet)^n \to (\lim M^\bullet)^{n+1}$ is the unique morphism induced by the family of morphisms $\{ d^n_{(j)} : M^n_{(j)} \to M^{n+1}_{(j)} \}_{j \in \mathcal{J}}$ via the universal property of limits.

    \item Similarly, if the \CrefAndHyperrefIfExist{definition:limit_and_colimit_of_a_diagram_in_a_category}{colimit} of the diagram of objects $j \mapsto M^n_{(j)}$ exists in $\mathcal{A}$ for every degree $n \in \mathbb{Z}$, then the colimit of the diagram $M^\bullet$ exists in $\operatorname{Ch}(\mathcal{A})$. It is computed termwise:
    \[
    \left( \operatorname{colim}_{j \in \mathcal{J}} M^\bullet_{(j)} \right)^n \cong \operatorname{colim}_{j \in \mathcal{J}} \left( M^n_{(j)} \right).
    \]
    The differential $\delta^n: (\operatorname{colim} M^\bullet)^n \to (\operatorname{colim} M^\bullet)^{n+1}$ is the unique morphism induced by the family of morphisms $\{ d^n_{(j)} : M^n_{(j)} \to M^{n+1}_{(j)} \}_{j \in \mathcal{J}}$ via the universal property of colimits.
\end{enumerate}
\end{proposition}
\begin{proof}
    See \Cref{proposition:limits_and_colimits_in_functor_categories_may_be_computed_pointwise} \TODO{It must be taken into account how example the categories of chain complexes are regardable as functor categories}
\end{proof}


\subsection{(Co)Homology of chain complexes}

\begin{definition} \label{definition:boundary_cycle_coboundary_cocyble_of_a_chain_cochain_complex}
    Let $\mathcal{A}$ be an \CrefAndHyperrefIfExist{definition:abelian_category}{abelian category}.
    \begin{enumerate}
        \item Let $C_\bullet = (\dots \to C_{n+1} \xrightarrow{d_{n+1}} C_n \xrightarrow{d_n} C_{n-1} \to \dots)$ be a \CrefAndHyperrefIfExist{definition:chain_complex_of_objects_in_an_additive_category}{chain complex} in $\mathcal{A}$.
        For each integer $n$, we define:
        \begin{itemize}
            \item The object of \hldef{$n$-cycles}, denoted \hl{$Z_n(C)$}, is the \CrefAndHyperrefIfExist{definition:kernel_and_cokernel_of_a_morphism_in_a_category}{kernel} of the differential leaving $C_n$:
            $$Z_n(C) := \ker(d_n: C_n \to C_{n-1}).$$
            \item The object of \hldef{$n$-boundaries}, denoted \hl{$B_n(C)$}, is the \CrefAndHyperrefIfExist{definition:image_coimage_of_a_morphism_in_a_category}{image} of the differential entering $C_n$:
            $$B_n(C) := \text{im}(d_{n+1}: C_{n+1} \to C_n).$$
        \end{itemize}
        Since $d_n \circ d_{n+1} = 0$, there is a canonical \CrefAndHyperrefIfExist{definition:monomorphism_and_epimorphism_in_categories}{monomorphism} $B_n(C) \hookrightarrow Z_n(C)$.

        \item Let $C^\bullet = (\dots \to C^{n-1} \xrightarrow{d^{n-1}} C^n \xrightarrow{d^n} C^{n+1} \to \dots)$ be a \hldef{cochain complex in $\mathcal{A}$}.
        For each integer $n$, we define:
        \begin{itemize}
            \item The object of \hldef{$n$-cocycles}, denoted \hl{$Z^n(C)$}, is the \CrefAndHyperrefIfExist{definition:kernel_and_cokernel_of_a_morphism_in_a_category}{kernel} of the differential leaving $C^n$:
            $$Z^n(C) := \ker(d^n: C^n \to C^{n+1}).$$
            \item The object of \hldef{$n$-coboundaries}, denoted \hl{$B^n(C)$}, is the \CrefAndHyperrefIfExist{definition:image_coimage_of_a_morphism_in_a_category}{image>} of the differential entering $C^n$:
            $$B^n(C) := \text{im}(d^{n-1}: C^{n-1} \to C^n).$$
        \end{itemize}
        Since $d^n \circ d^{n-1} = 0$, there is a canonical \CrefAndHyperrefIfExist{definition:monomorphism_and_epimorphism_in_categories}{monomorphism} $B^n(C) \hookrightarrow Z^n(C)$.
    \end{enumerate}
\end{definition}
% \begin{definition}[Chain complexes and their (co)homology objects] \label{definition:homology_and_cohomology_objects_for_a_chain_complex_in_an_additive_category}
%     Let $\mathcal{A}$ be an \hyperrefIfExists{definition:abelian_category}{abelian category}.
    
%     \begin{itemize}
%         \item For a cohomologically indexed chain complex $K^\bullet$, its \hldef{cohomology object in degree $i$} is defined by
%         $$ \hlin{H^i(K^\bullet) := \ker(d^i) / \operatorname{im}(d^{i-1})} $$
%         where the kernel and image are taken in $\mathcal{A}$.

%         \item For a homologically indexed chain complex $K_\bullet$, its \hldef{homology object in degree $i$} is defined by
%         $$ \hlin{H_i(K_\bullet) := \ker(d_i) / \operatorname{im}(d_{i+1})} $$
%         where the kernel and image are taken in $\mathcal{A}$.
%     \end{itemize}
% \end{definition}

\begin{definition}[Chain complexes and their (co)homology objects] \label{definition:homology_and_cohomology_objects_for_a_chain_complex_in_an_additive_category}
    Let $\mathcal{A}$ be an \hyperrefIfExists{definition:abelian_category}{abelian category}.
    
    \begin{itemize}
        \item For a \CrefAndHyperrefIfExist{definition:cochain_complex}{cochain complex} $K^\bullet$, its \hldef{cohomology object in degree $i$} is defined as the quotient of the object of $i$-cocycles by the object of $i$-coboundaries:
        $$ \hlin{H^i(K^\bullet) := Z^i(K) / B^i(K) = \ker(d^i) / \operatorname{im}(d^{i-1}).} $$

        \item For a \CrefAndHyperrefIfExist{definition:chain_complex}{chain complex} $K_\bullet$, its \hldef{homology object in degree $i$} is defined as the quotient of the object of $i$-cycles by the object of $i$-boundaries:
        $$ \hlin{H_i(K_\bullet) := Z_i(K) / B_i(K). = \ker(d_i) / \operatorname{im}(d_{i+1}).} $$
    \end{itemize}
\end{definition}


\begin{definition}[Acyclic complex] \label{definition:acyclic_complex_of_objects_in_an_abelian_category}
Let $\mathcal{A}$ be an \CrefAndHyperrefIfExist{definition:additive_category}{additive category}, and let $(C_\bullet, d_\bullet)$ be a \CrefAndHyperrefIfExist{definition:chain_complex_of_objects_in_an_additive_category}{complex} in $\mathcal{A}$:
\[
\cdots \xrightarrow{d_{n+1}} C_n \xrightarrow{d_n} C_{n-1} \xrightarrow{d_{n-1}} \cdots.
\]

The complex $(C_\bullet, d_\bullet)$ is called \hldef{acyclic at $C_n$} (or sometimes synonymously \hldef{exact at $C_n$})  if we have $\ker d_n \cong \mathrm{im} d_{n+1}$ \CrefIfExists{definition:kernel_and_cokernel_of_a_morphism_in_a_category}  \CrefIfExists{definition:image_coimage_of_a_morphism_in_a_category}. 

If $\calA$ is an \CrefAndHyperrefIfExist{definition:abelian_category}{abelian category}, then this is equivalent to the condition that the \CrefAndHyperrefIfExist{definition:homology_and_cohomology_objects_for_a_chain_complex_in_an_additive_category}{(co)homology objects} $H^n(C_\bullet) := \ker d_n / \operatorname{im} d_{n+1}$ are zero in $\mathcal{A}$.

We furthermore say that the complex $(C_\bullet, d_\bullet)$ is \hldef{acyclic} or \hldef{exact} if it is acyclic/exact everywhere.
\end{definition}

\begin{proposition} \label{proposition:exact_sequence_in_a_diagram_category_of_an_abelian_category_indexed_by_a_small_category_is_given_by_componentwise_exact_sequences}
    Let $\calA$ be a \CrefAndHyperrefIfExist{definition:abelian_category}{abelian category} and let $J$ be a \CrefAndHyperrefIfExist{definition:locally_small_category}{small category}. 

    Given a sequence $A \to B \to C$ of objects in the \CrefAndHyperrefIfExist{definition:diagram_in_a_category_indexed_by_a_small_category}{diagram category} $\calA^J$, the sequence is \CrefAndHyperrefIfExist{definition:acyclic_complex_of_objects_in_an_abelian_category}{exact} at $B$ if and only if all the sequences $A(j) \to B(j) \to C(j)$ are exact at $B(j)$ for every $j \in \Ob(J)$. 
\end{proposition}

\begin{definition}[Boundedness conditions on chain complexes] \label{definition:bounded_complexes_on_an_additive_category_and_homologically_bounded_objects_on_an_abelian_category}
Let $\mathcal{A}$ be an additive category.

\noindent\textbf{Cohomological convention:} Let $K^\bullet = (K^i, d^i)_{i \in \mathbb{Z}}$ be a cohomologically indexed chain complex in $\mathcal{A}$.
\begin{itemize}
    \item $K^\bullet$ is \hldef{bounded above} if there exists $n \in \mathbb{Z}$ such that $K^i = 0$ for all $i > n$.
    \item $K^\bullet$ is \hldef{bounded below} if there exists $m \in \mathbb{Z}$ such that $K^i = 0$ for all $i < m$.
    \item $K^\bullet$ is \hldef{bounded} if it both bounded above and below, or equivalently if $K^i = 0$ for all but finitely many $i \in \mathbb{Z}$.
    \item Assuming that $\calA$ is an \hyperrefIfExists{definition:abelian_category}{abelian category}, 
    \begin{itemize}
        \item $K^\bullet$ is \hldef{cohomologically bounded above} if there exists $n \in \mathbb{Z}$ such that $H^i(K^\bullet) = 0$ for all $i > n$.
        \item $K^\bullet$ is \hldef{cohomologically bounded below} if there exists $m \in \mathbb{Z}$ such that $H^i(K^\bullet) = 0$ for all $i < m$.
        \item $K^\bullet$ is \hldef{cohomologically bounded} if it is both cohomologically bounded above and below, or equivalently if the \hyperrefIfExists{definition:homology_and_cohomology_objects_for_a_chain_complex_in_an_additive_category}{cohomology objects} $H^i(K^\bullet)$ vanish for all but finitely many $i \in \mathbb{Z}$.
    \end{itemize}
\end{itemize}

\noindent\textbf{Homological convention:} Let $K_\bullet = (K_i, d_i)_{i \in \mathbb{Z}}$ be a homologically indexed chain complex in $\mathcal{A}$.
\begin{itemize}
    \item $K_\bullet$ is \hldef{bounded above} if there exists $n \in \mathbb{Z}$ such that $K_i = 0$ for all $i > n$.
    \item $K_\bullet$ is \hldef{bounded below} if there exists $m \in \mathbb{Z}$ such that $K_i = 0$ for all $i < m$.

    % \item $K^\bullet$ is \hldef{bounded} if it both bounded above and below, equivalently if $K^i = 0$ for all but finitely many $i \in \mathbb{Z}$.
    \item $K_\bullet$ is \hldef{bounded} if it both bounded above and below, or equivalently if $K_i = 0$ for all but finitely many $i \in \mathbb{Z}$.
    \item Assuming that $\calA$ is an \hyperrefIfExists{definition:abelian_category}{abelian category},
    \begin{itemize}
        \item $K_\bullet$ is \hldef{homologically bounded above} if there exists $n \in \mathbb{Z}$ such that $H_i(K_\bullet) = 0$ for all $i > n$.
        \item $K_\bullet$ is \hldef{homologically bounded below} if there exists $m \in \mathbb{Z}$ such that $H_i(K_\bullet) = 0$ for all $i < m$.

        \item $K_\bullet$ is \hldef{homologically bounded} if it is both homologically bounded above and below, or equivalently if the \hyperrefIfExists{definition:homology_and_cohomology_objects_for_a_chain_complex_in_an_additive_category}{homology objects} $H_i(K_\bullet)$ vanish for all but finitely many $i \in \mathbb{Z}$.
    \end{itemize}
\end{itemize}
\end{definition}

% \begin{definition}[Boundedness conditions on chain complexes] \label{definition:bounded_complexes_on_an_additive_category_and_homologically_bounded_objects_on_an_abelian_category}
% Let $\mathcal{A}$ be an additive category.
% \noindent\textbf{Cohomological convention:} Let $K^\bullet = (K^i, d^i)_{i \in \mathbb{Z}}$ be a cohomologically indexed chain complex in $\mathcal{A}$.
% \begin{itemize}
%     \item $K^\bullet$ is \hldef{bounded} if $K^i = 0$ for all but finitely many $i \in \mathbb{Z}$.
%     \item $K^\bullet$ is \hldef{bounded above} if there exists $n \in \mathbb{Z}$ such that $K^i = 0$ for all $i > n$.
%     \item $K^\bullet$ is \hldef{bounded below} if there exists $m \in \mathbb{Z}$ such that $K^i = 0$ for all $i < m$.
%     \item Assuming that $\calA$ is an \hyperrefIfExists{definition:abelian_category}{abelian category}, $K^\bullet$ is \hldef{cohomologically bounded} if the \hyperrefIfExists{definition:homology_and_cohomology_objects_for_a_chain_complex_in_an_additive_category}{cohomology objects} $H^i(K^\bullet)$ vanish for all but finitely many $i \in \mathbb{Z}$.
% \end{itemize}
% \noindent\textbf{Homological convention:} Let $K_\bullet = (K_i, d_i)_{i \in \mathbb{Z}}$ be a homologically indexed chain complex in $\mathcal{A}$.
% \begin{itemize}
%     \item $K_\bullet$ is \hldef{bounded} if $K_i = 0$ for all but finitely many $i \in \mathbb{Z}$.
%     \item $K_\bullet$ is \hldef{bounded above} if there exists $n \in \mathbb{Z}$ such that $K_i = 0$ for all $i < n$.
%     \item $K_\bullet$ is \hldef{bounded below} if there exists $m \in \mathbb{Z}$ such that $K_i = 0$ for all $i > m$.
%     \item Assuming that $\calA$ is an \hyperrefIfExists{definition:abelian_category}{abelian category}, $K_\bullet$ is \hldef{homologically bounded} if the \hyperrefIfExists{definition:homology_and_cohomology_objects_for_a_chain_complex_in_an_additive_category}{homology objects} $H_i(K_\bullet)$ vanish for all but finitely many $i \in \mathbb{Z}$.
% \end{itemize}
% \end{definition}


\begin{definition}[Quasi-isomorphism] \label{definition:quasi_isomorphism_of_chain_complexes_of_objects_in_an_abelian_category}
    Let $\mathcal{A}$ be an \hyperrefIfExists{definition:abelian_category}{abelian category}\CrefIfExists{definition:abelian_category}, and let
    \[
    f_\bullet: (C_\bullet, d_\bullet^C) \to (D_\bullet, d_\bullet^D)
    \]
    be a \hyperrefIfExists{definition:chain_complex_of_objects_in_an_additive_category}{chain map between complexes}\CrefIfExists{definition:chain_complex_of_objects_in_an_additive_category} in $\mathcal{A}$.

    The morphism $f_\bullet$ is called a \hldef{quasi-isomorphism} if it induces isomorphisms on all cohomology objects, i.e., for every integer $n$, the induced morphism on \hyperrefIfExists{definition:homology_and_cohomology_objects_for_a_chain_complex_in_an_additive_category}{homology}\CrefIfExists{definition:homology_and_cohomology_objects_for_a_chain_complex_in_an_additive_category} (or cohomology, depending on the convention)
    \[
    H^n(f_\bullet): H^n(C_\bullet) \to H^n(D_\bullet)
    \]
    is an isomorphism in $\mathcal{A}$.

    Note that all of these notions are applicable to the cohomological convention as well\CrefIfExists{remark:cohomological_vs_homological_conventions}.
\end{definition}

\begin{lemma} \label{lemma:additive_functor_between_abelian_categories_is_exact_if_and_only_if_it_preserves_kernels_and_cokernels}
    Let \[ F: \mathcal{A} \to \mathcal{B} \] be an \hyperrefIfExists{definition:additive_functor_between_additive_categories}{additive functor}\CrefIfExists{definition:additive_functor_between_additive_categories} between \hyperrefIfExists{definition:abelian_category}{abelian categories}\CrefIfExists{definition:abelian_category}. It is \CrefAndHyperrefIfExist{definition:exact_functor_between_abelian_categories}{exact} if and only if it preserves \CrefAndHyperrefIfExist{definition:kernel_and_cokernel_of_a_morphism_in_a_category}{kernels and cokernels}.
\end{lemma}
\begin{theorem}[Long Exact Sequence in Homology] \label{theorem:long_exact_sequence_in_homology_cohomology_of_a_short_exact_sequence_of_chain_complexes_in_an_abelian_category}

\begin{enumerate}
    \item 
    Let $(C_\bullet, d^C_\bullet)$, $(D_\bullet, d^D_\bullet)$, and $(E_\bullet, d^E_\bullet)$ be \CrefAndHyperrefIfExist{definition:chain_complex_of_objects_in_an_additive_category}{chain complexes} in an \CrefAndHyperrefIfExist{definition:abelian_category}{abelian category}. Recall that \CrefAndHyperrefIfExist{definition:chain_complex_of_objects_in_an_additive_category}{$\mathbf{Ch}(\calA)$} is itself an abelian category (\Cref{proposition:category_of_chain_complexes_in_an_additive_category_is_additive}).
    Assume that 
    \[ 0 \longrightarrow C_\bullet \xrightarrow{\alpha_\bullet} D_\bullet \xrightarrow{\beta_\bullet} E_\bullet \longrightarrow 0 \]
    is a \CrefAndHyperrefIfExist{definition:short_exact_sequence_in_an_additive_category}{short exact sequence} of chain complexes. Equivalently, for each integer $n$, 
    \[ 0 \to C_n \xrightarrow{\alpha_n} D_n \xrightarrow{\beta_n} E_n \to 0 \]
    is an exact sequence of $R$-modules.

    Then there exists a natural \hldef{long exact sequence} in \CrefAndHyperrefIfExist{definition:homology_and_cohomology_objects_for_a_chain_complex_in_an_additive_category}{homology}:
    \[
    \cdots \longrightarrow H_{n+1}(E_\bullet) \xrightarrow{\delta_{n+1}} H_n(C_\bullet)
    \xrightarrow{H_n(\alpha)} H_n(D_\bullet)
    \xrightarrow{H_n(\beta)} H_n(E_\bullet)
    \xrightarrow{\delta_n} H_{n-1}(C_\bullet)
    \longrightarrow \cdots
    \]
    The homomorphisms \hl{$\delta_n: H_n(E_\bullet) \to H_{n-1}(C_\bullet)$} are called the \hldef{connecting homomorphisms} induced by the short exact sequence of chain complexes.

    Moreover, this long exact sequence is natural with respect to \CrefAndHyperrefIfExist{definition:chain_complex_of_objects_in_an_additive_category}{morphisms} of short exact sequences of chain complexes.

    \item Let $(C^\bullet, d_C^\bullet)$, $(D^\bullet, d_D^\bullet)$, and $(E^\bullet, d_E^\bullet)$ be 
    \CrefAndHyperrefIfExist{definition:chain_complex_of_objects_in_an_additive_category}{cochain complexes} 
    in an \CrefAndHyperrefIfExist{definition:abelian_category}{abelian category}. 
    Recall that \CrefAndHyperrefIfExist{definition:chain_complex_of_objects_in_an_additive_category}{$\mathbf{Ch}(\calA)$} 
    is itself an abelian category 
    (\Cref{proposition:category_of_chain_complexes_in_an_additive_category_is_additive}).

    Assume that 
    \[
    0 \longrightarrow C^\bullet \xrightarrow{\alpha^\bullet} D^\bullet \xrightarrow{\beta^\bullet} E^\bullet \longrightarrow 0
    \]
    is a \CrefAndHyperrefIfExist{definition:short_exact_sequence_in_an_additive_category}{short exact sequence} 
    of cochain complexes. Equivalently, for each integer $n$, 
    \[
    0 \to C^n \xrightarrow{\alpha^n} D^n \xrightarrow{\beta^n} E^n \to 0
    \]
    is an exact sequence of $R$-modules.

    Then there exists a natural \hldef{long exact sequence} in \CrefAndHyperrefIfExist{definition:homology_and_cohomology_objects_for_a_chain_complex_in_an_additive_category}{cohomology}:
    \[
    \cdots \longrightarrow H^{n-1}(E^\bullet) 
    \xrightarrow{\delta^{n-1}} H^{n}(C^\bullet)
    \xrightarrow{H^{n}(\alpha)} H^{n}(D^\bullet)
    \xrightarrow{H^{n}(\beta)} H^{n}(E^\bullet)
    \xrightarrow{\delta^{n}} H^{n+1}(C^\bullet)
    \longrightarrow \cdots
    \]
    The morphisms \hl{$\delta^{n}: H^{n}(E^\bullet) \to H^{n+1}(C^\bullet)$} are called the 
    \hldef{connecting homomorphisms} induced by the short exact sequence of cochain complexes.

    Moreover, this long exact sequence is natural with respect to 
    \CrefAndHyperrefIfExist{definition:morphism_of_cochain_complexes_of_objects_in_an_additive_category}{morphisms} 
    of short exact sequences of cochain complexes.

\end{enumerate}
\end{theorem}

\begin{proof}
    \TODO{why kernel and cokernel are left/right exact}
Consider the following commutative diagram in $\mathcal{A}$ with exact rows, where the vertical maps are the differentials of the respective complexes:
\begin{equation} \label{eq:snake_setup}
\begin{tikzcd}
0 \ar[r] & C_n / \operatorname{im}(d_{n+1}^C) \ar[r, "\bar{\alpha}_n"] \ar[d, "\bar{d}_n^C"] & D_n / \operatorname{im}(d_{n+1}^D) \ar[r, "\bar{\beta}_n"] \ar[d, "\bar{d}_n^D"] & E_n / \operatorname{im}(d_{n+1}^E) \ar[r] \ar[d, "\bar{d}_n^E"] & 0 \\
0 \ar[r] & \ker(d_{n-1}^C) \ar[r, "\alpha_{n-1}"] & \ker(d_{n-1}^D) \ar[r, "\beta_{n-1}"] & \ker(d_{n-1}^E) \ar[r] & 0
\end{tikzcd}
\end{equation}

The rows are exact because the original sequence of complexes is exact and the functors $\ker(d_{n-1})$ and $\coker(d_{n+1})$ are respectively left and right exact. Specifically, the bottom row is exact at the right because $\beta_n$ is surjective and the complexes satisfy the cycle condition. \TODO{why is the bottom row exact at the right and why is the top row exact at the left}

We now apply the \CrefAndHyperrefIfExist{proposition:snake_lemma}{Snake Lemma} to this diagram. To identify the resulting terms, we calculate the kernel and cokernel of the vertical map $\bar{d}_n^C$ (and similarly for $D$ and $E$):
\begin{itemize}
    \item \textbf{Kernel:} The kernel of $\bar{d}_n^C: C_n / \operatorname{im}(d_{n+1}^C) \to \ker(d_{n-1}^C)$ consists of elements\footnote{Since elements are invoked in this argument, the \CrefAndHyperref{theorem:freyd_mitchell_embedding_theorem_for_small_abelian_categories}{Freyd-Mitchell embedding theorem} should technically be used for this argument; nevertheless, one can argue that $\ker(\bar{d}_n^C) = H_n(C_\bullet)$ purely categorically.} $x + \operatorname{im}(d_{n+1}^C)$ such that $d_n^C(x) = 0$, i.e. $x \in \ker(d_n^C)$. Therefore,
    \[ \ker(\bar{d}_n^C) = \ker(d_n^C) / \operatorname{im}(d_{n+1}^C) = H_n(C_\bullet). \]
    
    \item \textbf{Cokernel:} The cokernel of $\bar{d}_n^C$ is the quotient of the target by the image. The target is $\ker(d_{n-1}^C)$ and the image is $\operatorname{im}(d_n^C)$. Thus,
    \[ \coker(\bar{d}_n^C) = \ker(d_{n-1}^C) / \operatorname{im}(d_n^C) = H_{n-1}(C_\bullet). \]
\end{itemize}

Applying the Snake Lemma to \eqref{eq:snake_setup} yields the exact sequence:
\[
\ker(\bar{d}_n^C) \to \ker(\bar{d}_n^D) \to \ker(\bar{d}_n^E) \xrightarrow{\delta_n} \coker(\bar{d}_n^C) \to \coker(\bar{d}_n^D) \to \coker(\bar{d}_n^E)
\]
Substituting the homology groups identified above, we obtain:
\[
H_n(C_\bullet) \xrightarrow{H_n(\alpha)} H_n(D_\bullet) \xrightarrow{H_n(\beta)} H_n(E_\bullet) \xrightarrow{\delta_n} H_{n-1}(C_\bullet) \xrightarrow{H_{n-1}(\alpha)} H_{n-1}(D_\bullet) \xrightarrow{H_{n-1}(\beta)} H_{n-1}(E_\bullet)
\]
Since this construction exists for every $n \in \bbZ$, we can splice these sequences together to form the bi-infinite long exact sequence. The naturality of the sequence follows from the naturality of the Snake Lemma and the functoriality of the kernel and cokernel constructions.
\end{proof}


\subsection{Homotopy between maps between chain complexes}


\begin{definition} \label{definition:chain_homotopy_between_chain_maps_between_complexes}
Let $\mathcal{A}$ be an \hyperrefIfExists{definition:additive_category_preadditive_category}{additive category}\CrefIfExists{definition:additive_category_preadditive_category}. 
\begin{enumerate}

    \item Let $f_\bullet, g_\bullet: C_\bullet \to D_\bullet$ be \hyperrefIfExists{definition:chain_complex_of_objects_in_an_additive_category}{chain maps between complexes}\CrefIfExists{definition:chain_complex_of_objects_in_an_additive_category} in $\mathcal{A}$. A \hldef{chain homotopy from $f_\bullet$ to $g_\bullet$} is a collection of morphisms $\{ s_n : C_n \to D_{n+1} \}$ such that for all $n$,
    \[
    f_n - g_n = d_{n+1}^D \circ s_n + s_{n-1} \circ d_n^C.
    \]
    If such an $s_\bullet$ exists, we say that $f_\bullet$ and $g_\bullet$ are \hldef{chain homotopic} and write \hl{$f_\bullet \simeq g_\bullet$}.

    \item Let $f_\bullet: C_\bullet \to D_\bullet$ be a \hyperrefIfExists{definition:chain_complex_of_objects_in_an_additive_category}{chain map between complexes}\CrefIfExists{definition:chain_complex_of_objects_in_an_additive_category} in $\mathcal{A}$. A \hldef{chain contraction} is a chain homotopy from $f_\bullet$ to the zero complex. The chain map $f_\bullet$ is said to be \hldef{null homotopic} if a chain contraction of $f_\bullet$ exists, i.e. $f_\bullet$ is chain homotopic to the $0$ chain complex. 

    \item Let $f_\bullet: C_\bullet \to D_\bullet$ be a chain map between complexes. We say that $f_\bullet$ is a \hldef{chain homotopy equivalence} if there exists a chain map and $h_\bullet: D_\bullet \to C_\bullet$ such that
    $$fg \simeq \id_{D_\bullet} \quad \text{and} \quad gf \simeq \id_{C_\bullet}.$$
    In this case, it is appropriate to call $f$ and $g$ \hldef{chain homotopy inverses of each other}.
\end{enumerate}
One similarly defines the above notions for cochain complexes and their morphisms.
\end{definition}

\begin{lemma} \label{lemma:chain_homotopy_between_chain_maps_between_complexes_is_equivalence_relation}
    Let $\calA$ be an \CrefAndHyperrefIfExist{definition:additive_category_preadditive_category}{additive category}. Being \CrefAndHyperrefIfExist{definition:chain_homotopy_between_chain_maps_between_complexes}{chain homotopic} is an equivalence relation on \CrefAndHyperrefIfExist{definition:chain_complex_of_objects_in_an_additive_category}{chain complexes} of objects in $\calA$.
    \TODOdef{equivalence relation}
\end{lemma}


\subsection{Truncations of chain complexes}
\begin{definition} \label{definition:stupid_truncation_of_a_chain_complex_in_an_additive_category}
    Let $\calA$ be an \CrefAndHyperrefIfExist{definition:additive_category_preadditive_category}{additive category}.
\begin{enumerate}
    \item  
    Let $A_\bullet$ be a \CrefAndHyperrefIfExist{definition:chain_complex_of_objects_in_an_additive_category}{chain complex} in 
    \CrefAndHyperrefIfExist{definition:chain_complex_of_objects_in_an_additive_category}{$\mathbf{Ch}(\mathcal{A})$}, 
    and let $n \in \mathbb{Z}$. The \hldef{stupid/brutal truncations of $A_\bullet$} are defined as follows.

    \hlalign{
    \begin{align*}
    (\sigma_{\le n} A)_i &=
    \begin{cases}
        0, & i > n, \\
        A_i, & i \le n,
    \end{cases}
    &
    (\sigma_{\ge n} A)_i &=
    \begin{cases}
        A_i, & i \ge n, \\
        0, & i < n.
    \end{cases}
    \end{align*}
    }

    Both are complexes with differentials induced from that of $A_\bullet$. 
    If $\calA$ is an \CrefAndHyperrefIfExist{definition:abelian_category}{abelian category}, then they satisfy canonical isomorphisms
    $$
    \sigma_{\le n} A_\bullet / \sigma_{\le n-1} A_\bullet \cong A_n[-n]
    \quad \text{and} \quad
    \sigma_{\ge n} A_\bullet / \sigma_{\ge n+1} A_\bullet \cong A_n[-n].
    $$

    \item Let $A^\bullet$ be a 
    \CrefAndHyperrefIfExist{definition:cochain_complex_of_objects_in_an_additive_category}{cochain complex} 
    in $\mathbf{Ch}(\mathcal{A})$, i.e.
    $$
    \cdots \xrightarrow{d^{n-2}} A^{n-1} \xrightarrow{d^{n-1}} A^n \xrightarrow{d^n} A^{n+1} \xrightarrow{d^{n+1}} \cdots
    $$
    with $d^{n+1} \circ d^n = 0$ for all $n \in \mathbb{Z}$. 
    The \hldef{stupid truncations of $A^\bullet$} are defined by the rules

    \hlalign{
    \begin{align*}
    (\sigma_{\ge n} A)^i &=
    \begin{cases}
        0, & i < n, \\
        A^i, & i \ge n,
    \end{cases}
    &
    (\sigma_{\le n} A)^i &=
    \begin{cases}
        A^i, & i \le n, \\
        0, & i > n.
    \end{cases}
    \end{align*}
    }

    These are \CrefAndHyperrefIfExist{definition:chain_complex_of_objects_in_an_additive_category}{cochain complexes} with differentials inherited from $A^\bullet$. 
    If $\calA$ is abelian, then they fit canonically into short exact sequences of cochain complexes
    $$
    0 \longrightarrow \sigma_{\ge n+1} A^\bullet \longrightarrow \sigma_{\ge n} A^\bullet \longrightarrow A^n[-n] \longrightarrow 0
    \quad\text{and}\quad
    0 \longrightarrow A^n[-n] \longrightarrow \sigma_{\le n} A^\bullet \longrightarrow \sigma_{\le n-1} A^\bullet \longrightarrow 0.
    $$

\end{enumerate}
\end{definition}
\begin{definition} \label{definition:canonical_truncation_of_chain_complexes_of_objects_in_an_abelian_category}

    Let $\calA$ be an \CrefAndHyperrefIfExist{definition:abelian_category}{abelian category}.

    \begin{enumerate}
        \item  
    The \hldef{canonical truncations} of a 
    \CrefAndHyperrefIfExist{definition:chain_complex_of_objects_in_an_additive_category}{chain complex} 
    $A_\bullet$ in 
    \CrefAndHyperrefIfExist{definition:chain_complex_of_objects_in_an_additive_category}{$\mathbf{Ch}(\mathcal{A})$} 
    are defined by

    \hlalign{
    \begin{align*}
    (\tau_{\ge n} A)_i &=
    \begin{cases}
        A_i, & i > n, \\
        \ker(d_n : A_n \to A_{n-1}), & i = n, \\
        0, & i < n,
    \end{cases}
    &
    (\tau_{\le n} A)_i &=
    \begin{cases}
        0, & i > n, \\
        \mathrm{coker}(d_{n+1} : A_{n+1} \to A_n), & i = n, \\
        A_i, & i < n.
    \end{cases}
    \end{align*}
    }

    The differentials are the restrictions and/or quotient maps induced from $A_\bullet$. 
    In particular,
    $$
    H_i(\tau_{\ge n} A_\bullet) = 
    \begin{cases}
        H_i(A_\bullet), & i \ge n, \\
        0, & i < n,
    \end{cases}
    \quad\text{and}\quad
    H_i(\tau_{\le n} A_\bullet) = 
    \begin{cases}
        H_i(A_\bullet), & i \le n, \\
        0, & i > n.
    \end{cases}
    $$
    The assignments $A_\bullet \mapsto \tau_{\ge n} A_\bullet$ and $A_\bullet \mapsto \tau_{\le n} A_\bullet$ extend to endofunctors

    \hlalign{
    \begin{align*}
    \tau_{\ge n},\, \tau_{\le n} : \mathbf{Ch}(\mathcal{A}) \to \mathbf{Ch}(\mathcal{A}),
    \end{align*}
    }

    called the \hldef{truncation functors}. They are natural in both $A_\bullet$ and $n$, and fit into canonical morphisms of complexes
    $$
    \tau_{\ge n} A_\bullet \longrightarrow A_\bullet \longrightarrow \tau_{\le n} A_\bullet.
    $$

    \item 
    \noindent
    Similarly, let $A^\bullet$ be a 
    \CrefAndHyperrefIfExist{definition:cochain_complex_of_objects_in_an_additive_category}{cochain complex} 
    in $\mathbf{Ch}(\mathcal{A})$, i.e.
    $$
    \cdots \xrightarrow{d^{n-2}} A^{n-1} \xrightarrow{d^{n-1}} A^n \xrightarrow{d^n} A^{n+1} \xrightarrow{d^{n+1}} \cdots
    $$
    with $d^{n+1} \circ d^n = 0$ for all $n \in \mathbb{Z}$. 
    The \hldef{canonical truncations of $A^\bullet$} are defined by

    \hlalign{
    \begin{align*}
    (\tau_{\le n} A)^i &=
    \begin{cases}
        A^i, & i < n, \\
        \ker(d^n : A^n \to A^{n+1}), & i = n, \\
        0, & i > n,
    \end{cases}
    &
    (\tau_{\ge n} A)^i &=
    \begin{cases}
        0, & i < n, \\
        \mathrm{coker}(d^{n-1} : A^{n-1} \to A^n), & i = n, \\
        A^i, & i > n.
    \end{cases}
    \end{align*}
    }

    The differentials are the restrictions or quotient maps induced by those of $A^\bullet$. 
    These truncations satisfy
    $$
    H^i(\tau_{\le n} A^\bullet) = 
    \begin{cases}
        H^i(A^\bullet), & i \le n, \\
        0, & i > n,
    \end{cases}
    \quad\text{and}\quad
    H^i(\tau_{\ge n} A^\bullet) = 
    \begin{cases}
        0, & i < n, \\
        H^i(A^\bullet), & i \ge n.
    \end{cases}
    $$

    They also extend to endofunctors

    \hlalign{
    \begin{align*}
    \tau_{\le n},\, \tau_{\ge n} : \mathbf{Ch}(\mathcal{A}) \to \mathbf{Ch}(\mathcal{A}),
    \end{align*}
    }

    natural in both $A^\bullet$ and $n$, fitting into canonical morphisms of cochain complexes
    $$
    \tau_{\le n} A^\bullet \longrightarrow A^\bullet \longrightarrow \tau_{\ge n} A^\bullet.
    $$

    \end{enumerate}

%     Let $\calA$ be an \CrefAndHyperrefIfExist{definition:abelian_category}{abelian category}.
% The \hldef{canonical truncations} of a \CrefAndHyperrefIfExist{definition:chain_complex_of_objects_in_an_additive_category}{chain complex} $A_\bullet$ in \CrefAndHyperrefIfExist{definition:chain_complex_of_objects_in_an_additive_category}{$\mathbf{Ch}(\mathcal{A})$} are defined by

% \hlalign{
% \begin{align*}
% (\tau_{\ge n} A)_i &=
%   \begin{cases}
%     A_i, & i > n, \\
%     \ker(d_n : A_n \to A_{n-1}), & i = n, \\
%     0, & i < n,
%   \end{cases}
% &
% (\tau_{\le n} A)_i &=
%   \begin{cases}
%     0, & i > n, \\
%     \mathrm{coker}(d_{n+1} : A_{n+1} \to A_n), & i = n, \\
%     A_i, & i < n.
%   \end{cases}
% \end{align*}
% }

% The differentials are the restrictions and/or quotient maps induced from $A_\bullet$. 
% In particular,
% $$
% H_i(\tau_{\ge n} A_\bullet) = 
%   \begin{cases}
%     H_i(A_\bullet), & i \ge n, \\
%     0, & i < n,
%   \end{cases}
% \quad\text{and}\quad
% H_i(\tau_{\le n} A_\bullet) = 
%   \begin{cases}
%     H_i(A_\bullet), & i \le n, \\
%     0, & i > n.
%   \end{cases}
% $$

% The assignments $A_\bullet \mapsto \tau_{\ge n} A_\bullet$ and $A_\bullet \mapsto \tau_{\le n} A_\bullet$ extend to endofunctors

% \hlalign{
% \begin{align*}
% \tau_{\ge n},\, \tau_{\le n} : \mathbf{Ch}(\mathcal{A}) \to \mathbf{Ch}(\mathcal{A}),
% \end{align*}
% }

% called the \hldef{truncation functors}. They are natural in both $A_\bullet$ and $n$, and fit into canonical morphisms of complexes
% $$
% \tau_{\ge n} A_\bullet \longrightarrow A_\bullet \longrightarrow \tau_{\le n} A_\bullet.
% $$
\end{definition}
\begin{proposition}[Exhaustion of Complexes by Brutal Truncations] \label{proposition:chain_complexes_in_additive_categories_are_limits_and_colimits_of_truncations}
    Let $\mathcal{A}$ be an additive category and let $X$ be a chain complex in $\operatorname{Ch}(\mathcal{A})$. 
    \begin{enumerate}
        \item 
        For any integer $n \in \mathbb{Z}$, let $\sigma_{\le n} X$ and $\sigma_{\ge n} X$ denote the \CrefAndHyperrefIfExist{definition:stupid_truncation_of_a_chain_complex_in_an_additive_category}{brutal truncations} defined componentwise by:
        $$ 
        (\sigma_{\le n} X)_k = \begin{cases} X_k & \text{if } k \le n \\ 0 & \text{if } k > n \end{cases} 
        \quad \text{and} \quad 
        (\sigma_{\ge n} X)_k = \begin{cases} X_k & \text{if } k \ge n \\ 0 & \text{if } k < n. \end{cases} 
        $$
        The canonical inclusions $\iota_n: \sigma_{\le n} X \to \sigma_{\le n+1} X$ define a \CrefAndHyperrefIfExist{definition:system_in_a_category_indexed_by_a_directed_poset}{directed system}, and the canonical projections $\pi_n: \sigma_{\ge n} X \to \sigma_{\ge n+1} X$ define an inverse system.
        
        The complex $X$ is canonically isomorphic to both the \CrefAndHyperrefIfExist{definition:limit_and_colimit_of_a_diagram_in_a_category}{colimit} of its "bounded above" truncations and the \CrefAndHyperrefIfExist{definition:limit_and_colimit_of_a_diagram_in_a_category}{limit} of its "bounded below" truncations:
        $$ X \cong \varinjlim_{n \to +\infty} \sigma_{\le n} X \quad \text{and} \quad X \cong \varprojlim_{n \to -\infty} \sigma_{\ge n} X.  $$

        \item 
        For any integer $n \in \mathbb{Z}$, let $\tau_{\le n} X$ and $\tau_{\ge n} X$ denote the \CrefAndHyperrefIfExist{definition:canonical_truncation_of_chain_complexes_of_objects_in_an_abelian_category}{canonical truncations} (or Postnikov sections) defined by:
        \begin{align*}
            (\tau_{\le n} X)_k &= \begin{cases} X_k & \text{if } k < n \\ \operatorname{Ker}(d_n) & \text{if } k = n \\ 0 & \text{if } k > n \end{cases} \\
            (\tau_{\ge n} X)_k &= \begin{cases} X_k & \text{if } k > n \\ \operatorname{Coker}(d_{n+1}) & \text{if } k = n \\ 0 & \text{if } k < n. \end{cases}
        \end{align*}
        The canonical morphisms $\iota_n: \tau_{\le n} X \to \tau_{\le n+1} X$ define a \CrefAndHyperrefIfExist{definition:system_in_a_category_indexed_by_a_directed_poset}{directed system}, and the canonical projections $\pi_n: \tau_{\ge n} X \to \tau_{\ge n+1} X$ define an inverse system (where the limit is taken as $n$ decreases).
    
        The complex $X$ is canonically isomorphic to both the \CrefAndHyperrefIfExist{definition:limit_and_colimit_of_a_diagram_in_a_category}{colimit} of its "homologically bounded above" truncations and the \CrefAndHyperrefIfExist{definition:limit_and_colimit_of_a_diagram_in_a_category}{limit} of its "homologically bounded below" truncations:
        $$ X \cong \varinjlim_{n \to +\infty} \tau_{\le n} X \quad \text{and} \quad X \cong \varprojlim_{n \to -\infty} \tau_{\ge n} X. $$

    \end{enumerate}
    
\end{proposition}



\subsection{Mapping cone of maps of chain complexes}

\begin{definition} \label{definition:mapping_cone_of_a_map_of_chain_cochain_complexes}
    \begin{enumerate}
        \item Let $f : (C_\bullet, d^C_\bullet) \to (D_\bullet, d^D_\bullet)$ be a \CrefAndHyperrefIfExist{definition:chain_complex_of_objects_in_an_additive_category}{morphism of chain complexes} in an \CrefAndHyperrefIfExist{definition:additive_category_preadditive_category}{additive category $\mathcal{A}$}.

        The \hldef{mapping cone of $f$}, denoted \hl{$\operatorname{Cone}(f)$}, is the chain complex defined by:
        \begin{itemize}
        \item Objects: For each $n$, 
        $$\operatorname{Cone}(f)_n = D_n \oplus C_{n-1}.$$
        \item Differential: For each $n$, define
        $$d^{\operatorname{Cone}(f)}_n : \operatorname{Cone}(f)_n \to \operatorname{Cone}(f)_{n-1}$$
        by the matrix morphism
        $$d^{\operatorname{Cone}(f)}_n = \begin{pmatrix} d^D_n & f_{n-1} \\ 0 & -d^C_{n-1} \end{pmatrix} : D_n \oplus C_{n-1} \to D_{n-1} \oplus C_{n-2}.$$
        \end{itemize}

        This construction defines a chain complex, i.e., $d^{\operatorname{Cone}(f)}_{n-1} \circ d^{\operatorname{Cone}(f)}_n = 0$.

        \item Dually, let $g : (C^\bullet, d_C^\bullet) \to (D^\bullet, d_D^\bullet)$ be a \CrefAndHyperrefIfExist{definition:chain_complex_of_objects_in_an_additive_category}{morphism of cochain complexes} in $\mathcal{A}$. 

        The \hldef{mapping cone of $g$}, denoted \hl{$\operatorname{Cone}(g)$}, is the \CrefAndHyperrefIfExist{definition:chain_complex_of_objects_in_an_additive_category}{cochain complex} defined by:
        \begin{itemize}
        \item Objects: For each $n$, 
        $$\operatorname{Cone}(g)^n = D^n \oplus C^{n+1}.$$
        \item Differential: For each $n$, define
        $$d_{\operatorname{Cone}(g)}^n : \operatorname{Cone}(g)^n \to \operatorname{Cone}(g)^{n+1}$$
        by the matrix morphism
        $$d_{\operatorname{Cone}(g)}^n = \begin{pmatrix} d_D^n & g^{n+1} \\ 0 & -d_C^{n+1} \end{pmatrix} : D^n \oplus C^{n+1} \to D^{n+1} \oplus C^{n+2}.$$
        \end{itemize}

        This construction defines a cochain complex, i.e., $d_{\operatorname{Cone}(g)}^{n+1} \circ d_{\operatorname{Cone}(g)}^n = 0$.
            \end{enumerate}
\end{definition}





\section{Right/left derived functors of left/right exact additive functors between abelian categories}

\subsection{Injective/projective objects}


\begin{definition}[Injective and Projective objects in a general category] \label{definition:injective_and_projective_objects_in_a_category}
Let $\mathcal{C}$ be a \CrefAndHyperrefIfExist{definition:category}{category}
\begin{itemize}
    \item An object $I \in \mathcal{C}$ is called \hldef{injective} if for every \CrefAndHyperrefIfExist{definition:monomorphism_and_epimorphism_in_categories}{monomorphism} $m: A \to B$ in $\mathcal{C}$ and every morphism $f: A \to I$, there exists a morphism $\tilde{f}: B \to I$ such that the diagram
    \[
    \begin{tikzcd}
    A \arrow[r, "f"] \arrow[d, "m", hook] & I \\
    B \arrow[ru, dashed, "\tilde{f}"'] &
    \end{tikzcd}
    \]
    commutes, i.e., $\tilde{f} \circ m = f$.

    \item Dually, an object $P \in \mathcal{C}$ is called \hldef{projective} if for every \CrefAndHyperrefIfExist{definition:monomorphism_and_epimorphism_in_categories}{epimorphism} $e: X \to Y$ in $\mathcal{C}$ and every morphism $g: P \to Y$, there exists a morphism $\tilde{g}: P \to X$ such that the diagram
    \[
    \begin{tikzcd}
    & P \arrow[ld, dashed, "\tilde{g}"'] \arrow[d, "g"] \\
    X \arrow[r, "e", two heads] & Y
    \end{tikzcd}
    \]
    commutes, i.e., $e \circ \tilde{g} = g$.
\end{itemize}
\end{definition}

% See Also
% lemma:injective_objects_in_a_category_are_projective_in_the_opposite_category
% lemma:injective_projective_objects_on_a_locally_small_category_induce_hom_functors_that_take_monomorphisms_epimoprhisms_to_surjections
% lemma:injective_projective_objects_in_an_abelian_category_have_exact_Hom_functors
\begin{lemma}[cf. e.g. {\cite[Lemma 2.3.4]{weibel}}] \label{lemma:injective_objects_in_a_category_are_projective_in_the_opposite_category}
    Let $\calC$ be a \CrefAndHyperrefIfExist{definition:category}{(large) category}. \TODO{Does this generalize to a category enriched in a symmetric monoidal category} 
    The following are equivalent for an object $I \in \calC$:
    \begin{enumerate}
        \item $I$ is \CrefAndHyperrefIfExist{definition:injective_and_projective_objects_in_a_category}{injective} in $\calC$
        \item $I$ is \CrefAndHyperrefIfExist{definition:injective_and_projective_objects_in_a_category}{projective} in \CrefAndHyperrefIfExist{definition:opposite_category_of_a_category}{$\calC^{\op}$}
    \end{enumerate}
\end{lemma}

\begin{proof}
    The notions of injectivity and projectivity of objects in a category are dual objects, so it is clear that $I$ is injective in $\calC$ if and only if it is projective in $\calC^{\op}$.
\end{proof}
\begin{lemma} \label{lemma:examples_of_projective_and_injective_objects}
    \begin{enumerate}
        \item 
        Let $R$ be a \CrefAndHyperrefIfExist{definition:ring}{(not necessarily commutative) ring}.
        \begin{enumerate}
            \item In the category $R\text{-}\mathbf{Mod}$ of left $R$-modules, projective objects are precisely direct summands of free modules.
            % \item In $R\text{-}\mathbf{Mod}$, injective objects are the ones satisfying Baer's criterion: $I$ is injective if and only if every $R$-linear map from a left ideal of $R$ to $I$ extends to $R$.
            \item For $R = \mathbb{Z}$, projective $\mathbb{Z}$-modules are the free abelian groups, and injective ones are the divisible abelian groups, e.g. $\mathbb{Q}$ or $\mathbb{Q}/\mathbb{Z}$.
        \end{enumerate}

        \item In the category $\mathbf{Set}$ of sets, every object is both projective and injective. Indeed, epimorphisms and monomorphisms in $\mathbf{Set}$ are surjective and injective maps, respectively, both of which split.
        \item In the category $\mathbf{Grp}$ of groups, projective objects are precisely the free groups, since every homomorphism from a free group lifts along surjective homomorphisms.
        \item In $\mathbf{Grp}$, the only injective object is the trivial group, since injectivity requires extensions along all inclusions, which only the terminal object satisfies.
    \end{enumerate}
\end{lemma}


\begin{theorem}[Baer's Criterion] \label{theorem:baers_criterion_for_injectivity_of_module_over_a_ring}
Let $R$ be a \CrefAndHyperrefIfExist{definition:ring}{(not necessarily commutative) ring} with unit and let $I$ be an \CrefAndHyperrefIfExist{definition:module_of_a_ring}{$R$-module}. Then $I$ is an \CrefAndHyperrefIfExist{definition:injective_and_projective_objects_in_a_category}{injective} $R$-module if and only if for every \CrefAndHyperrefIfExist{definition:ideal_of_a_ring}{left ideal} $J \subseteq R$, every \CrefAndHyperrefIfExist{definition:homomorphism_of_modules_over_a_ring}{$R$-module homomorphism}
$$f : J \to I$$
extends to an $R$-module homomorphism
$$\tilde{f} : R \to I.$$
\end{theorem}

Intuitively, Baer's criterion reduces the problem of verifying injectivity of a module to checking extension properties from the simplest possible submodules of the ring itself — its left ideals. This characterization is fundamental in homological algebra and module theory, and is widely used to identify and construct injective modules.


\begin{lemma} \label{lemma:injective_projective_objects_on_a_locally_small_category_induce_hom_functors_that_take_monomorphisms_epimoprhisms_to_surjections}
    Let $\calC$ be a \CrefAndHyperrefIfExist{definition:locally_small_category}{locally small category}. \TODO{Does this generalize to a category enriched in a symmetric monoidal category} 
    \begin{enumerate}
        \item  The following are equivalent for an object $I \in \calC$:
        \begin{itemize}
            \item $I$ is \CrefAndHyperrefIfExist{definition:injective_and_projective_objects_in_a_category}{injective} in $\calC$.
            \item The contravariant functor 
            $$\Hom_\calC(-,I): \calC^{\op} \to \Sets$$
            takes \CrefAndHyperrefIfExist{definition:monomorphism_and_epimorphism_in_categories}{monomorphisms} to surjections, i.e. for any monomorphism $A \hookrightarrow B$, the induced map 
            $$\Hom_\calC(B,I) \to \Hom_\calC(A,I)$$
            of sets is a surjection.
        \end{itemize}
        \item  The following are equivalent for an object $P \in \calC$:
        \begin{itemize}
            \item $P$ is \CrefAndHyperrefIfExist{definition:injective_and_projective_objects_in_a_category}{projective} in $\calC$.
            \item The covariant functor 
            $$\Hom_\calC(P,-): \calC \to \Sets$$
            takes \CrefAndHyperrefIfExist{definition:monomorphism_and_epimorphism_in_categories}{epimorphisms} to surjections, i.e. for any epimorphism $A \hookrightarrow B$, the induced map 
            $$\Hom_\calC(P,A) \to \Hom_\calC(A,B)$$
            of sets is a surjection.
        \end{itemize}
    \end{enumerate}
\end{lemma}
\begin{proof}
    These follow immediately from the definitions of injective and projective objects.
\end{proof}

\begin{lemma}[cf. e.g. {\cite[Lemma 2.3.4]{weibel}}] \label{lemma:injective_projective_objects_in_an_abelian_category_have_exact_Hom_functors}
    Let $\calA$ be a \CrefAndHyperrefIfExist{definition:abelian_category}{abelian category}. \TODO{Does this generalize to a category enriched in a symmetric monoidal category} 
    \begin{enumerate}
        \item  The following are equivalent for an object $I \in \calA$:
        \begin{itemize}
            \item $I$ is \CrefAndHyperrefIfExist{definition:injective_and_projective_objects_in_a_category}{injective} in $\calA$.
            \item The contravariant functor 
            $$\Hom_\calA(-,I): \calA^{\op} \to \Ab$$
            is \CrefAndHyperrefIfExist{definition:exact_functor_between_abelian_categories}{exact}.
        \end{itemize}
        \item  The following are equivalent for an object $P \in \calA$:
        \begin{itemize}
            \item $P$ is \CrefAndHyperrefIfExist{definition:injective_and_projective_objects_in_a_category}{projective} in $\calA$.
            \item The covariant functor 
            $$\Hom_\calA(P,-): \calA \to \Ab$$
            is \CrefAndHyperrefIfExist{definition:exact_functor_between_abelian_categories}{exact}.
        \end{itemize}
    \end{enumerate}
\end{lemma}
\begin{proof}
    \TODO{}
\end{proof}

\subsection{Resolutions of objects by chain complexes}

In this section, we list out some basic facts about \CrefAndHyperref{definition:left_right_resolution_of_a_class_of_objects_in_an_abelian_category}{resolutions}, particular of \CrefAndHyperref{definition:injective_and_projective_objects_in_a_category}{injective or projective} objects. These will be needed to define \CrefAndHyperref{definition:left_right_derived_functors_of_a_right_left_exact_functor_between_abelian_categories_where_source_has_enough_projectives_injectives}{derived functors}

\begin{definition} \label{definition:has_enough_objects_of_a_class_on_the_left_right_for_an_abelian_category}
    Let $\mathcal{A}$ be an \CrefAndHyperrefIfExist{definition:abelian_category}{abelian category} and let $\mathcal{X}$ be a class of objects in $\mathcal{A}$.

    We say that $\calA$ \hldef{has enough objects of class $\calX$ on the left (resp. on the right)} if for any object $M \in \calA$, there exists an object $X$ of the class $\calX$ and an \CrefAndHyperrefIfExist{definition:monomorphism_and_epimorphism_in_categories}{epimorphism} $X \twoheadrightarrow M$ (resp. a monomorphism $M \hookrightarrow X$). 
\end{definition}
\begin{definition} \label{definition:has_enough_injectives_or_projectives_for_an_abelian_category}
Let $\mathcal{A}$ be an \CrefAndHyperrefIfExist{definition:abelian_category}{abelian category}.
\begin{enumerate}
    \item $\mathcal{A}$ is said to \hldef{have enough injectives} if for every object $A$ in $\calA$, there is an \CrefAndHyperrefIfExist{definition:monomorphism_and_epimorphism_in_categories}{monomorphism} $A \to I$ with $I$ an \CrefAndHyperrefIfExist{definition:injective_and_projective_objects_in_a_category}{injective object} of $\calA$. \TextIfExistsElse{definition:has_enough_objects_of_a_class_on_the_left_right_for_an_abelian_category}{Equivalently, $\calA$ has enough injectives if it has enough objects of the class of injectives on the right (\Cref{definition:has_enough_objects_of_a_class_on_the_left_right_for_an_abelian_category})}

    \item $\mathcal{A}$ is said to \hldef{have enough projectives} if for every object $A$ in $\calA$, there is a \CrefAndHyperrefIfExist{definition:monomorphism_and_epimorphism_in_categories}{epimorphism} $P \to A$ with $P$ a \CrefAndHyperrefIfExist{definition:injective_and_projective_objects_in_a_category}{projective object} of $\calA$. \TextIfExistsElse{definition:has_enough_objects_of_a_class_on_the_left_right_for_an_abelian_category}{Equivalently, $\calA$ has enough projectives if it has enough objects of the class of projectives on the left (\Cref{definition:has_enough_objects_of_a_class_on_the_left_right_for_an_abelian_category})}

\end{enumerate}
\end{definition}

% See Also
% definition:left_right_derived_functors_of_a_right_left_exact_functor_between_abelian_categories_where_source_has_enough_projectives_injectives 

\begin{theorem} \label{theorem:examples_of_abelian_categories_with_enough_injectives_or_projectives}
\begin{enumerate}
    \item Examples of abelian categories with enough injectives include:
    \begin{itemize}
        \item The category of abelian groups.
        \item The category of modules over a ring.
        \item The category of sheaves of abelian groups on a ringed space or on an essentially small site.
    \end{itemize}

    \item Examples of abelian categories with enough projectives include:
    \begin{itemize}
        \item The category of modules over a ring with enough projectives (e.g., rings with unity and suitable properties). \TODO{make this more precise}
        \item The category of finitely generated modules over a semisimple ring.
    \end{itemize}
\end{enumerate}
% These conditions ensure the existence of derived functors such as Ext and Tor.
\end{theorem}
\begin{definition} \label{definition:left_right_resolution_of_a_class_of_objects_in_an_abelian_category}
Let $\mathcal{A}$ be an \CrefAndHyperrefIfExist{definition:abelian_category}{abelian category} and let $\mathcal{X}$ be a class of objects in $\mathcal{A}$. Let $M$ be an object of $\calA$.

\begin{enumerate}
    \item A \hldef{right resolution of $M$} is a \CrefAndHyperrefIfExist{definition:chain_complex_of_objects_in_an_additive_category}{cochain complex} $I^\bullet$ with $I^i = 0$ for $i < 0$ and a map $M \to I^0$ such that the augmented complex
    $$0 \to M \to I^0 \to I^1 \to I^2 \to \cdots$$
    is \CrefAndHyperrefIfExist{definition:acyclic_complex_of_objects_in_an_abelian_category}{exact}.

    % \item An \hldef{injective resolution of $M$} is a right resolution $I^\bullet$ for which the objects $I^i$ are all \CrefAndHyperrefIfExist{definition:injective_and_projective_objects_in_a_category}{injective}.

    \item A \hldef{left resolution of $M$} is a \CrefAndHyperrefIfExist{definition:chain_complex_of_objects_in_an_additive_category}{chain complex} $P_\bullet$ with $P_i = 0$ for $i < 0$ and a map $P_0 \to M$ such that the augmented complex
    $$\cdots P_2 \to P_1 \to P_0 \to M \to 0$$
    is \CrefAndHyperrefIfExist{definition:acyclic_complex_of_objects_in_an_abelian_category}{exact}.
    
    \item An \hldef{$\mathcal{X}$-left resolution} of an object $M \in \mathcal{A}$ a \CrefAndHyperrefIfExist{definition:left_right_resolution_of_a_class_of_objects_in_an_abelian_category}{left resolution} by objects of $X$, i.e. an exact complex
    $$ \cdots \to X_2 \to X_1 \to X_0 \to M \to 0 $$
    with each $X_i \in \mathcal{X}$.

    \item An \hldef{$\mathcal{X}$-right resolution} of an object $M \in \mathcal{A}$ a \CrefAndHyperrefIfExist{definition:left_right_resolution_of_a_class_of_objects_in_an_abelian_category}{right resolution} by objects of $X$, i.e. an exact complex
    $$ 0 \to M \to X^0 \to X^1 \to X^2 \to \cdots $$
    with each $X_i \in \mathcal{X}$.

    \item A \hldef{projective resolution of $M$} is a left resolution $P^\bullet$ for which the objects $P^i$ are all \CrefAndHyperrefIfExist{definition:injective_and_projective_objects_in_a_category}{projective}.

    \item An \hldef{injective resolution of $M$} is a right resolution $I^\bullet$ for which the objects $I^i$ are all \CrefAndHyperrefIfExist{definition:injective_and_projective_objects_in_a_category}{injective}.
\end{enumerate}

\end{definition}
\begin{lemma} \label{lemma:projective_injective_objects_in_an_abelian_category_always_have_a_projective_injctive_resolution}
    Let $\calA$ be an \CrefAndHyperrefIfExist{definition:abelian_category}{abelian category}.
    \begin{enumerate}
        \item A \CrefAndHyperrefIfExist{definition:injective_and_projective_objects_in_a_category}{projective object} $\calP$ always has a \CrefAndHyperrefIfExist{definition:left_right_resolution_of_a_class_of_objects_in_an_abelian_category}{projective resolution} given by 
        $$\cdots \to 0 \to \calP \xrightarrow{\id} \calP \to 0.$$
        \item A \CrefAndHyperrefIfExist{definition:injective_and_projective_objects_in_a_category}{injective object} $\calI$ always has a \CrefAndHyperrefIfExist{definition:left_right_resolution_of_a_class_of_objects_in_an_abelian_category}{injective resolution} given by 
        $$0 \to \calI \xrightarrow{\id} \calI \to 0 \to \cdots.$$
    \end{enumerate}
\end{lemma}

\begin{proof}
    This is clear.
\end{proof}

\begin{lemma} [cf. {\cite[Lemma 2.2.5, Lemma 2.3.6]{weibel}}] \label{lemma:an_object_of_abelian_category_with_enough_objects_of_a_class_on_the_right_left_has_right_left_resolution_by_the_class}
Let $\mathcal{A}$ be an \CrefAndHyperrefIfExist{definition:abelian_category}{abelian category} and let $\calX$ be a class of objects in $\calA$.

\begin{enumerate}
    \item If $\mathcal{A}$ \CrefAndHyperrefIfExist{definition:has_enough_objects_of_a_class_on_the_left_right_for_an_abelian_category}{has enough objects of class $\calX$ on the right}, then for every object $A \in \mathcal{A}$ there exists an \CrefAndHyperrefIfExist{definition:left_right_resolution_of_a_class_of_objects_in_an_abelian_category}{$\calX$-right resolution of $A$}.

    \item If $\mathcal{A}$ \CrefAndHyperrefIfExist{definition:has_enough_objects_of_a_class_on_the_left_right_for_an_abelian_category}{has enough objects of class $\calX$ on the left}, then for every object $A \in \mathcal{A}$ there exists an \CrefAndHyperrefIfExist{definition:left_right_resolution_of_a_class_of_objects_in_an_abelian_category}{$\calX$-left resolution of $A$}.

\end{enumerate}

Note that this is a special case of \Cref{proposition:abelian_category_with_enough_objets_of_a_class_on_the_right_left_has_resolutions_of_complexes} obtained by letting the complex $M^\bullet$ be the complex such that
$$M^i = \begin{cases} A &\text{if } i = 0 \\ 0 &\text{otherwise} \end{cases}.$$


In particular,
\begin{itemize}
    \item If $\mathcal{A}$ \CrefAndHyperrefIfExist{definition:has_enough_injectives_or_projectives_for_an_abelian_category}{has enough injective objects}, then for every object $A \in \mathcal{A}$ there exists an \CrefAndHyperrefIfExist{definition:left_right_resolution_of_a_class_of_objects_in_an_abelian_category}{injective resolution of $A$}.
    \item If $\mathcal{A}$ \CrefAndHyperrefIfExist{definition:has_enough_injectives_or_projectives_for_an_abelian_category}{has enough projective objects}, then for every object $A \in \mathcal{A}$ there exists a \CrefAndHyperrefIfExist{definition:left_right_resolution_of_a_class_of_objects_in_an_abelian_category}{projective resolution of $A$}.
    \item If $F: \calA \to \calB$ is a \CrefAndHyperrefIfExist{definition:exact_functor_between_abelian_categories}{left (resp. right) exact functor} between abelian categories and $\calA$ has enough $F$-acyclic objects on the right (resp. left), then for every object $A \in \calA$, there exists an \CrefAndHyperrefIfExist{definition:F_acyclic_resolution_for_a_right_left_exact_functor_between_abelian_categories}{right (resp. left) $F$-acyclic resolution} of $A$.
\end{itemize}

\end{lemma}
\begin{proof}
    \begin{enumerate}
        \item Let $A \in \calA$ be an object. Since $\calA$ has enough objects of class $\calX$ of the right, there is an object $X_0$ of $\calX$ and a monomorphism $\varepsilon_0: A \to X_0$. Let \CrefAndHyperrefIfExist{definition:kernel_and_cokernel_of_a_morphism_in_a_category}{$A_0 = \operatorname{coker} \varepsilon_0$}. Inductively, given an object $A_{n-1}$ of $\calA$, choose an object $X_n$ of $\calX$ and a monomorphism $\varepsilon_{n}: A_{n-1} \hookrightarrow X_n$. Let $A_n = \operatorname{coker} \varepsilon_n$. In particular, there is a surjection $X_n \twoheadrightarrow A_n$. Let $d_n$ be the composition
        $$X_{n-1} \twoheadrightarrow A_{n-1} \xrightarrow{\varepsilon_n} X_n.$$
        The chain complex
        $$0 \to A \xrightarrow{\varepsilon_0} X_0 \xrightarrow{d_0} X_1 \xrightarrow{d_1} \cdots$$
        is thus an $\calX$-right resolution of $A$.

        \item This is simply the dual statement of the next statement.

    \end{enumerate} 
\end{proof}

In fact, a generalization is possible: if the abelian category $\calA$ has enough objects in a class $\calX$ (on the right/left) \emph{complex}, then the complex has a ``resolution'' by objects of $\calX$. 

\begin{proposition} \label{proposition:abelian_category_with_enough_objets_of_a_class_on_the_right_left_has_resolutions_of_complexes}
    Let $\mathcal{A}$ be an \CrefAndHyperrefIfExist{definition:abelian_category}{abelian category} and let $\calX$ be a class of objects in $\calA$.
    \begin{enumerate}
        \item If $\mathcal{A}$ \CrefAndHyperrefIfExist{definition:has_enough_objects_of_a_class_on_the_left_right_for_an_abelian_category}{has enough objects of class $\calX$ on the right}, then for every \CrefAndHyperrefIfExist{definition:bounded_complexes_on_an_additive_category_and_homologically_bounded_objects_on_an_abelian_category}{bounded below} complex $M^\bullet$ of objects in $\calA$, there exists a bounded below complex $I^\bullet$ of objects in $\calX$ and a \CrefAndHyperrefIfExist{definition:quasi_isomorphism_of_chain_complexes_of_objects_in_an_abelian_category}{quasi-isomorphism} $M^\bullet \to I^\bullet$.
        
        \item If $\mathcal{A}$ \CrefAndHyperrefIfExist{definition:has_enough_objects_of_a_class_on_the_left_right_for_an_abelian_category}{has enough objects of class $\calX$ on the left}, then for every \CrefAndHyperrefIfExist{definition:bounded_complexes_on_an_additive_category_and_homologically_bounded_objects_on_an_abelian_category}{bounded above} complex $M^\bullet$ of objects in $\calA$, there exists a bounded above complex $P^\bullet$ of objects in $\calX$ and a \CrefAndHyperrefIfExist{definition:quasi_isomorphism_of_chain_complexes_of_objects_in_an_abelian_category}{quasi-isomorphism} $P^\bullet \to M^\bullet$.
    \end{enumerate}
\end{proposition}
\begin{proof}
    We prove that if $\calA$ has enough objects of class $\calX$ on the left, then there exists a complex $P^\bullet$ of objects in $\calX$ and a quasi-isomorphism $P^\bullet \to M^\bullet$. The other statement can be proven basically symmetrically.

    First suppose that $M^\bullet$ is \CrefAndHyperrefIfExist{definition:bounded_complexes_on_an_additive_category_and_homologically_bounded_objects_on_an_abelian_category}{bounded above}; say that $M^i = 0$ for all $i > n$. We inductively construct $P^\bullet$ and the quasi-isomoprhism to $M^\bullet$. Choose an object $P^n$ from $\calX$ and a surjective morphism $\epsilon_n: P^n \twoheadrightarrow M^n$, and let $d: P^n \to P^{n+1}$ be the zero map. Assume inductively that we have constructed the complex $P^\bullet$ and maps $\epsilon_i: P^i \to M^i$ for $i = k+1,k+2,\ldots,n$. We want to construct $P^k$, the differential $d:P^k \to P^{k+1}$ and the map $\epsilon_k: P^k \to M^k$. 

    Let $L_k = Z^{k+1}(P) \times_{Z^{k+1}(M)} M^k$
    \begin{center}
    \begin{tikzcd}
        L_k \ar[r] \ar[d] & Z^{k+1}(P) \ar[d, "\epsilon_{k+1}"] \\
        M^k \ar[r, "d" ] & Z^{k+1}(M)
    \end{tikzcd}
    \end{center}
    
    where $Z^{i}$ denotes the $i$th \CrefAndHyperrefIfExist{definition:boundary_cycle_coboundary_cocyble_of_a_chain_cochain_complex}{cycle} of a complex \CrefIfExists{definition:homology_and_cohomology_objects_for_a_chain_complex_in_an_additive_category}; recall that abelian categories have finite limits by \Cref{lemma:abelian_categories_are_finitely_complete_and_finitely_cocomplete}, so \CrefAndHyperrefIfExist{definition:cartesian_product_of_two_objects_in_a_category_over_an_object}{fiber products} exist. Choose an object $P^k$ from $\calX$ and a surjective moprhism $\pi: P^k \twoheadrightarrow L_k$. Set the differential $d: P^k \to P^{k+1}$ to be $\operatorname{proj}_{Z^{k+1}(P)} \circ \pi$ and the map $\epsilon_k: P^k \to M^k$ to be $\operatorname{proj}_{M^k} \circ \pi$. 
    
    We verify that the square
    \begin{center}
        \begin{tikzcd}
            P^k \ar[r, "d"] \ar[d, "\epsilon_k"]  & P^{k+1} \ar[d, "\epsilon_{k+1}"] \\
            M^k \ar[r, "d"] & M^{k+1} 
        \end{tikzcd}
    \end{center}
    commutes, i.e. that $\epsilon_{k+1} \circ d = d \circ \varepsilon_k$. The left is $\epsilon_{k+1} \circ \operatorname{proj}_{Z^{k+1}(P)} \circ \pi$ and the right is $d \circ \operatorname{proj}_{M^k} \circ \pi$, which do indeed coincide. 

    We verify that the chain map $\epsilon$ induces isomorphisms $H^{*} (P) \xrightarrow{\sim} H^{*}(M)$ on cohomology objects. We first show that the induced maps on cohomology are epimorphisms. Let $u:Z^k(M) \to L_k = Z^{k+1}(P) \times_{Z^{k+1}(M)} M^k$ be the unique morphism corresponding to the morphisms $0: Z^k(M) \to Z^{k+1}(P)$ and $Z^k(M) \hookrightarrow M^k$. Let $Y$ be the pullback of $\pi$ along $u$:
    \begin{center}
    \begin{tikzcd}
        Y \ar[r, "\tilde{\pi}"] \ar[d, "\tilde{u}"] & Z^k(M) \ar[d, "u"] \\
        P^k \ar[r, "\pi"] & L_k.
    \end{tikzcd}
    \end{center}
    Note that $\operatorname{im} \tilde{u}$ is a subobject of $Z^k(P) = \ker(d: P^k \to P^{k+1})$ because
    $$d \circ \tilde{u} = \operatorname{proj}_{Z^{k+1}(P)} \circ \pi \circ \tilde{u} = \operatorname{proj}_{Z^{k+1}(P)} \circ u \circ \tilde{\pi} = 0.$$
    Therefore, $\tilde{u}$ factors through $Z^k(P)$. Writing $[\tilde{u}]$ for the composition $Y \xrightarrow{\tilde{u}} Z^k(P) \twoheadrightarrow H^k(P)$, note that 
    $$H^k(\epsilon) \circ [\tilde{u}] = [\epsilon_k \circ \tilde{u}] = [\operatorname{proj}_{M^k} \circ \pi \circ \tilde{u}] = [\operatorname{proj}_{M^k} \circ u \circ \tilde{\pi}] = [(\id: Z^k(M) \to Z^k(M)) \circ \tilde{\pi}].$$
    The right most expression is the composition 
    $$Y \xrightarrow{\tilde{\pi}} Z^k(M) \twoheadrightarrow H^k(M).$$
    Since $\pi$ is an epimorphism, $\tilde{\pi}$ is an epimorphism, so the above composition is an epimorphism. We have thus shown that $H^k(\epsilon) \circ [\tilde{u}]$ is an epimorphism, so $H^k(\epsilon)$ is an epimorphism.

    We now show that $H^{k+1}(\epsilon): H^{k+1}(P) \to H^{k+1}(M)$ is a monomorphism. Let $K$ be the kernel of $Z^{k+1}(P) \xrightarrow{\epsilon_{k+1}} Z^{k+1}(M) \twoheadrightarrow H^{k+1}(M)$; this kernel coincides with ``the (k+1)-cycles of $P$ mapping to (k+1)-boundaries of $M$''. More precisely, $K$ can be regarded as the fiber product
    \begin{center}
    \begin{tikzcd}
        K \ar[r] \ar[d] & Z^{k+1}(P) \ar[d, "\epsilon_{k+1}"] \\ 
        B^{k+1}(M) \ar[r, hookrightarrow] & Z^{k+1}(M),
    \end{tikzcd}
    \end{center}
    and note that this Cartesian diagram displays $K$ as a subobject of $Z^{k+1}(P)$. Furthe note that the morphism $d: M^k \to B^{k+1}(M)$ naturally induces a morphism $L_k \to K$; in fact, $K$ is then the image of the projection map $\operatorname{proj}_{Z^{k+1}(P)}: L_k \to Z^{k+1}(P)$. On the other hand, by definition, 
    $$B^{k+1}(P) = \operatorname{im}(d: P^k \to Z^{k+1}(P)) = \operatorname{im}(\operatorname{proj}_{Z^{k+1}(P)} \circ \pi).$$
    Since $\pi$ is an epimorphism, this image in turn equals $\operatorname{im}(\operatorname{proj}_{Z^{k+1}(P)})$, which equals $K$ as we have seen. Therefore, $K$ coincides with $B^{k+1}(P)$, which means that the map $Z^{k+1}(P) \to H^{k+1}(M)$, whose kernel is $K$ by definition, naturally induces a monomorphism $H^{k+1}(P) \to H^{k+1}(M)$ as desired.

\end{proof}


\begin{lemma}[cf. {\cite[Porism 2.2.7]{weibel}}] \label{lemma:projective_injective_complex_with_map_to_from_object_with_left_right_resolution_lifts_uniquely_up_to_chain_homotopy}
    Let $\calA$ be an \CrefAndHyperrefIfExist{definition:abelian_category}{abelian category}.
    \begin{enumerate}
        \item Let
        $$\cdots \to P_2 \to P_1 \to P_0 \to M \to 0$$
        be a \CrefAndHyperrefIfExist{definition:chain_complex_of_objects_in_an_additive_category}{chain complex} with $P_i$ \CrefAndHyperrefIfExist{definition:injective_and_projective_objects_in_a_category}{projective}. For every \CrefAndHyperrefIfExist{definition:left_right_resolution_of_a_class_of_objects_in_an_abelian_category}{left resolution} $Q_\bullet \to N$ of an object $N$, every map $M \to N$ lifts to a \CrefAndHyperrefIfExist{definition:chain_complex_of_objects_in_an_additive_category}{complex map} $P_\bullet \to Q_\bullet$ unique up to \CrefAndHyperrefIfExist{definition:chain_homotopy_between_chain_maps_between_complexes}{chain homotopy}.

        \item Let
        $$0 \to M \to I^0 \to I^1 \to I^2 \to \cdots$$
        be a \CrefAndHyperrefIfExist{definition:chain_complex_of_objects_in_an_additive_category}{(co)chain complex} with $I^i$ \CrefAndHyperrefIfExist{definition:injective_and_projective_objects_in_a_category}{injective}. For every \CrefAndHyperrefIfExist{definition:left_right_resolution_of_a_class_of_objects_in_an_abelian_category}{right resolution} $N \to Q^\bullet$ of an object $N$, every map $N \to M$ lifts to a \CrefAndHyperrefIfExist{definition:chain_complex_of_objects_in_an_additive_category}{complex map} $Q^\bullet \to I^\bullet$ unique up to \CrefAndHyperrefIfExist{definition:chain_homotopy_between_chain_maps_between_complexes}{chain homotopy}.
    \end{enumerate}
\end{lemma}

\begin{proof}
    \begin{enumerate}
        \item The map $P_0 \to M \to N$ lifts to a map $P_0 \to Q_0$ because $P_0$ is projective and $Q_0 \to N$ is an epimorphism. 
        Inductively suppose that there are morphisms $P_i \to Q_i$ for $0 \leq i \leq n$, where $n \geq 0$ that make 
        \begin{center}
        \begin{tikzcd}
            P_n \ar[r] \ar[d] & P_{n-1} \ar[r] \ar[d] & \cdots \ar[r] & P_0 \ar[r] \ar[d] & M \ar[r] \ar[d] & 0 \\
            Q_n \ar[r] & Q_{n-1} \ar[r] & \cdots \ar[r] & Q_0 \ar[r] & N \ar[r] & 0 \\
        \end{tikzcd}
        \end{center}
        into a commuting diagram are established. The morphism $Q_{n} \to Q_{n-1}$ (where we let $Q_{-1} = N$ and $P_{-1} = M$ here in case that $n = 0$) acts as $0$ when restricted to \CrefAndHyperrefIfExist{definition:image_coimage_of_a_morphism_in_a_category}{$\mathfrak{I} \coloneq \operatorname{im} (P_{n+1} \to P_n \to Q_n)$} because the composition 
        $$P_{n+1} \to P_n \to Q_n \to Q_{n-1}$$
        equals the composition 
        $$P_{n+1} \to P_n \to P_{n-1} \to Q_{n-1}.$$
        In other words, $\mathfrak{I}$ is a \CrefAndHyperrefIfExist{definition:subobject_of_an_object_of_an_additive_category}{subobject} of \CrefAndHyperrefIfExist{definition:kernel_and_cokernel_of_a_morphism_in_a_category}{$\ker(Q_n \to Q_{n-1})$}, which is isomorphic to $\operatorname{im}(Q_{n+1} \to Q_n)$ by the acyclicity of the sequence of the $Q_i$'s. Therefore, we have a map $P_{n+1} \twoheadrightarrow \mathfrak{I}\hookrightarrow \operatorname{im}(Q_{n+1} \to Q_n)$ along with an epimorphism $Q_{n+1} \twoheadrightarrow \operatorname{im}(Q_{n+1} \to Q_n)$. Since $P_{n+1}$ is projective, the former map lifts to a map $P_{n+1} \to Q_{n+1}$ in a way that is compatible with the latter, i.e. the following commutes:
        \begin{center}
        \begin{tikzcd}
            P_{n+1} \ar[rd] \ar[d,dotted] & \\
            Q_{n+1} \ar[r] & \operatorname{im}(Q_{n+1} \to Q_n).
        \end{tikzcd}
        \end{center}
        By induction, this shows thta $M \to N$ lifts to a morphism $P_\bullet \to Q_\bullet$ of complexes.

        We show that the morphism of complexes is unique up to chain homotopy, i.e. if $f_1, f_2: P_\bullet \to Q_\bullet$ are two morphisms of complexes, then $h \coloneq f_1 - f_2$ is null homotopic. We construct a \CrefAndHyperrefIfExist{definition:chain_homotopy_between_chain_maps_between_complexes}{chain contraction} $\{s_n: P_n \to Q_{n+1}\}$ of $h$ by induction on $n$. If $n < 0$, then set $s_n = 0$. If $n = 0$, note that the composition $P_0 \xrightarrow{h_0} Q_0 \to N$ equals the composition $P_0 \to M \xrightarrow{0} N$, so $\operatorname{im}(h_0)$ is a subobject of $\ker(Q_0 \to N) \cong \operatorname{im}(Q_1 \to Q_0)$. The projectivitiy of $P_0$ thus yields a lift $s_0: P_0 \to Q_1$ such that $h_0$ equals the composition $P_0 \xrightarrow{s_0} X_1 \xrightarrow{d} Q_0$:
        \begin{center}
        \begin{tikzcd}
           &  P_0 \ar[dl, "s_0", dotted] \ar[d, "h_0"] \\
           Q_1 \ar[r, "d"] & Q_0 
        \end{tikzcd}
        \end{center}
        Note moreover that $h_0 = ds_0 + s_{-1} d$ because $s_{-1} = 0$. Inductively suppose that we have maps $s_i$ for $i \leq n$  such that $h_n = d s_{n} + s_{n-1} d$ or equivalently that $ds_{n} = h_n - s_{n-1} d$. Consider the map $h_{n+1} - s_{n} d: P_{n+1} \to Q_{n+1}$. Compute
        $$d(h_{n+1} - s_{n} d) = dh_{n+1} - d s_{n} d = dh_{n+1} - (h_n - s_{n-1}d)d = (dh_{n+1} - h_n d) + s_{n-1} d d = 0$$
        Therefore, $\operatorname{im} (h_{n+1} - s_n d)$ is a subobject of $\ker(Q_{n+1} \to Q_{n}) \cong \operatorname{im}(Q_{n+2} \to Q_{n+1})$, which is in turn a quotient of $Q_{n+2}$. Since $P_{n+1}$ is projective, there is a morphism $s_{n+1}: P_{n+1} \to Q_{n+2}$ such that $d s_{n+1} = h_{n+1} - s_{n} d$. 
        \begin{center}
        \begin{tikzcd}
           &  P_{n+1} \ar[dl, dotted, "s_{n+1}"] \ar[d, "h_n - s_{n-1} d = ds_{n}"] \\
           Q_{n+2} \ar[r, "d"]  &\operatorname{im}(Q_{n+2} \to Q_{n+1}) \cong \ker(Q_{n+1} \to Q_{n})
        \end{tikzcd}
        \end{center}
        The $s_n$ thus form a chain contraction as needed.

        \item This is simply dual to the previous part.
    \end{enumerate}
\end{proof}



\subsection{Derived functors of right or left exact functors between abelian categories where the source category has enough projectives or injectives}



\begin{definition} \label{definition:left_right_derived_functors_of_a_right_left_exact_functor_between_abelian_categories_where_source_has_enough_projectives_injectives}
    \TODO{I think that the definition of derived categories might be doable for more general kinds of resolutions? Perhaps it is that if I have a right exact functor $F$, then $L^i F$ can be computed with resolutions of $F$-acyclic objects? \CrefIfExists{definition:F_acyclic_object_for_a_left_or_right_functor_between_abelian_categories}}
    \TODO{Apparently, left/right derived functors may be defined for functors that are additive and preserve finite coproducts, and not necessarily right/left exact; the exactness condition ensures that the zeroth derived functor agrees with $F$.}
Let $\mathcal{A}$ and $\mathcal{B}$ be \CrefAndHyperrefIfExist{definition:abelian_category}{abelian categories}, and let 
$$F: \mathcal{A} \to \mathcal{B}$$ 
be an \CrefAndHyperrefIfExist{definition:additive_functor_between_additive_categories}{additive functor}.

\begin{enumerate}
    \item Suppose that the functor $F$ is \CrefAndHyperrefIfExist{definition:exact_functor_between_abelian_categories}{right exact} and suppose that $A \in \calA$ is an object for which a \CrefAndHyperrefIfExist{definition:left_right_resolution_of_a_class_of_objects_in_an_abelian_category}{projective resolution}
    $$\cdots \to P_2 \to P_1 \to P_0 \to A \to 0$$
    exists in $\calA$. We define the \hldef{left derived object} \hl{$L_n F A \in \calB$} by applying $F$ to obtain a complex
    $$\cdots \to F(P_2) \to F(P_1) \to F(P_0) \to 0$$
    and letting $L_n F(A)$ be the \CrefAndHyperrefIfExist{definition:homology_and_cohomology_objects_for_a_chain_complex_in_an_additive_category}{$n$-th homology object} of this complex in $\mathcal{B}$:
    $$L_n F(A) := H_n(F(P_\bullet)).$$
    The object $L_n F(A)$ is independent of the choice of projective resolution up to natural isomorphism (\Cref{proposition:left_right_derived_objects_for_a_right_left_exact_functor_between_abelian_categories_are_well_defined}). 

    By convention, set $L_n F = 0$ for $n < 0$.

    The \hldef{higher left derived objects} refer to the object $L_n F(A)$ for $n > 0$. 

    \item  Suppose that the functor $F$ is \CrefAndHyperrefIfExist{definition:exact_functor_between_abelian_categories}{right exact} and that $\calA$ \CrefAndHyperrefIfExist{definition:has_enough_injectives_or_projectives_for_an_abelian_category}{has enough projectives}. The \hldef{left derived functors} refer to the family of functors
    $$\hlin{L_n F : \mathcal{A} \to \mathcal{B}, \quad A \mapsto L_n F(A).}$$
    The \hldef{higher left derived functors} refer to the functors $L_n F$ for $n > 0$. 

    \item Suppose that the functor $F$ is \CrefAndHyperrefIfExist{definition:exact_functor_between_abelian_categories}{right exact} and suppose that $A \in \calA$ is an object for which a \CrefAndHyperrefIfExist{definition:left_right_resolution_of_a_class_of_objects_in_an_abelian_category}{injective resolution}
    $$0 \to A \to I^0 \to I^1 \to I^2 \to \cdots$$
    exists in $\calA$. We define the \hldef{right derived object} \hl{$R_n F A \in \calB$}, also often denoted by \hl{$R^n FA$}, by applying $F$ to obtain a complex
    $$0 \to F(I^0) \to F(I^1) \to F(I^2) \to \cdots.$$
    and letting $R_n F(A)$ be the \CrefAndHyperrefIfExist{definition:homology_and_cohomology_objects_for_a_chain_complex_in_an_additive_category}{$n$-th cohomology object} of this complex in $\mathcal{B}$:
    $$R_n F(A) := H^n(F(I_\bullet)).$$
    The object $R_n F(A)$ is independent of the choice of injective resolution up to natural isomorphism (\Cref{proposition:left_right_derived_objects_for_a_right_left_exact_functor_between_abelian_categories_are_well_defined}). 

    By convention, set $R_n F = 0$ for $n < 0$.

    The \hldef{higher right derived objects} refer to the object $R_n F(A)$ for $n > 0$. 

    \item  Suppose that the functor $F$ is \CrefAndHyperrefIfExist{definition:exact_functor_between_abelian_categories}{right exact} and that $\calA$ \CrefAndHyperrefIfExist{definition:has_enough_injectives_or_projectives_for_an_abelian_category}{has enough injectives}. The \hldef{right derived functors} refer to the family of functors
    $$\hlin{R_n F : \mathcal{A} \to \mathcal{B}, \quad A \mapsto R_n F(A).}$$
    The right derived functors are also often denoted by \hl{$R^n F$}.
    The \hldef{higher right derived functors} refer to the functors $R_n F$ for $n > 0$. 

    
    % If the functor $F$ is right exact and $\calA$ \CrefAndHyperrefIfExist{definition:has_enough_injectives_or_projectives_for_an_abelian_category}{has enough projectives}, then its \hldef{left derived functors} are a family of functors
    % $$\hlin{L_n F : \mathcal{A} \to \mathcal{B}, \quad n \geq 0,}$$
    % which are defined for each object $A$ in $\mathcal{A}$ by choosing (\Cref{lemma:an_object_of_abelian_category_with_enough_objects_of_a_class_on_the_right_left_has_right_left_resolution_by_the_class}) a \CrefAndHyperrefIfExist{definition:left_right_resolution_of_a_class_of_objects_in_an_abelian_category}{projective resolution}
    % $$\cdots \to P_2 \to P_1 \to P_0 \to A \to 0$$
    % in $\mathcal{A}$ and applying $F$ to obtain a complex
    % $$\cdots \to F(P_2) \to F(P_1) \to F(P_0) \to 0.$$
    % Then $L_n F(A)$ is defined to be the \CrefAndHyperrefIfExist{definition:homology_and_cohomology_objects_for_a_chain_complex_in_an_additive_category}{$n$-th homology object} of this complex in $\mathcal{B}$:
    % $$L_n F(A) := H_n(F(P_\bullet)).$$
    % The functors $L_n F$ are independent of the choice of projective resolution up to natural isomorphism. 

    % By convention, set $L_n F = 0$ for $n < 0$.

    % \item If the functor $F$ is \CrefAndHyperrefIfExist{definition:exact_functor_between_abelian_categories}{left exact} and $\calA$ \CrefAndHyperrefIfExist{definition:has_enough_injectives_or_projectives_for_an_abelian_category}{has enough injectives}, then its \hldef{right derived functors} are a family of functors
    % $$\hlin{R^n F : \mathcal{A} \to \mathcal{B}, \quad n \geq 0,}$$
    % which are defined for each object $A$ in $\mathcal{A}$ by choosing (\Cref{lemma:an_object_of_abelian_category_with_enough_objects_of_a_class_on_the_right_left_has_right_left_resolution_by_the_class}) an \CrefAndHyperrefIfExist{definition:left_right_resolution_of_a_class_of_objects_in_an_abelian_category}{injective resolution}
    % $$0 \to A \to I^0 \to I^1 \to I^2 \to \cdots$$
    % in $\mathcal{A}$ and applying $F$ to obtain a complex
    % $$0 \to F(I^0) \to F(I^1) \to F(I^2) \to \cdots.$$
    % Then $R^n F(A)$ is defined to be the \CrefAndHyperrefIfExist{definition:homology_and_cohomology_objects_for_a_chain_complex_in_an_additive_category}{$n$-th cohomology object} of this complex in $\mathcal{B}$:
    % $$R^n F(A) := H^n(F(I^\bullet)).$$
    % The functors $R^n F$ are independent of the choice of injective resolution up to natural isomorphism.

    % By convention, set $R^n F = 0$ for $n < 0$.
\end{enumerate}
\end{definition}

% \TODO{The below would require some notion of K-flatness, see stacks 06Y7, 06XZ}
% \begin{proposition} 
%     Let $\calA$ be an abelian category satisfying \CrefAndHyperrefIfExist{definition:grothendiecks_additional_axioms_for_abelian_categories}{AB5} and let $\calX$ be a class of objects in $\calA$ that is closed under direct sums.
%     \begin{enumerate}
%         \item If $\mathcal{A}$ \CrefAndHyperrefIfExist{definition:has_enough_objects_of_a_class_on_the_left_right_for_an_abelian_category}{has enough objects of class $\calX$ on the right}, then for every  complex $M^\bullet$ of objects in $\calA$, there exists a complex $I^\bullet$ of objects in $\calX$ and a \CrefAndHyperrefIfExist{definition:quasi_isomorphism_of_chain_complexes_of_objects_in_an_abelian_category}{quasi-isomorphism} $M^\bullet \to I^\bullet$.
        
%         \item If $\mathcal{A}$ \CrefAndHyperrefIfExist{definition:has_enough_objects_of_a_class_on_the_left_right_for_an_abelian_category}{has enough objects of class $\calX$ on the left}, then for every complex $M^\bullet$ of objects in $\calA$, there exists a complex $P^\bullet$ of objects in $\calX$ and a \CrefAndHyperrefIfExist{definition:quasi_isomorphism_of_chain_complexes_of_objects_in_an_abelian_category}{quasi-isomorphism} $P^\bullet \to M^\bullet$.
%     \end{enumerate}
% \end{proposition}
% \begin{proof}
%     \TODO{show that complex is the filtered directed limit of its brutal truncations}

%     We show the case of that $\calA$ has enough objects of class $\calX$ on the left; the other case is dual.  

%     We inductively build a system of complexes of objects of $\calX$ via \CrefAndHyperrefIfExist{definition:stupid_truncation_of_a_chain_complex_in_an_additive_category}{brutal truncations} of $M^\bullet$. Start with some bounded above complex $P_{n-1}^\bullet$ of $\tau_{\leq n-1}M^\bullet$ such that there is a \CrefAndHyperrefIfExist{definition:quasi_isomorphism_of_chain_complexes_of_objects_in_an_abelian_category}{quasi-isomorphism} $f_{n-1}: P_{n-1}^\bullet \to \tau_{\leq {n-1}} M^\bullet$, which exist by \Cref{proposition:abelian_category_with_enough_objets_of_a_class_on_the_right_left_has_resolutions_of_complexes}. \TODO{define truncation triangle} Consider the truncation triangle
%     $$\tau_{\leq n-1}M^\bullet \to \tau_{\leq n}M^\bullet \to M^n[-n] \to.$$
%     Take a complex $F^\bullet$ of objects from $\calX$ and a quasi-isomorphism $F^\bullet \to M^n[-n]$. In particular, we have the distinguished triangle
%     $$M^n[-n][-1] \to \tau_{n-1} M^\bullet \to \tau_{\leq n} M^\bullet.$$
%     Take the map $F^\bullet[-1] \to P_{n-1}^\bullet$ in the \CrefAndHyperrefIfExist{definition:derived_category_of_an_abelian_category}{derived category} and construct $P_n^\bullet$ as its mapping cone. It can then be represented 
%     \TODO{mapping cone and how it makes sense in the homotopy category and in the derived category; also try to show why the derived category works in the context of having enough objects here.}
% \end{proof}


\begin{lemma}[Horseshoe lemma, cf. {\cite[Horsehoe Lemma 2.2.8]{weibel}}] \label{lemma:horseshoe_lemma_of_projective_injective_resolutions_in_abelian_categories}
    Let $\calA$ be an \CrefAndHyperrefIfExist{definition:abelian_category}{abelian category}. 
    \begin{enumerate}
        \item Suppose that
            $$0 \to A' \xrightarrow{i_A} A \xrightarrow{\pi_A} A'' \to 0$$
            is a short exact sequence in $\calA$, and that $\varepsilon': P'_\bullet \to A'$ and $\varepsilon'': P''_{\bullet} \to A''$ are respectively \CrefAndHyperrefIfExist{lemma:flat_resolution_lemma_of_tor_objects_of_abelian_categories_with_a_right_exact_bifunctor_assuming_that_category_has_enough_projectives_or_flats}{projective resolutions}. 
            \begin{center}
                \begin{tikzcd}
                & & & 0 \ar[d] & \\
                \cdots P_2' \ar[r] & P_1' \ar[r] & P_0' \ar[r, "\varepsilon'"] & A' \ar[r] \ar[d, "i_A"] & 0 \\
                & & & A \ar[d, "\pi_A"] & \\
                \cdots P_2'' \ar[r] & P_1'' \ar[r] & P_0'' \ar[r, "\varepsilon''"] & A'' \ar[r] \ar[d] & 0 \\ 
                & & & 0 & 
                \end{tikzcd}
            \end{center}
            Let $P_\bullet = P'_\bullet \oplus P''_\bullet$. The complex $P_\bullet$ is a projective resolution of $A$, and the short exact sequence lifts to an exact esquence of complexes
            $$0 \to P' \xrightarrow{i} P \xrightarrow{\pi} P'' \to 0$$
            where $i_n: P_n' \to P_n$ and $\pi_n: P_n \to P_n''$ are the natural inclusion and projection respectively.
        \item Suppose that
        $$0 \to A' \xrightarrow{i_A} A \xrightarrow{\pi_A} A'' \to 0$$
        is a short exact sequence in $\calA$, and that $\eta': A' \to I'^\bullet$ and $\eta'': A'' \to I''^\bullet$ are respectively \CrefAndHyperrefIfExist{lemma:injective_resolution_lemma}{injective resolutions}. 

        \begin{center}
            \begin{tikzcd} 
            & 0 \ar[d] & & & \\
            0 \ar[r] & A' \ar[r, "\eta'"] \ar[d, "i_A"] & {I'}^{0} \ar[r] & {I'}^{1} \ar[r] & {I'}^{2} \cdots \\
            & A \ar[d, "\pi_A"] & & & \\
            0 \ar[r] & A' \ar[r, "\eta''"] \ar[d] & {I''}^{0} \ar[r] & {I''}^{1} \ar[r] & {I''}^{2} \cdots \\
            & 0 & & & 
            \end{tikzcd}
        \end{center}
        Let $I^\bullet = I'^\bullet \oplus I''^\bullet$. The complex $I^\bullet$ is an injective resolution of $A$, and the short exact sequence lifts to a short exact sequence of complexes
        $$0 \to I'^\bullet \xrightarrow{i} I^\bullet \xrightarrow{\pi} I''^\bullet \to 0,$$
        where $i^n: I'^n \to I^n$ and $\pi^n: I^n \to I''^n$ are the natural inclusion and projection at each degree \(n\).
    \end{enumerate}
\end{lemma}

% \TODO{delete the following; it seems we genuinely need the projectivity/injectivity}
% \begin{lemma}
%     Let $\calA$ be an \CrefAndHyperrefIfExist{definition:abelian_category}{abelian category}. 
%     Let $\calX$ be a class of objects in $\calA$ such that $\calX$ is closed under \CrefAndHyperrefIfExist{definition:additive_category_preadditive_category}{direct sums}. 
%     \begin{enumerate}
%         \item Suppose that
%             $$0 \to A' \xrightarrow{i_A} A \xrightarrow{\pi_A} A'' \to 0$$
%             is a short exact sequence in $\calA$, and that $\varepsilon': P'_\bullet \to A'$ and $\varepsilon'': P''_{\bullet} \to A''$ are respectively \CrefAndHyperrefIfExist{definition:left_right_resolution_of_a_class_of_objects_in_an_abelian_category}{$\calX$-left resolutions}. 
%             \begin{center}
%                 \begin{tikzcd}
%                 & & & 0 \ar[d] & \\
%                 \cdots P_2' \ar[r] & P_1' \ar[r] & P_0' \ar[r, "\varepsilon'"] & A' \ar[r] \ar[d, "i_A"] & 0 \\
%                 & & & A \ar[d, "\pi_A"] & \\
%                 \cdots P_2'' \ar[r] & P_1'' \ar[r] & P_0'' \ar[r, "\varepsilon''"] & A'' \ar[r] \ar[d] & 0 \\ 
%                 & & & 0 & 
%                 \end{tikzcd}
%             \end{center}
%             Let $P_\bullet = P'_\bullet \oplus P''_\bullet$. The complex $P_\bullet$ is an $\calX$-left resolution of $A$, and the short exact sequence lifts to an exact esquence of complexes
%             $$0 \to P' \xrightarrow{i} P \xrightarrow{\pi} P'' \to 0$$
%             where $i_n: P_n' \to P_n$ and $\pi_n: P_n \to P_n''$ are the natural inclusion and projection respectively.
%         \item Suppose that
%         $$0 \to A' \xrightarrow{i_A} A \xrightarrow{\pi_A} A'' \to 0$$
%         is a short exact sequence in $\calA$, and that $\eta': A' \to I'^\bullet$ and $\eta'': A'' \to I''^\bullet$ are respectively \CrefAndHyperrefIfExist{definition:left_right_resolution_of_a_class_of_objects_in_an_abelian_category}{$\calX$-right resolutions}. 

%         \begin{center}
%             \begin{tikzcd} 
%             & 0 \ar[d] & & & \\
%             0 \ar[r] & A' \ar[r, "\eta'"] \ar[d, "i_A"] & {I'}^{0} \ar[r] & {I'}^{1} \ar[r] & {I'}^{2} \cdots \\
%             & A \ar[d, "\pi_A"] & & & \\
%             0 \ar[r] & A' \ar[r, "\eta''"] \ar[d] & {I''}^{0} \ar[r] & {I''}^{1} \ar[r] & {I''}^{2} \cdots \\
%             & 0 & & & 
%             \end{tikzcd}
%         \end{center}
%         Let $I^\bullet = I'^\bullet \oplus I''^\bullet$. The complex $I^\bullet$ is an $\calX$-right resolution of $A$, and the short exact sequence lifts to a short exact sequence of complexes
%         $$0 \to I'^\bullet \xrightarrow{i} I^\bullet \xrightarrow{\pi} I''^\bullet \to 0,$$
%         where $i^n: I'^n \to I^n$ and $\pi^n: I^n \to I''^n$ are the natural inclusion and projection at each degree \(n\).
%     \end{enumerate}

% \end{lemma}

\begin{proof}
    \TODO{}
\end{proof}
\begin{proposition}[cf.{\cite[Lemma 2.4.1]{weibel}}] \label{proposition:left_right_derived_objects_for_a_right_left_exact_functor_between_abelian_categories_are_well_defined}
    Let $F: \calA \to \calB$ be an \CrefAndHyperrefIfExist{definition:additive_functor_between_additive_categories}{additive functor} between \CrefAndHyperrefIfExist{definition:abelian_category}{abelian categories}. Let $A$ be an object of $\calA$. 
    \begin{enumerate}
        \item Suppose that $F$ is \CrefAndHyperrefIfExist{definition:exact_functor_between_abelian_categories}{right exact}, and suppose that a \CrefAndHyperrefIfExist{definition:left_right_resolution_of_a_class_of_objects_in_an_abelian_category}{projective resolution}
        $$\cdots \to P_2 \to P_1 \to P_0 \to A \to 0$$
        of $A$ exists in $\calA$. Let 
        $$\cdots \to Q_2 \to Q_1 \to Q_0 \to A \to 0$$
        be any projective resolution of $A$ in $\calA$. For all $n$, there are natural isomorphisms
        $$H_n(F(P_\bullet)) \cong H_n(F(Q_\bullet)).$$
        In other words, the \CrefAndHyperrefIfExist{definition:left_right_derived_functors_of_a_right_left_exact_functor_between_abelian_categories_where_source_has_enough_projectives_injectives}{left derived objects $L_n F(A)$} is well defined.

        \item Suppose that $F$ is \CrefAndHyperrefIfExist{definition:exact_functor_between_abelian_categories}{left exact}, and suppose that a \CrefAndHyperrefIfExist{definition:left_right_resolution_of_a_class_of_objects_in_an_abelian_category}{injective resolution}
        $$0 \to A \to I^0 \to I^1 \to I^2 \to \cdots$$
        of $A$ exists in $\calA$. Let 
        $$0 \to A \to Q^0 \to Q^1 \to Q^2 \to \cdots$$
        be any injective resolution of $A$ in $\calA$. For all $n$, there are natural isomorphisms
        $$H_n(F(I^\bullet)) \cong H_n(F(Q^\bullet)).$$
        In other words, the \CrefAndHyperrefIfExist{definition:left_right_derived_functors_of_a_right_left_exact_functor_between_abelian_categories_where_source_has_enough_projectives_injectives}{right derived objects $R_n F(A)$} is well defined.
    \end{enumerate}
\end{proposition}

\begin{proof}
    \begin{enumerate}
        \item By \Cref{lemma:projective_injective_complex_with_map_to_from_object_with_left_right_resolution_lifts_uniquely_up_to_chain_homotopy}, there is a lift $f: P_\bullet \to Q_\bullet$ of the identity map $A \to A$ unique up to chain homotopy. There are then induced natural maps $H_n(F(f)): H_n(F(P_\bullet)) \to H_n(F(Q_\bullet))$. There is also a lift $f': Q_\bullet \to P_\bullet$ of the identity map $A \to A$ unique up to chain homotopy, and this also induces natural maps $H_n(F(f')): H_n(F(Q_\bullet)) \to H_n(F(P_\bullet))$. The chain maps $f$ and $f'$ are in fact \CrefAndHyperrefIfExist{definition:chain_homotopy_between_chain_maps_between_complexes}{chain homotopy inverses} because \Cref{lemma:projective_injective_complex_with_map_to_from_object_with_left_right_resolution_lifts_uniquely_up_to_chain_homotopy} also implies that any lifts $P_\bullet \to P_\bullet$ and $Q_\bullet \to Q_\bullet$ of the identity map $A \to A$ are chain homotopic to the identity chain maps. Therefore, $H_n(F(f))$ and $H_n(F(f'))$ are inverses of each other as morphisms in $\calB$.  \TODO{prove basic facts about the fucntoriality of homology/cohomology of chain complexes}

        \item This is dual to the previous part.
    \end{enumerate}
\end{proof}

\subsection{Homological and cohomological $\delta$ functors}

\CrefAndHyperref{definition:left_right_derived_functors_of_a_right_left_exact_functor_between_abelian_categories_where_source_has_enough_projectives_injectives}{Derived functors} are the main example of \CrefAndHyperrefIfExist{definition:homological_cohomological_delta_functor_between_abelian_categories}{$\delta$-functors}, which associate long exact sequences to short exact sequences.

% Definition: Homological Delta Functor
\begin{definition} \label{definition:homological_cohomological_delta_functor_between_abelian_categories}
Let $\mathcal{A}$ and $\mathcal{B}$ be \CrefAndHyperrefIfExist{definition:abelian_category}{abelian categories}. 
\begin{enumerate}
    \item  A \hldef{homological $\delta$ functor from $\mathcal{A}$ to $\mathcal{B}$} is a pair $(T_n, \delta_n)_{n\ge 0}$ consisting of:
    \begin{itemize}
        \item a sequence of \CrefAndHyperrefIfExist{definition:additive_functor_between_additive_categories}{additive functors} $T_n : \mathcal{A} \to \mathcal{B}$ for each integer $n \ge 0$, and
        \item for every \CrefAndHyperrefIfExist{definition:short_exact_sequence_in_an_additive_category}{short exact sequence} $0 \to A' \xrightarrow{u} A \xrightarrow{v} A'' \to 0$ in $\mathcal{A}$, a \hldef{connecting morphism}
        $$\hlin{\delta_n(A',A,A'') : T_n(A'') \to T_{n-1}(A'})$$
        in $\mathcal{B}$ for each $n > 0$
    \end{itemize}
    that make the induced sequence
    $$\cdots \to T_{n+1}(A'') \xrightarrow{\delta_{n+1}} T_n(A') \to T_n(A) \to T_n(A'') \xrightarrow{\delta_n} T_{n-1}(A') \to \cdots$$
    \CrefAndHyperrefIfExist{definition:acyclic_complex_of_objects_in_an_abelian_category}{exact} in $\mathcal{B}$, and are natural in short exact sequences. That is, for any morphism of short exact sequences, the induced morphisms between these long exact sequences commute.

    \item A \hldef{cohomological $\delta$-functor from $\mathcal{A}$ to $\mathcal{B}$} is defined dually: it consists of \CrefAndHyperrefIfExist{definition:additive_functor_between_additive_categories}{additive functors} $T^n: \mathcal{A} \to \mathcal{B}$ for $n \ge 0$ and \hldef{connecting morphisms}
    $$\delta^n(A',A,A''): T^n(A') \to T^{n+1}(A'')$$
    such that for each \CrefAndHyperrefIfExist{definition:short_exact_sequence_in_an_additive_category}{short exact sequence} $0 \to A' \to A \to A'' \to 0$ the resulting sequence
    $$0 \to T^0(A') \to T^0(A) \to T^0(A'') \xrightarrow{\delta^0} T^1(A') \to T^1(A) \to T^1(A'') \xrightarrow{\delta^1} \cdots$$
    is \CrefAndHyperrefIfExist{definition:acyclic_complex_of_objects_in_an_abelian_category}{exact} and the construction is natural with respect to morphisms of short exact sequences.

    \item Let $(T_n,\delta_n)$ and $(S_n,\partial_n)$ be homological $\delta$-functors from $\mathcal{A}$ to $\mathcal{B}$. A \hldef{morphism of (homological) $\delta$-functors}
    $$\eta: T_\bullet \to S_\bullet$$
    is a family of \CrefAndHyperrefIfExist{definition:natural_transformation_between_functors_between_categories}{natural transformations} $\eta_n: T_n \Rightarrow S_n$ for each $n \ge 0$ such that for every short exact sequence $0 \to A' \xrightarrow{u} A \xrightarrow{v} A'' \to 0$, the following diagram in $\mathcal{B}$ commutes for all $n > 0$:
    $$
    \begin{array}{ccccccccc}
    T_n(A'') & \xrightarrow{\delta_n} & T_{n-1}(A') & & \\
    \downarrow\eta_n & & \downarrow\eta_{n-1} & & \\
    S_n(A'') & \xrightarrow{\partial_n} & S_{n-1}(A') & &
    \end{array}$$
    The dual notion (for cohomological $\delta$-functors) is defined analogously, reversing the direction of the connecting morphisms.
\end{enumerate}
\end{definition}
\begin{theorem}[cf. {\cite[Theorem 2.4.6]{weibel}}] \label{theorem:long_exact_sequence_of_left_right_derived_functors_and_homological_delta_functors}
    Let $F: \calA \to \calB$ be an \CrefAndHyperrefIfExist{definition:additive_functor_between_additive_categories}{additive functor} between \CrefAndHyperrefIfExist{definition:abelian_category}{abelian categories}.
    \begin{enumerate}
        \item Suppose that $F$ is \CrefAndHyperrefIfExist{definition:exact_functor_between_abelian_categories}{right exact}. 
        \begin{enumerate}
            \item Let 
            $$0 \to A' \to A \to A'' \to 0$$
            be a \CrefAndHyperrefIfExist{definition:short_exact_sequence_in_an_additive_category}{short exact sequence} in $\calA$. Suppose that there exist \CrefAndHyperrefIfExist{definition:left_right_resolution_of_a_class_of_objects_in_an_abelian_category}{projective resolutions} $P' \to A'$ and $P'' \to A''$. There exists a long exact sequence
            $$\cdots \xrightarrow{\partial} L_i F(A') \to L_iF(A) \to L_iF(A'') \xrightarrow{\partial} L_{i-1} F(A') \to L_{i-1}F(A) \to L_{i-1}F(A'') \xrightarrow{\partial} \cdots$$
            of \CrefAndHyperrefIfExist{definition:left_right_derived_functors_of_a_right_left_exact_functor_between_abelian_categories_where_source_has_enough_projectives_injectives}{derived objects} and this long exact sequence is natural.

            \item If $\calA$ \CrefAndHyperrefIfExist{definition:has_enough_injectives_or_projectives_for_an_abelian_category}{has enough projectives}, then the derived functor $L_* F$ form a \CrefAndHyperrefIfExist{definition:homological_cohomological_delta_functor_between_abelian_categories}{homological $\delta$-functor}.
        \end{enumerate}

        \item Suppose that $G$ is \CrefAndHyperrefIfExist{definition:exact_functor_between_abelian_categories}{left exact}. 
            \begin{enumerate}
                \item Let 
                $$0 \to A' \to A \to A'' \to 0$$
                be a \CrefAndHyperrefIfExist{definition:short_exact_sequence_in_an_additive_category}{short exact sequence} in $\calA$. Suppose that there exist \CrefAndHyperrefIfExist{definition:left_right_resolution_of_a_class_of_objects_in_an_abelian_category}{injective resolutions} $I' \to A'$, $I \to A$, and $I'' \to A''$. There exists a long exact sequence
                $$
                \cdots \xrightarrow{\partial} R^i G(A') \to R^i G(A) \to R^i G(A'') \xrightarrow{\partial} R^{i+1} G(A') \to R^{i+1} G(A) \to R^{i+1} G(A'') \xrightarrow{\partial} \cdots
                $$
                of \CrefAndHyperrefIfExist{definition:left_right_derived_functors_of_a_right_left_exact_functor_between_abelian_categories_where_source_has_enough_projectives_injectives}{derived objects}, and this long exact sequence is natural.

                \item If $\calA$ \CrefAndHyperrefIfExist{definition:has_enough_injectives_or_projectives_for_an_abelian_category}{has enough injectives}, then the derived functors $R^* G$ form a \CrefAndHyperrefIfExist{definition:homological_cohomological_delta_functor_between_abelian_categories}{cohomological $\delta$-functor}.
            \end{enumerate}
    \end{enumerate}
\end{theorem}

\begin{proof}
\begin{enumerate}
    \item 
    \begin{enumerate}
        \item By the \CrefAndHyperrefIfExist{lemma:horseshoe_lemma_of_projective_injective_resolutions_in_abelian_categories}{Horseshoe lemma}, there is a projective resolution $P \to A$ fitting into a short exact sequence $0 \to P' \to P \to P'' \to 0$. Since the $P_n''$ are projective, each sequence $0 \to P_n' \to P_n \to P_n'' \to 0$ is split exact. \TODO{show that SES's ending in projective objects are split}. Since $F$ is additive, each sequence 
        $$0 \to F(P_n') \to F(P_n) \to F(P_n'') \to 0$$
        is split exact in $\calB$ \TODO{show why}. Therefore,
        $$0 \to F(P') \to F(P) \to F(P'') \to 0$$
        is a short exact sequence of chain complex. Its associated \CrefAndHyperrefIfExist{theorem:long_exact_sequence_of_left_right_derived_functors_and_homological_delta_functors}{long exact sequence in homology} is the desired long exact sequence. 

        We now show that the long exact sequence is natural. \TODO{continue}
    \end{enumerate}
\end{enumerate}
\end{proof}

\TODO{define universal delta functors}
\TODO{Put a staetment about how derived functors are universal delta functors}

\subsection{Homological and cohomological dimension of an additive functor}

\begin{definition}[Homological dimension] \label{definition:homological_cohomological_dimension_of_objects_in_abelian_categories_with_enough_projectives_injectives}
Let $\mathcal{A}$ be an \CrefAndHyperrefIfExist{definition:abelian_category}{abelian category} and let $M \in \mathcal{A}$ be an object.
\begin{enumerate}
    \item Suppose that $M$ has a \CrefAndHyperrefIfExist{definition:left_right_resolution_of_a_class_of_objects_in_an_abelian_category}{projective resolution}.

    The \hldef{homological/projective dimension of $M$}, denoted by notations such as \hl{$\mathrm{hdim}(M)$} and \hl{$\mathrm{hd}(M)$}, is the smallest integer $n \geq 0$ such that there exists a projective resolution of $M$ of length $n$, i.e., an \CrefAndHyperrefIfExist{definition:acyclic_complex_of_objects_in_an_abelian_category}{exact sequence}
    $$ 0 \to P_n \to P_{n-1} \to \cdots \to P_0 \to M \to 0, $$
    with each $P_i$ \CrefAndHyperrefIfExist{definition:injective_and_projective_objects_in_a_category}{projective}. 

    If no such finite $n$ exists, we say $\mathrm{hdim}(M) = \infty$.

    \item Suppose that $M$ has an \CrefAndHyperrefIfExist{definition:left_right_resolution_of_a_class_of_objects_in_an_abelian_category}{injective resolution}.

    The \hldef{cohomological/injective dimension of $M$}, denoted by notations such as \hl{$\mathrm{cdim}(M)$} or \hl{$\mathrm{cd}(M)$}, is the smallest integer $n \geq 0$ such that there exists a injective resolution of $M$ of length $n$, i.e., an \CrefAndHyperrefIfExist{definition:acyclic_complex_of_objects_in_an_abelian_category}{exact sequence}
    $$0 \to M \to I^1 \to \cdots \to I^{n-1} \to I^n \to 0, $$
    with each $I^i$ \CrefAndHyperrefIfExist{definition:injective_and_projective_objects_in_a_category}{injective}. 

    If no such finite $n$ exists, we say $\mathrm{cdim}(M) = \infty$.
\end{enumerate}
\end{definition}
\begin{definition}[Cohomological/homological dimension of a functor] \label{definition:cohomological_homological_dimension_of_a_left_right_exact_functor_on_an_abelian_category_with_enough_projectives_injectives}
Let $F : \mathcal{A} \to \mathcal{B}$ be an \CrefAndHyperrefIfExist{definition:additive_functor_between_additive_categories}{additive functor} between \CrefAndHyperrefIfExist{definition:abelian_category}{abelian categories}. 
\begin{enumerate}
    \item Assume that $F$ is \CrefAndHyperrefIfExist{definition:exact_functor_between_abelian_categories}{left exact} and that $\calA$ \CrefAndHyperrefIfExist{definition:has_enough_injectives_or_projectives_for_an_abelian_category}{has enough projectives}. The \hldef{cohomological dimension of $F$}, denoted \hl{$\operatorname{cdim}(F)$} or \hl{$\operatorname{cd}(F)$}, is the smallest integer $n \geq 0$ such that the \CrefAndHyperrefIfExist{definition:left_right_derived_functors_of_a_right_left_exact_functor_between_abelian_categories_where_source_has_enough_projectives_injectives}{right derived functors $R^i F$} vanish for all $i > n$, i.e.,
    $$
    R^i F = 0 \quad \text{for all } i > n.
    $$

    If no such finite $n$ exists, then $\mathrm{cdim}(F) = \infty$.

    \item Assume that $F$ is \CrefAndHyperrefIfExist{definition:exact_functor_between_abelian_categories}{right exact} and that $\calA$ \CrefAndHyperrefIfExist{definition:has_enough_injectives_or_projectives_for_an_abelian_category}{has enough injectives}. The \hldef{homological dimension of $F$}, denoted \hl{$\operatorname{hdim}(F)$} or \hl{$\operatorname{hd}(F)$}, is the smallest integer $n \geq 0$ such that the \CrefAndHyperrefIfExist{definition:left_right_derived_functors_of_a_right_left_exact_functor_between_abelian_categories_where_source_has_enough_projectives_injectives}{left derived functors $L_i F$} vanish for all $i > n$, i.e.,
    $$
    L_i F = 0 \quad \text{for all } i > n.
    $$

    If no such finite $n$ exists, then $\mathrm{hdim}(F) = \infty$.
\end{enumerate}
\end{definition}

% \begin{definition}[Tor dimension]
% Let $R$ be a ring, and let $M$ be an $R$-module.

% The \hldef{Tor dimension} of $M$, denoted $\mathrm{tdim}_R(M)$, is the smallest integer $n \geq 0$ such that
% $$
% \mathrm{Tor}^R_i(M, -) = 0 \quad \text{for all } i > n,
% $$
% where $\mathrm{Tor}^R_i$ are the left derived functors of the tensor product.

% Equivalently, it is the minimal length of a flat resolution of $M$. If such an $n$ does not exist, $\mathrm{tdim}_R(M) = \infty$.
% \end{definition}


\subsection{$F$-acyclic objects}

Recall by definition that \CrefAndHyperref{definition:left_right_derived_functors_of_a_right_left_exact_functor_between_abelian_categories_where_source_has_enough_projectives_injectives}{derived functors} are computed using \CrefAndHyperrefIfExist{lemma:projective_injective_objects_in_an_abelian_category_always_have_a_projective_injctive_resolution}{projective/injective} \CrefAndHyperrefIfExist{definition:left_right_resolution_of_a_class_of_objects_in_an_abelian_category}{resolutions}. In \Cref{proposition:left_right_derived_objects_for_a_right_left_exact_functor_between_abelian_categories_may_be_computed_with_acyclic_resolutions}, we see that derived functors can be more generally computed with resolutions of \CrefAndHyperref{definition:F_acyclic_object_for_a_left_or_right_functor_between_abelian_categories}{$F$-acyclic}. 

\begin{definition}[{\texorpdfstring{$F$}{F}-acyclic object for a functor of categories}] \label{definition:F_acyclic_object_for_a_left_or_right_functor_between_abelian_categories}
Let $\mathcal{A}$ and $\mathcal{B}$ be \CrefAndHyperrefIfExist{definition:abelian_category}{abelian categories}. 
\begin{enumerate}
    \item Let
    $$ F : \mathcal{A} \to \mathcal{B} $$
    be a \CrefAndHyperrefIfExist{definition:exact_functor_between_abelian_categories}{right exact functor}.

    An object $A \in \mathcal{A}$ for which a \CrefAndHyperrefIfExist{definition:left_right_resolution_of_a_class_of_objects_in_an_abelian_category}{projective resolution} exists is called \hldef{$F$-acyclic} if for all integers $n > 0$, its \CrefAndHyperrefIfExist{definition:left_right_derived_functors_of_a_right_left_exact_functor_between_abelian_categories_where_source_has_enough_projectives_injectives}{higher left derived functors} vanish:
    $$ L_n F (A) = 0 \text{ for all } n > 0  $$

    \item Let
    $$ F : \mathcal{A} \to \mathcal{B} $$
    be a \CrefAndHyperrefIfExist{definition:exact_functor_between_abelian_categories}{left exact functor}.

    An object $A \in \mathcal{A}$ for which a \CrefAndHyperrefIfExist{definition:left_right_resolution_of_a_class_of_objects_in_an_abelian_category}{injective resolution} exists is called \hldef{$F$-acyclic} if for all integers $n > 0$, its \CrefAndHyperrefIfExist{definition:left_right_derived_functors_of_a_right_left_exact_functor_between_abelian_categories_where_source_has_enough_projectives_injectives}{higher right derived functors} vanish:
    $$ R_n F (A) = 0 \text{ for all } n > 0  $$

\end{enumerate}
\end{definition}
\begin{definition}[$F$-acyclic resolution for a right/left exact functor] \label{definition:F_acyclic_resolution_for_a_right_left_exact_functor_between_abelian_categories}
Let $\mathcal{A}$ and $\mathcal{B}$ be \CrefAndHyperrefIfExist{definition:abelian_category}{abelian categories} and let
$$
F : \mathcal{A} \to \mathcal{B}
$$
be an \CrefAndHyperrefIfExist{definition:additive_functor_between_additive_categories}{additive functor}.
\begin{enumerate}
    \item Suppose that $F$ is a \CrefAndHyperrefIfExist{definition:exact_functor_between_abelian_categories}{right exact functor}. \hldef{An (left) $F$-acyclic resolution} of an object $A \in \mathcal{A}$ is a \CrefAndHyperrefIfExist{definition:left_right_resolution_of_a_class_of_objects_in_an_abelian_category}{left resolution of $A$ for} the class of \CrefAndHyperrefIfExist{definition:F_acyclic_object_for_a_left_or_right_functor_between_abelian_categories}{$F$-acyclic objects}.

    \item Suppose that $F$ is a \CrefAndHyperrefIfExist{definition:exact_functor_between_abelian_categories}{left exact functor}. \hldef{An (right) $F$-acyclic resolution} of an object $A \in \mathcal{A}$ is a \CrefAndHyperrefIfExist{definition:left_right_resolution_of_a_class_of_objects_in_an_abelian_category}{right resolution of $A$ for} the class of \CrefAndHyperrefIfExist{definition:F_acyclic_object_for_a_left_or_right_functor_between_abelian_categories}{$F$-acyclic objects}.
\end{enumerate}
\end{definition}
\begin{lemma} \label{lemma:any_projective_injective_object_of_an_abelian_category_is_acyclic_for_a_right_left_exact_functor}
    Let $F: \calA \to \calB$ be an \CrefAndHyperrefIfExist{definition:additive_functor_between_additive_categories}{additive functor} between \CrefAndHyperrefIfExist{definition:abelian_category}{abelian categories}.
    \begin{enumerate}
        \item If $F$ is \CrefAndHyperrefIfExist{definition:exact_functor_between_abelian_categories}{right exact}, then any \CrefAndHyperrefIfExist{definition:injective_and_projective_objects_in_a_category}{projective object} $\calP$ is \CrefAndHyperrefIfExist{definition:F_acyclic_object_for_a_left_or_right_functor_between_abelian_categories}{$F$-acyclic}. 

        \item If $F$ is \CrefAndHyperrefIfExist{definition:exact_functor_between_abelian_categories}{left exact}, then any \CrefAndHyperrefIfExist{definition:injective_and_projective_objects_in_a_category}{injective object} $\calI$ is \CrefAndHyperrefIfExist{definition:F_acyclic_object_for_a_left_or_right_functor_between_abelian_categories}{$F$-acyclic}. 
    \end{enumerate}
\end{lemma}

\begin{proof}
    Note that any projective/injective object has a \CrefAndHyperrefIfExist{definition:left_right_resolution_of_a_class_of_objects_in_an_abelian_category}{projective/injective resolution} (\Cref{lemma:projective_injective_objects_in_an_abelian_category_always_have_a_projective_injctive_resolution}), so we may talk about the \CrefAndHyperrefIfExist{definition:left_right_derived_functors_of_a_right_left_exact_functor_between_abelian_categories_where_source_has_enough_projectives_injectives}{derived objects $L_n F (\calP)$ and $R^n F (\calI)$} of projective objects $\calP$ and injective objects $\calI$ of $\calA$. In fact, the projective/injective resolution is simply given by 
    $$\cdots \to 0 \to \calP \xrightarrow{\id} \calP \to 0$$
    and
    $$0 \to \calI \xrightarrow{\id} \calI \to 0 \to \cdots$$
    so we compute $L_n F(\calP) = 0$ and $R^n F (\calI) = 0$. 
\end{proof}

\begin{lemma}[Dimension shifting, cf. {\cite[Exercise 2.4.3]{weibel}}] \label{lemma:dimension_shifting_lemma_for_acyclic_syzygies}
    Let $F: \calA \to \calB$ be an \CrefAndHyperrefIfExist{definition:additive_functor_between_additive_categories}{additive functor} between \CrefAndHyperrefIfExist{definition:abelian_category}{abelian categories}. Let $A$ be an object of $\calA$. 
    \begin{enumerate}
        \item Suppose that $F$ is \CrefAndHyperrefIfExist{definition:exact_functor_between_abelian_categories}{right exact}. 
        \begin{enumerate}

            \item Suppose that $0 \to M \to C \to A \to 0$ is an exact sequence in $\calA$ where $C$ is $F$-acyclic, and that $A$ and $M$ have \CrefAndHyperrefIfExist{definition:left_right_resolution_of_a_class_of_objects_in_an_abelian_category}{projective resolutions}. We have $L_i F(A) \cong L_{i-1} F(M)$ for $i \geq 2$ and $L_1 F(A) \cong \ker(F(M) \to F(C))$.  

            \item Suppose that $\calA$ \CrefAndHyperrefIfExist{definition:has_enough_injectives_or_projectives_for_an_abelian_category}{has enough projectives}.
            Let
            $$0 \to M_m \to C_m \to C_{m-1} \to \cdots \to C_0 \to A \to 0$$
            be an \CrefAndHyperrefIfExist{definition:acyclic_complex_of_objects_in_an_abelian_category}{acyclic complex} with the $C_i$ \CrefAndHyperrefIfExist{definition:F_acyclic_object_for_a_left_or_right_functor_between_abelian_categories}{$F$-acyclic}. We have
            % and such that $M_m$ has a projective resolution. We have 
            \begin{enumerate}
                \item $L_i F(A) \cong L_{i-m-1} F(M_m)$ for $i \geq m+2$ and 
                \item $L_{m+1} F(A) \cong \ker (F(M_m) \to F(C_m))$. 
            \end{enumerate}


        \end{enumerate}

        \item \TODO{dual statement}

    \end{enumerate}

\end{lemma}

\begin{proof}
    \begin{enumerate}
        \item Suppose that $F$ is right exact
        \begin{enumerate}
            \item Since $C$ is $F$-acyclic, the \CrefAndHyperrefIfExist{theorem:long_exact_sequence_of_left_right_derived_functors_and_homological_delta_functors}{long exact sequence of derived functors associated to} $0 \to M \to C \to A \to 0$ yields an exact sequence
            $$0 \to L_1 F(A) \to F(M) \to F(C) \to F(A) \to 0$$
            along with isomorphisms $L_{i} F(A) \cong L_{i-1} F(M)$ for $i \geq 2$. In particular, $L_1F(A) = \ker (F(M) \to F(C))$.

            \item We now proceed by induction. The above proves the base case of $m = 0$. Associated to the acyclic complex
            $$0 \to M_m \to C_m \to C_{m-1} \to \cdots \to C_0 \to A \to 0$$
            is a short exact sequence
            $$0 \to M_m \to C_m \to M_{m-1} \to 0$$
            where $M_{m-1} = \operatorname{coker}(M_m \to C_{m})$ and an acyclic complex
            $$0 \to M_{m-1} \to C_{m-1} \to \cdots \to C_0 \to A \to 0.$$  
            The derived functor long exact sequence of the short exact sequence yields isomorphisms 
            $$L_iF(M_{m-1}) \cong L_{i-1} F(M_m)$$
            for all $i \geq 1$. By induction, we also have $L_iF(A) \cong L_{i-(m-1)-1} F(M_{m-1}) = L_{i-m} F(M_{m-1})$ for $i \geq m+1$, so we in fact have
            $$L_i F(A) \cong L_{i-m} F(M_{m-1}) \cong L_{i-m-1} F(M_m)$$
            for all $i \geq m+1$. Moreover, the derived functor long exact sequence also includes
            $$0 \to L_1 F(M_{m-1}) \to F(M_m) \to F(C_m) \to F(M_{m-1}) \to 0,$$
            so $L_1 F(M_{m-1}) \cong \ker (F(M_m) \to F(C_m))$. We found that $L_1F(M_{m-1}) \cong L_{m+1}F(A)$, so $L_{m+1} F(A)$ is the kernel of $F(M_m) \to F(C_m)$ as desired.
        \end{enumerate}

        \item  These hold dually.
    \end{enumerate}

    
    
    
\end{proof}

\begin{proposition}[cf.{\cite[Exercise 2.4.3]{weibel}}] \label{proposition:left_right_derived_objects_for_a_right_left_exact_functor_between_abelian_categories_may_be_computed_with_acyclic_resolutions}
    Let $F: \calA \to \calB$ be an \CrefAndHyperrefIfExist{definition:additive_functor_between_additive_categories}{additive functor} between \CrefAndHyperrefIfExist{definition:abelian_category}{abelian categories}. Let $A$ be an object of $\calA$. 
    \begin{enumerate}
        \item Suppose that $F$ is \CrefAndHyperrefIfExist{definition:exact_functor_between_abelian_categories}{right exact}, and suppose that a \CrefAndHyperrefIfExist{definition:left_right_resolution_of_a_class_of_objects_in_an_abelian_category}{projective resolution}
        $$\cdots \to P_2 \to P_1 \to P_0 \to A \to 0$$
        of $A$ exists in $\calA$. Let 
        $$\cdots \to Q_2 \to Q_1 \to Q_0 \to A \to 0$$
        be any \CrefAndHyperrefIfExist{definition:F_acyclic_resolution_for_a_right_left_exact_functor_between_abelian_categories}{$F$-acyclic (left) resolution} of $A$ in $\calA$. For all $n$, there are natural isomorphisms
        $$H_n(F(P_\bullet)) \cong H_n(F(Q_\bullet)).$$
        In particular, the left derived objects $L_n F (A)$ may be computed using any $F$-acyclic (left) resolution of $A$ and are independent of the choice of left resolution. In particular, since all projective objects are $F$-acyclic (\Cref{lemma:any_projective_injective_object_of_an_abelian_category_is_acyclic_for_a_right_left_exact_functor}), $L_n F(A)$ is well defined.

        \item Suppose that $F$ is \CrefAndHyperrefIfExist{definition:exact_functor_between_abelian_categories}{left exact}, and suppose that a \CrefAndHyperrefIfExist{definition:left_right_resolution_of_a_class_of_objects_in_an_abelian_category}{injective resolution}
        $$0 \to A \to I^0 \to I^1 \to I^2 \to \cdots$$
        of $A$ exists in $\calA$. Let 
        $$0 \to A \to Q^0 \to Q^1 \to Q^2 \to \cdots$$
        be any \CrefAndHyperrefIfExist{definition:F_acyclic_resolution_for_a_right_left_exact_functor_between_abelian_categories}{$F$-acyclic (right) resolution} of $A$ in $\calA$. For all $n$, there are natural isomorphisms
        $$H_n(F(I^\bullet)) \cong H_n(F(Q^\bullet)).$$
        In particular, the right derived objects $R_n F (A)$ may be computed using any $F$-acyclic (right) resolution of $A$ and are independent of the choice of right resolution. In particular, since all injectives objects are $F$-acyclic (\Cref{lemma:any_projective_injective_object_of_an_abelian_category_is_acyclic_for_a_right_left_exact_functor}), $R_n F(A)$ is well defined.
    \end{enumerate}
\end{proposition}

\begin{proof}
    \TODO{There is a ``dimension shifting'' fact that goes into proving this kind of thing.}
\end{proof}


\subsection{Double complexes}

Just as we define and compute \CrefAndHyperref{definition:left_right_derived_functors_of_a_right_left_exact_functor_between_abelian_categories_where_source_has_enough_projectives_injectives}{derived functors} of (\CrefAndHyperref{definition:exact_functor_between_abelian_categories}{left/right exact}) single-variate \CrefAndHyperref{definition:additive_functor_between_additive_categories}{additive functors} by (projective/injective/$F$-acyclic) \CrefAndHyperref{definition:left_right_resolution_of_a_class_of_objects_in_an_abelian_category}{resolutions} of objects, we can define and compute \CrefAndHyperref{definition:generalized_derived_functors_for_biadditive_functors_of_abelian_categories_via_resolutions_by_flat_objects}{derived functors} of \CrefAndHyperref{definition:n_ary_additive_functor_between_additive_categories}{bi-additive} functors via \CrefAndHyperref{definition:flat_resolution_of_an_object_in_an_abelian_category_with_respect_to_a_right_exact_monoidal_product_functor}{flat resolutions}. To show that such derived functors of bi-additive functors are well defined under mild conditions (see \Cref{theorem:balancing_generalized_derived_functors_of_a_biadditive_functor_of_abelian_categories_computed_via_flat_resolutions}), it is useful (see \Cref{lemma:applying_biadd_func_to_res_is_defined_up_to_quasi_iso_and_agrees_with_the_tot_complex_of_the_double_complex_of_the_biadd_func_on_the_two_res}) to use obtain (\Cref{definition:double_complex_associated_to_biadditive_functor_and_chain_complexes}) \CrefAndHyperref{definition:double_complex_of_objects_in_an_additive_category}{double complexes} from the biadditive functor.

\begin{definition} \label{definition:double_complex_of_objects_in_an_additive_category}
Let $\mathcal{A}$ be an \CrefAndHyperrefIfExist{definition:additive_category}{additive category}. A \hldef{double complex} (also called a \hldef{bicomplex}) in $\mathcal{A}$ is a collection of objects
$$ \{ A^{p,q} \}_{(p,q)\in\mathbb{Z}^2} $$
together with morphisms
$$ d'_A{}^{p,q} : A^{p,q} \to A^{p+1,q}, \quad d''_A{}^{p,q} : A^{p,q} \to A^{p,q+1}, $$
satisfying the identities
$$ (d'_A)^{p+1,q} \circ (d'_A)^{p,q} = 0, \qquad (d''_A)^{p,q+1} \circ (d''_A)^{p,q} = 0, $$
and the \hldef{anti-commutativity relation}
$$ (d''_A)^{p+1,q} \circ (d'_A)^{p,q} + (d'_A)^{p,q+1} \circ (d''_A)^{p,q} = 0.  $$

An alternative (equivalent) convention defines a double complex $(A^{p,q},d'_A,d''_A)$ so that
$$ (d''_A)^{p+1,q} \circ (d'_A)^{p,q} - (d'_A)^{p,q+1} \circ (d''_A)^{p,q} = 0.  $$
This convention differs by a sign and corresponds to replacing $d''_A$ by $-d''_A$. Thus, both conventions are equivalent up to the natural isomorphism $A^{p,q}\mapsto A^{p,q}$, $d'_A\mapsto d'_A$, $d''_A\mapsto -d''_A$.
\end{definition}

\begin{definition} \label{definition:morphism_of_double_complex_of_objects_in_an_additive_category}
Let $\mathcal{A}$ be an \CrefAndHyperrefIfExist{definition:additive_category_preadditive_category}{additive category}, and let
$$(A^{p,q}, d'_A, d''_A)_{(p,q)\in \mathbb{Z}^2}, \quad (B^{p,q}, d'_B, d''_B)_{(p,q)\in \mathbb{Z}^2}$$
be \CrefAndHyperrefIfExist{definition:double_complex_of_objects_in_an_additive_category}{double complexes} in $\mathcal{A}$.

A \hldef{morphism of double complexes} $f : A \to B$ is a collection of morphisms
$$ f^{p,q} : A^{p,q} \to B^{p,q}, $$
for all $(p,q) \in \mathbb{Z}^2$, such that the following diagrams commute:

    \hlalign{
    \begin{align*}
    d'_B{}^{p,q} \circ f^{p,q} &= f^{p+1,q} \circ d'_A{}^{p,q}, \\ 
    d''_B{}^{p,q} \circ f^{p,q} &= f^{p,q+1} \circ d''_A{}^{p,q}.
    \end{align*}
    }

In other words, $f$ respects both horizontal and vertical differentials of the double complexes.

The double complexes and their morphisms form a category, sometimes denoted by \hl{$\mathbf{DC}(\mathcal{A})$}.
\end{definition}

\begin{theorem} \label{theorem:category_of_double_complexes_of_objects_of_an_additive_category_is_naturally_isomorphic_to_the_category_of_chain_complexes_of_chain_complexes}
Let $\mathcal{A}$ be an \CrefAndHyperrefIfExist{definition:additive_category}{additive category}. Consider the category \CrefAndHyperrefIfExist{definition:morphism_of_double_complex_of_objects_in_an_additive_category}{$\mathbf{DC}(\mathcal{A})$} of \CrefAndHyperrefIfExist{definition:double_complex_of_objects_in_an_additive_category}{double complexes} in $\mathcal{A}$ with \CrefAndHyperrefIfExist{definition:morphism_of_double_complex_of_objects_in_an_additive_category}{morphisms}, and the category $\mathbf{Ch}(\mathbf{Ch}(\mathcal{A}))$ of \CrefAndHyperrefIfExist{definition:chain_complex_of_objects_in_an_additive_category}{chain complexes} in the category of chain complexes over $\mathcal{A}$.

\TODO{natural isomorphism of categories}
There exists a natural isomorphism of categories
$$ \mathbf{DC}(\mathcal{A}) \cong \mathbf{Ch}(\mathbf{Ch}(\mathcal{A})), $$
given by the sign trick: for a double complex $(A^{p,q}, d'_A, d''_A)$, define the associated chain complex of chain complexes by adjusting the vertical differentials as
$$ d''^{p,q}_{\mathrm{new}} := (-1)^p d''^{p,q}_A, $$
while keeping the horizontal differentials $d'_A$ unchanged.

This identification respects morphisms of double complexes and chain complexes of chain complexes, making the two categories canonically equivalent.
\end{theorem}

Given two chain complexes and a biadditive functor of their underlying additive categories, we can obtain a double complex.

\begin{lemma}[Product of Additive Categories is Additive] \label{lemma:small_product_of_additive_categories_is_additive}
Let $I$ be any set and let $\{\mathcal{A}_i\}_{i \in I}$ be a family of \CrefAndHyperrefIfExist{definition:additive_category}{additive categories}. 
\begin{enumerate}
    \item The \CrefAndHyperrefIfExist{definition:product_category_of_a_family_of_categories}{product category} $\prod_{i \in I} \mathcal{A}_i$ is additive. Explicitly:
        \begin{itemize}
            \item Each hom-set 
            $$\mathrm{Hom}_{\prod_i \mathcal{A}_i}((A_i)_i,(B_i)_i) = \prod_{i \in I} \mathrm{Hom}_{\mathcal{A}_i}(A_i,B_i)$$
            is an abelian group under componentwise addition.
            \item Composition is bilinear with respect to this group structure.
            \item The family $(0_i)_{i \in I}$ of zero objects in each $\mathcal{A}_i$ is a zero object in $\prod_i \mathcal{A}_i$.
            \item Finite direct sums exist and are given componentwise.
        \end{itemize}
    \item If each $\calA_i$ is \CrefAndHyperrefIfExist{definition:abelian_category}{abelian}, then so is $\prod_{i \in I} \mathcal{A}_i$. \TODO{If each $\calA_i$ satisfies Ab1-Ab5, does the product also?} Explicitly,
    \begin{itemize}
    \item Kernels and cokernels exist and are computed componentwise:
    $$\ker((f_i)_i) = (\ker(f_i))_i, \quad \mathrm{coker}((f_i)_i) = (\mathrm{coker}(f_i))_i.$$
    \item Every monomorphism is the kernel of its cokernel, and every epimorphism is the cokernel of its kernel.
    \end{itemize}
\end{enumerate}
\end{lemma}

\begin{definition}[n-ary Additive Functor] \label{definition:n_ary_additive_functor_between_additive_categories}
Let $I$ be a finite set with $|I| = n$. Let $\{\mathcal{A}_i\}_{i\in I}$ be \CrefAndHyperrefIfExist{definition:additive_category_preadditive_category}{additive categories} and let $\mathcal{B}$ be an additive category. An \hldef{n-ary additive functor} (or \hldef{multilinear functor})
$$F : \prod_{i\in I}\mathcal{A}_i \to \mathcal{B}$$
\CrefIfExists{definition:product_category_of_a_family_of_categories} is a functor such that for each fixed collection of all but one variable, the resulting functor in the remaining variable is \CrefAndHyperrefIfExist{definition:additive_functor_between_additive_categories}{additive}. Equivalently, for every $j\in I$ and objects $(A_i)_{i\in I}$ and morphisms $f_1,f_2:A_j\to A'_j$ in $\mathcal{A}_j$, we have
\begin{align*}
&F(A_1,\dots,A_{j-1}, f_1+f_2, A_{j+1},\dots,A_n)
\\
=& F(A_1,\dots,A_{j-1}, f_1, A_{j+1},\dots,A_n)  \\
& + F(A_1,\dots,A_{j-1}, f_2, A_{j+1},\dots,A_n),
\end{align*}
and $F$ preserves zero morphisms componentwise:
$$F(A_1,\dots,0_{A_j,A'_j},\dots,A_n) = 0_{F(A_1,\dots),F(A'_1,\dots)}.$$
A bifunctor that satisfies this property for $n=2$ is simply called a \hldef{biadditive functor}.
\end{definition}


\begin{definition}[Double Complex associated to biadditive functor and chain complexes] \label{definition:double_complex_associated_to_biadditive_functor_and_chain_complexes}
    Let $\mathcal{A}$, $\mathcal{B}$, and $\calC$ be \CrefAndHyperrefIfExist{definition:additive_category}{additive categories}, and suppose that
    $$F : \mathcal{A} \times \mathcal{B} \to \mathcal{C} $$
    is a \CrefAndHyperrefIfExist{definition:n_ary_additive_functor_between_additive_categories}{biadditive functor}. Let $X_\bullet$ and $Y_\bullet$ be \CrefAndHyperrefIfExist{definition:chain_complex_of_objects_in_an_additive_category}{chain complexes} of objects in $\calA$ and $\calB$ respectively.

    Construct the \CrefAndHyperrefIfExist{definition:double_complex_of_objects_in_an_additive_category}{double complex} \(Z_{\bullet,\bullet} = (Z_{n,m}, d^h_{n,m}, d^v_{n,m})\) in \(C\) associated to \(F\), \(X_\bullet\), and \(Y_\bullet\) as follows:
    \[ Z_{n,m} := F(X_n, Y_m), \]
    with horizontal differentials
    \[ d^h_{n,m} := F(d^X_n, \mathrm{id}_{Y_m}): Z_{n,m} \to Z_{n-1,m}, \]
    and vertical differentials
    \[ d^v_{n,m} := (-1)^n F(\mathrm{id}_{X_n}, d^Y_m): Z_{n,m} \to Z_{n,m-1}.  \]
    These differentials indeed satisfy the double complex conditions:
    \[ d^h \circ d^h = 0, \quad d^v \circ d^v = 0, \quad \text{and} \quad d^h \circ d^v + d^v \circ d^h = 0.  \]
    We may often denote the double complex $Z_{\bullet,\bullet}$ by \hl{$F(X_\bullet, Y_\bullet)$}. In particular, $F$ induces a bifunctor  
    $$F: \mathbf{Ch}(\calA) \times \mathbf{Ch}(\calB) \to \mathbf{DC}(\calC)$$
    (\Cref{definition:morphism_of_double_complex_of_objects_in_an_additive_category})
    that is in fact a biadditive functor of additive categories (see \Cref{proposition:category_of_chain_complexes_in_an_additive_category_is_additive}) \TODO{verify that we indeed get a biadditive functor}

    In particular, we may speak of the \CrefAndHyperrefIfExist{definition:total_complexes_of_a_double_complex_of_objects_in_an_additive_category}{total complexes} $\Tot^{\oplus}(F(X_\bullet, Y_\bullet))$ and $\Tot^{\prod}(F(X_\bullet, Y_\bullet))$, and these specify biadditive functors
    $$\Tot^{\oplus}(F(-,-)), \Tot^{\prod}(F(-,-)): \mathbf{Ch}(\calA) \times \mathbf{Ch}(\calB) \to \mathbf{Ch}(\calC).$$

\end{definition}



\begin{definition} \label{definition:bounded_double_complex_of_objects_in_an_additive_category}
Let $(A^{p,q},d'_A,d''_A)$ be a double complex in an additive category $\mathcal{A}$.
\begin{itemize}
  \item The double complex is called \hldef{bounded above} if there exist integers $p_0,q_0$ such that $A^{p,q}=0$ whenever $p>p_0$ or $q>q_0$.
  \item The double complex is called \hldef{bounded below} if there exist integers $p_0,q_0$ such that $A^{p,q}=0$ whenever $p<p_0$ or $q<q_0$.

  \item The double complex is called \hldef{bounded} if it is both bounded above and below. 
  \item The double complex is said to be in the \hldef{first quadrant} (also called \emph{first-quadrant double complex}) if $A^{p,q}=0$ whenever $p<0$ or $q<0$. In particular, any first quadrant double complex is bounded below.
  \item The double complex is said to be in the \hldef{third quadrant} (also called \emph{third-quadrant double complex}) if $A^{p,q}=0$ whenever $p>0$ or $q>0$. In particular, any third quadrant double complex is bounded above.

  \item Let us say that the double complex is \hldef{locally finite along diagonals} or \hldef{locally bounded along diagonals}\footnote{These do not seem to be standard teminology.} if for each integer $n$, there exist at most finitely many pairs $(p,q)$ with $p+q = n$ such thta $A^{p,q} \neq 0$. 
  
  \item Let us say that the double complex is \hldef{bounded in total degree}\footnote{This does not seem to be standard teminology.} if there exist integers $m$ and $M$ such that $A^{p,q} = 0$ whenever $m \leq p+q \leq M$. 
\end{itemize}
\end{definition}

\begin{definition} \label{definition:total_complexes_of_a_double_complex_of_objects_in_an_additive_category}
Let $(A^{p,q},d'_A,d''_A)$ be a \CrefAndHyperrefIfExist{definition:double_complex_of_objects_in_an_additive_category}{double complex} in an \CrefAndHyperrefIfExist{definition:additive_category_preadditive_category}{additive category} $\mathcal{A}$. For each integer $n$, define:

    \hlalign{
    \begin{align*}
    \Tot_{\oplus}^n(A) &= \bigoplus_{p+q=n} A^{p,q}, \\
    \Tot_{\Pi}^n(A) &= \prod_{p+q=n} A^{p,q}.
    \end{align*}
    }

assuming that the direct sum $\bigoplus_{p+q=n} A^{p,q}$ and the product $\prod_{p+q=n} A^{p,q}$ respectively exist.
These are called the \hldef{direct-sum total complex} and the \hldef{product total complex}, respectively.

Define the differentials categorically as follows:

\begin{itemize}
  \item For the direct sum total complex, the differential $d^n : \Tot_{\oplus}^n(A) \to \Tot_{\oplus}^{n+1}(A)$ is the unique morphism such that, for each $(p,q)$ with $p+q=n$, we have
    \[
    d^n \circ \iota_{p,q} =
      \iota_{p+1,q} \circ d'_A {}^{p,q}
      + (-1)^p\, \iota_{p,q+1} \circ d''_A {}^{p,q},
    \]
    where $\iota_{p,q} : A^{p,q} \to \Tot_{\oplus}^n(A)$ is the canonical inclusion.

  \item For the product total complex, the differential $d^n : \Tot_{\Pi}^n(A) \to \Tot_{\Pi}^{n+1}(A)$ is the unique morphism such that, for each $(p,q)$ with $p+q=n+1$, we have
    \[
    \pi_{p,q} \circ d^n =
      d'_A {}^{p-1,q} \circ \pi_{p-1,q}
      + (-1)^{p-1}\, d''_A {}^{p,q-1} \circ \pi_{p,q-1},
    \]
    where $\pi_{p,q} : \Tot_{\Pi}^{n+1}(A) \to A^{p,q}$ is the canonical projection.
\end{itemize}

Then $(\Tot_{\oplus}^\bullet(A), d)$ and $(\Tot_{\Pi}^\bullet(A), d)$ are 
\CrefAndHyperrefIfExist{definition:chain_complex_of_objects_in_an_additive_category}{chain complexes} in~$\mathcal{A}$, whenever the corresponding sums or products exist. 
The complexes $\Tot_{\oplus}^\bullet(A)$ and $\Tot_{\Pi}^\bullet(A)$ are also denoted by \hl{$\Tot^{\oplus}(A)$} and \hl{$\Tot^{\Pi}(A)$}.
\end{definition}

\begin{lemma} \label{lemma:total_complexes_of_double_complex_with_bounded_diagonals_are_canonically_isomorphic}
    Let $C$ be a \CrefAndHyperrefIfExist{definition:double_complex_of_objects_in_an_additive_category}{double complex} of objects in an \CrefAndHyperrefIfExist{definition:additive_category}{additive category} $\calA$. If $C$ is \CrefAndHyperrefIfExist{definition:bounded_double_complex_of_objects_in_an_additive_category}{locally bounded along diagonals},
    % for each $n$, the total number of pairs $(p,q)$ such that $p+q = n$ and $C^{p,q} \neq 0$ is finite, 
    then the complexes \CrefAndHyperrefIfExist{definition:total_complexes_of_a_double_complex_of_objects_in_an_additive_category}{$\Tot^{\bigoplus}(A)$ and $\Tot^{\prod}(A)$} are naturally isomorphic.
\end{lemma}

\begin{proof}
    Since $C$ is locally bounded along diagonals, the degreee $n$ components
    \begin{align*}
    \left(\Tot^{\oplus}\right)^n(A) &= \bigoplus_{p+q=n} A^{p,q}, \\
    \left(\Tot^{\Pi}\right)^n(A) &= \prod_{p+q=n} A^{p,q}.
    \end{align*}
    are finite direct sums and finite products respectively and hence are naturally isomorphic (\Cref{lemma:finite_products_and_finite_coproducts_coincide_in_preadditive_categories}). The differential maps of the two total complexes also naturally coincide.
\end{proof}

\begin{lemma}[cf. {\cite[Acyclic Assembly Lemma 2.7.3]{weibel}}] \label{lemma:acyclic_assembly_lemma_for_bounded_double_complexes_with_exact_rows_or_columns}
    \TODO{It may be the case that this is generalizable beyond first quadrant double complexes, but I don't have a slick way to show this. See the commented out code for the statements; also, it may be necessary to assume something like AB4$*$ for such statements}
    Let $\calA$ be an \CrefAndHyperrefIfExist{definition:abelian_category}{abelian category} for which (small) \CrefAndHyperrefIfExist{definition:projective_and_inductive_limits_in_categories}{filtered colimits} which exist are exact (e.g. which holds if $\calA$ satisfies \CrefAndHyperrefIfExist{definition:grothendiecks_additional_axioms_for_abelian_categories}{Ab5}). Let $C$ be a \CrefAndHyperrefIfExist{definition:double_complex_of_objects_in_an_additive_category}{double complex} in $\calA$.

    % Let $C$ be a \CrefAndHyperrefIfExist{definition:double_complex_of_objects_in_an_additive_category}{double complex} in an \CrefAndHyperrefIfExist{definition:abelian_category}{abelian category} $\calA$ for which (small) filtered colimits which exist are exact (e.g. which holds if $\calC$ satisfies \CrefAndHyperrefIfExist{definition:grothendiecks_additional_axioms_for_abelian_categories}{Ab5}). 

    If $C$ has exact columns or has exact rows and $C$ is a \CrefAndHyperrefIfExist{definition:bounded_double_complex_of_objects_in_an_additive_category}{bounded below or bounded above double complex}, then $\Tot^{\Pi}(C)$ is an \CrefAndHyperrefIfExist{definition:acyclic_complex_of_objects_in_an_abelian_category}{acyclic chain complex}.



    % \begin{enumerate}
    %     \item 
    %     %Assuming that $\calA$ has arbitrary (small) products (in particular, the \CrefAndHyperrefIfExist{definition:total_complexes_of_a_double_complex_of_objects_in_an_additive_category}{total complex $\Tot^{\Pi}(C)$} exists), and products are exact (e.g. by virtue of $\calA$ satisfying \CrefAndHyperrefIfExist{definition:grothendiecks_additional_axioms_for_abelian_categories}{Ab$4^*$}),
    %     \begin{enumerate}
    %         \item if $C$ has exact columns or has exact rows and $C$ is a \CrefAndHyperrefIfExist{definition:bounded_double_complex_of_objects_in_an_additive_category}{bounded below or bounded above double complex}, then $\Tot^{\Pi}(C)$ is an \CrefAndHyperrefIfExist{definition:acyclic_complex_of_objects_in_an_abelian_category}{acyclic chain complex}.
    % \TODO{Fix the statements below to correctly specify what kind of boundedness of the double complex is needed. }
    %         \item if $C$ has exact rows and $C$ is a \CrefAndHyperrefIfExist{definition:bounded_double_complex_of_objects_in_an_additive_category}{bounded double complex}, then $\Tot^{\Pi}(C)$ is an \CrefAndHyperrefIfExist{definition:acyclic_complex_of_objects_in_an_abelian_category}{acyclic chain complex}.
    %     \end{enumerate}

    %     \item 
    %     %Assuming that $\calA$ has arbitrary (small) coproducts (in particular, the \CrefAndHyperrefIfExist{definition:total_complexes_of_a_double_complex_of_objects_in_an_additive_category}{total complex $\Tot^{\oplus}(C)$} exists), and (not necessarily finite) coproducts are exact (e.g. by virtue of $\calA$ satisfying \CrefAndHyperrefIfExist{definition:grothendiecks_additional_axioms_for_abelian_categories}{Ab4})
    %     \begin{enumerate}
    %         \item if $C$ has exact rows and $C$ is a \CrefAndHyperrefIfExist{definition:bounded_double_complex_of_objects_in_an_additive_category}{bounded double complex}, then $\Tot^{\oplus}(C)$ is an \CrefAndHyperrefIfExist{definition:acyclic_complex_of_objects_in_an_abelian_category}{acyclic chain complex}.
    %         \item if $C$ has exact columns and $C$ is a \CrefAndHyperrefIfExist{definition:bounded_double_complex_of_objects_in_an_additive_category}{bounded double complex}, then $\Tot^{\oplus}(C)$ is an \CrefAndHyperrefIfExist{definition:acyclic_complex_of_objects_in_an_abelian_category}{acyclic chain complex}.
    %     \end{enumerate}
    % \end{enumerate}

    % \begin{enumerate}
    %     \item Assuming that $\calA$ has arbitrary (small) products (in particular, the \CrefAndHyperrefIfExist{definition:total_complexes_of_a_double_complex_of_objects_in_an_additive_category}{total complex $\Tot^{\Pi}(C)$} exists), and products are exact,
    %     \begin{enumerate}
    %         \item if $C$ has exact columns and every diagonal of $C$ is bounded on the lower right (e.g. $C$ is an upper half-plane complex), then $\Tot^{\Pi}(C)$ is an \CrefAndHyperrefIfExist{definition:acyclic_complex_of_objects_in_an_abelian_category}{acyclic chain complex}.
    %         \item if $C$ has exact rows and every diagonal of $C$ is bounded on the upper left (e.g. $C$ is an right half-plane complex), then $\Tot^{\Pi}(C)$ is an \CrefAndHyperrefIfExist{definition:acyclic_complex_of_objects_in_an_abelian_category}{acyclic chain complex}.
    %     \end{enumerate}

    %     \item Assuming that the \CrefAndHyperrefIfExist{definition:total_complexes_of_a_double_complex_of_objects_in_an_additive_category}{total complex $\Tot^{\oplus}(C)$} exists (which holds e.g. when $\calA$ has all finite direct sums), 
    %     \begin{enumerate}
    %         \item if $C$ has exact rows and every diagonal of $C$ is bounded on the lower right (e.g. $C$ is an upper half-plane complex), then $\Tot^{\oplus}(C)$ is an \CrefAndHyperrefIfExist{definition:acyclic_complex_of_objects_in_an_abelian_category}{acyclic chain complex}.
    %         \item if $C$ has exact columns and every diagonal of $C$ is bounded on the upper left (e.g. $C$ is an right half-plane complex), then $\Tot^{\oplus}(C)$ is an \CrefAndHyperrefIfExist{definition:acyclic_complex_of_objects_in_an_abelian_category}{acyclic chain complex}.
    %     \end{enumerate}
    % \end{enumerate}
    
\end{lemma}

\begin{proof}

    % Assume that $\calA$ has arbitrary (small) products, and that products are exact in $\calA$.
    We show that if $C$ has exact columns and 
    $C$ is bounded below, 
    % every diagonal of $C$ is bounded on the lower right,
    then $\Tot^{\Pi}(C)$ is an \CrefAndHyperrefIfExist{definition:acyclic_complex_of_objects_in_an_abelian_category}{acyclic chain complex}; it can then be argued symmetrically that if $C$ has exact columns and $C$ is bounded above, then $\Tot^{\Pi}(C)$ is acyclic. Moreover, the case of exact rows can be deduced by reflecting the rows and columns of double complexes. 

    Note that since $C$ is assumed to be bounded below and hence is \CrefAndHyperrefIfExist{definition:bounded_double_complex_of_objects_in_an_additive_category}{locally bounded along diagonals}, $\Tot^{\Pi}(C)$ and $\Tot^{\oplus}(C)$ exist, are constructed by finite products (which are also finite coproducts), and are naturally isomorphic by Lemma \ref{lemma:total_complexes_of_double_complex_with_bounded_diagonals_are_canonically_isomorphic}. Further recall that finite coproducts in an abelian category are exact. 

    Define the sub-double complexes $F^k C$ of $C$ by 
    \begin{align*}
    (F^k C)^{p,q} = \begin{cases} C^{p,q} &\text{if } p \leq k \\ 0 &\text{otherwise} \end{cases}.
    \end{align*}
    This yields a filtration 
    $$\cdots \subseteq F^{k-1} C \subseteq F^k C \subseteq F^{k+1} \subseteq \cdots \subseteq C.$$
    Moreover, for each $n$,
    $$\left( \Tot^\Pi(F^k C) \right)^n = \prod_{p \leq k, p+q = n} C^{p,q}.$$
    For each $n$, the above stabilizes as $k \to \infty$ to $\Tot^{\Pi}(C)^n$. Now let 
    $$D^k = F^k C / F^{k-1} C = \begin{cases} C^{p,q} &\text{if } p = k \\ 0 &\text{otherwise}.\end{cases}$$
    Since each column $C^{k,*}$ is exact by assumption, the total complex $\Tot^{\Pi}(D^k)$ is acyclic. Note that we have short exact sequences
    $$0 \to F^{k-1} C \to F^k C \to D^k \to 0$$
    of double complexes. The totalization functor $\Tot^{\Pi}(-)$ in this case is exact because all of the double complexes are \CrefAndHyperrefIfExist{definition:bounded_double_complex_of_objects_in_an_additive_category}{locally bounded along diagonals}.
    % $\Tot^{\Pi}(-): \mathrm{Ch}(\mathrm{Ch}(\calA)) \to \mathrm{Ch}(\calA)$ (cf. \Cref{theorem:category_of_double_complexes_of_objects_of_an_additive_category_is_naturally_isomorphic_to_the_category_of_chain_complexes_of_chain_complexes}) is exact 
    We hence have a short exact sequence
    $$0 \to \Tot^{\Pi}(F^{k-1} C) \to \Tot^{\Pi}(F^k C) \to \Tot^{\Pi}(D^k) \to 0.$$
    Since $\Tot^{\Pi}(D^k)$ is \CrefAndHyperrefIfExist{definition:acyclic_complex_of_objects_in_an_abelian_category}{acyclic}, the \CrefAndHyperrefIfExist{theorem:long_exact_sequence_of_left_right_derived_functors_and_homological_delta_functors}{long exact cohomology sequences} yield isomorphisms
    $$H^n(\Tot^{\Pi}(F^{k-1} C)) \cong H^n(\Tot^{\Pi}(F^{k} C)).$$
    Since $C$ is assumed to be bounded, the subcomplex $F^k C$ is zero for sufficiently negative $k$, in which case $\Tot^{\Pi}(F^k C)$ is acyclic. By induction on $k$, $\Tot^{\Pi}(F^k C)$ remains acyclic for all $k$. Moroever, the filtered colimit $\varinjlim_k \Tot^{\Pi}(F^k C)$ is $\Tot^{\Pi}(C)$. The assumed exactness of filtered colimits in $\calA$ concludes that $\Tot^{\Pi}(C)$ is acyclic.

    By symmetry, if $C$ instead has exact rows, then $\Tot^{\Pi}(C)$ is an acyclic chain complex.
\end{proof}



% \subsubsection{Tensor product of chain complexes}


\subsection{Derived functors of biadditive functors}


Given a biadditive functor $F: \calA \times \calB \to \calC$ of abelian categories, we consider ``derived'' functors (in a broad sense) obtained by fixing an object $A$ or $B$ either of $\calA$ or $\calB$ respectively and applying $F(A,-)$ or $F(-,B)$ to a suitable resolution. Such derived functors are not a priori independent of the choice of resolution, so we state sufficient conditions for such derived functors to be well defined.

\subsubsection{Flat objects with respect to biadditive functors}

\begin{definition}[Flat object with respect to a tensor (monoidal) product] \label{definition:flat_object_in_an_abelian_category_with_respect_to_a_right_exact_monoidal_product_functor}
Let $\mathcal{A}, \calB, \calC$ be \CrefAndHyperrefIfExist{definition:abelian_category}{abelian categories}, and let $F : \mathcal{A} \times \mathcal{B} \to \mathcal{C}$ be a \CrefAndHyperrefIfExist{definition:n_ary_additive_functor_between_additive_categories}{biadditive functor} (in practice, the following definitions are usually considered when $F$ is some kind of ``tensor product'' $\otimes$ and is \CrefAndHyperrefIfExist{definition:exact_functor_between_abelian_categories}{right exact} in each variable). 
% that is often, but not necessarily, \CrefAndHyperrefIfExist{definition:exact_functor_between_abelian_categories}{right exact} in each variable. 
\begin{enumerate}
    \item An object $X \in \mathcal{A}$ is called \hldef{flat (with respect to $F$ on the left)} if the functor $F(X, -): \calB \to \calC$ if \CrefAndHyperrefIfExist{definition:exact_functor_between_abelian_categories}{exact}.
    \item An object $Y \in \mathcal{B}$ is called \hldef{flat (with respect to $F$ on the right)} if the functor $F(-, Y): \calA \to \calC$ if \CrefAndHyperrefIfExist{definition:exact_functor_between_abelian_categories}{exact}.
\end{enumerate}
If $\calA = \calB = \calC$ and $F$ makes $\calA$ into a \CrefAndHyperrefIfExist{definition:symmetric_monoidal_category}{symmetric monoidal category}, then certainly $F(X,-)$ is exact if and only if $F(-,Y)$ is exact, i.e. flatness is equivalent on the two sides of $\otimes$.
\end{definition}
\begin{definition}[Having enough flat objects in an abelian category] \label{definition:has_enough_flat_objects_for_an_abelian_category_with_respect_to_a_right_exact_bifunctor}
Let $\mathcal{A}, \calB, \calC$ be \CrefAndHyperrefIfExist{definition:abelian_category}{abelian categories}, and let $\otimes : \mathcal{A} \times \mathcal{B} \to \mathcal{C}$ be a \CrefAndHyperrefIfExist{definition:n_ary_additive_functor_between_additive_categories}{biadditive functor}
% that is \CrefAndHyperrefIfExist{definition:exact_functor_between_abelian_categories}{right exact} in each variable..

\begin{enumerate}
    \item We say that \hldef{$\mathcal{A}$ has enough flat objects with respect to $\otimes$} if for every object $A \in \mathcal{A}$ there exists an epimorphism
    $$
    F \twoheadrightarrow A
    $$
    in $\mathcal{A}$ where $F$ is \CrefAndHyperrefIfExist{definition:flat_object_in_an_abelian_category_with_respect_to_a_right_exact_monoidal_product_functor}{flat with respect to $\otimes$}, i.e. $F \otimes -: \calB \to \calC$ is exact. \TextIfExists{definition:has_enough_objects_of_a_class_on_the_left_right_for_an_abelian_category}{Equivalently, $\calA$ has enough flat objects with respect to $\otimes$ if it \CrefAndHyperrefIfExist{definition:has_enough_objects_of_a_class_on_the_left_right_for_an_abelian_category}{has enough objects of the class of objects of $\calA$ with are flat with respect to $\otimes$} on the left.} 

    \item Similarly, we say that \hldef{$\mathcal{B}$ has enough flat objects with respect to $\otimes$} if for every $B \in \mathcal{B}$ there exists an epimorphism
    $$
    G \twoheadrightarrow B
    $$
    in $\calB$ with $G$ flat with respect to $\otimes$, i.e. $- \otimes G: \calA \to \calC$ is exact. \TextIfExists{definition:has_enough_objects_of_a_class_on_the_left_right_for_an_abelian_category}{Equivalently, $\calB$ has enough flat objects with respect to $\otimes$ if it \CrefAndHyperrefIfExist{definition:has_enough_objects_of_a_class_on_the_left_right_for_an_abelian_category}{has enough objects of the class of objects of $\calB$ with are flat with respect to $\otimes$} on the left.} 

\end{enumerate}
% Having enough flat objects ensures the ability to compute the left derived functors of $\otimes$ via flat resolutions, generalizing the classical notion of having enough projectives.
\end{definition}
\begin{definition} \label{definition:flat_resolution_of_an_object_in_an_abelian_category_with_respect_to_a_right_exact_monoidal_product_functor}
Let $\mathcal{A}, \calB, \calC$ be \CrefAndHyperrefIfExist{definition:abelian_category}{abelian categories}, and let $\otimes : \mathcal{A} \times \mathcal{B} \to \mathcal{C}$ be a bifunctor that is \CrefAndHyperrefIfExist{definition:additive_functor_between_additive_categories}{additive} and \CrefAndHyperrefIfExist{definition:exact_functor_between_abelian_categories}{right exact} in each variable. 
\begin{enumerate}
    \item A \hldef{flat (left-)resolution of an object $F \in \mathcal{A}$} is a \CrefAndHyperrefIfExist{definition:left_right_resolution_of_a_class_of_objects_in_an_abelian_category}{left resolution of $F$} consisting of \CrefAndHyperrefIfExist{definition:flat_object_in_an_abelian_category_with_respect_to_a_right_exact_monoidal_product_functor}{flat objects} in $\calA$ with respect to $\otimes$.

    \item A \hldef{flat (left-)resolution of an object $G \in \mathcal{B}$} is a \CrefAndHyperrefIfExist{definition:left_right_resolution_of_a_class_of_objects_in_an_abelian_category}{left resolution of $G$} consisting of \CrefAndHyperrefIfExist{definition:flat_object_in_an_abelian_category_with_respect_to_a_right_exact_monoidal_product_functor}{flat objects} in $\calB$ with respect to $\otimes$.
\end{enumerate}
Note that we are reserving these default term ``flat resolution'' for ``left resolutions with flat objects''. We may still use \hldef{flat right-resolution} to be a \CrefAndHyperrefIfExist{definition:left_right_resolution_of_a_class_of_objects_in_an_abelian_category}{right resolution} consisting of \CrefAndHyperrefIfExist{definition:flat_object_in_an_abelian_category_with_respect_to_a_right_exact_monoidal_product_functor}{flat objects}.
\end{definition}
% \begin{lemma}[cf. {\cite[Flat Resolution Lemma 3.2.8]{weibel}}] \label{lemma:flat_resolution_lemma_of_tor_objects_of_abelian_categories_with_a_right_exact_bifunctor_assuming_that_category_has_enough_projectives_or_flats}
Let $\mathcal{A}, \calB, \calC$ be \CrefAndHyperrefIfExist{definition:abelian_category}{abelian categories}, and let $\otimes : \mathcal{A} \times \mathcal{B} \to \mathcal{C}$ be a \CrefAndHyperrefIfExist{definition:n_ary_additive_functor_between_additive_categories}{biadditive functor}
that is \CrefAndHyperrefIfExist{definition:exact_functor_between_abelian_categories}{right exact} in each variable. 
Let $A \in \calA$ and $B \in \calB$ be objects.
\begin{enumerate}
    % \item Suppose that $A \in \calA$ is an object with a \CrefAndHyperrefIfExist{definition:left_right_resolution_of_a_class_of_objects_in_an_abelian_category}{projective resolution}. For any object $B \in \calB$, the objects
    % $$\mathrm{Tor}_n^{\mathcal{A}}(A, B) \in \calC$$
    % obtained via a \CrefAndHyperrefIfExist{definition:left_right_resolution_of_a_class_of_objects_in_an_abelian_category}{projective resolution of $A$} and via a \CrefAndHyperrefIfExist{definition:flat_resolution_of_an_object_in_an_abelian_category_with_respect_to_a_right_exact_monoidal_product_functor}{flat resolution of $A$} are naturally isomorphic. 

    \item Suppose that $\mathcal{A}$ \CrefAndHyperrefIfExist{definition:has_enough_injectives_or_projectives_for_an_abelian_category}{has enough projective objects}.
    % or \CrefAndHyperrefIfExist{definition:has_enough_flat_objects_for_an_abelian_category_with_respect_to_a_right_exact_bifunctor}{has enough flat objects}.
    Suppose that $A$ has some \CrefAndHyperrefIfExist{definition:flat_resolution_of_an_object_in_an_abelian_category_with_respect_to_a_right_exact_monoidal_product_functor}{flat resolution}. The \CrefAndHyperrefIfExist{definition:tor_functors_of_a_right_exact_monoidal_functor_on_an_abelian_category}{Tor objects} 
    $$\mathrm{Tor}_n(A, B) \in \calC$$
    obtained via a \CrefAndHyperrefIfExist{definition:left_right_resolution_of_a_class_of_objects_in_an_abelian_category}{projective resolution of $A$} and via a \CrefAndHyperrefIfExist{definition:flat_resolution_of_an_object_in_an_abelian_category_with_respect_to_a_right_exact_monoidal_product_functor}{flat resolution of $A$} are naturally isomorphic. In particular, either notion of $\Tor_n$ is well defined.


    % \item Suppose that $B \in \calB$ is an object with a \CrefAndHyperrefIfExist{definition:left_right_resolution_of_a_class_of_objects_in_an_abelian_category}{projective resolution}. For any object $A \in \calA$, the objects
    % $$\mathrm{Tor}_n^{\mathcal{A}}(A, B) \in \calC$$
    % obtained via a \CrefAndHyperrefIfExist{definition:left_right_resolution_of_a_class_of_objects_in_an_abelian_category}{projective resolution of $B$} and via a \CrefAndHyperrefIfExist{definition:flat_resolution_of_an_object_in_an_abelian_category_with_respect_to_a_right_exact_monoidal_product_functor}{flat resolution of $B$} are naturally isomorphic. 

    \item Suppose that $\mathcal{B}$ \CrefAndHyperrefIfExist{definition:has_enough_injectives_or_projectives_for_an_abelian_category}{has enough projective objects}.
    % or \CrefAndHyperrefIfExist{definition:has_enough_flat_objects_for_an_abelian_category_with_respect_to_a_right_exact_bifunctor}{has enough flat objects}.
    Suppose that $B$ has some \CrefAndHyperrefIfExist{definition:flat_resolution_of_an_object_in_an_abelian_category_with_respect_to_a_right_exact_monoidal_product_functor}{flat resolution}. The Tor objects 
    $$\mathrm{Tor}_n(A, B) \in \calC$$
    obtained via a \CrefAndHyperrefIfExist{definition:left_right_resolution_of_a_class_of_objects_in_an_abelian_category}{projective resolution of $B$} and via a \CrefAndHyperrefIfExist{definition:flat_resolution_of_an_object_in_an_abelian_category_with_respect_to_a_right_exact_monoidal_product_functor}{flat resolution of $B$} are naturally isomorphic. In particular, either notion of $\Tor_n$ is well defined.


\end{enumerate}
\end{lemma}

\begin{proof}

    We show $1$. Write $\Tor_n(A,B)$ for the tor object obtained via a projective resolution of $A$. Write $H_n(F_\bullet \otimes B)$ for the tor object obtained via a flat resolution $F_\bullet \to A$ of $A$. We argue by induction. For $n = 0$, we have $\Tor_0(A,B) \cong H_0(F_\bullet \otimes B)$ because $- \otimes B$ is right exact by assumption. 
   
    Let $K$ be such that $0 \to K \to F_0 \to A$ is exact. write $E_\bullet = (\cdots \to F_2 \to F_1)$ so that $E_\bullet \to K$ is a resolution of $K$ by flat objects (\Cref{lemma:an_object_of_abelian_category_with_enough_objects_of_a_class_on_the_right_left_has_right_left_resolution_by_the_class}). Since $\calA$ has enough projectives, further note that all flat objects are \CrefAndHyperrefIfExist{definition:F_acyclic_object_for_a_left_or_right_functor_between_abelian_categories}{acyclic} for the functor $- \otimes B$ by \Cref{lemma:flat_objects_are_acyclic_for_tensoring_by_any_object_if_enough_projectives_or_flats_exist}.
    
    By \Cref{lemma:dimension_shifting_lemma_for_acyclic_syzygies}, we have 
    $$\Tor_1(A,B) = \ker(K \otimes B \to F_0 \otimes B) = \ker \left\{ \frac{F_1 \otimes B}{\operatorname{im}(F_2 \otimes B) \to F_0 \otimes B} = H_1(F \otimes B)  \right\},$$
    thus establishing the desired isomorphism for $n = 1$. For $n \geq 2$, use induction to see that 
    $$\Tor_n(A,B) \cong \Tor_{n-1}(K, B) \cong H_{n-1}(E_\bullet \otimes B) = H_n(F \otimes B).$$

\end{proof}


\subsubsection{Derived functors for biadditive functors computed via resolutions of flat objects} \label{subsubsection:derived_functors_for_biadditive_functors_computed_via_resolutions_of_flat_objects}

\begin{definition}[Derived functors for biadditive functors of abelian categories via resolutions by flat objects] \label{definition:generalized_derived_functors_for_biadditive_functors_of_abelian_categories_via_resolutions_by_flat_objects}
    Let $F: \calA \times \calB \to \calC$ be a \CrefAndHyperrefIfExist{definition:n_ary_additive_functor_between_additive_categories}{biadditive functor} of \CrefAndHyperrefIfExist{definition:abelian_category}{abelian categories}.
    \begin{enumerate}
        \item Fix an object $A$ of $\calA$. Let $B$ be an object of $\calB$. 
        \begin{enumerate}
            \item Let \hl{$L_n^{II}F(A, B) = L_n^{II}(F(A, -))(B)$} be an object of $\calC$ obtained as follows: take a \CrefAndHyperrefIfExist{definition:left_right_resolution_of_a_class_of_objects_in_an_abelian_category}{left resolution} 
            $$P_\bullet \to B$$
            of $B$ of objects $P_i$ for which the functors $F(-, P_i) : \calA \to \calC$ are \CrefAndHyperrefIfExist{definition:exact_functor_between_abelian_categories}{exact}, i.e. $P_\bullet$ is a \CrefAndHyperrefIfExist{definition:flat_resolution_of_an_object_in_an_abelian_category_with_respect_to_a_right_exact_monoidal_product_functor}{flat resolution of $B$}. Let 
            $$L_n^{II}(F(A, -))(B) \coloneq H_n(F(A, P_\bullet))$$
            (\Cref{definition:homology_and_cohomology_objects_for_a_chain_complex_in_an_additive_category}).

            \item Let \hl{$R_{II}^{n}F(A, B) = R_{II}^{n}(F(A, -))(B)$} be an object of $\calC$ obtained as follows: take a \CrefAndHyperrefIfExist{definition:left_right_resolution_of_a_class_of_objects_in_an_abelian_category}{right resolution} 
            $$B \to I^\bullet$$
            of $B$ of objects $I^i$ for which the functors $F(-, I^i) : \calA \to \calC$ are \CrefAndHyperrefIfExist{definition:exact_functor_between_abelian_categories}{exact}, i.e. $I^\bullet$ is a \CrefAndHyperrefIfExist{definition:flat_resolution_of_an_object_in_an_abelian_category_with_respect_to_a_right_exact_monoidal_product_functor}{flat right-resolution of $B$}. Let 
            $$R_{II}^{n}(F(A, -))(B) \coloneq H^n(F(A, I^\bullet))$$
            (\Cref{definition:homology_and_cohomology_objects_for_a_chain_complex_in_an_additive_category}).
        \end{enumerate}

        \item Fix an object $B$ of $\calB$. Let $A$ be an object of $\calA$. 
        \begin{enumerate}
            \item Let \hl{$L_n^{I}F(A, B) = L_n^{I}(F(-, B))(A)$} be an object of $\calC$ obtained as follows: take a \CrefAndHyperrefIfExist{definition:left_right_resolution_of_a_class_of_objects_in_an_abelian_category}{left resolution} 
            $$P_\bullet \to A$$
            of $A$ of objects $P_i$ for which the functors $F(P_i, -) : \calB \to \calC$ are \CrefAndHyperrefIfExist{definition:exact_functor_between_abelian_categories}{exact}, i.e. $P_\bullet$ is a \CrefAndHyperrefIfExist{definition:flat_resolution_of_an_object_in_an_abelian_category_with_respect_to_a_right_exact_monoidal_product_functor}{flat resolution} of $A$. Let 
            $$L_n^{I}(F(-, B))(A) \coloneq H_n(F(P_\bullet, B))$$
            (\Cref{definition:homology_and_cohomology_objects_for_a_chain_complex_in_an_additive_category}).

            \item Let \hl{$R_{I}^{n}F(A, B) = R_{I}^{n}(F(-, B))(A)$} be an object of $\calC$ obtained as follows: take a \CrefAndHyperrefIfExist{definition:left_right_resolution_of_a_class_of_objects_in_an_abelian_category}{right resolution} 
            $$A \to I^\bullet$$
            of $A$ of objects $I^i$ for which the functors $F(I^i, -) : \calB \to \calC$ are \CrefAndHyperrefIfExist{definition:exact_functor_between_abelian_categories}{exact}, i.e. $I^\bullet$ is a \CrefAndHyperrefIfExist{definition:flat_resolution_of_an_object_in_an_abelian_category_with_respect_to_a_right_exact_monoidal_product_functor}{flat right-resolution of $A$}. Let 
            $$R_{I}^{n}(F(-, B))(A) \coloneq H^n(F(I^\bullet, B))$$
            (\Cref{definition:homology_and_cohomology_objects_for_a_chain_complex_in_an_additive_category}).
        \end{enumerate}
    \end{enumerate}
    A priori, the objects $L_n^{II} F(A,B)$, $R_{II}^nF(A,B)$, $L_n^{I} F(A,B)$, and $R_{I}^nF(A,B)$ are not well defined, i.e. they may possibly depend on the choice of resolution. See \Cref{theorem:balancing_generalized_derived_functors_of_a_biadditive_functor_of_abelian_categories_computed_via_flat_resolutions}, which asserts that the pair $L_n^{II} F(A,B)$ and $L_n^{I} F(A,B)$ (resp. $R_{II}^nF(A,B)$ and $R_{I}^nF(A,B)$) are in agreement and are well defined under mild conditions.

    In case that $F$ is a functor thought of as some kind of ``tensor product'' and denoted by $\otimes$ (and usually, but not necessarily, such that $\otimes$ is right exact in each variable), it is customary to denote $L_n^{II} F(A,B)$ and $L_n^{I} F(A,B)$ by \hl{$\Tor_n^{II}(A,B)$} and \hl{$\Tor_n^{I}(A,B)$} respectively and to call these objects \hldef{Tor objects}. In other words, Tor objects are obtained by taking \CrefAndHyperrefIfExist{definition:flat_object_in_an_abelian_category_with_respect_to_a_right_exact_monoidal_product_functor}{flat} resolutions with respect to $\otimes$

    In case that $F$ is a functor thought of as some kind of ``Hom'' and denoted as (some variation of) $\Hom: \calA^{\op} \times \calB \to \calC$ (most usually, $\calA = \calB$ for such Hom functors), it is customary to denote $R_n^{II} F(A,B)$ and $R_n^{I} F(A,B)$ by \hl{$\mathrm{Ext}_n^{II}(A,B)$} and \hl{$\mathrm{Ext}_n^{I}(A,B)$} respectivley and to call these objects \hldef{Ext objects}. 
\end{definition}
\begin{lemma} \label{lemma:applying_biadd_func_to_res_is_defined_up_to_quasi_iso_and_agrees_with_the_tot_complex_of_the_double_complex_of_the_biadd_func_on_the_two_res}
    Let $F: \calA \times \calB \to \calC$ be a \CrefAndHyperrefIfExist{definition:n_ary_additive_functor_between_additive_categories}{biadditive functor} of \CrefAndHyperrefIfExist{definition:abelian_category}{abelian categories}. Assume that (small) filtered colimits which exist in $\calC$ are exact (e.g. which holds if $\calC$ satisfies \CrefAndHyperrefIfExist{definition:grothendiecks_additional_axioms_for_abelian_categories}{Ab5}).

    Let $A \in \calA$ and $B \in \calB$ be objects. 
    \begin{enumerate}
        \item Suppose that \CrefAndHyperrefIfExist{definition:left_right_resolution_of_a_class_of_objects_in_an_abelian_category}{left resolutions} $P_{A,\bullet} \to A$ and $P_{B,\bullet} \to B$ exist such that $P_{A,i}$ and $P_{B,i}$ are \CrefAndHyperrefIfExist{definition:flat_object_in_an_abelian_category_with_respect_to_a_right_exact_monoidal_product_functor}{flat} with respect to $F$ on the left and right respectively, i.e. $F(P_{A,i}, -): \calB \to \calC$ and $F(-, P_{B,i}): \calA \to \calC$ are exact for all $i$. 

        The complexes $F(P_{A,\bullet}, B)$ and $F(A, P_{B,\bullet})$ are \CrefAndHyperrefIfExist{definition:quasi_isomorphism_of_chain_complexes_of_objects_in_an_abelian_category}{quasi-isomorphic} to the complex $\Tot(F(P_{A,\bullet}, P_{B,\bullet}))$\CrefIfExists{definition:total_complexes_of_a_double_complex_of_objects_in_an_additive_category}\CrefIfExists{definition:double_complex_associated_to_biadditive_functor_and_chain_complexes}.

        \item Suppose that \CrefAndHyperrefIfExist{definition:left_right_resolution_of_a_class_of_objects_in_an_abelian_category}{right resolutions} $A \to I^{A,\bullet}$ and $B \to I^{B,\bullet}$ exist such that $I^{A,i}$ and $I^{B,i}$ are \CrefAndHyperrefIfExist{definition:flat_object_in_an_abelian_category_with_respect_to_a_right_exact_monoidal_product_functor}{flat} with respect to $F$ on the left and right respectively, $F(I^{A,i}, -): \calB \to \calC$ and $F(-, I^{B,i}): \calA \to \calC$ are exact for all $i$. 

        The complexes $F(I^{A,\bullet}, B)$ and $F(A, I^{B,\bullet})$ are \CrefAndHyperrefIfExist{definition:quasi_isomorphism_of_chain_complexes_of_objects_in_an_abelian_category}{quasi-isomorphic} to the complex $\Tot(F(I^{A,\bullet}, I^{B,\bullet}))$\CrefIfExists{definition:total_complexes_of_a_double_complex_of_objects_in_an_additive_category}\CrefIfExists{definition:double_complex_associated_to_biadditive_functor_and_chain_complexes}.
    \end{enumerate}
\end{lemma}

\begin{proof}
    We prove 1. The other part is the dual statement.

    Choose resolutions $P_{A,\bullet} \xrightarrow{\varepsilon} A$ and $P_{B,\bullet} \xrightarrow{\eta} B$ such that $F(P_{A,i}, -): \calB \to \calC$ and $F(-, P_{B,i}): \calA \to \calC$ are exact for all $i$. 
    Identifying $A$ and $B$ with complexes concentrated in degree $0$, we can \CrefAndHyperrefIfExist{definition:double_complex_associated_to_biadditive_functor_and_chain_complexes}{form} the three \CrefAndHyperrefIfExist{definition:double_complex_of_objects_in_an_additive_category}{double complexes} $F(P_{A,\bullet}, P_{B,\bullet})$, $F(A, P_{B,\bullet})$, and $F(P_{A,\bullet} , B)$. Note that the augmentation morphisms $\varepsilon$ and $\eta$ induce morphisms $P_{A,\bullet} \otimes P_{B,\bullet} \to A \otimes P_{B,\bullet}, P_{A,\bullet} \otimes B$.

    Let $C$ be the double complex of objects in $\calC$ obtained from $F(P_{A,\bullet}, P_{B,\bullet})$ by adding $F(A, P_{B,\bullet}[-1])$ in the column $p = -1$. One can show that the translate $\Tot(C)[1]$ is the \CrefAndHyperrefIfExist{definition:mapping_cone_of_a_map_of_chain_cochain_complexes}{mapping cone} of the map 
    $$\Tot(F(P_{A,\bullet}, P_{B,\bullet})) \xrightarrow{\varepsilon \otimes \id} \Tot(F(A, P_{B,\bullet})) = F(A, P_{B,\bullet}).$$
    Moreover, since each $F(-, P_{B,i})$ is an exact functor, every row of $C$ is exact, so $\Tot(C)$ is exact by \Cref{lemma:acyclic_assembly_lemma_for_bounded_double_complexes_with_exact_rows_or_columns}. Therefore, $F(\varepsilon, \id)$ is a quasi-isomorphism and hence 
    $$H_*(\Tot(F(P_{A,\bullet}, P_{B,\bullet}))) \xrightarrow{H_*(F(\varepsilon, P_{B,\bullet}))} H_*(F(A, P_{B,\bullet}))$$
    is a natural isomorphism. 

    By symmetry, there is a natural isomorphism $H_*(\Tot(F(P_{A,\bullet} P_{B,\bullet}))) \xrightarrow{} H_*(F(P_{A,\bullet}, B))$.
\end{proof}

\begin{theorem}[Balancing generalized derived functors of a biadditive functor of abelian categories computed via flat resolutions] \label{theorem:balancing_generalized_derived_functors_of_a_biadditive_functor_of_abelian_categories_computed_via_flat_resolutions}
    Let $F: \calA \times \calB \to \calC$ be a \CrefAndHyperrefIfExist{definition:n_ary_additive_functor_between_additive_categories}{biadditive functor} of \CrefAndHyperrefIfExist{definition:abelian_category}{abelian categories}. Assume that (small) filtered colimits which exist in $\calC$ are exact (e.g. which holds if $\calC$ satisfies \CrefAndHyperrefIfExist{definition:grothendiecks_additional_axioms_for_abelian_categories}{Ab5}).

    Let $A \in \calA$ and $B \in \calB$ be objects. 
    \begin{enumerate}
        \item Suppose that \CrefAndHyperrefIfExist{definition:left_right_resolution_of_a_class_of_objects_in_an_abelian_category}{left resolutions} $P_{A,\bullet} \to A$ and $P_{B,\bullet} \to B$ exist such that $F(P_{A,i}, -): \calB \to \calC$ and $F(-, P_{B,i}): \calA \to \calC$ are exact for all $i$, i.e. $P_{A,\bullet}$ and $P_{B,\bullet}$ are \CrefAndHyperrefIfExist{definition:flat_resolution_of_an_object_in_an_abelian_category_with_respect_to_a_right_exact_monoidal_product_functor}{flat resolutions} of $A$ and $B$ respectively. 
        \begin{enumerate}
            \item The objects \CrefAndHyperrefIfExist{definition:generalized_derived_functors_for_biadditive_functors_of_abelian_categories_via_resolutions_by_flat_objects}{$L_n^{I}F(A,B)$} and $L_n^{II}F(A,B)$ are naturally isomorphic.
            \item The objects $L_n^{I}F(A,B)$ and $L_n^{II}F(A,B)$ are well defined (up to natural isomorphism), i.e. do not depend on the choice of left resolutions of $A$ and $B$ respectively.
        \end{enumerate}

        \item Suppose that \CrefAndHyperrefIfExist{definition:left_right_resolution_of_a_class_of_objects_in_an_abelian_category}{right resolutions} $A \to I^{A,\bullet}$ and $B \to I^{B,\bullet}$ exist such that $F(I^{A,i}, -): \calB \to \calC$ and $F(-, I^{B,i}): \calA \to \calC$ are exact for all $i$. 
        \begin{enumerate}
            \item The objects \CrefAndHyperrefIfExist{definition:generalized_derived_functors_for_biadditive_functors_of_abelian_categories_via_resolutions_by_flat_objects}{$R^n_{I}F(A,B)$} and $R^n_{II}F(A,B)$ are naturally isomorphic.
            \item The objects $R^n_{I}F(A,B)$ and $R^n_{II}F(A,B)$ are well defined (up to natural isomorphism), i.e. do not depend on the choice of left resolutions of $A$ and $B$ respectively.
        \end{enumerate}
    \end{enumerate}
\end{theorem}

\begin{proof}

    We prove 1. The other part is the dual statement.

    Choose resolutions $P_{A,\bullet} \xrightarrow{\varepsilon} A$ and $P_{B,\bullet} \xrightarrow{\eta} B$ such that $F(P_{A,i}, -): \calB \to \calC$ and $F(-, P_{B,i}): \calA \to \calC$ are exact for all $i$. As per \Cref{lemma:applying_biadd_func_to_res_is_defined_up_to_quasi_iso_and_agrees_with_the_tot_complex_of_the_double_complex_of_the_biadd_func_on_the_two_res}, $F(\varepsilon, \id)$ is a quasi-isomorphism and hence $H_*(\Tot(F(P_{A,\bullet}, P_{B,\bullet}))) \xrightarrow{H_*(F(\varepsilon, P_{B,\bullet}))} H_*(F(A, P_{B,\bullet}))$ is a natural isomorphism.  By symmetry, there is a natural isomorphism $H_*(\Tot(F(P_{A,\bullet} P_{B,\bullet}))) \xrightarrow{} H_*(F(P_{A,\bullet}, B))$. Therefore, $L_n^I(A,B)$ and $L_n^{II}(A,B)$ are naturally isomorphic as claimed. In particular, $L_n^{I}(A,B)$ and $L_n^{II}(A,B)$ are independent of the choice of resolution of $A$ and $B$ respectively.

\end{proof}

\subsection{Homological and Cohomological amplitude}

\begin{definition}[Homological and cohomological amplitude of a functor]
    \TODO{}
Let $\mathcal{A}$ and $\mathcal{B}$ be abelian categories, and let $F \colon D(\mathcal{A}) \to D(\mathcal{B})$ be a triangulated functor between their derived categories. Suppose $F$ is \hldef{bounded} on cohomological degrees of objects in $D(\mathcal{A})$ in the sense explained below.

\medskip

We say that $F$ has \hldef{homological amplitude contained in the interval} $[a,b] \subset \mathbb{Z}$ if for every complex $X^\bullet \in D(\mathcal{A})$ whose cohomology vanishes outside degrees $m \leq n$, the complex $F(X^\bullet)$ has vanishing cohomology outside degrees $m+a$ through $n+b$. Equivalently, $F$ shifts the cohomology of any complex by at least $a$ and at most $b$ degrees.

\medskip

Dually, $F$ has \hldef{cohomological amplitude contained in the interval} $[c,d] \subset \mathbb{Z}$ if its right or left derived functor (whichever is defined in context) satisfies the analogous bounds on cohomological degrees.

\medskip

More explicitly, for integers $a \leq b$, $F$ has homological amplitude in $[a,b]$ if and only if for all $X^\bullet$ in $D(\mathcal{A})$,
$$
F(X^\bullet) \in D^{[m+a, n+b]}(\mathcal{B}) \quad \text{whenever} \quad X^\bullet \in D^{[m,n]}(\mathcal{A}),
$$
where $D^{[m,n]}(\mathcal{A})$ denotes complexes with cohomology supported between degrees $m$ and $n$.
\end{definition}


\TODO{try defining flat dimensions via derived tensor products}
\begin{definition} \label{definition:flat_dimension_of_objects_of_abelian_categories_with_}
    Let $\mathcal{A}, \calB, \calC$ be \CrefAndHyperrefIfExist{definition:abelian_category}{abelian categories}, and let $\otimes : \mathcal{A} \times \mathcal{B} \to \mathcal{C}$ be a \CrefAndHyperrefIfExist{definition:n_ary_additive_functor_between_additive_categories}{biadditive functor} that is \CrefAndHyperrefIfExist{definition:exact_functor_between_abelian_categories}{right exact} in each variable. 
    \begin{enumerate}
        \item 
        Suppose that $\calB$ \CrefAndHyperrefIfExist{definition:has_enough_injectives_or_projectives_for_an_abelian_category}{has enough projectives} or \CrefAndHyperrefIfExist{definition:has_enough_flat_objects_for_an_abelian_category_with_respect_to_a_right_exact_bifunctor}{has enough flats}.

        Let $M$ be an object of $\calA$. The \hldef{Tor dimension of $M$}, denoted by \hl{$\mathrm{tdim}(M)$} or \hl{$\mathrm{td}(M)$}, is the smallest integer $n \geq 0$ such that 
        $$\mathrm{Tor}_i(M,-) = 0$$
        for all $i > n$, where \CrefAndHyperrefIfExist{definition:tor_functors_of_a_right_exact_monoidal_functor_on_an_abelian_category}{$\mathrm{Tor}_i(M,-): \calB \to \calC$} is computed via projective or flat resolutions of objects of $\calB$; by \Cref{lemma:flat_resolution_lemma_of_tor_objects_of_abelian_categories_with_a_right_exact_bifunctor_assuming_that_category_has_enough_projectives_or_flats}, it does not matter whether a projective resolution or a flat resolution is used.

        \item 
        Symmetrically, suppose that $\calA$ \CrefAndHyperrefIfExist{definition:has_enough_injectives_or_projectives_for_an_abelian_category}{has enough projectives} or \CrefAndHyperrefIfExist{definition:has_enough_flat_objects_for_an_abelian_category_with_respect_to_a_right_exact_bifunctor}{has enough flats}.

        Let $N$ be an object of $\calB$. The \hldef{Tor dimension of $N$}, denoted by \hl{$\mathrm{tdim}(N)$} or \hl{$\mathrm{td}(N)$}, is the smallest integer $n \geq 0$ such that 
        $$\mathrm{Tor}_i(-,N) = 0$$
        for all $i > n$, where \CrefAndHyperrefIfExist{definition:tor_functors_of_a_right_exact_monoidal_functor_on_an_abelian_category}{$\mathrm{Tor}_i(-,N): \calA \to \calC$} is computed via projective or flat resolutions of objects of $\calA$; by \Cref{lemma:flat_resolution_lemma_of_tor_objects_of_abelian_categories_with_a_right_exact_bifunctor_assuming_that_category_has_enough_projectives_or_flats}, it does not matter whether a projective resolution or a flat resolution is used.

    \end{enumerate}
\end{definition}


\subsection{Tor/derived functor of a (right exact) biadditive functor via projective resolutions or flat resolutions}

Recall that the \CrefAndHyperref{definition:generalized_derived_functors_for_biadditive_functors_of_abelian_categories_via_resolutions_by_flat_objects}{derived functor} of a \CrefAndHyperref{definition:n_ary_additive_functor_between_additive_categories}{bi-additive functor} may be computed via \CrefAndHyperref{definition:flat_resolution_of_an_object_in_an_abelian_category_with_respect_to_a_right_exact_monoidal_product_functor}{flat resolutions}; this seems to be a departure from the notion of a \CrefAndHyperref{definition:left_right_derived_functors_of_a_right_left_exact_functor_between_abelian_categories_where_source_has_enough_projectives_injectives}{left derived functor} of a single-variate \CrefAndHyperrefIfExist{definition:exact_functor_between_abelian_categories}{right exact} additive functor, which a priori uses \CrefAndHyperrefIfExist{definition:injective_and_projective_objects_in_a_category}{projective} \CrefAndHyperrefIfExist{definition:left_right_resolution_of_a_class_of_objects_in_an_abelian_category}{resolutions}. The motivation for this is that certain common abelian categories, such as those of \CrefAndHyperref{definition:sheaf_on_a_site}{sheaves} of abelian groups or \CrefAndHyperref{definition:module_over_a_sheaf_of_rings_on_a_site}{modules}, may \CrefAndHyperrefIfExist{definition:has_enough_flat_objects_for_an_abelian_category_with_respect_to_a_right_exact_bifunctor}{have enough flats} (with respect to the bi-additive functor of interest, which is a right exact ``tensor product'' $\otimes$ in common contexts) but not \CrefAndHyperrefIfExist{definition:has_enough_injectives_or_projectives_for_an_abelian_category}{enough projectives}, so projective resolutions have not be available to compute the derived functors. Nevertheless, in the case that the abelian categories involved may both have enough flats and have enough projectives (e.g. when the abelian categories are the categories of abelian groups or modules over rings), then using either flat resolutions or projective resolutions to compute the derived functors of the bi-additive functor (which again is usually some kind of right exact ``tensor product'' $\otimes$) will yield the same result, see \Cref{corollary:right_exact_bifunctor_on_abelian_categories_with_enough_projectives_and_flats_yield_unambiguous_tors}. 


\begin{definition}[General circumstances for the existence of Tor functors] \label{definition:tor_functors_of_a_right_exact_monoidal_functor_on_an_abelian_category}
Let $\mathcal{A}, \calB, \calC$ be \CrefAndHyperrefIfExist{definition:abelian_category}{abelian categories}, and let $\otimes : \mathcal{A} \times \mathcal{B} \to \mathcal{C}$ be a \CrefAndHyperrefIfExist{definition:n_ary_additive_functor_between_additive_categories}{biadditive functor}
that is \CrefAndHyperrefIfExist{definition:exact_functor_between_abelian_categories}{right exact} in each variable.

\TODO{Do notation for tor in two categories, one category, modules over rings}

% Suppose $\mathcal{A}$ \CrefAndHyperrefIfExist{definition:has_enough_injectives_or_projectives_for_an_abelian_category}{has enough projective objects} or \TODO{define} has enough flat objects. 
For objects $A \in \calA$ and $B \in \calB$, the \hldef{Tor objects}
$$\hlin{\mathrm{Tor}_n(A, B) = \Tor_n^{\calA, \calB}(A,B)}$$
may be defined in one of several not-necessarily-equivalent ways:

\begin{enumerate}
    \item as the \CrefAndHyperrefIfExist{definition:left_right_derived_functors_of_a_right_left_exact_functor_between_abelian_categories_where_source_has_enough_projectives_injectives}{left derived functors} of $A \otimes -$ computed via \CrefAndHyperrefIfExist{definition:left_right_resolution_of_a_class_of_objects_in_an_abelian_category}{projective resolutions} of $B$ in $\mathcal{B}$, assuming that such a projective resolution exists.

    \item as the \CrefAndHyperrefIfExist{definition:left_right_derived_functors_of_a_right_left_exact_functor_between_abelian_categories_where_source_has_enough_projectives_injectives}{left derived functors} of $- \otimes B$ computed via \CrefAndHyperrefIfExist{definition:left_right_resolution_of_a_class_of_objects_in_an_abelian_category}{projective resolutions} of $B$ in $\mathcal{A}$, assuming that such a projective resolution exists.

    \item as the ``left derived functors'' of $A \otimes -$ computed via \CrefAndHyperrefIfExist{definition:flat_resolution_of_an_object_in_an_abelian_category_with_respect_to_a_right_exact_monoidal_product_functor}{flat resolutions} of $B$ in $\calB$, assuming that such a flat resolution exists. More precisely, $\Tor_n(A,B) = H_n(A \otimes F_\bullet)$ in this case where $F_\bullet \to B$ is a flat resolution. \TextIfExists{definition:generalized_derived_functors_for_biadditive_functors_of_abelian_categories_via_resolutions_by_flat_objects}{Equivalently, $\Tor_n(A,B) = \Tor_n^{II}(A,B)$ as in \Cref{definition:generalized_derived_functors_for_biadditive_functors_of_abelian_categories_via_resolutions_by_flat_objects}.}A priori, this might depend on the choice of flat resolution.

    \item as the ``left derived functors'' of $- \otimes B$ computed via \CrefAndHyperrefIfExist{definition:flat_resolution_of_an_object_in_an_abelian_category_with_respect_to_a_right_exact_monoidal_product_functor}{flat resolutions} of $A$ in $\calA$. More precisely, $\Tor_n(A,B) = H_n(F_\bullet \otimes B)$ in this case where $F_\bullet \to A$ is a flat resolution. \TextIfExists{definition:generalized_derived_functors_for_biadditive_functors_of_abelian_categories_via_resolutions_by_flat_objects}{Equivalently, $\Tor_n(A,B) = \Tor_n^{I}(A,B)$ as in \Cref{definition:generalized_derived_functors_for_biadditive_functors_of_abelian_categories_via_resolutions_by_flat_objects}.}A priori, this might depend on the choice of flat resolution.

\end{enumerate}
See \Cref{theorem:balancing_of_tor_on_abelian_categories_with_right_exact_bifunctor} and \Cref{lemma:flat_resolution_lemma_of_tor_objects_of_abelian_categories_with_a_right_exact_bifunctor_assuming_that_category_has_enough_projectives_or_flats}, which describe sufficient conditions under which the Tor functors defined via flat resolutions are independent of the choice of flat resolution. In particular \Cref{corollary:right_exact_bifunctor_on_abelian_categories_with_enough_projectives_and_flats_yield_unambiguous_tors} shows that all of the above notions are in natural agreement if $\calA$ and $\calB$ \CrefAndHyperrefIfExist{definition:has_enough_injectives_or_projectives_for_an_abelian_category}{have enough projectives} and \CrefAndHyperrefIfExist{definition:has_enough_flat_objects_for_an_abelian_category_with_respect_to_a_right_exact_bifunctor}{have enough flats} and filtered direct limits that exist in $\calC$ are exact.

For any of the above notions of $\Tor_n$, note that
\begin{enumerate}
    \item for fixed $A \in \calA$, if $\Tor_n(A,B)$ exists and is well defined for any $B \in \calB$, then $\Tor_n(A,-)$ is an additive functor $\calB \to \calC$. 
    \item for fixed $B \in \calB$, if $\Tor_n(A,B)$ exists and is well defined for any $A \in \calA$, then $\Tor_n(A,-)$ is an additive functor $\calA \to \calC$. 
    \item if $\Tor_n(A,B)$ exists and is well defined for any $A \in \calA$ and $B \in \calB$, then $\Tor_n(-,-)$ is an biadditive $\calA \times \calB \to \calC$. 
\end{enumerate}


% If $\mathcal{A}$ instead has enough flat objects and $\otimes$ is right exact and preserves finite direct sums, then $\mathrm{Tor}_n^{\mathcal{A}}(A, B)$ can be computed using flat resolutions in place of projective ones.

In the case that $\calA$ is the category of $R-S$-modules, $\calB$ is the category of $S-T$-modules, $\calC$ is the category of $R-T$-modules for some \CrefAndHyperrefIfExist{definition:ring}{(not necessarily commutative) rings $R,S,T$}, and $\otimes$ is the usual tensor product between $R-S$-modules and $S-T$-modules producing $R-T$-modules, then the Tor functors may be denoted by 
$$ \hlin{\mathrm{Tor}_n^S(A, B)} $$
for $A \in \mathcal{A}$ and $B \in \mathcal{B}$.

% \CrefAndHyperrefIfExist{definition:module_of_a_ring}{$R$-modules} for some \CrefAndHyperrefIfExist{definition:ring}{(not necessarily commutative ring} $R$ and $\otimes$ is the usual tensor product of (left/right/two-sided)$R$-modules, \TODO{define tensor product of $R$-modules} then the Tor functors may be denoted by 
% for $R$-modules $A$ and $B$. 
\end{definition}
\begin{theorem}[Balancing of Tor, cf. {\cite[Theorem 2.7.2]{weibel}}] \label{theorem:balancing_of_tor_on_abelian_categories_with_right_exact_bifunctor}
    Let $\mathcal{A}, \calB, \calC$ be \CrefAndHyperrefIfExist{definition:abelian_category}{abelian categories}, and let $\otimes : \mathcal{A} \times \mathcal{B} \to \mathcal{C}$ be a \CrefAndHyperrefIfExist{definition:n_ary_additive_functor_between_additive_categories}{biadditive functor}
    that is \CrefAndHyperrefIfExist{definition:exact_functor_between_abelian_categories}{right exact} in each variable. 
    Assume that (small) filtered colimits which exist in $\calC$ are exact (e.g. which holds if $\calC$ satisfies \CrefAndHyperrefIfExist{definition:grothendiecks_additional_axioms_for_abelian_categories}{Ab5}).

    Given $A \in \calA$ and $B \in \calB$ for which \CrefAndHyperrefIfExist{definition:flat_resolution_of_an_object_in_an_abelian_category_with_respect_to_a_right_exact_monoidal_product_functor}{flat resolutions exist}, let $\Tor_n^{I}(A,B)$ and $\Tor_n^{II}(A,B)$ respectively be the \CrefAndHyperrefIfExist{definition:tor_functors_of_a_right_exact_monoidal_functor_on_an_abelian_category}{Tor objects $\mathrm{Tor}_n^{\calA}(A,B)$}\CrefIfExists{definition:generalized_derived_functors_for_biadditive_functors_of_abelian_categories_via_resolutions_by_flat_objects} computed via \CrefAndHyperrefIfExist{definition:flat_resolution_of_an_object_in_an_abelian_category_with_respect_to_a_right_exact_monoidal_product_functor}{flat resolutions} of $A$ in $\calA$ and of $B$ in $\calB$.
    \begin{enumerate}
        \item $\Tor_n^{I}(A,B)$ and $\Tor_n^{II}(A,B)$ are naturally isomorphic.
        \item $\Tor_n^{I}(A,B)$ and $\Tor_n^{II}(A,B)$ are independent of the choice of flat resolution of $A$ and $B$ respectively.
    \end{enumerate}
    In particular, we may identify the objects $\Tor_n^{I}(A,B)$ and $\Tor_n^{II}(A,B)$ and simply write $\Tor_n(A,B)$ for either.
\end{theorem}

\begin{proof}
    This follows from \Cref{theorem:balancing_generalized_derived_functors_of_a_biadditive_functor_of_abelian_categories_computed_via_flat_resolutions}.
    % Choose flat resolutions $P_\bullet \xrightarrow{\varepsilon} A$ and $Q_\bullet \xrightarrow{\eta} B$. Identifying $A$ and $B$ with complexes concentrated in degree $0$, we can \CrefAndHyperrefIfExist{definition:double_complex_associated_to_biadditive_functor_and_chain_complexes}{form} the three tensor product \CrefAndHyperrefIfExist{definition:double_complex_of_objects_in_an_additive_category}{double complexes} $P_\bullet \otimes Q_\bullet$, $A \otimes Q_\bullet$, and $P_\bullet \otimes B$. Note that the augmentation morphisms $\varepsilon$ and $\eta$ induced morphisms $P_\bullet \otimes Q_\bullet \to A \otimes Q_\bullet, P_\bullet \otimes B$.

    % Let $C$ be the double complex of objects in $\calC$ obtained from $P_\bullet \otimes Q_\bullet$ by adding $A \otimes Q_{\bullet}[-1]$ in the column $p = -1$. The translate $\Tot(C)[1]$ is the mapping cone of the map 
    % $$\Tot(P_\bullet \otimes Q_\bullet) \xrightarrow{\varepsilon \otimes Q} \Tot(A \otimes Q_\bullet) = A \otimes Q_\bullet.$$
    % \TODO{why is the translate the mapping cone of $\varepsilon \otimes Q$?} Moreover, since each $- \otimes Q_q$ is an exact functor, every row of $C$ is exact, so $\Tot(C)$ is exact by \Cref{lemma:acyclic_assembly_lemma_for_bounded_double_complexes_with_exact_rows_or_columns}. Therefore, $\varepsilon \otimes Q$ is a quasi-isomorphism and hence $H_*(\Tot(P_\bullet \otimes Q_\bullet)) \xrightarrow{H_*(\varepsilon \otimes Q_\bullet)} H_*(A \otimes Q_\bullet)$ is a natural isomorphism. 

    % By symmetry, there is a natural isomorphism $H_*(\Tot(P_\bullet \otimes Q_\bullet)) \xrightarrow{} H_*(P_\bullet \otimes B)$. Therefore, $\Tor_n^{I}(A,B)$ and $\Tor_n^{II}(A,B)$ are naturally isomorphic as claimed. In particular, $\Tor_n^{I}(A,B)$ and $\Tor_n^{II}(A,B)$ are independent of the choice of flat resolution of $A$ and $B$ respectively.
\end{proof}


\begin{remark}
    Some common abelian categories do not \CrefAndHyperrefIfExist{definition:has_enough_injectives_or_projectives_for_an_abelian_category}{have enough projectives}, but nevertheless \CrefAndHyperrefIfExist{definition:has_enough_flat_objects_for_an_abelian_category_with_respect_to_a_right_exact_bifunctor}{have enough flats} and hence Tor objects are well defined as per \Cref{theorem:balancing_of_tor_on_abelian_categories_with_right_exact_bifunctor}. One such abelian category is the category of torsion abelian groups. Similarly, some categories of sheaves of torsion abelian groups on some sites may also have enought flats, but not enough projectives.
\end{remark}


\begin{lemma} \label{lemma:flat_objects_are_acyclic_for_tensoring_by_any_object_if_enough_projectives_or_flats_exist}
    Let $\mathcal{A}, \calB, \calC$ be \CrefAndHyperrefIfExist{definition:abelian_category}{abelian categories}, and let $\otimes : \mathcal{A} \times \mathcal{B} \to \mathcal{C}$ be a \CrefAndHyperrefIfExist{definition:n_ary_additive_functor_between_additive_categories}{biadditive functor}
    that is \CrefAndHyperrefIfExist{definition:exact_functor_between_abelian_categories}{right exact} in each variable.
    \begin{enumerate}
        \item Say that $\calB$ \CrefAndHyperrefIfExist{definition:has_enough_injectives_or_projectives_for_an_abelian_category}{has enough projectives} and let $\Tor_n(A,B)$ be the Tor object computed as the \CrefAndHyperrefIfExist{definition:left_right_derived_functors_of_a_right_left_exact_functor_between_abelian_categories_where_source_has_enough_projectives_injectives}{left derived functor} of $A \otimes -$ applied to $B$. 
        If $A$ is \CrefAndHyperrefIfExist{definition:flat_object_in_an_abelian_category_with_respect_to_a_right_exact_monoidal_product_functor}{flat}, then $\Tor_n(A,B) = 0$ for all $n \neq 0$ and all $B$.

        
        % The following are equivalent:
        % \begin{enumerate}
        %     \item $A$ is \CrefAndHyperrefIfExist{definition:flat_object_in_an_abelian_category_with_respect_to_a_right_exact_monoidal_product_functor}{flat}
        %     \item $\Tor_n(A,B) = 0$ for all $n \neq 0$ and all $B$.
        %     \item $\Tor_1(A,B) = 0$ for all $B$.
        % \end{enumerate}

        \item Say that $\calA$ \CrefAndHyperrefIfExist{definition:has_enough_injectives_or_projectives_for_an_abelian_category}{has enough projectives} and let $\Tor_n(A,B)$ be the Tor object computed as the left derived functor of $- \otimes B$ applied to $A$. 
        If $B$ is \CrefAndHyperrefIfExist{definition:flat_object_in_an_abelian_category_with_respect_to_a_right_exact_monoidal_product_functor}{flat}, then $\Tor_n(A,B) = 0$ for all $n \neq 0$ and all $A$.
        % The following are equivalent:
        % \begin{enumerate}
        %     \item $B$ is \CrefAndHyperrefIfExist{definition:flat_object_in_an_abelian_category_with_respect_to_a_right_exact_monoidal_product_functor}{flat}
        %     \item $\Tor_n(A,B) = 0$ for all $n \neq 0$ and all $A$.
        %     \item $\Tor_1(A,B) = 0$ for all $A$.
        % \end{enumerate}

        \item Say that $\calB$ \CrefAndHyperrefIfExist{definition:has_enough_flat_objects_for_an_abelian_category_with_respect_to_a_right_exact_bifunctor}{has enough flats} $F_\bullet \to B$ and let $\Tor_n(A,B)$ be the Tor object computed as $H_n(A \otimes F_\bullet)$. 
        If $A$ is \CrefAndHyperrefIfExist{definition:flat_object_in_an_abelian_category_with_respect_to_a_right_exact_monoidal_product_functor}{flat}, then $\Tor_n(A,B) = 0$ for all $n \neq 0$ and all $B$.
        % The following are equivalent:
        % \begin{enumerate}
        %     \item $A$ is \CrefAndHyperrefIfExist{definition:flat_object_in_an_abelian_category_with_respect_to_a_right_exact_monoidal_product_functor}{flat}
        %     \item $\Tor_n(A,B) = 0$ for all $n \neq 0$ and all $B$.
        %     \item $\Tor_1(A,B) = 0$ for all $B$.
        % \end{enumerate}

        \item Say that $\calA$ \CrefAndHyperrefIfExist{definition:has_enough_flat_objects_for_an_abelian_category_with_respect_to_a_right_exact_bifunctor}{has enough flats} $F_\bullet \to A$ and let $\Tor_n(A,B)$ be the Tor object computed as $H_n(F_\bullet \otimes B)$. 
        If $B$ is \CrefAndHyperrefIfExist{definition:flat_object_in_an_abelian_category_with_respect_to_a_right_exact_monoidal_product_functor}{flat}, then $\Tor_n(A,B) = 0$ for all $n \neq 0$ and all $A$.
        % The following are equivalent:
        % \begin{enumerate}
        %     \item $B$ is \CrefAndHyperrefIfExist{definition:flat_object_in_an_abelian_category_with_respect_to_a_right_exact_monoidal_product_functor}{flat}
        %     \item $\Tor_n(A,B) = 0$ for all $n \neq 0$ and all $A$.
        %     \item $\Tor_1(A,B) = 0$ for all $A$.
        % \end{enumerate}
    \end{enumerate}
\end{lemma}

\begin{proof}
    We prove the first part. The other parts hold similarly.
    
    If $A$ is flat, then for any $B \in \calB$, letting $P_\bullet \to B$ be some \CrefAndHyperrefIfExist{definition:left_right_resolution_of_a_class_of_objects_in_an_abelian_category}{projective resolution}, the exactness of $A \otimes -$ implies that $\Tor_n(A,B) = H_n(A \otimes P_\bullet) = 0$ for all $n \neq 0$.
\end{proof}



\begin{lemma}[cf. {\cite[Flat Resolution Lemma 3.2.8]{weibel}}] \label{lemma:flat_resolution_lemma_of_tor_objects_of_abelian_categories_with_a_right_exact_bifunctor_assuming_that_category_has_enough_projectives_or_flats}
Let $\mathcal{A}, \calB, \calC$ be \CrefAndHyperrefIfExist{definition:abelian_category}{abelian categories}, and let $\otimes : \mathcal{A} \times \mathcal{B} \to \mathcal{C}$ be a \CrefAndHyperrefIfExist{definition:n_ary_additive_functor_between_additive_categories}{biadditive functor}
that is \CrefAndHyperrefIfExist{definition:exact_functor_between_abelian_categories}{right exact} in each variable. 
Let $A \in \calA$ and $B \in \calB$ be objects.
\begin{enumerate}
    % \item Suppose that $A \in \calA$ is an object with a \CrefAndHyperrefIfExist{definition:left_right_resolution_of_a_class_of_objects_in_an_abelian_category}{projective resolution}. For any object $B \in \calB$, the objects
    % $$\mathrm{Tor}_n^{\mathcal{A}}(A, B) \in \calC$$
    % obtained via a \CrefAndHyperrefIfExist{definition:left_right_resolution_of_a_class_of_objects_in_an_abelian_category}{projective resolution of $A$} and via a \CrefAndHyperrefIfExist{definition:flat_resolution_of_an_object_in_an_abelian_category_with_respect_to_a_right_exact_monoidal_product_functor}{flat resolution of $A$} are naturally isomorphic. 

    \item Suppose that $\mathcal{A}$ \CrefAndHyperrefIfExist{definition:has_enough_injectives_or_projectives_for_an_abelian_category}{has enough projective objects}.
    % or \CrefAndHyperrefIfExist{definition:has_enough_flat_objects_for_an_abelian_category_with_respect_to_a_right_exact_bifunctor}{has enough flat objects}.
    Suppose that $A$ has some \CrefAndHyperrefIfExist{definition:flat_resolution_of_an_object_in_an_abelian_category_with_respect_to_a_right_exact_monoidal_product_functor}{flat resolution}. The \CrefAndHyperrefIfExist{definition:tor_functors_of_a_right_exact_monoidal_functor_on_an_abelian_category}{Tor objects} 
    $$\mathrm{Tor}_n(A, B) \in \calC$$
    obtained via a \CrefAndHyperrefIfExist{definition:left_right_resolution_of_a_class_of_objects_in_an_abelian_category}{projective resolution of $A$} and via a \CrefAndHyperrefIfExist{definition:flat_resolution_of_an_object_in_an_abelian_category_with_respect_to_a_right_exact_monoidal_product_functor}{flat resolution of $A$} are naturally isomorphic. In particular, either notion of $\Tor_n$ is well defined.


    % \item Suppose that $B \in \calB$ is an object with a \CrefAndHyperrefIfExist{definition:left_right_resolution_of_a_class_of_objects_in_an_abelian_category}{projective resolution}. For any object $A \in \calA$, the objects
    % $$\mathrm{Tor}_n^{\mathcal{A}}(A, B) \in \calC$$
    % obtained via a \CrefAndHyperrefIfExist{definition:left_right_resolution_of_a_class_of_objects_in_an_abelian_category}{projective resolution of $B$} and via a \CrefAndHyperrefIfExist{definition:flat_resolution_of_an_object_in_an_abelian_category_with_respect_to_a_right_exact_monoidal_product_functor}{flat resolution of $B$} are naturally isomorphic. 

    \item Suppose that $\mathcal{B}$ \CrefAndHyperrefIfExist{definition:has_enough_injectives_or_projectives_for_an_abelian_category}{has enough projective objects}.
    % or \CrefAndHyperrefIfExist{definition:has_enough_flat_objects_for_an_abelian_category_with_respect_to_a_right_exact_bifunctor}{has enough flat objects}.
    Suppose that $B$ has some \CrefAndHyperrefIfExist{definition:flat_resolution_of_an_object_in_an_abelian_category_with_respect_to_a_right_exact_monoidal_product_functor}{flat resolution}. The Tor objects 
    $$\mathrm{Tor}_n(A, B) \in \calC$$
    obtained via a \CrefAndHyperrefIfExist{definition:left_right_resolution_of_a_class_of_objects_in_an_abelian_category}{projective resolution of $B$} and via a \CrefAndHyperrefIfExist{definition:flat_resolution_of_an_object_in_an_abelian_category_with_respect_to_a_right_exact_monoidal_product_functor}{flat resolution of $B$} are naturally isomorphic. In particular, either notion of $\Tor_n$ is well defined.


\end{enumerate}
\end{lemma}

\begin{proof}

    We show $1$. Write $\Tor_n(A,B)$ for the tor object obtained via a projective resolution of $A$. Write $H_n(F_\bullet \otimes B)$ for the tor object obtained via a flat resolution $F_\bullet \to A$ of $A$. We argue by induction. For $n = 0$, we have $\Tor_0(A,B) \cong H_0(F_\bullet \otimes B)$ because $- \otimes B$ is right exact by assumption. 
   
    Let $K$ be such that $0 \to K \to F_0 \to A$ is exact. write $E_\bullet = (\cdots \to F_2 \to F_1)$ so that $E_\bullet \to K$ is a resolution of $K$ by flat objects (\Cref{lemma:an_object_of_abelian_category_with_enough_objects_of_a_class_on_the_right_left_has_right_left_resolution_by_the_class}). Since $\calA$ has enough projectives, further note that all flat objects are \CrefAndHyperrefIfExist{definition:F_acyclic_object_for_a_left_or_right_functor_between_abelian_categories}{acyclic} for the functor $- \otimes B$ by \Cref{lemma:flat_objects_are_acyclic_for_tensoring_by_any_object_if_enough_projectives_or_flats_exist}.
    
    By \Cref{lemma:dimension_shifting_lemma_for_acyclic_syzygies}, we have 
    $$\Tor_1(A,B) = \ker(K \otimes B \to F_0 \otimes B) = \ker \left\{ \frac{F_1 \otimes B}{\operatorname{im}(F_2 \otimes B) \to F_0 \otimes B} = H_1(F \otimes B)  \right\},$$
    thus establishing the desired isomorphism for $n = 1$. For $n \geq 2$, use induction to see that 
    $$\Tor_n(A,B) \cong \Tor_{n-1}(K, B) \cong H_{n-1}(E_\bullet \otimes B) = H_n(F \otimes B).$$

\end{proof}
 
\begin{corollary} \label{corollary:right_exact_bifunctor_on_abelian_categories_with_enough_projectives_and_flats_yield_unambiguous_tors}
Let $\mathcal{A}, \calB, \calC$ be \CrefAndHyperrefIfExist{definition:abelian_category}{abelian categories}, and let $\otimes : \mathcal{A} \times \mathcal{B} \to \mathcal{C}$ be a \CrefAndHyperrefIfExist{definition:n_ary_additive_functor_between_additive_categories}{biadditive functor}
that is \CrefAndHyperrefIfExist{definition:exact_functor_between_abelian_categories}{right exact} in each variable.
Suppose that $\calA$ and $\calB$ \CrefAndHyperrefIfExist{definition:has_enough_injectives_or_projectives_for_an_abelian_category}{have enough projectives} and \CrefAndHyperrefIfExist{definition:has_enough_flat_objects_for_an_abelian_category_with_respect_to_a_right_exact_bifunctor}{flats}. Further suppose that (small) filtered colimits which exist in $\calC$ are exact (e.g. which holds if $\calC$ satisfies \CrefAndHyperrefIfExist{definition:grothendiecks_additional_axioms_for_abelian_categories}{Ab5}).

For any objects $A \in \calA$ and $B \in \calB$, all notions of $\Tor_n(A,B)$ as defined in \Cref{definition:tor_functors_of_a_right_exact_monoidal_functor_on_an_abelian_category} naturally agree with one another.
\end{corollary}

\begin{proof}
    This follows from \Cref{theorem:balancing_of_tor_on_abelian_categories_with_right_exact_bifunctor} and \Cref{lemma:flat_resolution_lemma_of_tor_objects_of_abelian_categories_with_a_right_exact_bifunctor_assuming_that_category_has_enough_projectives_or_flats}.
\end{proof}










% \begin{definition} \label{definition:tensor_product_of_chain_complexes_of_objects_of_additive_categories_with_an_additive_bifunctor}
    \TODO{this definition can in fact be generalized to more general biadditive functors, not just ones thought of as tensor products}
Let $\mathcal{A}$, $\mathcal{B}$, and $\calC$ be \CrefAndHyperrefIfExist{definition:additive_category_preadditive_category}{additive categories}, and suppose
$$ \otimes : \mathcal{A} \times \mathcal{B} \to \mathcal{C} $$
is a \CrefAndHyperrefIfExist{definition:n_ary_additive_functor_between_additive_categories}{biadditive functor}.

Let $(C_\bullet, d^C)$ be a \CrefAndHyperrefIfExist{definition:chain_complex_of_objects_in_an_additive_category}{chain complex} in $\mathcal{A}$ and $(D_\bullet, d^D)$ be a chain complex in $\mathcal{B}$, i.e.,
$$ C_\bullet = \{ C_n, d^C_n : C_n \to C_{n-1} \}_{n \in \mathbb{Z}}, \quad D_\bullet = \{ D_m, d^D_m : D_m \to D_{m-1} \}_{m \in \mathbb{Z}}.  $$

\begin{enumerate}
    \item The \hldef{tensor product double complex} $(T^{\bullet, \bullet}, d'_T, d''_T)$ in $\mathcal{C}$ is defined as follows:
    \begin{itemize}
    \item Objects:
    $$
    T^{p,q} := C_p \otimes D_q,
    $$
    for all integers $p,q$.
    \item Horizontal differentials:
    $$
    d'_T{}^{p,q} := d^C_p \otimes \mathrm{id}_{D_q} : T^{p,q} \to T^{p-1,q}.
    $$
    \item Vertical differentials:
    $$
    d''_T{}^{p,q} := (-1)^p \cdot (\mathrm{id}_{C_p} \otimes d^D_q) : T^{p,q} \to T^{p,q-1}.
    $$
    \end{itemize}
    These satisfy the conditions for a \CrefAndHyperrefIfExist{definition:double_complex_of_objects_in_an_additive_category}{double complex}: the horizontal and vertical differentials each square to zero and anticommute appropriately due to the sign $(-1)^p$. The complex $T^{p,q}$ is often denoted by \hl{$C_\bullet \otimes D_\bullet$}.

    \item The \hldef{(total) tensor product chain complex} \hl{$(\Tot(T)_{\bullet}, d^{\Tot})$} associated to the tensor product double complex $(T^{\bullet,\bullet}, d'_T, d''_T)$ is defined by:
    \begin{itemize}
    \item Objects (in degree $n$):
    $$ \Tot(T)_n := \bigoplus_{p+q=n} T^{p,q} = \bigoplus_{p+q=n} C_p \otimes D_q.  $$
    \item Differential:
    $$ d^{\Tot}_n : \Tot(T)_n \to \Tot(T)_{n-1} $$
    acts on homogeneous elements $x \otimes y \in C_p \otimes D_q$ with $p+q=n$ by
    $$
    d^{\Tot}_n (x \otimes y) := d^C_p(x) \otimes y + (-1)^p x \otimes d^D_q(y).
    $$
    \end{itemize}
    Equivalently, $\Tot(T)_{\bullet}$ is the \CrefAndHyperrefIfExist{definition:total_complexes_of_a_double_complex_of_objects_in_an_additive_category}{total complex} \hl{$\Tot^{\oplus}(P \otimes Q)$} of the double complex $P \otimes Q$. 

    This construction yields a biadditive functor
    $$ \otimes : \mathbf{Ch}(\mathcal{A}) \times \mathbf{Ch}(\mathcal{B}) \to \mathbf{Ch}(\mathcal{C}), $$
    where \CrefAndHyperrefIfExist{definition:chain_complex_of_objects_in_an_additive_category}{$\mathbf{Ch}(\mathcal{A})$} denotes the category of chain complexes in $\mathcal{A}$.
\end{enumerate}

\end{definition}


\section{Derived categories}




\import{../_excerpts}{excerpts_derived_categories.tex}


\subsection{Total derived functors of biadditive functors}

We extend the ideas from \Cref{subsubsection:derived_functors_for_biadditive_functors_computed_via_resolutions_of_flat_objects}


\begin{definition} \label{definition:left_and_right_total_derived_functors_of_biadditive_functors_of_abelian_categories}
    Let $\mathcal{A}, \calB, \calC$ be \CrefAndHyperrefIfExist{definition:abelian_category}{abelian categories}, and let $F: \mathcal{A} \times \mathcal{B} \to \mathcal{C}$ be a \CrefAndHyperrefIfExist{definition:n_ary_additive_functor_between_additive_categories}{biadditive functor}. 

    There are notions of derived functors that we may consider depending on whether $\calA$ or $\calB$  has enough projectives or injectives or \CrefAndHyperrefIfExist{definition:flat_object_in_an_abelian_category_with_respect_to_a_right_exact_monoidal_product_functor}{flats} with respect to $F$. \TODO{define for flats}

    \TODO{I really should be letting $M$ and $N$ be complxes, not just objects.}

    \begin{enumerate}
    \item         
    \end{enumerate}


    \begin{enumerate}
        \item 
        \begin{enumerate}
            \item Let $M \in \calA$. Assume that $\calB$ \CrefAndHyperrefIfExist{definition:has_enough_injectives_or_projectives_for_an_abelian_category}{has enough projectives}. By \Cref{corollary:additive_functor_bewteen_abelian_categories_has_a_right_or_left_total_derived_functor_if_the_source_abelian_category_has_enough_injectives_or_projectives}, the functor $F(M, -): \calB \to \calC$ has a \CrefAndHyperrefIfExist{definition:total_derived_functor_of_an_exact_functor_between_homotopy_categories_of_abelian_categories}{left derived functor} $L(F(M,-)): D^-(\calB) \to D(\calC)$\CrefIfExists{definition:derived_category_of_an_abelian_category}, which we may write as \hl{$LF(M,-)$}.

            \item Symmetrically, let $N \in \calB$. Assume that $\calA$ \CrefAndHyperrefIfExist{definition:has_enough_injectives_or_projectives_for_an_abelian_category}{has enough projectives}. By \Cref{corollary:additive_functor_bewteen_abelian_categories_has_a_right_or_left_total_derived_functor_if_the_source_abelian_category_has_enough_injectives_or_projectives}, the functor $F(-, N): \calA \to \calC$ has a \CrefAndHyperrefIfExist{definition:total_derived_functor_of_an_exact_functor_between_homotopy_categories_of_abelian_categories}{left derived functor} $L(F(-, N)): D^-(\calA) \to D(\calC)$\CrefIfExists{definition:derived_category_of_an_abelian_category}, which we may write as \hl{$LF(-,N)$}.
        \end{enumerate}
        Assuming that $\calA$ and $\calB$ have enough projectives, the notations $LF(M,N)$ above are in agreement (\Cref{proposition:total_derived_functors_for_biadditive_functors_on_abelian_categories_are_well_defined_if_both_source_categories_have_enough_projectives_injectives})


        \item
        \begin{enumerate}
            \item Let $M \in \calA$. Assume that $\calB$ \CrefAndHyperrefIfExist{definition:has_enough_injectives_or_projectives_for_an_abelian_category}{has enough injectives}. By \Cref{corollary:additive_functor_bewteen_abelian_categories_has_a_right_or_left_total_derived_functor_if_the_source_abelian_category_has_enough_injectives_or_projectives}, the functor $F(M, -): \calB \to \calC$ has a \CrefAndHyperrefIfExist{definition:total_derived_functor_of_an_exact_functor_between_homotopy_categories_of_abelian_categories}{right derived functor} $R(F(M,-)): D^+(\calB) \to D(\calC)$\CrefIfExists{definition:derived_category_of_an_abelian_category}, which we may write as \hl{$RF(M,-)$}.

            \item Symmetrically, let $N \in \calB$. Assume that $\calA$ \CrefAndHyperrefIfExist{definition:has_enough_injectives_or_projectives_for_an_abelian_category}{has enough injectives}. By \Cref{corollary:additive_functor_bewteen_abelian_categories_has_a_right_or_left_total_derived_functor_if_the_source_abelian_category_has_enough_injectives_or_projectives}, the functor $F(-, N): \calA \to \calC$ has a \CrefAndHyperrefIfExist{definition:total_derived_functor_of_an_exact_functor_between_homotopy_categories_of_abelian_categories}{right derived functor} $R(F(-, N)): D^+(\calA) \to D(\calC)$\CrefIfExists{definition:derived_category_of_an_abelian_category}, which we may write as \hl{$RF(-,N)$}.
        \end{enumerate}
        Assuming that $\calA$ and $\calB$ have enough injectives, the notations $RF(M,N)$ above are in agreement (\Cref{proposition:total_derived_functors_for_biadditive_functors_on_abelian_categories_are_well_defined_if_both_source_categories_have_enough_projectives_injectives})

        \item Let $M \in \calA$. Assume that $\calB$ \CrefAndHyperrefIfExist{definition:has_enough_flat_objects_for_an_abelian_category_with_respect_to_a_right_exact_bifunctor}{has enough flats} with respect to $\otimes$. Define \hl{$LF(M, -): D^-(\calB) \to D(\calC)$} as follows:
    \end{enumerate}
    See \Cref{definition:derived_tensor_product_on_bounded_above_derived_categories_of_abelian_categories_for_a_biadditive_right_exact_functor} for notation used in the case that $F$ is written as a tensor product.
\end{definition}

\begin{proposition} \label{proposition:total_derived_functors_for_biadditive_functors_on_abelian_categories_are_well_defined_if_both_source_categories_have_enough_projectives_injectives}
    Let $\mathcal{A}, \calB, \calC$ be \CrefAndHyperrefIfExist{definition:abelian_category}{abelian categories}, and let $F: \mathcal{A} \times \mathcal{B} \to \mathcal{C}$ be a \CrefAndHyperrefIfExist{definition:n_ary_additive_functor_between_additive_categories}{biadditive functor}. 
    \begin{enumerate}
        \item Suppose that $\calA$ and $\calB$ both \CrefAndHyperrefIfExist{definition:has_enough_injectives_or_projectives_for_an_abelian_category}{have enough projectives}. Given objects $M \in D^-(\calA)$ and $N \in D^-(\calA)$\CrefIfExists{definition:derived_category_of_an_abelian_category}, the objects $LF(M,N)$ obtained as $(LF(M,-))(N)$ and $(LF(-,N))(M)$ are naturally isomorphic. Thus, the two definitions of $LF(M,N)$ in \Cref{definition:left_and_right_total_derived_functors_of_biadditive_functors_of_abelian_categories} are in agreement.
        \item Dually, suppose that $\calA$ and $\calB$ both \CrefAndHyperrefIfExist{definition:has_enough_injectives_or_projectives_for_an_abelian_category}{have enough injectives}. Given objects $M \in D^+(\calA)$ and $N \in D^+(\calA)$\CrefIfExists{definition:derived_category_of_an_abelian_category}, the objects $RF(M,N)$ obtained as $(RF(M,-))(N)$ and $(RF(-,N))(M)$ are naturally isomorphic. Thus, the two definitions of $RF(M,N)$ in \Cref{definition:left_and_right_total_derived_functors_of_biadditive_functors_of_abelian_categories} are in agreement.
    \end{enumerate}
\end{proposition}
\begin{proof}
    We prove 1. The other part is dual. By \Cref{theorem:derived_categories_can_be_identified_with_homotopy_categories_of_injectives_or_projectives}, note that $D^-(\calA)$ and $D^-(\calB)$ are respectively equivalent to the categories \CrefAndHyperrefIfExist{definition:category_of_bounded_complexes_of_injectives_projectives}{$K^-(\calP_\calA)$ and $K^-(\calP_\calB)$} of cohomologically bounded above complexes of projectives in $\calA$ and $\calB$ respectively. Let 
    $$P_\bullet \to M$$
    and 
    $$Q_\bullet \to N$$
    be projective resolutions.
    By \Cref{corollary:additive_functor_bewteen_abelian_categories_has_a_right_or_left_total_derived_functor_if_the_source_abelian_category_has_enough_injectives_or_projectives}, 
    $$(LF(M,-)(N) \cong q(K(F(M,-)))(N)$$
    $$(LF(-,N)(M) \cong q(K(F(-,N)))(M)$$
    The former is represented by the complex $F(M,Q_\bullet)$ and the latter is represented by the complex $F(P_\bullet, N)$. These are quasi-isomorphic by \Cref{lemma:applying_biadd_func_to_res_is_defined_up_to_quasi_iso_and_agrees_with_the_tot_complex_of_the_double_complex_of_the_biadd_func_on_the_two_res}.
\end{proof}




\subsubsection{Derived tensor product}




\begin{definition} \label{definition:derived_tensor_product_on_bounded_above_derived_categories_of_abelian_categories_for_a_biadditive_right_exact_functor}
    Let $\mathcal{A}, \calB, \calC$ be \CrefAndHyperrefIfExist{definition:abelian_category}{abelian categories}, and let $\otimes : \mathcal{A} \times \mathcal{B} \to \mathcal{C}$ be a \CrefAndHyperrefIfExist{definition:n_ary_additive_functor_between_additive_categories}{biadditive functor} written as tensor product. 

    \TODO{need a definition with enough flats}
    \begin{enumerate}
        \item Let $M \in \calA$. Assume that $\calB$ \CrefAndHyperrefIfExist{definition:has_enough_injectives_or_projectives_for_an_abelian_category}{has enough projectives}. We write \hl{$M \otimes^L -$} for \CrefAndHyperref{definition:left_and_right_total_derived_functors_of_biadditive_functors_of_abelian_categories}{$LF(M, -): D^-(\calB) \to D(\calC)$} in the case that $F = M \otimes -: \calB \to \calC$.
        
        \item Symmetrically, let $N \in \calB$. Assume that $\calA$ \CrefAndHyperrefIfExist{definition:has_enough_injectives_or_projectives_for_an_abelian_category}{has enough projectives}. We write \hl{$- \otimes^L N$} for \CrefAndHyperref{definition:left_and_right_total_derived_functors_of_biadditive_functors_of_abelian_categories}{$LF(-, N): D^+(\calA) \to D(\calC)$} in the case that $F = - \otimes N: \calA \to \calC$.

        \TODO{show that projectives vs. flats yield the same thing}
        \TODO{show that flats in each variable yield the same thing}
        \item Let $M \in \calA$. Assume that $\calB$ \CrefAndHyperrefIfExist{definition:has_enough_flat_objects_for_an_abelian_category_with_respect_to_a_right_exact_bifunctor}{has enough flats}. We alternatively define \hl{$M \otimes^L -: D^-(\calB) \to D(\calC)$} as follows --- given an object $N \in D^-(\calB)$, say that $Q^\bullet$ is a complex of flat objects in $\calB$ representing $N$ \TODO{show that such a thing exists}, and let $M \otimes^L N$ be the object of $D(\calC)$ represented by the complex $M \otimes Q^\bullet$.  \TODO{show that this is well defined, i.e. does not depend on the choice of flat resolution}

        \item Symmetrically, let $N \in \calB$. Assume that $\calA$ \CrefAndHyperrefIfExist{definition:has_enough_flat_objects_for_an_abelian_category_with_respect_to_a_right_exact_bifunctor}{has enough flats}. We alternatively define \hl{$- \otimes^L N: D^-(\calA) \to D(\calC)$} as follows --- given an object $M \in D^-(\calA)$, say that $P^\bullet$ is a complex of flat objects in $\calA$ representing $M$ \TODO{show that such a thing exists}, and let $M \otimes^L N$ be the object of $D(\calC)$ represented by the complex $P^\bullet \otimes N$.  \TODO{show that this is well defined, i.e. does not depend on the choice of flat resolution}

    \end{enumerate}
    The first two notions agree assuming that $\calA$ and $\calB$ have enough projectives. \TODO{comment on the next two notions agreeing}
\end{definition}


\begin{definition} \label{definition:K_flat_complex_in_an_abelian_category_with_respect_to_a_biadditive_functor}
    Let $\calA, \calB, \calC$ be \CrefAndHyperrefIfExist{definition:abelian_category}{abelian categories}, and let $F: \calA \times \calB \to \calC$ be a \CrefAndHyperrefIfExist{definition:n_ary_additive_functor_between_additive_categories}{bi-additive functor}. Let us say that a \CrefAndHyperrefIfExist{definition:chain_complex_of_objects_in_an_additive_category}{complex} $K^\bullet$ of $\calA$ (resp. of $\calB$) is \hl{$K$-flat (with respect to $F$ on the left, resp. right)} if for every \CrefAndHyperrefIfExist{definition:acyclic_complex_of_objects_in_an_abelian_category}{acyclic complex} $M^\bullet$ of $\calB$ (resp. of $\calA$), the \CrefAndHyperrefIfExist{definition:total_complexes_of_a_double_complex_of_objects_in_an_additive_category}{direct sum total complex} $\Tot^\oplus (F(K^\bullet,M^\bullet))$\CrefIfExists{definition:double_complex_associated_to_biadditive_functor_and_chain_complexes} (resp. $\Tot^\oplus (F(M^\bullet,K^\bullet))$) is acyclic.
\end{definition}

\begin{lemma} \label{lemma:bi_additive_functor_on_additive_categories_that_preserves_colimits_or_limits_preserve_colimits_or_limtis_as_functor_on_chain_categories}
    Let $F: \calA \times \calB \to \calC$ be a \CrefAndHyperrefIfExist{definition:n_ary_additive_functor_between_additive_categories}{bi-additive functor} where $\calA, \calB, \calC$ are \CrefAndHyperrefIfExist{definition:additive_category}{additive categories}. Let $J$ be a \CrefAndHyperrefIfExist{definition:category}{category}.
    \begin{enumerate}
        \item  Assume that $F(X,-): \calB \to \calC$ preserves \CrefAndHyperrefIfExist{definition:limit_and_colimit_of_a_diagram_in_a_category}{colimits} of \CrefAndHyperrefIfExist{definition:diagram_in_a_category_indexed_by_a_small_category}{diagrams of shape $J$} for all objects $X \in \calA$. Then the induced functor $F(X^\bullet,-): \mathbf{Ch}(\calB) \to \mathbf{DC}(\calC)$ (\Cref{definition:double_complex_associated_to_biadditive_functor_and_chain_complexes})\CrefIfExists{definition:chain_complex_of_objects_in_an_additive_category}\CrefIfExists{definition:morphism_of_double_complex_of_objects_in_an_additive_category} also preserves colimits of diagrams of shape $J$ for all objects $X^\bullet \in \mathbf{Ch}(\calA)$.


        \item Dually, assume that $F(-,Y): \calA \to \calC$ preserves \CrefAndHyperrefIfExist{definition:limit_and_colimit_of_a_diagram_in_a_category}{colimits} of \CrefAndHyperrefIfExist{definition:diagram_in_a_category_indexed_by_a_small_category}{diagrams of shape $J$} for all objects $Y \in \calB$.
        Then the induced functor $F(-,Y^\bullet): \mathbf{Ch}(\calA) \to \mathbf{DC}(\calC)$ (\Cref{definition:double_complex_associated_to_biadditive_functor_and_chain_complexes})\CrefIfExists{definition:chain_complex_of_objects_in_an_additive_category}\CrefIfExists{definition:morphism_of_double_complex_of_objects_in_an_additive_category} also preserves colimits of diagrams of shape $J$ for all objects $Y^\bullet \in \mathbf{Ch}(\calB)$.

        \item  Assume that $F(X,-): \calB \to \calC$ preserves \CrefAndHyperrefIfExist{definition:limit_and_colimit_of_a_diagram_in_a_category}{limits} of \CrefAndHyperrefIfExist{definition:diagram_in_a_category_indexed_by_a_small_category}{diagrams of shape $J$} for all objects $X \in \calA$. Then the induced functor $F(X^\bullet,-): \mathbf{Ch}(\calB) \to \mathbf{DC}(\calC)$ (\Cref{definition:double_complex_associated_to_biadditive_functor_and_chain_complexes})\CrefIfExists{definition:chain_complex_of_objects_in_an_additive_category}\CrefIfExists{definition:morphism_of_double_complex_of_objects_in_an_additive_category} also preserves limits of diagrams of shape $J$ for all objects $X^\bullet \in \mathbf{Ch}(\calA)$.


        \item Dually, assume that $F(-,Y): \calA \to \calC$ preserves \CrefAndHyperrefIfExist{definition:limit_and_colimit_of_a_diagram_in_a_category}{limits} of \CrefAndHyperrefIfExist{definition:diagram_in_a_category_indexed_by_a_small_category}{diagrams of shape $J$} for all objects $Y \in \calB$.
        Then the induced functor $F(-,Y^\bullet): \mathbf{Ch}(\calA) \to \mathbf{DC}(\calC)$ (\Cref{definition:double_complex_associated_to_biadditive_functor_and_chain_complexes})\CrefIfExists{definition:chain_complex_of_objects_in_an_additive_category}\CrefIfExists{definition:morphism_of_double_complex_of_objects_in_an_additive_category} also preserves limits of diagrams of shape $J$ for all objects $Y^\bullet \in \mathbf{Ch}(\calB)$.
    \end{enumerate}
    
\end{lemma}
\begin{proof}
    We prove that if $F(X,-): \calB \to \calC$ preserves colimtis of diagrams of shape $J$ for all $X \in \calA$, then so does $F(X^\bullet, -): \mathbf{Ch}(\calB) \to \mathbf{DC}(\calC)$ for all $X^\bullet \in \mathbf{Ch}(\calA)$. Given $X^\bullet \in \mathbf{Ch}(\calA)$ and $Y^\bullet \in \mathbf{Ch}(\calB)$, the term of $F(X^\bullet, Y^\bullet)$ at position $(p,q)$ is $F(X^p, Y^q)$. Let $J \to \mathbf{Ch}(\calB): j \mapsto Y_{(j)}^\bullet$ be some \CrefAndHyperrefIfExist{definition:diagram_in_a_category_indexed_by_a_small_category}{diagram} whose colimit is $Y^\bullet$. In fact, since colimits of chain complexes are computed termwise (\Cref{proposition:limit_and_colimit_or_diagram_of_chain_complexes_is_computed_pointwise}), there are canonical isomorphisms
    $$Y^q \cong \colim_{j \in J} Y_{(j)}^q $$
    in $\calB$ .

    We wish to show that $F(X^\bullet, Y^\bullet)$ is the colimit of the diagram $j \mapsto F(X^\bullet, Y_{(j)}^\bullet)$ in $\mathbf{DC}(\mathcal{C})$. The colimit in $\mathbf{DC}(\mathcal{C})$ is also computed termwise (note that $\mathbf{DC}(\mathcal{C})$ is \CrefAndHyperref{theorem:category_of_double_complexes_of_objects_of_an_additive_category_is_naturally_isomorphic_to_the_category_of_chain_complexes_of_chain_complexes}{identifiable} as $\mathbf{Ch}(\mathbf{Ch}(\calC))$), it suffices to verify the isomorphism at each position $(p,q)$ and ensure it respects the differentials.

    Consider the term at position $(p,q)$. The functor $F(X^\bullet, -)$ maps the colimit diagram in $\mathbf{Ch}(\mathcal{B})$ to a \CrefAndHyperrefIfExist{definition:limit_and_colimit_of_a_diagram_in_a_category}{cocone} in $\mathbf{DC}(\mathcal{C})$. This induces a canonical comparison morphism:
    \[ \phi: \operatorname{colim}_{j \in J} F(X^\bullet, Y_{(j)}^\bullet) \longrightarrow F(X^\bullet, \operatorname{colim}_{j \in J} Y_{(j)}^\bullet) = F(X^\bullet, Y^\bullet). \]
    At the position $(p,q)$, this morphism is the map:
    \[ \phi^{p,q}: \operatorname{colim}_{j \in J} F(X^p, Y_{(j)}^q) \longrightarrow F(X^p, Y^q). \]
    By hypothesis, $F(X^p, -): \mathcal{B} \to \mathcal{C}$ preserves colimits of shape $J$. Therefore, $\phi^{p,q}$ is an isomorphism for all $p,q \in \mathbb{Z}$.
    
    Since a morphism of double complexes is an isomorphism if and only if it is an isomorphism at every term $(p,q)$, we conclude that $\phi$ is an isomorphism. Thus, $F(X^\bullet, -)$ preserves colimits of shape $J$.
    
    % Note that
    % $$F(X^p, Y^q) \cong F(X^p, \colim_{j \in J} Y_{(j)}^q) \cong \colim_{j \in J} F(X^p, Y_{(j)}^q);$$
    % these isomorphisms are functorial in $X^p$ due to Lemma \ref{lemma:bifunctors_induce_functors_to_a_functor_category_and_natural_transformations}, so they in fact induce an isomorphism
    % $$F(X^\bullet, Y^\bullet) \cong F(X^\bullet, \colim_j Y_{(j)}^\bullet)$$
    % functorial in the first argument and natural in the second.
\end{proof}


\begin{lemma}
    Let $\calA, \calB, \calC$ be \CrefAndHyperrefIfExist{definition:abelian_category}{abelian categories}, and let $F: \calA \times \calB \to \calC$ be a \CrefAndHyperrefIfExist{definition:n_ary_additive_functor_between_additive_categories}{bi-additive functor}. Assume that $\calC$ is \CrefAndHyperrefIfExist{definition:grothendiecks_additional_axioms_for_abelian_categories}{Ab5}.
    \begin{enumerate}
        \item
        
        Suppose that $F(X,-): \calB \to \calC$ preserves \CrefAndHyperrefIfExist{definition:projective_and_inductive_limits_in_categories}{filtered colimits} for all objects $X \in \calA$.
        % is a \CrefAndHyperrefIfExist{definition:adjoint_functors_between_categories_unit_counit_of_adjoint_functors}{}right adjoint functor for all 
        Let $P^\bullet$ be a \CrefAndHyperrefIfExist{definition:bounded_complexes_on_an_additive_category_and_homologically_bounded_objects_on_an_abelian_category}{bounded above} complex of objects of $\calA$ that are \CrefAndHyperrefIfExist{definition:flat_object_in_an_abelian_category_with_respect_to_a_right_exact_monoidal_product_functor}{flat} with respect to $F$ on the left. The complex $P^\bullet$ is \CrefAndHyperrefIfExist{definition:K_flat_complex_in_an_abelian_category_with_respect_to_a_biadditive_functor}{$K$-flat} with respect to $F$ on the left.

        \item Suppose that $F(-,Y): \calA \to \calC$ preserves \CrefAndHyperrefIfExist{definition:projective_and_inductive_limits_in_categories}{filtered colimits} for all objects $Y \in \calB$. Let $P^\bullet$ be a bounded above complex of objects in $\calB$ that are flat with respect to $F$ on the right. The complex $P^\bullet$ is $K$-flat with respect to $F$ on the right.
    \end{enumerate}
\end{lemma}
\begin{proof}
    We show the claimed statement in the case that $P^\bullet$ is a complex of flat objects of $\calA$ with respect to $F$ on the left; the other case can be argued symmetrically. Let $M^\bullet$ be an \CrefAndHyperrefIfExist{definition:acyclic_complex_of_objects_in_an_abelian_category}{acyclic complex} of objects in $\calA$. Writing $\tau_{\leq k} M^\bullet$ for the \CrefAndHyperrefIfExist{definition:stupid_truncation_of_a_chain_complex_in_an_additive_category}{brutal truncation} of $M^\bullet$, note that $M^\bullet$ is the filtered colimit of these truncations (\Cref{proposition:chain_complexes_in_additive_categories_are_limits_and_colimits_of_truncations}):
    $$M^\bullet = \varinjlim_k \tau_{\leq k} M^\bullet.$$
    Since $F(X,-): \calB \to \calC$ preserves filtered colimits, so does $F(X^\bullet, -): \mathbf{Ch}(\calB) \to \mathbf{DC}(\calC)$\CrefIfExists{definition:double_complex_associated_to_biadditive_functor_and_chain_complexes}\CrefIfExists{definition:chain_complex_of_objects_in_an_additive_category}\CrefIfExists{definition:morphism_of_double_complex_of_objects_in_an_additive_category} by \Cref{lemma:bi_additive_functor_on_additive_categories_that_preserves_colimits_or_limits_preserve_colimits_or_limtis_as_functor_on_chain_categories}. Therefore,
    $$\Tot^{\oplus} F(P^\bullet, M^\bullet) \cong \Tot^{\oplus} F(P^\bullet, \varinjlim_k \tau_{\leq k} M^\bullet) \cong \varinjlim_{k} \Tot^{\oplus} F(P^\bullet, \tau_{\leq k} M^\bullet).$$
    Since $\tau_{\leq k} M^\bullet$ is bounded above and is of flat objects, the double complex \CrefAndHyperrefIfExist{definition:double_complex_associated_to_biadditive_functor_and_chain_complexes}{$F(P^\bullet, \tau_{\leq k} M^\bullet)$} is \CrefAndHyperrefIfExist{definition:bounded_double_complex_of_objects_in_an_additive_category}{bounded above}, and each row is acyclic. Since $\calC$ satisfies Ab$5$, $\Tot^{\oplus} F(P^\bullet, \tau_{\leq k} M^\bullet)$ is acyclic by Lemma \ref{lemma:acyclic_assembly_lemma_for_bounded_double_complexes_with_exact_rows_or_columns}. That $\calC$ satisfies AB$5$ further implies that 
    $$H^n(\Tot^{\oplus}  F(P^\bullet, M^\bullet)) \cong \varinjlim_{k} H^n \Tot^{\oplus} F(P^\bullet, \tau_{\leq k} M^\bullet) = 0,$$
    i.e. $\Tot^{\oplus}  F(P^\bullet, M^\bullet)$ is acyclic. Therefore, $P^\bullet$ is $K$-flat as desired. 
\end{proof}



\subsection{Ext functors}

\subsection{Serre subcategories}

\begin{definition}[Serre subcategory] \label{definition:serre_subcategory_of_an_abelian_category}
    Let $\mathcal{A}$ be an \CrefAndHyperrefIfExist{definition:abelian_category}{abelian category}. A \CrefAndHyperrefIfExist{definition:full_subcategory_of_a_category}{full subcategory} $\mathcal{S} \subseteq \mathcal{A}$ is called a \hldef{Serre subcategory} (or sometimes a \hldef{thick subcategory}) if it satisfies the following conditions:
 
    \begin{enumerate}
        \item For any \CrefAndHyperrefIfExist{definition:short_exact_sequence_in_an_additive_category}{short exact sequence}
        $$ 0 \to A' \to A \to A'' \to 0 $$
        in $\mathcal{A}$, the object $A$ lies in $\mathcal{S}$ if and only if both $A'$ and $A''$ lie in $\mathcal{S}$.
        
        \TODO{extension}
        \item Equivalently, $\mathcal{S}$ is closed under taking \CrefAndHyperrefIfExist{definition:subobject_of_an_object_of_an_additive_category}{subobjects}, \CrefAndHyperrefIfExist{definition:quotient_object_of_an_object_of_an_abelian_category_by_a_subobject}{quotients}, and extensions in $\mathcal{A}$.
    \end{enumerate}

    In other words, $\mathcal{S}$ is a Serre subcategory if for every exact sequence in $\mathcal{A}$, the presence of any two of the objects in $\mathcal{S}$ forces the third to be in $\mathcal{S}$.

    The notion of a thick subcategory of an abelian category should not be confused for the notion of a \CrefAndHyperrefIfExist{definition:thick_subcategory_of_a_triangulated_category}{thick subcategory} of a triangulated category.
\end{definition}


\begin{definition}[Thick subcategory] \label{definition:thick_subcategory_of_a_triangulated_category}
    Let $\mathcal{T}$ be a \CrefAndHyperrefIfExist{definition:triangulated_category}{triangulated category}. A \CrefAndHyperrefIfExist{definition:full_subcategory_of_a_category}{full} triangulated subcategory $\mathcal{S} \subseteq \mathcal{T}$ is called \hldef{thick} (also \hldef{epaisse}) if it is closed under \CrefAndHyperrefIfExist{definition:additive_category_preadditive_category}{direct summands}. More explictly, for any object $X \in \mathcal{T}$, if $X$ is isomorphic to a \CrefAndHyperrefIfExist{definition:additive_category_preadditive_category}{direct sum}
    $$ X \cong Y \oplus Z, $$
    and $X$ lies in $\mathcal{S}$, then both $Y$ and $Z$ lie in $\mathcal{S}$.
    % Equivalently, a thick subcategory is a triangulated subcategory $\mathcal{S}$ such that whenever an object is a direct summand of an object in $\mathcal{S}$, it also belongs to $\mathcal{S}$.

    The notion of a thick subcategory of a triangulated category should not be confused for the notion of a \CrefAndHyperrefIfExist{definition:serre_subcategory_of_an_abelian_category}{thick subcategory} of an abelian category.
\end{definition}

\begin{proposition} \label{proposition:subcategory_of_homotopy_category_of_an_abelian_category_whose_cohomology_objects_belong_to_a_serre_subcategory_is_a_triangulated_subcategory}
    
    Let $\calA$ be an abelian category and let $\calA'$ be a \CrefAndHyperrefIfExist{definition:serre_subcategory_of_an_abelian_category}{Serre subcategory} of $\calA$. The full subcategories of \CrefAndHyperrefIfExist{definition:homotopy_category_of_chain_complexes_of_an_additive_category}{$K(\calA)$, $K^?(\calA)$ for $? \in \{+,-,b\}$, and $K^{?,??}(\calA)$ for $? \in \{\infty, +, -, b\}$ and $?? \in \{+,-,b,\emptyset\}$} consisting of complexes $X^\bullet$ such that \CrefAndHyperrefIfExist{definition:homology_and_cohomology_objects_for_a_chain_complex_in_an_additive_category}{$H^i(X^\bullet)$} are objects of $\calA'$ for all $i$ are a \CrefAndHyperrefIfExist{definition:triangulated_category}{triangulated subcategories} of their respective triangulated parent categories. 
\end{proposition}

\begin{proof}
    \TODO{cf. \cite[Page 272 and 6.1.2]{fu_ect}}
\end{proof}

\begin{notation}\label{notation:homotopy_categories_of_an_abelian_category_whose_homology_objects_are_in_a_serre_subcategory}
    Let $\calA$ be an abelian category and let $\calA'$ be a \CrefAndHyperrefIfExist{definition:serre_subcategory_of_an_abelian_category}{Serre subcategory} of $\calA$. We may write \hl{$K_{\calA'}(\calA)$} for the triangulated subcategory of \CrefAndHyperrefIfExist{definition:homotopy_category_of_chain_complexes_of_an_additive_category}{$K(\calA)$} consisting of complexes $X^\bullet$ such that \CrefAndHyperrefIfExist{definition:homology_and_cohomology_objects_for_a_chain_complex_in_an_additive_category}{$H^i(X^\bullet)$} are objects of $\calA'$ for all $i$. Similarly, we may write \hl{$K_{\calA'}^?(\calA)$} for $? \in \{+,-,b\}$, and \hl{$K_{\calA'}^{?,??}(\calA)$} for $? \in \{\infty, +, -, b\}$ and $?? \in \{+,-,b,\emptyset\}$ for the corresponding subcategories of \CrefAndHyperrefIfExist{definition:homotopy_category_of_chain_complexes_of_an_additive_category}{$K^?(\calA)$ and $K^{?,??}(\calA)$} respectively.
\end{notation}

\begin{notation} \label{notation:subcategory_of_derived_category_of_an_abelian_category_whose_homology_objects_belong_to_a_serre_subcategory}
    Let $\calA$ be an abelian category and let $\calA'$ be a \CrefAndHyperrefIfExist{definition:serre_subcategory_of_an_abelian_category}{Serre subcategory} of $\calA$. We may write \hl{$D_{\calA'}^?(\calA)$} for $? \in \{(\text{blank}), +,-,b\}$ for the full subcategory of \CrefAndHyperrefIfExist{definition:derived_category_of_an_abelian_category}{$D^?(\calA)$} of objects from \CrefAndHyperrefIfExist{notation:homotopy_categories_of_an_abelian_category_whose_homology_objects_are_in_a_serre_subcategory}{$K_{\calA'}(\calA)$}. These are triangulated subcategories of their respective parent categories.
\end{notation}



\section{Sheaves and sheaf cohomology}

\subsection{Sites and sheaves}


\begin{definition}[{\cite[Expos\'e I D\'efinition 4.1]{SGA4_I}}] \label{definition:sieve_on_an_object_in_a_category}
Let $C$ be a \CrefAndHyperrefIfExist{definition:category}{(large) category}. 

\begin{enumerate}
    \item A \hldef{sieve $S$ on the category $C$} is a \CrefAndHyperrefIfExist{definition:full_subcategory_of_a_category}{full subcategory} $D$ of $C$ such that for any object $U$ of $C$ there exists an object $V$ of \TODO{correctly parse the definiton}
    \item A \hldef{sieve $S$ on an object $U \in \operatorname{Ob}(C)$} is a collection of morphisms in $C$ with codomain $U$ that is closed under precomposition by any compatible morphism in $C$. In other words, $S$ is a sieve if for every $f : V \to U$ in $S$ and morphism $g : W \to V$ in $C$, the composition $f \circ g : W \to U$ is also in $S$. 

    Given a morphism $f: V \to U$ in a sieve $S$, we also say that \hldef{$f$ factors through $U$}.
\end{enumerate}
\end{definition}

% \begin{definition}[Generated sieve] \label{definition:sieve_on_an_object_of_a_category_generated_by_a_family_of_morphisms}
%     Let $\mathcal{C}$ be a (large) category, $X \in \mathcal{C}$ an object, and $S$ a \CrefAndHyperrefIfExist{definition:sieve_on_an_object_in_a_category}{sieve} on $X$. A sieve $S$ is said to be \hldef{generated} by a family of morphisms $\{f_i : U_i \to X\}_{i \in I}$ if $S$ is the smallest sieve on $X$ containing all the morphisms $f_i$, i.e., $S$ consists precisely of all morphisms $g : Y \to X$ such that $g$ factors through some $f_i$.
% \end{definition}

\begin{definition} \label{definition:sieve_on_an_object_of_a_category_generated_by_a_family_of_morphisms}
Let $\mathcal{C}$ be a \CrefAndHyperrefIfExist{definition:category}{category} and $U \in \mathcal{C}$ an object. Let $\mathcal{S} = \{f_i: U_i \to U\}_{i \in I}$ be a family of morphisms with codomain $U$. 

The \hldef{sieve generated by $\mathcal{S}$}, denoted \hl{$(\mathcal{S})$} or \hl{$\langle \mathcal{S} \rangle$}, is the smallest \CrefAndHyperrefIfExist{definition:sieve_on_an_object_in_a_category}{sieve on $U$} containing all the morphisms in $\mathcal{S}$.

Explicitly, a morphism $h: V \to U$ belongs to the generated sieve if and only if $h$ factors through some morphism in $\mathcal{S}$. That is, there exists an index $i \in I$ and a morphism $g: V \to U_i$ such that
$$ h = f_i \circ g. $$
\end{definition}

\begin{definition} \label{definition:pullback_sieve_of_an_object_in_a_category_via_a_morphism_to_the_object}
Let $C$ be a category, let $U \in \operatorname{Ob}(C)$, and let $S$ be a \CrefAndHyperrefIfExist{definition:sieve_on_an_object_in_a_category}{sieve on $U$}.
For a morphism $f : V \to U$ in $C$, the \hldef{pullback sieve} \hl{$f^*S$} (or \hldef{basechange sieve} \hl{$S \times_U V$}) on $V$ is defined by
\[
f^*S = \{ g : W \to V \mid f \circ g \in S \}.
\]
In other words, $f^*S$ consists of all morphisms into $V$ whose composite with $f$ belongs to the sieve $S$ on $U$.
\end{definition}
% \begin{definition}[Grothendieck topology] \label{definition:grothendieck_topology_on_a_category_site_covering_sieve_topologically_generating_family}
%     Let $\mathscr{U}$ be a \hyperrefIfExists{definition:grothendieck_universe}{universe}\CrefIfExists{definition:grothendieck_universe} and let $\calC$ be a \hyperrefIfExists{definition:locally_small_category}{locally small category}\CrefIfExists{definition:locally_small_category}.

%     \begin{enumerate}
%         \item \textbf{(Grothendieck Topology via Sieves)}
%         A \hldef{Grothendieck topology} $J$ on $\calC$ is an assignment to each object $U \in \calC$ of a collection $J(U)$ of \CrefAndHyperrefIfExist{definition:sieve_on_an_object_in_a_category}{sieves} on $U$, called \hldef{covering sieves}, satisfying:
%         \begin{enumerate}
%             \item (Maximality) The maximal \CrefAndHyperrefIfExist{definition:sieve_on_an_object_in_a_category}{sieve} $\{ f : V \to U \mid V \in \calC \}$ is in $J(U)$.
%             \item (Stability) If $S \in J(U)$ and $f : V \to U$ is any morphism, then the \CrefAndHyperrefIfExist{definition:pullback_sieve_of_an_object_in_a_category_via_a_morphism_to_the_object}{pullback sieve} $f^{*}S$ is in $J(V)$.
%             \item (Transitivity/Local Character) If $S$ is a sieve on $U$ and there exists a covering sieve $R \in J(U)$ such that for every morphism $f : V \to U$ in $R$, the pullback sieve $f^{*}S$ is in $J(V)$, then $S \in J(U)$.
%         \end{enumerate}

%         % \item \textbf{(Grothendieck Pretopology / Basis)}
%         % If $\calC$ admits fiber products, one can define a topology via \hldef{covering families}. A \hldef{Grothendieck pretopology} (or basis) is a collection $K(U)$ of families $\{U_i \to U\}_{i \in I}$ for each object $U$, satisfying:
%         % \begin{itemize}
%         %     \item (Isomorphism) $\{U' \xrightarrow{\sim} U\} \in K(U)$ for any isomorphism.
%         %     \item (Stability) If $\{U_i \to U\} \in K(U)$ and $V \to U$ is a morphism, then $\{U_i \times_U V \to V\} \in K(V)$.
%         %     \item (Composition) If $\{U_i \to U\} \in K(U)$ and for each $i$, $\{V_{ij} \to U_i\} \in K(U_i)$, then the composite family $\{V_{ij} \to U\} \in K(U)$.
%         % \end{itemize}
%         % Every pretopology generates a unique Grothendieck topology $J$, where $S \in J(U)$ iff $S$ contains a covering family from the pretopology.

%         \item A \hldef{site} is a pair $(\calC, J)$ consisting of a category $\calC$ and a Grothendieck topology $J$.

%         \item A family of objects $\mathcal{G} = \{G_\alpha\}$ in a site $(\calC, J)$ is called a \hldef{topologically generating family} if for every object $X \in \calC$, there exists a covering sieve $S \in J(X)$ \CrefAndHyperrefIfExist{definition:sieve_on_an_object_of_a_category_generated_by_a_family_of_morphisms}{generated by} morphisms with domains in $\mathcal{G}$. Equivalently, every object $X$ admits a cover $\{U_i \to X\}$ where each $U_i \in \mathcal{G}$.

%         \item A \hldef{$\mathscr{U}$-site} is a site whose underlying category is $\mathscr{U}$-locally small and which admits a $\mathscr{U}$-small topologically generating family.
%     \end{enumerate}
% \end{definition}

\begin{definition}[Grothendieck topology] \label{definition:grothendieck_topology_on_a_category_site_covering_sieve_topologically_generating_family}
    Let $\mathscr{U}$ be a \hyperrefIfExists{definition:grothendieck_universe}{universe}\CrefIfExists{definition:grothendieck_universe}.
    \begin{enumerate}
        % \item Let $C$ be a \hyperrefIfExists{definition:locally_small_category}{locally small category}\CrefIfExists{definition:locally_small_category}. A \hldef{Grothendieck topology on $C$} assigns to each object $U$ of $C$ a collection of families of morphisms $\{U_i \to U\}_{i \in I}$, called \hldef{coverings of $U$}, satisfying:
        % \begin{itemize}
        %     \item (Isomorphism) If $f: V \to U$ is an isomorphism in $C$, then $\{f: V \to U\}$ is a covering of $U$.
        %     \item (Stability under base change) If $\{U_i \to U\}_{i \in I}$ is a covering of $U$ and $V \to U$ is any morphism, then the family $\{ U_i \times_U V \to V \}_{i \in I}$ is a covering of $V$.
        %     \item (Transitivity) If $\{U_i \to U\}_{i \in I}$ is a covering of $U$ and for each $i$, $\{V_{ij} \to U_i\}_{j \in J_i}$ is a covering of $U_i$, then the family $\{ V_{ij} \to U \}_{i \in I,\, j \in J_i}$ is a covering of $U$.
        % \end{itemize}

        \item (See \cite[Expos\'e II, D\'efinition 1.1]{SGA4_I}) Let $\calC$ be a \CrefAndHyperrefIfExist{definition:category}{category}. A \hldef{Grothendieck topology on $\calC$} assigns to each object $U$ of $\calC$ a collection \hl{$J(U)$} of \CrefAndHyperrefIfExist{definition:sieve_on_an_object_in_a_category}{sieves} $\{U_i \to U\}_{i \in I}$, each called a \hldef{covering sieve of $U$}, satisfying:
        \begin{enumerate}
            \item (Stability under ``base change''): If $S \in J(U)$ is a covering sieve of an object $U$, and $f: V \to U$ is any morphism in $\calC$, then the \CrefAndHyperrefIfExist{definition:pullback_sieve_of_an_object_in_a_category_via_a_morphism_to_the_object}{pullback sieve} $f^* S$ is a covering sieve of $U$.
            % \item (Local character condition) If $F$ is a sieve on $U$ such that the sieve $\bigcup_...$ \TODO{}
            \item (Local character condition) If $S$ is a sieve on $U$, and if there exists a covering sieve $R \in J(U)$ such that for all $f: V \to U$ in $R$ the \CrefAndHyperrefIfExist{definition:pullback_sieve_of_an_object_in_a_category_via_a_morphism_to_the_object}{pullback sieve} $f^* S$ is in $J(V)$, then $S \in J(U)$. 
            
            \item The \CrefAndHyperrefIfExist{definition:maximal_sieve_on_an_object_in_a_category}{maximal sieve} is a covering sieve.
        \end{enumerate}


        % Equivalently, a Grothendieck topology $J$ on a category $C$ is an assignment of a collection $J(U)$ of \CrefAndHyperrefIfExist{definition:sieve_on_an_object_in_a_category}{sieves} on each object $U \in \operatorname{Ob}(C)$ such that:
        % \begin{enumerate}
        %     \item the maximal \CrefAndHyperrefIfExist{definition:sieve_on_an_object_in_a_category}{sieve} $\{ f : V \to U \mid f \in \operatorname{Mor}(C) \}$ belongs to $J(U)$,
        %     \item if $S \in J(U)$ and $f : V \to U$, then the \CrefAndHyperrefIfExist{definition:pullback_sieve_of_an_object_in_a_category_via_a_morphism_to_the_object}{pullback sieve $f^{*}S$} on $V$ belongs to $J(V)$,
        %     \item if $S$ is a sieve on $U$, and if there exists $R \in J(U)$ such that for all $f : V \to U$ in $R$ the \CrefAndHyperrefIfExist{definition:pullback_sieve_of_an_object_in_a_category_via_a_morphism_to_the_object}{pullback sieve $f^{*}S$} is in $J(V)$, then $S \in J(U)$.
        % \end{enumerate}

        Some will refer to a Grothendieck topology as simply a \hldef{topology}, not to be confused with the related, but less general, notion of a \CrefAndHyperrefIfExist{definition:topological_space}{topology on a set}.


        \item (See \cite[Expos\'e II, 1.1.5]{SGA4_I}) A \hldef{site} is a category $\calC$ equipped with a Grothendieck topology.

        When we are working with a \CrefAndHyperref{definition:basis_and_grothendieck_pretopology_for_a_grothendieck_topology_on_a_category}{Grothendieck pretopology} $K$ on a category $\calC$, we may regard $\calC$ as a site by equipping it with the \CrefAndHyperref{definition:grothendieck_topology_generated_by_a_pretopology}{Grothendieck topology generated by} $K$. 

        \item (See \cite[Expos\'e II, D\'efinition 1.2]{SGA4_I}) Let $(\calC, J)$ be a site. A family of morphisms $(U_i \to U)_{i \in I}$ is called a \hldef{covering family of $U$ (with respect to the site/topology)} or a \hldef{cover of $U$ (with respect to the site/topology)} if the \CrefAndHyperrefIfExist{definition:sieve_on_an_object_of_a_category_generated_by_a_family_of_morphisms}{sieve generated by} the family is a covering sieve of $U$. 

        \item (See \cite[Expos\'e II, D\'efinition 3.0.1]{SGA4_I}) Let $(\calC, J)$ be a \CrefAndHyperrefIfExist{definition:grothendieck_topology_on_a_category_site_covering_sieve_topologically_generating_family}{site}, where $J$ is a Grothendieck topology on $\calC$.

        A family $G$ of objects $\calC$ is called a \hldef{topologically generating family of the site/topology} or a \hldef{generating family/collection of the site/topology} if for every object $X \in \calC$, there is a covering family $\{X_\alpha \to X\}_{\alpha \in A}$ of $X$ such that every $X_\alpha$ is a member of $G$.  

        Equivalently, the Grothendieck topology $J$ is the smallest Grothendieck topology containing all covers of the $U_i$. Also equivalently, for any $S \in J(X)$, the sieve $S$ contains a covering family $\{V_i \to X\}$ such that each morphism $V_i \to X$ factors through some member of $G$. \TODO{Verify that these claimed equivalences are indeed equivalences}
        
        % A family of objects $\{U_i\}_{i \in I}$ in $\calC$ is called a \hldef{topologically generating family} if for every object $X \in \calC$ and every covering sieve $S \in J(X)$, the sieve $S$ is \CrefAndHyperrefIfExist{definition:sieve_on_an_object_of_a_category_generated_by_a_family_of_morphisms}{generated by} pullbacks of covering families from the family $\{U_i\}$.

        % More precisely, this means that for any $S \in J(X)$, the sieve $S$ contains a covering family $\{V_j \to X\}$ such that each morphism $V_j \to X$ factors through some $U_i$, and the covering families of the $U_i$ generate the topology $J$. 
        % Equivalently, the Grothendieck topology $J$ is the smallest Grothendieck topology containing all coverings of the $U_i$.

        % When one speaks of a \hldef{generating family/collection} of a site, one usually refers to the above notion of a topologically generating family.

        \item (See \cite[Expos\'e II, D\'efinition 3.0.2]{SGA4_I}) A \hldef {$\mathscr{U}$-site} is a site whose underlying category $\calC$ is \hyperrefIfExists{definition:locally_small_category}{$\mathscr{U}$-locally small}\CrefIfExists{definition:locally_small_category} and which has a $\mathscr{U}$-small topologically generating family. A $\mathscr{U}$-site is called \hldef{$\mathscr{U}$-small} if its underlying category is $\mathscr{U}$-small. Similarly, a \hldef{small site} is a site whose underlying category is a set and a \hldef{locally small site} is a site whose underlying category is \CrefAndHyperrefIfExist{definition:locally_small_category}{locally small}.
    \end{enumerate}
\end{definition}

\begin{definition} \label{definition:essentially_small_site}
    An \hldef{essentially small site} is a \CrefAndHyperrefIfExist{definition:grothendieck_topology_on_a_category_site_covering_sieve_topologically_generating_family}{site} whose underlying category is \CrefAndHyperrefIfExist{definition:essentially_small_category}{essentially small}.
\end{definition}
\begin{definition}[Presheaf on a category] \label{definition:presheaf_on_a_category}
    Let $C$ and $\mathcal{A}$ be \hyperrefIfExists{definition:category}{(large) categories}\CrefIfExists{definition:category}. 
    \begin{enumerate}
        \item A \hldef{presheaf $\mathcal{F}$ on $C$ with values in $\mathcal{A}$} is a functor
        \[
        \mathcal{F}: C^{\mathrm{op}} \to \mathcal{A}.
        \]
        In other words, a presheaf $\calF$ on $C$ with values in $\calA$ is simply a \CrefAndHyperrefIfExist{definition:functor_between_categories}{contravariant functor} from $C$ to $\calA$. 
        Explicitly, for every object $U$ in $C$, one has an object $\mathcal{F}(U)$ in $\mathcal{A}$ (called the \hldef{$U$-valued sections/sections evaluated at $U$ of $\calF$}\TextIfExists{definition:sections_of_a_presheaf_on_a_category_valued_in_a_category}{, cf. \Cref{definition:sections_of_a_presheaf_on_a_category_valued_in_a_category}}), and for every morphism $f: V \to U$ in $C$, one has a morphism (called the \hldef{restriction map})
        \[
        \mathcal{F}(f): \mathcal{F}(U) \to \mathcal{F}(V)
        \]
        in $\mathcal{A}$, such that for all composable morphisms $W \xrightarrow{g} V \xrightarrow{f} U$ in $C$, the following diagram in $\mathcal{A}$ commutes:
        \[
        \begin{tikzcd}
        \mathcal{F}(U) \arrow[r, "\mathcal{F}(f)"] \arrow[rr, bend left, "\mathcal{F}(f \circ g)"] & \mathcal{F}(V) \arrow[r, "\mathcal{F}(g)"] & \mathcal{F}(W)
        \end{tikzcd}
        \]
        That is,
        \[
        \mathcal{F}(g) \circ \mathcal{F}(f) = \mathcal{F}(f \circ g),
        \]
        and for every object $U$ in $C$, $\mathcal{F}(\mathrm{id}_U) = \mathrm{id}_{\mathcal{F}(U)}$.


        \item 
        Let $\mathcal{F},\mathcal{G}: C^{\mathrm{op}} \to \mathcal{A}$ be two presheaves on $C$ with values in $\mathcal{A}$. A \hldef{morphism of presheaves}
        \[
        \varphi: \mathcal{F} \to \mathcal{G}
        \]
        is a \hyperrefIfExists{definition:natural_transformation_between_functors_between_categories}{natural transformation of functors}\CrefIfExists{definition:natural_transformation_between_functors_between_categories}: for each object $U$ of $C$, one has a morphism
        \[
        \varphi_U: \mathcal{F}(U) \to \mathcal{G}(U)
        \]
        in $\mathcal{A}$, such that for every morphism $f: V \to U$ in $C$, the diagram
        \[
        \begin{tikzcd}
        \mathcal{F}(U) \arrow[r, "\mathcal{F}(f)"] \arrow[d, "\varphi_U"'] & \mathcal{F}(V) \arrow[d, "\varphi_V"] \\
        \mathcal{G}(U) \arrow[r, "\mathcal{G}(f)"'] & \mathcal{G}(V)
        \end{tikzcd}
        \]
        commutes, i.e.,
        \[
        \varphi_V \circ \mathcal{F}(f) = \mathcal{G}(f) \circ \varphi_U
        \]
        for all objects and morphisms in $C$.

        \item Given a \hyperrefIfExists{definition:grothendieck_universe}{universe}\CrefIfExists{definition:grothendieck_universe} $U$, a \hldef{$U$-presheaf on $\calC$} typically refers to a presheaf of $U$-sets on $C$.

        \item The \hldef{presheaf category/category of $\calA$-valued presheaves on $\calC$} is the (large) category whose objects are the presheaves on $C$ with values in $\calA$ and whose morphisms are the presheaf morphisms. Common notations for the presheaf category include, but are not limited to: \hl{$\calA^{\calC^{\op}}$}, \hl{$\PreShv(\calC, \calA)$}, \hl{$[\calC^{\op}, \calA]$}. If the value category $\calA$ is clear from context, then notations such as \hl{$\PreShv(\calC)$} are also common. \TextIfExists{definition:diagram_in_a_category_indexed_by_a_small_category}{Note that the presheaf category $\PreShv(\calC, \calA)$ is equivalent to the \CrefAndHyperrefIfExist{definition:diagram_in_a_category_indexed_by_a_small_category}{category of functors} $\calC^{\op} \to \calA$ and hence notations for the functor categories are applicable as notations for presheaf categories.}

    \end{enumerate}
\end{definition}

\begin{definition}[Sheaf on a site] \label{definition:sheaf_on_a_site}

% \TODO{There might be some need to say that $\calA$ is a category for which sheaves on the site ``can be defined''}
% \TODO{go through statements using the notion of sheaves and make sure that the value categories have small products and that the categories have small generating families.}

Let $(\calC, J)$ be a \CrefAndHyperrefIfExist{definition:grothendieck_topology_on_a_category_site_covering_sieve_topologically_generating_family}{site}. Let $\calA$ be a \CrefAndHyperrefIfExist{definition:category}{(large) category}.
\begin{enumerate}
    \item A \CrefAndHyperrefIfExist{definition:presheaf_on_a_category}{presheaf} $\calF: \calC^{\op} \to \calA$\CrefIfExists{definition:opposite_category_of_a_category} is called a \hldef{sheaf on the site $(\calC, J)$ valued in $\calA$} if, for every object $U$ of $\calC$ and every \CrefAndHyperrefIfExist{definition:grothendieck_topology_on_a_category_site_covering_sieve_topologically_generating_family}{covering sieve} $S \in J(U)$, the \CrefAndHyperrefIfExist{definition:limit_and_colimit_of_a_diagram_in_a_category}{limit}
    $$\varprojlim_{(V \to U) \in (\calD_S)^{\op}} \calF|_{\calD_S}(V),$$
    exists and the canonical natural morphism
    $$\calF(U) \to \varprojlim_{(V \to U) \in (\calD_S)^{\op}} \calF|_{\calD_S}(V)$$
    is an isomorphism. Here, $\calD_S \hookrightarrow \calC/U$\CrefIfExists{definition:category_of_objects_over_under_a_fixed_object_in_a_category} is the full \CrefAndHyperrefIfExist{definition:downward_upward_closed_subcategory_of_a_category}{downward-closed subcategory} such that $\operatorname{Ob}(\calD_S) = \{(f: V \to U): f \in S(V)\}$,

    In particular, when we are working with a \CrefAndHyperref{definition:basis_and_grothendieck_pretopology_for_a_grothendieck_topology_on_a_category}{Grothendieck pretopology} $K$ on a category $\calC$, we may speak of sheaves on the site whose Grothendieck topology is the \CrefAndHyperref{definition:grothendieck_topology_generated_by_a_pretopology}{one generated by} $K$.

    \item Given sheaves $\calF, \calG: \calC^{\op} \to \calA$ on the site $(\calC, J)$, a \hldef{morphism between the sheaves} is a \CrefAndHyperrefIfExist{definition:presheaf_on_a_category}{morphism} between $\calF$ and $\calG$ as presheaves.


    \item Let $U$ be a \hyperrefIfExists{definition:grothendieck_universe}{universe}\CrefIfExists{definition:grothendieck_universe}. A \hldef{$U$-sheaf} typically refers to a $U$-presheaf that is a sheaf for a $U$-site. In other words, a $U$-sheaf is a sheaf on a site whose underlying category is \hyperrefIfExists{definition:locally_small_category}{$U$-locally small}\CrefIfExists{definition:locally_small_category} and which has a $U$-small topologically generating family such that the sheaf is valued in $U$-sets.

    \item The \hldef{sheaf category/category of $\calA$-valued sheaves on $\calC$} is the (large) category defined as the full subcategory of $\PreShv(\calC, \calA)$ whose objects are the sheaves on $\calC$ with values in $\calA$. Common notations for the sheaf category include \hl{$\Shv(\calC, \calA)$}, \hl{$\Shv(\calC, J, \calA)$}, \hl{$\Sh(\calC, \calA)$}, \hl{$\Sh(\calC, J, \calA)$}. If the value category $\calA$ is clear from context, then notations such as \hl{$\Shv(\calC)$}, \hl{$\Shv(\calC, J)$}, \hl{$\Sh(\calC)$}, \hl{$\Sh(\calC, J)$} are also common.

\end{enumerate}

% Let $(\calC, J)$ be a \CrefAndHyperrefIfExist{definition:grothendieck_topology_on_a_category_site_covering_sieve_topologically_generating_family}{site} with a small \CrefAndHyperrefIfExist{definition:grothendieck_topology_on_a_category_site_covering_sieve_topologically_generating_family}{topological generating family} (or a $U$-small topologically generating family if a \CrefAndHyperrefIfExist{definition:grothendieck_universe}{universe} $U$ is available) and let $\mathcal{A}$ be a \CrefAndHyperrefIfExist{definition:category}{(large) category} that has all \CrefAndHyperrefIfExist{definition:locally_small_category}{small} \CrefAndHyperrefIfExist{definition:product_and_coproduct_of_objects_in_a_category}{products} (Some common examples of categories that have small products and thus play the role of $\calA$ here include $\mathcal{A} = \text{Set}$, $\text{Ab}$, $R\mathbf{-mod}$ for a fixed ring $R$, $\text{rings}$). 
% \begin{enumerate}

%     \item For any object $U$ of $\calC$ and every covering $\{U_i \to U\}_{i \in I}$ in $J$, note that there are morphisms $U_i \times_U U_j \to U_i$ for every $i,j \in I$. 
%     % Consider the subcategory of $C$ consisting of the objects $U_i$ and $U_i \times_U U_j$, together with these morphisms.
%     Given any presheaf $\calF: C^{\op} \to \calA$, there is a \CrefAndHyperrefIfExist{definition:diagram_in_a_category_indexed_by_a_small_category}{diagram} in $\calA$ consisting of objects $\calF(U_i)$ and $\calF(U_i \times_U U_j)$ and morphisms $\calF(U_i) \to \calF(U_i \times_U U_j)$. The presheaf $\calF$ is called a \hldef{sheaf on the site $(\calC, J)$ valued in $\calA$} if, for every object $U$ of $\calC$ and every covering $\{U_i \to U\}_{i \in I}$ in $J$, the sections object $\calF(U)$ is the \CrefAndHyperrefIfExist{definition:limit_and_colimit_of_a_diagram_in_a_category}{limit} of the aforementioned diagram:
    
%     % A \hyperrefIfExists{definition:presheaf_on_a_category}{presheaf}\CrefIfExists{definition:presheaf_on_a_category} $\mathcal{F}: C^{\mathrm{op}} \to \mathcal{A}$ is a \hldef{sheaf on the site $(\calC,J)$ valued in $\calA$} if, for every object $U$ of $\calC$ and every covering $\{U_i \to U\}_{i \in I}$ in $J$, the sections object $\calF(U)$ is the \CrefAndHyperrefIfExist{definition:limit_and_colimit_of_a_diagram_in_a_category}{limit} of the sections objects $\calF(U_i)$:
%     % $$\calF(U) \cong \varprojlim_{}$$
    
%     % following sequence is an \CrefAndHyperrefIfExist{definition:equalizer_and_coequalizer_of_morphisms_in_a_category}{equalizer} in $\mathcal{A}$:
%     % \[
%     % \mathcal{F}(U) \to \prod_{i} \mathcal{F}(U_i) \rightrightarrows \prod_{i, j} \mathcal{F}(U_i \times_U U_j)
%     % \]
%     % where the first map sends $s$ to $(\mathcal{F}(U_i \to U)(s))_i$ and the arrows to $(\mathcal{F}(U_i \times_U U_j \to U_i)(s_i))_{i,j}$ and $(\mathcal{F}(U_i \times_U U_j \to U_j)(s_j))_{i,j}$, respectively.

%     % \item A \hyperrefIfExists{definition:presheaf_on_a_category}{presheaf}\CrefIfExists{definition:presheaf_on_a_category} $\mathcal{F}: C^{\mathrm{op}} \to \mathcal{A}$ is a \hldef{sheaf on the site $(\calC,J)$ valued in $\calA$} if, for every object $U$ of $\calC$ and every covering $\{U_i \to U\}_{i \in I}$ in $J$, the following sequence is an \CrefAndHyperrefIfExist{definition:equalizer_and_coequalizer_of_morphisms_in_a_category}{equalizer} in $\mathcal{A}$:
%     % \[
%     % \mathcal{F}(U) \to \prod_{i} \mathcal{F}(U_i) \rightrightarrows \prod_{i, j} \mathcal{F}(U_i \times_U U_j)
%     % \]
%     % where the first map sends $s$ to $(\mathcal{F}(U_i \to U)(s))_i$ and the arrows to $(\mathcal{F}(U_i \times_U U_j \to U_i)(s_i))_{i,j}$ and $(\mathcal{F}(U_i \times_U U_j \to U_j)(s_j))_{i,j}$, respectively.

%     \item A \hldef{morphism of sheaves} $\calF: \calC^{\op} \to \calA$ is a \hyperrefIfExists{definition:presheaf_on_a_category}{morphism as presheaves}\CrefIfExists{definition:presheaf_on_a_category}. 


%     \item Let $U$ be a \hyperrefIfExists{definition:grothendieck_universe}{universe}\CrefIfExists{definition:grothendieck_universe}. A \hldef{$U$-sheaf} typically refers to a $U$-presheaf that is a sheaf for a $U$-site. In other words, a $U$-sheaf is a sheaf on a site whose underlying category is \hyperrefIfExists{definition:locally_small_category}{$U$-locally small}\CrefIfExists{definition:locally_small_category} and which has a $U$-small topologically generating family such that the sheaf is valued in $U$-sets.

%     \item The \hldef{sheaf category/category of $\calA$-valued sheaves on $\calC$} is the (large) category defined as the full subcategory of $\PreShv(\calC, \calA)$ whose objects are the sheaves on $C$ with values in $\calA$. Common notations for the sheaf category include \hl{$\Shv(\calC, \calA)$}, \hl{$\Shv(\calC, J, \calA)$}, \hl{$\Sh(\calC, \calA)$}, \hl{$\Sh(\calC, J, \calA)$}. If the value category $\calA$ is clear from context, then notations such as \hl{$\Shv(\calC)$}, \hl{$\Shv(\calC, J)$}, \hl{$\Sh(\calC)$}, \hl{$\Sh(\calC, J)$} are also common.

% \end{enumerate}
\end{definition}

\begin{definition}[Sheaf on a topological space] \label{definition:sheaf_on_a_topological_space_valued_in_a_category_with_a_terminal_object}

    Let $X$ be a \CrefAndHyperrefIfExist{definition:topological_space}{topological space}, let $\calD$ be a \CrefAndHyperrefIfExist{definition:category}{category} with a \CrefAndHyperrefIfExist{lemma:initial_or_final_object_in_a_category_that_is_also_in_a_full_subcategory_is_initial_or_final_in_the_subcategory}{terminal object}, and let $\mathcal{F}$ be a \CrefAndHyperrefIfExist{definition:presheaf_on_a_topological_space}{presheaf valued in $\calD$ on $X$}.  
    Then $\mathcal{F}$ is a \hldef{sheaf} if it satisfies the following additional condition (known as the \hldef{sheaf axioms}):

    For every open set $U \subseteq X$ and every \CrefAndHyperrefIfExist{definition:open_covering_of_a_topological_space}{open cover} $\{U_i\}_{i \in I}$ of $U$, let $\calJ$ be the \CrefAndHyperrefIfExist{definition:diagram_in_a_category_indexed_by_a_small_category}{diagram} in the \CrefAndHyperrefIfExist{definition:category_of_opens_of_a_topological_space}{category of opens} of $U$ consisting of the inclusions $U_i \cap U_j \hookrightarrow U_i$ for all $i,j \in I$. Then $\calF$ is a sheaf if the \CrefAndHyperrefIfExist{definition:limit_and_colimit_of_a_diagram_in_a_category}{limit} of the diagram $\calF \circ \calJ$ exists in $\calD$ and the natural morphism
    $$\calF(U) \to \lim_{j \in \calJ} \calF(j)$$
    is an isomorphism. More precisely, $\calJ:J \to \operatorname{Open}(U)$ should be the functor whose index category $J$ consists of
    \begin{enumerate}
        \item An object $i$ for every $i \in I$ and an object $(i,j)$ for every pair $i,j \in I$,
        \item Morphisms $p_1: (i,j) \to i$ and $p_2: (i,j) \to j$ for every pair $i,j \in I$
    \end{enumerate}
    and which sends the objects and morphisms as follows:
    \begin{enumerate}
        \item $\calJ(i) = U_i$
        \item $\calJ(i,j) = U_i \cap U_j$
        \item $\calJ(p_1): U_i \cap U_j \hookrightarrow U_i$
        \item $\calJ(p_2): U_i \cap U_j \hookrightarrow U_j$.
    \end{enumerate}
    In particular, taking $U = \emptyset$ and taking the empty open cover of the empty set, $\calF(\emptyset)$ must be the \CrefAndHyperrefIfExist{lemma:initial_or_final_object_in_a_category_that_is_also_in_a_full_subcategory_is_initial_or_final_in_the_subcategory}{terminal object} of $\calD$

    In the case that $\calD$ admits all \CrefAndHyperrefIfExist{definition:small_and_finite_limits_and_colimits_in_a_category}{small limits}, the sheaf condition is equivalent to the following: For every open set $U \subset X$ and every open cover $\{U_i\}_{i \in I}$ of $U$, the following \CrefAndHyperrefIfExist{definition:equalizer_and_coequalizer_of_morphisms_in_a_category}{equalizer diagram is exact}:
    $$\calF(U) \to \prod_{i \in I} \calF(U_i) \rightrightarrows \prod_{i,j \in I} \calF(U_i \cap U_j).$$
    Here, the morphism $\calF(U) \to \prod_{i \in I} \calF(U_i)$ and the two morphisms $\prod_{i \in I} \calF(U_i) \rightrightarrows \prod_{i,j \in I} \calF(U_i \cap U_j)$ are induced by the \CrefAndHyperrefIfExist{definition:presheaf_on_a_topological_space}{restriction maps} $\calF(U) \to \calF(U_i)$ and $\calF(U_i) \to \calF(U_i \cap U_j)$.

    In the case that $\calD$ is some \CrefAndHyperrefIfExist{definition:subcategory_of_a_category}{subcategory} of the \CrefAndHyperrefIfExist{definition:category_of_sets}{category of sets}, the sheaf condition is equivalent to the following: For every open set $U \subseteq X$ and every \CrefAndHyperrefIfExist{definition:open_covering_of_a_topological_space}{open cover} $\{U_i\}_{i \in I}$ of $U$, 
    \begin{itemize}
        \item (Locality) If $s, t \in \mathcal{F}(U)$ are such that $s|_{U_i} = t|_{U_i}$ for all $i$, then $s = t$.
        \item (Gluing) If for each $i$ there is $s_i \in \mathcal{F}(U_i)$ such that for all $i,j$ one has $s_i|_{U_i \cap U_j} = s_j|_{U_i \cap U_j}$, then there exists a unique $s \in \mathcal{F}(U)$ such that $s|_{U_i} = s_i$ for all $i$.
    \end{itemize}

    \TextIfExists{definition:sheaf_on_a_site}{
        Equivalently, a sheaf on a topological space $X$ may be defined as a \CrefAndHyperrefIfExist{definition:sheaf_on_a_site}{sheaf on} the \CrefAndHyperrefIfExist{definition:grothendieck_topology_on_a_category_site_covering_sieve_topologically_generating_family}{site} \CrefAndHyperrefIfExist{definition:site_of_opens_on_a_topological_space}{of opens on $X$}. 
        % whose objects are open subsets $U$ of $X$ and whose morphisms are inclusions $U_1 \hookrightarrow U_2$ of open subsets of $X$, and whose Grothendieck topology is the in which a cover of an open subset $U \subseteq X$ is given by a collection $\{U_i \to U\}_{i \in I}$ in which $\bigcup U_i = U$. 
    }

\end{definition}


\begin{definition} \label{definition:categories_of_presheaves_and_sheaves_on_a_topological_space_valued_in_a_category}
    \TODO{Move these notations to the defintions of presheaves and sheaves on topological spaces}
    Let $X$ be a \CrefAndHyperrefIfExist{definition:topological_space}{topological space}, and let $\calD$ be a \CrefAndHyperrefIfExist{definition:category}{category} with a \CrefAndHyperrefIfExist{lemma:initial_or_final_object_in_a_category_that_is_also_in_a_full_subcategory_is_initial_or_final_in_the_subcategory}{terminal object}.

    The \CrefAndHyperrefIfExist{definition:presheaf_on_a_topological_space}{presheaves on $X$ valued in $\calD$}, along with the \CrefAndHyperrefIfExist{definition:morphism_of_presheaves_on_a_topological_space}{morphisms} thereof, form a (in general large) \CrefAndHyperrefIfExist{definition:category}{category} often denoted by notations such as \hl{$\PreShv(X, \calD)$} \TODO{include more notations} (or \hl{$\PreShv(X)$} if the category $\calD$ is clear). If $\calD$ is \CrefAndHyperrefIfExist{definition:locally_small_category}{locally small}, then so is $\PreShv(X, \calD)$.


    Similarly, the \CrefAndHyperrefIfExist{definition:sheaf_on_a_topological_space_valued_in_a_category_with_a_terminal_object}{sheaves on $X$ valued in $\calD$}, along with the \CrefAndHyperrefIfExist{definition:morphism_of_presheaves_on_a_topological_space}{morphisms} thereof, form a (in general large) \CrefAndHyperrefIfExist{definition:category}{category} often denoted by notations such as \hl{$\Shv(X, \calD)$} \TODO{include more notations} (or \hl{$\Shv(X)$} if the category $\calD$ is clear). The category $\Shv(X, \calD)$ is a \CrefAndHyperrefIfExist{definition:full_subcategory_of_a_category}{full subcategory} of $\PreShv(X, \calD)$.
    
    \TextIfExists{definition:sheaf_on_a_site}{
    Equivalently, the categories of presheaves and sheaves are the categories $\PreShv(\mathbf{Open}(X), \calD)$ and $\Shv(\mathbf{Open}(X), \calD)$ of \CrefAndHyperrefIfExist{definition:presheaf_on_a_category}{presheaves} and \CrefAndHyperrefIfExist{definition:sheaf_on_a_site}{sheaves} where \CrefAndHyperrefIfExist{definition:category_of_opens_of_a_topological_space}{$\mathbf{Open}(X)$} is the category of open subsets of $X$ equipped with its \CrefAndHyperrefIfExist{definition:site_of_opens_on_a_topological_space}{usual} \CrefAndHyperrefIfExist{definition:basis_and_grothendieck_pretopology_for_a_grothendieck_topology_on_a_category}{Grothendieck pretopology}.
    }

\end{definition}


\begin{definition} \label{definition:continuous_functor_of_sites}
Let $(\calC,J)$ and $(\calD,K)$ be \CrefAndHyperrefIfExist{definition:grothendieck_topology_on_a_category_site_covering_sieve_topologically_generating_family}{sites}. 

A functor $u : \calC \to \calD$ is said to be a \hldef{continuous functor of sites} if, for every object $U \in \operatorname{Ob}(\calD)$ and every \CrefAndHyperrefIfExist{definition:grothendieck_topology_on_a_category_site_covering_sieve_topologically_generating_family}{covering sieve} $S \in K(U)$, the \CrefAndHyperrefIfExist{definition:pullback_sieve_of_an_object_in_a_category_via_a_morphism_to_the_object}{pullback sieve $u^*S$} belongs to $J(V)$ for all $V \in \calC$ with a morphism $u(V) \to U$ in $\calD$.

Equivalently, $u$ is continuous if for every \CrefAndHyperrefIfExist{definition:sheaf_on_a_site}{sheaf} of sets $F$ on $\calD$, the \CrefAndHyperrefIfExist{definition:presheaf_on_a_category}{presheaf} $\calC^{\op} \to \Sets, X \mapsto F(u(X))$ is a sheaf on $\calC$. 
\TODO{show these are equivalent}
\TODO{define morphism of sites and recheck ref's to this definition}

% A \hldef{morphism of sites} $f: (\calD, K) \to (\calC, J)$ 

% Synonymously, we call a continuous functor $u: C \to D$ a \hldef{morphism of sites}.
\end{definition}
% \begin{definition} \label{definition:inverse_image_of_a_sheaf_under_a_continuous_functor_of_sites_or_a_site_morphism}
% Let $(\calC,J)$ and $(\calD,K)$ be \CrefAndHyperrefIfExist{definition:grothendieck_topology_on_a_category_site_covering_sieve_topologically_generating_family}{sites}  with small \CrefAndHyperrefIfExist{definition:grothendieck_topology_on_a_category_site_covering_sieve_topologically_generating_family}{topological generating families} (or $U$-small topologically generating families if a \CrefAndHyperrefIfExist{definition:grothendieck_universe}{universe} $U$ is available), and let $u : \calC \to \calD$ be a \CrefAndHyperrefIfExist{definition:continuous_functor_of_sites}{continuous functor of sites}. Let $\mathcal{A}$ be a (large) category which has all small (or $U$-small) \CrefAndHyperrefIfExist{definition:product_and_coproduct_of_objects_in_a_category}{products}. For any \CrefAndHyperrefIfExist{definition:sheaf_on_a_site}{sheaf} 
% \[
% \mathcal{G} \in \operatorname{Sh}(\calC,J;\mathcal{A}),
% \]
% the \hldef{direct image/pushforward sheaf of $\mathcal{G}$ under $u$} is defined by
% $$\hlin{u_*\mathcal{G} : \calD^{\mathrm{op}} \to \mathcal{A}, \quad V \mapsto \varprojlim_{(u \downarrow V)^{\op}} \mathcal{G}(U),}$$
% where the \CrefAndHyperrefIfExist{definition:projective_and_inductive_limits_in_categories}{limit} is taken over the \CrefAndHyperrefIfExist{definition:opposite_category_of_a_category}{opposite} of the \CrefAndHyperrefIfExist{definition:comma_category_of_two_functors_to_a_category}{comma category $(u \downarrow V)$} of whose objects are pairs $(U, u(U) \to V)$ with $U \in \calC$ and $u(U) \to V$ a moprhism in $\calD$. 

% The assignment $\mathcal{G} \mapsto u_*\mathcal{G}$ defines the \hldef{direct image functor}
% $$\hlin{u_* : \operatorname{Sh}(\calC,J;\mathcal{A}) \to \operatorname{Sh}(\calD,K;\mathcal{A})}.$$

% If $u$ is the functor underlying a \CrefAndHyperrefIfExist{definition:morphism_of_sites}{site morphism} $f: (\calD, K) \to (\calC, J)$, we may alternatively denote $u_* \calG$ by \hl{$f^* \calG$} and call it the \hldef{inverse image/pullback of $\calG$ under $f$}; the assignment $\calG \mapsto f^* \calG$ is then the \hldef{inverse image/pullback functor}.
% $$\hlin{f^* : \operatorname{Sh}(\calC,J;\mathcal{A}) \to \operatorname{Sh}(D,K;\mathcal{A}).}$$

% We further note that $u_*$ is ``categorical'' notation whereas $f^*$ is ``geometric'' notation; loosely speaking, given a morphism $f: X \to Y$ of topological spaces or schemes, 
% \begin{itemize}
%     \item we may have a continuous functor $u: \mathbf{C}(Y) \to \mathbf{C}(X)$ where $\mathbf{C}(X), \mathbf{C}(Y)$ are appropriate sites induced by $X$ and $Y$ respectively,
%     \item $u$ may underlie a site morphism $f: \mathbf{C}(X) \to \mathbf{C}(Y)$ roughly given by pullbacks under the morphism $f: X \to Y$, and
%     \item given a sheaf $\calG$ on $\mathbf{C}(Y)$, we may speak of its direct image $f^* \calG$ on $\mathbf{C}(X)$.
% \end{itemize}

% \end{definition}

\begin{definition} \label{definition:inverse_image_of_a_sheaf_under_a_continuous_functor_of_sites_or_a_site_morphism}
    % \begin{definition} \label{definition:inverse_image_of_a_sheaf_on_a_site_under_a_continuous_functor_of_sites}
Let $(\calC,J)$ and $(\calD,K)$ be \CrefAndHyperrefIfExist{definition:grothendieck_topology_on_a_category_site_covering_sieve_topologically_generating_family}{sites} with small \CrefAndHyperrefIfExist{definition:grothendieck_topology_on_a_category_site_covering_sieve_topologically_generating_family}{topological generating families}, and let $u : \calC \to \calD$ be a \CrefAndHyperrefIfExist{definition:continuous_functor_of_sites}{continuous functor of sites}. Let $\mathcal{A}$ be a (large) category such that the \CrefAndHyperrefIfExist{definition:presheaf_on_a_category}{presheaf category} $\operatorname{PreSh}(\calD,K;\mathcal{A})$ has \CrefAndHyperrefIfExist{definition:sheafification_functor_on_a_site}{sheafification}.

% Let $\mathcal{A}$ be a category admitting all small colimits and finite limits. 
For any \CrefAndHyperrefIfExist{definition:sheaf_on_a_site}{sheaf} 
\[
\mathcal{G} \in \operatorname{Sh}(\calC,J;\mathcal{A}),
\]
the \hldef{inverse image/pullback sheaf of $\mathcal{G}$ under $u$} is defined, assuming that all colimits below exist, as:
$$\hlin{u_s \mathcal{G} : \calD^{\mathrm{op}} \to \mathcal{A}, \quad V \mapsto a \left( \varinjlim_{(V \downarrow u)} \mathcal{G}(U) \right),}$$
where $a$ is the \CrefAndHyperrefIfExist{definition:definition:sheafification_functor_on_a_site}{sheafification} functor of presheaves and the \CrefAndHyperrefIfExist{definition:limit_and_colimit_of_a_diagram_in_a_category}{colimit} is taken over the \CrefAndHyperrefIfExist{definition:comma_category_of_two_functors_to_a_category}{comma category $(V \downarrow u)$} of pairs $(U, V \to u(U))$ with $U \in \calC$.

The assignment $\mathcal{G} \mapsto u_s\mathcal{G}$ defines the \hldef{inverse image/pullback functor}
$$\hlin{u_s : \operatorname{Sh}(\calC,J;\mathcal{A}) \to \operatorname{Sh}(\calD,K;\mathcal{A})}.$$

If $u$ is the functor underlying a \CrefAndHyperrefIfExist{definition:morphism_of_sites}{site morphism} $f: (\calD, K) \to (\calC, J)$, we may alternatively denote $u_s \calG$ by \hl{$f^{*} \calG$} (or sometimes by \hl{$f^{-1} \calG$}) and call it the \hldef{inverse image/pullback of $\calG$ under $f$}.


Note that while the continuous functor $u$ and the site morphism $f$ point in opposite directions, the identification $f^* := u_s$ ensures that $f^*$ corresponds to the standard geometric pullback. In the case of topological spaces, this recovers the usual construction involving colimits over open neighborhoods to obtain stalks followed by sheafification.

\end{definition}
% \end{definition}
\begin{definition} \label{definition:direct_image_of_a_sheaf_on_a_site_under_a_continuous_functor_of_sites_or_a_site_morphism}
Let $(\calC,J)$ and $(\calD,K)$ be \CrefAndHyperrefIfExist{definition:grothendieck_topology_on_a_category_site_covering_sieve_topologically_generating_family}{sites}  with small \CrefAndHyperrefIfExist{definition:grothendieck_topology_on_a_category_site_covering_sieve_topologically_generating_family}{topological generating families} (or $U$-small topologically generating families if a \CrefAndHyperrefIfExist{definition:grothendieck_universe}{universe} $U$ is available), and let $u : \calC \to \calD$ be a \CrefAndHyperrefIfExist{definition:continuous_functor_of_sites}{continuous functor of sites}. 
%Let $\mathcal{A}$ be a (large) category which has all small (or $U$-small) \CrefAndHyperrefIfExist{definition:product_and_coproduct_of_objects_in_a_category}{products}.

For any \CrefAndHyperrefIfExist{definition:sheaf_on_a_site}{sheaf} 
\[
\mathcal{F} \in \operatorname{Sh}(\calD,K;\mathcal{A}),
\]
Define the \hldef{pushforward/direct image sheaf} \hl{$u^s \calF$} by 
$$\hlin{u^s \mathcal{F} := \mathcal{F} \circ u : \calC^{\mathrm{op}} \to \mathcal{A}.}$$
Because $u$ is continuous, $u^s\mathcal{F}$ is a sheaf on $(\calC,J)$ valued in $\mathcal{A}$. The assignment $\mathcal{F} \mapsto u^s\mathcal{F}$ defines the \hldef{direct image/pushforward functor}
$$\hlin{u^s : \operatorname{Sh}(\calD,K;\mathcal{A}) \to \operatorname{Sh}(\calC,J;\mathcal{A}).}$$

If $u$ is the functor underlying a \CrefAndHyperrefIfExist{definition:morphism_of_sites}{site morphism} $f: (\calD, K) \to (\calC, J)$, we may alternatively denote $u^s \calF$ by \hl{$f_* \calF$} and call it the \hldef{direct image/pushforward of $\calF$ under $f$}; the assignment $\calF \mapsto f_* \calF$ is then the \hldef{direct image/pushforward functor}.
$$\hlin{f_* : \operatorname{Sh}(\calD,K;\mathcal{A}) \to \operatorname{Sh}(\calC,J;\mathcal{A}).}$$

Note that while the continuous functor $u$ and the site morphism $f$ point in opposite directions, the definition $f_* := u^s$ ensures that $f_*$ corresponds to the standard geometric pushforward used in topology and algebraic geometry.



% We further note that $u^*$ is ``categorical'' notation whereas $f_*$ is ``geometric'' notation; loosely speaking, given a morphism $f: X \to Y$ of topological spaces or schemes, 
% \begin{itemize}
%     \item we may have a continuous functor $u: \mathbf{C}(Y) \to \mathbf{C}(X)$ where $\mathbf{C}(X), \mathbf{C}(Y)$ are appropriate sites induced by $X$ and $Y$ respectively,
%     \item $u$ may underlie a site morphism $f: \mathbf{C}(X) \to \mathbf{C}(Y)$ roughly given by pullbacks under the morphism $f: X \to Y$, and
%     \item given a sheaf $\calF$ on $\mathbf{C}(X)$, we may speak of its direct image $f_* \calF$ on $\mathbf{C}(Y)$.
% \end{itemize}
\end{definition}
\begin{theorem} \label{theorem:adjunction_between_inverse_images_and_direct_images_of_sheaves_under_continuous_functor_of_sites}
    \TODO{It is likely that some more restrictions are needed; e.g. must $C$ and $D$ be small and $\calA$ locally small to ensure that we can talk about hom's between sheaves?}
Let $(C,J)$ and $(D,K)$ be \CrefAndHyperrefIfExist{definition:grothendieck_topology_on_a_category_site_covering_sieve_topologically_generating_family}{sites}, and let $u : C \to D$ be a \CrefAndHyperrefIfExist{definition:continuous_functor_of_sites}{continuous functor} of sites. Let $\mathcal{A}$ be a (large) category such that the \CrefAndHyperrefIfExist{definition:presheaf_on_a_category}{presheaf category} $\operatorname{PreSh}(\calD,K;\mathcal{A})$ has \CrefAndHyperrefIfExist{definition:sheafification_functor_on_a_site}{sheafification}.


% with all small limits and colimits in which \CrefAndHyperrefIfExist{definition:sheaf_on_a_site}{sheaves} are valued.

% Let $\mathcal{A}$ be a (large) category with all small limits and colimits in which \CrefAndHyperrefIfExist{definition:sheaf_on_a_site}{sheaves} are valued.

Then the \CrefAndHyperrefIfExist{definition:direct_image_of_a_sheaf_on_a_site_under_a_continuous_functor_of_sites_or_a_site_morphism}{inverse image functor}
\[
u_s : \operatorname{Sh}(D,K;\mathcal{A}) \to \operatorname{Sh}(C,J;\mathcal{A})
\]
(assuming that the inverse images of all sheaves on $(D,K)$ valued in $\calA$ exist by virtue of $\calA$ admitting enough \CrefAndHyperrefIfExist{definition:projective_and_inductive_limits_in_categories}{colimits}) and the \CrefAndHyperrefIfExist{definition:inverse_image_of_a_sheaf_under_a_continuous_functor_of_sites_or_a_site_morphism}{direct image functor}
\[
u^s : \operatorname{Sh}(C,J;\mathcal{A}) \to \operatorname{Sh}(D,K;\mathcal{A})
\]
form an \CrefAndHyperrefIfExist{definition:adjoint_functors_between_categories_unit_counit_of_adjoint_functors}{adjoint pair}, i.e., for any sheaves $\mathcal{F} \in \operatorname{Sh}(D,K;\mathcal{A})$ and $\mathcal{G} \in \operatorname{Sh}(C,J;\mathcal{A})$, there is a natural isomorphism
\[
\operatorname{Hom}_{\operatorname{Sh}(C,J)}(u_s\mathcal{F}, \mathcal{G}) \cong \operatorname{Hom}_{\operatorname{Sh}(D,K)}(\mathcal{F}, u^s\mathcal{G}).
\]
\end{theorem}


\begin{definition}[Pushforward (direct image) of a sheaf] \label{definition:direct_image_of_a_sheaf_on_a_topological_space}
    Let $f : X \to Y$ be a \CrefAndHyperrefIfExist{definition:continuous_map_between_open_subsets_of_euclidean_spaces}{continuous map} between \CrefAndHyperrefIfExist{definition:topological_space}{topological spaces}, and let $\mathcal{F}$ be a \CrefAndHyperrefIfExist{definition:presheaf_on_a_topological_space}{presheaf on $X$ valued in a category $\mathcal{D}$} with a \CrefAndHyperrefIfExist{lemma:initial_or_final_object_in_a_category_that_is_also_in_a_full_subcategory_is_initial_or_final_in_the_subcategory}{terminal object}.  
    The \hldef{pushforward} or \hldef{direct image presheaf} 
    \hl{$f_* \mathcal{F}$} on $Y$ is the \CrefAndHyperrefIfExist{definition:presheaf_on_a_category}{presheaf valued in $\calD$ on $Y$} defined as follows: For every open set $V \subseteq Y$, the value of the pushforward is given by
    \[ 
    f_* \mathcal{F}(V) := \mathcal{F}(f^{-1}(V)). 
    \]
    For an inclusion of open sets $V' \subseteq V$ in $Y$, the restriction morphism 
    \[ 
    \operatorname{res}_{V, V'}^{f_* \mathcal{F}} : f_* \mathcal{F}(V) \to f_* \mathcal{F}(V') 
    \]
    is defined as the restriction morphism of $\mathcal{F}$ associated with the inclusion of preimages $f^{-1}(V') \subseteq f^{-1}(V)$ in $X$:
    \[ 
    \operatorname{res}_{f^{-1}(V), f^{-1}(V')}^{\mathcal{F}} : \mathcal{F}(f^{-1}(V)) \to \mathcal{F}(f^{-1}(V')). 
    \]
    
    % by
    % $$f_* \mathcal{F}(V) := \mathcal{F}(f^{-1}(V))$$
    % for every open set $V \subseteq Y$, with restriction maps induced from those of $\mathcal{F}$ via preimages.  

    If $\calF$ is a \CrefAndHyperrefIfExist{definition:sheaf_on_a_topological_space_valued_in_a_category_with_a_terminal_object}{sheaf}, then so is $f_* \calF$. \TextIfExists{definition:direct_image_of_a_sheaf_on_a_site_under_a_continuous_functor_of_sites_or_a_site_morphism}{In this case, it is equivalent to define $f_* \calF$ as the \CrefAndHyperrefIfExist{definition:direct_image_of_a_sheaf_on_a_site_under_a_continuous_functor_of_sites_or_a_site_morphism}{direct image} $(f^{-1})^s \calF$ of $\calF$ under the \CrefAndHyperrefIfExist{definition:continuous_functor_between_sites_of_opens_on_topological_spaces_induced_by_continuous_map}{continuous functor $f^{-1}: \operatorname{Open} Y \to \operatorname{Open} X$}\CrefIfExists{definition:continuous_functor_between_sites_of_opens_on_topological_spaces_induced_by_continuous_map}\CrefIfExists{definition:site_of_opens_on_a_topological_space}, which is also equivalent to the \CrefAndHyperrefIfExist{definition:direct_image_of_a_sheaf_on_a_site_under_a_continuous_functor_of_sites_or_a_site_morphism}{direct image} of $\calF$ under the \CrefAndHyperrefIfExist{definition:morphism_of_sites}{site morphism} $\operatorname{Open} X \to \operatorname{Open} Y$ whose underlying continuous functor is $f^{-1}$.}
    
\end{definition}


\begin{definition}[Pullback (inverse image) of a sheaf] \label{definition:inverse_image_of_a_sheaf_on_a_topological_space}
    Let $f : X \to Y$ be a \CrefAndHyperrefIfExist{definition:continuous_map_between_open_subsets_of_euclidean_spaces}{continuous map} between \CrefAndHyperrefIfExist{definition:topological_space}{topological spaces}. Let $\calD$ be a category  with a \CrefAndHyperrefIfExist{lemma:initial_or_final_object_in_a_category_that_is_also_in_a_full_subcategory_is_initial_or_final_in_the_subcategory}{terminal object}.
    
    \begin{enumerate}
        \item  Let $\mathcal{G}$ be a \CrefAndHyperrefIfExist{definition:presheaf_on_a_topological_space}{presheaf on $Y$ valued in a $\calD$}.  
        The \hldef{pullback} or \hldef{inverse image presheaf} 
        \hl{$f^{-1} \mathcal{G}$} on $X$ is defined as the presheaf 
        $$U \mapsto \varinjlim_{V \supseteq f(U)} \mathcal{G}(V)$$
        where $U$ ranges over open subsets of $X$ and the colimit is taken over all open subsets $V \subseteq Y$ containing $f(U)$.  
        \TextIfExists{definition:stalk_of_a_presheaf_on_a_topological_space_at_a_point}{This construction admits a natural isomorphism
        $$(f^{-1}\mathcal{G})_x \to \mathcal{G}_{f(x)}$$
        of \CrefAndHyperrefIfExist{definition:stalk_of_a_presheaf_on_a_topological_space_at_a_point}{stalks} for every $x \in X$.}  

        \item 
        If $\calG$ is a \CrefAndHyperrefIfExist{definition:sheaf_on_a_topological_space_valued_in_a_category_with_a_terminal_object}{sheaf} valued in $\calD$, then we can define the \hldef{pullback} or \hldef{inverse image sheaf} \hl{$f^* \calG$} on $X$ as the \CrefAndHyperrefIfExist{definition:sheafification_functor_on_a_site}{sheaf associated to the presheaf} $f^{-1} \calG$, assuming it exists.
        \TextIfExistsElse{definition:direct_image_of_a_sheaf_on_a_site_under_a_continuous_functor_of_sites_or_a_site_morphism}{

            Assuming that a \CrefAndHyperrefIfExist{definition:sheafification_functor_on_a_site}{sheafification functor} exists, one may equivalently define $f^* \calG$ via \Cref{definition:inverse_image_of_a_sheaf_under_a_continuous_functor_of_sites_or_a_site_morphism} --- More concretely, $f^* \calG$ is the following equivalent constructions:
            \begin{itemize}
                \item The \CrefAndHyperrefIfExist{definition:inverse_image_of_a_sheaf_under_a_continuous_functor_of_sites_or_a_site_morphism}{direct image} $(f^{-1})_s \calG$ of $\calG$ under the continuous functor $f^{-1}: \operatorname{Open} Y \to \operatorname{Open} X, \quad W \mapsto f^{-1}(W)$\CrefIfExists{definition:site_of_opens_on_a_topological_space}\CrefIfExists{definition:continuous_functor_between_sites_of_opens_on_topological_spaces_induced_by_continuous_map}. 

                \item The \CrefAndHyperrefIfExist{definition:inverse_image_of_a_sheaf_under_a_continuous_functor_of_sites_or_a_site_morphism}{inverse image} of $\calG$ under the \CrefAndHyperrefIfExist{definition:morphism_of_sites}{site morphism} $\operatorname{Open} X \to \operatorname{Open} Y$ whose underlying \CrefAndHyperrefIfExist{definition:continuous_functor_of_sites}{continuous functor} is $f^{-1}$

            \end{itemize}
            
        }{
        }

    \end{enumerate}
\end{definition}

\begin{theorem} \label{theorem:adjunction_between_inverse_image_and_direct_image_functors_for_sheaves_on_topological_spaces_under_a_continuous_map}
    Let $f : X \to Y$ be a \CrefAndHyperrefIfExist{definition:continuous_map_between_open_subsets_of_euclidean_spaces}{continuous map} between \CrefAndHyperrefIfExist{definition:topological_space}{topological spaces}. Let $\calA$ be a \CrefAndHyperrefIfExist{definition:locally_small_category}{locally small category} such that $\calA$ is \CrefAndHyperrefIfExist{definition:complete_and_cocomplete_category}{cocomplete}, i.e. admits \CrefAndHyperrefIfExist{definition:small_and_finite_limits_and_colimits_in_a_category}{small colimits}, and such that $\PreShv(Y, \calA)$ admits \CrefAndHyperrefIfExist{definition:sheafification_of_a_presheaf_on_a_topological_space}{sheafification}.
    
    The \CrefAndHyperrefIfExist{definition:inverse_image_of_a_sheaf_on_a_topological_space}{inverse image functor} 
    $$f^*: \Sh(Y, \calA) \to \Sh(X, \calA)$$
    and the \CrefAndHyperrefIfExist{definition:direct_image_of_a_sheaf_on_a_topological_space}{direct image functor}
    $$f_*: \Sh(X, \calA) \to \Sh(Y, \calA)$$
    form an \CrefAndHyperrefIfExist{definition:adjoint_functors_between_categories_unit_counit_of_adjoint_functors}{adjoint pair} $f^* \dashv f_*$, i.e., for any sheaves $\calF \in \Sh(Y, \calA)$ and $\calG \in \Sh(X, \calA)$, there is a natural isomorphism
    $$\Hom_{\Sh(X, \calA)}(f^* \calF, \calG) \cong \Hom_{\Sh(Y, \calA)}(\calF, f_* \calG).$$
\end{theorem}

\begin{theorem} [Comparison Lemma, cf. {{\cite[Expos\'e III, Th\'eor\`eme 4.1]{SGA4_I}}}] \label{theorem:comparison_lemma_of_sheaves_on_a_site_and_sheaves_on_a_small_topologically_generating_family}
    Let $\calC$ be a \CrefAndHyperrefIfExist{definition:locally_small_category}{small category}, let $(\calD, K)$ be a \CrefAndHyperrefIfExist{definition:grothendieck_topology_on_a_category_site_covering_sieve_topologically_generating_family}{site}, let $u: \calC \to \calD$ be a \CrefAndHyperrefIfExist{definition:full_and_faithful_functor_between_locally_small_categories}{fully faithful functor}, and equip $\calC$ with the \CrefAndHyperrefIfExist{definition:grothendieck_topology_on_a_category_site_covering_sieve_topologically_generating_family}{Grothendieck topology} $J$ \CrefAndHyperrefIfExist{definition:induced_topology_on_a_category_by_a_functor_to_a_site}{induced by} $u$. Let $\calA$ be a \CrefAndHyperref{definition:locally_small_category}{locally small} \CrefAndHyperrefIfExist{definition:complete_and_cocomplete_category}{complete category}, i.e. a category which admits all \CrefAndHyperrefIfExist{definition:small_and_finite_limits_and_colimits_in_a_category}{small limits}.

    If the objects of $\calC$ form a \CrefAndHyperrefIfExist{definition:grothendieck_topology_on_a_category_site_covering_sieve_topologically_generating_family}{topologically generating family}\TODO{I need to check that the condition of covering in SGA really matches the notion of topologically generating family} of $\calD$, then the \CrefAndHyperrefIfExist{definition:direct_image_of_a_sheaf_on_a_site_under_a_continuous_functor_of_sites_or_a_site_morphism}{direct image}/restriction functor 
    $$u_s: \Sh(\calD, K; \calA) \to \Sh(\calC, J; \calA)$$
    is an \CrefAndHyperrefIfExist{definition:equivalence_of_categories}{equivalence of categories}. In particular, $\Sh(\calD, K; \calA)$ is a \CrefAndHyperrefIfExist{definition:locally_small_category}{locally small category}.
\end{theorem}
\begin{proof}
We explicitly construct the inverse functor and prove the equivalence without assuming \textit{a priori} that $\text{Sh}(\mathcal{D}, K)$ is locally small.

Step 1: Construction of the Inverse Functor $u_*$

We define the functor $u_*: \text{Sh}(\mathcal{C}, J) \to \text{Sh}(\mathcal{D}, K)$ as the \textbf{Right Kan Extension} of a sheaf $F$ along $u$.

For a sheaf $F \in \text{Sh}(\mathcal{C}, J)$ and an object $d \in \mathcal{D}$, define:
\[ u_*F(d) = \lim_{(c, f) \in (u \downarrow d)^{op}} F(c) \]
Here, the limit is taken over the opposite of the comma category $(u \downarrow d)$, whose objects are pairs $(c, f: u(c) \to d)$.

\textit{Verification of Well-Definedness (Size):} Since $\mathcal{C}$ is a small category and $\mathcal{D}$ is locally small, the collection of objects in $(u \downarrow d)$ is a set (indexed by objects of $\mathcal{C}$ and hom-sets of $\mathcal{D}$). Therefore, the limit defining $u_*F(d)$ is a \textbf{small limit of sets}, which exists in $\mathbf{Set}$. Thus, $u_*F$ takes values in $\mathbf{Set}$ rather than proper classes.

\textit{Verification that $u_*F$ is a Sheaf:} Since limits commute with limits, and the sheaf condition is a limit condition, the Right Kan Extension of a sheaf along a continuous functor is a sheaf. The density condition ensures that covers in $\mathcal{D}$ are "seen" by $\mathcal{C}$, ensuring the sheaf condition is preserved.

Step 2: The Unit of Adjunction ($u^* u_* \cong \text{Id}$)

We examine $u^* u_* F$ for $F \in \text{Sh}(\mathcal{C}, J)$. Evaluating at an object $c_0 \in \mathcal{C}$:
\[ (u^* u_* F)(c_0) = u_*F(u(c_0)) = \lim_{(c, f) \in (u \downarrow u(c_0))^{op}} F(c) \]
Since $u$ is \textbf{fully faithful}, the comma category $(u \downarrow u(c_0))$ has a terminal object: $(c_0, \text{id}_{u(c_0)})$. The limit over a category with a terminal object is isomorphic to the value at that object. Thus, $(u^* u_* F)(c_0) \cong F(c_0)$, concluding $u^* u_* \cong \text{Id}_{\text{Sh}(\mathcal{C})}$.

Step 3: The Counit of Adjunction ($H \cong u_* u^* H$)

This is the critical step that uses the \textbf{Density Condition} to control the size of sheaves on $\mathcal{D}$. Let $H \in \text{Sh}(\mathcal{D}, K)$. We construct a map $\eta_d: H(d) \to u_*(u^*H)(d)$. By definition:
\[ u_*(u^*H)(d) = \lim_{u(c) \to d} H(u(c)) \]
There is a canonical map $\eta_d$ induced by the morphisms $H(f): H(d) \to H(u(c))$ for each $f: u(c) \to d$. 

To see that $\eta_d$ is an isomorphism:
\begin{enumerate}
    \item \textbf{Density as a Cover:} By hypothesis, the family of morphisms $\mathcal{S} = \{ f: u(c) \to d \mid c \in \mathcal{C} \}$ generates a covering sieve $S$ on $d$.
    \item \textbf{Sheaf Property:} Since $H$ is a sheaf on $(\mathcal{D}, K)$, $H(d) \cong \text{Match}(S, H)$.
    \item \textbf{Matching Families:} The limit over the comma category $(u \downarrow d)$ is exactly the set of compatible families indexed by the generators of the sieve. Because $\mathcal{S}$ generates $S$, the data of a matching family on $\mathcal{S}$ extends uniquely to the whole sieve $S$.
\end{enumerate}
The canonical map $H(d) \to \lim_{u(c) \to d} H(u(c))$ is therefore an isomorphism, so $H \cong u_* (u^* H)$.

Step 4: Conclusion of Equivalence and Local Smallness

We have established natural isomorphisms $u^* u_* \cong \text{Id}$ and $\text{Id} \cong u_* u^*$, establishing an equivalence. Since $\mathcal{C}$ is small, $\text{Sh}(\mathcal{C}, J)$ is a locally small category. Since $\text{Sh}(\mathcal{D}, K)$ is equivalent to it, $\text{Sh}(\mathcal{D}, K)$ is itself locally small. Specifically:
\[ \text{Hom}_{\mathcal{D}}(H, G) \cong \text{Hom}_{\mathcal{C}}(u^*H, u^*G) \]
where the latter is a set.



    Since $\calC$ is small and $\calA$ is locally small, the category of presheaves on $\calC$ valued in $\calA$ is locally small by \Cref{lemma:category_of_presheaves_on_a_small_category_of_locally_small_value_is_locally_small}. Therefore, the category of sheaves is locally small.
\end{proof}


\begin{definition} \label{definition:ringed_site}
    \TODO{there are places where sites and sheaves of rings on them are used, but it would be better to just have them be ringed sites.}

    A \hldef{ringed site} is a \CrefAndHyperrefIfExist{definition:grothendieck_topology_on_a_category_site_covering_sieve_topologically_generating_family}{site} $(\mathcal{C}, J)$ with a small \CrefAndHyperrefIfExist{definition:grothendieck_topology_on_a_category_site_covering_sieve_topologically_generating_family}{topological generating family} equipped with a \CrefAndHyperrefIfExist{definition:sheaf_on_a_site}{sheaf} of (not necessarily commutative) rings $\mathcal{O}$. If the Grothendieck topology $J$ is clear in context, one may even write that $(C, \calO)$ is a ringed site.

    A \hldef{morphism of ringed sites}
    $$ ((\mathcal{C},J),\mathcal{O}) \to ((\mathcal{C}',J'),\mathcal{O}') $$
    consists of a \CrefAndHyperrefIfExist{definition:morphism_of_sites}{morphism of sites} $f : (\mathcal{C},J) \to (\mathcal{C}',J')$ and a \CrefAndHyperrefIfExist{definition:sheaf_on_a_site}{morphism of sheaves} of rings $f^\# : \mathcal{O}' \to f_*\mathcal{O}$ \CrefIfExists{definition:inverse_image_of_a_sheaf_under_a_continuous_functor_of_sites_or_a_site_morphism}.
\end{definition}

\subsection{Sheaf cohomology}

\begin{definition} \label{definition:sections_of_a_presheaf_on_a_category_valued_in_a_category}
Let $\mathcal{C}$ be a \CrefAndHyperrefIfExist{definition:category}{(large) category}, and let $\mathcal{D}$ be a \CrefAndHyperrefIfExist{definition:category}{(large) category}. Let $\mathcal{F} : \mathcal{C}^{op} \to \mathcal{D}$ be a \CrefAndHyperrefIfExist{definition:presheaf_on_a_category}{presheaf valued in $\mathcal{D}$}.

\begin{enumerate}
    \item For an object $U \in \mathcal{C}$, the \hldef{sections functor evaluated at $U$} is the functor
    $$\hlin{\Gamma(U, -) : \mathrm{PSh}(\mathcal{C}, \mathcal{D}) \to \mathcal{D}}$$
    defined by
    $$\Gamma(U, \mathcal{F}) := \mathcal{F}(U),$$
    i.e., the value of the presheaf $\mathcal{F}$ at the object $U$.

    \item The \hldef{global sections of $\calF$} is the object \hl{$\Gamma(\calF)$} of $\calD$ defined as the \CrefAndHyperrefIfExist{definition:limit_and_colimit_of_a_diagram_in_a_category}{limit}
    $$\Gamma(\calF) = \varprojlim_{U \in \calC^{\op}} \calF(U)$$
    assuming that such a limit exists, where the limit is taken over objects $U \in \calC$ and the restriction morphisms $\calF(V) \to \calF(U)$ in $\calD$ for  morphisms $U \to V$ in $\calC$. 
    
    If a \CrefAndHyperrefIfExist{definition:initial_final_zero_objects_of_a_category}{final object} $\ast \in \mathcal{C}$ exists, then $\Gamma(\calF)$ exists and coincides with $\Gamma(\ast, \calF) = \calF(\ast)$. The construction $\Gamma(\calF)$ is functorial; in particular, if $\Gamma(\calF)$ exists for all $\calF$ in $\mathrm{PSh}(\calC, \calD)$, e.g. if \CrefAndHyperrefIfExist{definition:limit_and_colimit_of_a_diagram_in_a_category}{limits of} diagrams in $\calD$ indexed by $\calC$ exist, then $\Gamma$ is a functor 
    $$\hlin{\Gamma : \mathrm{PSh}(\mathcal{C}, \mathcal{D}) \to \mathcal{D}}$$
    called the \hldef{global sections functor on $\mathrm{PSh}(\mathcal{C}, \mathcal{D})$}.
\end{enumerate}
\end{definition}

\begin{proposition} \label{proposition:final_object_of_category_of_locally_small_objects_represents_a_final_object_of_the_category_of_set_valued_presheaves}
    Let $\calC$ be a \CrefAndHyperrefIfExist{definition:locally_small_category}{locally small category} such that $\calC$ has a \CrefAndHyperrefIfExist{definition:initial_final_zero_objects_of_a_category}{final object} $*$.
    The \CrefAndHyperrefIfExist{definition:presheaf_on_a_category}{presheaf} \CrefAndHyperrefIfExist{definition:representable_functor_on_a_category_enriched_in_a_monoidal_category}{represented by} $*$ is a final object in the category $\mathrm{PSh}(\calC, \Sets)$ of set-valued presheaves.
\end{proposition}

\begin{proposition} \label{proposition:sections_functors_on_presheaves_vlaued_in_an_abelian_category_are_left_exact}
Let $\calC$ be an \CrefAndHyperrefIfExist{definition:essentially_small_category}{essentially small category} and let $\mathcal{A}$ be an \CrefAndHyperrefIfExist{definition:abelian_category}{abelian category}. 
\begin{enumerate}
    \item Let $U \in \Ob(\calC)$ be some fixed object. The \CrefAndHyperrefIfExist{definition:sections_of_a_presheaf_on_a_category_valued_in_a_category}{sections functor}
    $$\Gamma(U, -) : \mathrm{PSh}(\mathcal{C}, \mathcal{A}) \to \mathcal{A}$$ 
    is \CrefAndHyperrefIfExist{definition:exact_functor_between_abelian_categories}{left exact}. \TODO{state that presheaves and sheaves valued in an abelian category form abelian categories}

    \item Assume that $\Gamma(\calF)$ exists for all $\calF$ in $\mathrm{PSh}(\calC, \calA)$ so that $\Gamma$ is a functor 
    $$\Gamma: \mathrm{PSh}(\mathcal{C}, \calA) \to \calA.$$
    The functor $\Gamma$ is left exact.

\end{enumerate}
\end{proposition}
\begin{proof}
    Recall that $\mathrm{PSh}(\mathcal{C}, \mathcal{D})$ is an abelian category (\Cref{proposition:examples_of_abelian_categories}) since $\calC$ is essentially small.  \TODO{Talk about how limits are left exact and }
\end{proof}

\begin{theorem}[e.g. see {\cite[Tag 01DU]{stacks-project}}] \label{theorem:category_of_modules_over_a_sheaf_of_rings_on_a_site_on_an_essentially_small_category_has_enough_injectives}
    For any \CrefAndHyperrefIfExist{definition:grothendieck_topology_on_a_category_site_covering_sieve_topologically_generating_family}{site} $(\calC, J)$ on an \CrefAndHyperrefIfExist{definition:essentially_small_category}{essentially small category} $\mathcal{C}$ and a \CrefAndHyperrefIfExist{definition:sheaf_on_a_site}{sheaf of rings} $\calO$ on $\calC$, the category $\mathbf{Mod}(\mathcal{O})$ of \CrefAndHyperrefIfExist{definition:module_over_a_sheaf_of_rings_on_a_site}{$\calO$-modules} is an abelian category that \CrefAndHyperrefIfExist{definition:has_enough_injectives_or_projectives_for_an_abelian_category}{has enough injectives}. In fact, there is a functorial injective embedding \TODO{ what does this mean?}
\end{theorem}


\begin{definition}[Category of objects over a fixed object] \label{definition:category_of_objects_over_under_a_fixed_object_in_a_category}
Let $\mathcal{C}$ be a \hyperrefIfExists{definition:category}{category}\CrefIfExists{definition:category} and let $X \in \operatorname{Ob}(\mathcal{C})$ be a fixed object.
\begin{enumerate}
    \item 
        The \hldef{category of objects over $X$} (or synonymously the \hldef{slice category of $X$ in $\calC$} or the \hldef{over category of $X$ in $\calC$}), commonly denoted \hl{$\mathcal{C}/X$}, \hl{$\mathcal{C}_{/X}$}, or \hl{$(\mathcal{C} \downarrow X)$} is the category defined as follows:
        \begin{itemize}
            \item An object of $\mathcal{C}/X$ is a morphism $f \colon A \to X$ in $\mathcal{C}$, where $A \in \operatorname{Ob}(\mathcal{C})$.
            \item A morphism from $f \colon A \to X$ to $g \colon B \to X$ in $\mathcal{C}/X$ is a morphism $h \colon A \to B$ in $\mathcal{C}$ such that the following diagram commutes:
            $$
            \begin{aligned}
            \xymatrix{
            A \ar[dr]_f \ar[r]^h & B \ar[d]^g \\
            & X
            }
            \end{aligned}
            $$
            i.e. such that $g \circ h = f$.
            \item The identity morphisms and composition in $\mathcal{C}/X$ are inherited from $\mathcal{C}$.
        \end{itemize}

    \item 
    The \hldef{category of objects under $X$} (or synonymously the \hldef{coslice category of $X$ in $\calC$} or the \hldef{under category of $X$ in $\calC$}), commonly denoted \hl{$X/\mathcal{C}$}, \hl{$X \backslash \calC$}, \hl{$\mathcal{C}_{X/}$}, or \hl{$(X \downarrow \calC)$}, is the category defined as follows:
    \begin{itemize}
        \item An object of $X/\mathcal{C}$ is a morphism $f \colon X \to A$ in $\mathcal{C}$, where $A \in \operatorname{Ob}(\mathcal{C})$.
        \item A morphism from $f \colon X \to A$ to $g \colon X \to B$ in $X/\mathcal{C}$ is a morphism $h \colon A \to B$ in $\mathcal{C}$ such that the following diagram commutes:
        $$
        \begin{aligned}
        \xymatrix{
        X \ar[dr]^g \ar[r]^f & A \ar[d]^h \\
        & B
        }
        \end{aligned}
        $$
        i.e. such that $h \circ f = g$.
        \item The identity morphisms and composition in $X/\mathcal{C}$ are inherited from $\mathcal{C}$.
    \end{itemize}

\end{enumerate} 
\TextIfExists{definition:comma_category_of_two_functors_to_a_category}{Both notions are special cases of \CrefAndHyperrefIfExist{definition:comma_category_of_two_functors_to_a_category}{comma categories}.}
\end{definition}

\begin{definition}[Slice site] \label{definition:site_induced_by_a_site_on_an_over_category}
Let $(\mathcal{C}, \tau)$ be a \CrefAndHyperrefIfExist{definition:grothendieck_topology_on_a_category_site_covering_sieve_topologically_generating_family}{site}, where $\tau$ is a Grothendieck topology on the (\CrefAndHyperrefIfExist{definition:locally_small_category}{locally small or $U$-locally small}, if a \CrefAndHyperrefIfExist{definition:grothendieck_universe}{universe} $U$ is available) category $\mathcal{C}$. For a fixed object $X$ in $\mathcal{C}$, the \hldef{slice site} (or the \hldef{over site}, the \hldef{site on the slice category $\mathcal{C}_{/X}$}, the \hldef{site induced on the over category $\mathcal{C}_{/X}$}, the \hldef{localization of the site $\calC$ at the object $X$}, etc.) $(\mathcal{C}_{/X}, \tau_{/X})$ is the site whose underlying category is the \CrefAndHyperrefIfExist{definition:category_of_objects_over_under_a_fixed_object_in_a_category}{slice category $\mathcal{C}_{/X}$}, and whose Grothendieck topology \hl{$\tau_{/X}$} (also denoted by notations such as \hl{$\tau|_{X}$} or \hl{$\tau/X$}) is defined by declaring a family of morphisms $\{f_i : Y_i \to Y\}$ in $\mathcal{C}_{/X}$ to be a covering if and only if the family $\{f_i : Y_i \to Y\}$ is a covering in $(\mathcal{C}, \tau)$.

% The forgetful functor 
% $$\hlin{j_X: \calC/X \to \calC}$$
% is \CrefAndHyperrefIfExist{definition:continuous_cocontinuous_functor_between_categories}{cocontinuous and continuous} 

\end{definition}
\begin{definition} \label{definition:sheaf_cohomology_group_of_a_sheaf_of_modules_over_a_sheaf_of_rings_on_a_site}

Let $(\mathcal{C}, J)$ be a \CrefAndHyperrefIfExist{definition:grothendieck_topology_on_a_category_site_covering_sieve_topologically_generating_family}{site} on a \CrefAndHyperrefIfExist{definition:locally_small_category}{locally small category} or a $U$-site for some \CrefAndHyperrefIfExist{definition:grothendieck_universe}{universe} $U$. Let $\calO$ be a \CrefAndHyperrefIfExist{definition:sheaf_on_a_site}{sheaf of rings} on $\calC$, so that $(\calC, J, \calO)$ is a \CrefAndHyperrefIfExist{definition:ringed_site}{ringed site}. Recall that the category \CrefAndHyperrefIfExist{definition:module_over_a_sheaf_of_rings_on_a_site}{$\mathbf{Mod}(\mathcal{O})$} of $\calO$-modules is abelian and \CrefAndHyperrefIfExist{definition:has_enough_injectives_or_projectives_for_an_abelian_category}{has enough injectives} (\Cref{theorem:category_of_modules_over_a_sheaf_of_rings_on_a_site_on_an_essentially_small_category_has_enough_injectives}).

Assume that \CrefAndHyperrefIfExist{definition:sections_of_a_presheaf_on_a_category_valued_in_a_category}{global sections objects $\Gamma(\calG)$} exist for all objects $\calG$ of $\mathrm{Sh}(\mathcal{C}, \mathbf{Ab})$\footnote{for example, this occurs when $\calC$ is \CrefAndHyperrefIfExist{definition:essentially_small_category}{essentially small}} so that $\Gamma$ is a functor
$$\Sh(\calC, \mathbf{Ab}) \to \mathbf{Ab},$$
which is a \CrefAndHyperrefIfExist{definition:exact_functor_between_abelian_categories}{left exact functor} (\Cref{proposition:sections_functors_on_presheaves_vlaued_in_an_abelian_category_are_left_exact}).  Note that $\Gamma$ restricts to a left exact functor 
$$\mathbf{Mod}(\mathcal{O}) \to \mathbf{Ab}.$$
If $\calC$ has a \CrefAndHyperrefIfExist{definition:initial_final_zero_objects_of_a_category}{final object} $\ast$ as well, then recall that $\Gamma(\calF) = \calF(\ast)$. 

Let $\calF$ be an object of $\mathbf{Mod}(\mathcal{O})$. 
\begin{enumerate}
    \item For each integer $n \geq 0$, the \hldef{$n$-th (abelian) (global) sheaf cohomology group of $\mathcal{F}$} is
    $$\hlin{H^n(\mathcal{C}, J; \mathcal{F}) := R^n \Gamma(\mathcal{F}),}$$
    where $R^n \Gamma$ is the $n$-th \CrefAndHyperrefIfExist{definition:left_right_derived_functors_of_a_right_left_exact_functor_between_abelian_categories_where_source_has_enough_projectives_injectives}{right derived functor} of the \CrefAndHyperrefIfExist{definition:sections_of_a_presheaf_on_a_category_valued_in_a_category}{global sections functor $\Gamma$}.

    In particular, each $H^n$ is a functor 
    $$H^n: \mathbf{Mod}(\mathcal{O})  \to \mathbf{Ab}.$$

    \item Given an object $U \in \calC$ and for each integer $n \geq 0$, the \hldef{$n$-th (abelian) sheaf cohomology group of $\mathcal{F}$ of sections at $U$} is
    $$\hlin{H^n(U, \mathcal{F}) := (R^n \Gamma(U,-))(\mathcal{F}),}$$
    where $R^n \Gamma(U,-)$ is the $n$-th \CrefAndHyperrefIfExist{definition:left_right_derived_functors_of_a_right_left_exact_functor_between_abelian_categories_where_source_has_enough_projectives_injectives}{right derived functor} of the \CrefAndHyperrefIfExist{definition:sections_of_a_presheaf_on_a_category_valued_in_a_category}{sections functor $\Gamma(U,-)$ evaluated at $U$}.

    In particular, $H^n(U,\calF)$ can be regarded as the $n$th global sheaf cohomology group of the \CrefAndHyperrefIfExist{definition:restriction_of_a_sheaf_on_a_site_to_an_object_of_the_underlying_category_of_the_site}{restriction $\calF|_U$} of $\calF$ to $U$.
    % to the 
    % \CrefAndHyperrefIfExist{definition:site_induced_by_a_site_on_an_over_category}{site induced by $(\calC, J)$} on the \CrefAndHyperrefIfExist{definition:category_of_objects_over_under_a_fixed_object_in_a_category}{over category $\calC_{/U}$}.

\end{enumerate}

% In the case that $\calA = R\mathbf{-mod}$, the category of (left/right/two-sided)modules over some fixed (not necessarily commutative) ring $R$, recall that $R\mathbf{-Mod}$ is complete (and cocomplete) \TODO{Is it precisely completeness that I need or cocompleteness? In other words, for the limits defining $\Gamma$, are those limits projective limits or colimtis?} \TODO{talk about how $R$-modules are complete and cocomplete}, so all global sections objects $\Gamma(\calG)$ exist. \TODO{continue talking about this context of modules}

% The \hldef{sheaf cohomology groups of $\mathcal{F}$} on the site $(\mathcal{C}, J)$ are defined as the \CrefAndHyperrefIfExist{definition:left_right_derived_functors_of_a_right_left_exact_functor_between_abelian_categories_where_source_has_enough_projectives_injectives}{right derived functors} of the global sections functor
% $$\Gamma : \mathrm{Sh}(\mathcal{C}, J) \to \mathbf{Ab}$$
% where $\mathrm{Sh}(\mathcal{C}, J)$ \CrefAndHyperrefIfExist{definition:sheaf_on_a_site}{denotes} the category of sheaves of abelian groups on $(\mathcal{C}, J)$, and $\ast$ denotes the \CrefAndHyperrefIfExist{definition:initial_final_zero_objects_of_a_category}{final object} in $\mathcal{C}$ if it exists.

% More precisely, 

% If $\mathcal{C}$ has no final object, $H^n(\mathcal{C}, J; \mathcal{F})$ is defined by choosing an injective resolution of $\mathcal{F}$ and taking cohomology of the resulting complex obtained by applying $\Gamma$.

% These groups measure the extent to which global sections fail to be exact, and generalize classical sheaf cohomology defined on topological spaces to arbitrary sites.
\end{definition}

We show that sheaf cohomology commutes with filtered colimits.

\begin{definition} \label{definition:coherent_site}
A \CrefAndHyperrefIfExist{definition:grothendieck_topology_on_a_category_site_covering_sieve_topologically_generating_family}{site} $(\mathcal{C}, J)$ is called a \hldef{coherent site} if it satisfies the following conditions:
\begin{enumerate}
    \item The category $\mathcal{C}$ is locally small with a small \CrefAndHyperrefIfExist{definition:grothendieck_topology_on_a_category_site_covering_sieve_topologically_generating_family}{topologically generating family}.
    \item The category $\mathcal{C}$ admits all finite \CrefAndHyperrefIfExist{definition:limit_and_colimit_of_a_diagram_in_a_category}{limits}.
    \item Every object in $\mathcal{C}$ is \CrefAndHyperrefIfExist{definition:quasi_compact_object_in_a_locally_small_site}{quasi-compact relative to} the topology $J$.
    \item The collection of quasi-compact objects relative to $J$ is closed under \CrefAndHyperrefIfExist{definition:cartesian_product_of_two_objects_in_a_category_over_an_object}{fiber products}. Explicitly, if $X \to Z$ and $Y \to Z$ are morphisms where $X, Y, Z$ are quasi-compact, then $X \times_Z Y$ is also quasi-compact.
\end{enumerate}
\end{definition}
\begin{definition} \label{definition:coherent_topos}
A \CrefAndHyperrefIfExist{definition:topos}{Grothendieck topos} $\mathcal{E}$ is called a \hldef{coherent topos} if there exists a \CrefAndHyperrefIfExist{definition:coherent_site}{coherent site} $(\mathcal{C}, J)$ such that $\mathcal{E}$ is equivalent to the \CrefAndHyperrefIfExist{definition:sheaf_on_a_site}{category of sheaves} $\mathcal{E} \simeq \text{Sh}(\mathcal{C}, J)$ on that site.

Alternatively, an intrinsic characterization 
\cite[Expos\'e VI]{SGA4_II}
is that $\mathcal{E}$ is coherent if there exists a \CrefAndHyperrefIfExist{definition:generator_of_a_category}{generating family} $\mathcal{G}$ of objects in $\mathcal{E}$ such that:
\begin{enumerate}
    \item Every object in $\mathcal{G}$ is \CrefAndHyperrefIfExist{definition:coherent_object_in_a_locally_small_site}{coherent} (\CrefAndHyperrefIfExist{definition:quasi_compact_object_in_a_locally_small_site}{quasi-compact} and \CrefAndHyperrefIfExist{definition:quasi_seperated_morphism_in_a_locally_small_site}{quasi-separated}).
    \item The full subcategory generated by $\mathcal{G}$ is closed under \CrefAndHyperrefIfExist{definition:cartesian_product_of_two_objects_in_a_category_over_an_object}{fiber products}.
\end{enumerate}
It is a theorem that the full subcategory of \textit{all} coherent objects in a coherent topos is closed under finite limits and finite colimits (forming a so-called \hldef{pretopos}).
\end{definition}

\begin{theorem}[Standard Examples of Coherent Sites] \label{theorem:examples_of_coherent_sites}
The following are standard examples of coherent sites. In each case, the topos of sheaves on the site is a \CrefAndHyperrefIfExist{definition:coherent_sheaf_topos}{coherent topos}.
\begin{enumerate}
    \item \textbf{The Zariski Site of a Coherent Scheme}:
    Let $X$ be a \hldef{coherent scheme} \TODO{define coherent scheme} (e.g., any \hldef{Noetherian scheme}). Let $\mathcal{C}$ be the category of open subsets of $X$ that are \hldef{quasi-compact} (i.e., finite unions of affine opens). Let $J$ be the standard finite open cover topology.
    
    Then $(\mathcal{C}, J)$ is a coherent site. The sheaves on this site are equivalent to the category of sheaves on the full Zariski site of $X$.
    
    \item \textbf{The Étale Site of a Coherent Scheme}:
    Let $X$ be a coherent scheme. Let $\mathcal{C}$ be the category of schemes \hldef{étale} over $X$ which are themselves coherent (quasi-compact and quasi-separated over $X$) \TODO{define étale morphism}. Let $J$ be the topology generated by finite surjective families of étale maps.
    
    Then $(\mathcal{C}, J)$ is a coherent site.

    \item \textbf{The Nisnevich Site of a Coherent Scheme}:
    Let $X$ be a coherent scheme. The Nisnevich site over $X$ is a coherent site.
    
    \item \textbf{The Site of Finite Sets}:
    Let $\mathcal{C} = \mathbf{FinSet}$ be the category of finite sets. Let $J$ be the \hldef{canonical topology} (where covering families are jointly surjective finite families).
    
    This is a coherent site. The corresponding topos is $\mathbf{Set}$ itself.
    
    \item \textbf{The Syntactic Site of a Coherent Theory}:
    Let $\mathbb{T}$ be a \hldef{coherent theory} in first-order logic \TODO{define coherent theory}. Let $\mathcal{C}_{\mathbb{T}}$ be the \hldef{syntactic category} of $\mathbb{T}$, whose objects are formulas modulo equivalence and morphisms are provable functional relations.
    
    Equipped with the topology where covers correspond to finite disjunctions ($\phi \leftrightarrow \bigvee_{i=1}^n \psi_i$), this forms a coherent site. The sheaf topos is the \hldef{classifying topos} of $\mathbb{T}$.
\end{enumerate}
\end{theorem}

\begin{theorem}[Sheaf cohomology of modules on a coherent site commutes with filtered colimits {\cite[Expos\'e VI Corollaire 5.2]{SGA4_II}}] \label{theorem:sheaf_cohomology_on_a_ringed_site_with_a_final_object_commutes_with_filtered_colimits}
Let $\mathcal{E}$ be a \CrefAndHyperrefIfExist{definition:coherent_topos}{coherent topos}. In particular, the \CrefAndHyperrefIfExist{definition:grothendieck_topology_on_a_category_site_covering_sieve_topologically_generating_family}{site} underlying $\calE$ has a final object and so we may speak of the \CrefAndHyperrefIfExist{definition:sheaf_cohomology_group_of_a_sheaf_of_modules_over_a_sheaf_of_rings_on_a_site}{sheaf cohomology functors} $H^q(\mathcal{E}, -)$ as the derived functors of the global sections functor. 

The functors $H^q(\mathcal{E}, -)$ commute with \CrefAndHyperrefIfExist{definition:projective_and_inductive_limits_in_categories}{filtered colimits}. 

Explicitly, let $\{\mathcal{F}_i\}_{i \in I}$ be a \CrefAndHyperrefIfExist{definition:system_in_a_category_indexed_by_a_directed_poset}{filtered system} of abelian sheaves in $\mathcal{E}$. Then the canonical map
$$ \colim_{i \in I} H^q(\mathcal{E}, \mathcal{F}_i) \xrightarrow{\cong} H^q(\mathcal{E}, \colim_{i \in I} \mathcal{F}_i) $$
is an isomorphism for all $q \geq 0$. In particular, the global sections functor $\Gamma(\mathcal{E}, -) = H^0(\mathcal{E}, -)$ commutes with filtered colimits.
\end{theorem}


\section{Spectral Sequences}

\begin{definition} \label{definition:graded_object_in_an_abelian_category}
    Let $\mathcal{A}$ be an \CrefAndHyperrefIfExist{definition:abelian_category}{abelian category} . 
    A \hldef{graded object in $\mathcal{A}$} is a collection of objects \hl{$M = \{M_n\}_{n \in \mathbb{Z}}$} in $\mathcal{A}$ indexed by the integers. 
\end{definition}
\begin{definition} \label{definition:morphism_of_graded_objects_in_an_abelian_category}
    Let $\mathcal{A}$ be an \CrefAndHyperrefIfExist{definition:abelian_category}{abelian category}. Let $M = \{M_n\}_{n \in \mathbb{Z}}$ and $N = \{N_n\}_{n \in \mathbb{Z}}$ be \CrefAndHyperrefIfExist{definition:graded_object_in_an_abelian_category}{graded objects} in $\mathcal{A}$.
    A \hldef{morphism of graded objects} $f: M \to N$ is a family of morphisms in $\mathcal{A}$
    $$\hlin{f = \{f_n: M_n \to N_n\}_{n \in \mathbb{Z}}.}$$
    Composition of such morphisms is defined component-wise: $(g \circ f)_n = g_n \circ f_n$.
\end{definition}
\begin{definition} \label{definition:category_of_graded_objects_in_an_abelian_category}
    Let $\mathcal{A}$ be an \CrefAndHyperrefIfExist{definition:abelian_category}{abelian category}.
    The \hldef{category of graded objects in $\mathcal{A}$}, denoted \hl{$\text{Gr}(\mathcal{A})$}, is the category whose objects are graded objects in $\mathcal{A}$ and whose morphisms are morphisms of graded objects.
\end{definition}
\begin{proposition} \label{proposition:category_of_graded_objects_in_an_abelian_category_is_abelian}
    Let $\mathcal{A}$ be an \CrefAndHyperrefIfExist{definition:abelian_category}{abelian category}. 
    The category \CrefAndHyperrefIfExist{definition:category_of_graded_objects_in_an_abelian_category}{$\text{Gr}(\mathcal{A})$} is an abelian category.
    \CrefAndHyperrefIfExist{definition:limit_and_colimit_of_a_diagram_in_a_category}{Limits, colimits}, \CrefAndHyperrefIfExist{definition:kernel_and_cokernel_of_a_morphism_in_a_category}{kernels, and cokernels} in \hl{$\text{Gr}(\mathcal{A})$} are formed component-wise in $\mathcal{A}$. 
    Explicitly, a sequence of graded objects
    $$0 \to L \xrightarrow{f} M \xrightarrow{g} N \to 0$$
    is \CrefAndHyperrefIfExist{definition:short_exact_sequence_in_an_additive_category}{exact} in $\text{Gr}(\mathcal{A})$ if and only if for every $n \in \mathbb{Z}$, the sequence of components
    $$0 \to L_n \xrightarrow{f_n} M_n \xrightarrow{g_n} N_n \to 0$$
    is exact in $\mathcal{A}$.
\end{proposition}

\begin{definition} \label{definition:graded_subobject_of_a_graded_object_of_an_abelian_category}
    Let $\mathcal{A}$ be an \CrefAndHyperrefIfExist{definition:abelian_category}{abelian category}. Let $M = \{M_n\}_{n \in \mathbb{Z}}$ be a \CrefAndHyperrefIfExist{definition:graded_object_in_an_abelian_category}{graded object} in $\mathcal{A}$. 
    A \hldef{graded subobject of $M$} is a graded object $N = \{N_n\}_{n \in \mathbb{Z}}$ in $\mathcal{A}$ together with a family of \CrefAndHyperrefIfExist{definition:monomorphism_and_epimorphism_in_categories}{monomorphisms} 
    $$\hlin{\iota_n: N_n \hookrightarrow M_n}$$
    for each $n \in \mathbb{Z}$. If we view $M$ as an object in the \CrefAndHyperrefIfExist{definition:category_of_graded_objects_in_an_abelian_category}{category of graded objects $\text{Gr}(\mathcal{A})$}, this is equivalent to saying that $N$ is a \CrefAndHyperrefIfExist{definition:subobject_of_an_object_of_an_additive_category}{subobject} of $M$ in $\text{Gr}(\mathcal{A})$.
\end{definition}
\begin{definition} \label{definition:graded_module_of_rings}
    Let $R$, $S$ be \CrefAndHyperrefIfExist{definition:ring}{(not necessarily commutative) rings}. 
    A \hldef{graded $R$-$S$-bimodule} is a collection of \CrefAndHyperrefIfExist{definition:module_of_a_ring}{$R$-$S$-bimodules} $M = \{M_n\}_{n \in \mathbb{Z}}$ indexed by the integers. 
    Equivalently, a graded $R$-$S$-bimodule is a \CrefAndHyperrefIfExist{definition:graded_object_in_an_abelian_category}{graded object} in the abelian \CrefAndHyperrefIfExist{definition:category_of_modules_and_bimodules_over_rings}{category of $R$-$S$-bimodules}.
    An element $x \in M_n$ is called a \hldef{homogeneous element of degree $n$}.
\end{definition}
\begin{definition} \label{definition:bigraded_object_in_an_abelian_category}
    Let $\mathcal{A}$ be an \CrefAndHyperrefIfExist{definition:abelian_category}{abelian category}.
    A \hldef{bigraded object in $\mathcal{A}$} is a collection of objects \hl{$E = \{E_{p,q}\}_{p,q \in \mathbb{Z}}$} in $\mathcal{A}$ indexed by pairs of integers.
\end{definition}
\begin{definition} \label{definition:differential_on_a_bigraded_object_in_an_abelian_category}
    Let $E$ be a \CrefAndHyperrefIfExist{definition:bigraded_object_in_an_abelian_category}{bigraded object} in an \CrefAndHyperrefIfExist{definition:abelian_category}{abelian category} $\mathcal{A}$. 
    
    A \hldef{differential of bidegree $(\delta_p, \delta_q)$ on $E$} is a family of morphisms in $\mathcal{A}$
    $$\hlin{d: E_{p,q} \to E_{p+\delta_p, q+\delta_q}}$$
    (or $d: E^{p,q} \to E^{p+\delta_p, q+\delta_q}$ depending on notation) defined for all $p,q \in \mathbb{Z}$, such that $d \circ d = 0$. The pair $(E, d)$ is called a \hldef{differential bigraded object}.
    
    Two specific conventions are standard in spectral sequence theory (where $r \ge 0$ is the page index):
    \begin{enumerate}
        \item \textbf{Homological convention:} The objects are denoted \hl{$E_{p,q}$}. The differential usually has bidegree \hl{$(-r, r-1)$}. Thus, $d$ \textbf{decreases} the total degree $p+q$ by $1$.
        
        \item \textbf{Cohomological convention:} The objects are denoted \hl{$E^{p,q}$}. The differential usually has bidegree \hl{$(r, 1-r)$}. Thus, $d$ \textbf{increases} the total degree $p+q$ by $1$.
    \end{enumerate}
\end{definition}
\begin{definition} \label{definition:homology_and_cohomology_of_a_differential_bigraded_object_in_an_abelian_category}
    Let $(E,d)$ be a \CrefAndHyperrefIfExist{definition:differential_on_a_bigraded_object_in_an_abelian_category}{differential bigraded object} in an \CrefAndHyperrefIfExist{definition:abelian_category}{abelian category} $\mathcal{A}$.
    Since $\mathcal{A}$ is abelian, the condition $d \circ d = 0$ implies that for every $(p,q)$, the \CrefAndHyperrefIfExist{definition:image_coimage_of_a_morphism_in_a_category}{image} of the incoming differential is a subobject of the \CrefAndHyperrefIfExist{definition:kernel_and_cokernel_of_a_morphism_in_a_category}{kernel} of the outgoing differential. We define the \hldef{homology} (or \hldef{cohomology}) \hl{$H(E, d)$} as the bigraded object of quotients $\ker(d)/\text{im}(d)$. Explicitly:

    \begin{enumerate}
        \item If $d$ has \hldef{homological bidegree} $(-r, r-1)$, then:
        $$\hlin{H_{p,q}(E, d) = \frac{\ker(d: E_{p,q} \to E_{p-r, q+r-1})}{\text{im}(d: E_{p+r, q-r+1} \to E_{p,q})}.}$$
        
        \item If $d$ has \hldef{cohomological bidegree} $(r, 1-r)$, then:
        $$\hlin{H^{p,q}(E, d) = \frac{\ker(d: E^{p,q} \to E^{p+r, q-r+1})}{\text{im}(d: E^{p-r, q+r-1} \to E^{p,q})}.}$$
    \end{enumerate}
\end{definition}


\begin{definition} \label{definition:filtration_on_an_object_of_an_abelian_category}
    Let $M$ be an object in an \CrefAndHyperrefIfExist{definition:abelian_category}{abelian category} $\mathcal{A}$. 
    A \hldef{filtration $F$ on $M$} is a family of subobjects of $M$ indexed by $\mathbb{Z}$. There are two standard conventions:

    \begin{enumerate}
        \item An \hldef{increasing filtration} (or \hldef{ascending filtration}) is denoted by subscripts $\{F_p M\}_{p \in \mathbb{Z}}$ such that:
        $$\hlin{\dots \hookrightarrow F_{p-1}M \hookrightarrow F_pM \hookrightarrow F_{p+1}M \hookrightarrow \dots \hookrightarrow M.}$$
        
        \item A \hldef{decreasing filtration} (or \hldef{descending filtration}) is denoted by superscripts $\{F^p M\}_{p \in \mathbb{Z}}$ such that:
        $$\hlin{\dots \hookrightarrow F^{p+1}M \hookrightarrow F^pM \hookrightarrow F^{p-1}M \hookrightarrow \dots \hookrightarrow M.}$$
    \end{enumerate}

    The pair $(M, F)$ is called a \hldef{filtered object}. 
\end{definition}


\begin{definition} \label{definition:exhaustive_separated_filtrations_on_an_abelian_category}
    Let $\mathcal{A}$ be an \CrefAndHyperrefIfExist{definition:abelian_category}{abelian category}. Let $(M, F)$ be a \CrefAndHyperrefIfExist{definition:filtration_on_an_object_of_an_abelian_category}{filtered object} in $\mathcal{A}$.
    
    \begin{enumerate}
        \item Let $F$ be an \textbf{increasing} filtration $\{F_p M\}_{p \in \mathbb{Z}}$.
        \begin{itemize}
            \item We say $F$ is \hldef{exhaustive} if the \CrefAndHyperrefIfExist{definition:limit_and_colimit_of_a_diagram_in_a_category}{colimit} of the diagram $\dots \hookrightarrow F_p M \hookrightarrow \dots$ exists in $\mathcal{A}$ and the canonical morphism induced by the inclusions $F_p M \hookrightarrow M$ is an isomorphism:
            $$\text{colim}_{p \to \infty} F_p M \xrightarrow{\cong} M.$$
            \item We say $F$ is \hldef{separated} (or Hausdorff) if the limit of the diagram $\dots \hookrightarrow F_p M \hookrightarrow \dots$ exists in $\mathcal{A}$ and is the zero object:
            $$\lim_{p \to -\infty} F_p M \cong 0.$$
        \end{itemize}

        \item Let $F$ be a \textbf{decreasing} filtration $\{F^p M\}_{p \in \mathbb{Z}}$.
        \begin{itemize}
            \item We say $F$ is \hldef{exhaustive} if the colimit of the diagram $\dots \hookrightarrow F^p M \hookrightarrow \dots$ exists in $\mathcal{A}$ and the canonical morphism is an isomorphism:
            $$\text{colim}_{p \to -\infty} F^p M \xrightarrow{\cong} M.$$
            \item We say $F$ is \hldef{separated} if the limit of the diagram $\dots \hookrightarrow F^p M \hookrightarrow \dots$ exists in $\mathcal{A}$ and is zero:
            $$\lim_{p \to \infty} F^p M \cong 0.$$
        \end{itemize}
    \end{enumerate}
    When $\mathcal{A}$ is the category of $R$-modules, these conditions correspond to $\bigcup F_p M = M$ and $\bigcap F_p M = 0$ (increasing case).
\end{definition}




\begin{definition} \label{definition:filtered_graded_object_in_an_abelian_category}
    Let $(M, F)$ be a \CrefAndHyperrefIfExist{definition:filtration_on_an_object_of_an_abelian_category}{(increasing or decreasing) filtered object} in an \CrefAndHyperrefIfExist{definition:abelian_category}{abelian category} $\mathcal{A}$. 
    If $M$ is a \CrefAndHyperrefIfExist{definition:graded_object_in_an_abelian_category}{graded object}, then the filtration is said to be \hldef{compatible with the grading} if each $F_p M$ (or $F^p M$) is a \CrefAndHyperrefIfExist{definition:graded_subobject_of_a_graded_object_of_an_abelian_category}{graded subobject} of $M$.
    A graded object equipped with a filtration compatible with the grading is called a \hldef{filtered graded object}.
\end{definition}



\begin{definition} \label{definition:bigraded_object_associated_to_a_filtered_graded_object}
    Let $(M, F)$ be a \CrefAndHyperrefIfExist{definition:filtered_graded_object_in_an_abelian_category}{filtered graded object} in an \CrefAndHyperrefIfExist{definition:abelian_category}{abelian category} $\mathcal{A}$. 
    The \hldef{associated graded object}, denoted \hl{$\text{Gr}^F(M)$}, is the \CrefAndHyperrefIfExist{definition:bigraded_object_in_an_abelian_category}{bigraded object} defined by the \CrefAndHyperrefIfExist{definition:short_exact_sequence_in_an_additive_category}{short exact sequences}:
    $$0 \to F_{p-1} M_{p+q} \to F_p M_{p+q} \to \text{Gr}^F_{p,q}(M) \to 0.$$
    In other words, $\text{Gr}^F_{p,q}(M)$ is the cokernel of the inclusion $F_{p-1} M_{p+q} \hookrightarrow F_p M_{p+q}$.
\end{definition}


\begin{definition} \label{definition:spectral_sequence_in_an_abelian_category}
    Let $\mathcal{A}$ be an \CrefAndHyperrefIfExist{definition:abelian_category}{abelian category}. Let $r_0$ be an integer (typically $0, 1$, or $2$).
    \begin{enumerate}
    \item A \hldef{homological spectral sequence (starting at page $r_0$) in $\mathcal{A}$} is a sequence of \CrefAndHyperrefIfExist{definition:bigraded_object_in_an_abelian_category}{bigraded objects} $E = \{E^r, d^r\}_{r \ge r_0}$, where for each $r \ge r_0$:
    \begin{enumerate}
        \item $d^r: E^r_{p,q} \to E^r_{p-r, q+r-1}$ is a \CrefAndHyperrefIfExist{definition:differential_on_a_bigraded_object_in_an_abelian_category}{differential} on $E^r$ of bidegree $(-r, r-1)$;
        \item There is an isomorphism of bigraded objects $E^{r+1} \cong H(E^r, d^r)$, i.e. for all indices $(p,q)$, we have isomorphisms $E^{r+1}_{p,q} \cong H_{p,q}(E^r, d^r)$\CrefIfExists{definition:homology_and_cohomology_of_a_differential_bigraded_object_in_an_abelian_category}.
    \end{enumerate}
    The object $E^r_{p,q}$ is called the term of bidegree $(p,q)$ on the \hldef{$r$-th page}.

    \item A \hldef{cohomological spectral sequence (starting at page $r_0$) in $\mathcal{A}$} is a sequence of \CrefAndHyperrefIfExist{definition:bigraded_object_in_an_abelian_category}{bigraded objects} $E = \{E_r, d_r\}_{r \ge r_0}$, where for each $r \ge r_0$:
    \begin{enumerate}
        \item $d_r: E_r^{p,q} \to E_r^{p+r, q-r+1}$ is a differential on $E_r$ of bidegree $(r, 1-r)$;
        \item There is an isomorphism of bigraded objects $E_{r+1} \cong H(E_r, d_r)$, i.e. for all indices $(p,q)$, we have isomorphisms $E_{r+1}^{p,q} \cong H^{p,q}(E_r, d_r)$\CrefIfExists{definition:homology_and_cohomology_of_a_differential_bigraded_object_in_an_abelian_category}.
    \end{enumerate}
    \end{enumerate}
\end{definition}


\begin{definition} \label{definition:stabilize_for_a_spectral_sequence_in_an_abelian_category_and_infinity_page}
    Let $\mathcal{A}$ be an \CrefAndHyperrefIfExist{definition:abelian_category}{abelian category}.
    \begin{enumerate}
        \item Let $E = \{E^r, d^r\}_{r \ge r_0}$ be a \CrefAndHyperrefIfExist{definition:spectral_sequence_in_an_abelian_category}{homological spectral sequence} in $\mathcal{A}$. 
        A term $E^r_{p,q}$ is said to \hldef{stabilize} if there exists an integer $r(p,q) \ge r_0$ such that for all $r \ge r(p,q)$, the differentials entering and leaving $E^r_{p,q}$ are zero:
        $$d^r: E^r_{p,q} \to E^r_{p-r, q+r-1} = 0 \quad \text{and} \quad d^r: E^r_{p+r, q-r+1} \to E^r_{p,q} = 0.$$
        In this case, we have isomorphisms $E^{r+1}_{p,q} \cong E^r_{p,q}$ for all $r \ge r(p,q)$.
        The \hldef{limit term} (or \hldef{infinity page}), denoted \hl{$E^\infty_{p,q}$}, is defined as this stable object:
        $$E^\infty_{p,q} = E^{r(p,q)}_{p,q}.$$

        \item Let $E = \{E_r, d_r\}_{r \ge r_0}$ be a \CrefAndHyperrefIfExist{definition:spectral_sequence_in_an_abelian_category}{cohomological spectral sequence} in $\mathcal{A}$.
        A term $E_r^{p,q}$ is said to \hldef{stabilize} if there exists an integer $r(p,q) \ge r_0$ such that for all $r \ge r(p,q)$, the differentials entering and leaving $E_r^{p,q}$ are zero:
        $$d_r: E_r^{p,q} \to E_r^{p+r, q-r+1} = 0 \quad \text{and} \quad d_r: E_r^{p-r, q+r-1} \to E_r^{p,q} = 0.$$
        In this case, we have isomorphisms $E_{r+1}^{p,q} \cong E_r^{p,q}$ for all $r \ge r(p,q)$.
        The \hldef{limit term} (or \hldef{infinity page}), denoted \hl{$E_\infty^{p,q}$}, is defined as this stable object:
        $$E_\infty^{p,q} = E_{r(p,q)}^{p,q}.$$
    \end{enumerate}
\end{definition}

\begin{definition} \label{definition:converges_to_a_graded_object_for_a_spectral_sequence_in_an_abelian_category}
    Let $\mathcal{A}$ be an \CrefAndHyperrefIfExist{definition:abelian_category}{abelian category} and let $H = \{H_n\}_{n \in \mathbb{Z}}$ be a \CrefAndHyperrefIfExist{definition:graded_object_in_an_abelian_category}{graded object} in $\mathcal{A}$.

    \begin{enumerate}
        \item Let $E = \{E^r, d^r\}_{r \ge r_0}$ be a \CrefAndHyperrefIfExist{definition:spectral_sequence_in_an_abelian_category}{homological spectral sequence}. We say that $E$ \hldef{converges to $H$}, denoted \hl{$E^r_{p,q} \Rightarrow H_{p+q}$}, if:
        \begin{enumerate}
            \item For every pair $(p,q)$, the term $E^r_{p,q}$ \CrefAndHyperrefIfExist{definition:stabilize_for_a_spectral_sequence_in_an_abelian_category_and_infinity_page}{stabilizes} to a limit $E^\infty_{p,q}$.
            \item Each object $H_n$ is equipped with a finite, \CrefAndHyperrefIfExist{definition:exhaustive_separated_filtrations_on_an_abelian_category}{exhaustive, and separated} \CrefAndHyperrefIfExist{definition:filtration_on_an_object_of_an_abelian_category}{increasing filtration} $F$ \CrefAndHyperrefIfExist{definition:filtered_graded_object_in_an_abelian_category}{compatible with the grading}, such that there is an isomorphism:
            $$E^\infty_{p,q} \cong \text{Gr}^F_{p,q}(H) := \frac{F_p H_{p+q}}{F_{p-1} H_{p+q}}.$$
        \end{enumerate}

        \item Let $E = \{E_r, d_r\}_{r \ge r_0}$ be a \CrefAndHyperrefIfExist{definition:spectral_sequence_in_an_abelian_category}{cohomological spectral sequence}. We say that $E$ \hldef{converges to $H$}, denoted \hl{$E_r^{p,q} \Rightarrow H^{p+q}$}, if:
        \begin{enumerate}
            \item For every pair $(p,q)$, the term $E_r^{p,q}$ \CrefAndHyperrefIfExist{definition:stabilize_for_a_spectral_sequence_in_an_abelian_category_and_infinity_page}{stabilizes} to a limit $E_\infty^{p,q}$.
            \item Each object $H^n$ (where $H$ is viewed as co-graded, $H^n = H_{-n}$) is equipped with a finite, exhaustive, and separated \CrefAndHyperrefIfExist{definition:filtration_on_an_object_of_an_abelian_category}{decreasing filtration} $F$ \CrefAndHyperrefIfExist{definition:filtered_graded_object_in_an_abelian_category}{compatible with the grading}, such that there is an isomorphism:
            $$E_\infty^{p,q} \cong \text{Gr}_F^{p,q}(H) := \frac{F^p H^{p+q}}{F^{p+1} H^{p+q}}.$$
        \end{enumerate}
    \end{enumerate}
    In either case, the object $H$ is called the \hldef{abutment of the spectral sequence}.
    Usually, $H$ is explicitly realized as the homology $H_*(C)$ or cohomology $H^*(C)$ of a filtered complex $C$.
\end{definition}

\begin{definition} \label{definition:strong_convergence__to_a_filtered_graded_object_for_a_spectral_sequence_in_an_abelian_category}
    Let $E^r_{p,q} \Rightarrow H_{p+q}$ be a \CrefAndHyperrefIfExist{definition:spectral_sequence_in_an_abelian_category}{spectral sequence} \CrefAndHyperrefIfExist{definition:converges_to_a_graded_object_for_a_spectral_sequence_in_an_abelian_category}{converging} to a \CrefAndHyperrefIfExist{definition:filtered_graded_object_in_an_abelian_category}{filtered graded object} $(H, F)$.
    The convergence is called \hldef{strong convergence} if the filtration $F$ on the \CrefAndHyperrefIfExist{definition:converges_to_a_graded_object_for_a_spectral_sequence_in_an_abelian_category}{abutment} $H$ is \CrefAndHyperrefIfExist{definition:exhaustive_separated_filtrations_on_an_abelian_category}{exhaustive and separated}.

\end{definition}
\begin{remark}
    Strong convergence ensures that the abutment $H$ is "fully" determined by the spectral sequence (up to extension problems). Without these conditions, non-zero elements in $H$ could be "invisible" to the spectral sequence (e.g., elements in $\cap F_p H$ or elements not captured by $\cup F_p H$).
    Explicitly, for an increasing filtration, strong convergence implies that $H_n$ is built from the factors $E^\infty_{p, n-p}$ via a sequence of short exact sequences:
    $$0 \to F_{p-1} H_n \to F_p H_n \to E^\infty_{p, n-p} \to 0.$$
\end{remark}



\begin{definition} \label{definition:first_quadrant_spectral_sequence}
    Let $\mathcal{A}$ be an abelian category.
    \begin{enumerate}
        \item A \CrefAndHyperrefIfExist{definition:spectral_sequence_in_an_abelian_category}{homological spectral sequence} $E = \{E^r\}_{r \ge r_0}$ is called a \hldef{first quadrant spectral sequence} if the terms $E^{r_0}_{p,q}$ vanish unless $p \ge 0$ and $q \ge 0$.
        In this case, for any fixed $(p,q)$, the differentials $d^r$ entering or leaving $E^r_{p,q}$ eventually vanish because the indices $(p-r, q+r-1)$ and $(p+r, q-r+1)$ eventually land outside the first quadrant for large $r$. Thus, every term \CrefAndHyperrefIfExist{definition:stabilize_for_a_spectral_sequence_in_an_abelian_category_and_infinity_page}{stabilizes}.
        
        \item A \CrefAndHyperrefIfExist{definition:spectral_sequence_in_an_abelian_category}{cohomological spectral sequence} $E = \{E_r\}_{r \ge r_0}$ is called a \hldef{first quadrant spectral sequence} if the terms $E_{r_0}^{p,q}$ vanish unless $p \ge 0$ and $q \ge 0$.
        Similarly, the differentials $d_r$ eventually vanish as the indices $(p+r, q-r+1)$ and $(p-r, q+r-1)$ leave the first quadrant, guaranteeing \CrefAndHyperrefIfExist{definition:stabilize_for_a_spectral_sequence_in_an_abelian_category_and_infinity_page}{stabilization}.
    \end{enumerate}
\end{definition}

\begin{definition}
    Let $\mathcal{A}$ be an \CrefAndHyperrefIfExist{definition:abelian_category}{abelian category}.
    \begin{enumerate}
        \item A \CrefAndHyperrefIfExist{definition:spectral_sequence_in_an_abelian_category}{homological spectral sequence} $E$ is called \hldef{bounded} if for every integer $n$, there are only finitely many pairs $(p,q)$ with $p+q=n$ such that $E^{r_0}_{p,q} \neq 0$.
        Boundedness ensures that the \CrefAndHyperrefIfExist{definition:filtration_on_an_object_of_an_abelian_category}{filtration} on the \CrefAndHyperrefIfExist{definition:converges_to_a_graded_object_for_a_spectral_sequence_in_an_abelian_category}{abutment} $H_n$ is finite, which is a key condition for \CrefAndHyperrefIfExist{definition:strong_convergence__to_a_filtered_graded_object_for_a_spectral_sequence_in_an_abelian_category}{strong convergence}.

        \item A \CrefAndHyperrefIfExist{definition:spectral_sequence_in_an_abelian_category}{cohomological spectral sequence} $E$ is called \hldef{bounded} if for every integer $n$, there are only finitely many pairs $(p,q)$ with $p+q=n$ such that $E_{r_0}^{p,q} \neq 0$.
        Boundedness ensures that the filtration on the abutment $H^n$ is finite.
    \end{enumerate}
\end{definition}






\TODO{

[STRUCTURAL] Exact Couple $(D, E, i, j, k)$
- A pair of bigraded modules $D, E$ and morphisms $i: D \to D$, $j: D \to E$, $k: E \to D$ forming an exact triangle
- [Purpose: the algebraic machine that generates spectral sequences]
- [Dependencies: Exact sequence, bigraded module]

[LEMMA] Derivation of a Spectral Sequence from an Exact Couple
- Given an exact couple, $d = j \circ k$ is a differential on $E$, and a derived couple $(D', E')$ exists
- [Purpose: shows how to iterate the process to get pages $E^1, E^2, \dots$]
- [Dependencies: Exact couple]

[SPECIALIZED] Filtered Chain Complex $(C_*, d, F)$
- A chain complex $C_*$ equipped with a filtration $F_p C_*$ preserved by the differential $d$
- [Purpose: the most common source of spectral sequences in nature]
- [Dependencies: Chain complex, filtered module]

[THEOREM] Spectral Sequence of a Filtered Complex
- For a filtered complex $(C_*, F)$, there exists a spectral sequence with $E^0_{p,q} = F_p C_{p+q} / F_{p-1} C_{p+q}$ and $E^1_{p,q} = H_{p+q}(E^0_{p,q})$ converging to $H_*(C)$
- [Purpose: the fundamental existence theorem used in applications]
- [Dependencies: Filtered chain complex, convergence]

[SPECIALIZED] Double Complex (Bicomplex) $C_{p,q}$
- A bigraded module with two commuting (or anticommuting) differentials $d_h: C_{p,q} \to C_{p-1,q}$ and $d_v: C_{p,q} \to C_{p,q-1}$
- [Purpose: a specific type of filtered complex that yields two spectral sequences]
- [Dependencies: Chain complex]

[STRUCTURAL] Total Complex of a Double Complex $\text{Tot}(C)_n$
- The chain complex defined by $\text{Tot}(C)_n = \bigoplus_{p+q=n} C_{p,q}$ with differential $D = d_h + d_v$
- [Purpose: the object to which the double complex spectral sequences converge]
- [Dependencies: Double complex]

[THEOREM] Spectral Sequences of a Double Complex
- There are two spectral sequences associated to $C_{p,q}$, one filtering by rows ($^{I}E$) and one by columns ($^{II}E$), both converging to $H_*(\text{Tot}(C))$
- [Purpose: allows computation of unknown homology by comparing two different filtrations]
- [Dependencies: Double complex, total complex, convergence]

[SPECIALIZED] Edge Homomorphisms
- Natural maps $H_n \to E^\infty_{n,0} \hookrightarrow E^2_{n,0}$ and $E^2_{0,n} \twoheadrightarrow E^\infty_{0,n} \to H_n$ (for first quadrant)
- [Purpose: relates the spectral sequence terms directly to the abutment]
- [Dependencies: Convergence, first quadrant spectral sequence]

[STRUCTURAL] Five-Term Exact Sequence
- For a first quadrant spectral sequence converging to $H_*$, the sequence $H_2 \to E^2_{2,0} \xrightarrow{d^2} E^2_{0,1} \to H_1 \to E^2_{1,0} \to 0$ is exact
- [Purpose: extracts concrete low-dimensional data without analyzing the full sequence]
- [Dependencies: Edge homomorphisms, convergence]

[SPECIALIZED] Transgression $\tau$
- The differential $d^n: E^n_{n,0} \to E^n_{0, n-1}$ in a first quadrant spectral sequence
- [Purpose: the longest possible non-zero differential connecting the axes]
- [Dependencies: First quadrant spectral sequence]

[SPECIALIZED] Collapse of a Spectral Sequence
- A situation where $E^r_{p,q} = E^\infty_{p,q}$ for some finite $r$ (often $r=2$)
- [Purpose: simplifying condition where the calculation terminates early]
- [Dependencies: Limit page]

[SPECIALIZED] Extension Problem
- The algebraic problem of reconstructing the group $H_n$ from the graded pieces $E^\infty_{p,q}$ where $p+q=n$
- [Purpose: explains why convergence to $E^\infty$ is not explicitly convergence to $H_*$]
- [Dependencies: Convergence, filtration]

}

\appendix


\section{Categories enriched in monoidal categories}

\begin{definition} \label{definition:monoidal_category}
A \hldef{monoidal category} is a \CrefAndHyperrefIfExist{definition:category}{(large) category} $\mathcal{C}$ equipped with:
\begin{itemize}
    \item a \CrefAndHyperrefIfExist{definition:n_ary_functor}{bifunctor} $\otimes : \mathcal{C} \times \mathcal{C} \to \mathcal{C}$\CrefIfExists{definition:product_category_of_a_family_of_categories} (called the \hldef{tensor product});
    \item an object $\mathbb{I} \in \mathrm{Ob}(\mathcal{C})$ (often called the \hldef{unit object}); common notations for the unit object include \hl{$\mathbb{I}$} and \hl{$\mathds{1}$}.
    \item natural isomorphisms (\hldef{associator}) $\alpha_{X,Y,Z} : (X \otimes Y) \otimes Z \to X \otimes (Y \otimes Z)$ for all $X, Y, Z \in \mathcal{C}$;
    \item natural isomorphisms (\hldef{left and right unitors}) $\lambda_X : \mathbb{I} \otimes X \to X$, $\rho_X : X \otimes \mathbb{I} \to X$ for all $X \in \mathcal{C}$;
\end{itemize}
% \TODO{TODO: add the pentagon and triangle coherence diagrams}

% such that the pentagon and triangle coherence diagrams commute.

such that the following coherence diagrams commute:

\bigskip

\textbf{Pentagon coherence:} For all $W,X,Y,Z \in \mathcal{C}$, the diagram
\[
\begin{tikzcd}[column sep=huge]
((W \otimes X) \otimes Y) \otimes Z \arrow[r, "\alpha_{W\otimes X, Y, Z}"] \arrow[d, "\alpha_{W,X,Y} \otimes \mathrm{id}_Z"'] & (W \otimes X) \otimes (Y \otimes Z) \arrow[r, "\alpha_{W,X,Y \otimes Z}"] & W \otimes (X \otimes (Y \otimes Z)) \\
(W \otimes (X \otimes Y)) \otimes Z \arrow[rr, "\alpha_{W,X \otimes Y, Z}"'] & & W \otimes ((X \otimes Y) \otimes Z) \arrow[u, "\mathrm{id}_W \otimes \alpha_{X,Y,Z}"'].
\end{tikzcd}
\]

\bigskip

\textbf{Triangle coherence:} For all $X,Y \in \mathcal{C}$, the diagram
\[
\begin{tikzcd}[column sep=large]
(X \otimes \mathbb{I}) \otimes Y \arrow[r, "\alpha_{X, \mathbb{I}, Y}"] \arrow[dr, "\rho_X \otimes \mathrm{id}_Y"'] & X \otimes (\mathbb{I} \otimes Y) \arrow[d, "\mathrm{id}_X \otimes \lambda_Y"] \\
& X \otimes Y.
\end{tikzcd}
\]
\end{definition}


\begin{definition}[Category enriched in a monoidal category] \label{definition:category_enriched_in_a_monoidal_category}
Let $(\mathcal{V}, \otimes, \mathbf{1})$ be a \CrefAndHyperrefIfExist{definition:monoidal_category}{monoidal category}. A \hldef{category enriched in $\mathcal{V}$} (or a \hldef{$\mathcal{V}$-enriched category} or a \hldef{$\mathcal{V}$-category}) $\mathcal{C}$ consists of the following data:
\begin{itemize}
    \item A class \hl{$\operatorname{Ob}(\mathcal{C})$} of \hldef{objects}. As with \hyperrefIfExists{definition:category}{regular categories}, we may write \hl{$X \in \operatorname{Ob}(\mathcal{C})$} or \hl{$X \in \calC$} to mean that $X$ is an object of $\calC$.  
    \item For each pair of objects $X, Y \in \operatorname{Ob}(\mathcal{C})$, an object \hl{$\underline{\operatorname{Hom}}_{\mathcal{C}}(X,Y) \in \operatorname{Ob}(\mathcal{V})$} of \hldef{morphisms}; it is an object of the monoidal category $\mathcal{V}$. It is also often denoted by notations such as \hl{$\calC(X,Y)$}, \hl{$\Hom(X,Y) = \Hom_\calC(X,Y)$}, or \hl{$\operatorname{Mor}(X,Y) = \operatorname{Mor}_{\calC}(X,Y)$}.
    \item For each triple $X,Y,Z \in \operatorname{Ob}(\mathcal{C})$, a \hldef{composition morphism} 
    $$\mu_{X,Y,Z} : \underline{\operatorname{Hom}}_{\mathcal{C}}(Y,Z) \otimes \underline{\operatorname{Hom}}_{\mathcal{C}}(X,Y) \to \underline{\operatorname{Hom}}_{\mathcal{C}}(X,Z).$$
    It is a morphism in $\mathcal{V}$.
    \item For each object $X$, a \hldef{unit morphism} \hl{$\eta_X : \mathbf{1} \to \underline{\operatorname{Hom}}_{\mathcal{C}}(X,X)$} in $\mathcal{V}$.
\end{itemize}
These data satisfy the following axioms:
\begin{itemize}
    \item (Associativity) For all $W,X,Y,Z \in \operatorname{Ob}(\mathcal{C})$, the following diagram in $\mathcal{V}$ commutes:
    $$
    \begin{tikzcd}[column sep=large,row sep=large]
    \bigl(\underline{\operatorname{Hom}}_{\mathcal{C}}(Z,W) \otimes \underline{\operatorname{Hom}}_{\mathcal{C}}(Y,Z)\bigr) \otimes \underline{\operatorname{Hom}}_{\mathcal{C}}(X,Y) \ar[r,"\alpha"] \ar[d,"\mu \otimes \mathrm{id}"]
    & \underline{\operatorname{Hom}}_{\mathcal{C}}(Z,W) \otimes \bigl(\underline{\operatorname{Hom}}_{\mathcal{C}}(Y,Z) \otimes \underline{\operatorname{Hom}}_{\mathcal{C}}(X,Y)\bigr) \ar[d,"\mathrm{id} \otimes \mu"] \\
    \underline{\operatorname{Hom}}_{\mathcal{C}}(Y,W) \otimes \underline{\operatorname{Hom}}_{\mathcal{C}}(X,Y) \ar[d,"\mu"] 
    & \underline{\operatorname{Hom}}_{\mathcal{C}}(Z,W) \otimes \underline{\operatorname{Hom}}_{\mathcal{C}}(X,Z) \ar[d,"\mu"] \\
    \underline{\operatorname{Hom}}_{\mathcal{C}}(X,W) \ar[r,equal] & \underline{\operatorname{Hom}}_{\mathcal{C}}(X,W)
    \end{tikzcd}
    $$
    where $\alpha$ is the associativity constraint in $\mathcal{V}$.
    \item (Unit) For all $X,Y \in \operatorname{Ob}(\mathcal{C})$, the following diagrams commute:
    \begin{center}
    \begin{tikzcd}[column sep=large]
    \mathbf{1} \otimes \underline{\operatorname{Hom}}_{\mathcal{C}}(X,Y) \ar[r,"\eta_Y \otimes \mathrm{id}"] \ar[dr,"\lambda"']
    & \underline{\operatorname{Hom}}_{\mathcal{C}}(Y,Y) \otimes \underline{\operatorname{Hom}}_{\mathcal{C}}(X,Y) \ar[d,"\mu"] \\
    & \underline{\operatorname{Hom}}_{\mathcal{C}}(X,Y)
    \end{tikzcd}
    \begin{tikzcd}[column sep=large]
    \underline{\operatorname{Hom}}_{\mathcal{C}}(X,Y) \otimes \mathbf{1} \ar[r,"\mathrm{id} \otimes \eta_X"] \ar[dr,"\rho"']
    & \underline{\operatorname{Hom}}_{\mathcal{C}}(X,Y) \otimes \underline{\operatorname{Hom}}_{\mathcal{C}}(X,X) \ar[d,"\mu"] \\
    & \underline{\operatorname{Hom}}_{\mathcal{C}}(X,Y)
    \end{tikzcd}
    \end{center}
    % $$
    % \quad\quad
    % $$
    where $\lambda$ and $\rho$ are the left and right unit constraints in $\mathcal{V}$.
\end{itemize}
\end{definition}


\begin{definition} \label{definition:enriched_functor_between_categories_enriched_in_a_}
Let $\mathcal{V}$ be a \CrefAndHyperrefIfExist{definition:monoidal_category}{monoidal category} and let $\calC$ and $\calD$ be \CrefAndHyperrefIfExist{definition:category_enriched_in_a_monoidal_category}{categories enriched in $\calV$}. A \hldef{$\mathcal{V}$-functor between $\mathcal{C}$ and $\mathcal{D}$} or an \hldef{enriched functor between $\calC$ and $\calD$}, written
$$F : \mathcal{C} \to \mathcal{D},$$
consists of:
\begin{itemize}
    \item a function on objects $F : \mathrm{Ob}(\mathcal{C}) \to \mathrm{Ob}(\mathcal{D})$,
    \item for all $A,B \in \mathrm{Ob}(\mathcal{C})$, morphisms in $\mathcal{V}$,
        $$F_{A,B} : \mathcal{C}(A,B) \to \mathcal{D}(FA,FB),$$
\end{itemize}
such that the following diagrams in $\mathcal{V}$ commute:
\begin{align*}
\mathcal{C}(B,C) \otimes \mathcal{C}(A,B) &\xrightarrow{\circ_{\mathcal{C}}} \mathcal{C}(A,C) \\
\downarrow{F_{B,C} \otimes F_{A,B}} \quad & \quad \downarrow{F_{A,C}} \\
\mathcal{D}(FB,FC) \otimes \mathcal{D}(FA,FB) &\xrightarrow{\circ_{\mathcal{D}}} \mathcal{D}(FA,FC)
\end{align*}
and for all $A \in \mathrm{Ob}(\mathcal{C})$, the unit compatibility condition holds:
$$\iota_{FA} = F_{A,A} \circ \iota_A.$$
\end{definition}

\begin{definition} \label{definition:product_of_categories_enriched_in_a_monoidal_category}
    More generally, let \(\mathcal{M} = (\mathcal{M}, \otimes, \mathds{1})\) be a \CrefAndHyperrefIfExist{definition:monoidal_category}{monoidal category}, and let \(\{\mathcal{C}_i\}_{i \in I}\) be a family of \CrefAndHyperrefIfExist{definition:category_enriched_in_a_monoidal_category}{$\mathcal{M}$-enriched categories} indexed by a class $I$ such that $\calM$ is closed under all \CrefAndHyperrefIfExist{definition:product_and_coproduct_of_objects_in_a_category}{products} indexed by $I$. The \hldef{product $\mathcal{M}$-enriched category of the family}, denoted
    $$ \hlin{\prod_{i \in I} \mathcal{C}_i}, $$
    is defined as follows:
    \begin{itemize}
        \item The class of objects is
        $$
        \mathrm{Ob}\Big(\prod_{i \in I} \mathcal{C}_i\Big) = \prod_{i \in I} \mathrm{Ob}(\mathcal{C}_i),
        $$
        i.e., an object is a family \((A_i)_{i \in I}\) with \(A_i \in \mathrm{Ob}(\mathcal{C}_i)\).

        \item For two objects \((A_i)_i\) and \((B_i)_i\), the hom-object in \(\mathcal{M}\) is
        $$
        \left(\prod_{i \in I} \mathcal{C}_i\right)\big((A_i)_i, (B_i)_i\big) := \prod_{i \in I} \mathcal{C}_i(A_i, B_i).
        $$
        Here, the product on the right is taken inside \(\mathcal{M}\).

        \item Composition morphisms in \(\mathcal{M}\) are defined componentwise using the composition morphisms of each \(\mathcal{C}_i\):
        $$
        \circ_{(A_i),(B_i),(C_i)}: \left(\prod_i \mathcal{C}_i(B_i, C_i)\right) \otimes \left(\prod_i \mathcal{C}_i(A_i, B_i)\right) \to \prod_i \mathcal{C}_i(A_i, C_i),
        $$
        given by the universal property of products and the componentwise compositions
        $$
        \circ_i : \mathcal{C}_i(B_i, C_i) \otimes \mathcal{C}_i(A_i, B_i) \to \mathcal{C}_i(A_i, C_i).
        $$

        \item For each object \((A_i)_i\), the identity morphism is given by the family
        $$
        (\mathrm{id}_{A_i})_i : \mathds{1} \to \prod_i \mathcal{C}_i(A_i, A_i),
        $$
        where $\mathds{1}$ is the unit object of \(\mathcal{M}\).
    \end{itemize}

In case that \(I\) is finite, the notation \(\times\) may be used for product \(\mathcal{M}\)-enriched categories, e.g. \hl{$\mathcal{C}_i \times \mathcal{C}_j$} denotes the product of two \(\mathcal{M}\)-enriched categories \(\mathcal{C}_i \times \mathcal{C}_j\).
\end{definition}


\begin{definition}[n-ary Functor of \(\mathcal{V}\)-Enriched Categories] \label{definition:n_ary_functor_of_categories_enriched_in_a_monoidal_category}
Let \(\mathcal{V}\) be a \CrefAndHyperrefIfExist{definition:monoidal_category}{monoidal category} that is closed under finite \CrefAndHyperrefIfExist{definition:product_and_coproduct_of_objects_in_a_category}{products}, and let \(I\) be a finite set with \(|I|=n\). Suppose \(\{\mathcal{C}_i\}_{i \in I}\) are \CrefAndHyperrefIfExist{definition:category_enriched_in_a_monoidal_category}{$\mathcal{V}$-enriched categories}, and \(\mathcal{D}\) is another \(\mathcal{V}\)-enriched category. An \hldef{$n$-ary enriched functor} (or \hldef{multivariable enriched functor}) from \(\{\mathcal{C}_i\}_{i\in I}\) to \(\mathcal{D}\) is a \CrefAndHyperrefIfExist{definition:enriched_functor_between_categories_enriched_in_a_}{$\mathcal{V}$-enriched functor}
\[ F : \prod_{i \in I} \mathcal{C}_i \to \mathcal{D}, \]
where \(\prod_{i\in I} \mathcal{C}_i\) is the \CrefAndHyperrefIfExist{definition:product_of_categories_enriched_in_a_monoidal_category}{$\mathcal{V}$-enriched product category}. 

That is, \(F\) assigns:
\begin{itemize}
    \item to each object \(((A_i)_{i \in I})\) in \(\prod_{i \in I} \mathcal{C}_i\), an object \(F((A_i)_{i \in I})\) in \(\mathcal{D}\),
    \item to each family of objects \(( (A_i)_i, (B_i)_i )\), a morphism in \(\mathcal{V}\)
    \[
    F_{(A_i),(B_i)} : \prod_{i\in I} \mathcal{C}_i(A_i,B_i) \to \mathcal{D}(F((A_i)_i), F((B_i)_i)),
    \]
    respecting the enriched composition and unit axioms.
\end{itemize}

When \(n=2\), an $n$-ary enriched functor is called a \emph{bifunctor} enriched over \(\mathcal{V}\), etc.

\end{definition}


\begin{definition} \label{definition:opposite_of_a_category_enriched_in_a_monoidal_category}
Let $\mathcal{V}$ be a \CrefAndHyperrefIfExist{definition:monoidal_category}{monoidal category} and $\mathcal{C}$ a \CrefAndHyperrefIfExist{definition:category_enriched_in_a_monoidal_category}{$\mathcal{V}$-category}. The \hldef{opposite $\mathcal{V}$-category of $\mathcal{C}$}, denoted by \hl{$\mathcal{C}^{\mathrm{op}}$}, is defined as follows:
\begin{itemize}
    \item Objects: $\mathrm{Ob}(\mathcal{C}^{\mathrm{op}}) = \mathrm{Ob}(\mathcal{C})$.
    \item Hom-objects: For all $A,B \in \mathrm{Ob}(\mathcal{C})$, set
    $$\mathcal{C}^{\mathrm{op}}(A,B) = \mathcal{C}(B,A).$$
    \item Composition: For all $A,B,C \in \mathrm{Ob}(\mathcal{C})$, the composition morphism
    $$\circ^{\mathrm{op}}_{A,B,C} : \mathcal{C}^{\mathrm{op}}(B,C) \otimes \mathcal{C}^{\mathrm{op}}(A,B) \to \mathcal{C}^{\mathrm{op}}(A,C)$$
    is defined by the composite in $\mathcal{V}$
    $$\mathcal{C}(C,B) \otimes \mathcal{C}(B,A) \xrightarrow{s_{\mathcal{C}(C,B),\mathcal{C}(B,A)}} \mathcal{C}(B,A) \otimes \mathcal{C}(C,B) \xrightarrow{\circ_{C,B,A}} \mathcal{C}(C,A).$$
    \item Units: For each $A \in \mathrm{Ob}(\mathcal{C})$, the unit morphism in $\mathcal{C}^{\mathrm{op}}$ is the same as that in $\mathcal{C}$,
    $$\iota_A^{\mathrm{op}} = \iota_A : I \to \mathcal{C}(A,A) = \mathcal{C}^{\mathrm{op}}(A,A).$$
\end{itemize}
The associativity and unit axioms for $\mathcal{C}^{\mathrm{op}}$ follow from those for $\mathcal{C}$ and the naturality and symmetry properties of $s_{X,Y}$.
\end{definition}


\section{Schemes}


\begin{definition}[Locally ringed space] \label{definition:locally_ringed_space_on_a_topological_space}
A \hldef{locally ringed space} is a \CrefAndHyperrefIfExist{definition:ringed_space}{ringed space} $(X, \mathcal{O}_X)$ such that for every point $x \in X$, the \CrefAndHyperrefIfExist{definition:stalk_of_a_presheaf_on_a_topological_space_at_a_point}{stalk $\mathcal{O}_{X,x}$} is a \CrefAndHyperrefIfExist{definition:local_ring}{local ring}.  
The unique \CrefAndHyperrefIfExist{definition:prime_and_maximal_ideal_of_a_ring}{maximal ideal} of $\mathcal{O}_{X,x}$ is often denoted \hl{$\mathfrak{m}_x$} and is called the \hldef{maximal ideal at $x$}.
\end{definition}


\begin{definition}[Scheme] \label{definition:scheme}
    A \hldef{scheme} is a \CrefAndHyperrefIfExist{definition:locally_ringed_space_on_a_topological_space}{locally ringed space} $(X, \mathcal{O}_X)$ that admits an open cover $\{U_i\}_{i \in I}$ such that each $(U_i, \mathcal{O}_X|_{U_i})$ is \CrefAndHyperrefIfExist{definition:morphism_of_locally_ringed_spaces}{isomorphic (as a locally ringed space)} to an \CrefAndHyperrefIfExist{definition:affine_scheme}{affine scheme $(\mathrm{Spec}(A_i), \mathcal{O}_{\mathrm{Spec}(A_i)})$} for some \CrefAndHyperrefIfExist{ring}{commutative ring} $A_i$.  
    In other words, a scheme is a locally ringed space locally isomorphic to affine schemes.

    
\end{definition}

\begin{definition} \label{definition:quasi_coherent_sheaf_on_a_general_scheme}
Let $X$ be a \CrefAndHyperrefIfExist{definition:scheme}{scheme} and $\mathcal{F}$ a \CrefAndHyperrefIfExist{definition:module_over_a_sheaf_of_rings_on_a_site}{sheaf of $\mathcal{O}_X$-modules}. The sheaf $\mathcal{F}$ is called \hldef{quasi-coherent} if for every open affine subset $U = \operatorname{Spec} A$ of $X$, there exists an $A$-module $M_U$ such that the restriction $\mathcal{F}|_U$ is isomorphic to the sheaf \CrefAndHyperrefIfExist{definition:quasi_coherent_sheaf_on_an_affine_scheme}{$\widetilde{M_U}$} associated to $M_U$.

For a scheme $X$, we denote by
$$\hlin{\operatorname{QCoh}(X)}$$
the full subcategory of $\text{Mod}(\mathcal{O}_X)$ consisting of all quasi-coherent $\mathcal{O}_X$-modules.


\TextIfExists{definition:quasi_coherent_sheaf_on_a_ringed_space}{
Equivalently, a quasi-coherent sheaf on the \CrefAndHyperrefIfExist{definition:scheme}{scheme} $\Spec A$ is a \CrefAndHyperrefIfExist{definition:quasi_coherent_sheaf_on_a_ringed_space}{quasi-coherent sheaf} on $(X, \calO_X)$ \CrefAndHyperrefIfExist{definition:scheme}{as a} \CrefAndHyperrefIfExist{definition:ringed_space}{ringed space}. 
}


\end{definition}




\section{Miscellaneous definitions}


\begin{definition}[Groups] \label{definition:group}
A \hldef{group} is a pair $(G,\cdot)$ where $G$ is a set and $\cdot : G \times G \to G$ is a binary operation, subject to the following conditions:

1. (Associativity) For all $g,h,k \in G$ one has 
$$ (g \cdot h) \cdot k = g \cdot (h \cdot k). $$

2. (Identity element) There exists an element \hl{$e \in G$} such that for all $g \in G$, 
$$ e \cdot g = g \cdot e = g. $$

3. (Inverse element) For all $g \in G$ there exists an element \hl{$g^{-1} \in G$} such that 
$$ g \cdot g^{-1} = g^{-1} \cdot g = e. $$

The element $e$ is called the \hldef{identity element of $G$}, and $g^{-1}$ is called the \hldef{inverse of $g$}.

Equivalently, a group is a \CrefAndHyperrefIfExist{definition:monoid}{monoid} with inverse elements.

\TextIfExists{definition:group_object_in_a_category_with_a_final_object}{Equivalently, a group is a \CrefAndHyperrefIfExist{definition:group_object_in_a_category_with_a_final_object}{group object} in the \CrefAndHyperrefIfExist{definition:category_of_sets}{category of sets}.}

A group $(G, \cdot)$ is often simply written as $G$, when the notation for the binary operation $\cdot$ is clear. 

An \hldef{abelian group} or synonymously, a \hldef{commutative group}, is a group $(G,\cdot)$ whose binary operation $\cdot$ is \CrefAndHyperrefIfExist{definition:commutative_binary_operation}{\hldef{abelian} or \hldef{commutative}}, i.e. satisfies 
$$g \cdot h = h \cdot g$$
for all $g,h \in G$. 


\TextIfExists{definition:module_of_a_ring}{An abelian group is equivalent to a $\bbZ$-module.}

\end{definition}


\begin{definition}\label{definition:ring}
    A \hldef{ring} is a triple $(R, +, \cdot)$ where 
    \begin{enumerate}
        \item $(R,+)$ is a \CrefAndHyperrefIfExist{definition:group}{commutative group}, and
        \item $(R, \cdot)$ is a \CrefAndHyperrefIfExist{definition:monoid}{monoid}. 
        \item $\cdot$ is distributive over $+$, i.e. for all $a,b,c \in R$, we have
        $$a \cdot (b+c) = a \cdot b + a \cdot c \quad \text{and} \quad (a+b) \cdot c = a \cdot c + b \cdot c.$$
    \end{enumerate}

    Equivalently, a ring is a triple $(R,+,\cdot)$ where $+,\cdot: R \times R \to R$ are binary operations satisfying
    \begin{enumerate}
        \item $(a+b)+c = a+(b+c)$ and $(ab)c = a(bc)$ for all $a,b,c \in R$
        \item There exists an element \hl{$0 \in R$} such that $a+0 = a = 0 + a$ for all $a \in R$.
        \item For every $a \in R$, there exists an element \hl{$-a \in R$} such that $a+(-a) = 0 = (-a) + a$ for all $a \in R$.
        \item There exists an element \hl{$1 \in R$} such that $a \cdot 1 = a = 1 \cdot a$ for all $a \in R$.
        \item For all $a,b,c \in R$, we have
        $$a \cdot (b+c) = a \cdot b + a \cdot c \quad \text{and} \quad (a+b) \cdot c = a \cdot c + b \cdot c.$$
    \end{enumerate} 

    The operation $+$ is often called \hldef{addition} and the operation $\cdot$ is often called \hldef{multiplication}. Accordingly, the identity element $0$ of $+$ is often called the \hldef{additive identity} and the identity element $1$ of $\cdot$ is often called the \hldef{multiplicative identity}.

    % If $\cdot$ is additionally a \CrefAndHyperrefIfExist{definition:commutative_binary_operation}{commutative operation}, i.e. $a \cdot b = b \cdot a$ for all $a,b \in R$, then we call the ring \hldef{commutative}.  


\end{definition}
\begin{remark}
    Some writers might not require a ring to have a multiplicative identity element, i.e. would define a ring so that $(R,+)$ is a commutative group, $(R, \cdot)$ is a semigroup, and $\cdot$ is distributive over $+$. Such writers would call the notion of ring in \Cref{definition:ring} a \hldef{unitary ring} to emphasize the existence of the multiplicative identity $1$. 
\end{remark}


\begin{definition} \label{definition:module_of_a_ring}
Let $R$ be a \CrefAndHyperrefIfExist{definition:ring}{not-necessarily commutative ring}. 
\begin{enumerate}
    \item A \hldef{left $R$-module} is an abelian group $(M,+)$ together with an operation $R \times M \to M$, denoted $(r,m) \mapsto rm$, such that for all $r,s \in R$ and $m,n \in M$:
    \begin{itemize}
        \item $r(m+n) = rm + rn$,
        \item $(r+s)m = rm + sm$,
        \item $(rs)m = r(sm)$,
        \item $1_R m = m$ where $1_R$ is the multiplicative identity of $R$.
    \end{itemize}

    \item A \hldef{right $R$-module} is defined similarly as an abelian group $(M,+)$ with an operation $M \times R \to M$, denoted $(m,r) \mapsto mr$, such that for all $r,s \in R$ and $m,n \in M$:
    \begin{itemize}
        \item $(m+n)r = mr + nr$,
        \item $m(r+s) = mr + ms$,
        \item $m(rs) = (mr)s$,
        \item $m 1_R = m$.
    \end{itemize}

    \item Let $R$ and $S$ be  (not necessarily commutative) \CrefAndHyperrefIfExist{definition:ring}{rings}.

    An \hldef{$R$-$S$-bimodule} (or an \hldef{$R$-$S$-module} or an $(R,S)$-module, etc.)is an \CrefAndHyperrefIfExist{definition:group}{abelian group} $(M,+)$ equipped with
    \begin{enumerate}
        \item a left action of $R$:
        $$\hlin{R \times M \to M, \quad (r,m) \mapsto r \cdot m},$$
        making $M$ a \CrefAndHyperrefIfExist{definition:module_of_a_ring}{left $R$-module},
        \item a right action of $S$:
        $$\hlin{M \times S \to M, \quad (m,s) \mapsto m \cdot s},$$
        making $M$ a right $S$-module,
    \end{enumerate}
    such that the left and right actions commute; that is, for all $r \in R$, $s \in S$, and $m \in M$,
    $$ r \cdot (m \cdot s) = (r \cdot m) \cdot s.  $$

    \item A \hldef{two-sided $R$-module} (or \hldef{$R$-bimodule}) is an $R$-$R$-bimodule.
    
    % an abelian group $(M,+)$ which is simultaneously a left $R$-module and a right $R$-module, such that $(rm)s = r(ms)$ for all $r,s \in R$, $m \in M$. Equivalently, a two-sided $R$-module is an \hldef{$R$-$R$-bimodule}\CrefIfExists{definition:module_of_a_ring}


\end{enumerate}
If $R$ is a \CrefAndHyperrefIfExist{definition:commutative_ring}{commutative ring}, then a left/right $R$-module can automatically be regarded as a two-sided $R$-module. As such, we simply talk about \hldef{$R$-modules} in this case. 

Any abelian group is equivalent to a two-sided $\bbZ$-module. Moreover, any left $R$-module is equivalent to an \CrefAndHyperrefIfExist{definition:module_of_a_ring}{$R-\bbZ$-bimodule} and any right $R$-module is equivalent to an \CrefAndHyperrefIfExist{definition:module_of_a_ring}{$\bbZ-R$-bimodule}. Given a left/right/two-sided $R$-module, its \hldef{natural bimodule structure} will refer to its structure as a $R$-$\bbZ$/$\bbZ$-$R$/$R$-$R$ bimodule. In this way, many definitions associated with the notions of left/right/two-sided $R$-modules can be defined as special cases for definitions for $R$-$S$-bimodules.
\end{definition}


\begin{definition} \label{definition:ideal_of_a_ring}
Let $R$ be a (not necessarily commutative, possibly nonunital) \CrefAndHyperrefIfExist{definition:ring}{ring}.  
A \hldef{left ideal of $R$} is a subset $I \subseteq R$ such that
\begin{itemize}
    \item $(I,+)$ is an additive \CrefAndHyperrefIfExist{definition:subgroup_of_a_group}{subgroup} of $(R,+)$,
    \item $RI \subseteq I$, i.e., for all $r \in R$ and $x \in I$, one has $rx \in I$.
\end{itemize}
Similarly, a \hldef{right ideal of $R$} is a subset $I \subseteq R$ such that
\begin{itemize}
    \item $(I,+)$ is an additive subgroup of $(R,+)$,
    \item $IR \subseteq I$, i.e., for all $r \in R$ and $x \in I$, one has $xr \in I$.
\end{itemize}
A \hldef{two-sided ideal} (or simply an \hldef{ideal}) of $R$ is a subset $I \subseteq R$ which is both a left ideal and a right ideal of $R$. We denote by \hl{$I \unlhd R$} the relation expressing that $I$ is a two-sided ideal of $R$.

\TextIfExists{definition:module_of_a_ring}{Equivalently, an left/right/two-sided ideal of $R$ is a \CrefAndHyperrefIfExist{definition:submodule_of_a_module_over_a_ring}{submodule} of $R$ as an \CrefAndHyperrefIfExist{definition:module_of_a_ring}{$R$-module}.}

A left/right/two-sided ideal is said to be \hldef{proper} if it is strictly contained in $R$.

Note that every left or right ideal of a commutative ring is a two-sided ideal.
\end{definition}

\begin{definition} \label{definition:homomorphism_of_modules_over_a_ring}
Let $R,S$ be \CrefAndHyperrefIfExist{definition:ring}{(not-necessarily commutative) rings}. 
\begin{enumerate}
    \item Let $M$ and $N$ be \CrefAndHyperrefIfExist{definition:module_of_a_ring}{$R$-$S$-bimodules}. A function $\varphi: M \to N$ is called an \hldef{$R$-$S$-bimodule homomorphism} or \hldef{$R$-$S$-linear} if it is a \CrefAndHyperrefIfExist{definition:group_homomorphism}{group homomorphism} of the underlying abelian groups of $M$ and $N$ and respects the scalar actions as follows: 
    for all $m_1,m_2 \in M$, $r \in R$, and $s \in S$,
        \begin{align*}
        % \varphi(m_1 + m_2) &= \varphi(m_1) + \varphi(m_2), \\
        \varphi(r \cdot m_1) &= r \cdot \varphi(m_1), \\
        \varphi(m_1 \cdot s) &= \varphi(m_1) \cdot s.
        \end{align*}

    \item Let $M$ and $N$ be \CrefAndHyperrefIfExist{definition:module_of_a_ring}{left/right/two-sided $R$-modules}. A function $\varphi: M \to N$ is called a \hldef{left/right/two-sided $R$-module homomorphism} if it is an bimodule homomorphism on the \CrefAndHyperrefIfExist{definition:module_of_a_ring}{natural bimodule structures} of $M$ and $N$.
    %  $R$-$\bbZ$/$\bbZ$-$R$/$R$-$R$-bimodule homomorphism. 
     Such a function is also called \hldef{$R$-linear}.

\end{enumerate}

Modules and homomorphisms of a fixed type (i.e. $R$-$S$-bimodules or left/righ/two-sided $R$-modules) form a \CrefAndHyperrefIfExist{definition:locally_small_category}{locally small} \CrefAndHyperrefIfExist{definition:category}{category}.

% Let $M$ and $N$ be \CrefAndHyperrefIfExist{definition:module_of_a_ring}{left/right/two-sided $R$-modules or $R$-$S$-bidmodules}. 

% \begin{enumerate}
%     \item A function $\varphi : M \to N$ is called a \hldef{left/right/two-sided module homomorphism} or \hldef{$R$-linear} if it is additive (more precisely, a \CrefAndHyperrefIfExist{definition:group_homomorphism}{group homomorphism} of \CrefAndHyperrefIfExist{definition:group}{abelian groups}) and respects the scalar action(s) as follows: for all $m_1,m_2 \in M$, $r \in R$, and $s \in S$,
%     \begin{align*}
%     % \varphi(m_1 + m_2) &= \varphi(m_1) + \varphi(m_2), \\
%     \varphi(r \cdot m_1) &= r \cdot \varphi(m_1), \\
%     \varphi(m_1 \cdot s) &= \varphi(m_1) \cdot s.
%     \end{align*}

%     \item 
% \end{enumerate}

% \begin{enumerate}
%     \item If $M$ and $N$ are left $R$-modules, then for all $m_1,m_2 \in M$ and $r \in R$,
%     \begin{align*}
%     \varphi(m_1 + m_2) &= \varphi(m_1) + \varphi(m_2), \\
%     \varphi(r \cdot m_1) &= r \cdot \varphi(m_1).
%     \end{align*}

%     \item If $M$ and $N$ are right $R$-modules, then for all $m_1,m_2 \in M$ and $r \in R$,
%     \begin{align*}
%     \varphi(m_1 + m_2) &= \varphi(m_1) + \varphi(m_2), \\
%     \varphi(m_1 \cdot r) &= \varphi(m_1) \cdot r.
%     \end{align*}

%     \item If $M$ and $N$ are two-sided $R$-modules, then for all $m_1,m_2 \in M$, and $r_1,r_2 \in R$,
%     \begin{align*}
%     \varphi(m_1 + m_2) &= \varphi(m_1) + \varphi(m_2), \\
%     \varphi(r_1 \cdot m_1) &= r_1 \cdot \varphi(m_1), \\
%     \varphi(m_1 \cdot r_2) &= \varphi(m_1) \cdot r_2.
%     \end{align*}

%     \item If $M$ and $N$ are $(R,S)$-bimodules, then for all $m_1,m_2 \in M$, $r \in R$, and $s \in S$,
%     \begin{align*}
%     \varphi(m_1 + m_2) &= \varphi(m_1) + \varphi(m_2), \\
%     \varphi(r \cdot m_1) &= r \cdot \varphi(m_1), \\
%     \varphi(m_1 \cdot s) &= \varphi(m_1) \cdot s.
%     \end{align*}
% \end{enumerate}
\end{definition}
% \begin{definition} \label{definition:kernel_image_cokernel_of_a_module_homomorphism_over_a_ring}
% Let $R,S$ be \CrefAndHyperrefIfExist{definition:ring}{(not-necessarily commutative) rings with unity}, and let $M,N$ be \CrefAndHyperrefIfExist{definition:module_of_a_ring}{$R$-$S$-bimodules}. Let 
% $$\varphi : M \to N$$ 
% be a \CrefAndHyperrefIfExist{definition:homomorphism_of_modules_over_a_ring}{homomorphism of $R$-$S$-bimodules}. We define:

% \begin{enumerate}
%     \item The \hldef{kernel of $\varphi$} is the \CrefAndHyperrefIfExist{definition:submodule_of_a_module_over_a_ring}{submodule of $M$} given by
%     $$\hlin{\ker(\varphi) := \{ m \in M \mid \varphi(m) = 0 \} \subseteq M.}$$

%     \item The \hldef{image of $\varphi$} is the submodule of $N$ given by
%     $$\hlin{\operatorname{im}(\varphi) := \{ \varphi(m) \mid m \in M \} \subseteq N.}$$

%     \item The \hldef{cokernel of $\varphi$} is the \CrefAndHyperrefIfExist{definition:quotient_module_of_a_module_by_a_module_of_a_ring}{quotient $R$-module of $N$} defined by
%     $$\hlin{\operatorname{coker}(\varphi) := N / \operatorname{im}(\varphi).}$$


% \end{enumerate}
% It is not difficult to see that each of these are indeed $R$-$S$ bimodules. In case $M$ and $N$ are left/right/two-sided $R$-modules, the \hldef{kernel, image, and cokernel} of a module homomorphism $\varphi: M \to N$ are respectively defined to be the kernel, image, and cokernel for the \CrefAndHyperrefIfExist{definition:module_of_a_ring}{natural bimodule structures} of $M$ and $N$.
% \TODO{describe how these are categorical kernels/images/cokernels}
% \end{definition}

\begin{definition} \label{definition:kernel_image_cokernel_coimage_of_a_module_homomorphism}
Let $R,S$ be \CrefAndHyperrefIfExist{definition:ring}{(not-necessarily commutative) rings with unity}, and let $M,N$ be \CrefAndHyperrefIfExist{definition:module_of_a_ring}{$R$-$S$-bimodules}. Let 
$$\varphi : M \to N$$ 
be a \CrefAndHyperrefIfExist{definition:homomorphism_of_modules_over_a_ring}{homomorphism of $R$-$S$-bimodules}. We define:

\begin{enumerate}
    \item The \hldef{kernel of $\varphi$} is the \CrefAndHyperrefIfExist{definition:submodule_of_a_module_over_a_ring}{submodule of $M$} given by
    $$\hlin{\ker(\varphi) := \{ m \in M \mid \varphi(m) = 0 \} \subseteq M.}$$

    \item The \hldef{image of $\varphi$} is the submodule of $N$ given by
    $$\hlin{\operatorname{im}(\varphi) := \{ \varphi(m) \mid m \in M \} \subseteq N.}$$

    \item The \hldef{cokernel of $\varphi$} is the \CrefAndHyperrefIfExist{definition:quotient_module_of_a_module_by_a_module_of_a_ring}{quotient module of $N$} defined by
    $$\hlin{\operatorname{coker}(\varphi) := N / \operatorname{im}(\varphi).}$$

    \item The \hldef{coimage of $\varphi$} is the \CrefAndHyperrefIfExist{definition:quotient_module_of_a_module_by_a_module_of_a_ring}{quotient module of $M$} defined by
    $$\hlin{\operatorname{coim}(\varphi) := M / \ker(\varphi).}$$
\end{enumerate}
It is not difficult to see that each of these are indeed $R$-$S$ bimodules. In case $M$ and $N$ are left/right/two-sided $R$-modules, the \hldef{kernel, image, cokernel, and coimage} of a module homomorphism $\varphi: M \to N$ are respectively defined to be the kernel, image, cokernel, and coimage for the \CrefAndHyperrefIfExist{definition:module_of_a_ring}{natural bimodule structures} of $M$ and $N$.

The kerel, cokernel, image, and coimage of $f$ are respectively the categorical \CrefAndHyperrefIfExist{definition:kernel_and_cokernel_of_a_morphism_in_a_category}{kernel, cokernel}, \CrefAndHyperrefIfExist{definition:image_coimage_of_a_morphism_in_a_category}{image, and coimage} (\Cref{lemma:kernel_cokernel_image_coimage_of_modules_over_rings_are_categorical}).

\end{definition}

\begin{definition} \label{definition:category_of_modules_and_bimodules_over_rings}
    Let $R$ and $S$ be \CrefAndHyperrefIfExist{definition:ring}{(not necessarily commutative) rings}.     
    \begin{enumerate}
        \item The \hldef{category of $(R,S)$-bimodules} (or $R$-$S$-bimodules), denoted by notations such as \hl{${}_R\mathsf{Mod}_S$}, is the category whose objects are \CrefAndHyperrefIfExist{definition:module_of_a_ring}{$(R,S)$-bimodules} and whose \CrefAndHyperrefIfExist{definition:homomorphism_of_modules_over_a_ring}{$R$-$S$-bimodule homomorphisms}.

        \item The \hldef{category of left $R$-modules}, denoted by notations such as \hl{${}_R\mathsf{Mod}$} or \hl{$R-\mathbf{Mod}$}, is the category ${}_R\mathsf{Mod}_\bbZ$, i.e. the category whose objects are \CrefAndHyperrefIfExist{definition:module_of_a_ring}{left $R$-modules} and whose morphisms are \CrefAndHyperrefIfExist{definition:homomorphism_of_modules_over_a_ring}{left $R$-linear maps}.

        \item The \hldef{category of right $R$-modules}, denoted by notations such as \hl{$\mathsf{Mod}_R$} or \hl{$\mathbf{Mod}-R$}, is the category ${}_\bbZ\mathsf{Mod}_R$, i.e. the category whose objects are \CrefAndHyperrefIfExist{definition:module_of_a_ring}{right $R$-modules} and whose morphisms are \CrefAndHyperrefIfExist{definition:homomorphism_of_modules_over_a_ring}{right $R$-linear maps}.
    \end{enumerate}

    The category of bimodules can be canonically identified with module categories over \CrefAndHyperrefIfExist{definition:tensor_product_of_a_ring_and_an_algebra_over_a_ring}{tensor product rings}:
    \begin{itemize}
        \item ${}_R\mathsf{Mod}_S$ is isomorphic to the category of left modules over the ring $R \otimes_{\mathbb{Z}} S^{\operatorname{op}}$.
        \item ${}_R\mathsf{Mod}_S$ is isomorphic to the category of right modules over the ring $R^{\operatorname{op}} \otimes_{\mathbb{Z}} S$.
    \end{itemize}
    Consequently, standard module-theoretic concepts (such as projective objects, injective objects, and flat objects) in ${}_R\mathsf{Mod}_S$ correspond exactly to the respective concepts in ${}_{R \otimes S^{\operatorname{op}}}\mathsf{Mod}$.

    % \paragraph{Relation to One-Sided Modules.}
    Note that there are canonical isomorphisms of categories:
    \[
        {}_R\mathsf{Mod} \cong {}_R\mathsf{Mod}_{\mathbb{Z}} \quad \text{and} \quad \mathsf{Mod}_S \cong {}_{\mathbb{Z}}\mathsf{Mod}_S.
    \]
    That is, left $R$-modules are exactly $(R, \mathbb{Z})$-bimodules, and right $S$-modules are exactly $(\mathbb{Z}, S)$-bimodules.
\end{definition}
\begin{definition}[Coproduct of Modules] \label{definition:coproduct_of_modules_of_rings}
    Let $R$ and $S$ be \CrefAndHyperrefIfExist{definition:ring}{(not necessarily commutative) rings}, and let $\{ M_i \}_{i \in I}$ be a (possibly infinite but small) family of $(R,S)$-bimodules.
    % \CrefAndHyperrefIfExist{definition:module_of_a_ring}{left $R$-modules, right $R$-modules, or $(R,S)$-bimodules}, respectively.

    The \hldef{coproduct (direct sum) of the family $\{M_i\}_{i \in I}$}, denoted by \hl{$\bigoplus_{i \in I} M_i$}, is constructed as
    $$ \bigoplus_{i \in I} M_i := \left\{ (m_i)_{i \in I} \in \prod_{i \in I} M_i \mid m_i = 0 \text{ for all but finitely many } i \in I \right\} $$
    \CrefIfExists{definition:product_of_modules_of_rings}
    consisting of all tuples with only finitely many nonzero entries.

    Addition and scalar multiplication in $\bigoplus_{i \in I} M_i$ are defined componentwise as in the \CrefAndHyperrefIfExist{definition:product_of_modules_of_rings}{direct product}:
    $$ (m_i)_{i \in I} + (n_i)_{i \in I} := (m_i + n_i)_{i \in I}, \qquad r \cdot (m_i)_{i \in I} \cdot s := (r \cdot m_i \cdot s)_{i \in I}, \qquad r \in R,\, s \in S.  $$
    In all cases, the zero element is $(0)_{i \in I}$, and additive inverses are given by $-(m_i)_{i \in I} := (-m_i)_{i \in I}$. 

    Note that we can define the coproduct of a family $\{M_i\}_{i \in I}$ of left/right/two-sided $R$-modules by taking the \CrefAndHyperrefIfExist{definition:module_of_a_ring}{natural bimodule structure} of each module.
    
    \TODO{submodule}
    Note that $\bigoplus_{i \in I} M_i$ is a submodule of $\prod_{i \in I} M_i$. Moreover, $\bigoplus_{i \in I} M_i$ is the \CrefAndHyperrefIfExist{definition:product_and_coproduct_of_objects_in_a_category}{coproduct} in the appropriate \CrefAndHyperrefIfExist{definition:category_of_modules_and_bimodules_over_rings}{category of modules}.

    For finitely many modules $M_1,\ldots,M_n$, the direct sum $\bigoplus_{j=1}^{n} M_j$, which may also be written as \hl{$M_1 \oplus \cdots \oplus M_n$}, is simply the usual \CrefAndHyperrefIfExist{definition:product_of_modules_of_rings}{Cartesian product $\prod_{j=1}^n M_j$} of the modules, as every tuple automatically has only finitely many nonzero entries.
\end{definition}
\begin{definition}[Opposite ring] \label{definition:opposite_ring_of_a_ring}
Let $R = (R, +, \cdot, 0, 1)$ be a \CrefAndHyperrefIfExist{definition:ring}{ring} with addition $+$, multiplication $\cdot$, additive identity $0$, and multiplicative identity $1$ (not necessarily commutative).

The \hldef{opposite ring of $R$}, denoted $R^{\mathrm{op}}$, is the ring with the same underlying set $R$ and the same addition $+$ and additive identity $0$, but with multiplication defined by
$$ r \star s := s \cdot r $$
for all $r, s \in R$.

That is, multiplication in $R^\mathrm{op}$ is the multiplication of $R$ reversed in order.

If $R$ is \CrefAndHyperrefIfExist{definition:commutative_ring}{commutative}, then $R$ and $R^{\op}$ are naturally isomorphic to each other.
\end{definition}
\begin{definition}[Tensor product of bimodules] \label{definition:tensor_product_of_bimodules_of_rings}
Let $R,S,T$ be \CrefAndHyperrefIfExist{definition:ring}{(not necessarily commutative) rings}, let $M$ be an \CrefAndHyperrefIfExist{definition:module_of_a_ring}{$R$-$S$ bimodule}, and let $N$ be an $S$-$T$ bimodule. In the \CrefAndHyperrefIfExist{definition:free_abelian_group_generated_by_a_set}{free abelian group} $\bbZ[M \times N]$ generated by the \CrefAndHyperrefIfExist{definition:product_of_sets}{Cartesian product $M \times N$}, let $U$ be the subgroup generated by elements of the form
\TODO{subgroup generated}
\begin{align*}
&(m+m',n) - (m,n) - (m',n),\\
&(m,n+n') - (m,n) - (m,n'),\\
&(m \cdot s, n) - (m, s \cdot n),
\end{align*}
for all $m,m' \in M$, $n,n' \in N$, and $s \in S$. The \hldef{tensor product of $M$ and $N$ over $S$} is the \CrefAndHyperrefIfExist{definition:quotient_of_a_group_by_a_normal_subgroup}{quotient} abelian group
$$M \otimes_S N := \mathbb{Z}[M \times N] / U.$$
The image of an element of the form $(m,n) \in M \times N$ in $M \otimes_S N$ is denoted \hl{$m \otimes n$} and called a \hldef{pure tensor}. In general, the elements of $M \otimes_S N$ are finite sums 
$$\sum_{i=1}^n m_i \otimes n_i \quad m_i \in M, n_i \in N$$
of pure tensors. Thus, the pure tensors satisfy the following relations:
\begin{align*}
    (m + m') \otimes n &= m \otimes n + m' \otimes n \\ 
    m \otimes (n + n') &= m \otimes n + m \otimes n' \\
    (m \cdot s) \otimes n &= m \otimes (s \cdot n)
\end{align*}

This tensor product becomes naturally an $R$-$T$ bimodule with left action and right action defined by
\begin{align*}
r \cdot (m \otimes n) &= (r \cdot m) \otimes n, \\
(m \otimes n) \cdot t &= m \otimes (n \cdot t),
\end{align*}
for all $r \in R$, $t \in T$, $m \in M$, and $n \in N$.

Inductively, given rings $R_0,\ldots,R_k$ and $R_{i-1}-R_i$-bimodules $M_i$ for $i = 1,\ldots,k$, we may speak of the tensor product
$$M_0 \otimes_{R_1} M_1 \otimes_{R_2} \cdots \otimes_{R_{k-1}} M_k;$$
tensor products are associative\TODO{}, so parentheses are not strictly needed to notate them. Its \hldef{pure tensors} are elements of the form $m_0 \otimes m_1 \otimes \cdots \otimes m_k$ for $m_i \in M_i$, and its general elements are finite sums
$$\sum_{j=1}^n m_{0j} \otimes m_{1j} \otimes \cdots m_{kj} \quad m_{ij} \in M_i.$$
of pure tensors. It also has a natural $R_0-R_k$-bimodule structure.

\TextIfExists{definition:n_ary_additive_functor_between_additive_categories}{In general, $(M_0,\ldots,M_k) \mapsto M_0 \otimes_{R_1} M_1 \otimes_{R_2} \cdots \otimes_{R_{k-1}} M_k$ defines a \CrefAndHyperrefIfExist{definition:n_ary_additive_functor_between_additive_categories}{$(k+1)$-ary additive functor}
$${}_{R_0}\mathbf{Mod}_{R_1} \times \cdots \times {}_{R_{k-1}}\mathbf{Mod}_{R_k} \to {}_{R_0} \mathbf{Mod}_{R_k}$$
(\Cref{theorem:the_category_of_R_S_bimodules_is_a_grothendieck_abelian_category_and_AB4_star}).}


Given a ring $R$ and a two-sided $R$-module $M$, we may also speak of the \hldef{$n$-fold tensor product} \hl{$M^{\otimes n} = M^{\otimes_R n}$}

\end{definition}
\begin{definition} \label{definition:tensor_product_of_a_ring_and_an_algebra_over_a_ring}
    Let $k$ be a \CrefAndHyperrefIfExist{definition:ring}{not necessarily commutative ring}. Let $R$ and $S$ be \CrefAndHyperrefIfExist{definition:ring_homomorphism}{$k$-rings} (not necessarily commutative). Assume that at least one of $R$ or $S$ is a \CrefAndHyperrefIfExist{definition:algebra_of_a_ring}{$k$-algebra}. The \hldef{tensor product ring} \hl{$R \otimes_k S$} is the $k$-module \CrefAndHyperrefIfExist{definition:tensor_product_of_bimodules_of_rings}{$R \otimes_k S$} equipped with a multiplication defined on simple tensors by
    \[
        (r_1 \otimes s_1) \cdot (r_2 \otimes s_2) = (r_1 r_2) \otimes (s_1 s_2)
    \]
    and extended by linearity. This multiplication is well-defined and makes $R \otimes_k S$ into a $k$-ring under the ring homomorphism 
    $$k \to R \otimes_k S, \quad a \mapsto a \otimes 1 = 1 \otimes a.$$
    The unit element is $1_R \otimes 1_S$.

    In this ring, the \CrefAndHyperrefIfExist{definition:subring_of_a_ring}{subrings} $R \otimes 1$ and $1 \otimes S$ commute with each other; that is, for all $r \in R$ and $s \in S$,
    \[
        (r \otimes 1) \cdot (1 \otimes s) = r \otimes s = (1 \otimes s) \cdot (r \otimes 1).
    \]

    If $R$ and $S$ are both $k$-algebras, then $R \otimes_k S$ is also a $k$-algebra.
\end{definition}
\begin{theorem} \label{theorem:category_of_R_S_bimodules_is_equivalent_to_the_category_of_left_R_otimes_Z_S_op_modules}
    Let $R$ and $S$ be  (not necessarily commutative) \CrefAndHyperrefIfExist{definition:ring}{rings}. 
    The \CrefAndHyperrefIfExist{definition:category_of_modules_and_bimodules_over_rings}{category of} \CrefAndHyperrefIfExist{definition:module_of_a_ring}{$R$-$S$-bimodules} is \CrefAndHyperrefIfExist{definition:equivalence_of_categories}{equivalent} to the category of left $R \otimes_\bbZ S^{\op}$\CrefIfExists{definition:tensor_product_of_a_ring_and_an_algebra_over_a_ring}\CrefIfExists{definition:opposite_ring_of_a_ring} modules.
\end{theorem}


\begin{definition}[Grothendieck Universe] \label{definition:grothendieck_universe}
    Let $U$ be a set. We say $U$ is a \hldef{Grothendieck universe} (or just a \hldef{universe}) if the following conditions hold:
    \begin{enumerate}
        \item If $x \in U$ and $y \in x$, then $y \in U$ (transitivity).
        \item If $x,y \in U$, then $\{x,y\} \in U$ (closed under pair formation).
        \item If $x \in U$, then the power set $\mathcal{P}(x) \in U$.
        \item If $I \in U$ and $(x_\alpha)_{\alpha \in I}$ is a family with each $x_\alpha \in U$, then $\bigcup_{\alpha \in I} x_\alpha \in U$.
    \end{enumerate}
    A set $X$ is called \hldef{$U$-small} or a \hldef{$U$-set} if $X \in U$.
\end{definition}


\begin{definition}[Reflecting a type of morphism] \label{definition:reflects_a_type_of_morphism_for_a_functor_between_categories}
Let $F : \mathcal{C} \to \mathcal{D}$ be a \CrefAndHyperrefIfExist{definition:functor_between_categories}{functor between (large) categories}, and let $\mathcal{P}$ be a property of morphisms (or more generally a property of sequences or families of morphisms) that is stable under \CrefAndHyperrefIfExist{definition:isomorphism_in_a_category}{isomorphism} (e.g. \CrefAndHyperrefIfExist{definition:monomorphism_and_epimorphism_in_categories}{monomorphism, epimorphism}, isomorphism, etc.). We say that $F$ \hldef{reflects $\mathcal{P}$-morphisms} if for every morphism $f : x \to y$ in $\mathcal{C}$, whenever $F(f)$ has property $\mathcal{P}$ in $\mathcal{D}$, it follows that $f$ has property $\mathcal{P}$ in $\mathcal{C}$.
\end{definition}


\begin{definition} \label{definition:adjoint_functors_between_categories_unit_counit_of_adjoint_functors}
Let $\mathcal{C}$ and $\mathcal{D}$ be \CrefAndHyperrefIfExist{definition:category}{categories}. Let $F : \mathcal{C} \to \mathcal{D}$ and $G : \mathcal{D} \to \mathcal{C}$ be functors. 

An \hldef{adjunction between $F$ and $G$} consists of two \CrefAndHyperrefIfExist{definition:natural_transformation_between_functors_between_categories}{natural transformations}: $\eta : \mathrm{Id}_{\mathcal{C}} \implies GF$ (the \hldef{unit}), and  $\varepsilon : FG \implies \mathrm{Id}_{\mathcal{D}}$ (the \hldef{counit})

These must satisfy the triangle identities: For every object $X \in \mathcal{C}$ 
and $Y \in \mathcal{D}$, 
$$\varepsilon_{FX} \circ F(\eta_X) = \text{id}_{FX}$$
$$G(\varepsilon_Y) \circ \eta_{GY} = \text{id}_{GY}.$$
In diagrammatic form, the triangle identities assert that the following are commutative diagrams:
\begin{center}
\begin{tikzcd}
F(X) \arrow[r, "F(\eta_X)"] \arrow[rd, "\text{id}_{F(X)}"'] & FGF(X) \arrow[d, "\varepsilon_{F(X)}"] \\
& F(X)
\end{tikzcd}
\begin{tikzcd}
G(Y) \arrow[r, "\eta_{G(Y)}"] \arrow[rd, "\text{id}_{G(Y)}"'] & GFG(Y) \arrow[d, "G(\varepsilon_Y)"] \\
& G(Y)
\end{tikzcd}
\end{center}

We say that $F$ is a \hldef{left adjoint to $G$} and $G$ is a \hldef{right adjoint to $F$} (written \hl{$F \dashv G$}). 

% for every object $A$ in $\mathcal{C}$ and $B$ in $\mathcal{D}$ there is a \CrefAndHyperrefIfExist{definition:natural_transformation_between_functors_between_categories}{natural isomorphism}
% \begin{align*}
% \operatorname{Hom}_{\mathcal{D}}(F(A), B) \cong \operatorname{Hom}_{\mathcal{C}}(A, G(B))
% \end{align*}
% that is natural in both $A$ and $B$.


In the case that $\mathcal{C}$ and $\mathcal{D}$ are \CrefAndHyperrefIfExist{definition:locally_small_category}{locally small} categories (or $U$-locally small categories if a \CrefAndHyperrefIfExist{definition:grothendieck_universe}{universe} $U$ is available), we have an adjunction $F \dashv G$ if and only if for every object $X$ in $\mathcal{C}$ and $Y$ in $\mathcal{D}$ there is a \CrefAndHyperrefIfExist{definition:natural_transformation_between_functors_between_categories}{natural isomorphism}
\begin{align*}
\operatorname{Hom}_{\mathcal{D}}(F(X), Y) \cong \operatorname{Hom}_{\mathcal{C}}(X, G(Y))
\end{align*}
that is natural in both $X$ and $Y$. In this case, the \hldef{unit of the adjunction} is the natural transformation $\eta : \mathrm{Id}_{\mathcal{C}} \Rightarrow G F$ such that, 
\begin{enumerate}
    \item for every $X \in \calC$, the morphism $\eta_X: X \to GF(X)$ (each called a \hldef{unit morphism}) in $\calC$ is obtained as the image of $\id_{F(X)}$ via the adjoint isomorphism
    $$\Hom_\calD(F(X), F(X)) \cong \Hom_\calC(X, GF(X)). $$

    \item for every $Y \in \calD$, the morphism $\epsilon_Y: FG(Y) \to Y$ (each called a \hldef{counit morphism}) in $\calD$ is obtained as the image of $\id_{G(Y)}$ via the adjoint isomorphism 
    $$\Hom_\calC(G(Y), G(Y)) \cong \Hom_\calD(FG(Y), Y).$$

\end{enumerate}


% Let $F : \mathcal{C} \to \mathcal{D}$ and $G : \mathcal{D} \to \mathcal{C}$ be functors. 
% $F$ is a \hldef{left adjoint to $G$} and $G$ is a \hldef{right adjoint to $F$} (written \hl{$F \dashv G$}) if for every object $A$ in $\mathcal{C}$ and $B$ in $\mathcal{D}$ there is a \CrefAndHyperrefIfExist{definition:natural_transformation_between_functors_between_categories}{natural isomorphism}
% \begin{align*}
% \operatorname{Hom}_{\mathcal{D}}(F(A), B) \cong \operatorname{Hom}_{\mathcal{C}}(A, G(B))
% \end{align*}
% that is natural in both $A$ and $B$.
\end{definition}
\begin{definition} \label{definition:continuous_cocontinuous_functor_between_categories}
    Let $F: \mathcal{C} \to \mathcal{D}$ be a \CrefAndHyperrefIfExist{definition:functor_between_categories}{functor} between \CrefAndHyperrefIfExist{definition:category}{(large) categories} $\mathcal{C}$ and $\mathcal{D}$.
    \begin{itemize}
        \item The functor $F$ is called \hldef{continuous} if it preserves all small limits that exist in $\mathcal{C}$. That is, for every small category $\mathcal{J}$ and every diagram $D: \mathcal{J} \to \mathcal{C}$ having a limit in $\mathcal{C}$, $F$ \CrefAndHyperrefIfExist{definition:functor_preserving_limit_colimit_of_a_diagram}{preserves the limit} of $D$.
        \item The functor $F$ is called \hldef{cocontinuous} if it preserves all small colimits that exist in $\mathcal{C}$. That is, for every small category $\mathcal{J}$ and every diagram $D: \mathcal{J} \to \mathcal{C}$ having a colimit in $\mathcal{C}$, $F$ \CrefAndHyperrefIfExist{definition:functor_preserving_limit_colimit_of_a_diagram}{preserves the colimit} of $D$.
    \end{itemize}
\end{definition}


\begin{definition}[Noetherian conditions for a ring] \label{definition:noetherian_ring}
Let $R$ be a \CrefAndHyperrefIfExist{definition:ring}{ring}. We say:
\begin{itemize}
    \item $R$ is \hldef{left-Noetherian} if every ascending chain of \CrefAndHyperrefIfExist{definition:ideal_of_a_ring}{left ideals} of $R$ stabilizes, i.e., if for any chain
    $$I_1 \subseteq I_2 \subseteq I_3 \subseteq \cdots$$
    of left ideals, there exists $n$ such that $I_m = I_n$ for all $m \geq n$.
    
    \item $R$ is \hldef{right-Noetherian} if every ascending chain of right ideals of $R$ stabilizes.
    
    \item $R$ is \hldef{Noetherian} if it is both left-Noetherian and right-Noetherian.
\end{itemize}
\TODO{finitely generated ideal}
If $R$ is \CrefAndHyperrefIfExist{definition:commutative_ring}{commutative}, then $R$ is Noetherian if and only if every ideal is finitely generated.
\end{definition}



\begin{definition}[Finitely generated modules and bimodules]  \label{definition:finitely_generated_modules_over_rings}

    Let $R$ and $S$ be \CrefAndHyperrefIfExist{definition:ring}{(not necessarily commutative) rings}. 
    \begin{enumerate}
        \item An $R$-$S$-bimodule $M$ is \hldef{finitely generated} if it has a \CrefAndHyperrefIfExist{definition:span_a_module_over_a_ring_for_elements_of_the_module}{finite spanning set}. 
        
        % there exists a finite set $\{m_1,\ldots,m_n\} \subseteq M$ such that the \CrefAndHyperrefIfExist{definition:submodule_of_a_module_generated_by_elements}{submodule of $M$ generated by} this set is $M$ itself or equivalently every element $m \in M$ is a linear combination of $m_1,\ldots,m_n$.

        \item A left/right/two-sided $R$-module is \hldef{finitely generated} if has a \CrefAndHyperrefIfExist{definition:span_a_module_over_a_ring_for_elements_of_the_module}{finite spanning set}, or equivalently if its \CrefAndHyperrefIfExist{definition:module_of_a_ring}{natural bimodule structure} is finitely generated.
    \end{enumerate}
\end{definition}


\begin{proposition}[Pointwise Computation of Limits and Colimits in Functor Categories] \label{proposition:limits_and_colimits_in_functor_categories_may_be_computed_pointwise}
    Let $\mathcal{C}$ and $\mathcal{D}$ be categories. Let $\operatorname{Fun}(\mathcal{C}, \mathcal{D})$ denote the category of functors from $\mathcal{C}$ to $\mathcal{D}$ (also denoted $\mathcal{D}^{\mathcal{C}}$). Let $\mathcal{J}$ be a small category and let $F: \mathcal{J} \to \operatorname{Fun}(\mathcal{C}, \mathcal{D})$ be a diagram of functors, denoted by $j \mapsto F_j$.
    
    \begin{enumerate}
        \item \textbf{Limits are computed pointwise:}
        Suppose that for every object $C \in \mathcal{C}$, the limit of the diagram $j \mapsto F_j(C)$ exists in $\mathcal{D}$. Then the limit of the diagram $F$ exists in $\operatorname{Fun}(\mathcal{C}, \mathcal{D})$ and is computed pointwise. That is, there is an isomorphism in $\operatorname{Fun}(\mathcal{C}, \mathcal{D})$:
        \[
        \left( \lim_{j \in \mathcal{J}} F_j \right)(C) \cong \lim_{j \in \mathcal{J}} (F_j(C)).
        \]
        The action of this limit functor on a morphism $f: C \to C'$ in $\mathcal{C}$ is the unique morphism induced by the family $\{ F_j(f) \}_{j \in \mathcal{J}}$ via the universal property of limits in $\mathcal{D}$.
        
        \item \textbf{Colimits are computed pointwise:}
        Suppose that for every object $C \in \mathcal{C}$, the colimit of the diagram $j \mapsto F_j(C)$ exists in $\mathcal{D}$. Then the colimit of the diagram $F$ exists in $\operatorname{Fun}(\mathcal{C}, \mathcal{D})$ and is computed pointwise. That is, there is an isomorphism in $\operatorname{Fun}(\mathcal{C}, \mathcal{D})$:
        \[
        \left( \operatorname{colim}_{j \in \mathcal{J}} F_j \right)(C) \cong \operatorname{colim}_{j \in \mathcal{J}} (F_j(C)).
        \]
        The action of this colimit functor on a morphism $f: C \to C'$ in $\mathcal{C}$ is the unique morphism induced by the family $\{ F_j(f) \}_{j \in \mathcal{J}}$ via the universal property of colimits in $\mathcal{D}$.
    \end{enumerate}
\end{proposition}


\begin{definition} \label{definition:cartesian_product_of_two_objects_in_a_category_over_an_object}
    Let $\mathcal{C}$ be a \CrefAndHyperrefIfExist{definition:category}{category}, let $Z$ be an object, and let $X, Y$ be objects of $\mathcal{C}$ \CrefAndHyperrefIfExist{definition:category_of_objects_over_under_a_fixed_object_in_a_category}{over} $Z$, i.e. morphisms $X \to Z$ and $Y \to Z$ are fixed. A \hldef{cartesian product of $X$ and $Y$ over $Z$ in $\mathcal{C}$} (or \hldef{fiber product} or \hldef{pullback diagram}) is an object, often denoted by \hl{$X \times_Z Y$}, with \hldef{projection morphisms} $X \times_Z Y \to X$ and $X \times_Z Y \to Y$ that are universal. 
    More precisely, for any object $T$ of $\mathcal{C}$ and morphisms $f_X : T \to X$, $f_Y : T \to Y$, there exists a unique morphism $u : T \to X \times_Z Y$ such that the following diagram commutes:
        \begin{center}
        \begin{tikzcd}
            T \ar[rd, dotted, "u" ] \ar[rrd, "f_X", bend left] \ar[ddr, "f_Y", bend right] & & \\
            & X \times_Z Y \ar[r] \ar[d] &  \ar[d] X \\
            & Y \ar[r] & Z
        \end{tikzcd}
        \end{center}
        Equivalently, $X \times_Z Y$ is the \CrefAndHyperrefIfExist{definition:limit_and_colimit_of_a_diagram_in_a_category}{limit} of the \CrefAndHyperrefIfExist{definition:diagram_in_a_category_indexed_by_a_small_category}{diagram}
        \begin{center}
            \begin{tikzcd}
            & X \ar[d] \\
            Y \ar[r] & Z
            \end{tikzcd}
        \end{center}
        in $\calC$. 

        The commutative diagram 
        \begin{center}
        \begin{tikzcd}
        X \times_Z Y \ar[r] \ar[d] & X \ar[d] \\
        Y \ar[r] & Z
        \end{tikzcd} 
        \end{center}
        may be referred to as a \hldef{cartesian square}.

\end{definition}

\begin{definition} \label{definition:basis_and_grothendieck_pretopology_for_a_grothendieck_topology_on_a_category}
Let $\mathcal{C}$ be a \CrefAndHyperrefIfExist{definition:category}{category}.
A \hldef{basis for a Grothendieck topology} (also called a \hldef{Grothendieck pretopology} or simply a \hldef{pretopology}) on $\mathcal{C}$ is a collection of families $K(U)$ of morphisms for each object $U \in \mathcal{C}$, called \hldef{coverings} or \hldef{covering families}, satisfying the following axioms:
\begin{enumerate}
    \item \textbf{(Identity)} For every isomorphism $U' \to U$, the singleton family $\{U' \to U\}$ is in $K(U)$.
    \item \textbf{(Base Change)} If $\{U_i \to U\}_{i \in I}$ is a covering family in $K(U)$ and $V \to U$ is any morphism in $\mathcal{C}$, then the \CrefAndHyperrefIfExist{definition:cartesian_product_of_two_objects_in_a_category_over_an_object}{fiber products} $U_i \times_U V$ exist, and the family of projections $\{U_i \times_U V \to V\}_{i \in I}$ is in $K(V)$.
    \item \textbf{(Composition)} If $\{U_i \to U\}_{i \in I}$ is in $K(U)$ and for each $i \in I$, $\{V_{ij} \to U_i\}_{j \in J_i}$ is in $K(U_i)$, then the composite family $\{V_{ij} \to U_i \to U\}_{i \in I, j \in J_i}$ is in $K(U)$.
\end{enumerate}

\end{definition}
\begin{definition} \label{definition:grothendieck_topology_generated_by_a_pretopology}
Let $\mathcal{C}$ be a category equipped with a \CrefAndHyperrefIfExist{definition:basis_and_grothendieck_pretopology_for_a_grothendieck_topology_on_a_category}{Grothendieck pretopology} $K$. The \hldef{Grothendieck topology generated by $K$}, denoted \hl{$J_K$}, is the smallest \CrefAndHyperrefIfExist{definition:grothendieck_topology_on_a_category_site_covering_sieve_topologically_generating_family}{Grothendieck topology} on $\mathcal{C}$ such that every family in $K(U)$ is a covering family for $J_K$.

Explicitly, a \CrefAndHyperrefIfExist{definition:sieve_on_an_object_in_a_category}{sieve} $S$ on an object $U$ belongs to $J_K(U)$ if and only if there exists a \CrefAndHyperrefIfExist{definition:basis_and_grothendieck_pretopology_for_a_grothendieck_topology_on_a_category}{covering family} $\{U_i \to U\}_{i \in I} \in K(U)$ such that for every $i \in I$, the morphism $U_i \to U$ belongs to $S$.

The condition that $S$ contains the family $\{U_i \to U\}$ is equivalent to saying that the sieve generated by this family is a sub-sieve of $S$.
\end{definition}
\begin{definition} \label{definition:symmetric_monoidal_category}
A \hldef{symmetric monoidal category} is a \hyperrefIfExists{definition:monoidal_category}{monoidal category} $(\mathcal{C}, \otimes, \mathbb{I})$ together with a natural isomorphism (symmetry)
$$\hlin{\gamma_{X,Y}: X \otimes Y \xrightarrow{\cong} Y \otimes X}$$
for all $X, Y \in \mathcal{C}$, such that for all $X, Y, Z \in \mathcal{C}$ the following holds:
\begin{itemize}
    \item $\gamma_{Y,X} \circ \gamma_{X,Y} = \mathrm{id}_{X \otimes Y}$ (involutivity);
    % % \TODO{TODO: add the hexagon and symmetry coherence diagrams}
    % \item the hexagon and symmetry coherence diagrams commute.
        \item the \textbf{hexagon coherence diagrams} commute:
        \[
        \begin{tikzcd}[column sep=large]
        (X \otimes Y) \otimes Z \arrow[r, "\alpha_{X,Y,Z}"] \arrow[d, "\gamma_{X,Y} \otimes \mathrm{id}_Z"'] & X \otimes (Y \otimes Z) \arrow[r, "\gamma_{X, Y \otimes Z}"] & (Y \otimes Z) \otimes X \\
        (Y \otimes X) \otimes Z \arrow[rr, "\alpha_{Y,X,Z}"'] & & Y \otimes (X \otimes Z) \arrow[u, "\mathrm{id}_Y \otimes \gamma_{X,Z}"']
        \end{tikzcd}
        \]
        and the analogous hexagon with inverse braiding:
        \[
        \begin{tikzcd}[column sep=large]
        X \otimes (Y \otimes Z) \arrow[r, "\alpha^{-1}_{X,Y,Z}"] \arrow[d, "\mathrm{id}_X \otimes \gamma_{Y,Z}"'] & (X \otimes Y) \otimes Z \arrow[r, "\gamma_{X \otimes Y, Z}"] & Z \otimes (X \otimes Y) \\
        X \otimes (Z \otimes Y) \arrow[rr, "\alpha^{-1}_{X,Z,Y}"'] & & (X \otimes Z) \otimes Y \arrow[u, "\gamma_{X,Z} \otimes \mathrm{id}_Y"']
        \end{tikzcd}
        \]
    \item the \textbf{symmetry coherence diagram} commutes:
        \[
        \begin{tikzcd}
        X \otimes Y \arrow[r, "\gamma_{X,Y}"] \arrow[dr, swap, "\mathrm{id}_{X \otimes Y}"] & Y \otimes X \arrow[d, "\gamma_{Y,X}"] \\
        & X \otimes Y
        \end{tikzcd}
        \]
\end{itemize}
A \hldef{closed symmetric monoidal category} usually refers to a symmetric monoidal category that is \hyperrefIfExists{definition:closed_monoidal_category}{closed as a monoidal category}. 
\end{definition}