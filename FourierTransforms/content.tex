
The purpose of this document is to compile various notions of Fourier transforms and to compare and contrast them.

\section{Fourier transform of a complex valued function on \texorpdfstring{$\bbR$}{R}} \label{section:fourier_transforms_for_complex_valued_functions_on_R}

We can first define the Fourier transform of functions $f \in L^1(\mathbb{R})$

\begin{definition} \label{definition:integrable_complex_valued_function_on_the_real_line}
Let \( f: \mathbb{R} \to \mathbb{C} \) be a complex-valued function defined on the real line.
The function \( f \) is said to be \hldef{integrable} if it belongs to the space \( L^{1}(\mathbb{R}) \), i.e., if
\[
\int_{-\infty}^{\infty} |f(t)| \, dt < \infty.
\]
\end{definition}

There are multiple conventions for defining the Fourier transform and developing its theory. We compile some common conventions below: 

\begin{definition} \label{definition:fourier_transform_of_an_L_1_function_on_R}
    Let $f,g: \bbR \to \bbC$ be a measurable functions. The \hldef{(forward) Fourier transform of $f$} and the \hldef{inverse Fourier transform of $f$} are the functions \hl{$\calF[f](t) = \hat{f}(t): \bbR \to \bbC$} and \hl{$\calF^{-1}[g](t): \bbR \to \bbC$} defined by one of the following conventions (notation for the variables vary amongst the conventions as the variables represent different actors):

    \begin{center}
    \begin{tabular}{|p{4cm}|c|c|} \hline
        Convention name & Fourier Transform $\hat{f}(t)$ & Inverse Fourier Transform $\calF^{-1}[g](t)$  \\ \hline Standard mathematical convention & $\hat{f}(t) = \int_{-\infty}^\infty f(x) e^{-ixt} dx$ & $\calF^{-1}[g](x) = \frac{1}{2\pi} \int_{-\infty}^{\infty} g(t) e^{ixt} dt$ \\ 
        \hline Symmetric math convention & $\hat{f}(t) = \frac{1}{\sqrt{2\pi}} \int_{-\infty}^{\infty} f(x) e^{-ixt}  dx$ & $\calF^{-1}[g](x) = \frac{1}{\sqrt{2\pi}} \int_{-\infty}^\infty g(t) e^{ixt} dt$ \\
        \hline Time frequency convention & $\hat{f}(\omega) = \frac{1}{2\pi} \int_{-\infty}^{\infty} f(t) e^{i \omega t} dt$ & $\calF^{-1}[g](t) = \int_{-\infty}^\infty g(\omega ) e^{-i \omega t} d \omega $ \\ 
        \hline Position wavenumber convention & $\hat{f}(k) = \frac{1}{2\pi} \int_{-\infty}^{\infty} f(x) e^{-i k x }  dx$ & $\calF^{-1}[g](x) = \int_{-\infty}^\infty g(k) e^{i k x} dk$ \\
        \hline Communications and signal processing convention & $\hat{f}(\nu) = \int_{-\infty}^{\infty} f(t) e^{-2\pi i \nu t}  dt$ & $\calF^{-1}[g](t) = \int_{-\infty}^\infty g(\nu) e^{2 \pi i \nu t} d\nu$ \\\hline 
    \end{tabular}
    \end{center}

    If $f \in L^1(\bbR)$, then these are all well-defined.
\end{definition}

\begin{convention}
    We use the symmetric mathematical convention in the rest of the current section, which is the convention used in \cite[Chapter 9]{rudin_real_and_complex}; note that in loc.~cit.~, the notation 
    $$\int_{-\infty}^{\infty} f(x) dm(x)$$
    is used to denote
    $$\frac{1}{\sqrt{2\pi}} \int_{-\infty}^{\infty} f(x) dx$$
    by letting $m$ denote, unlike in other previous chapters of the book, the Lebesgue measure on $\bbR$ divided by $\sqrt{2\pi}$.
\end{convention}

\begin{remark}
    The Fourier transform and the inverse Fourier transform for the standard mathematical convention and the time frequency convention are reversed.
\end{remark}

\begin{theorem}[The Fourier inversion theorem, see e.g. {\cite[Theorem 9.11]{rudin_real_and_complex}}]
    Let $f \in L^1(\bbR)$. If $\hat{f} \in L^1(\bbR)$, then $\calF^{-1}[\hat{f}]$ is a continuous function that vanishes at infinity (Definition \ref{definition:vanish_at_infinity_for_a_complex_valued_function_on_a_locally_compact_hausdorff_space}) and $f = g$ almost everywhere.
\end{theorem}

\begin{theorem}[The Fourier transform uniqueness theorem, see e.g. {\cite[Theorem 9.12]{rudin_real_and_complex}}]
    If $f \in L^{1}(\bbR)$ and $\hat{f}(t)=0$ for all $t \in \bbR$, then $f(x)=0$ a.e.
\end{theorem}

We now define the Fourier transform of a function $f \in L^2(\bbR)$ by taking a limit of Fourier transforms of $L^1(\bbR)$ functions that approximate $f$.  

\begin{definition} \label{definition:fourier_transform_of_an_L_2_function}
    Let \( f \in L^2(\mathbb{R}) \). The \hldef{Fourier transform} \hl{$\hat{f}$} of \( f \) is defined as the limit in the \( L^2 \)-norm of the Fourier transforms of a sequence \( \{f_n\} \subset L^1(\mathbb{R}) \cap L^2(\mathbb{R}) \) such that

    \[
    \lim_{n \to \infty} \|f - f_n\|_{L^2(\mathbb{R})} = 0;
    \]
    such a sequence exists due to Corollary \ref{corollary:intersection_of_L_p_s_is_dense_in_L_p}.  Specifically,

    $$ \hat{f} := \lim_{n \to \infty} \hat{f}_n \text{ in } L^2(\bbR).  $$

    where for each \( n \), the Fourier transform \( \hat{f}_n \) is given by Definition \ref{definition:fourier_transform_of_an_L_1_function_on_R}.
\end{definition}

\begin{convention}
    Unless otherwise stated, given by function $f \in L^2(\bbR)$, $\hat{f}$ refers to the Fourier transform as defined in \Cref{definition:fourier_transform_of_an_L_2_function} rather than the Fourier transform as defiend in \Cref{definition:fourier_transform_of_an_L_1_function_on_R}; recall that the latter notion is applicable to general measurable functions (although the integral defining the Fourier transform may not be well defined). \Cref{theorem:fourier_transform_of_an_L2_function_can_be_obtained_through_the_regular_integral_definition_if_L1} shows that the two definitions in fact coincide almost everywhere for $f \in L^2(\bbR)$ under nice enough conditions.
\end{convention}

\begin{theorem}[see e.g. {\cite[Theorem 9.14]{rudin_real_and_complex}}] \label{theorem:fourier_transform_of_an_L2_function_can_be_obtained_through_the_regular_integral_definition_if_L1}
Let \( f \in L^{2}(\mathbb{R}) \). Let $\hat{f}$ be the Fourier transform $\hat{f}$ in the sense of Definition \ref{definition:fourier_transform_of_an_L_2_function}. If $\hat{f} \in L^1$, then 
$$\hat{f}(t) = \frac{1}{\sqrt{2\pi}} \int_{-\infty}^{\infty} f(x) e^{-ixt}  dx \quad \text{almost everywhere}$$
i.e. $\hat{f}$ almost everywhere coincides with the Fourier transform in the sense of Definition \ref{definition:fourier_transform_of_an_L_1_function_on_R}.
\end{theorem}

Plancherel's theorem shows that the Fourier transform is a unitary operator on $L^2(\bbR)$. 

\begin{theorem}[Plancherel's theorem, see e.g. {\cite[Theorem 9.13]{rudin_real_and_complex}}]
Let \( f \in L^{2}(\mathbb{R}) \). 

\begin{enumerate}
    \item \[ \| f \|_{L^{2}(\mathbb{R})} = \| \hat{f} \|_{L^{2}(\mathbb{R})}.  \]

    \item The Fourier  transform $f \mapsto \hat{f}$ is a Hilbert space (see Convention \ref{convention:L2_is_a_hilbert_space}) isomorphism $L^2(\mathbb{R}) \to L^2(\mathbb{R})$.
\end{enumerate}
\end{theorem}

\begin{remark}
    As we are working with vector fields over $\bbR$ and $\bbC$ (in particular which are not of characteristic $2$), saying that \[ \| f \|_{L^{2}(\mathbb{R})} = \| \hat{f} \|_{L^{2}(\mathbb{R})}\] is equivalent to saying that the inner product is preserved under Fourier transforms, i.e. $$(f,g) = (\hat{f}, \hat{g})$$ for all $f,g \in L^2(\bbR)$. In other words, the Fourier transform is a unitary operator on $L^2(\bbR)$.
\end{remark}

\section{Fourier transform of a function on a locally compact abelian group}

We now extend the analytic notion of Fourier transform in Section \ref{section:fourier_transforms_for_complex_valued_functions_on_R} to apply to functions on locally compact abelian groups. In particular, the discussion in Section \ref{section:fourier_transforms_for_complex_valued_functions_on_R} is essentially a specialization of the discussion in this section in the case that the locally compact abelian group $G$ is $\bbR$.

\begin{definition} \label{definition:locally_compact_group}
    A \hldef{locally compact group} is an topological group that is locally compact Hausdorff. It is called a \hldef{locally compact abelian group} if it is abelian as well.
\end{definition}


\begin{definition} \label{definition:haar_measure_on_a_locally_compact_group}
    \TODO{Define Borel sigma algebra, measure, regular Borel measure}
    Let \(G\) be a \hyperrefIfExists{definition:locally_compact_group}{locally compact group}\CrefIfExists{definition:locally_compact_group} with Borel \(\sigma\)-algebra \(\mathcal{B}(G)\). A measure \(\mu : \mathcal{B}(G) \to [0, \infty]\) is called a \hldef{left Haar measure} if it satisfies:
    \begin{enumerate}
    \item \(\mu\) is a regular Borel measure, finite on compact sets,
    \item \(\mu\) is left-translation invariant:
    \[
        \mu(gS) = \mu(S) \quad \text{for all } g \in G, \, S \in \mathcal{B}(G),
    \]
    where \(gS = \{g \cdot s : s \in S \}\).
    \end{enumerate}
    The notion of a \hldef{right Haar measure} may be defined similarly. If \(\mu\) also both a left and right Haar measure, then it is called \hldef{two-sidedinvariant} and may simply be referred to as a \hldef{Haar measure}.
\end{definition}



\begin{lemma} \label{lemma:left_or_right_haar_measure_on_a_locally_compact_abelian_group_is_a_haar_measure}
    If $G$ is a \hyperrefIfExists{definition:locally_compact_group}{locally compact abelian group}, then any left/right Haar measure is a Haar measure.
\end{lemma}

\begin{lemma} \label{lemma:left_haar_measures_on_a_locally_compact_group_are_unique_up_to_scalars}
    Let $G$ be a \hyperrefIfExists{definition:locally_compact_group}{locally compact group}. 
    \begin{enumerate}
        \item There exists a left Haar measure on $G$. There exists a right Haar measure on $G$.
        \item Any two left Haar measures on $G$ are positive scalar multiples of each other. Any two right Haar measures on $G$ are positive scalar multiples of each other
    \end{enumerate}
\end{lemma}


\begin{notation}
    Given a \hyperrefIfExists{definition:locally_compact_group}{locally compact group} $G$, fix a left Haar measure on $G$; integration with respect to this Haar measure would often be denoted by \hl{$\int_G dg$}
\end{notation}

\begin{definition}
    Let \( G \) be a \hyperrefIfExists{definition:locally_compact_group}{locally compact group} with a fixed left Haar measure \( \mu \). For functions $f_1, f_2: G \to \bbC$, their \hldef{convolution} \( f_1 * f_2 \in L^1(G) \) is defined by

    $$\hlin{(f_1 * f_2)(x) = \int_G f_1(y) f_2(y^{-1} x) \, d\mu(y)}$$

    If $f_1,f_2 \in L^1(G)$, then this integral converges absolutely for almost every $x \in G$.
\end{definition}

\TODO{TODO: define a Banach algebra}
\begin{lemma}
    Let $G$ be a \hyperrefIfExists{definition:locally_compact_group}{locally compact group}. 
    \begin{enumerate}
        \item For $f_1,f_2 \in L^1(G)$, we have \[ \|f_1 * f_2\|_1 \leq \|f_1\|_1 \|f_2\|_1.  \]
        \item The convolution $*$ makes $L^1(G)$ into a Banach algebra (without unit).
        \item If $G$ is abelian, then $f_1 * f_2 = f_2 * f_1$ almost everywhere for $f_1,f_2 \in L^1(G)$.
        \item If $G$ is abelian, then $*$ makes $L^1(G)$ into a commutative Banach algebra (without unit).
    \end{enumerate}
\end{lemma}


\begin{definition} \label{definition:quasi_character_of_a_locally_compact_hausdorff_group}
    Let \( G \) be a locally compact Hausdorff group.
    \begin{enumerate}
        \item A \hldef{quasicharacter of $G$} is a continuous group homomorphism $\chi : G \to \mathbb{C}^\times$.
        \item A quasicharacter \(\chi\) is \hldef{unitary} if its image lies in the unit circle $S^1 \subset \bbC^\times$, i.e. $|\chi(g)| = 1$ for all $g \in G$. Such a quasicharacter is also simply called a \hldef{character}.
        \item A (quasi)character is \hldef{finite} if its image is finite, i.e. its kernel has finite index in its domain.
    \end{enumerate}

    When $G$ is finite, $G$ is usually equipped with the discrete topology --- a character (and even a quasicharacter) is thus simply a group homomorphism $G \to \bbC^\times$ such that $|\chi(g)| = 1$ for all $g \in G$.

    \TextIfExists{definition:additive_and_multiplicative_characters_of_a_ring}{
    Compare against \Cref{definition:additive_and_multiplicative_characters_of_a_ring}}
\end{definition}

\begin{definition} \label{definition:pontryagin_dual_of_a_locally_compact_abelian_group}
    Let \( G \) be a locally compact abelian group. The \hldef{Pontryagin dual of $G$} is the group
    $$\hlin{\widehat{G} = \{\, \chi : G \to S^1 \mid \chi \text{ is a continuous group homomorphism} \,\},}$$
    equipped with pointwise multiplication and the compact-open topology.
\end{definition}

\begin{remark}
    A posteriori, we will equip $\widehat{G}$ with a Haar measure, see \Cref{convention:haar_measure_on_pontryagin_dual_of_a_locally_compact_abelian_group}
\end{remark}

\begin{lemma}
    Let \( G \) be a locally compact abelian group.
    \begin{enumerate}
        \item $\widehat{G}$ is a locally compact abelian group. 
        \item There is a canonical isomorphism 
        \begin{align*}
        G &\xrightarrow{\sim} \widehat{\widehat{G}} \\
        g &\mapsto (\chi \mapsto \chi(g))
        \end{align*}
        of topological groups. This is usually called the \hldef{Pontryagin map}.
    \end{enumerate}
\end{lemma}

\begin{notation}\label{notation:bilinear_pairing_notation_for_evaluation_of_a_character}
    Let \( G \) be a \hyperrefIfExists{definition:locally_compact_group}{locally compact abelian group}. Given $g \in G$ and \hyperrefIfExists{definition:pontryagin_dual_of_a_locally_compact_abelian_group}{$\chi \in \widehat{G}$}, write \hl{$\langle g, \chi \rangle$} for $\chi(g) \in \bbC$. 
\end{notation}


\begin{definition} \label{definition:fourier_transform_of_a_function_on_a_locally_compact_abelian_group}
    Let \(G\) be a locally compact abelian group with a fixed Haar measure \(\mu\).

    For a function $f: G \to \bbC$, the \hldef{Fourier transform} \hl{$\calF[f] = \hat{f} : \widehat{G} \to \mathbb{C}$} is defined by

    \[ \hat{f}(\chi) = \int_{x \in G} f(x) \overline{\langle x, \chi \rangle} \, d\mu(x), \]

    % \CrefIfExists{notation:bilinear_pairing_notation_for_evaluation_of_a_character}
    % \ref{notation:bilinear_pairing_notation_for_evaluation_of_a_character}
    \CrefIfExists{notation:bilinear_pairing_notation_for_evaluation_of_a_character} for all \(\chi \in \widehat{G}\). If $f \in L^1(G)$, then the integral converges absolute and hence $\hat{f}(\chi)$ is well defined for all $\chi \in \hat{G}$. Moreover, If $f \in L^1(G)$, then $\hat{f} \in C_0(\widehat{G})$, i.e. is a function on $\widehat{G}$ that vanishes at infinity\CrefIfExists{definition:vanish_at_infinity_for_a_complex_valued_function_on_a_locally_compact_hausdorff_space}.

    For a function $g: \widehat{G} \to \bbC$, the \hldef{inverse Fourier transform} \hl{$\calF^{-1}[g]: G \to \bbC$} is defined by 

    \[ \calF^{-1}[g](x) = \int_{\chi \in \widehat{G}} g(\chi) {\langle x, \chi \rangle} \, d\nu(\chi).\]

    for $x \in G$, where $\nu$ is a Haar measure on $\widehat{G}$. Similarly as before, if $g \in L^1(\widehat{G})$, then the integral converges absolute and hence $\calF^{-1}[g](x)$ is well defined for all $x \in G$. Moreover, If $g \in L^1(\widehat{G})$, then $\calF^{-1}[g] \in C_0(G)$, i.e. is a function on $G$ that vanishes at infinity (Definition \ref{definition:vanish_at_infinity_for_a_complex_valued_function_on_a_locally_compact_hausdorff_space}).
\end{definition}

\begin{theorem}[Fourier Inversion Formula] \label{theorem:fourier_inversion_formula_for_locally_compact_abelian_group}
    Let \(G\) be a locally compact abelian group with Haar measure \(\mu\). There exists a unique Haar measure $\nu$ in $\widehat{G}$ for which the following holds: for any \(f \in L^1(G)\), if \(\hat{f} \in L^1(\widehat{G})\), then 
    
    $$f(x) = \calF^{-1}[g](x)$$

    and the integral defining $\calF^{-1}[g](x)$ converges absolutely for almost every \(x \in G\).
\end{theorem}

\begin{convention} \label{convention:haar_measure_on_pontryagin_dual_of_a_locally_compact_abelian_group}
    Given a locally compact abelian group $G$ and a choice of Haar measure $\mu$, we always equip $\widehat{G}$ with the unique Haar measure $\nu$ in Theorem \ref{theorem:fourier_inversion_formula_for_locally_compact_abelian_group}.
\end{convention}

\TODO{TODO: state the Plancherel theorem}

\section{Mellin transform of a complex valued function on the positive real numbers}

\begin{definition}[Mellin transform] \label{definition:mellin_transform_of_a_measurable_function_on_the_positive_real_numbers}
Let $f: \mathbb{R}_{>0} \to \mathbb{C}$ be a measurable function. The \hldef{Mellin transform of $f$} is the function \hl{$\mathcal{M}[f] : \mathbb{C} \to \mathbb{C}$} defined by
\[
\mathcal{M}[f](s) := \int_0^\infty f(x) \, x^{s-1} \, dx,
\]
whenever the integral exists (absolutely or as an improper integral).

Although this definition is stated for general measurable functions $f: \mathbb{R}_{>0} \to \mathbb{C}$, it is intuitively meaningful when in the case that $f$ is a function into $\bbC^\times$ --- the function $f$ would be a function from $\bbR_{>0}$, the identity component of the multiplicative group $\bbR^\times$, to the multiplicative group $\bbC^\times$.  It is further intuitively meaningful to express the integral as
$$\int_0^\infty f(x) \, x^{s} \, \frac{dx}{x},$$
and regarding $\frac{dx}{x}$ as a \CrefAndHyperrefIfExist{definition:haar_measure_on_a_locally_compact_group}{Haar measure} \TODO{on what?}

\end{definition}

\TODO{TODO: sufficient conditions for the Mellin transform to converge}
\TODO{TODO: the inverse Mellin transform to converge}

\begin{theorem}[Relation between Mellin and Fourier transforms]
Let $f: \mathbb{R}_{>0} \to \mathbb{C}$ be a measurable function such that the Mellin transform $\mathcal{M}[f]$ converges for $s = \sigma + i t \in \mathbb{C}$ in some vertical strip $\sigma_1 < \sigma < \sigma_2$.

Let $g : \mathbb{R} \to \mathbb{C}$ be the function
$$ g(u) := f(e^{u}).  $$

Then the Fourier transform $\mathcal{F}[g](\omega)$ of $g$, defined by
\TODO{TODO: according to my convention, the Fourier transform is slightly different.}
$$
\mathcal{F}[g](\omega) := \int_{-\infty}^\infty g(u) \, e^{-i \omega u} \, du,
$$
exists (appropriately as an improper integral or in distributional sense) for $\omega \in \mathbb{R}$ in a corresponding domain, and satisfies the identity
$$
\mathcal{F}[g](\omega) = \mathcal{M}[f]\bigl(\sigma - i \omega\bigr),
$$
where the Mellin transform is analytically continued if necessary to the point $\sigma - i \omega$.

In other words, the Fourier transform of the function $u \mapsto f(e^{u})$ is equal to the Mellin transform of $f$ evaluated on the vertical line in the complex plane parameterized by $s = \sigma - i \omega$.

\end{theorem}

\section{Arithmetic Fourier transform}

\subsection{Deligne's Fourier transform of perverse sheaves on the additive group over a finite field}

\begin{definition} \label{definition:fourier_transform_functor_on_the_cohomologically_bounded_derived_category_of_ell_adic_sheaves_on_the_additive_group_over_a_finite_field_associated_to_a_character_on_the_additive_group_of_the_finite_field}
    Let $\Fq$ be a finite field. Let $\psi: \bbG_a(\Fq) \to \Qellbar^*$ be a nontrivial \hyperrefIfExists{definition:quasi_character_of_a_locally_compact_hausdorff_group}{character}. Let $m: \bbG_a \times \bbG_a \to \bbG_a$ be the scheme morphism given by $(x,y) \mapsto xy$.

    The \hldef{Fourier transform functor associated to $\psi$} (also called \hldef{Deligne's (arithmetic) Fourier transform associated to $\psi$}) is the functor 
    $$\hlin{T_\psi: \Dbc(\bbG_a, \Qellbar) \to \Dbc(\bbG_a, \Qellbar)}$$
    defined by 
    $$T_\psi(M) = R\pi_{1!} \left( \pi_{2}^* (M) \otimes m^* (\mathscr{L}(\psi))\right)[1]$$
    \CrefIfExists{defintion:artin_schreier_sheaf_of_a_character_on_the_additive_group_of_a_finite_field}
    where $\pi_1$ and $\pi_2$ are the projection morphisms $\bbA^1 \times \bbA^1 \to \bbA^1$ and \hyperrefIfExists{defintion:artin_schreier_sheaf_of_a_character_on_the_additive_group_of_a_finite_field}{$\mathscr{L}(\psi)$} is the \hyperrefIfExists{defintion:artin_schreier_sheaf_of_a_character_on_the_additive_group_of_a_finite_field}{Artin-Schreier sheaf of $\psi$}\CrefIfExists{definition_artin_schreier_sheaf_of_a_character_on_the_additive_group_of_a_finite_field}. %is the \hyperrefIfExists{}{}. 
\end{definition}

\TODO{TODO: Put in generalized definitions of Deligne's fourier transform functor}

\subsection{Fourier transform of the trace function of a cohomologically bounded complex of constructible sheaves on a comutative algebraic group over a finite field}

We give an abstract definition for the Fourier transform of a function on a finite group of rational points on an algebraic group. The case of interest\CrefIfExists{definition:arithmetic_fourier_transform_of_a_cohomologically_bounded_complex_of_constructible_ell_adic_sheaves_on_a_finite_type_separated_scheme_over_a_finite_field} is when base field is finite and the function is the \hyperrefIfExists{definition:frobenius_trace_of_a_cohomologically_bounded_complex_of_constructible_ell_adic_sheaves_on_a_finite_type_separated_scheme_over_a_finite_field}{Frobenius trace function}\CrefIfExists{definition:frobenius_trace_of_a_cohomologically_bounded_complex_of_constructible_ell_adic_sheaves_on_a_finite_type_separated_scheme_over_a_finite_field}.   

\begin{definition} \label{definition:fourier_transform_of_a_function_on_a_finite_group_of_rational_points_of_an_algebraic_group_over_a_field}
    % Let $G/\bbF_q$ be a 
    Let $G/k$ be an algebraic group over a field. Let $F$ be a field. Let $f: G(k) \to F$ be a function. Write $\widehat{G}(k)$ for the set of characters/group homomorphisms $\chi: G(k) \to F^\times$.
    Assuming that $G(k)$ is a finite set, define the \hldef{(discrete) Fourier transform of $f$} to be the function \hl{$\calF[f] = \hat{f} :\widehat{G}(k) \to F^\times$} defined by 
    $$\hat{f}(\chi) = \sum_{x \in G(k)} \chi(x) f(x).$$
\end{definition}

In studying arithmetic Fourier transforms for functions on algebraic groups over finite fields, the main functions of interest are Frobenius trace functions:

\begin{definition} \label{definition:arithmetic_and_geometric_frobenius_automorphisms_of_algebraic_extensions_of_finite_fields}
    Let $k$ be either a finite field $\Fq$ of $q$ elements or the algebraic closure $\Fqbar$ thereof and let $p = \Char k$. 
    \begin{enumerate}

        \item For $n \geq 0$, the \hldef{$p^n$th power Frobenius automorphism on $k$} is the \hyperrefIfExists{definition:absolute_frobenius_endomorphism_of_a_ring}{absolute $p^n$th Frobenius endomorphism of $k$}\CrefIfExists{definition:absolute_frobenius_endomorphism_of_a_ring}; it is a field automorphism.

        \item When $k$ is a finite field and $l$ is an algebraic extension of $k$, the \hldef{arithmetic Frobenius automorphism on $l$ (over the base field $k$)} usually refers to the $|k|$th power Frobenius automorphism on $l$; it is an element of $\Gal(l/k)$ that we usually denote by \hl{$\Frob_{l/k}$}. We often simply let the finite field $k$ be the base field and also denote $\Frob_{l/k}$ by \hl{$\Frob_{l}$}. 
        \item the \hldef{geometric Frobenius automorphism on $l$} refers to the element $\Frob_{l/k}^{-1} \in \Gal(l/k)$.
    \end{enumerate}
    A corresponding automorphism $\Spec l \to \Spec l$ may also be referred to as an \hldef{arithmetic/geometric Frobenius automorphism} as appropriate.
\end{definition}



\begin{definition}[e.g. {\cite[Section A.4]{forey_fresan_kowalski_aftff}}] \label{definition:frobenius_trace_of_a_cohomologically_bounded_complex_of_constructible_ell_adic_sheaves_on_a_finite_type_separated_scheme_over_a_finite_field}
    \TODO{TODO: notate the bounded derived category of constructible sheaves}
    \TODO{TODO: define stalks}
    Let $X$ be a finite type and separated scheme over a finite field $\bbF_q$, let $\ell \neq \Char \Fq$ be a prime number, and let $M \in D_c^b(X,\Qellbar)$. For points $x \in X(\bbF_q)$, write $\barx$ for a geometric point above $x$. 
    
    % The \hyperrefIfExists{definition:arithmetic_and_geometric_frobenius_automorphisms_of_algebraic_extensions_of_finite_fields}{geometric Frobenius automorphism}\CrefIfExists{definition:arithmetic_and_geometric_frobenius_automorphisms_of_algebraic_extensions_of_finite_fields} $\operatorname{Fr}_{q}: \Fqbar \to \Fqbar $ acts on the stalk $M_{\barx}$ and this action is independent of the choice of $\barx$ up to conjugation. 
    The \hldef{Frobenius trace function of $M$ over $\Fqn$} is the function $G(\Fqn) \to \Qellbar$ defined by
    $$\hlin{t_M(x;\Fqn) = \sum_{i \in \bbZ} (-1)^i \operatorname{Tr}(\operatorname{Frob}_{q^n}|H^i(M)_{\barx} )}$$
    where $\Frob_{q^n}$ is the \hyperrefIfExists{definition:geometric_frobenius_action_of_a_derived_object_on_a_scheme_of_characteristic_p_at_a_stalk}{geometric Frobenius action of the sheaf $H^i(M)$ at the stalk at $\barx$}\CrefIfExists{definition:geometric_frobenius_action_of_a_derived_object_on_a_scheme_of_characteristic_p_at_a_stalk}.
    This is independent of the choice of geometric point $\barx$ above $x$. 
\end{definition}

\begin{notation} \label{notation:characters_of_the_points_of_an_algebraic_group_over_a_finite_field}
    Given an algebraic group $G$ over a finite field $\bbF_q$, and when a prime $\ell \neq \Char \bbF_q$ is understood in context, let \hl{$\widehat{G}(\Fqn)$} denote the set of \hyperrefIfExists{definition:quasi_character_of_a_locally_compact_hausdorff_group}{(unitary) characters} $\chi: G \to \Qellbar^\times$. Let \hl{$\widehat{G}$} denote the union
    $\bigcup_{n\geq 1} \widehat{G}(\Fqn)$.
\end{notation}

\begin{definition} \label{definition:arithmetic_fourier_transform_of_a_cohomologically_bounded_complex_of_constructible_ell_adic_sheaves_on_a_finite_type_separated_scheme_over_a_finite_field}
    Let $G$ be an algebraic group over a finite field $\bbF_q$, let $\ell \neq \Char \Fq$ be a prime number, and let $M \in D_c^b(G,\Qellbar)$. 

    The \hldef{arithmetic Fourier transform function of $M$} is the function \hl{$S(M, -): \widehat{G} \to \Qellbar$}\CrefIfExists{notation:characters_of_the_points_of_an_algebraic_group_over_a_finite_field} defined by \CrefAndHyperrefIfExist{definition:fourier_transform_of_a_function_on_a_finite_group_of_rational_points_of_an_algebraic_group_over_a_field}{Fourier transforms} of \CrefAndHyperrefIfExist{definition:frobenius_trace_of_a_cohomologically_bounded_complex_of_constructible_ell_adic_sheaves_on_a_finite_type_separated_scheme_over_a_finite_field}{$t_M(-;\bbF_q)$}, i.e. if $\chi \in \widehat{G}(\Fqn)$, then 
    $$S(M,\chi) = \sum_{x \in G(\Fqn)} \chi(x) t_M(x;\Fqn).$$
    % In the case that $\chi$ is the trivial character, we might also denote $S(M,\chi)$ by \hl{$S(M)$}.
\end{definition}

\subsection{Lang's isogeny theorem and Kummer sheaves associated to characters on connected commutative algebraic groups over finite fields}

% Definition: Lang isogeny
\begin{definition} \label{definition:lang_map_of_an_endomorphism_of_an_algebraic_group}
Let $G$ be an \hyperrefIfExists{definition:algebraic_group_over_a_field}{algebraic group over some scheme $S$}, and let $F: G \to G$ be an \hyperrefIfExists{definition:homomorphism_of_algebraic_groups_over_a_scheme}{endomorphism}. The \hldef{Lang map associated to $F$} is the morphism \hl{$L: G \to G$} defined by
$$ L(g) = g^{-1} F(g) $$
for every $g \in G$. 
% See Also
% theorem:lang_steinberg_theorem

In case that $S$ is the spectrum of a finite field and $F$ is the \hyperrefIfExists{definition:relative_frobenius_morphism_of_a_scheme_over_a_scheme_of_prime_characteristic}{$q$th power relative Frobenius morphism on $G/\Fq$}\CrefIfExists{definition:relative_frobenius_morphism_of_a_scheme_over_a_scheme_of_prime_characteristic}, the Lang map associated to $F$ is also called the \hldef{Lang isogeny of $G$}.
\end{definition}

\begin{theorem}[See e.g. {\cite[Chapter V Corollary 16.4]{borel_lag}}] \label{theorem:langs_theorem_for_connected_algebraic_groups_over_finite_fields}
Let $\Fq$ be a finite field of characteristic $p$ with $q$ elements, let $G$ be a connected algebraic group over $\Fq$, and let $F$ be the \hyperrefIfExists{definition:relative_frobenius_morphism_of_a_scheme_over_a_scheme_of_prime_characteristic}{relative $q$th power Frobenius endomorphism on $G/\Fq$}\CrefIfExists{definition:relative_frobenius_morphism_of_a_scheme_over_a_scheme_of_prime_characteristic}. The \hyperrefIfExists{definition:lang_map_of_an_endomorphism_of_an_algebraic_group}{Lang map associated to $F$}\CrefIfExists{definition:lang_map_of_an_endomorphism_of_an_algebraic_group} is a separable Galois \'etale (surjective) isogeny whose kernel is precisely $G(\Fq)$, the group of $\Fq$-points of $G$.
\TODO{TODO: Borel's book states the Lang isogeny as separable, and surjective, but I don't know what shows the Galoisness and \'etaleness}
% See Also
% theorem:lang_steinberg_theorem
\end{theorem}

\begin{definition} \label{definition:kummer_sheaf_associated_to_a_character_on_points_of_a_connected_commutative_algebraic_group_over_a_finite_field}
    Let $G/\Fq$ be a conected commutative algebraic group. Let $\ell \neq \Char \Fq$ be a prime number. Let $n \geq 1$.
    By \hyperrefIfExists{theorem:langs_theorem_for_connected_algebraic_groups_over_finite_fields}{Lang's theorem}\CrefIfExists{theorem:langs_theorem_for_connected_algebraic_groups_over_finite_fields}, given a connected commutative algebraic group $G/\Fq$, we have we have a short exact sequence
    $$1 \to G(\Fqn) \to G_{\Fqn} = G \times_{\Fq} \Fqn \xrightarrow{L} G_{\Fqn} \to 1$$
    where $L$ is the \hyperrefIfExists{definition:lang_map_of_an_endomorphism_of_an_algebraic_group}{Lang isogeny}\CrefIfExists{definition:lang_map_of_an_endomorphism_of_an_algebraic_group}. Since this identifies $G_{\Fqn}$ with a finite \'etale $G(\Fqn)$-cover of $G_{\Fqn}$, we have a corresponding surjective group homomorphism $\pioneet(G_{\Fqn}, e) \to G(\Fqn)$\CrefIfExists{theorem:finite_etale_covers_of_a_connected_scheme_correspond_to_finite_sets_with_an_action_of_the_etale_fundamental_group}. 
    
    Let $\chi: G(\Fqn) \to \Qellbar^\times$ be a character. There is a lisse sheaf \hl{$\calL_\chi$} of rank $1$ on $G_{\Fqn} = G \times_{\Fq} \Fqn$ corresponding to the composed group homomorphism
    $$\pioneet(G_{\Fqn}, e) \to G(\Fqn) \xrightarrow{\chi} \Qellbar^\times.$$
    We might often call this lisse sheaf the \hldef{Kummer sheaf associated to $\chi$}.

\end{definition}

\begin{definition}
Let $G$ be a connected commutative algebraic group over a finite field $\Fq$, and let $\ell \neq \Char \Fq$ be a prime. Let $\chi \in \widehat{G}(\Fqn)$ be a character for some $n \geq 1$.  
\begin{itemize}
    \item Let $M \in \Dbc(G, \Qellbar)$. The \hldef{twist} \hl{$M_\chi$} of $M$ by $\chi$ is the object 
    $$M_\chi = M_{\Fqn} \otimes \calL_{\chi}$$
    where $M_{\Fqn}$ is the pullback of $M$ under the base change map $G_{\Fqn} \to G$ and \hyperrefIfExists{definition:kummer_sheaf_associated_to_a_character_on_points_of_a_connected_commutative_algebraic_group_over_a_finite_field}{$\calL_{\chi}$} is the \hyperrefIfExists{definition:kummer_sheaf_associated_to_a_character_on_points_of_a_connected_commutative_algebraic_group_over_a_finite_field}{Kummer sheaf associated to $\chi$}\CrefIfExists{definition:kummer_sheaf_associated_to_a_character_on_points_of_a_connected_commutative_algebraic_group_over_a_finite_field}.

    \item More generally, let $\pi: X \to G_{\Fqn}$ be some morphism of schemes over $\Fqn$ and let $M \in \Dbc(X, \Qellbar)$. The \hl{twist} \hldef{$M_\chi$} of $M$ by $\chi$ is the object
    $$M_\chi = M \otimes \calL_\chi.$$
\end{itemize} 
\end{definition}

\begin{lemma}[{\cite[Lemma 1.17]{forey_fresan_kowalski_aftff}}] \label{lemma:twisting_cohomologically_bounded_derived_object_on_a_scheme_over_a_finite_field_by_a_character_on_a_connected_commutative_group_scheme_is_t_exact_in_the_standard_and_perverse_t_structures}
    \TODO{re-verify this citation for the final version of FFK}
    Let $f: X \rightarrow G$ be a morphism from an algebraic variety $X$ to a connected commutative algebraic group $G$, both defined over $\Fq$. Let $\chi \in \widehat{G}$ be a character. Then the functor $M \mapsto M_\chi$ on $D_c^{b}(X, \Qellbar)$ or $\mathrm{D}_c^{\mathrm{b}}\left(X_{\Fqbar}, \Qellbar\right)$ is $t$-exact for the standard and perverse $t$-structures. In particular, if $M$ is perverse (resp. semiperverse) then so is $M_\chi$.
\end{lemma}

\begin{proposition} \label{proposition:some_properties_of_the_arithmetic_fourier_transform}
Let $G$ be a connected commutative algebraic group over a finite field $\Fq$, let $\ell \neq \Char \Fq$ be a prime, let $M,N \in \Dbc(G, \Qellbar)$, and let $n \geq 1$. Write $1_n \in \widehat{G}(\Fqn)$ for the trivial character.

\TODO{TODO: define $*_!$, negligible objects, generic subset}
\begin{enumerate}
    \item $S(M, \chi) = S(M_\chi, 1_n)$ for any \hyperrefIfExists{notation:characters_of_the_points_of_an_algebraic_group_over_a_finite_field}{$\chi \in \widehat{G}(\Fqn)$}. 
    
    \item Given any homomorphism $f: G \to H$ of connected commutative algebraic groups over $\Fq$ and for any $n \geq 1$, we have 
    $$S(Rf_! M , 1_{H,n}) = S(M,1_{G,n})$$
    where $1_{G,n} \in \widehat{G}(\Fqn)$ and $1_{H,n} \in \widehat{H}(\Fqn)$ denote the trivial characters.

    \item $S(M *_! N, \chi) = S(M, \chi) \cdot S(N,\chi)$ for any \hyperrefIfExists{notation:characters_of_the_points_of_an_algebraic_group_over_a_finite_field}{$\chi \in \widehat{G}(\Fqn)$}. 

    \item If $M$ is negligible, then $S(M,\chi) = 0$ for $\chi$ in a generic subset of $\widehat{G}$. The converse holds if $M$ is perverse.
\end{enumerate}
\end{proposition}

\begin{proof}
    \begin{enumerate}
        \item 
        \begin{align*}
        S(M_\chi, 1_n) &= \sum_{x \in G(\Fqn)} t_{M_\chi}(x; \Fqn) \cdot 1_n(x) \\
        &= \sum_{x \in G(\Fqn)} t_{M \otimes \calL_\chi}(x; \Fqn) \\
        &= \sum_{x \in G(\Fqn)} t_{M}(x; \Fqn) \cdot t_{\calL_\chi}(x; \Fqn)\\
        &= \sum_{x \in G(\Fqn)} t_{M}(x; \Fqn) \cdot \chi(x) \\
        &= S(M, 1_n).
        \end{align*}

        \item 
        \begin{align*}
            S(Rf_! M, 1_n) &= \sum_{y \in H(\Fqn)} t_{Rf_! M}(y; \Fqn) \\
            &= \sum_{y \in H(\Fqn)} \sum_{\substack{x \in G(\Fqn) \\ f(x) = y}} t_M(x;\Fqn) \\
            &= \sum_{x \in G(\Fqn)} t_M(x;\Fqn) \\
            &= S(M, 1_n)
        \end{align*}

        \item
        \begin{align*}
            S(M *_! N, \chi) &= \sum_{x \in G(\Fqn)} t_{M *_! N}(x;\Fqn) \cdot \chi(x)  \\
            &= \sum_{x \in G(\Fqn)} \left( \sum_{y \in G(\Fqn)} t_M(y;k_n)  \cdot t_N(y^{-1} x; k_n) \right)  \cdot \chi(x)  \\
            &= \sum_{x \in G(\Fqn)} \left( \sum_{y \in G(\Fqn)} t_M(y;k_n) \cdot \chi(y) \cdot t_N(y^{-1} x; k_n) \cdot \chi(y^{-1} x) \right)  \\
            &= \sum_{y \in G(\Fqn)} \sum_{x \in G(\Fqn)}   t_M(y;k_n) \cdot \chi(y) \cdot t_N(y^{-1} x; k_n) \cdot \chi(y^{-1} x)   \\
            &= \sum_{y \in G(\Fqn)} \left( t_M(y;k_n) \cdot \chi(y) \sum_{x \in G(\Fqn)} t_N(y^{-1} x; k_n) \cdot \chi(y^{-1} x)  \right) \\
        \end{align*}
        using the change of variables $z = y^{-1} x$, the above equals
        \begin{align*}
            & \sum_{y \in G(\Fqn)} \left( t_M(y;k_n) \cdot \chi(y) \sum_{z \in G(\Fqn)} t_N(z; k_n) \cdot \chi(z) \right)  \\
            =& \left( \sum_{y \in G(\Fqn)}  t_M(y;k_n) \cdot \chi(y) \right) \cdot \left( \sum_{z \in G(\Fqn)} t_N(z; k_n) \cdot \chi(z) \right) \\
            =& S(M, \chi) \cdot S(N,\chi).
        \end{align*}

        \item 
        For generic $\chi \in \widehat{G}$, we have $H_c^i(G_{\Fqbar}, M_\chi) = 0$ for all $i > 0$ by the generic vanishing theorem. The Grothendieck-Lefschetz trace formula shows that 
        $$S(M_\chi, 1_n) = \sum_{x \in G_{\bbF_{q^{n}}}} t_{M_\chi} (x;\bbF_{q^{n}}) = \sum_{i \in \bbZ} (-1)^i \operatorname{tr}(\Frob_{q^n} | H_c^i(G_{\Fqbar}, M_\chi)).$$
        Therefore, 
        $$S(M_\chi, 1_n) = \operatorname{tr}(\Frob_{q^n} | H_c^0(G_{\Fqbar}, M_\chi))$$
        for generic $\chi$. This value is generically $0$ if and only if $H_c^0(G_{\Fqbar}, M_\chi))$ is generically $0$.  

        If $M$ is perverse, then this is true if and only if $M$ is negligible. If $M$ is a more general negligible object of $\Dbc(G, \Qellbar)$, then its perverse cohomology objects ${}^p\calH^i(M)$ are all negligible by definition. Recall that $[M] = \sum_{i \in \bbZ} (-1)^i [{}^p\calH^i(M)]$ in the Grothendieck group $K(\Dbc(G, \Qellbar))$. Moreover, $S(-, \chi)$ is additive in distinguished triangles because $t(-; \Fqn)$ is additive in distinguished triangles. Therefore, 
        $$S(M,\chi) = \sum_{i \in \bbZ} (-1)^i S({}^p\calH^i(M), \chi) = 0.$$
    \end{enumerate}
\end{proof}

\begin{lemma} \label{lemma:characters_on_product_of_connected_commutative_algebraic_groups_over_a_finite_field_correspond_to_tuples_of_characters_of_the_individual_groups}
    Let $\{G_i\}_{i}$ be a finite collction of connected commutative algebraic groups over a finite field $\Fq$. Write $G = \prod_i G_i$. Let $\ell \neq \Char \Fq$ be a prime. Let $n \geq 1$. 
    \begin{enumerate}
        \item The characters $\chi \in \widehat{G}(\Fqn)$ are in bijective correspondence with tuples $(\chi_i)_{i}$ of characters $\chi_i \in \widehat{G}_i(\Fqn)$ under the inverse maps
        \begin{align*}
            \widehat{G}(\Fqn) &\to \prod_i \widehat{G}_i(\Fqn) \\
             \chi & \mapsto (\chi \circ \iota_i)_i \\
             \left( (g_i)_i \mapsto \prod_i \chi_i(g_i) \right) & \mapsfrom (\chi_i)_i
        \end{align*}
        where $\iota_i: G_i(\Fqn) \to G(\Fqn)$ sends $g_i \in G_i(\Fqn)$ to the point of $G(\Fqn)$ whose $i$th coordinate is $g_i$ and whose other coordinates are all identity elements.

        \item Let $\chi \in \widehat{G}(\Fqn)$ be a character and let $(\chi_i)_i \in \prod_i \widehat{G}_i(\Fqn)$ be the corresponding tuple of characters. There is a natural isomorphism
        $$\calL_{\chi} \cong \bigotimes_i \operatorname{pr}_i^* \calL_{\chi_i}$$
        where $\operatorname{pr}_i: G \to G_i$ is the projection morphism.
    \end{enumerate}
\end{lemma}
\begin{proof}
    \begin{enumerate}
        \item This is clear. 
        \item Since inverse images preserve tensor products, it suffices to show this in the case that the collction $(G_i)_{i}$ consists of two groups $G_1$ and $G_2$. It also suffices to show this in the case of $n = 1$ by base change. Take the commutative diagram
        \begin{center}
            \begin{tikzcd}
                G_1 \times G_2 \ar[r, "\varphi"] \ar[d] & G_1 \times G_2 \ar[r] \ar[d, "\psi"] & G_2 \ar[d, "L_2"] \\
                G_1 \times G_2 \ar[r] \ar[d] & G_1 \times G_2 \ar[r] \ar[d] & G_2 \ar[d] \\
                G_1 \ar[r, "L_1"] & G_1 \ar[r] & \Spec \Fq
            \end{tikzcd}
        \end{center}
        whose squares are all Cartesian where we have written $L_i: G_i \to G_i$ to denote the Lang isogeny for $i = 1,2$. The morphisms labeled $\varphi, \psi: G_1 \times G_2 \to G_1 \times G_2$ in the diagram are base changes of $L_1$ and $L_2$ respectively and hence are Galois \'etale of Galois groups $G_1(\Fq)$ and $G_2(\Fq)$ respectively. In fact, these Galois \'etale covers correspond to the surjective group homomorphisms
        \begin{align} \label{eq:surjective_group_homomorphisms_from_etale_fundamental_group_of_product_group_to_Fq_points_of_each_component_group_from_lang_isogeny}
           \pioneet(G_{1} \times G_2, e) \to   \pioneet(G_{1}, e) &\twoheadrightarrow G_1(\Fq) \\
            \pioneet(G_{1} \times G_2, e) \to \pioneet(G_{2}, e) &\twoheadrightarrow G_2(\Fq) \nonumber
        \end{align}
        where the maps from $\pioneet(G_{1} \times G_2, e)$ are induced by the natural projection maps on the group schemes. Note that base changing the Lang isogeny exact sequence 
        $$1 \to G(\Fq) \to G \xrightarrow{L} G \to 1$$
        to $\Fqbar$ results in the exact sequence
        $$1 \to G(\Fq) \to G_{\Fqbar} \xrightarrow{L} G_{\Fqbar} \to 1$$
        describing a Galois \'etale cover of connected group schemes, so the homomorphisms \eqref{eq:surjective_group_homomorphisms_from_etale_fundamental_group_of_product_group_to_Fq_points_of_each_component_group_from_lang_isogeny} are in fact surjective even when restricted the subgroup $\pioneet((G_1 \times G_2)_{\Fqbar}, e)$ of $\pioneet(G_1 \times G_2, e)$.  

        Therefore, the product group homomorphism 
        \begin{equation} \label{eq:group_homomorphism_from_etale_fundamental_group_of_product_of_group_schemes_to_product_of_Fq_rational_points_via_lang_isogenies}
        \pioneet(G_{1} \times G_2, e) \to G_1(\Fq) \times G_2(\Fq)
        \end{equation}
        induced by the two group homomorphisms of \eqref{eq:surjective_group_homomorphisms_from_etale_fundamental_group_of_product_group_to_Fq_points_of_each_component_group_from_lang_isogeny} is surjective and corresponds to the composed cover 
        $$G_1 \times G_2 \xrightarrow{\varphi} G_1 \times G_2 \xrightarrow{\psi} G_1 \times G_2.$$
        Note that $\varphi = (L_1, \id_{G_2})$ and $\psi = (\id_{G_1} \times L_2)$, so the composed covering morphism above is $(L_1, L_2)$, which is exactly the Lang isogeny of $(G_1 \times G_2)$ over $\Fq$.

        By construction, the lisse sheaf $\calL_{\chi_i}$ corresponds to the composed group homomorphism
        $$\pioneet(G_i, e) \to G_i(\Fq) \xrightarrow{\chi_i} \Qellbar^\times$$
        and hence its pullback $\operatorname{pr}_i^* \calL_{\chi_i}$ to $G_1 \times G_2$ corresponds to the composition
        $$\pioneet(G_{1} \times G_2, e) \to \pioneet(G_{i}, e) \twoheadrightarrow G_i(\Fq) \xrightarrow{\chi_i} \Qellbar^\times$$
        of the group homomorphism in \eqref{eq:surjective_group_homomorphisms_from_etale_fundamental_group_of_product_group_to_Fq_points_of_each_component_group_from_lang_isogeny} with $\chi_i$. The tensor product $\operatorname{pr}_1^* \calL_{\chi_1} \times \operatorname{pr}_2^* \calL_{\chi_2}$ thus corresponds to the homomorphism

        \begin{equation} \label{eq:homomorphism_from_fundamental_group_of_product_of_algebraic_groups_over_finite_field_via_product_of_characters}
        \pioneet(G_{1} \times G_2, e) \to \pioneet(G_{1}, e) \times \pioneet(G_{2}, e) \twoheadrightarrow G_1(\Fq) \times G_2(\Fq) \xrightarrow{(g_1,g_2) \mapsto \chi_1(g_1) \cdot \chi_2(g_2)} \Qellbar^\times.
        \end{equation}

        On the other hand, $\calL_\chi$ corresponds to the group homomorphism 
        \begin{equation}\label{eq:homomorphism_from_fundamental_group_of_product_of_algebraic_groups_over_finite_field_via_character_on_product}
        \pioneet(G_1 \times G_2, e) \to G_1(\Fq) \times G_2(\Fq) \cong (G_1 \times G_2)(\Fq) \xrightarrow{\chi} \Qellbar
        \end{equation}
        where the first homomorphism is \eqref{eq:group_homomorphism_from_etale_fundamental_group_of_product_of_group_schemes_to_product_of_Fq_rational_points_via_lang_isogenies}. Since $\chi(g_1,g_2) = \chi_1(g_1) \cdot \chi_2(g_2)$, and since the homomoprhism \eqref{eq:group_homomorphism_from_etale_fundamental_group_of_product_of_group_schemes_to_product_of_Fq_rational_points_via_lang_isogenies} is the product of the homomorphisms of \eqref{eq:surjective_group_homomorphisms_from_etale_fundamental_group_of_product_group_to_Fq_points_of_each_component_group_from_lang_isogeny}, the two homomorphisms \eqref{eq:homomorphism_from_fundamental_group_of_product_of_algebraic_groups_over_finite_field_via_product_of_characters} and \eqref{eq:homomorphism_from_fundamental_group_of_product_of_algebraic_groups_over_finite_field_via_character_on_product} coincide. The two lisse sheaves $\calL_\chi$ and $\operatorname{pr}_1^* \calL_{\chi_1} \otimes \operatorname{pr}_2^* \calL_{\chi_2}$ are thus naturally isomorphic.
    \end{enumerate}
\end{proof}

\begin{proposition}
    Let $\{G_i\}_{i}$ be a finite collction of connected commutative algebraic groups over a finite field $\Fq$. Write $G = \prod_i G_i$ and write $\operatorname{pr}_i: G \to G_i$ for the projection morphism. Let $\ell \neq \Char \Fq$ be a prime. 
    Let $M \in \Perv(G)$ be arithmetically semisimple and pure of weight $0$. Suppose for each $i$ that $R\operatorname{pr}_{i!} M$ is arithmetically isomorphic to a direct sum $\delta_{x_i} \oplus N_i$ where $\delta_{x_i}$ is a skyscraper sheaf of rank $1$ supported at a point $x_i \in G_i(\Fq)$ and $N_i$ is a negligible object on $G_i$. Further suppose that there exists some $i_0$ such that $H_c^*(G_{i_0, \Fqbar}, N_{i_0}) = 0$.
    
    The object $M$ itself must be direct sum of a skyscraper sheaf of rank $1$ and a negligible object. 
    
\end{proposition}
\begin{proof}

    The hypotheses and conclusion are preserved under base change by a finite extension of $\Fq$; we may thus apply such base changes and replace $\Fq$ with appropriate finite extensions in the process. 

    Say that $R\operatorname{pr}_{i!} M$ is the direct sum $\delta_{x_i} \oplus N_i$ of a skyscraper sheaf $\delta_{x_i}$ supported at a point $x_i \in G_i(\Fq)$ and a negligible object $N_i$ on $G_i$. 
    
    Given a character $\chi_i \in \widehat{G}_i(\Fqn)$, \Cref{proposition:some_properties_of_the_arithmetic_fourier_transform} shows that
    $$S(M, \operatorname{pr}_i^* \chi_i) = S(M_{\operatorname{pr}_i^* \chi_i}, 1_n) = S(R\operatorname{pr}_{i!} (M_{\operatorname{pr}_i^* \chi_i}), 1_n).$$
    By the projection formula \TODO{cite the projection formula}, this equals
    \begin{align*}
    S((R\operatorname{pr}_{i!} M)_{\chi_i}, 1_n) &= S(R\operatorname{pr}_{i!} M, \chi_i) \\
    &= S(\delta_{x_i} \oplus N_i, \chi_i) \\
    &= S(\delta_{x_i}, \chi_i) \oplus  S(N_i, \chi_i).
    \end{align*}
    For generic $\chi_i \in \widehat{G}_i$, the value $S(N_i, \chi_i)$ equals $0$, so the above equals 
    $S(\delta_{x_i}, \chi_i)$. Since $\delta_{x_i}$ is skyscraper, we have 
    $$S(\delta_{x_i}, \chi_i) = \sum_{y_i \in G_i(\Fqn)} t_{\delta_{x_i}}(y_i; \Fqn) \cdot \chi_i(y_i) = t_{\delta_{x_i}}(x_i; \Fqn) \cdot \chi_i(x_i) = S(\delta_{x_i}, 1_n) \cdot \chi_i(x_i).$$
    In turn, this equals 
    $$S(\delta_{x_i}, 1_n) \cdot \chi_i(x_i) = S(R\operatorname{pr}_{i!} M, 1_n) \cdot \chi_i(x_i) = S(M, 1_n) \cdot \chi_i(x_i).$$
    We have thus shown that 
    \begin{equation} \label{eq:applying_character_on_component_group_to_arithmetic_fourier_transform_of_an_object_whose_pushforward_is_skyscraper_plus_negligible_multiplies_the_value_by_the_character_value_at_the_point}
    S(M, \operatorname{pr}_i^* \chi_i) = S(M, 1_n) \cdot \chi_i(x_i)
    \end{equation}
    holds for generic $\chi_i \in \widehat{G}_i$. 

    Let $x \in G(\Fq)$ be the point $(x_i)_i$. For each $i$, the skyscraper sheaf $\delta_{x_i}$ corresponds to a continuous character $\Gal(\Fqbar/\Fq) \to \Qellbar^\times$; in particular, the Frobenius element must be sent to a root of unity. Replace $\Fq$ by a finite extension by base change so that each skyscraper sheaf $\delta_{x_i}$ corersponds to the trivial character $\Gal(\Fqbar/\Fq) \to \Qellbar^\times$. Let $\delta_x$ be the trivial skyscraper sheaf of rank $1$ at $x$. By assumption, there is some $i_0$ such that $H_c^*(G_{i_0, \Fqbar}, N_{i_0}) = 0$, so in particular $S(N_{i_0}, 1_{n}) = 0$ for every $n \geq 1$ by the Grothendieck-Lefschetz trace formula. Further note that 
    \begin{align} \label{eq:object_whose_pushforwards_are_skyscraper_mod_negligibles_and_well_chosen_skyscraper_have_equal_arithmetic_fourier_transform}
    S(M, 1_n) &= S(R\operatorname{pr}_{i_0 !} M, 1_n) \\ \nonumber
    &= S(\delta_{x_{i_0}} \oplus N_{i_0} , 1_n) \\ \nonumber
    &= S(\delta_{x_{i_0}}, 1_n) + S(N_{i_0}, 1_n) \\ \nonumber
    &= S(\delta_{x_{i_0}}, 1_n) + 0 \\ \nonumber
    &= S(R\operatorname{pr}_{i_0 !} \delta_x, 1_n) \\ \nonumber
    &= S(\delta_x, 1_n). \nonumber
    \end{align}
    
    The ideas used to show \eqref{eq:applying_character_on_component_group_to_arithmetic_fourier_transform_of_an_object_whose_pushforward_is_skyscraper_plus_negligible_multiplies_the_value_by_the_character_value_at_the_point} show that 
    \begin{equation} \label{eq:applying_character_on_component_group_to_arithmetic_fourier_transform_of_a_skyscraper_multiplies_the_value_by_the_character_value_at_the_point}
    S(\delta_x, \operatorname{pr}_i^* \chi_i) = S(\delta_x, 1_n) \cdot \chi_i(x_i)
    \end{equation}
    for any $\chi_i \in \widehat{G}_i(\Fqn)$. 

    Let $\chi \in \widehat{G}(\Fqn)$ be a character and let it correspond to the tuple $(\chi_i)_i$ of $\prod_i \widehat{G}_i(\Fqn)$ via \Cref{lemma:characters_on_product_of_connected_commutative_algebraic_groups_over_a_finite_field_correspond_to_tuples_of_characters_of_the_individual_groups}. \eqref{eq:applying_character_on_component_group_to_arithmetic_fourier_transform_of_an_object_whose_pushforward_is_skyscraper_plus_negligible_multiplies_the_value_by_the_character_value_at_the_point} and \eqref{eq:applying_character_on_component_group_to_arithmetic_fourier_transform_of_a_skyscraper_multiplies_the_value_by_the_character_value_at_the_point} show that  
    \begin{align*}
    S(M, \calL_\chi) &= S(M, 1_n) \cdot \prod_i \chi_i(x_i) \quad \text{for generic } \chi \in \widehat{G} \\
    S(\delta_x, \calL_\chi) &= S(\delta_x, 1_n) \cdot \prod_i \chi_i(x_i) \quad \text{for all } \chi \in \widehat{G}(Fqn).
    \end{align*}
    By \eqref{eq:object_whose_pushforwards_are_skyscraper_mod_negligibles_and_well_chosen_skyscraper_have_equal_arithmetic_fourier_transform}, $S(M, \calL_\chi)$ and $S(\delta_x, \calL_\chi)$ thus coincide for generic $
    \chi \in \widehat{G}$. The generic Fourier invertibility theorem \cite[Theorem 6.11]{forey_fresan_kowalski_aftff} \TODO{ref the invertibility theorem} thus concludes that $M$ and $\delta_x$ are arithmetically isomorphic as objects of $\mathbf{P}_{\mathrm{int}}(G)$. 
    \TODO{re-verify this citation for the final version of FFK}
\end{proof}



\section{Fourier-Mukai equivalence}

\TODO{TODO: Read the statements in this section to verify}


\begin{definition}
Let \(X\) and \(Y\) be smooth projective varieties over a field, and let 
\[
K \in D^b(\mathrm{Coh}(X \times Y))
\]
be an object in the bounded derived category of coherent sheaves on the product \(X \times Y\).

Denote by 
\[
p : X \times Y \to Y, \quad q : X \times Y \to X
\]
the projection maps onto the second and first factors, respectively.

The \hldef{Fourier--Mukai transform} associated to the kernel \(K\) is the exact functor
\[
\Phi_K : D^b(\mathrm{Coh}(X)) \to D^b(\mathrm{Coh}(Y))
\]
defined by
\[
\Phi_K(\mathcal{F}) = R p_*\bigl( q^* \mathcal{F} \otimes^{\mathbf{L}} K \bigr),
\]
where 
\begin{itemize}
  \item \(q^*\) is the (derived) pullback functor,
  \item \(\otimes^{\mathbf{L}}\) denotes the derived tensor product,
  \item \(R p_*\) is the derived pushforward functor.
\end{itemize}

This construction generalizes the classical Fourier transform to the setting of derived categories and algebraic geometry.

\end{definition}

\begin{definition}
Let $X$ and $Y$ be smooth projective varieties over a field $k$. Denote by $D^b(\mathrm{Coh}(X))$ and $D^b(\mathrm{Coh}(Y))$ the bounded derived categories of coherent sheaves on $X$ and $Y$, respectively. Let $K \in D^b(\mathrm{Coh}(X \times Y))$. The \emph{Fourier--Mukai transform} with kernel $K$ is the exact functor
$$
\Phi_K : D^b(\mathrm{Coh}(X)) \to D^b(\mathrm{Coh}(Y)), \quad \Phi_K(\mathcal{F}) = Rp_*\big(q^*\mathcal{F} \otimes^{\mathbf{L}} K \big),
$$
where $q: X \times Y \to X$ and $p: X \times Y \to Y$ are the natural projection morphisms.
\end{definition}

\begin{proposition}[Composition of Fourier--Mukai Functors]
Let $X, Y, Z$ be smooth projective varieties, and let $K \in D^b(\mathrm{Coh}(X \times Y))$, $L \in D^b(\mathrm{Coh}(Y \times Z))$. Define the object $K \star L \in D^b(\mathrm{Coh}(X \times Z))$ by
$$
K \star L = Rp_{13*}\bigl( p_{12}^* K \otimes^{\mathbf{L}} p_{23}^* L \bigr),
$$
where $p_{12}, p_{23}, p_{13}$ are the projections from $X \times Y \times Z$ onto the corresponding factors. Then
$$
\Phi_L \circ \Phi_K \cong \Phi_{K \star L}.
$$
\end{proposition}

\begin{theorem}[Orlov's Representability Theorem for Fully Faithful Functors]
Let $X$ and $Y$ be smooth projective varieties over a field. Every exact fully faithful functor
$$
F: D^b(\mathrm{Coh}(X)) \to D^b(\mathrm{Coh}(Y))
$$
is isomorphic to a Fourier--Mukai functor $\Phi_K$ for some $K \in D^b(\mathrm{Coh}(X \times Y))$.
\end{theorem}

\begin{theorem}[Equivalences as Fourier--Mukai Transforms]
Let $X$ and $Y$ be smooth projective varieties over a field. If $\Phi_K : D^b(\mathrm{Coh}(X)) \to D^b(\mathrm{Coh}(Y))$ is an exact equivalence of derived categories, then its quasi-inverse is isomorphic to a Fourier--Mukai functor $\Phi_{K'}$, where
$$
K' = R\mathcal{H}om_{X \times Y}(K, p_X^! \mathcal{O}_X)
$$
(up to dualization and permutation of factors), and $\Phi_{K'} \circ \Phi_K \cong \mathrm{Id}$, $\Phi_K \circ \Phi_{K'} \cong \mathrm{Id}$.
\end{theorem}


\begin{theorem}[Mukai's Equivalence Theorem]
Let $A$ be an abelian variety of dimension $g$ over an algebraically closed field, and let $\widehat{A} = \operatorname{Pic}^0(A)$ denote its dual abelian variety. Let $\mathcal{P}$ denote the normalized Poincar\'e line bundle on $A \times \widehat{A}$. The Fourier--Mukai functor
$$
\Phi_{\mathcal{P}}: D^b(\mathrm{Coh}(A)) \to D^b(\mathrm{Coh}(\widehat{A}))
$$
is an exact equivalence of triangulated categories, where $p: A\times \widehat{A} \to A$ and $\widehat{p} : A \times \widehat{A} \to \widehat{A}$ are the projections.
\end{theorem}

\begin{proposition}[Parseval Formula for Fourier--Mukai Functors]
Let $X$ and $Y$ be smooth projective varieties over a field, let $K \in D^b(\mathrm{Coh}(X \times Y))$, and let $\Phi_K: D^b(\mathrm{Coh}(X)) \to D^b(\mathrm{Coh}(Y))$ be the associated Fourier--Mukai functor. For all $\mathcal{F}, \mathcal{G} \in D^b(\mathrm{Coh}(X))$ and all $i \in \mathbb{Z}$,
$$
\operatorname{Hom}_{D^b(\mathrm{Coh}(X))}(\mathcal{F}, \mathcal{G}[i]) \cong \operatorname{Hom}_{D^b(\mathrm{Coh}(Y))}(\Phi_K(\mathcal{F}), \Phi_K(\mathcal{G})[i])
$$
if $\Phi_K$ is fully faithful. In particular, this isomorphism holds if $\Phi_K$ is an equivalence.
\end{proposition}

\begin{theorem}[Generic Vanishing for Abelian Varieties]
Let $A$ be a complex abelian variety, and $\mathcal{F}$ a coherent sheaf on $A$. For each $i \geq 0$, define the \emph{cohomological support loci}
$$
V^i(\mathcal{F}) = \{ \alpha \in \widehat{A} : H^i(A, \mathcal{F} \otimes \mathcal{P}_\alpha) \neq 0 \},
$$
where $\mathcal{P}_\alpha$ is the restriction of the Poincar\'e bundle to $A \times \{\alpha\}$. Then for a generic $\alpha \in \widehat{A}$,
$$
H^i(A, \mathcal{F} \otimes \mathcal{P}_\alpha) = 0 \quad \text{for all } i > \dim \operatorname{Supp} \mathcal{F}.
$$
\end{theorem}

\begin{theorem}[Index Theorem for the Fourier--Mukai Transform]
Let $A$ be an abelian variety over an algebraically closed field with dual $\widehat{A}$ and normalized Poincar\'e bundle $\mathcal{P}$. Then for any $x \in A$, consider the skyscraper sheaf $\mathcal{O}_x$. Its image under the Fourier--Mukai equivalence $\Phi_{\mathcal{P}}$ is a line bundle:
$$
\Phi_{\mathcal{P}}(\mathcal{O}_x) \cong \mathcal{P}_x[-g],
$$
where $\mathcal{P}_x$ is the restriction of $\mathcal{P}$ to $\{x\} \times \widehat{A}$, and $g = \dim A$. In particular,
$$
H^i(A, \mathcal{L}) = 0 \text{ for all } i \neq \operatorname{ind}(\mathcal{L})
$$
for $\mathcal{L}$ a sufficiently ample line bundle, where $\operatorname{ind}(\mathcal{L})$ is the unique $i$ for which $H^i(A, \mathcal{L}) \ne 0$.
\end{theorem}




\appendix

\section{Comparisons}

% Here are thus the comparisons 


\section{Analysis and \texorpdfstring{$L^p$}{Lp}-norms}


\subsection{(Extended) norms and metrics}

\begin{definition}[Absolute Value on a Field] \label{definition:absolute_value_on_a_field}
    Let \( F \) be a \CrefAndHyperrefIfExist{definition:field}{field}. An \hldef{absolute value on $F$} is a function 
    \[
    | \cdot | : F \to \mathbb{R}_{\geq 0}
    \]
    that satisfies the following properties for all \( a, b \in F \):

    \begin{enumerate}
        \item Non-negativity: \( |a| \geq 0 \),
        \item Positive-definiteness: \( |a| = 0 \iff a = 0 \),
        \item Multiplicativity: \( |ab| = |a| \cdot |b| \),
        \item Triangle inequality: \( |a + b| \leq |a| + |b| \).
    \end{enumerate}

    Here, \( 0 \) denotes the additive identity of the field \( F \), and the codomain \( \mathbb{R}_{\geq 0} \) consists of non-negative real numbers.
\end{definition}

\begin{remark}
    For our analytic discussions, the valued field will usually be $\bbR$ or $\bbC$ with the usual absolute values. In particular, for $a+bi \in \bbC$ with $a,b \in bbR$, $|a+bi| = \sqrt{a^2+b^2}$.
\end{remark}

\begin{definition}[Extended Norm] \label{definition:extended_norm_on_a_vector_space_over_a_field_with_absolute_value}
    Let \( V \) be a \CrefAndHyperrefIfExist{definition:vector_space_over_a_field}{vector space} over a \CrefAndHyperrefIfExist{definition:field}{field} \( F \) equipped with an \CrefAndHyperrefIfExist{definition:absolute_value_on_a_field}{absolute value}
    \[
    |\cdot| : F \to [0, \infty).
    \]
    An \hldef{extended norm on $V$} is a function
    \[ \|\cdot\| : V \to [0, \infty] \]
    satisfying for all \( x,y \in V \) and all scalars \( \alpha \in F \):
    \begin{enumerate}
        \item \textbf{Positive definiteness:} \quad \(\|x\| = 0\) if and only if \(x = 0\).
        \item \textbf{Homogeneity:} \quad \(\|\alpha x\| = |\alpha| \cdot \|x\|\).
        \item \textbf{Triangle inequality:} \quad \(\|x + y\| \leq \|x\| + \|y\|\),
    \end{enumerate}
    where arithmetic is extended to allow sums involving \(\infty\) with the convention that \(a + \infty = \infty\) for any \(a \in [0, \infty]\). A vector space with an extended norm over a field with an absolute value is called an \hldef{extended normed (vector) space}.

    If the range of the extended norm is contained in $[0,\infty)$, then the extended norm is a \hldef{norm} in the usual sense and $V$ may be called a \hldef{normed (vector) space}.
\end{definition}


\begin{definition}[Topology induced by a norm] \label{definition:topology_induced_by_a_norm_on_a_vector_space_over_a_field_with_absolute_value}
Let $V$ be a vector space over a field $K$ with \hyperrefIfExists{definition:absolute_value_on_a_field}{absolute value}\CrefIfExists{definition:absolute_value_on_a_field} $|\cdot|$, and let $\|\cdot\|$ be an \hyperrefIfExists{definition:extended_norm_on_a_vector_space_over_a_field_with_absolute_value}{extended norm on $V$}\CrefIfExists{definition:extended_norm_on_a_vector_space_over_a_field_with_absolute_value}. The \hldef{topology induced by the extended norm $\|\cdot\|$ on $V$} is defined by declaring a subset $U \subseteq V$ to be open if for every $x \in U$, there exists $\varepsilon > 0$ such that 
$$\hlin{B(x, \varepsilon) := \{ y \in V : \|y - x\| < \varepsilon \}}$$
is contained in $U$. The set $B(x, \varepsilon)$ is called the \hldef{open ball of radius $\varepsilon$ around $x$}.
The collection of all such open sets forms a topology on $V$.

\TextIfExists{definition:topology_induced_by_an_extended_metric_and_open_ball_for_an_extended_metric}{
Equivalently, the topology on $V$ induced by the extended norm $\|\cdot\|$ is the \hyperrefIfExists{definition:topology_induced_by_an_extended_metric_and_open_ball_for_an_extended_metric}{topology on $V$ induced by}\CrefIfExists{definition:topology_induced_by_an_extended_metric_and_open_ball_for_an_extended_metric} the extended metric $d: V \times V \to [0,\infty]$ \hyperrefIfExists{definition:extended_metric_induced_by_an_extended_norm_on_a_vector_space_over_a_field_with_absolute_value}{induced by $\|\cdot\|$} \CrefIfExists{definition:extended_metric_induced_by_an_extended_norm_on_a_vector_space_over_a_field_with_absolute_value}.
}

\end{definition}

\begin{definition}[Extended Metric] \label{definition:extended_metric_on_a_set}
    Let \(M\) be a set. An \hldef{extended metric} on \(M\) is a function
    \[
    d : M \times M \to [0, \infty]
    \]
    such that for all \(x,y,z \in M\):
    \begin{enumerate}
        \item \textbf{Non-negativity:} \quad \(d(x,y) \geq 0\), and \(d(x,y) = 0\) if and only if \(x = y\).
        \item \textbf{Symmetry:} \quad \(d(x,y) = d(y,x)\).
        \item \textbf{Triangle inequality:} \quad \(d(x,z) \leq d(x,y) + d(y,z)\),
    \end{enumerate}
    again adopting the convention that sums involving \(\infty\) behave so that \(a + \infty = \infty\). A set equipped with an extended metric is called an \hldef{extended metric space}.

    If the range of the extended metric is contained in $[0,\infty)$, then the extended metric is a \hldef{metric} in the usual sense and $V$ may be called a \hldef{metric space}.
\end{definition}


\begin{remark}
    If the codomain of an extended norm or an extended metric is $[0, \infty)$, then it is a (usual) norm or metric.
\end{remark}

\begin{definition}[Extended Metric Induced by an Extended Norm] \label{definition:extended_metric_induced_by_an_extended_norm_on_a_vector_space_over_a_field_with_absolute_value}
    Let \( V \) be a vector space over a field \( F \) equipped with an \hyperrefIfExists{definition:absolute_value_on_a_field}{absolute value}\CrefIfExists{definition:absolute_value_on_a_field}
    \[
    |\cdot| : F \to [0, \infty),
    \]
    and let \( \|\cdot\| : V \to [0, \infty] \) be an \hyperrefIfExists{definition:extended_norm_on_a_vector_space_over_a_field_with_absolute_value}{extended norm on \( V \)}\CrefIfExists{definition:extended_norm_on_a_vector_space_over_a_field_with_absolute_value}.  
    Then the \hldef{extended metric induced by the extended norm} is the function
    $$\hlin{d : V \times V \to [0, \infty]}$$
    defined by
    \[
    d(x, y) := \|x - y\|.
    \]
    It is indeed an extended metric. If $\|\cdot\|$ is a norm, then $d$ is a metric.
\end{definition}

\begin{definition}[Convergence and Limits in an Extended Metric Space] \label{definition:convergence_and_limits_in_extended_metric_space}
    Let \((X, d)\) be an \CrefAndHyperrefIfExist{definition:extended_metric_on_a_set}{extended metric space}.
    A sequence \((x_n)_{n=1}^\infty\) in \(X\) \hldef{converges to a point $x \in X$} if for every \(\varepsilon > 0\), there exists an integer \(N \in \mathbb{N}\) such that for all \(n \geq N\),
    \[
    d(x_n, x) < \varepsilon.
    \]
    In that case, \(x\) is called the \hldef{limit of the sequence $(x_n)$}, and we write
    $$\hlin{\lim_{n \to \infty} x_n = x \quad \text{or} \quad x_n \to x}$$
\end{definition}

\subsection{\texorpdfstring{$L^p$}{Lp}-norms and \texorpdfstring{$L^p$}{Lp}-spaces on measure spaces.}

\begin{definition} \label{definition:L_p_norm_on_a_measure_space}
Let \( (X, \mathcal{A}, \mu) \) be a measure space. 
    \begin{enumerate}
        \item Let \( 1 \leq p < \infty \). Given a measurable function $f: X \to \bbC$, the \hldef{$L^{p}$-norm} is the \hyperrefIfExists{definition:extended_norm_on_a_vector_space_over_a_field_with_absolute_value}{norm}\CrefIfExists{definition:extended_norm_on_a_vector_space_over_a_field_with_absolute_value} defined by
        $$\hlin{\|f\|_{p} = \left( \int_{X} |f(x)|^{p} \, d\mu(x) \right)^{\frac{1}{p}}.}$$
        

        \item Let \( p = \infty \). Given a measurable function $f: X \to \bbC$, the \hldef{$L^\infty$-norm} is the \hyperrefIfExists{definition:extended_norm_on_a_vector_space_over_a_field_with_absolute_value}{norm}\CrefIfExists{definition:extended_norm_on_a_vector_space_over_a_field_with_absolute_value} defined by
        $$\hlin{\|f\|_{\infty} = \inf \{ M \geq 0 : |f(x)| \leq M \quad \text{for } \mu\text{-almost every } x \in X \}.}$$

        \item Let $1 \leq p \leq \infty$. The \hldef{$L^{p}$ space on $X$} is defined as the set
        $$\hlin{L^{p}(X) = L^{p}(X, \mu) =\left\{ f : X \to \mathbb{C} \mid f \text{ is measurable and } \|f\|_p < \infty \right\}.}$$
        An element $f \in L^p(X)$ is called an \hldef{$L^p$-function on the measure space $X$.}

        % The \hldef{$L^\infty$ space on $X$} is defined as the set

        % \hl{$L^{\infty}(X) = L^{\infty}(X, \mu)$} consists of essentially bounded measurable functions with the norm
        % In other words, $f \in L^\infty(X,\mu)$ if and only if there exists a constant $C \geq 0$ such that $|f(x)| \leq C$ for $\mu$-almost every $x \in X$.
    \end{enumerate}
\end{definition}

\begin{remark} \label{remark:space_of_measurable_functions_is_an_extended_metric_space_for_any_Lp_metric}
    Let $(X,\mathcal{A},\mu)$ be a measure space, and let $1 \leq p \leq \infty$. The $L^p$-norm $\|\cdot\|_p$ is an extended norm on the space of all measurable functions $f: X \to \bbC$. It thus induces an extended metric on the space of all measurable functions.
\end{remark}

\begin{theorem}[See for example {\cite[Theorem 3.11]{rudin_real_and_complex}}]
    Let $(X,\calA, \mu)$ be a measure space. Let $1 \leq p \leq \infty$. The $L^p$-norm $\|\cdot\|_p$ induces a complete metric on $L^p(X)$.
\end{theorem}

\begin{convention}
    Let $1 \leq p \leq \infty$. Unless otherwise specified, 
    \begin{enumerate}
        \item the $L^p$-space on a measure space is assumed to be made into a metric space by equipping it with the \hyperrefIfExists{definition:extended_metric_induced_by_an_extended_norm_on_a_vector_space_over_a_field_with_absolute_value}{metric induced by}\CrefIfExists{definition:extended_metric_induced_by_an_extended_norm_on_a_vector_space_over_a_field_with_absolute_value} the \hyperrefIfExists{definition:L_p_norm_on_a_measure_space}{$L^p$-norm}\CrefIfExists{definition:L_p_norm_on_a_measure_space}.  
        \item the space of all measurable functions on $X$ is assumed to be made into an extended metric space by equipping it with the extended metric induced by the $L^p$-norm.  
    \end{enumerate}
\end{convention}

\begin{lemma} \label{lemma:compactly_supported_simple_measurable_functoins_are_in_L_p}
    Let $(X,\calA, \mu)$ be a measure space. Any (complex) measurable simple function $s: X \to \bbC$ on $X$ such that $$ \mu(\{x: s(x) \neq 0\})<\infty$$ is in $L^p(X, \mu)$. 
\end{lemma}

For finite measure spaces, we have the following inclusions of the $L^p(X,\mu)$ for varying $1 \leq p$; however, we are not concerned with such inclusions as we are concerned with $L^1$ and $L^2$-functions on the infinite measure space $\bbR$. 

\begin{proposition}
    Let \((X, \mathcal{A}, \mu)\) be a measure space with finite measure \( \mu(X) < \infty \). For \(1 \leq p < q \leq \infty\), the following inclusion holds:
    \[
    L^q(X, \mu) \subseteq L^p(X, \mu)
    \]
    and there exists a constant \(C > 0\) such that for all \(f \in L^q(X, \mu)\),
    \[
    \|f\|_{L^p} \leq C \, \|f\|_{L^q}.
    \]
\end{proposition}

We also can speak of measurable functions converging, with respect to an $L^p$-norm for some $p$, to a measurable function.


\begin{definition} \label{definition:sequence_of_measurable_functions_on_a_measure_space_that_converges_to_a_function_in_the_L_p_norm}
    Let \((X, \mathcal{A}, \mu)\) be a measure space and \(1 \leq p \leq \infty\). 
    %Since $L^p(X, \mu)$ is a (complete) measure space with respect to the measure induced by the $L^p$-norm, we may speak of convergence and  
    
    A sequence of measurable functions \(\{f_n\}_{n=1}^\infty\), where each \(f_n : X \to \mathbb{R}\) (or \(f_n : X \to \mathbb{C}\)), \hldef{converges to a function $f: X \to \mathbb{R}$ (or $f: X \to \mathbb{C}$) in the $L^p$ norm} if and only if the sequence $(f_n)_{n=1}^\infty$ converges to $f$ in the sense of (see Definition \ref{definition:convergence_and_limits_in_extended_metric_space}) the extended metric space which (see Remark \ref{remark:space_of_measurable_functions_is_an_extended_metric_space_for_any_Lp_metric}) is the space of all measurable functions equipped with the metric induced by the $L^p$ norm, i.e. 

    \[ \lim_{n \to \infty} \|f_n - f\|_p = 0. \]

    We may thus also say that $f$ is the \hldef{limit of $(f_n)_{n=1}^\infty$ for the $L^p$ norm on $X$}, and write 
    $$\hlin{f = \lim_{n \to \infty} f_n \text{ in } L^p}$$
    or 
    $$\hlin{f_n \to f \text{ in } L^p}.$$
\end{definition}

\begin{lemma}
    Let $(X,\calA,\mu)$ be a measure space, let $\{f_n\}_{n=1}^\infty$ be a sequence of measurable functions $f_n: X \to \bbR$ (or $f_n: X \to \bbC$), and let $1 \leq p \leq \infty$. 
    \begin{enumerate}
        \item If there exists some function $f: X \to \bbR$ (or $f: X \to \bbC$) such that $f_n \to f$ in $L^p$, then $f$ is measurable.
        \item If $f,g$ are functions $X \to \bbR$ (or $X \to \bbC$ such that $f_n \to f$ and $f_n \to g$ in $L^p$, then $f = g$ $\mu$-almost everywhere.
    \end{enumerate}
\end{lemma}

% \begin{definition}[Norm on a vector space]
% Let $V$ be a vector space over a field $K$ equipped with an absolute value $|\cdot|$. A \hldef{norm} on $V$ is a function
% \[
% \|\cdot\| : V \to \mathbb{R}_{\geq 0}
% \]
% such that for all $u,v \in V$ and $\alpha \in K$, the following hold:
% \begin{enumerate}
%     \item $\|v\| = 0$ if and only if $v = 0$,
%     \item $\|\alpha v\| = |\alpha| \cdot \|v\|$,
%     \item $\|u + v\| \leq \|u\| + \|v\|$ (triangle inequality).
% \end{enumerate}
% \end{definition}


Any element in the $L^p$-space $L^p(X, \mu)$ of a general measure space can be $L^p$-approximated by simple functions with finite measure support. We formulate this idea in terms of \emph{density} of subspaces of $L^p(X, \mu)$ with respect to the \hyperrefIfExists{definition:topology_induced_by_a_norm_on_a_vector_space_over_a_field_with_absolute_value}{topology induced by}\CrefIfExists{definition:topology_induced_by_a_norm_on_a_vector_space_over_a_field_with_absolute_value} the \hyperrefIfExists{definition:L_p_norm_on_a_measure_space}{$L^p$-norm}\CrefIfExists{definition:L_p_norm_on_a_measure_space}

\begin{notation}
    Let $(X,\calA, \mu)$ be a measure space. Let \hl{$C_c(X)$} be the class of all continuous (complex) functions on $X$ whose support is compact. 
\end{notation}

\begin{theorem}[See for example {\cite[Theorems 3.13, 3.14]{rudin_real_and_complex}}] \label{theorem:compactly_supported_simple_measurable_functions_are_dense_in_L_p_mu}
    Let $(X,\calA, \mu)$ be a measure space. Let $1 \leq p < \infty$.
    \begin{enumerate}
        \item Let $S$ be the class of all (complex), measurable, simple functions $s: X \to \bbC$ on $X$ such that
        $$ \mu(\{x: s(x) \neq 0\})<\infty.$$
        The class $S$ is dense in $L^{P}(\mu)$.
        \item  The class $C_c(X)$ is dense in $L^{P}(\mu)$.
    \end{enumerate} 
\end{theorem}

\begin{corollary} \label{corollary:intersection_of_L_p_s_is_dense_in_L_p}
    Let $(X, \calA, \mu)$ be a measure space. Let $1 \leq p,q < \infty$. The space $L^p(X, \mu) \cap L^q(X, \mu)$ is dense in $L^p (X, \mu)$. In particular, for any $f \in L^p(X,\mu)$, there exists a sequence $\{f_n\}_{n=1}^\infty$ functions $f_n \in L^q(x,\mu)$ such that $f^n \to f$ in $L^p$.
\end{corollary}
\begin{proof}
    By Lemma \ref{lemma:compactly_supported_simple_measurable_functoins_are_in_L_p}, all compactly supported simple measurable functions on $X$ are in $L^p(X,\mu) \cap L^q(X,\mu)$. By Theorem \ref{theorem:compactly_supported_simple_measurable_functions_are_dense_in_L_p_mu}, $L^p(X,\mu) \cap L^q(X,\mu)$ is thus dense in $L^p(X,\mu)$.
\end{proof}

\section{Hilbert spaces and Banach spaces}

\begin{definition} \label{definition:inner_product_on_a_vector_space_over_the_reals_or_complex_numbers}
An \hldef{inner product on a vector space $V$ over a field $\mathbb{F}$} (either $\mathbb{R}$ or $\mathbb{C}$) is a function
\[
\langle \cdot, \cdot \rangle : V \times V \to \mathbb{F}
\]
satisfying, for all $u,v,w \in V$ and all scalars $\alpha \in \mathbb{F}$:
\begin{enumerate}
    \item \textbf{Conjugate symmetry:} \quad $\langle u, v \rangle = \overline{\langle v, u \rangle}$
    \item \textbf{Linearity in the first argument:} \quad $\langle \alpha u + v, w \rangle = \alpha \langle u, w \rangle + \langle v, w \rangle$
    \item \textbf{Positive-definiteness:} \quad $\langle v, v \rangle \geq 0$ and $\langle v, v \rangle = 0$ if and only if $v = 0$
\end{enumerate}
A vector space over $\bbF$ equipped with such an inner product is called an \hldef{inner product space}.
\end{definition}


\begin{definition} \label{definition:norm_induced_by_an_inner_product_on_a_real_or_complex_vector_space}
    Let $V$ be a vector space over a field $\bbF \in \{\bbR, \bbC\}$ and let $\langle \cdot, \cdot \rangle$ be an \hyperrefIfExists{definition:inner_product_on_a_vector_space_over_the_reals_or_complex_numbers}{inner product}\CrefIfExists{definition:inner_product_on_a_vector_space_over_the_reals_or_complex_numbers}. The \hldef{norm induced by $\langle \cdot, \cdot \rangle$} is the \hyperrefIfExists{definition:extended_norm_on_a_vector_space_over_a_field_with_absolute_value}{norm}\CrefIfExists{definition:extended_norm_on_a_vector_space_over_a_field_with_absolute_value}
    $$\hlin{\|\cdot\|: V \to \mathbb{R}_{\geq 0}}$$
    defined by
    $$\| v \| = \sqrt{\langle v, v \rangle}$$
    for any $v \in V$.
\end{definition}

\begin{definition} \label{definition:hilbert_space}
    \TODO{is a hilbert space definable over a more general field}
    A \hldef{Hilbert space} $\mathcal{H}$ is a vector space over $\mathbb{R}$ or $\mathbb{C}$ equipped with an inner product such that \hyperrefIfExists{definition:extended_metric_induced_by_an_extended_norm_on_a_vector_space_over_a_field_with_absolute_value}{metric induced by} the \hyperrefIfExists{definition:norm_induced_by_an_inner_product_on_a_real_or_complex_vector_space}{norm induced by} the inner product makes $\mathcal{H}$ a complete metric space.
\end{definition}


\begin{proposition} \label{proposition:L2_is_a_hilbert_space}
    Let $(X,\calA, \mu)$ be a measure space. The space $L^2(X, \mu)$ is a Hilbert space when equipped with the inner product
    $$\hlin{(f, g)=\int_{X} f \overline{g} d \mu}$$
    and the normed induced by this inner product is simply the $L^2$-norm $\|\cdot\|_2$. 
\end{proposition}

\begin{convention} \label{convention:L2_is_a_hilbert_space}
    Given a measure space $(X,\calA, \mu)$, assume that $L^2(X, \mu)$ is equipped with the inner product of Proposition \ref{proposition:L2_is_a_hilbert_space}, unless otherwise specified.
\end{convention}

% \begin{definition} \label{definition:banach_space}
% A \hldef{Banach space} is a vector space \(X\) over the field \(\mathbb{R}\) or \(\mathbb{C}\) equipped with a norm \(\|\cdot\|\) such that \(X\) is complete with respect to the metric induced by the norm.
% \end{definition}

\begin{definition}[Banach space over a complete valued field] \label{definition:banach_space_over_a_complete_valued_field}
Let $k$ be a field equipped with a non-trivial absolute value $|\cdot| : k \to \mathbb{R}_{\geq 0}$ such that $k$ is complete with respect to the metric induced by $|\cdot|$.  
A \hldef{Banach space over $k$} is a vector space $X$ over $k$ equipped with a norm $\|\cdot\| : X \to \mathbb{R}_{\geq 0}$ satisfying:
\begin{align*}
\|x\| = 0 &\iff x = 0, \\
\|x+y\| &\leq \|x\| + \|y\| \quad \text{for all } x,y \in X, \\
\|\lambda x\| &= |\lambda|\cdot\|x\| \quad \text{for all } \lambda \in k, x \in X,
\end{align*}
such that the metric $d(x,y) := \|x-y\|$ makes $(X,d)$ into a complete metric space.
\end{definition}


\begin{proposition}
    \begin{enumerate}
    \item All Hilbert spaces are Banach spaces.
    \item For any measure space $(X, \calA, \mu)$, $L^p(X,\mu)$ is a Banach space when equipped with the $L^p$-norm $\|\cdot\|_p$.
    \end{enumerate}
\end{proposition}


\section{Miscellaneous definitions}

\begin{definition} \label{definition:vanish_at_infinity_for_a_complex_valued_function_on_a_locally_compact_hausdorff_space}
    A function $f: X \to \bbC$ on a locally compact Hausdorff space $X$ is said to \hldef{vanish at infinity} if to every $\epsilon>0$ there exists a compact set $K \subset X$ such that $|f(x)|< \epsilon$ for all $x$ not in $K$.

    Let \hl{$C_0(X)$} denote the space of functions $f: X \to \bbC$ that vanish at infinity.
\end{definition}

\begin{definition} \label{definition:artin_schreier_morphism_of_the_additive_group_over_a_scheme}
Let $S$ be a scheme of prime characteristic $p$. The \hldef{Artin--Schreier morphism} is the scheme morphism
\[
\hlalign{
\begin{align*}
\bbG_{a,S} &\to \bbG_{a,S} \\
T &\mapsto T^p - T
\end{align*}
}
\]
sometimes denoted by \hl{$\operatorname{AS}$} whose corresponding map $\operatorname{AS}^\sharp : k[T] \to k[T]$ of $k$-algebras  is given by $T \mapsto T^p - T$.
\end{definition}

\begin{definition} \label{definition:artin_schreier_sheaf_of_a_character_on_the_additive_group_of_a_finite_field}
Let $\Fq$ be a finite field. Let $\psi: \bbG_a(\Fq) \to \Qellbar^*$ be a nontrivial \hyperrefIfExists{definition:quasi_character_of_a_locally_compact_hausdorff_group}{character}. 
Note that the \hyperrefIfExists{definition:artin_schreier_morphism_of_the_additive_group_over_a_scheme}{Artin-Schreier morphism} on $\bbA^1/\Fq$ \hyperrefIfExists{theorem:finite_etale_covers_of_a_connected_scheme_correspond_to_finite_sets_with_an_action_of_the_etale_fundamental_group}{corresponds to}\CrefIfExists{theorem:finite_etale_covers_of_a_connected_scheme_correspond_to_finite_sets_with_an_action_of_the_etale_fundamental_group} a surjective group homomorphism $\pioneet(\bbG_a, 0) \to \bbG_a(\Fq)$\CrefIfExists{definition:etale_fundamental_group_of_a_connected_scheme}\CrefIfExists{lemma:artin_schreier_morphism_has_galois_group_the_additive_group}. The rank $1$ \hyperrefIfExists{definition:local_system_of_Lambda_modules_on_a_scheme}{local system} on $\bbG_a/\Fq$ corresponding to the composition
$$\pioneet(\bbG_a, 0) \to \bbG_a(\Fq) \xrightarrow{\psi} \Qellbar^*$$
\TODO{TODO: hyperlink a correspondence between local systems, applicable for $\Qellbar$-coefficients and representations of etale fundamental groups}
is called the \hldef{Artin-Schreier sheaf of $\psi$} or \hldef{Artin-Schreier local system of $\psi$}. It may commonly be denoted by \hl{$\calL_\psi$}.
\end{definition}

\begin{definition} \label{definition:relative_frobenius_morphism_of_a_scheme_over_a_scheme_of_prime_characteristic}
    Let $S$ be a scheme of prime characteristic $p$. Let $X$ be an $S$-scheme and let $\varphi: X \to S$ be a structure morphism. Let $n \in \bbZ$.
    The \hldef{relative $p^n$th power Frobenius morphism of $X/S$} is the morphism
    $$(F_{X,p^n}, \varphi): X \to  X \times_S S_{F,p^n} = X^{(p^n)/S}$$
    where
    \begin{itemize}
        \item $X^{(p^n)/S}$ denotes the \hyperrefIfExists{definition:extension_of_scalars_of_a_scheme_of_prime_characteristic_by_frobenius}{extension of scalars of $X/S$ by the $p^n$th power Frobenius}\CrefIfExists{definition:extension_of_scalars_of_a_scheme_of_prime_characteristic_by_frobenius}
        \item $S_{F,p^n}$ denotes the \hyperrefIfExists{definition:restriction_of_scalars_by_Frobenius_for_a_scheme}{restriction of scalars of $S$} (as an $S$-scheme)\CrefIfExists{definition:restriction_of_scalars_by_Frobenius_for_a_scheme}, and 
        \item $\varphi$ is regarded as a morphism $X \to S_{F,p^n}$ by regarding $S_{F,p^n}$ as a copy of $S$. 
    \end{itemize}
    The relative Frobenius morphism above may commonly be denoted by notations such as \hl{$F_{X/S, p^n}$}, \hl{$F_{X|S, p^n}$}, \hl{$Frob_{X/S, p^n}$}, and \hl{$Frob_{X|S, p^n}$}. For $n < 0$, note that this definition is only well defined under the assumption that $F_{S,p^{|n|}}$ is invertible. 


    In the case that $n = 1$, this is referred to as the \hldef{relative Frobenius morphism of $X/S$} and may commonly be denoted by notations such as \hl{$F_{X/S}$}, \hl{$F_{X|S}$}, \hl{$Frob_{X/S}$}, and \hl{$Frob_{X|S}$}.
\end{definition}

\begin{definition} \label{definition:algebraic_group_scheme_over_a_scheme}
Let $S$ be a scheme. An \hl{algebraic group scheme over $S$} (or an \hl{$S$-group scheme}) is a group object $G$ in the category of schemes over $S$; that is, $G$ is an $S$-scheme equipped with $S$-morphisms:
\hl{$m: G \times_S G \to G$} (\hldef{multiplication}), \hl{$i: G \to G$} (\hldef{inverse}), and \hl{$e: S \to G$} (\hldef{identity}),
satisfying the group axioms expressed by the commutativity of the following diagrams:

\begin{enumerate}
    \item \textbf{Associativity}\quad
    $
    \begin{tikzcd}[column sep=small]
      G \times_S G \times_S G \arrow{r}{m \times \mathrm{id}} \arrow{d}[swap]{\mathrm{id} \times m} & G \times_S G \arrow{d}{m} \\
      G \times_S G \arrow{r}{m} & G
    \end{tikzcd}
    $
    \item \textbf{Identity}\quad
    $
    \begin{tikzcd}[column sep=small]
      G \times_S S \arrow{r}{\mathrm{id} \times e} \arrow{dr}[swap]{\simeq} & G \times_S G \arrow{d}{m} \\
      & G
    \end{tikzcd}
    \qquad
    \begin{tikzcd}[column sep=small]
      S \times_S G \arrow{r}{e \times \mathrm{id}} \arrow{dr}[swap]{\simeq} & G \times_S G \arrow{d}{m} \\
      & G
    \end{tikzcd}
    $
    \item \textbf{Inverse}\quad
    $
    \begin{tikzcd}[column sep=small]
      G \arrow{r}{(\mathrm{id}, i)} \arrow{d}[swap]{\mathrm{id}} & G \times_S G \arrow{d}{m} \\
      G \arrow{r}{e \circ \pi} & G
    \end{tikzcd}
    $
    where $\pi: G \to S$ is the structure morphism and $e \circ \pi$ sends $g$ to the identity section.
\end{enumerate}

\TextIfExists{definition:group_object_in_a_category_with_a_final_object}{Equivalently, a group scheme over $S$ is a \CrefAndHyperrefIfExist{definition:group_object_in_a_category_with_a_final_object}{group object} in the \CrefAndHyperrefIfExist{definition:scheme_over_a_scheme}{category of $S$-schemes}}

If $G$ is \CrefAndHyperrefIfExist{definition:affine_morphism_of_schemes}{affine over} $S$, we call it an \hldef{affine group scheme over $S$}.

If the base scheme $S$ is the spectrum of a field $k$, then we call $G$ a \hldef{$k$-algebraic group} or an \hldef{algebraic group (scheme) over $k$}. If $G$ is additionally a $k$-variety, then we call $G$ a \hldef{$k$-group variety}.
\end{definition}

% \begin{definition} \label{definition:algebraic_group_over_a_field}
%     \TODO{TODO: define variety, group object}
% Let $k$ be a field. An \hl{$k$-algebraic group} (or an \hl{algebraic group over $k$}) is a group object $G$ in the category of $k$-schemes; that is, $G$ is a scheme over $k$ equipped with morphisms $m: G \times G \to G$ (multiplication), $i: G \to G$ (inverse), and $e: \operatorname{Spec} k \to G$ (identity), satisfying the group axioms expressed by the commutativity of the following diagrams:

% \begin{itemize}
%     \item[(Associativity)]
%     \begin{center}
%     \begin{tikzcd}
%     G \times G \times G \arrow{r}{m \times \mathrm{id}} \arrow{d}[swap]{\mathrm{id} \times m} & G \times G \arrow{d}{m} \\
%     G \times G \arrow{r}{m} & G
%     \end{tikzcd}
%     \end{center}

%     \item[(Identity)] 
%     \begin{center}
%     \begin{tikzcd}
%     G \times \operatorname{Spec} k \arrow{r}{\mathrm{id} \times e} \arrow{dr}[swap]{\simeq} & G \times G \arrow{d}{m} \\
%     & G
%     \end{tikzcd}
%     \qquad
%     \begin{tikzcd}
%     \operatorname{Spec} k \times G \arrow{r}{e \times \mathrm{id}} \arrow{dr}[swap]{\simeq} & G \times G \arrow{d}{m} \\
%     & G
%     \end{tikzcd}
%     \end{center}

%     \item[(Inverse)] 
%     \begin{center}
%         \begin{tikzcd}
%         G \arrow{r}{(\mathrm{id}, i)} \arrow{d}[swap]{\mathrm{id}} & G \times G \arrow{d}{m} \\
%         G \arrow{r}{e \circ \pi} & G
%         \end{tikzcd}
%     \end{center}
%     where $\pi: G \to \operatorname{Spec} k$ is the structure morphism and $e \circ \pi$ sends $g$ to the identity.
% \end{itemize}
% If $G$ is a $k$-variety, when we call it a \hldef{$k$-group variety}. If $G$ is an affine $k$-scheme, then we call it an \hldef{affine algebraic $k$-group}.
% \end{definition}

\begin{definition} \label{definition:geometric_frobenius_action_of_a_derived_object_on_a_scheme_of_characteristic_p_at_a_stalk}
    Let $X$ be a scheme of characteristic $p$. Let $x \in |X|$ be a closed point whose residue field is a finite field. Let $\barx$ be a geometric point over $X$. Let $K \in D(X)$ be an object in the derived category of sheaves of abelian groups on $X$. 
 %Let $n \geq 1$ be an integer.
    
    The \hyperrefIfExists{definition:arithmetic_and_geometric_frobenius_automorphisms_of_algebraic_extensions_of_finite_fields}{geometric Frobenius automorphism} in $\Gal(\overline{\kappa(x)} / \kappa(x))$ can be identified as an automorphism 
    $$\Spec \overline{\kappa(x)} \to \Spec \overline{\kappa(x)}$$
    over $\Spec \kappa(x)$ and hence induces an action 
    $$\hlin{\Frob_{\barx}: K_{\barx} \to K_{\barx}}$$
    on the stalk. Up to isomorphism this action only depends on the closed point $x$ and not the choice of the geometric point $\barx$ above $x$.

    This automorphism may be called the \hldef{geometric Frobenius action on $K$ at the stalk at $\barx$}. In case that the size of the residue field $\kappa(x)$ is $q$ (or is considered as $q$, say by virtue of changing the base field over which $X$ is defined) we may also denote the geometric Frobenius action using notations such as \hl{$\Frob_{\barx, q}$} or \hl{$\Frob_{q}$} to emphasize the size of $\kappa(x)$.
\end{definition}


