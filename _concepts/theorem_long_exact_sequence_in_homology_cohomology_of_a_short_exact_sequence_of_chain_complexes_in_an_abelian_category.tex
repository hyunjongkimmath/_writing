\begin{theorem}[Long Exact Sequence in Homology] \label{theorem:long_exact_sequence_in_homology_cohomology_of_a_short_exact_sequence_of_chain_complexes_in_an_abelian_category}

\begin{enumerate}
    \item 
    Let $(C_\bullet, d^C_\bullet)$, $(D_\bullet, d^D_\bullet)$, and $(E_\bullet, d^E_\bullet)$ be \CrefAndHyperrefIfExist{definition:chain_complex_of_objects_in_an_additive_category}{chain complexes} in an \CrefAndHyperrefIfExist{definition:abelian_category}{abelian category}. Recall that \CrefAndHyperrefIfExist{definition:chain_complex_of_objects_in_an_additive_category}{$\mathbf{Ch}(\calA)$} is itself an abelian category (\Cref{proposition:category_of_chain_complexes_in_an_additive_category_is_additive}).
    Assume that 
    \[ 0 \longrightarrow C_\bullet \xrightarrow{\alpha_\bullet} D_\bullet \xrightarrow{\beta_\bullet} E_\bullet \longrightarrow 0 \]
    is a \CrefAndHyperrefIfExist{definition:short_exact_sequence_in_an_additive_category}{short exact sequence} of chain complexes. Equivalently, for each integer $n$, 
    \[ 0 \to C_n \xrightarrow{\alpha_n} D_n \xrightarrow{\beta_n} E_n \to 0 \]
    is an exact sequence of $R$-modules.

    Then there exists a natural \hldef{long exact sequence} in \CrefAndHyperrefIfExist{definition:homology_and_cohomology_objects_for_a_chain_complex_in_an_additive_category}{homology}:
    \[
    \cdots \longrightarrow H_{n+1}(E_\bullet) \xrightarrow{\delta_{n+1}} H_n(C_\bullet)
    \xrightarrow{H_n(\alpha)} H_n(D_\bullet)
    \xrightarrow{H_n(\beta)} H_n(E_\bullet)
    \xrightarrow{\delta_n} H_{n-1}(C_\bullet)
    \longrightarrow \cdots
    \]
    The homomorphisms \hl{$\delta_n: H_n(E_\bullet) \to H_{n-1}(C_\bullet)$} are called the \hldef{connecting homomorphisms} induced by the short exact sequence of chain complexes.

    Moreover, this long exact sequence is natural with respect to \CrefAndHyperrefIfExist{definition:chain_complex_of_objects_in_an_additive_category}{morphisms} of short exact sequences of chain complexes.

    \item Let $(C^\bullet, d_C^\bullet)$, $(D^\bullet, d_D^\bullet)$, and $(E^\bullet, d_E^\bullet)$ be 
    \CrefAndHyperrefIfExist{definition:chain_complex_of_objects_in_an_additive_category}{cochain complexes} 
    in an \CrefAndHyperrefIfExist{definition:abelian_category}{abelian category}. 
    Recall that \CrefAndHyperrefIfExist{definition:chain_complex_of_objects_in_an_additive_category}{$\mathbf{Ch}(\calA)$} 
    is itself an abelian category 
    (\Cref{proposition:category_of_chain_complexes_in_an_additive_category_is_additive}).

    Assume that 
    \[
    0 \longrightarrow C^\bullet \xrightarrow{\alpha^\bullet} D^\bullet \xrightarrow{\beta^\bullet} E^\bullet \longrightarrow 0
    \]
    is a \CrefAndHyperrefIfExist{definition:short_exact_sequence_in_an_additive_category}{short exact sequence} 
    of cochain complexes. Equivalently, for each integer $n$, 
    \[
    0 \to C^n \xrightarrow{\alpha^n} D^n \xrightarrow{\beta^n} E^n \to 0
    \]
    is an exact sequence of $R$-modules.

    Then there exists a natural \hldef{long exact sequence} in \CrefAndHyperrefIfExist{definition:homology_and_cohomology_objects_for_a_chain_complex_in_an_additive_category}{cohomology}:
    \[
    \cdots \longrightarrow H^{n-1}(E^\bullet) 
    \xrightarrow{\delta^{n-1}} H^{n}(C^\bullet)
    \xrightarrow{H^{n}(\alpha)} H^{n}(D^\bullet)
    \xrightarrow{H^{n}(\beta)} H^{n}(E^\bullet)
    \xrightarrow{\delta^{n}} H^{n+1}(C^\bullet)
    \longrightarrow \cdots
    \]
    The morphisms \hl{$\delta^{n}: H^{n}(E^\bullet) \to H^{n+1}(C^\bullet)$} are called the 
    \hldef{connecting homomorphisms} induced by the short exact sequence of cochain complexes.

    Moreover, this long exact sequence is natural with respect to 
    \CrefAndHyperrefIfExist{definition:morphism_of_cochain_complexes_of_objects_in_an_additive_category}{morphisms} 
    of short exact sequences of cochain complexes.

\end{enumerate}
\end{theorem}

\begin{proof}
    \TODO{why kernel and cokernel are left/right exact}
Consider the following commutative diagram in $\mathcal{A}$ with exact rows, where the vertical maps are the differentials of the respective complexes:
\begin{equation} \label{eq:snake_setup}
\begin{tikzcd}
0 \ar[r] & C_n / \operatorname{im}(d_{n+1}^C) \ar[r, "\bar{\alpha}_n"] \ar[d, "\bar{d}_n^C"] & D_n / \operatorname{im}(d_{n+1}^D) \ar[r, "\bar{\beta}_n"] \ar[d, "\bar{d}_n^D"] & E_n / \operatorname{im}(d_{n+1}^E) \ar[r] \ar[d, "\bar{d}_n^E"] & 0 \\
0 \ar[r] & \ker(d_{n-1}^C) \ar[r, "\alpha_{n-1}"] & \ker(d_{n-1}^D) \ar[r, "\beta_{n-1}"] & \ker(d_{n-1}^E) \ar[r] & 0
\end{tikzcd}
\end{equation}

The rows are exact because the original sequence of complexes is exact and the functors $\ker(d_{n-1})$ and $\coker(d_{n+1})$ are respectively left and right exact. Specifically, the bottom row is exact at the right because $\beta_n$ is surjective and the complexes satisfy the cycle condition. \TODO{why is the bottom row exact at the right and why is the top row exact at the left}

We now apply the \CrefAndHyperrefIfExist{proposition:snake_lemma}{Snake Lemma} to this diagram. To identify the resulting terms, we calculate the kernel and cokernel of the vertical map $\bar{d}_n^C$ (and similarly for $D$ and $E$):
\begin{itemize}
    \item \textbf{Kernel:} The kernel of $\bar{d}_n^C: C_n / \operatorname{im}(d_{n+1}^C) \to \ker(d_{n-1}^C)$ consists of elements\footnote{Since elements are invoked in this argument, the \CrefAndHyperref{theorem:freyd_mitchell_embedding_theorem_for_small_abelian_categories}{Freyd-Mitchell embedding theorem} should technically be used for this argument; nevertheless, one can argue that $\ker(\bar{d}_n^C) = H_n(C_\bullet)$ purely categorically.} $x + \operatorname{im}(d_{n+1}^C)$ such that $d_n^C(x) = 0$, i.e. $x \in \ker(d_n^C)$. Therefore,
    \[ \ker(\bar{d}_n^C) = \ker(d_n^C) / \operatorname{im}(d_{n+1}^C) = H_n(C_\bullet). \]
    
    \item \textbf{Cokernel:} The cokernel of $\bar{d}_n^C$ is the quotient of the target by the image. The target is $\ker(d_{n-1}^C)$ and the image is $\operatorname{im}(d_n^C)$. Thus,
    \[ \coker(\bar{d}_n^C) = \ker(d_{n-1}^C) / \operatorname{im}(d_n^C) = H_{n-1}(C_\bullet). \]
\end{itemize}

Applying the Snake Lemma to \eqref{eq:snake_setup} yields the exact sequence:
\[
\ker(\bar{d}_n^C) \to \ker(\bar{d}_n^D) \to \ker(\bar{d}_n^E) \xrightarrow{\delta_n} \coker(\bar{d}_n^C) \to \coker(\bar{d}_n^D) \to \coker(\bar{d}_n^E)
\]
Substituting the homology groups identified above, we obtain:
\[
H_n(C_\bullet) \xrightarrow{H_n(\alpha)} H_n(D_\bullet) \xrightarrow{H_n(\beta)} H_n(E_\bullet) \xrightarrow{\delta_n} H_{n-1}(C_\bullet) \xrightarrow{H_{n-1}(\alpha)} H_{n-1}(D_\bullet) \xrightarrow{H_{n-1}(\beta)} H_{n-1}(E_\bullet)
\]
Since this construction exists for every $n \in \bbZ$, we can splice these sequences together to form the bi-infinite long exact sequence. The naturality of the sequence follows from the naturality of the Snake Lemma and the functoriality of the kernel and cokernel constructions.
\end{proof}