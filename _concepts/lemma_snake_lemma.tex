\begin{lemma}[Snake lemma] \label{lemma:snake_lemma}
    Let $\calA$ be an \CrefAndHyperrefIfExist{definition:abelian_category}{abelian category}. Let
    Given the commutative diagram with exact rows:
    $$ \begin{array}{ccccccccc} & & A & \xrightarrow{f} & B & \xrightarrow{g} & C & \xrightarrow{} & 0 \\ & & \downarrow{a} & & \downarrow{b} & & \downarrow{c} & & \\ 0 & \xrightarrow{} & A' & \xrightarrow{f'} & B' & \xrightarrow{g'} & C' & & \\ \end{array} $$
    there exists an exact sequence connecting the kernels and cokernels:
    $$ \ker(a) \xrightarrow{\bar{f}} \ker(b) \xrightarrow{\bar{g}} \ker(c) \xrightarrow{d} \text{coker}(a) \xrightarrow{\bar{f}'} \text{coker}(b) \xrightarrow{\bar{g}'} \text{coker}(c) $$
    where $d$ is the connecting homomorphism. Furthermore, if $f$ is \CrefAndHyperrefIfExist{definition:monomorphism_and_epimorphism_in_categories}{monic}, then $\bar{f}$ is monic; if $g'$ is \CrefAndHyperrefIfExist{definition:monomorphism_and_epimorphism_in_categories}{epi}, then $\bar{g}'$ is epi.
\end{lemma}

\begin{proof}
    We first show this in the case that $\calA$ is the \CrefAndHyperrefIfExist{definition:category_of_modules_and_bimodules_over_rings}{category of $R$-modules} for any (not necessarily commutative) ring $R$. We perform a diagram chase on the elements of the modules.
    \begin{enumerate}
        \item \textbf{Construction of $d$}: Let $z \in \ker(c) \subseteq C$. Since $g$ is surjective, there exists $y \in B$ such that $g(y) = z$. By commutativity, $g'(b(y)) = c(g(y)) = c(z) = 0$. Thus $b(y) \in \ker(g')$. By exactness of the bottom row, there exists a unique $x' \in A'$ such that $f'(x') = b(y)$. Define $d(z) = [x'] \in A'/\text{im}(a) = \text{coker}(a)$. 

        We show that $[x']$ is independent of the choice of $y$. Suppose we choose another lift $y_1 \in B$ such that $g(y_1) = z$. Then $g(y - y_1) = g(y) - g(y_1) = z - z = 0$. By the exactness of the top row at $B$, there exists some $x \in A$ such that $f(x) = y - y_1$. Applying $b$, we have $b(y - y_1) = b(f(x))$. By the commutativity of the first square, $b(f(x)) = f'(a(x))$. Let $x'$ and $x'_1$ be the unique elements in $A'$ such that $f'(x') = b(y)$ and $f'(x'_1) = b(y_1)$. Then $f'(x' - x'_1) = b(y) - b(y_1) = b(y - y_1) = f'(a(x))$. Since $f'$ is injective (due to the exactness of the bottom row at $A'$), we must have $x' - x'_1 = a(x)$. This implies $x' \equiv x'_1 \pmod{\text{im}(a)}$. Therefore, $[x'] = [x'_1]$ in $\text{coker}(a)$, proving that the definition of $d(z)$ is independent of the choice of the preimage $y$.

        %Write $y_1 = y$ and say that $y_2 \in B$ is an element such that $g(y_2) = z$ as well. Note that $g(y_1-y_2) = z - z = 0$, so $y_1-y_2 \in \ker g = \operatorname{im} f$. Let $x \in A$ such that $f(x) = y_1-y_2$. When then have $f'(a(x)) = b(f(x)) = b(y_1-y_2)

        \item \textbf{Exactness at $\ker(b)$}: Clearly $\bar{g} \circ \bar{f} = 0$. If $y \in \ker(\bar{g})$, then $g(y) = 0$, so $y = f(x)$ for some $x \in A$. Then $f'(a(x)) = b(f(x)) = b(y) = 0$. Since $f'$ is injective, $a(x) = 0$, so $x \in \ker(a)$ and $y \in \text{im}(\bar{f})$.
        \item \textbf{Exactness at $\ker(c)$}: If $z = \bar{g}(y)$ for $y \in \ker(b)$, then in the construction of $d(z)$, we can choose this $y$. Since $b(y)=0$, we have $x'=0$, so $d(z)=0$. Conversely, if $d(z)=0$, then the lift $x'$ is in $\text{im}(a)$, say $x'=a(x)$. Then $b(f(x)) = f'(a(x)) = f'(x') = b(y)$. Thus $y - f(x) \in \ker(b)$ and $g(y - f(x)) = g(y) = z$.
        \item \textbf{Exactness at $\text{coker}(a)$}: Similar diagram chasing confirms $\bar{f}' \circ d = 0$ and the corresponding inclusion.

        We verify the exactness of the sequence at $\text{coker}(a)$, specifically the sequence $\ker(c) \xrightarrow{d} \text{coker}(a) \xrightarrow{\bar{f}'} \text{coker}(b)$. Let $z \in \ker(c)$. By the construction of the connecting homomorphism, $d(z) = [x']$ where $f'(x') = b(y)$ for some $y \in B$ with $g(y) = z$. Applying $\bar{f}'$, we get $\bar{f}'([x']) = [f'(x')] = [b(y)]$. Since $b(y) \in \text{im}(b)$, its class in $\text{coker}(b)$ is zero. Thus $\bar{f}' \circ d = 0$. \textbf{Inclusion $\ker(\bar{f}') \subseteq \text{im}(d)$}: Let $[x'] \in \text{coker}(a)$ such that $\bar{f}'([x']) = [0]$ in $\text{coker}(b)$. This implies $f'(x') \in \text{im}(b)$, so there exists $y \in B$ such that $b(y) = f'(x')$. We apply $g'$ to both sides: $g'(b(y)) = g'(f'(x'))$. Since the bottom row is exact, $g' \circ f' = 0$, so $g'(b(y)) = 0$. By commutativity of the second square, $c(g(y)) = g'(b(y)) = 0$. This means $g(y) \in \ker(c)$. Let $z = g(y)$. By the definition of the connecting homomorphism, $d(z)$ is found by lifting $z$ to $B$ (we choose $y$), applying $b$ (we get $b(y) = f'(x')$), and taking the preimage under $f'$ (which is $x'$). Thus, $d(z) = [x']$, which shows that $[x'] \in \text{im}(d)$.

        \item We verify the exactness of the sequence at $\text{coker}(b)$, specifically looking at the sequence $\text{coker}(a) \xrightarrow{\bar{f}'} \text{coker}(b) \xrightarrow{\bar{g}'} \text{coker}(c)$. Let $[x'] \in \text{coker}(a)$ where $x' \in A'$. Then $\bar{f}'([x']) = [f'(x')]$. Applying $\bar{g}'$, we have $\bar{g}'([f'(x')]) = [g'(f'(x'))]$. Since the bottom row is exact, $g' \circ f' = 0$, thus $[g'(f'(x'))] = [0]$, so $\bar{g}' \circ \bar{f}' = 0$. Let $[y'] \in \text{coker}(b)$ such that $\bar{g}'([y']) = [0]$ in $\text{coker}(c)$. This implies $g'(y') \in \text{im}(c)$, so there exists $z \in C$ such that $c(z) = g'(y')$. Since $g$ is surjective, there exists $y \in B$ such that $g(y) = z$. By commutativity, $g'(b(y)) = c(g(y)) = c(z) = g'(y')$. Thus $g'(y' - b(y)) = 0$, which means $y' - b(y) \in \ker(g')$. By exactness of the bottom row, there exists $x' \in A'$ such that $f'(x') = y' - b(y)$. Transitioning to the cokernel, $[y'] = [f'(x') + b(y)] = [f'(x')] + [b(y)]$. Since $b(y) \in \text{im}(b)$, its class $[b(y)] = 0$ in $\text{coker}(b)$. Therefore, $[y'] = [f'(x')] = \bar{f}'([x'])$, showing $[y'] \in \text{im}(\bar{f}')$.
    \end{enumerate}

    Let $\calB$ be the smallest \CrefAndHyperref{definition:full_subcategory_of_a_category}{full} abelian subcategory of $\calA$ containing the objects $A,B,C,A',B',C'$; it is a small category. Let choose a ring $R$ and a functor $F: \calB \to \Mod_R$ that is \CrefAndHyperrefIfExist{definition:exact_functor_between_abelian_categories}{exact} and \CrefAndHyperrefIfExist{definition:full_and_faithful_functor_between_locally_small_categories}{fully faithful} by the \CrefAndHyperrefIfExist{theorem:freyd_mitchell_embedding_theorem_for_small_abelian_categories}{Freyd-Mitchell Embedding Theorem}. Apply the functor $F$ to the diagram. Since $F$ is exact, it preserves kernels, cokernels, images, and the exactness of the rows. The result holds in $\Mod_R$ by the element-based proof above. Since $F$ is a full and faithful embedding, the morphism $d: \ker(c) \to \coker(a)$ is present in $\calB$. Moreover, by \Cref{proposition:full_and_faithful_additive_functor_between_abelian_categories_reflects_exactness}, $F$ \CrefAndHyperrefIfExist{definition:reflects_a_type_of_morphism_for_a_functor_between_categories}{reflects} exactness, so the exact sequence in $\Mod_R$ remains exact in $\calB$. In particular, the exact sequence exists and is exact in $\calA$.
\end{proof}
