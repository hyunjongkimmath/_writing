\begin{theorem}[Standard Examples of Coherent Sites] \label{theorem:examples_of_coherent_sites}
The following are standard examples of coherent sites. In each case, the topos of sheaves on the site is a \CrefAndHyperrefIfExist{definition:coherent_sheaf_topos}{coherent topos}.
\begin{enumerate}
    \item \textbf{The Zariski Site of a Coherent Scheme}:
    Let $X$ be a \hldef{coherent scheme} \TODO{define coherent scheme} (e.g., any \hldef{Noetherian scheme}). Let $\mathcal{C}$ be the category of open subsets of $X$ that are \hldef{quasi-compact} (i.e., finite unions of affine opens). Let $J$ be the standard finite open cover topology.
    
    Then $(\mathcal{C}, J)$ is a coherent site. The sheaves on this site are equivalent to the category of sheaves on the full Zariski site of $X$.
    
    \item \textbf{The Étale Site of a Coherent Scheme}:
    Let $X$ be a coherent scheme. Let $\mathcal{C}$ be the category of schemes \hldef{étale} over $X$ which are themselves coherent (quasi-compact and quasi-separated over $X$) \TODO{define étale morphism}. Let $J$ be the topology generated by finite surjective families of étale maps.
    
    Then $(\mathcal{C}, J)$ is a coherent site.

    \item \textbf{The Nisnevich Site of a Coherent Scheme}:
    Let $X$ be a coherent scheme. The Nisnevich site over $X$ is a coherent site.
    
    \item \textbf{The Site of Finite Sets}:
    Let $\mathcal{C} = \mathbf{FinSet}$ be the category of finite sets. Let $J$ be the \hldef{canonical topology} (where covering families are jointly surjective finite families).
    
    This is a coherent site. The corresponding topos is $\mathbf{Set}$ itself.
    
    \item \textbf{The Syntactic Site of a Coherent Theory}:
    Let $\mathbb{T}$ be a \hldef{coherent theory} in first-order logic \TODO{define coherent theory}. Let $\mathcal{C}_{\mathbb{T}}$ be the \hldef{syntactic category} of $\mathbb{T}$, whose objects are formulas modulo equivalence and morphisms are provable functional relations.
    
    Equipped with the topology where covers correspond to finite disjunctions ($\phi \leftrightarrow \bigvee_{i=1}^n \psi_i$), this forms a coherent site. The sheaf topos is the \hldef{classifying topos} of $\mathbb{T}$.
\end{enumerate}
\end{theorem}
