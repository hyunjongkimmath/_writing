\begin{theorem} \label{theorem:functor_category_of_an_abelian_category_indexed_by_a_small_category_is_abelian}
    Let $\mathcal{A}$ be an \CrefAndHyperrefIfExist{definition:abelian_category}{abelian category} and $\mathcal{J}$ be a \CrefAndHyperrefIfExist{definition:locally_small_category}{small category}. The \CrefAndHyperrefIfExist{definition:diagram_in_a_category_indexed_by_a_small_category}{functor category} $\mathcal{A}^{\mathcal{J}}$ inherits the structure of an abelian category from $\mathcal{A}$. Specifically:
    \begin{itemize}
        \item The \CrefAndHyperrefIfExist{definition:initial_final_zero_objects_of_a_category}{zero object} in $\mathcal{A}^{\mathcal{J}}$ is the constant functor at the zero object of $\mathcal{A}$.
        \item \CrefAndHyperrefIfExist{definition:kernel_and_cokernel_of_a_morphism_in_a_category}{Kernels}, \CrefAndHyperrefIfExist{definition:kernel_and_cokernel_of_a_morphism_in_a_category}{cokernels}, \CrefAndHyperrefIfExist{definition:product_and_coproduct_of_objects_in_a_category}{products}, and \CrefAndHyperrefIfExist{definition:product_and_coproduct_of_objects_in_a_category}{coproducts} are computed pointwise in $\mathcal{A}$. For example, if $\eta: F \to G$ is a \CrefAndHyperrefIfExist{definition:natural_transformation_between_functors_between_categories}{natural transformation}, the kernel is the functor $K: \mathcal{J} \to \mathcal{A}$ defined by $K(j) = \ker(\eta_j)$ for each $j \in \mathcal{J}$.
        \item A sequence of functors $0 \to F \to G \to H \to 0$ is \CrefAndHyperrefIfExist{definition:short_exact_sequence_in_an_additive_category}{exact} in $\mathcal{A}^{\mathcal{J}}$ if and only if for every object $j \in \mathcal{J}$, the sequence $0 \to F(j) \to G(j) \to H(j) \to 0$ is exact in $\mathcal{A}$.
    \end{itemize}
    Moreover, if $\mathcal{A}$ admits arbitrary limits (resp. colimits), then $\mathcal{A}^{\mathcal{J}}$ also admits arbitrary limits (resp. colimits), computed pointwise.
\end{theorem}
