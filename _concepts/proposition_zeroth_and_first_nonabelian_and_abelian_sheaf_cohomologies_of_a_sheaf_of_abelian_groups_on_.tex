\begin{proposition}{See {\cite[Tag 03AJ]{stacks-project}} for a statement\footnote{While the Stacks Project defines a site as a category eqiupped with a Grothendieck \emph{pre}topology, its proof of this statement should be applicable for more general sites}} \label{proposition:zeroth_and_first_nonabelian_and_abelian_sheaf_cohomologies_of_a_sheaf_of_abelian_groups_on_}
    Let $(\calC, J)$ be a \CrefAndHyperrefIfExist{definition:grothendieck_topology_on_a_category_site_covering_sieve_topologically_generating_family}{site}. 
    Assume that \CrefAndHyperrefIfExist{definition:sections_of_a_presheaf_on_a_category_valued_in_a_category}{global sections objects $\Gamma(\calG)$} exist for all objects $\calG$ of $\mathrm{Sh}(\mathcal{C}, \mathbf{Ab})$\footnote{for example, this occurs when $\calC$ is \CrefAndHyperrefIfExist{definition:essentially_small_category}{essentially small}} so that $\Gamma$ is a functor
    $$\Sh(\calC, \mathbf{Ab}) \to \mathbf{Ab}.$$

    Let $\calG$ be a \CrefAndHyperrefIfExist{definition:sheaf_on_a_site}{sheaf} of abelian groups on $(\calC, J)$. For $i = 0,1$, there is a canonical bijection between the set of isomorphism classes of \CrefAndHyperrefIfExist{definition:torsor_principal_homogeneous_space_of_a_sheaf_of_groups_on_a_site_over_an_object}{$\calG$-torsors} and the \CrefAndHyperrefIfExist{definition:sheaf_cohomology_group_of_a_sheaf_of_modules_over_a_sheaf_of_rings_on_a_site}{(abelian) sheaf cohomology group $H^i(\calC, J; \calG)$}. In other words, for $i = 0,1$, there is a canonical bijection between the \CrefAndHyperrefIfExist{definition:zeroth_and_first_nonabelian_sheaf_cohomology_of_a_sheaf_of_groups_on_a_site}{$i$th nonabelian sheaf cohomology} and the \CrefAndHyperrefIfExist{definition:sheaf_cohomology_group_of_a_sheaf_of_modules_over_a_sheaf_of_rings_on_a_site}{$i$th abelian sheaf cohomology} of $\calG$. 
\end{proposition}

\begin{proof}
    In the case of $i = 0$, both are identifiable with the groups of global sections. In the case of $i = 1$, see {\cite[Tag 03AJ]{stacks-project}}. 
\end{proof}
