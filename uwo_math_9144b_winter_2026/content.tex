%% Delete this \nocite command invocation to make the references section only list out the bibitems that are actually cited.
\nocite{*}

These are ``notes'' written for the course MATH 9144B Homological Algebra at the University of Western Ontario during the Winter 2026 term. The main text for the course is Weibel's \emph{An Introduction to Homological Algebra} \cite{weibel}, and I intend for the course to roughly follow parts of this text. This document aims to supplement the presentation of loc.~cit.~, which focuses much of its attention on left/right $R$-modules where $R$ is an ``associative ring''\footnote{The introduction of \cite{weibel} assumes that the reader has a background in graduate algebra "based on a text such as \emph{Jacobson's Basic Algebraic I}". Jacobson defines a ring to be both associative and unital (but not necessarily commutative), so ostensibly \cite{weibel} adopts this same convention.}, by discussing general definitions. 

Readers are recommended to read this document using a PDF reader (e.g. Adobe Acrobat/Reader) that supports Link Annotations (i.e. the boxes, usually red/green, that are rendered when using \texttt{\\ref} and \texttt{\\cite} commands).

My ``notes'' for \href{https://github.com/hyunjongkimmath/_writing/tree/main/HomologicalAlgebra}{Homological Algebra}, \href{https://github.com/hyunjongkimmath/_writing/tree/main/CategoryTheory}{Category Theory}, \href{https://github.com/hyunjongkimmath/_writing/tree/main/AbstractAlgebra}{Abstract Algebra}, etc. may be of interest.

\subsection{Disclaimer}

The definitions and statements in this document are generally written less as basic introductions to any given concept and more so to present them in generality. As such, they are not necessarily linearly presented. Readers should judiciously decide what to read and what to skip.

Many statements are based on AI generated ones, and errors may be abound due to my own lack of care when verifying them. Readers are advised to exercise caution when citing claims, which may be erroneous, in these notes.

The contents of these notes may change constantly.

\section{Basic category theory}

\subsection{Categories}

\begin{definition}[Category] \label{definition:category}
    A 
    \defin{category}{category}{
        name={Category},
        description={A nice enough collection of objects and morphisms (\Cref{definition:category})},
    }
    \hldef{category} $\mathcal{C}$ consists of the following data:
    \begin{itemize}
        \item A class of \defin{objects}{object_of_a_category}{
            name={Object of a category},
            description={\Cref{definition:category}},
        }
        denoted \notat{\operatorname{Ob}(\mathcal{C})}{class_of_objects_of_a_category}{
            name={$\operatorname{Ob}(\mathcal{C})$},
            description={Class of objects of a category $\calC$ \Cref{definition:category}},
            sort={Ob},
        }.
        % \hl{$\operatorname{Ob}(\mathcal{C})$}.
        \item For each pair of objects $X, Y \in \operatorname{Ob}(\mathcal{C})$, a class
        \notatin{\operatorname{Hom}_{\mathcal{C}}(X,Y)}{class_of_morphisms_between_two_objects_of_a_category}
        {
            name={$\operatorname{Hom}_{\mathcal{C}}(X,Y)$},
            description={Class of morphisms between objects $X$ and $Y$ of the category $\calC$ (\Cref{definition:category})},
            sort={Hom},
        }
        % $$\hlin{\operatorname{Hom}_{\mathcal{C}}(X,Y)}$$
        of \defin{morphisms}{morphism_between_objects_of_a_category}{
            name={Morphism between objects of a category},
            description={(\Cref{definition:category})},
        }
        (also called 
        \defin{arrows}{arrow_between_objects_of_a_category}{
            name={Arrow between objects of a category},
            description={Synonym for morphism (\Cref{definition:category})},
        }
        or
        \defin{homs}{hom_between_objects_of_a_category}{
            name={Hom between objects of a category},
            description={Synonym for morphism (\Cref{definition:category})},
        }). If the category $\calC$ is clear, then this \hldef{hom-class} is also denoted by \hl{$\operatorname{Hom}(X,Y)$}. It may also be denoted by \hl{$\operatorname{hom}_{\mathcal{C}}(X,Y)$} or \hl{$\operatorname{hom}(X,Y)$}, especially to distinguish from other types of hom's (e.g. \hyperrefIfExists{definition:internal_hom_object_in_a_category}{internal hom's})
        \item For each triple of objects $X,Y,Z$, a composition law
        $$ \circ : \operatorname{Hom}_{\mathcal{C}}(Y,Z) \times \operatorname{Hom}_{\mathcal{C}}(X,Y) \to \operatorname{Hom}_{\mathcal{C}}(X,Z), $$
        denoted \hl{$(g,f) \mapsto g \circ f$}.
        \item For each object $X$, an \hldef{identity morphism}
        $$\hlin{\operatorname{id}_X \in \operatorname{Hom}_{\mathcal{C}}(X,X).}$$
    \end{itemize}
    These data satisfy the following axioms:
    \begin{itemize}
        \item (Associativity) For all morphisms $f \in \operatorname{Hom}_{\mathcal{C}}(X,Y)$, $g \in \operatorname{Hom}_{\mathcal{C}}(Y,Z)$, and $h \in \operatorname{Hom}_{\mathcal{C}}(Z,W)$, 
        $$
        h \circ (g \circ f) = (h \circ g) \circ f.
        $$
        \item (Identity) For all $f \in \operatorname{Hom}_{\mathcal{C}}(X,Y)$,
        $$
        \operatorname{id}_Y \circ f = f = f \circ \operatorname{id}_X.
        $$
    \end{itemize}
    One often writes \hl{$X \in \calC$} synonymously with $X \in \Ob(\calC)$, i.e. to denote that $X$ is an object of of $\calC$. 

    We may call a category as above an \hldef{ordinary category} to distinguish this notion from the notions of \hyperrefIfExists{definition:category_enriched_in_a_monoidal_category}{\emph{categories enriched in monoidal categories}} or higher/$n$-categories.
    \TODO{TODO: define $n$-categories}

    A category as defined above may be called called a \hldef{large category} or a \hldef{class category} to emphasize that the hom-classes may be proper classes rather than sets (note, however, that the possibility that hom-classes are sets is not excluded for large categories). Accordingly, a \hldef{category} may often refer to a \hyperrefIfExists{definition:locally_small_category}{locally small category}\CrefIfExists{definition:locally_small_category}, which is a category whose hom-classes are all sets.
\end{definition}

% Later on, we refer to the \gls{category} again.

\begin{definition}[Locally small category] \label{definition:locally_small_category}
A \hyperrefIfExists{definition:category}{(large) category}\CrefIfExists{definition:category} $\mathcal{C}$ is called a \hldef{locally small category} if for every pair of objects $X, Y \in \operatorname{Ob}(\mathcal{C})$, the collection $\operatorname{Hom}_{\mathcal{C}}(X,Y)$ of morphisms between them is a (\CrefAndHyperrefIfExist{definition:small_set}{small}) \emph{set} (as opposed to a proper class). In other words, each hom-class is a set and may even be called a \hldef{hom-set}.

In some contexts, a locally small category may simply be called a \hldef{category}, especially when genuinely large categories are not considered.

A category $\mathcal{C}$ is called a \hldef{small category} if it is a locally small category and the class $\operatorname{Ob}(\mathcal{C})$ of objects is a set.

\TextIfExists{definition:grothendieck_universe}{
Given a \hyperrefIfExists{definition:grothendieck_universe}{universe}\CrefIfExists{definition:grothendieck_universe} $U$, we can define the notion of a \hldef{$U$-locally small category} and of a \hldef{$U$-small category} similarly. More explicitly, 
\begin{enumerate}
    \item a $U$-locally small category is a category such that for every pair of objects $X, Y \in \operatorname{Ob}(\mathcal{C})$, the collection $\operatorname{Hom}_{\mathcal{C}}(X,Y)$ of morphisms between them is a $U$-set.
    \item a $U$-small category is a category such that $\operatorname{Ob}(\mathcal{C})$ is a $U$-set and for every pair of objects $X, Y \in \operatorname{Ob}(\mathcal{C})$, the collection $\operatorname{Hom}_{\mathcal{C}}(X,Y)$ of morphisms between them is a $U$-set; in particular the collection of all objects and morhpisms in a $U$-small category is a $U$-set.
\end{enumerate}
}
\end{definition}

\begin{remark}
    Many ``concrete'' categories considered in ``classical mathematics'' or outside of more ``abstract'' category theory tend to be locally small. For example, the categories of sets, groups, $R$-modules, vector spaces, topological spaces, schemes, manifolds, sheaves on ``small enough'' sites are all locally small.
\end{remark}

\begin{example}
    Here is an example of a ``boring'' category:
    \begin{enumerate}
        \item There is only one object, say $X$.
        \item There is only one morphism, the identity $\id_X: X \to X$.
    \end{enumerate}
    The composed morphism $\id_X \circ \id_X$ is then just $\id_X$, and associativity automatically holds.
\end{example}

\begin{example}
    Any \CrefAndHyperrefIfExist{definition:partially_ordered_set}{poset} $(P,\leq)$ induces a \CrefAndHyperrefIfExist{definition:category}{category} --- let the objects be the elements of the set $P$, and let the morphisms/arrows be given as follows: there is a unique arrow $a \to b$ whenever $a \leq b$. Composition of arrows works as follows: given arrows $a \to b$ and $b \to c$, the composed arrow $a \to b \to c$ will be the unique arrow $a \to c$ corresponding to $a \leq c$. See \Cref{lemma:posets_correspond_to_small_filtered_thin_categories}
\end{example}


\begin{example}
    Here are common examples of categories:
    \begin{enumerate}
        \item The \CrefAndHyperref{definition:category_of_sets}{category of sets}, whose objects are sets and whose morphisms are \CrefAndHyperrefIfExist{definition:function_of_sets}{set maps/functions}.
        \item The \CrefAndHyperref{definition:category_of_groups_of_abelian_groups}{category of groups}, whose objects are \CrefAndHyperrefIfExist{definition:group}{groups} and whose morphisms are \CrefAndHyperrefIfExist{definition:group_homomorphism}{group homomorphisms}.
        \item The \CrefAndHyperref{definition:category_of_groups_of_abelian_groups}{category of abelian groups}, whose objects are \CrefAndHyperrefIfExist{definition:group}{abelian groups} and whose morphisms are \CrefAndHyperrefIfExist{definition:group_homomorphism}{group homomorphisms}.
        \item The \CrefAndHyperref{definition:category_of_topological_spaces}{category of topological spaces}, whose objects are \CrefAndHyperrefIfExist{definition:topological_space}{topological spaces} and whose morphisms are \CrefAndHyperrefIfExist{definition:continuous_map_of_topological_spaces}{continuous maps between topological spaces}.
.
        \item The \CrefAndHyperref{definition:pointed_topological_space}{category of pointed topological spaces}, whose objects $(X,x)$ are \CrefAndHyperrefIfExist{definition:pointed_topological_space}{pointed topological spaces} and whose morphisms $(X,x) \to (Y,y)$ are \CrefAndHyperrefIfExist{definition:continuous_map_of_topological_spaces}{continuous maps} $f: X \to Y$ such that $f(x) = y$.

        \item The \CrefAndHyperref{definition:category_of_vector_spaces_over_a_field}{category of vector spaces} over a fixed \CrefAndHyperrefIfExist{definition:field}{field} $k$ whose objects are \CrefAndHyperrefIfExist{definition:vector_space_over_a_field}{vector spaces} over $k$ and whose morphisms are \CrefAndHyperrefIfExist{definition:morphism_of_vector_spaces}{$k$-linear maps}.

        \item The \CrefAndHyperref{definition:category_of_vector_spaces_over_a_field}{category of finite dimensional vector spaces} over a fixed \CrefAndHyperrefIfExist{definition:field}{field} $k$ whose objects are \CrefAndHyperrefIfExist{definition:basis_of_a_vector_space_over_a_field}{finite dimensional} \CrefAndHyperrefIfExist{definition:vector_space_over_a_field}{vector spaces} over $k$ and whose morphisms are \CrefAndHyperrefIfExist{definition:morphism_of_vector_spaces}{$k$-linear maps}.

        \item Given a \CrefAndHyperrefIfExist{definition:ring}{ring} $R$, the \CrefAndHyperrefIfExist{definition:category_of_modules_and_bimodules_over_rings}{category of (either left or right) $R$-modules} whose objects are \CrefAndHyperrefIfExist{definition:module_of_a_ring}{$R$-modules} and whose morphisms are \CrefAndHyperrefIfExist{definition:homomorphism_of_modules_over_a_ring}{$R$-module homomorphisms}.

        \item The \CrefAndHyperrefIfExist{definition:category_of_rings}{category of rings} whose objects are \CrefAndHyperrefIfExist{definition:ring}{rings} and whose morphisms are \CrefAndHyperrefIfExist{definition:ring_homomorphism}{ring homomorphisms}.

        \item The \CrefAndHyperrefIfExist{definition:category_of_rings}{category of commutative rings} whose objects are \CrefAndHyperrefIfExist{definition:commutative_ring}{commutative rings} and whose morphisms are \CrefAndHyperrefIfExist{definition:ring_homomorphism}{ring homomorphisms}.

        \item The \CrefAndHyperrefIfExist{definition:category_of_small_categories}{category of small categories}, whose objects are the \CrefAndHyperrefIfExist{definition:locally_small_category}{small categories} and whose morphisms are \CrefAndHyperrefIfExist{definition:functor_between_categories}{functors}.

        \item Given a fixed topological space $X$, the \CrefAndHyperrefIfExist{definition:category_of_opens_of_a_topological_space}{category $\operatorname{Open}(X)$ of open subsets of $X$}, whose objects are the open subsets of $X$ and whose morphisms $U \to V$ are given exactly by inclusions $U \subseteq V$. More precisely, for each inclusion $U \subseteq V$ of open subsets of $X$, there is a unique morphism $U \to V$, and the composition $U \to V \to W$ is the unique morphism $U \to W$ corresponding to the inclusion $U \subseteq W$. 
    \end{enumerate}
\end{example}

\begin{definition} \label{definition:category_of_sets}
    The category of sets is the \CrefAndHyperrefIfExist{definition:locally_small_category}{(locally small)} \CrefAndHyperrefIfExist{definition:category}{category} 
    \begin{itemize}
        \item whose objects are \CrefAndHyperrefIfExist{definition:zermelo_fraenkel_set_theory}{sets}, and 
        \item whose morphisms $X \to Y$ are \CrefAndHyperrefIfExist{definition:function_of_sets}{set functions} $X \to Y$. 
    \end{itemize}

    The category of sets is often denoted by notations such as \hl{$\mathrm{Set}$}, \hl{$\mathbf{Set}$}, \hl{$\mathrm{Sets}$}, \hl{$\mathbf{Sets}$}, \hl{$(\mathrm{Set})$}, \hl{$(\mathbf{Set})$}, \hl{$(\mathrm{Sets})$}, \hl{$(\mathbf{Sets})$}.
\end{definition}
\begin{definition} \label{definition:category_of_groups_of_abelian_groups}
    \begin{enumerate}
        \item The \hldef{category of groups} is the \CrefAndHyperrefIfExist{definition:locally_small_category}{locally small} \CrefAndHyperrefIfExist{definition:category}{category} whose objects are \CrefAndHyperrefIfExist{definition:group}{groups} and whose morphisms are \CrefAndHyperrefIfExist{definition:group_homomorphism}{group homomorphisms}. It is often denoted by notations such as \hl{$\mathbf{Grp}$}.

        \item The \hldef{category of abelian groups} is the \CrefAndHyperrefIfExist{definition:locally_small_category}{locally small} \CrefAndHyperrefIfExist{definition:category}{category} whose objects are \CrefAndHyperrefIfExist{definition:group}{abelian groups} and whose morphisms are \CrefAndHyperrefIfExist{definition:group_homomorphism}{group homomorphisms}. It is often denoted by notations such as \hl{$\mathbf{Ab}$}.
    \end{enumerate}
\end{definition}
\begin{definition} \label{definition:category_of_topological_spaces}
    The \hldef{category of topological spaces} is the \CrefAndHyperrefIfExist{definition:locally_small_category}{(locally small)} \CrefAndHyperrefIfExist{definition:category}{category}
    \begin{itemize}
        \item whose objects are \CrefAndHyperrefIfExist{definition:topological_space}{topological spaces}, and
        \item whose morphisms are \CrefAndHyperrefIfExist{definition:continuous_map_of_topological_spaces}{continuous maps}.
    \end{itemize}
    The category of topological spaces is often denoted by notations such as \hl{$\mathrm{Top}$}, \hl{$\mathbf{Top}$}, etc. 
\end{definition}
\begin{definition}[Pointed topological space] \label{definition:pointed_topological_space}
Let $X$ be a \CrefAndHyperrefIfExist{definition:topological_space}{topological space} and let $x_0 \in X$ be a chosen element of $X$.  
A \hldef{pointed/based (topological) space} is a pair $(X, x_0)$ consisting of the space $X$ together with the distinguished point $x_0$, called the \hldef{base point of $X$}. If the base point of a pointed space $(X,x_0)$ is understood, then it may be suppressed from notation; in particular, $X$ may be written as a pointed space as opposed to the full notation of $(X,x_0)$. 

A \hldef{morphism of pointed spaces} (or \hldef{based map}) or \hldef{continuous map} between pointed spaces $(X, x_0)$ and $(Y, y_0)$ is a \CrefAndHyperrefIfExist{definition:continuous_map_of_topological_spaces}{continuous map} 
$$f : X \to Y$$ 
such that $f(x_0) = y_0$.

The collection of pointed spaces with their morphisms form a \CrefAndHyperrefIfExist{definition:locally_small_category}{locally small} \CrefAndHyperrefIfExist{definition:category}{category}, often called the \hldef{category of pointed spaces}. This category is often denoted by notations such as \hl{$\mathrm{Top}_*$}, \hl{$\mathrm{Top}_\bullet$}, \hl{$\mathbf{Top}_*$}, \hl{$\mathbf{Top}_\bullet$}, etc. The set of continuous maps from pointed spaces $X$ to $Y$ may denoted by notations such as \hl{$C_*(X,Y)$}, \hl{$C_\bullet(X,Y)$}, \hl{$\mathrm{Top}_*(X,Y)$}, \hl{$\mathrm{Top}_\bullet(X,Y)$}, \hl{$\Hom_{\mathrm{Top}_\bullet}(X,Y)$}, etc. 
\end{definition}
\begin{definition} \label{definition:category_of_vector_spaces_over_a_field}
    Let $k$ be a \CrefAndHyperrefIfExist{definition:field}{field}. The \hldef{category of vector spaces over $k$} is the \CrefAndHyperrefIfExist{definition:locally_small_category}{locally small} \CrefAndHyperrefIfExist{definition:category}{category}
    \begin{itemize}
        \item whose objects are \CrefAndHyperrefIfExist{definition:vector_space_over_a_field}{vector spaces over $k$}, and
        \item whose morphisms are \CrefAndHyperrefIfExist{definition:morphism_of_vector_spaces}{$k$-linear maps}.
    \end{itemize}

    The $F$-vector spaces of finite \CrefAndHyperrefIfExist{definition:basis_of_a_vector_space_over_a_field}{dimension} form a \CrefAndHyperrefIfExist{definition:full_subcategory_of_a_category}{full subcategory}, called the \hldef{category of finite dimensional vector spaces over $k$}. Notations such as \hl{$\mathrm{Vec}_k$} or \hl{$\mathbf{Vec}_k$} are often used to denote either of these categories; when both categories are considered, notations such as \hl{$\mathrm{FinVec}_k$} or \hl{$\mathbf{FinVec}_k$} may be used to distinguish the category of finite dimensional $k$-vector spaces from the category of all $k$-vector spaces.
\end{definition}
\begin{definition} \label{definition:category_of_rings}
    \begin{enumerate}
        \item The \hldef{category of rings} is the \CrefAndHyperrefIfExist{definition:locally_small_category}{locally small} \CrefAndHyperrefIfExist{definition:category}{category} whose objects are \CrefAndHyperrefIfExist{definition:ring}{rings} $R$ and whose morphisms $R \to S$ are \CrefAndHyperrefIfExist{definition:ring_homomorphism}{ring homomorphisms}.
        The category of rings over $R$ is often denoted by notations such as \hl{$\mathbf{Ring}$}.

        \item The \hldef{category of commutative rings} is the \CrefAndHyperrefIfExist{definition:full_subcategory_of_a_category}{full subcategory} of $\mathbf{Ring}$ consisting of the \CrefAndHyperrefIfExist{definition:commutative_ring}{commutative rings}. It is denoted by notations such as \hl{$\mathbf{CommRing}$} or \hl{$\mathbf{CRing}$}. 
    \end{enumerate}

\end{definition}



\begin{definition}[Isomorphism in a category] \label{definition:isomorphism_in_a_category}
Let $\mathcal{C}$ be a \CrefAndHyperrefIfExist{definition:category}{(large) category}, and let $x,y \in \mathrm{Ob}(\mathcal{C})$.  
A morphism $f \in \mathcal{C}(x,y)$ is called an \hldef{isomorphism} if there exists a morphism $g \in \mathcal{C}(y,x)$ such that
$$ g \circ f = 1_x \qquad \text{and} \qquad f \circ g = 1_y.  $$
In this case, $g$ is called the \hldef{inverse of $f$}, and $x$ and $y$ are said to be \hldef{isomorphic objects} in $\mathcal{C}$. It is standard to write \hl{$x \cong y$} if there exists an isomorphism $f : x \to y$.

In practice, isomorphisms in specific categories may be defined in different, yet equivalent, ways.
\end{definition}



\subsection{Functors between categories}


\begin{definition}[Opposite category] \label{definition:opposite_category_of_a_category}

    Let $\mathcal{C}$ be a \hyperrefIfExists{definition:category}{(large) category}\CrefIfExists{definition:category}. The \hldef{opposite category} of $\mathcal{C}$, denoted \hl{$\mathcal{C}^{\mathrm{op}}$}, is defined as follows:
    \begin{itemize}
        \item The objects of $\mathcal{C}^{\mathrm{op}}$ are the same as those of $\mathcal{C}$.
        \item For any pair of objects $X,Y \in \mathcal{C}$, the morphisms from $X$ to $Y$ in $\mathcal{C}^{\mathrm{op}}$ are given by the morphisms from $Y$ to $X$ in $\mathcal{C}$:
        \[
        \mathrm{Hom}_{\mathcal{C}^{\mathrm{op}}}(X,Y) := \mathrm{Hom}_{\mathcal{C}}(Y,X).
        \]
        \item Composition in $\mathcal{C}^{\mathrm{op}}$ is defined by reversing the order of composition in $\mathcal{C}$. That is, for morphisms $f \in \mathrm{Hom}_{\mathcal{C}^{\mathrm{op}}}(X,Y)$ and $g \in \mathrm{Hom}_{\mathcal{C}^{\mathrm{op}}}(Y,Z)$, their composition is
        \[
        g \circ_{\mathcal{C}^{\mathrm{op}}} f := f \circ_{\mathcal{C}} g.
        \]
    \end{itemize}
    Intuitively, the category $\mathcal{C}^{\mathrm{op}}$ thus "reverses" the direction of all morphisms in $\mathcal{C}$.

\end{definition}

\begin{definition} \label{definition:functor_between_categories}
Let $\mathcal{C}$ and $\mathcal{D}$ be \CrefAndHyperrefIfExist{definition:category}{(large) categories}. 
\begin{enumerate}
  \item A \hldef{functor $F: \calC \to \calD$ (from $\mathcal{C}$ to $\mathcal{D}$)} consists of :
  \begin{itemize}
    \item For each object $X$ in $\mathcal{C}$, an object $F(X)$ in $\mathcal{D}$.
    \item For each morphism $f: X \to Y$ in $\mathcal{C}$, a morphism $F(f): F(X) \to F(Y)$ in $\mathcal{D}$,
  \end{itemize}
  such that:
  \begin{align*}
    F(\mathrm{id}_X) &= \mathrm{id}_{F(X)} \quad \text{for all objects } X \text{ in } \mathcal{C}, \\
    F(g \circ f) &= F(g) \circ F(f) \quad \text{for all } X,Y,Z \in \Ob(\calC) \text{ and all } f: X \to Y, g: Y \to Z \text{ in } \mathcal{C}.
  \end{align*}

  Functors as defined above are also referred to as \hldef{covariant functors} to distinguish them from contravariant functors

  \item A \hldef{contravariant functor from $\calC$ to $\calD$} refers to a covariant functor $F:\calC^{\op} \to \calD$. Equivalently, such a functor consists of 
  \begin{itemize}
    \item For each object $X$ in $\mathcal{C}$, an object $F(X)$ in $\mathcal{D}$.
    \item For each morphism $f: X \to Y$ in $\mathcal{C}$, a morphism $F(f): F(Y) \to F(X)$ in $\mathcal{D}$,
  \end{itemize}
  such that:
  \begin{align*}
    F(\mathrm{id}_X) &= \mathrm{id}_{F(X)} \quad \text{for all objects } X \text{ in } \mathcal{C}, \\
    F(g \circ f) &= F(f) \circ F(g) \quad \text{for all } X,Y,Z \in \Ob(\calC) \text{ and all } f: X \to Y, g: Y \to Z \text{ in } \mathcal{C}.
  \end{align*}
  \TextIfExists{definition:presheaf_on_a_category}{A synonym for a ``contravariant functor from $\calC$ to $\calD$'' is a ``\CrefAndHyperrefIfExist{definition:presheaf_on_a_category}{presheaf on $\calC$ with values in $\calD$}''.}
  
\end{enumerate}
Note that declarations such as ``Let $F: \calC^{\op} \to \calD$ be a contravariant functor'' can be common; such declarations usually mean ``Let $F$ be a contravariant functor from $\calC$ to $\calD$'' as opposed to ``Let $F$ be a contravariant functor from $\calC^{\op}$ to $\calD$''. further note that a contravariant functor from $\calC$ to $\calD$ is equivalent to a covariant functor from $\calC^{\op}$ to $\calD$.
\end{definition}



\begin{example}
    Here are some examples of \CrefAndHyperrefIfExist{definition:functor_between_categories}{functors}:
    \begin{enumerate}
        \item For any category $\calC$, its \CrefAndHyperrefIfExist{definition:identity_functor_on_a_category}{identity functor}.
        \item ``forgetful functors''; some \CrefAndHyperrefIfExist{definition:forgetful_functor_between_categories}{forgetful functors} include 
        \begin{enumerate}
            \item The forgetful functor $F:\mathbf{Grp} \to \mathbf{Sets}$\CrefIfExists{definition:category_of_groups_of_abelian_groups}\CrefIfExists{definition:category_of_sets} sending a \CrefAndHyperrefIfExist{definition:group}{group} $G$ to the underlying set of $G$, and sending a \CrefAndHyperrefIfExist{definition:group_homomorphism}{group homomorphism} $G_1 \to G_2$ to the \CrefAndHyperrefIfExist{definition:function_of_sets}{set function} $G_1 \to G_2$. One can verify that 
              \begin{align*}
                F(\mathrm{id}_X) &= \mathrm{id}_{F(X)} \quad \text{for all objects } G \text{ in } \mathbf{Grp}, \\
                F(g \circ f) &= F(g) \circ F(f) \quad \text{for all } G_1,G_2,G_3 \in \Ob(\mathbf{Grp}) \text{ and all } f: G_1 \to G_2, g: G_2 \to G_3 \text{ in } \mathbf{Grp}.
            \end{align*}

            \item Similarly, the forgetful functor $F: \Top \to \Sets$\CrefIfExists{definition:category_of_topological_spaces} \CrefIfExists{definition:category_of_sets} sending a \CrefAndHyperrefIfExist{definition:category_of_topological_spaces}{topological space} $X$ to the underlying set of $X$, and sending a \CrefAndHyperrefIfExist{definition:continuous_map_of_topological_spaces}{continuous map} $X \to Y$ to the \CrefAndHyperrefIfExist{definition:function_of_sets}{set function} $X \to Y$.

        \item The forgetful functor $F: \mathbf{Ab} \to \mathbf{Grp}$ \CrefIfExists{definition:category_of_groups_of_abelian_groups} sending an abelian group $A$ to itself, and sending a group homomorphism $A_1 \to A_2$ to itself.
        \end{enumerate}

    \item ``Free'' functors
    \begin{enumerate}
        \item There is a functor $\Sets \to \mathbf{Grp}$ sending a set $S$ to the \CrefAndHyperrefIfExist{definition:free_group_generated_by_a_set}{free group} $\langle S \rangle$ generated by $S$. The functor sends the morphism $f:S_1 \to S_2$ of sets, to the unique \CrefAndHyperrefIfExist{definition:group_homomorphism}{group homomorphism} $\langle S_1 \rangle \to \langle S_2 \rangle$ given by sending $s \in S_1$ to $f(s) \in S_2$.

        \item Similarly, there is a functor $\Sets \to \mathbf{Ab}$ sending a set $S$ to the \CrefAndHyperrefIfExist{definition:free_abelian_group_generated_by_a_set}{free abelian group} $\bbZ S$ generated by $S$. The functor sends the morphism $f:S_1 \to S_2$ of sets, to the unique \CrefAndHyperrefIfExist{definition:group_homomorphism}{group homomorphism} $\bbZ S_1 \to \bbZ S_2$ given by sending $s \in S_1$ to $f(s) \in S_2$.

    \end{enumerate}

    \item The fundamental group functor $\Top_\bullet \to \mathbf{Grp}$\CrefIfExists{definition:pointed_topological_space}; given a \CrefAndHyperrefIfExist{definition:pointed_topological_space}{pointed topological space} $(X,x)$, its associated \CrefAndHyperrefIfExist{definition:homotopy_groups_of_a_pointed_topological_space}{fundamental group} $\pi_1(X,x)$ is the group of homotopy classes of loops $\gamma: [0,1] \to X$. Given a morphism $(X,x) \to (Y,y)$ of pointed topological spaces, there is an (functorially) induced morphism $f: \pi_1(X,x) \to \pi_1(Y,Y)$ sending the homotopy class $[\gamma]$ of the loop $\gamma: [0,1] \to X$ to the homotopy class $[f \circ \gamma]$.
    \end{enumerate}
\end{example}

\begin{definition} \label{definition:identity_functor_on_a_category}
Let $\mathcal{C}$ be a \CrefAndHyperrefIfExist{definition:category}{category}. The \hldef{identity functor on $\mathcal{C}$} is the functor
\hl{$1_{\mathcal{C}} \colon \mathcal{C} \to \mathcal{C}$}
(also denoted by \hl{$\text{id}_{\mathcal{C}}$}) defined by the following data:
\begin{itemize}
    \item For every object $X \in \text{Ob}(\mathcal{C})$, $1_{\mathcal{C}}(X) = X$.
    \item For every morphism $f \colon X \to Y$ in $\mathcal{C}$, $1_{\mathcal{C}}(f) = f$.
\end{itemize}
It satisfies the functor axioms trivially: $1_{\mathcal{C}}(f \circ g) = f \circ g = 1_{\mathcal{C}}(f) \circ 1_{\mathcal{C}}(g)$ and $1_{\mathcal{C}}(\text{id}_X) = \text{id}_X$.
\end{definition}

\begin{definition} \label{definition:forgetful_functor_between_categories}
Let $\mathcal{C}$ and $\mathcal{D}$ be \CrefAndHyperrefIfExist{definition:category}{categories}. A \CrefAndHyperrefIfExist{definition:functor_between_categories}{functor} $U: \mathcal{C} \to \mathcal{D}$ is called a \hldef{forgetful functor} if it maps an object in $\mathcal{C}$ to an object in $\mathcal{D}$ by discarding some of its structure or properties, and maps morphisms accordingly. Common examples include the functor from the category of groups to the category of sets, or from the category of topological spaces to the category of sets.
\end{definition}

\begin{example} \label{example:examples_of_contravariant_functors}
    Here are some examples of \CrefAndHyperrefIfExist{definition:functor_between_categories}{contravariant functors}.
    \begin{enumerate}
        \item The dual of a vector space: given a \CrefAndHyperrefIfExist{definition:vector_space_over_a_field}{vector space} $V$ over a field $k$, the \CrefAndHyperrefIfExist{definition:dual_of_a_left_right_two_sided_module}{dual} is defined as $V^\vee \coloneq \Hom_k(V, k)$. The assignment $V \mapsto V^\vee$ specifies a contravariant functor $\VectorSpaces_k \to \VectorSpaces_k$ --- given a \CrefAndHyperrefIfExist{definition:morphism_of_vector_spaces}{linear map} $f:V_1 \to V_2$, there is an induced linear map $V_2^\vee \to V_1^\vee$ given by sending the linear map $\phi: V_2 \to k$, which is an element of $V_2^\vee$, to the linear map $V_2 \circ f V_1 \to k$, which is an element of $V_1^\vee$. 


        \item A \CrefAndHyperrefIfExist{definition:representable_functor_on_a_locally_small_category}{representable functor}: given any \CrefAndHyperrefIfExist{definition:locally_small_category}{locally small category} $\calC$ and any object $X$ of $\calC$, there is a contravariant functor $h_X: \calC^{\op} \to \Sets, T \mapsto \Hom_{\calC}(T,X)$; given a morphism $f:T_1 \to T_2$ in $\calC$, there is an induced map $h_X(T_2) \to h_X(T_1)$ of sets, i.e. a map $\Hom_\calC(T_2, X) \to \Hom_\calC(T_1, X)$, given by $\phi \mapsto \phi \circ f$.

        \item There is the contravariant power set functor $\calP: \Sets^{\op} \to \Sets$ that sends a set $S$ to its \CrefAndHyperrefIfExist{definition:power_set_of_a_set}{power set} $\calP(S)$ and that sends a set morphism $f: S \to T$ to the set morphism $f^*: \calP(T) \to \calP(S)$ given by $B \mapsto f^{-1}(B)$. 
        
        % Given a \CrefAndHyperrefIfExist{definition:continuous_map_of_topological_spaces}{continuous map} $f: X \to Y$ of \CrefAndHyperrefIfExist{definition:topological_space}{topological spaces}, there is a ``preimage functor'' $f^{-1}: \operatorname{Top}(Y) \to \operatorname{Top}

        \item The functor of continuous functions: let $\Top$ be the category of topological spaces and $\mathbb{R}\text{-}\operatorname{Alg}$ the category of \CrefAndHyperrefIfExist{definition:algebra_of_a_ring}{$\mathbb{R}$-algebras}. The assignment $X \mapsto C(X, \mathbb{R})$ of a space to its algebra of continuous real-valued functions is a contravariant functor. For any continuous map $g: X \to Y$, the induced algebra homomorphism $g^*: C(Y, \mathbb{R}) \to C(X, \mathbb{R})$ is given by the pullback $g^*(\phi) = \phi \circ g$ for $\phi \in C(Y, \mathbb{R})$.


    \end{enumerate}
\end{example}





\subsection{Natural transformation}

One overarching philosophy in various categories is that we only really care about objects ``up to equivalence'';  intuitively, we consider objects to be equivalent when they are \CrefAndHyperrefIfExist{definition:isomorphism_in_a_category}{isomorphic}. Similarly, we only really care about categories ``up to equivalence'' as well; there is a notion of \CrefAndHyperrefIfExist{definition:equivalence_of_categories}{equivalence} between categories. To define it, we first need to define the notion of \CrefAndHyperrefIfExist{definition:natural_transformation_between_functors_between_categories}{natural transformations} between functors --- a natural transformation is like a ``morphism'' between functors in a sense. 

\begin{definition} \label{definition:natural_transformation_between_functors_between_categories}
Let $\mathcal{C}$ and $\mathcal{D}$ be \CrefAndHyperrefIfExist{definition:category}{(large) categories}. 
Let $F, G : \mathcal{C} \to \mathcal{D}$ be \CrefAndHyperrefIfExist{definition:functor_between_categories}{functors}.

A \hldef{natural transformation $\eta$ between $F$ and $G$} is a family of morphisms $\eta_X: F(X) \to G(X)$ in $\mathcal{D}$, one for each object $X$ in $\mathcal{C}$, such that for every morphism $f: X \to Y$ in $\mathcal{C}$,
\begin{align*}
G(f) \circ \eta_X = \eta_Y \circ F(f)
\end{align*}
in $\mathcal{D}$. In other words, the following diagram commutes:
\begin{center}
\begin{tikzcd}
    F(X) \arrow[r, "F(f)"] \arrow[d, "\eta_X"']
    & F(Y) \arrow[d, "\eta_Y"] \\
    G(X) \arrow[r, "G(f)"']
    & G(Y)
\end{tikzcd}
\end{center}

We write such a natural transformation by \hl{$\eta: F \Rightarrow G$}.

If $\eta_X$ is an \CrefAndHyperrefIfExist{definition:isomorphism_in_a_category}{isomorphism} for all objects $X$ of $\calC$, then $\eta$ is said to be a \hldef{natural isomorphism}.
\end{definition}


\begin{example}
    Let $k$ be a \CrefAndHyperrefIfExist{definition:field}{field}. Note that there is a ``double dual'' functor $\left((-)^{\vee} \right)^\vee:\VectorSpaces_k \to \VectorSpaces_k$ given by $V \mapsto (V^{\vee})^\vee$\CrefIfExists{definition:dual_of_a_left_right_two_sided_module}. \CrefAndHyperref{example:examples_of_contravariant_functors}{Recall that} $V \mapsto V^\vee$ is a \CrefAndHyperrefIfExist{definition:functor_between_categories}{contravariant functor}, so the double dual functor is covariant. Recall that there is also the \CrefAndHyperrefIfExist{definition:identity_functor_on_a_category}{identity functor} $\id_{\VectorSpaces_k}: \VectorSpaces_k \to \VectorSpaces_k$ given by $V \mapsto V$.  

    For general vector spaces $V$, there is an injective \CrefAndHyperrefIfExist{definition:morphism_of_vector_spaces}{$k$-linear map}
    $$\alpha_V: V \to (V^{\vee})^\vee, \quad v \mapsto v^*, $$
    where $v^*: V^\vee \to k$ is the $k$-linear map given by $\phi \mapsto \phi(v)$. We note that if $V$ is \CrefAndHyperrefIfExist{definition:basis_of_a_vector_space_over_a_field}{finite dimensional}, then $\alpha_V$ is an isomorphism. In fact, $\alpha$ is a natural transformation $\id_{\VectorSpaces_k} \Rightarrow \left( (-)^\vee \right)^\vee$ --- one should verify that, for every $k$-linear map $f: V_1 \to V_2$, the following diagram commutes:
    \begin{center}
    \begin{tikzcd}
        V_1 \arrow[r, "f"] \arrow[d, "\alpha_{V_1}"']
        & V_2 \arrow[d, "\alpha_{V_2}"] \\
        (V_1^\vee)^\vee \arrow[r, "(f^\vee)^\vee"']
        & (V_2^\vee)^\vee 
    \end{tikzcd}
    \end{center}
\end{example}

\begin{example}
    Let $\operatorname{CRing}$ be the \CrefAndHyperrefIfExist{definition:category_of_rings}{category of commutative rings} and $\operatorname{Group}$ the \CrefAndHyperrefIfExist{definition:category_of_groups_of_abelian_groups}{category of groups}. For a fixed $n \ge 1$, we have a \CrefAndHyperrefIfExist{definition:functor_between_categories}{covariant functor} $\GL_n: \operatorname{CRing} \to \operatorname{Group}$ assigning a ring $R$ to the general linear group $\GL_n(R)$, and a functor $(\cdot)^\times: \operatorname{CRing} \to \operatorname{Group}$ assigning a ring to its \CrefAndHyperrefIfExist{definition:unit_of_a_ring}{group of units}. 
    
    The determinant $\det: \GL_n \Rightarrow (\cdot)^\times$ is a \CrefAndHyperrefIfExist{definition:natural_transformation_between_functors_between_categories}{natural transformation}. Its component at a ring $R$ is the group homomorphism $\det_R: \GL_n(R) \to R^\times$. For any ring homomorphism $f: R \to S$, the following diagram commutes:
    \begin{center}
    \begin{tikzcd}
        \GL_n(R) \arrow[r, "\GL_n(f)"] \arrow[d, "\det_R"'] & \GL_n(S) \arrow[d, "\det_S"] \\
        R^\times \arrow[r, "f|_{R^\times}"'] & S^\times
    \end{tikzcd}
    \end{center}
    This commutativity expresses that the determinant is defined by the same universal formula (a polynomial in the matrix entries) regardless of the ring $R$, and is thus preserved by the "change of scalars" $f$.
\end{example}

\begin{definition} \label{definition:equivalence_of_categories}
An \hldef{equivalence of categories} between two \CrefAndHyperrefIfExist{definition:category}{(large) categories} $\mathcal{C}$ and $\mathcal{D}$ consists of a pair of \CrefAndHyperrefIfExist{definition:functor_between_categories}{functors}
$$F : \mathcal{C} \to \mathcal{D} \quad \text{and} \quad G : \mathcal{D} \to \mathcal{C}$$
together with \CrefAndHyperrefIfExist{definition:natural_transformation_between_functors_between_categories}{natural isomorphisms}
$$\eta : \mathrm{Id}_{\mathcal{C}} \xrightarrow{\sim} G \circ F \quad \text{and} \quad \epsilon : F \circ G \xrightarrow{\sim} \mathrm{Id}_{\mathcal{D}}.$$
\CrefIfExists{definition:identity_functor_on_a_category} Such functors $F$ and $G$ may be called \hldef{(natural) inverses of each other}.

When $\calC$ and $\calD$ are \CrefAndHyperrefIfExist{definition:locally_small_category}{locally small categories}, $F$ is an equivalence of categories if and only if $F$ is \CrefAndHyperrefIfExist{definition:full_and_faithful_functor_between_locally_small_categories}{fully faithful} and \CrefAndHyperrefIfExist{definition:essentially_surjective_functor_between_categories}{essentially surjective}
\end{definition}

The ``correct'' notion for one category to embed into another is the notion of a \CrefAndHyperrefIfExist{definition:full_and_faithful_functor_between_locally_small_categories}{fully faithful functor}.


\begin{definition} \label{definition:full_and_faithful_functor_between_locally_small_categories}

Let $\mathcal{C}$ and $\mathcal{D}$ be \CrefAndHyperrefIfExist{definition:category}{(large)) categories}. Let $F : \mathcal{C} \to \mathcal{D}$ be a \CrefAndHyperrefIfExist{definition:functor_between_categories}{functor}. 
\begin{enumerate}
    \item $F$ is called \hldef{full} if for every pair of objects $x,y \in \mathrm{Ob}(\mathcal{C})$, the induced rule/assignment/class function
    $$ F_{x,y} : \Hom_\mathcal{C}(x,y) \to \Hom_\mathcal{D}(F(x), F(y)) $$
    on Hom-collections is ``surjective'', i.e. for all morphisms $g:F(x) \to F(y)$, there exists some morphism $f: x \to y$ such that $F(f) = g$. 

    \item $F$ is called \hldef{faithful} if for every pair of objects $x,y \in \mathrm{Ob}(\mathcal{C})$, 
    the induced class function (assignment)
    $$ F_{x,y} : \mathrm{Hom}_\mathcal{C}(x,y) \to \mathrm{Hom}_\mathcal{D}(F(x), F(y)) $$
    on Hom-collections is ``injective'', i.e., for any morphisms $f_1, f_2 \in \mathrm{Hom}_\mathcal{C}(x,y)$, 
    if $F(f_1) = F(f_2)$ in $\mathrm{Hom}_\mathcal{D}(F(x), F(y))$, then $f_1 = f_2$.

    \item $F$ is called \hldef{fully faithful} if it is both full and faithful.
\end{enumerate}

\end{definition}


\begin{definition}[Subcategory] \label{definition:subcategory_of_a_category}
    Let $\mathcal{C}$ be a \CrefAndHyperrefIfExist{definition:category}{(large) category}. A \hldef{subcategory} $\mathcal{D}$ of $\mathcal{C}$ consists of:
    \begin{itemize}
        \item a subclass of objects $\mathrm{Ob}(\mathcal{D}) \subseteq \mathrm{Ob}(\mathcal{C})$,
        \item for each pair of objects $X, Y \in \mathrm{Ob}(\mathcal{D})$, a subclass of morphisms
        $$\mathrm{Hom}_{\mathcal{D}}(X,Y) \subseteq \mathrm{Hom}_{\mathcal{C}}(X,Y),$$
    \end{itemize}
    such that
    \begin{itemize}
        \item for every object $X \in \mathrm{Ob}(\mathcal{D})$, the identity morphism $\mathrm{id}_X$ of $X$ in $\mathcal{C}$ lies in $\mathrm{Hom}_{\mathcal{D}}(X,X)$,
        \item the composition of morphisms in $\mathcal{D}$ is inherited from $\mathcal{C}$ and is closed in $\mathcal{D}$: for morphisms $f \in \mathrm{Hom}_{\mathcal{D}}(X,Y)$ and $g \in \mathrm{Hom}_{\mathcal{D}}(Y,Z)$, their composition $g \circ f \in \mathrm{Hom}_{\mathcal{D}}(X,Z)$.
    \end{itemize}
\end{definition}


\begin{definition}[Full subcategory] \label{definition:full_subcategory_of_a_category}
    Let $\mathcal{C}$ be a \CrefAndHyperrefIfExist{definition:category}{(large) category}. A \hldef{full subcategory} $\mathcal{D}$ of $\mathcal{C}$ is a \CrefAndHyperrefIfExist{definition:subcategory_of_a_category}{subcategory} such that for every pair of objects $X, Y \in \mathrm{Ob}(\mathcal{D})$, the morphism classes coincide:
    $$\mathrm{Hom}_{\mathcal{D}}(X,Y) = \mathrm{Hom}_{\mathcal{C}}(X,Y).$$
    In other words, a full subcategory includes all morphisms between its objects that exist in the ambient category $\mathcal{C}$.
\end{definition}


The ``correct'' notion for one category to ``surject'' onto another is the notion of a \CrefAndHyperrefIfExist{definition:essentially_surjective_functor_between_categories}{essentially surjective functor}.

\begin{definition}[Essential image of a functor] \label{definition:essential_image_of_a_functor_between_categories}
Let $F : \mathcal{C} \to \mathcal{D}$ be a functor between \CrefAndHyperrefIfExist{definition:category}{(large) categories}.  
The \hldef{essential image of $F$} is the \CrefAndHyperrefIfExist{definition:full_subcategory_of_a_category}{full subcategory} of $\mathcal{D}$ whose objects are those $d \in \mathrm{Ob}(\mathcal{D})$ for which there exists an object $c \in \mathrm{Ob}(\mathcal{C})$ such that
$$ F(c) \cong d.  $$
\CrefIfExists{definition:isomorphism_in_a_category} Equivalently, the essential image is given by
$$\hlin{\mathrm{EssIm}(F) = \{\, d \in \mathrm{Ob}(\mathcal{D}) \mid \exists c \in \mathrm{Ob}(\mathcal{C}), \, F(c) \cong d \,\},}$$
endowed with all morphisms $\mathcal{D}(d,d')$ between such objects.

\TextIfExists{definition:replete_subcategory_of_a_category}{Equivalently, the essential image of $F$ is the smallest \CrefAndHyperrefIfExist{definition:replete_subcategory_of_a_category}{replete} \CrefAndHyperrefIfExist{definition:full_subcategory_of_a_category}{full subcategory} of $\calD$ containing the \CrefAndHyperrefIfExist{definition:image_of_a_functor_between_categories}{image} of $F$}
\end{definition}

\begin{definition} \label{definition:essentially_surjective_functor_between_categories}
Let $F : \mathcal{C} \to \mathcal{D}$ be a functor between \CrefAndHyperrefIfExist{definition:category}{(large) categories}. It is said to be essentially surjective if its \CrefAndHyperrefIfExist{definition:essential_image_of_a_functor_between_categories}{essential image} coincides with $\calD$.
\end{definition}

\begin{example} \label{example:properties_of_functors}
    Below are examples illustrating various properties of functors:
    \begin{enumerate}
        \item \textbf{Faithful but not full:} The \CrefAndHyperrefIfExist{definition:forgetful_functor_between_categories}{forgetful functor} $U: \operatorname{Group} \to \operatorname{Set}$. It is \CrefAndHyperrefIfExist{definition:full_and_faithful_functor_between_locally_small_categories}{faithful} because group homomorphisms are distinct if they are distinct as functions. It is not full because not every function between groups is a group homomorphism (e.g., the constant function $x \mapsto g$ for $g \neq e$).

        \item \textbf{Full but not faithful:} The canonical functor $H: \mathsf{Top} \to h\mathsf{Top}$ from the \CrefAndHyperrefIfExist{definition:category_of_topological_spaces}{category of topological spaces} to the \CrefAndHyperrefIfExist{definition:homotopy_category_of_topological_spaces}{homotopy category} of topological spaces. It is the identity on objects and sends a continuous map $f$ to its homotopy class $[f]$. This is \CrefAndHyperrefIfExist{definition:full_and_faithful_functor_between_locally_small_categories}{full} by the definition of morphisms in $h\mathsf{Top}$, but not \CrefAndHyperrefIfExist{definition:full_and_faithful_functor_between_locally_small_categories}{faithful} because it identifies distinct but \CrefAndHyperrefIfExist{definition:homotopy_of_maps_of_topological_spaces_relative_to_a_subset}{homotopic maps} (e.g., any two paths in $\mathbb{R}^n$ with the same endpoints).

        \item \textbf{Fully faithful:} The inclusion functor $\iota: \operatorname{Ab} \to \operatorname{Group}$. It is \CrefAndHyperrefIfExist{definition:full_and_faithful_functor_between_locally_small_categories}{faithful} (it is an embedding) and it is \CrefAndHyperrefIfExist{definition:full_and_faithful_functor_between_locally_small_categories}{full} because any group homomorphism between two abelian groups is, by definition, a morphism in $\operatorname{Ab}$.

        \item \textbf{Essentially surjective:} Let $\mathbf{S}$ be the category whose objects are the standard sets $\underline{n} = \{0, \dots, n-1\}$ for each $n \in \mathbb{N}$ (and whose objects are the set functions between these sets), and $\mathbf{FinSet}$ be the category of all finite sets. The inclusion functor $I: \mathbf{S} \to \mathbf{FinSet}$ is essentially surjective because every finite set $X$ is isomorphic to $\underline{n}$ where $n$ is the cardinality of $X$.

        % \item \textbf{Essentially surjective:} Let $\operatorname{Vect}_k^{\text{fin}}$ be the category of finite-dimensional vector spaces and $\operatorname{Mat}_k$ the category whose objects are natural numbers $n \in \mathbb{N}$ and morphisms are $m \times n$ matrices. The functor $F: \operatorname{Vect}_k^{\text{fin}} \to \operatorname{Mat}_k$ sending $V$ to $\dim(V)$ is essentially surjective because every $n \in \mathbb{N}$ is the dimension of some vector space (namely $k^n$).
    \end{enumerate}
\end{example}

\begin{lemma} \label{lemma:functor_between_locally_small_categories_is_equivalence_if_and_only_if_fully_faithful_and_essentially_surjective}
Let $\calC$ and $\calD$ be \CrefAndHyperrefIfExist{definition:locally_small_category}{locally small categories}, and let $F: \calC \to \calD$ be a \CrefAndHyperrefIfExist{definition:functor_between_categories}{functor}. 

$F$ is an \CrefAndHyperrefIfExist{definition:equivalence_of_categories}{equivalence of categories} if and only if $F$ is \CrefAndHyperrefIfExist{definition:full_and_faithful_functor_between_locally_small_categories}{fully faithful} and \CrefAndHyperrefIfExist{definition:essentially_surjective_functor_between_categories}{essentially surjective}
\end{lemma}

\subsubsection{Yoneda lemma}

The Yoneda lemma basically expresses the idea that an object of a (locally small) category is essentially determined by its morphisms to other objects. 

\begin{definition} \label{definition:representable_functor_on_a_locally_small_category}
    Let $\calC$ be a \CrefAndHyperrefIfExist{definition:locally_small_category}{locally small category}. Given an object $X$ of $\calC$, the \hldef{functor of points} \hl{$h_X$} is the \CrefAndHyperrefIfExist{definition:functor_between_categories}{functor}/\CrefAndHyperrefIfExist{definition:presheaf_on_a_category}{presheaf} $\calC^{\op} \to \Sets$ given by 
    \begin{enumerate}
        \item sending an object $T$ of $\calC$ to the set $\Hom_\calC(T, X)$, and 
        \item sending a morphism $f:T_1 \to T_2$ in $\calC$ to the \CrefAndHyperrefIfExist{definition:function_of_sets}{set map} 
        $$\Hom_\calC(T_2, X) \to \Hom_\calC(T_1, X), \quad \phi \mapsto \phi \circ f.$$
    \end{enumerate}
    
    A functor $\calC^{\op} \to \Sets$ (or equivalently, a presheaf on $\calC$ valued in $\Sets$) is said to be \hldef{representable} if it is \CrefAndHyperrefIfExist{definition:natural_transformation_between_functors_between_categories}{naturally isomorphic} to some functor $h_X$ of points for an object $X$ of $\calC$.

    Dually, a functor $\calC \to \Sets$ is called \hldef{co-representable} if it is naturally isomorphic to a functor \hl{$h^X: \calC \to \Sets$} given by $T \mapsto \Hom_\calC(X, T)$. 

    \TextIfExists{definition:representable_functor_on_a_category_enriched_in_a_monoidal_category}{Note that the above notions of representability/co-representability are special caseiof those of \Cref{definition:representable_functor_on_a_category_enriched_in_a_monoidal_category}, where the monoidal category $\calV$ is the \CrefAndHyperrefIfExist{definition:symmetric_monoidal_category}{symmetric monoidal category} \CrefAndHyperrefIfExist{definition:category_of_sets}{$\Sets$}.}
\end{definition}


\begin{theorem}[Yoneda Lemma] \label{theorem:yoneda_lemma_on_a_locally_small_category}
Let $\mathcal{C}$ be a \CrefAndHyperrefIfExist{definition:locally_small_category}{locally small category}. Let $A$ be an object of $\mathcal{C}$, and let $F: \mathcal{C} \to \mathbf{Set}$ be a \CrefAndHyperrefIfExist{definition:functor_between_categories}{covariant functor} to the \CrefAndHyperrefIfExist{definition:category_of_sets}{category of sets}. Let $h^A: \mathcal{C} \to \mathbf{Set}$ denote the covariant \CrefAndHyperrefIfExist{definition:representable_functor_on_a_locally_small_category}{representable functor} defined by $h^A(X) = \operatorname{Hom}_{\mathcal{C}}(A, X)$.

There exists a bijection
$$y_{A, F} : \operatorname{Nat}(h^A, F) \xrightarrow{\cong} F(A)$$
between the set of \CrefAndHyperrefIfExist{definition:natural_transformation_between_functors_between_categories}{natural transformations} from $h^A$ to $F$ and the set $F(A)$. This bijection is given by the mapping
$$\alpha \mapsto \alpha_A(\operatorname{id}_A),$$
where $\alpha: h^A \to F$ is a natural transformation, $\alpha_A: h^A(A) \to F(A)$ is its component at $A$, and $\operatorname{id}_A \in h^A(A) = \operatorname{Hom}_{\mathcal{C}}(A, A)$ is the identity morphism.

Furthermore, this isomorphism is natural in both $A$ and $F$. Explicitly:
\begin{enumerate}
    \item For any morphism $f: A \to B$ in $\mathcal{C}$, the following diagram commutes:
    \begin{center}
    \begin{tikzcd}[row sep=large, column sep=large]
        \operatorname{Nat}(h^B, F) \arrow[r, "y_{B,F}"] \arrow[d, "-\circ h^f"'] & F(B) \arrow[d, "F(f)"] \\
        \operatorname{Nat}(h^A, F) \arrow[r, "y_{A,F}"] & F(A)
    \end{tikzcd}
    \end{center}
    where $h^f: h^B \to h^A$ is the natural transformation induced by pre-composition with $f$.

    \item For any natural transformation $\eta: F \to G$, the following diagram commutes:
    \begin{center}
    \begin{tikzcd}[row sep=large, column sep=large]
        \operatorname{Nat}(h^A, F) \arrow[r, "y_{A,F}"] \arrow[d, "\eta \circ -"'] & F(A) \arrow[d, "\eta_A"] \\
        \operatorname{Nat}(h^A, G) \arrow[r, "y_{A,G}"] & G(A)
    \end{tikzcd}
    \end{center}
\end{enumerate}
\end{theorem}
\begin{corollary}[Yoneda Embedding] \label{corollary:yoneda_embedding_on_a_locally_small_category}
Let $\mathcal{C}$ be a \CrefAndHyperrefIfExist{definition:locally_small_category}{locally small category}. The functor
$$h^\bullet: \mathcal{C}^{\operatorname{op}} \to \mathbf{Set}^{\mathcal{C}}$$
\CrefIfExists{definition:opposite_category_of_a_category} \CrefIfExists{definition:diagram_in_a_category_indexed_by_a_small_category}
defined on objects by $A \mapsto h^A = \operatorname{Hom}_{\mathcal{C}}(A, -)$\CrefIfExists{definition:representable_functor_on_a_locally_small_category} and on morphisms by $f \mapsto h^f = (-\circ f)$ is \CrefAndHyperrefIfExist{definition:full_and_faithful_functor_between_locally_small_categories}{fully faithful}. That is, for any objects $A, B$ in $\mathcal{C}$, the map
$$\operatorname{Hom}_{\mathcal{C}}(A, B) \to \operatorname{Nat}(h^B, h^A)$$
given by sending a morphism $f: A \to B$ to the \CrefAndHyperrefIfExist{definition:natural_transformation_between_functors_between_categories}{natural transformation} $h^f: h^B \to h^A$ (pre-composition by $f$) is a bijection.

Consequently, $\mathcal{C}^{\operatorname{op}}$ embeds as a \CrefAndHyperrefIfExist{definition:full_subcategory_of_a_category}{full subcategory} of the functor category $\mathbf{Set}^{\mathcal{C}}$.
\end{corollary}
\begin{theorem}[Contravariant Yoneda Lemma] \label{theorem:contravariant_yoneda_lemma_on_a_locally_small_category}
Let $\mathcal{C}$ be a \CrefAndHyperrefIfExist{definition:locally_small_category}{locally small category}. Let $A$ be an object of $\mathcal{C}$, and let $G: \mathcal{C}^{\operatorname{op}} \to \mathbf{Set}$ be a \CrefAndHyperrefIfExist{definition:functor_between_categories}{contravariant functor} (i.e. a \CrefAndHyperrefIfExist{definition:presheaf_on_a_category}{presheaf}). Let $h_A: \mathcal{C}^{\operatorname{op}} \to \mathbf{Set}$ denote the contravariant representable functor defined by $h_A(X) = \operatorname{Hom}_{\mathcal{C}}(X, A)$.

There exists a bijection natural in $A$ and $G$:
$$\operatorname{Nat}(h_A, G) \cong G(A)$$
given by $\alpha \mapsto \alpha_A(\operatorname{id}_A)$.
\end{theorem}
\begin{corollary}[Contravariant Yoneda Embedding] \label{corollary:contravariant_yoneda_embedding_on_a_locally_small_category}
Let $\mathcal{C}$ be a \CrefAndHyperrefIfExist{definition:locally_small_category}{locally small category}. The functor
$$h_\bullet: \mathcal{C} \to \mathbf{Set}^{\mathcal{C}^{\operatorname{op}}}$$
defined on objects by $A \mapsto h_A = \operatorname{Hom}_{\mathcal{C}}(-, A)$ and on morphisms by $f \mapsto h_f = (f \circ -)$ is \CrefAndHyperrefIfExist{definition:full_and_faithful_functor_between_locally_small_categories}{fully faithful}. That is, for any objects $A, B$ in $\mathcal{C}$, the map
$$\operatorname{Hom}_{\mathcal{C}}(A, B) \to \operatorname{Nat}(h_A, h_B)$$
given by sending a morphism $f: A \to B$ to the \CrefAndHyperrefIfExist{definition:natural_transformation_between_functors_between_categories}{natural transformation} $h_f: h_A \to h_B$ (post-composition by $f$) is a bijection.

Consequently, $\mathcal{C}$ embeds as a \CrefAndHyperrefIfExist{definition:full_subcategory_of_a_category}{full subcategory} of the category of presheaves $\mathbf{Set}^{\mathcal{C}^{\operatorname{op}}}$.
\end{corollary}


\begin{example} 
The algebra of \CrefAndHyperrefIfExist{definition:C_k_morphism_between_C_k_manifolds}{smooth functions} $C^\infty(M)$ on a \CrefAndHyperrefIfExist{definition:C_k_manifold}{smooth manifold} $M$ is represented by the real line $\mathbb{R}$ in the category of smooth manifolds $\mathsf{Man}$.
\begin{enumerate}
    \item \textbf{The Functor:} Consider the covariant \CrefAndHyperrefIfExist{definition:representable_functor_on_a_locally_small_category}{representable functor} $h^{\mathbb{R}} = \operatorname{Hom}_{\mathsf{Man}}(-, \mathbb{R})$. For any manifold $M$, we have 
    $$\operatorname{Hom}_{\mathsf{Man}}(M, \mathbb{R}) = C^\infty(M),$$ 
    i.e. the representable functor $h^{\bbR}$ assigns $M$ to the set of all smooth maps $f: M \to \mathbb{R}$, which is exactly $C^\infty(M)$.
    
    \item \textbf{The Yoneda Correspondence:} By the \CrefAndHyperrefIfExist{theorem:yoneda_lemma_on_a_locally_small_category}{Yoneda Lemma}, there is a natural bijection between the points of $M$ and the natural transformations from $h^M$ to $h^{\mathbb{R}}$:
    \[ \operatorname{Nat}(h^M, h^{\mathbb{R}}) \cong h^{\mathbb{R}}(M) = C^\infty(M) \]
    
    \item \textbf{Geometric Interpretation:} This identifies a smooth function $\psi \in C^\infty(M)$ with a "natural" way to turn $M$-valued points of any other smooth manifold $X$ into real-valued points. Specifically, if we have a "probe" $\alpha: X \to M$, the function $\psi$ induces a map:
    \[ \alpha \mapsto \psi \circ \alpha \]
    mapping $\operatorname{Hom}(X, M)$ to $\operatorname{Hom}(X, \mathbb{R})$.
\end{enumerate}
This illustrates that the "global" algebraic data of a manifold (its functions) is entirely captured by its relationship to the "simplest" non-trivial manifold, $\mathbb{R}$.
\end{example}


\subsection{The category of categories}

Intuitively, one might think of a \CrefAndHyperrefIfExist{definition:functor_between_categories}{functor} as a ``morphism'' between two categories. Indeed, one can consider a category of categories in which morphisms are functors.

\begin{definition} \label{definition:category_of_small_categories}
The \hldef{category of small categories} is the category defined by the following data:
\begin{itemize}
    \item The objects are all \CrefAndHyperrefIfExist{definition:locally_small_category}{small categories}.
    \item The morphisms between two small categories $\mathcal{C}$ and $\mathcal{D}$ are the \CrefAndHyperrefIfExist{definition:functor_between_categories}{functors} $F: \mathcal{C} \to \mathcal{D}$.
    \item The composition of morphisms is the standard composition of functors.
    \item The identity morphism for each object $\mathcal{C}$ is the identity functor $1_{\mathcal{C}}$.
\end{itemize}
This category is a \CrefAndHyperrefIfExist{definition:category}{large category} and is denoted by
\hl{$\mathbf{Cat}$}
\end{definition}

If we allow ourselves to use \CrefAndHyperrefIfExist{definition:grothendieck_universe}{Grothendieck universes}, then we can also talk about a category of categories in the sense of \Cref{definition:category_of_categories_relative_to_a_grothendieck_universe}.

\subsection{Miscellaneous categorical notions}

\begin{definition}[Product Category of a Family of Categories] \label{definition:product_category_of_a_family_of_categories}

    Let $\{\mathcal{C}_i\}_{i \in I}$ be a family of \CrefAndHyperrefIfExist{definition:category}{(large) categories} indexed by a class $I$. The \hldef{product category of the family}, denoted
        $$\hlin{\prod_{i\in I} \mathcal{C}_i},$$
        is the very large category \TODO{define very large categories} defined as follows:
        \begin{itemize}
            \item The class of objects is
            $$\mathrm{Ob}\Big(\prod_{i\in I} \mathcal{C}_i\Big) = \prod_{i\in I} \mathrm{Ob}(\mathcal{C}_i),$$
            i.e., an object is a family $(A_i)_{i\in I}$ with $A_i \in \mathrm{Ob}(\mathcal{C}_i)$.

            \item For two objects $(A_i)_i$ and $(B_i)_i$, the morphism class is
            $$\mathrm{Hom}_{\prod_{i\in I} \mathcal{C}_i}((A_i)_i,(B_i)_i) = \prod_{i\in I} \mathrm{Hom}_{\mathcal{C}_i}(A_i,B_i).$$
            In other words, a morphism $(f_i)_i : (A_i)_i \to (B_i)_i$ consists of morphisms $f_i: A_i \to B_i$ in each $\mathcal{C}_i$.

            \item For morphisms $(f_i)_i : (A_i)_i \to (B_i)_i$ and $(g_i)_i : (B_i)_i \to (C_i)_i$, composition is defined componentwise:
            $$(g_i)_i \circ (f_i)_i = (g_i \circ_i f_i)_i.$$

            \item For each object $(A_i)_i$, the identity morphism is given by the family $$(\mathrm{id}_{A_i})_i.$$
        \end{itemize}
        If $I$ is a set, then $\prod_{i \in I} \calC_i$ is a large category. If $I$ is a set and if each $\calC_i$ is \CrefAndHyperrefIfExist{definition:locally_small_category}{locally small}, then $\prod_{i \in I} \calC_i$ is locally small. 

        In case that $I$ is finite, the notation of \hl{$\times$} may be used for product categories, e.g. \hl{$\calC_i \times \calC_j$} denotes the product of two categories $\calC_i \times \calC_j$.

        \TODO{ordinal, $U_\alpha$}
        If $\alpha$ is an ordinal such that $\calC_i$ and $I$ are $U_\alpha$-large (i.e. they live in $U_{\alpha+1}$), then $\prod_{i \in I} \calC_i$ is $U_{\alpha+1}$-large.

\end{definition}

\section{Additive and abelian categories}

\subsection{Rings and modules}


\begin{definition} \label{definition:module_of_a_ring}
Let $R$ be a \CrefAndHyperrefIfExist{definition:ring}{not-necessarily commutative ring}. 
\begin{enumerate}
    \item A \hldef{left $R$-module} is an abelian group $(M,+)$ together with an operation $R \times M \to M$, denoted $(r,m) \mapsto rm$, such that for all $r,s \in R$ and $m,n \in M$:
    \begin{itemize}
        \item $r(m+n) = rm + rn$,
        \item $(r+s)m = rm + sm$,
        \item $(rs)m = r(sm)$,
        \item $1_R m = m$ where $1_R$ is the multiplicative identity of $R$.
    \end{itemize}

    \item A \hldef{right $R$-module} is defined similarly as an abelian group $(M,+)$ with an operation $M \times R \to M$, denoted $(m,r) \mapsto mr$, such that for all $r,s \in R$ and $m,n \in M$:
    \begin{itemize}
        \item $(m+n)r = mr + nr$,
        \item $m(r+s) = mr + ms$,
        \item $m(rs) = (mr)s$,
        \item $m 1_R = m$.
    \end{itemize}

    \item Let $R$ and $S$ be  (not necessarily commutative) \CrefAndHyperrefIfExist{definition:ring}{rings}.

    An \hldef{$R$-$S$-bimodule} (or an \hldef{$R$-$S$-module} or an $(R,S)$-module, etc.)is an \CrefAndHyperrefIfExist{definition:group}{abelian group} $(M,+)$ equipped with
    \begin{enumerate}
        \item a left action of $R$:
        $$\hlin{R \times M \to M, \quad (r,m) \mapsto r \cdot m},$$
        making $M$ a \CrefAndHyperrefIfExist{definition:module_of_a_ring}{left $R$-module},
        \item a right action of $S$:
        $$\hlin{M \times S \to M, \quad (m,s) \mapsto m \cdot s},$$
        making $M$ a right $S$-module,
    \end{enumerate}
    such that the left and right actions commute; that is, for all $r \in R$, $s \in S$, and $m \in M$,
    $$ r \cdot (m \cdot s) = (r \cdot m) \cdot s.  $$

    \item A \hldef{two-sided $R$-module} (or \hldef{$R$-bimodule}) is an $R$-$R$-bimodule.
    
    % an abelian group $(M,+)$ which is simultaneously a left $R$-module and a right $R$-module, such that $(rm)s = r(ms)$ for all $r,s \in R$, $m \in M$. Equivalently, a two-sided $R$-module is an \hldef{$R$-$R$-bimodule}\CrefIfExists{definition:module_of_a_ring}


\end{enumerate}
If $R$ is a \CrefAndHyperrefIfExist{definition:commutative_ring}{commutative ring}, then a left/right $R$-module can automatically be regarded as a two-sided $R$-module. As such, we simply talk about \hldef{$R$-modules} in this case. 

Any abelian group is equivalent to a two-sided $\bbZ$-module. Moreover, any left $R$-module is equivalent to an \CrefAndHyperrefIfExist{definition:module_of_a_ring}{$R-\bbZ$-bimodule} and any right $R$-module is equivalent to an \CrefAndHyperrefIfExist{definition:module_of_a_ring}{$\bbZ-R$-bimodule}. Given a left/right/two-sided $R$-module, its \hldef{natural bimodule structure} will refer to its structure as a $R$-$\bbZ$/$\bbZ$-$R$/$R$-$R$ bimodule. In this way, many definitions associated with the notions of left/right/two-sided $R$-modules can be defined as special cases for definitions for $R$-$S$-bimodules.
\end{definition}

\begin{definition} \label{definition:homomorphism_of_modules_over_a_ring}
Let $R,S$ be \CrefAndHyperrefIfExist{definition:ring}{(not-necessarily commutative) rings}. 
\begin{enumerate}
    \item Let $M$ and $N$ be \CrefAndHyperrefIfExist{definition:module_of_a_ring}{$R$-$S$-bimodules}. A function $\varphi: M \to N$ is called an \hldef{$R$-$S$-bimodule homomorphism} or \hldef{$R$-$S$-linear} if it is a \CrefAndHyperrefIfExist{definition:group_homomorphism}{group homomorphism} of the underlying abelian groups of $M$ and $N$ and respects the scalar actions as follows: 
    for all $m_1,m_2 \in M$, $r \in R$, and $s \in S$,
        \begin{align*}
        % \varphi(m_1 + m_2) &= \varphi(m_1) + \varphi(m_2), \\
        \varphi(r \cdot m_1) &= r \cdot \varphi(m_1), \\
        \varphi(m_1 \cdot s) &= \varphi(m_1) \cdot s.
        \end{align*}

    \item Let $M$ and $N$ be \CrefAndHyperrefIfExist{definition:module_of_a_ring}{left/right/two-sided $R$-modules}. A function $\varphi: M \to N$ is called a \hldef{left/right/two-sided $R$-module homomorphism} if it is an bimodule homomorphism on the \CrefAndHyperrefIfExist{definition:module_of_a_ring}{natural bimodule structures} of $M$ and $N$.
    %  $R$-$\bbZ$/$\bbZ$-$R$/$R$-$R$-bimodule homomorphism. 
     Such a function is also called \hldef{$R$-linear}.

\end{enumerate}

Modules and homomorphisms of a fixed type (i.e. $R$-$S$-bimodules or left/righ/two-sided $R$-modules) form a \CrefAndHyperrefIfExist{definition:locally_small_category}{locally small} \CrefAndHyperrefIfExist{definition:category}{category}.

% Let $M$ and $N$ be \CrefAndHyperrefIfExist{definition:module_of_a_ring}{left/right/two-sided $R$-modules or $R$-$S$-bidmodules}. 

% \begin{enumerate}
%     \item A function $\varphi : M \to N$ is called a \hldef{left/right/two-sided module homomorphism} or \hldef{$R$-linear} if it is additive (more precisely, a \CrefAndHyperrefIfExist{definition:group_homomorphism}{group homomorphism} of \CrefAndHyperrefIfExist{definition:group}{abelian groups}) and respects the scalar action(s) as follows: for all $m_1,m_2 \in M$, $r \in R$, and $s \in S$,
%     \begin{align*}
%     % \varphi(m_1 + m_2) &= \varphi(m_1) + \varphi(m_2), \\
%     \varphi(r \cdot m_1) &= r \cdot \varphi(m_1), \\
%     \varphi(m_1 \cdot s) &= \varphi(m_1) \cdot s.
%     \end{align*}

%     \item 
% \end{enumerate}

% \begin{enumerate}
%     \item If $M$ and $N$ are left $R$-modules, then for all $m_1,m_2 \in M$ and $r \in R$,
%     \begin{align*}
%     \varphi(m_1 + m_2) &= \varphi(m_1) + \varphi(m_2), \\
%     \varphi(r \cdot m_1) &= r \cdot \varphi(m_1).
%     \end{align*}

%     \item If $M$ and $N$ are right $R$-modules, then for all $m_1,m_2 \in M$ and $r \in R$,
%     \begin{align*}
%     \varphi(m_1 + m_2) &= \varphi(m_1) + \varphi(m_2), \\
%     \varphi(m_1 \cdot r) &= \varphi(m_1) \cdot r.
%     \end{align*}

%     \item If $M$ and $N$ are two-sided $R$-modules, then for all $m_1,m_2 \in M$, and $r_1,r_2 \in R$,
%     \begin{align*}
%     \varphi(m_1 + m_2) &= \varphi(m_1) + \varphi(m_2), \\
%     \varphi(r_1 \cdot m_1) &= r_1 \cdot \varphi(m_1), \\
%     \varphi(m_1 \cdot r_2) &= \varphi(m_1) \cdot r_2.
%     \end{align*}

%     \item If $M$ and $N$ are $(R,S)$-bimodules, then for all $m_1,m_2 \in M$, $r \in R$, and $s \in S$,
%     \begin{align*}
%     \varphi(m_1 + m_2) &= \varphi(m_1) + \varphi(m_2), \\
%     \varphi(r \cdot m_1) &= r \cdot \varphi(m_1), \\
%     \varphi(m_1 \cdot s) &= \varphi(m_1) \cdot s.
%     \end{align*}
% \end{enumerate}
\end{definition}
\begin{definition} \label{definition:category_of_modules_and_bimodules_over_rings}
    Let $R$ and $S$ be \CrefAndHyperrefIfExist{definition:ring}{(not necessarily commutative) rings}.     
    \begin{enumerate}
        \item The \hldef{category of $(R,S)$-bimodules} (or $R$-$S$-bimodules), denoted by notations such as \hl{${}_R\mathsf{Mod}_S$}, is the category whose objects are \CrefAndHyperrefIfExist{definition:module_of_a_ring}{$(R,S)$-bimodules} and whose \CrefAndHyperrefIfExist{definition:homomorphism_of_modules_over_a_ring}{$R$-$S$-bimodule homomorphisms}.

        \item The \hldef{category of left $R$-modules}, denoted by notations such as \hl{${}_R\mathsf{Mod}$} or \hl{$R-\mathbf{Mod}$}, is the category ${}_R\mathsf{Mod}_\bbZ$, i.e. the category whose objects are \CrefAndHyperrefIfExist{definition:module_of_a_ring}{left $R$-modules} and whose morphisms are \CrefAndHyperrefIfExist{definition:homomorphism_of_modules_over_a_ring}{left $R$-linear maps}.

        \item The \hldef{category of right $R$-modules}, denoted by notations such as \hl{$\mathsf{Mod}_R$} or \hl{$\mathbf{Mod}-R$}, is the category ${}_\bbZ\mathsf{Mod}_R$, i.e. the category whose objects are \CrefAndHyperrefIfExist{definition:module_of_a_ring}{right $R$-modules} and whose morphisms are \CrefAndHyperrefIfExist{definition:homomorphism_of_modules_over_a_ring}{right $R$-linear maps}.
    \end{enumerate}

    The category of bimodules can be canonically identified with module categories over \CrefAndHyperrefIfExist{definition:tensor_product_of_a_ring_and_an_algebra_over_a_ring}{tensor product rings}:
    \begin{itemize}
        \item ${}_R\mathsf{Mod}_S$ is isomorphic to the category of left modules over the ring $R \otimes_{\mathbb{Z}} S^{\operatorname{op}}$.
        \item ${}_R\mathsf{Mod}_S$ is isomorphic to the category of right modules over the ring $R^{\operatorname{op}} \otimes_{\mathbb{Z}} S$.
    \end{itemize}
    Consequently, standard module-theoretic concepts (such as projective objects, injective objects, and flat objects) in ${}_R\mathsf{Mod}_S$ correspond exactly to the respective concepts in ${}_{R \otimes S^{\operatorname{op}}}\mathsf{Mod}$.

    % \paragraph{Relation to One-Sided Modules.}
    Note that there are canonical isomorphisms of categories:
    \[
        {}_R\mathsf{Mod} \cong {}_R\mathsf{Mod}_{\mathbb{Z}} \quad \text{and} \quad \mathsf{Mod}_S \cong {}_{\mathbb{Z}}\mathsf{Mod}_S.
    \]
    That is, left $R$-modules are exactly $(R, \mathbb{Z})$-bimodules, and right $S$-modules are exactly $(\mathbb{Z}, S)$-bimodules.
\end{definition}


\begin{definition} \label{definition:submodule_of_a_module_over_a_ring}
Let $R,S$ be \CrefAndHyperrefIfExist{definition:ring}{not-necessarily commutative rings}. 
\begin{enumerate}
    \item Let $M$ be an \CrefAndHyperrefIfExist{module}{$R$-$S$-bimodule} whose \CrefAndHyperrefIfExist{definition:group}{abelian group} structure is given by the operator $+$. An \hldef{$R$-$S$-submodule of $M$} is a \CrefAndHyperrefIfExist{definition:subgroup_of_a_group}{subgroup} $N \subseteq (M,+)$ if for all $r \in R$, $s \in S$, and $n \in N$, we have $rn \in N$ and $ns \in N$; in this case, $N$ inherits an $R$-$S$ bimodule structure from $M$. 

    \item If $M$ is a \CrefAndHyperrefIfExist{module}{left/right/two-sided $R$-module}, then a \hldef{left/right/two-sided $R$-submodule of $M$} is a submodule of the \CrefAndHyperrefIfExist{definition:module_of_a_ring}{natural bimodule structure} of $M$.
    % $R$-$\bbZ$/$\bbZ$-$R$/$R$-$R$-submodule of $M$. 
\end{enumerate}

\TextIfExists{definition:subobject_of_an_object_of_an_additive_category}{A submodule of $M$ is a \CrefAndHyperrefIfExist{definition:subobject_of_an_object_of_an_additive_category}{categorical subobject} of $M$ in the appropriate category of modules.}

% and let $M$ be an \CrefAndHyperrefIfExist{module}{left/right/two-sided $R$-module or an $R$-$S$-bimodule}.
% In those respective cases, a \CrefAndHyperrefIfExist{definition:subgroup_of_a_group}{subgroup} $N \subseteq (M,+)$ is called a \hldef{left/right/two-sided $R$-submodule of $M$} or an \hldef{$R$-$S$-submodule of $M$} if for all $r \in R$, $s \in S$, and $n \in N$:
% \begin{itemize}
%     \item In the left $R$-module case: $rn \in N$,
%     \item In the right $R$-module case: $nr \in N$,
%     \item In the two-sided $R$-module case: both $rn \in N$ and $nr \in N$.
%     \item In the $R$-$S$-bimodule case $rn \in N$ and $ns \in N$. 
% \end{itemize}
% In particular, a two-sided $R$-module is 
\end{definition}

\begin{definition} \label{definition:quotient_module_of_a_module_by_a_module_of_a_ring}
\TODO{define coset, kernel of R-module homomorphism}
Let $R,S$ be \CrefAndHyperrefIfExist{definition:ring}{(not necessarily commutative) rings}. 
\begin{enumerate}
    \item Let $M$ be an \CrefAndHyperrefIfExist{definition:module_of_a_ring}{$R$-$S$-bimodule}. Let $N \subseteq M$ be a \CrefAndHyperrefIfExist{definition:submodule_of_a_module_over_a_ring}{submodule of $M$}.  

    The \CrefAndHyperrefIfExist{definition:quotient_of_a_group_by_a_normal_subgroup}{quotient group $M/N$}, which is well defined as $M$ is an \CrefAndHyperrefIfExist{definition:group}{abelian group} and hence $N$ is a \CrefAndHyperrefIfExist{definition:normal_subgroup_of_a_group}{normal subgroup}\CrefIfExists{proposition:subgroup_of_a_group_is_normal_subgroup_of_normalizer}, has the structure of an $R$-$S$-bimodule --- the (abelian) group structure is simply the group structure of $M/N$, whereas the $R$-$S$-bimodule structure is given as follows: for $m \in M$, $r \in R$, $s \in S$, we have
    $$r \cdot (m + N) \cdot s = r \cdot m \cdot s + N.$$

    This $R$-$S$-bidmodule structure on $M/N$ is called the \hldef{quotient $R$-$S$-bidmodule of $M$ by $N$} and is also denoted as \hl{$M/N$}.
    
    The canonical projection map
    $$\pi: M \to M/N, \quad m \mapsto m+N,$$
    is a surjective \CrefAndHyperrefIfExist{definition:homomorphism_of_modules_over_a_ring}{$R$-module homomorphism} with kernel $N$.


    \item Let $M$ be a left/right/two-sided $R$-module. Let $N \subseteq M$ be a submodule of $M$. The \hldef{quotient $R$-module} \hl{$M/N$} is the quotient of $M$ by $N$ for their respective \CrefAndHyperrefIfExist{definition:module_of_a_ring}{natural bimodule structures}.
\end{enumerate}

    \TextIfExists{definition:quotient_object_of_an_object_of_an_abelian_category_by_a_subobject}{The quotient $M/N$ is the \CrefAndHyperrefIfExist{definition:quotient_object_of_an_object_of_an_abelian_category_by_a_subobject}{categorical quotient object} of $M$ by the \CrefAndHyperrefIfExist{definition:subobject_of_an_object_of_an_additive_category}{subobject} $N$ in the appropriate category of modules}

\end{definition}
\begin{definition}[Submodule generated by elements in an $(R,S)$-bimodule] \label{definition:submodule_of_a_module_generated_by_elements}
    Let $R$ and $S$ be \CrefAndHyperrefIfExist{definition:ring}{(not necessarily commutative) rings}. 
    \begin{enumerate}
        \item 
        Let $M$ be an \CrefAndHyperrefIfExist{definition:module_of_a_ring}{$(R,S)$-bimodule}.

        Given a subset $X \subseteq M$, the \hldef{sub-bimodule of $M$ generated by $X$} is the smallest $(R,S)$-sub-bimodule of $M$ containing $X$. It is often denoted by notations such as \hl{$\langle X \rangle = \langle X \rangle_{R,S}$} and is more explicitly the intersection
        $$\langle X \rangle_{R,S} = \bigcap_{X \subseteq T \subseteq M, T \text{ is a } (R,S)\text{-submodule of } M} T$$
        of al $(R,S)$-submodules of $M$ containing $X$. 
        
        Equivalently, $\langle X \rangle_{R,S}$ consists of all \CrefAndHyperrefIfExist{definition:linear_combination_of_elements_in_a_module}{linear combinations} of $X$. 

        \item If $M$ is a left/right/two-sided $R$-module and given a subset $X \subseteq M$, the \hldef{submodule of $M$ generated by $X$} is the submodule of the \CrefAndHyperrefIfExist{definition:module_of_a_ring}{natural bimodule} of $M$ generated by $X$. It is denoted by notations such as $\langle X \rangle = \langle X \rangle_R$. 
    \end{enumerate}
    % Explicitly, this sub-bimodule consists of all finite sums of elements of the form
    % \[
    % \hl{$r \cdot x \cdot s$}
    % \]
    % where $r \in R$, $s \in S$, and $x \in X$. That is,
    % \[
    % \hl{$\langle X \rangle_{R,S} = \left\{ \sum_{i=1}^n r_i x_i s_i \mid n \in \mathbb{N}, r_i \in R, s_i \in S, x_i \in X \right\}$}.
    % \]
    % This sub-bimodule is the smallest $(R,S)$-bimodule containing $X$, closed under the actions of $R$ and $S$ and addition.
\end{definition}


% \begin{definition} \label{definition:kernel_image_cokernel_of_a_module_homomorphism_over_a_ring}
% Let $R,S$ be \CrefAndHyperrefIfExist{definition:ring}{(not-necessarily commutative) rings with unity}, and let $M,N$ be \CrefAndHyperrefIfExist{definition:module_of_a_ring}{$R$-$S$-bimodules}. Let 
% $$\varphi : M \to N$$ 
% be a \CrefAndHyperrefIfExist{definition:homomorphism_of_modules_over_a_ring}{homomorphism of $R$-$S$-bimodules}. We define:

% \begin{enumerate}
%     \item The \hldef{kernel of $\varphi$} is the \CrefAndHyperrefIfExist{definition:submodule_of_a_module_over_a_ring}{submodule of $M$} given by
%     $$\hlin{\ker(\varphi) := \{ m \in M \mid \varphi(m) = 0 \} \subseteq M.}$$

%     \item The \hldef{image of $\varphi$} is the submodule of $N$ given by
%     $$\hlin{\operatorname{im}(\varphi) := \{ \varphi(m) \mid m \in M \} \subseteq N.}$$

%     \item The \hldef{cokernel of $\varphi$} is the \CrefAndHyperrefIfExist{definition:quotient_module_of_a_module_by_a_module_of_a_ring}{quotient $R$-module of $N$} defined by
%     $$\hlin{\operatorname{coker}(\varphi) := N / \operatorname{im}(\varphi).}$$


% \end{enumerate}
% It is not difficult to see that each of these are indeed $R$-$S$ bimodules. In case $M$ and $N$ are left/right/two-sided $R$-modules, the \hldef{kernel, image, and cokernel} of a module homomorphism $\varphi: M \to N$ are respectively defined to be the kernel, image, and cokernel for the \CrefAndHyperrefIfExist{definition:module_of_a_ring}{natural bimodule structures} of $M$ and $N$.
% \TODO{describe how these are categorical kernels/images/cokernels}
% \end{definition}

\begin{definition} \label{definition:kernel_image_cokernel_coimage_of_a_module_homomorphism}
Let $R,S$ be \CrefAndHyperrefIfExist{definition:ring}{(not-necessarily commutative) rings with unity}, and let $M,N$ be \CrefAndHyperrefIfExist{definition:module_of_a_ring}{$R$-$S$-bimodules}. Let 
$$\varphi : M \to N$$ 
be a \CrefAndHyperrefIfExist{definition:homomorphism_of_modules_over_a_ring}{homomorphism of $R$-$S$-bimodules}. We define:

\begin{enumerate}
    \item The \hldef{kernel of $\varphi$} is the \CrefAndHyperrefIfExist{definition:submodule_of_a_module_over_a_ring}{submodule of $M$} given by
    $$\hlin{\ker(\varphi) := \{ m \in M \mid \varphi(m) = 0 \} \subseteq M.}$$

    \item The \hldef{image of $\varphi$} is the submodule of $N$ given by
    $$\hlin{\operatorname{im}(\varphi) := \{ \varphi(m) \mid m \in M \} \subseteq N.}$$

    \item The \hldef{cokernel of $\varphi$} is the \CrefAndHyperrefIfExist{definition:quotient_module_of_a_module_by_a_module_of_a_ring}{quotient module of $N$} defined by
    $$\hlin{\operatorname{coker}(\varphi) := N / \operatorname{im}(\varphi).}$$

    \item The \hldef{coimage of $\varphi$} is the \CrefAndHyperrefIfExist{definition:quotient_module_of_a_module_by_a_module_of_a_ring}{quotient module of $M$} defined by
    $$\hlin{\operatorname{coim}(\varphi) := M / \ker(\varphi).}$$
\end{enumerate}
It is not difficult to see that each of these are indeed $R$-$S$ bimodules. In case $M$ and $N$ are left/right/two-sided $R$-modules, the \hldef{kernel, image, cokernel, and coimage} of a module homomorphism $\varphi: M \to N$ are respectively defined to be the kernel, image, cokernel, and coimage for the \CrefAndHyperrefIfExist{definition:module_of_a_ring}{natural bimodule structures} of $M$ and $N$.

The kerel, cokernel, image, and coimage of $f$ are respectively the categorical \CrefAndHyperrefIfExist{definition:kernel_and_cokernel_of_a_morphism_in_a_category}{kernel, cokernel}, \CrefAndHyperrefIfExist{definition:image_coimage_of_a_morphism_in_a_category}{image, and coimage} (\Cref{lemma:kernel_cokernel_image_coimage_of_modules_over_rings_are_categorical}).

\end{definition}


Two fundamental functors on categories of modules are given by \CrefAndHyperrefIfExist{definition:hom_of_left_right_bi_modules_of_rings}{Hom's} and \CrefAndHyperrefIfExist{definition:tensor_product_of_bimodules_of_rings}{tensor products}

\begin{definition}[Hom of left/right/bi-modules] \label{definition:hom_of_left_right_bi_modules_of_rings}
Let $R,S,T$ be \CrefAndHyperrefIfExist{definition:ring}{(not necessarily commutative) rings}.
\begin{enumerate}
    \item Let $M$ and $N$ be \CrefAndHyperrefIfExist{definition:module_of_a_ring}{left $R$-modules}. The \hldef{homomorphism group of left $R$-modules from $M$ to $N$} is the abelian group
    $$\hlin{\Hom(M,N) = \mathrm{Hom}_R(M, N) := \{ f : M \to N \mid f \text{ is a left $R$-module homomorphism} \}.} $$
    \CrefIfExists{definition:homomorphism_of_modules_over_a_ring}

    \item Let $M$ and $N$ be \CrefAndHyperrefIfExist{definition:module_of_a_ring}{right $R$-modules}. The \hldef{homomorphism group of right $R$-modules from $M$ to $N$} is the abelian group
    $$\hlin{ \Hom(M,N) = \mathrm{Hom}_R(M, N) := \{ f : M \to N \mid f \text{ is a right $R$-module homomorphism} \}.}$$

    \item Let $S$ be a (not necessarily commutative ring) and let $M$ and $N$ be \CrefAndHyperrefIfExist{definition:module_of_a_ring}{$R-S$-bimodules}. The \hldef{homomorphism group of $R$-$S$-bimodules from $M$ to $N$} is the abelian group 
    $$\hlin{\Hom(M,N) = \Hom_{R-S}(M,N) \coloneq \{f: M \to N | f \text{ is a } R-S\text{-bimodule homomorphism} \}}$$
    \CrefIfExists{definition:hom_between_bimodules}
    \end{enumerate}

    In each case, $\Hom(M,N)$ has a natural structure of an \hldef{abelian group} given by \hldef{pointwise addition}: for $f, g \in \mathrm{Hom}(M, N)$,
    $$ (f + g)(m) := f(m) + g(m), $$
    and the zero morphism \hl{$0$} given by $0(m) := 0_N$ acts as the identity element.
    The additive inverse \hl{$-f$} is defined by $(-f)(m) := -f(m)$. Moreover, depending on bi-module structures that $M$ and $N$ may be carrying, $\Hom(M,N)$ may itself carry additional module structures:
    \begin{itemize}
        \item In case that $M$ is a \CrefAndHyperrefIfExist{definition:hom_between_bimodules}{$R-S$-bimodule} and $N$ is a $R-T$-bimodule, $\mathrm{Hom}_R(M, N)$, the group of left $R$-module homomorphisms, is an $S-T$-bimodule as follows:
        $$(s\cdot f\cdot t)(m) = f(m \cdot s) \cdot t \quad f \in \Hom_R(M,N), s \in S, t \in T.$$

        \item Dually, in case that $M$ is a \CrefAndHyperrefIfExist{definition:hom_between_bimodules}{$S-R$-bimodule} and $N$ is a $T-R$-bimodule, $\mathrm{Hom}_R(M, N)$, the group of right $R$-module homomorphisms, is an $S-T$-bimodule as follows:
        $$(s\cdot f\cdot t)(m) = f(s \cdot m) \cdot t \quad f \in \Hom_R(M,N), s \in S, t \in T.$$
    \end{itemize}
    Some cases of interest may be when $R$, $S$, or $T$ is in fact $\bbZ$ --- these allow us to see module structures on $\Hom(M,N)$ even when $M$ and $N$ are one-sided modules.

    \TODO{state this as a theorem}
    We furthermore note that $\Hom_R(-,-)$ yields \CrefAndHyperrefIfExist{definition:n_ary_additive_functor_between_additive_categories}{biadditive functors}
    $$\Hom_R(-,-): {}_R \mathbf{Mod}_{S}^{\op} \times {}_{R} \mathbf{Mod}_{T} \to {}_{S} \mathbf{Mod}_{T}$$
    $$\Hom_R(-,-): {}_S \mathbf{Mod}_{R}^{\op} \times {}_{T} \mathbf{Mod}_{R} \to {}_{S} \mathbf{Mod}_{T}.$$
    \CrefIfExists{definition:opposite_category_of_a_category}\CrefIfExists{definition:category_of_modules_and_bimodules_over_rings} \CrefIfExists{theorem:the_category_of_R_S_bimodules_is_a_grothendieck_abelian_category_and_AB4_star}
\end{definition}
\begin{definition}[Tensor product of bimodules] \label{definition:tensor_product_of_bimodules_of_rings}
Let $R,S,T$ be \CrefAndHyperrefIfExist{definition:ring}{(not necessarily commutative) rings}, let $M$ be an \CrefAndHyperrefIfExist{definition:module_of_a_ring}{$R$-$S$ bimodule}, and let $N$ be an $S$-$T$ bimodule. In the \CrefAndHyperrefIfExist{definition:free_abelian_group_generated_by_a_set}{free abelian group} $\bbZ[M \times N]$ generated by the \CrefAndHyperrefIfExist{definition:product_of_sets}{Cartesian product $M \times N$}, let $U$ be the subgroup generated by elements of the form
\TODO{subgroup generated}
\begin{align*}
&(m+m',n) - (m,n) - (m',n),\\
&(m,n+n') - (m,n) - (m,n'),\\
&(m \cdot s, n) - (m, s \cdot n),
\end{align*}
for all $m,m' \in M$, $n,n' \in N$, and $s \in S$. The \hldef{tensor product of $M$ and $N$ over $S$} is the \CrefAndHyperrefIfExist{definition:quotient_of_a_group_by_a_normal_subgroup}{quotient} abelian group
$$M \otimes_S N := \mathbb{Z}[M \times N] / U.$$
The image of an element of the form $(m,n) \in M \times N$ in $M \otimes_S N$ is denoted \hl{$m \otimes n$} and called a \hldef{pure tensor}. In general, the elements of $M \otimes_S N$ are finite sums 
$$\sum_{i=1}^n m_i \otimes n_i \quad m_i \in M, n_i \in N$$
of pure tensors. Thus, the pure tensors satisfy the following relations:
\begin{align*}
    (m + m') \otimes n &= m \otimes n + m' \otimes n \\ 
    m \otimes (n + n') &= m \otimes n + m \otimes n' \\
    (m \cdot s) \otimes n &= m \otimes (s \cdot n)
\end{align*}

This tensor product becomes naturally an $R$-$T$ bimodule with left action and right action defined by
\begin{align*}
r \cdot (m \otimes n) &= (r \cdot m) \otimes n, \\
(m \otimes n) \cdot t &= m \otimes (n \cdot t),
\end{align*}
for all $r \in R$, $t \in T$, $m \in M$, and $n \in N$.

Inductively, given rings $R_0,\ldots,R_k$ and $R_{i-1}-R_i$-bimodules $M_i$ for $i = 1,\ldots,k$, we may speak of the tensor product
$$M_0 \otimes_{R_1} M_1 \otimes_{R_2} \cdots \otimes_{R_{k-1}} M_k;$$
tensor products are associative\TODO{}, so parentheses are not strictly needed to notate them. Its \hldef{pure tensors} are elements of the form $m_0 \otimes m_1 \otimes \cdots \otimes m_k$ for $m_i \in M_i$, and its general elements are finite sums
$$\sum_{j=1}^n m_{0j} \otimes m_{1j} \otimes \cdots m_{kj} \quad m_{ij} \in M_i.$$
of pure tensors. It also has a natural $R_0-R_k$-bimodule structure.

\TextIfExists{definition:n_ary_additive_functor_between_additive_categories}{In general, $(M_0,\ldots,M_k) \mapsto M_0 \otimes_{R_1} M_1 \otimes_{R_2} \cdots \otimes_{R_{k-1}} M_k$ defines a \CrefAndHyperrefIfExist{definition:n_ary_additive_functor_between_additive_categories}{$(k+1)$-ary additive functor}
$${}_{R_0}\mathbf{Mod}_{R_1} \times \cdots \times {}_{R_{k-1}}\mathbf{Mod}_{R_k} \to {}_{R_0} \mathbf{Mod}_{R_k}$$
(\Cref{theorem:the_category_of_R_S_bimodules_is_a_grothendieck_abelian_category_and_AB4_star}).}


Given a ring $R$ and a two-sided $R$-module $M$, we may also speak of the \hldef{$n$-fold tensor product} \hl{$M^{\otimes n} = M^{\otimes_R n}$}

\end{definition}
\begin{definition} \label{definition:tensor_product_of_a_ring_and_an_algebra_over_a_ring}
    Let $k$ be a \CrefAndHyperrefIfExist{definition:ring}{not necessarily commutative ring}. Let $R$ and $S$ be \CrefAndHyperrefIfExist{definition:ring_homomorphism}{$k$-rings} (not necessarily commutative). Assume that at least one of $R$ or $S$ is a \CrefAndHyperrefIfExist{definition:algebra_of_a_ring}{$k$-algebra}. The \hldef{tensor product ring} \hl{$R \otimes_k S$} is the $k$-module \CrefAndHyperrefIfExist{definition:tensor_product_of_bimodules_of_rings}{$R \otimes_k S$} equipped with a multiplication defined on simple tensors by
    \[
        (r_1 \otimes s_1) \cdot (r_2 \otimes s_2) = (r_1 r_2) \otimes (s_1 s_2)
    \]
    and extended by linearity. This multiplication is well-defined and makes $R \otimes_k S$ into a $k$-ring under the ring homomorphism 
    $$k \to R \otimes_k S, \quad a \mapsto a \otimes 1 = 1 \otimes a.$$
    The unit element is $1_R \otimes 1_S$.

    In this ring, the \CrefAndHyperrefIfExist{definition:subring_of_a_ring}{subrings} $R \otimes 1$ and $1 \otimes S$ commute with each other; that is, for all $r \in R$ and $s \in S$,
    \[
        (r \otimes 1) \cdot (1 \otimes s) = r \otimes s = (1 \otimes s) \cdot (r \otimes 1).
    \]

    If $R$ and $S$ are both $k$-algebras, then $R \otimes_k S$ is also a $k$-algebra.
\end{definition}
\begin{proposition}[Universal Property of the Tensor Product of Bimodules] \label{proposition:universal_property_of_the_tensor_product_of_bimodules}
Let $R,S,T$ be \CrefAndHyperrefIfExist{definition:ring}{(not necessarily commutative) rings}. Let $M$ be an \CrefAndHyperrefIfExist{definition:module_of_a_ring}{$R$-$S$ bimodule} and let $N$ be an $S$-$T$ bimodule. Let $P$ be an $R$-$T$ bimodule. Then for every \CrefAndHyperrefIfExist{definition:multilinear_map_of_modules_over_rings}{$R$-$T$ bilinear map} 
$$ \beta : M \times N \to P, $$
that is, a map satisfying
\begin{align*}
\beta(m+m',n) &= \beta(m,n) + \beta(m',n), \\
\beta(m,n+n') &= \beta(m,n) + \beta(m,n'), \\
\beta(r \cdot m, n) &= r \cdot \beta(m,n), \\
\beta(m,n\cdot t) &= \beta(m,n)\cdot t, \\
\beta(m \cdot s, n) &= \beta(m,s \cdot n),
\end{align*}
for all $m,m' \in M$, $n,n' \in N$, $r \in R$, $s \in S$, $t \in T$, there exists a unique $R$-$T$ bimodule homomorphism
$$ \widetilde{\beta} : M \otimes_S N \to P $$
such that $\widetilde{\beta}(m \otimes n) = \beta(m,n)$ for all $m \in M$, $n \in N$.
\end{proposition}

\begin{definition} \label{definition:dual_of_a_left_right_two_sided_module}
Let $R$ be a \CrefAndHyperrefIfExist{definition:ring}{(not necessarily commutative) ring}. Depending on the module structure of $M$, we define its dual module as follows:
\begin{enumerate}
    \item If $M$ is a \CrefAndHyperrefIfExist{definition:module_of_a_ring}{left $R$-module}, then the \hldef{(right) dual module of $M$} is 
    $$\hlin{M^* = M^\vee \coloneq \operatorname{Hom}_R(M, R)}.$$
    \CrefIfExists{definition:hom_of_left_right_bi_modules_of_rings}
    Note that it is a right $R$-module, as $M$ is a $R-\bbZ$-bimodule and $R$ is an $R-R$-bimodule. 

    \item If $M$ is a \CrefAndHyperrefIfExist{definition:module_of_a_ring}{right $R$-module}, then the \hldef{(left) dual module of $M$} is 
    $$\hlin{{}^* M = {}^\vee M \coloneq \operatorname{Hom}_R(M, R)}.$$
    \CrefIfExists{definition:hom_of_left_right_bi_modules_of_rings}
    Note that it is a left $R$-module, as $M$ is a $\bbZ-R$-bimodule and $R$ is an $R-R$-bimodule. 

    % \item If $M$ is a \CrefAndHyperrefIfExist{definition:module_of_a_ring}{right $R$-module}, the set of all left $R$-linear maps from $M$ to $R$ is denoted by
    % $$\hlin{\,{}^*M = \operatorname{Hom}_R(M, R)},$$
    % and is called the \hldef{left dual module} of $M$.

    \item If $M$ is a \CrefAndHyperrefIfExist{definition:two_sided_module_of_a_ring}{two-sided $R$-module}, then the \hldef{dual of $M$} usually refers to either the right or the left dual as above.
    % then $M$ admits both left and right $R$-actions, and we may define two duals:
    % \begin{align*}
    %     M^* &= \operatorname{Hom}_{R\text{-left}}(M, R) \quad \text{(right dual module)},\\
    %     {}^*M &= \operatorname{Hom}_{\text{right-}R}(M, R) \quad \text{(left dual module)}.
    % \end{align*}
    % In particular, $M^*$ becomes a right $R$-module, and ${}^*M$ becomes a left $R$-module, via pointwise scalar multiplication:
    % $$ (f \cdot r)(m) = f(m)r, \quad (r \cdot f)(m) = rf(m). $$
\end{enumerate}
In any case, the functor $M \mapsto M^\vee$ is a \CrefAndHyperrefIfExist{definition:functor_between_categories}{contravariant functor} from the appropriate \CrefAndHyperrefIfExist{definition:category_of_modules_and_bimodules_over_rings}{category of modules} to itself.

If $R$ is a \CrefAndHyperrefIfExist{definition:field}{field} $F$ and $V$ is an \CrefAndHyperrefIfExist{definition:vector_space_over_a_field}{$F$-vector space}, then the dual module
$$\hlin{V^* = V^\vee \coloneq  \operatorname{Hom}_F(V, F)}$$
is called the \hldef{dual vector space of $V$}.
\end{definition}

\subsubsection{Extension/Restriction/co-extension of scalars}

\begin{definition}  \label{definition:extension_of_scalars_for_modules_along_ring_homomorphisms}
Let $R$ and $S$ be \CrefAndHyperrefIfExist{definition:ring}{rings} (not necessarily commutative), and let $f: R \to S$ be a \CrefAndHyperrefIfExist{definition:ring_homomorphism}{ring homomorphism}. This homomorphism gives $S$ the structure of an \CrefAndHyperrefIfExist{definition:module_of_a_ring}{$(R, R)$-bimodule} via \CrefAndHyperrefIfExist{definition:restriction_of_scalars_of_a_left_or_right_module_along_a_ring_homomorphism_from_the_base_ring}{restriction of scalars}. Let $T$ be a \CrefAndHyperrefIfExist{definition:ring}{ring}.

\begin{enumerate}
    \item 
    The \hldef{extension of scalars} (or \hldef{base change}) along $f$ is defined separately for modules:
    \begin{itemize}
        \item For an $(R-T)$-bimodule $M$, the extension of scalars of $M$ along $f$ is the $(S-T)$-bimodule \CrefAndHyperrefIfExist{definition:tensor_product_of_bimodules_of_rings}{$S \otimes_R M$} where $S$ is viewed as an $(S, R)$-bimodule. In particular, the action of $s' \in S$ on a \CrefAndHyperrefIfExist{definition:tensor_product_of_bimodules_of_rings}{simple tensor} $s \otimes m$ is given by $s' \cdot (s \otimes m) = (s's) \otimes m$.
        
        \item For a $(T-R)$-bimodule $M$, the extension of scalars of $M$ along $f$ is the $(T-S)$-bimodule $M \otimes_R S$ where $S$ is viewed as an $(R, S)$-bimodule. In particular, the action of $s' \in S$ on a simple tensor $m \otimes s$ is given by $(m \otimes s) \cdot s' = m \otimes (ss')$.
    \end{itemize}

    \item The \hldef{base change functor} (or \hldef{extension of scalars functor} or \hldef{induction functor}), denoted by \hl{$f^*$}, \hl{$S \otimes_R -$}, \hl{$- \otimes_R S$}, or \hl{$\operatorname{Ind}_R^S$}, is given by:
    \begin{itemize}
        \item For left $R$-modules:
        $$f^*: {}_R\mathsf{Mod}_T \to {}_S\mathsf{Mod}_T, \quad M \mapsto S \otimes_R M.$$
        \CrefIfExists{definition:category_of_modules_and_bimodules_over_rings} This is the \CrefAndHyperrefIfExist{definition:adjoint_functors_between_categories_unit_counit_of_adjoint_functors}{left adjoint} to the \CrefAndHyperrefIfExist{definition:restriction_of_scalars_of_a_left_or_right_module_along_a_ring_homomorphism_from_the_base_ring}{restriction of scalars} functor $f_*: {}_S\mathsf{Mod}_{T} \to {}_R\mathsf{Mod}_{T}$.
        
        \item For right $R$-modules:
        $$f^*: {}_T \mathsf{Mod}_R \to {}_T \mathsf{Mod}_S, \quad M \mapsto M \otimes_R S.$$
        This is the left adjoint to the restriction of scalars functor $f_*: {}_T \mathsf{Mod}_S \to {}_T \mathsf{Mod}_R$.
    \end{itemize}
    
    \item Let $A$ be an \CrefAndHyperrefIfExist{definition:ring_homomorphism}{$R$-ring}, i.e. a ring equipped with a ring homomorphism $R \to A$. Assume that $S$ or $A$ is an \CrefAndHyperrefIfExist{definition:algebra_of_a_ring}{$R$-algebra}.
    
    The \hldef{base change of the algebra $A$ along $f$} is the $S$-ring defined as
    $$\hlin{ A_S := S \otimes_R A }$$
    \CrefIfExists{definition:tensor_product_of_a_ring_and_an_algebra_over_a_ring}
    equipped with the natural homomorphism $S \to S \otimes_R A$ given by $s \mapsto s \otimes 1_A$. As a ring, the multiplication in $A_S$ is determined by $(s_1 \otimes a_1)(s_2 \otimes a_2) = (s_1 s_2) \otimes (a_1 a_2)$ for $s_1, s_2 \in S$ and $a_1, a_2 \in A$.

        \item 
        Then the base change construction induces functors in the following situations:
        \begin{enumerate}
            \item If $S$ is only an $R$-ring, then base change induces a functor
            \hl{$f^*: \mathbf{Alg}_R \to \mathbf{Ring}_S$}.
            \item If $S$ is an $R$-algebra, then base change induces a functor
            \hl{$f^*: \mathbf{Ring}_R \to \mathbf{Ring}_S$}
            which restricts to a functor $f^*: \mathbf{Alg}_R \to \mathbf{Alg}_S$
        \end{enumerate}
        In either case, the base change functor is defined as follows:
        \begin{itemize}
            \item On objects: For any $R$-algebra $(A, \varphi)$, $f^*(A)$ is the $S$-algebra $S \otimes_R A$ defined above.
            \item On morphisms: For any homomorphism of $R$-algebras $h: A \to B$, the image $f^*(h)$ is the map $\text{id}_S \otimes h: S \otimes_R A \to S \otimes_R B$, defined by $s \otimes a \mapsto s \otimes h(a)$.
        \end{itemize}
        \TODO{comment on adjunction}
        % This functor is the left adjoint to the restriction of scalars functor $f_*: \mathbf{Alg}_S \to \mathbf{Alg}_R$.
\end{enumerate}
\end{definition}
\begin{definition} \label{definition:restriction_of_scalars_of_a_left_or_right_module_along_a_ring_homomorphism_from_the_base_ring}
Let $R$ and $S$ be \CrefAndHyperrefIfExist{definition:ring}{associative rings with identity}, and let $\varphi: R \to S$ be a unital \CrefAndHyperrefIfExist{definition:ring_homomorphism}{ring homomorphism}. Let $T$ be a ring.

\begin{enumerate}
    \item 
    Let $(M, +)$ be an \CrefAndHyperrefIfExist{definition:group}{abelian group} equipped with either the structure of a \CrefAndHyperrefIfExist{definition:module_of_a_ring}{$S-T$-bimodule} $(M, +, \cdot_S)$ or the structure of a $T-S$-bimodule $(M, +, \cdot_S)$.

    The \hldef{restriction of scalars of $M$ along $\varphi$} is the $R$-module structure on the same underlying abelian group $(M, +)$ defined as follows:
    \begin{itemize}
        \item If $M$ is a $S-T$-bimodule, the restriction of scalars of $M$ along $\varphi$, often denoted by \hl{$\varphi_* M$} (or \hl{$_R M$}), is the $R-T$-bimodule whose $R$-action $\cdot_R: R \times M \to M$ is given by
        $$ r \cdot_R m := \varphi(r) \cdot_S m $$
        for all $r \in R$ and $m \in M$.
        \item If $M$ is a $T-S$-bimodule, the restriction of scalars of $M$ along $\varphi$, often denoted by \hl{$\varphi_* M$} (or \hl{$M_R$}), is the $T-R$-bimodule whose $R$-action $\cdot_R: M \times R \to M$ is given by
        $$ m \cdot_R r := m \cdot_S \varphi(r) $$
        for all $r \in R$ and $m \in M$.
    \end{itemize}

    \item 
    Let ${}_S \text{Mod}_T$ and ${}_T \text{Mod}_S$ be the \CrefAndHyperrefIfExist{definition:category_of_modules_and_bimodules_over_rings}{categories of $S-T$ and $T-S$-bimodules}, respectively, and similarly for $R$.

    The \hldef{restriction of scalars functor for modules} is the covariant \CrefAndHyperrefIfExist{definition:functor_between_categories}{functor} induced by $\varphi$, defined for both left and right modules:
    \begin{itemize}
        \item For left $S$-modules, it is the functor
        \hl{$ \varphi_*: {}_S\text{Mod}_T \to {}_R\text{Mod}_T $}
        defined as follows:
        \begin{enumerate}
            \item On objects: For any left $S$-module $M$, $\varphi_*(M)$ is the left $R$-module obtained by restriction of scalars along $\varphi$.
            \item On morphisms: For any \CrefAndHyperrefIfExist{definition:homomorphism_of_modules_over_a_ring}{homomorphism} of left $S$-modules $h: M \to N$, the image $\varphi_*(h): \varphi_*(M) \to \varphi_*(N)$ is the map $h$ itself, viewed as a homomorphism of left $R$-modules.
        \end{enumerate}

        \item For right $S$-modules, it is the functor
        \hl{$ \varphi_*: {}_T \text{Mod}_S \to {}_T \text{Mod}_R $}
        defined as follows:
        \begin{enumerate}
            \item On objects: For any right $S$-module $M$, $\varphi_*(M)$ is the right $R$-module obtained by restriction of scalars along $\varphi$.
            \item On morphisms: For any homomorphism of right $S$-modules $h: M \to N$, the image $\varphi_*(h): \varphi_*(M) \to \varphi_*(N)$ is the map $h$ itself, viewed as a homomorphism of right $R$-modules.
        \end{enumerate}
    \end{itemize}
    In either context, if $\varphi$ is an inclusion map (making $R$ a subring of $S$), this functor is often called the \hldef{forgetful functor}.

        \item 
        Let $\mathsf{Ring}_S$ (or $S/\mathsf{Ring}$) denote the \CrefAndHyperrefIfExist{definition:category_of_rings_over_a_ring}{category of $S$-rings}. Let $\mathsf{Ring}_R$ be defined similarly.

        The \hldef{restriction of scalars functor for rings}, denoted by
        \hl{$ \varphi_*: \mathsf{Ring}_S \to \mathsf{Ring}_R $}
        is the functor defined as follows:
        \begin{itemize}
            \item On objects: Let $(A, \psi)$ be an $S$-ring. Then $\varphi_*(A)$ is the $R$-ring $(A, \psi \circ \varphi)$, where the structure map is the composition $R \xrightarrow{\varphi} S \xrightarrow{\psi} A$.
            \item On morphisms: For any morphism of $S$-rings $h: (A, \psi_A) \to (B, \psi_B)$, the image $\varphi_*(h)$ is the map $h$ itself, which satisfies $h \circ (\psi_A \circ \varphi) = \psi_B \circ \varphi$ and is thus a morphism of $R$-rings.
        \end{itemize}
        This functor simply pre-composes the structure map with $\varphi$, effectively "forgetting" the factorization through $S$.

        The restriction of scalars functor restricts to a functor $\varphi_*: \mathbf{Alg}_S \to \mathbf{Alg}_R$\CrefIfExists{definition:algebra_of_a_ring}. 

\end{enumerate}
\end{definition}
\begin{definition} \label{definition:coextension_of_scalars_of_modules_along_a_ring_homomorphism}
Let $R$ and $S$ be \CrefAndHyperrefIfExist{definition:ring}{rings} (not necessarily commutative), and let $f: R \to S$ be a \CrefAndHyperrefIfExist{definition:ring_homomorphism}{ring homomorphism}. The homomorphism $f$ endows $S$ with two bimodule structures --- that of an $(R, S)$-bimodule and of an $(S,R)$-bimodule via \CrefAndHyperrefIfExist{definition:restriction_of_scalars_of_a_left_or_right_module_along_a_ring_homomorphism_from_the_base_ring}{restriction of scalars}. Let $T$ be a ring.
\begin{enumerate}
    \item 

    The \hldef{co-extension of scalars} (or simply \hldef{co-extension}) along $f$ is defined separately for modules:
    \begin{itemize}
        \item For an $R-T$-bimodule $M$, the co-extension of scalars of $M$ along $f$ is the $S-T$-bimodule \CrefAndHyperrefIfExist{definition:hom_of_left_right_bi_modules_of_rings}{$\Hom_R(S, M)$} of left $R$-module homomorphisms from the $R-S$-bimodule $S$ to $M$. 
        % As such, the left $S$-action on a morphism $\psi \in \Hom_R(S, M)$ is defined by $(s \cdot \psi)(t) = \psi(t s)$ for all $s, t \in S$.
        
        \item For a $T-R$-bimodule $M$, the co-extension of scalars of $M$ along $f$ is the $T-S$-bimodule $\Hom_R(S, M)$ of right $R$-module homomorphisms from the $S-R$-bimodule $S$ to $M$. 
        %As such, the right $S$-action on a morphism $\psi \in \Hom_R(S, M)$ is defined by $(\psi \cdot s)(t) = \psi(s t)$ for all $s, t \in S$.
    \end{itemize}

    \item The \hldef{co-extension of scalars functor} (or \hldef{coinduction functor}), denoted by \hl{$f^!$} or \hl{$\CoInd_R^S$}, is the functor given by:
    \begin{itemize}
        \item For left $R$-modules:
        $$f^!: {}_R\mathsf{Mod}_T \to {}_S\mathsf{Mod}_T, \quad M \mapsto \Hom_R(S, M).$$
        This functor is the \CrefAndHyperrefIfExist{definition:adjoint_functors}{right adjoint} to the \CrefAndHyperrefIfExist{definition:restriction_of_scalars}{restriction of scalars functor} $f_*: {}_S\mathsf{Mod} \to {}_R\mathsf{Mod}$.
        
        \item For right $R$-modules:
        $$f^!: {}_T \mathsf{Mod}_R \to {}_T \mathsf{Mod}_S, \quad M \mapsto \Hom_R(S, M).$$
        This functor is the right adjoint to the restriction of scalars functor $f_*: \mathsf{Mod}_S \to \mathsf{Mod}_R$.
    \end{itemize}
\end{enumerate}

\end{definition}

% \begin{theorem}[Extension-Restriction and Restriction-Coextension adjunction for modules] \label{theorem:extension_restriction_and_restriction_coextension_adjunction_for_modules_over_rings}
% Let $R$ and $S$ be \CrefAndHyperrefIfExist{definition:ring}{rings} (not necessarily commutative), and let $f: R \to S$ be a \CrefAndHyperrefIfExist{definition:ring_homomorphism}{ring homomorphism}.

% \begin{enumerate}
%     \item \textbf{Extension-Restriction adjunction:} The \CrefAndHyperrefIfExist{definition:extension_of_scalars_for_modules_along_ring_homomorphisms}{extension of scalars functor} $f^* \coloneqq S \otimes_R -$ is \CrefAndHyperrefIfExist{definition:adjoint_functors_between_categories_unit_counit_of_adjoint_functors}{left adjoint} to the \CrefAndHyperrefIfExist{definition:restriction_of_scalars_of_a_left_or_right_module_along_a_ring_homomorphism_from_the_base_ring}{restriction of scalars functor} $f_*$.
    
%     Explicitly, for any left $R$-module $M$ and any left $S$-module $N$, there is a \CrefAndHyperrefIfExist{definition:natural_transformation_between_functors_between_categories}{natural isomorphism} of abelian groups:
%     \[
%         \Hom_S(S \otimes_R M, N) \cong \Hom_R(M, f_*N)
%     \]
%     given by the mapping
%     \[
%         \Psi \mapsto (m \mapsto \Psi(1 \otimes m)).
%     \]

%     \item \textbf{Restriction-Coextension adjunction:} The \CrefAndHyperrefIfExist{definition:coextension_of_scalars_of_modules_along_a_ring_homomorphism}{co-extension of scalars functor} $f^! \coloneqq \Hom_R(S, -)$ is \CrefAndHyperrefIfExist{definition:adjoint_functors_between_categories_unit_counit_of_adjoint_functors}{right adjoint} to the restriction of scalars functor $f_*$.
    
%     Explicitly, for any left $S$-module $N$ and any left $R$-module $M$, there is a \CrefAndHyperrefIfExist{definition:natural_transformation_between_functors_between_categories}{natural isomorphism} of abelian groups:
%     \[ \Hom_R(f_*N, M) \cong \Hom_S(N, \Hom_R(S, M)) \]
%     where $S$ is viewed as an $(R, S)$-bimodule when constructing the the \CrefAndHyperrefIfExist{definition:hom_of_left_right_bi_modules_of_rings}{Hom-module $\Hom_R(S, M)$}.
% \end{enumerate}
% \end{theorem}


\begin{theorem}[Extension-Restriction and Restriction-Coextension adjunction for modules] \label{theorem:extension_restriction_and_restriction_coextension_adjunction_for_modules_over_rings}
Let $R$ and $S$ be \CrefAndHyperrefIfExist{definition:ring}{rings} (not necessarily commutative), and let $f: R \to S$ be a \CrefAndHyperrefIfExist{definition:ring_homomorphism}{ring homomorphism}. Let $T$ be a ring.

\begin{enumerate}
    \item \textbf{Extension-Restriction adjunction:}
    \begin{enumerate}
        \item \textbf{Restriction on the Left:} The \CrefAndHyperrefIfExist{definition:extension_of_scalars_for_modules_along_ring_homomorphisms}{extension of scalars functor}
        \[ f^* = S \otimes_R - : {}_R\mathsf{Mod}_T \to {}_S\mathsf{Mod}_T \]
        \CrefIfExists{definition:category_of_modules_and_bimodules_over_rings}
        is \CrefAndHyperrefIfExist{definition:adjoint_functors_between_categories_unit_counit_of_adjoint_functors}{left adjoint} to the \CrefAndHyperrefIfExist{definition:restriction_of_scalars_of_a_left_or_right_module_along_a_ring_homomorphism_from_the_base_ring}{restriction of scalars functor}
        \[ f_* : {}_S\mathsf{Mod}_T \to {}_R\mathsf{Mod}_T. \]
        Explicitly, for any $R-T$-bimodule $M$ and any $S-T$-bimodule $N$, there is a natural isomorphism:
        \[
            \Hom_{S-T}(S \otimes_R M, N) \cong \Hom_{R-T}(M, f_*N).
        \]

        \item \textbf{Restriction on the Right:} The extension of scalars functor
        \[ f^* = - \otimes_R S : {}_T\mathsf{Mod}_R \to {}_T\mathsf{Mod}_S \]
        is left adjoint to the restriction of scalars functor
        \[ f_* : {}_T\mathsf{Mod}_S \to {}_T\mathsf{Mod}_R. \]
        Explicitly, for any $T-R$-bimodule $M$ and any $T-S$-bimodule $N$, there is a natural isomorphism:
        \[
            \Hom_{T-S}(M \otimes_R S, N) \cong \Hom_{T-R}(M, f_*N).
        \]
    \end{enumerate}

    \item \textbf{Restriction-Coextension adjunction:}
    \begin{enumerate}
        \item \textbf{Restriction on the Left:} The \CrefAndHyperrefIfExist{definition:coextension_of_scalars_of_modules_along_a_ring_homomorphism}{co-extension of scalars functor}
        \[ f^! = \Hom_R(S, -) : {}_R\mathsf{Mod}_T \to {}_S\mathsf{Mod}_T \]
        is \CrefAndHyperrefIfExist{definition:adjoint_functors_between_categories_unit_counit_of_adjoint_functors}{right adjoint} to the restriction of scalars functor
        \[ f_* : {}_S\mathsf{Mod}_T \to {}_R\mathsf{Mod}_T. \]
        Explicitly, for any $S-T$-bimodule $N$ and any $R-T$-bimodule $M$, there is a natural isomorphism:
        \[ \Hom_{R-T}(f_*N, M) \cong \Hom_{S-T}(N, \Hom_R(S, M)). \]

        \item \textbf{Restriction on the Right:} The co-extension of scalars functor
        \[ f^! = \Hom_R(S, -) : {}_T\mathsf{Mod}_R \to {}_T\mathsf{Mod}_S \]
        is right adjoint to the restriction of scalars functor
        \[ f_* : {}_T\mathsf{Mod}_S \to {}_T\mathsf{Mod}_R. \]
        Explicitly, for any $T-S$-bimodule $N$ and any $T-R$-bimodule $M$, there is a natural isomorphism:
        \[ \Hom_{T-R}(f_*N, M) \cong \Hom_{T-S}(N, \Hom_R(S, M)) \]
        where here $\Hom_R(S, M)$ uses the left $R$-module structure on $S$ and the right $R$-module structure on $M$.
    \end{enumerate}
\end{enumerate}
\end{theorem}




\subsection{Limits and colimits in categories}

\subsubsection{Categorical products, coproducts, and direct sums}

It is easier to first learn about \CrefAndHyperrefIfExist{definition:product_and_coproduct_of_objects_in_a_category}{products and coproducts} before learning about general \CrefAndHyperrefIfExist{definition:limit_and_colimit_of_a_diagram_in_a_category}{limits and colimits}.

One of the nice features about many algebraic structures such as groups and modules, is that simply taking their Cartesian product produces an object in the same category. In fact, the notion of product is a categorical one. Let us first define the categorical notion of product and the dual notion of coproducts:

\begin{definition}[Product in a category] \label{definition:product_and_coproduct_of_objects_in_a_category}
Let $\mathcal{C}$ be a category and let $\{X_i\}_{i \in I}$ be a family of objects in $\mathcal{C}$ indexed by a class $I$. 
\begin{enumerate}
    \item A \hldef{product of the family $\{X_i\}$} is an object $P$ of $\mathcal{C}$ together with a ``universal'' family of morphisms
    $$\pi_i : P \to X_i, \quad \text{for each } i \in I. $$
    More precisely, for any object $Y$ and any family of morphisms $\{f_i : Y \to X_i\}_{i \in I}$, there exists a unique morphism
    $$f : Y \to P$$
    making the following diagram commute for all $i \in I$, i.e. $\pi_i \circ f = f_i$:
    \begin{center}
    \begin{tikzcd}[row sep=large, column sep=large]
        Y \arrow[d, "\exists ! f", dashed] \arrow[dr, "f_i"] & \\
        \prod X_i \arrow[r, "\pi_i"'] & X_i
    \end{tikzcd}
    \end{center}
    Such a product is often denoted by \hl{$\prod_{i \in I} X_i$}. If $\prod_{i \in I} X_i$ exists in $\calC$, then it is unique up to unique isomorphism by the universal property described above.
    
    Equivalently, the product $\prod_{i \in I} X_i$ is the \CrefAndHyperrefIfExist{definition:limit_and_colimit_of_a_diagram_in_a_category}{limit} of the \CrefAndHyperrefIfExist{definition:diagram_in_a_category_indexed_by_a_small_category}{diagram} $I \to \calC, i \mapsto X_i$, where $I$ is made into a category whose objects are the members of $I$ and whose morphisms are just the identity morphisms.


    \item A \hldef{coproduct} (or synonymously \hldef{direct sum}) of the family $\{X_i\}$ is an object $C$ of $\mathcal{C}$ together with a ``universal'' family of morphisms
    $$\iota_i : X_i \to C, \quad \text{for each } i \in I.$$
    More precisely, for any object $Y$ and any family of morphisms $\{g_i : X_i \to Y\}_{i \in I}$, there exists a unique morphism
    $$g : C \to Y$$
    making the following diagram commute for all $i \in I$, i.e. $g \circ \iota_i = g_i$:
    \begin{center}
    \begin{tikzcd}[row sep=large, column sep=large]
        X_i \arrow[r, "\iota_i"] \arrow[dr, "g_i"'] & \coprod X_i \arrow[d, "\exists ! g", dashed] \\
        & Y
    \end{tikzcd}
    \end{center}
    Such a coproduct is often denoted by \hl{$\coprod_{i \in I} X_i$} or \hl{$\oplus_{i \in I} X_i$}. If $\coprod_{i \in I} X_i$ exists in $\calC$, then it is unique up to unique isomorphism by the universal property described above.

    Equivalently, the coproduct $\coprod_{i \in I} X_i$ is the \CrefAndHyperrefIfExist{definition:limit_and_colimit_of_a_diagram_in_a_category}{colimit} of the \CrefAndHyperrefIfExist{definition:diagram_in_a_category_indexed_by_a_small_category}{diagram} $I \to \calC, i \mapsto X_i$, where $I$ is made into a category whose objects are the members of $I$ and whose morphisms are just the identity morphisms.
\end{enumerate}
\end{definition}

For \CrefAndHyperrefIfExist{definition:concrete_category_over_a_category}{concrete categories}, cartesian products often result in the categorical products:

\begin{definition}[Product of Sets] \label{definition:product_of_sets}
Let $I$ be a (possibly \CrefAndHyperrefIfExist{definition:countable_finite_uncountable_sets}{infinite} but \CrefAndHyperrefIfExist{definition:small_set}{small}) index set and let $\{A_i\}_{i \in I}$ be a family of sets indexed by $I$.  
The \hldef{Cartesian product of the family $\{A_i\}_{i \in I}$}, denoted by \hl{$\prod_{i \in I} A_i$}, is defined as the set of all tuples/\CrefAndHyperrefIfExist{definition:function_of_sets}{functions}
$$ \prod_{i \in I} A_i := \{ (a_i)_{i \in I} \mid a_i \in A_i \text{ for all } i \in I \}, $$
where $(a_i)_{i \in I}$ denotes a function from $I$ to $\bigcup_{i \in I} A_i$ such that $(a_i)_{i \in I}(i) = a_i \in A_i$ for each $i \in I$.  

\TextIfExists{definition:category_of_sets}{
The Cartesian product $\prod_{i \in I} A_i$ is the \CrefAndHyperrefIfExist{definition:product_and_coproduct_of_objects_in_a_category}{product} of the objects $A_i$ in the \CrefAndHyperrefIfExist{definition:category_of_sets}{category of sets}.
}

The self product of a set $A$ indexed by $I$ is often denoted by \hl{$A^I$}. Note that elements of $A^I$ can be identified with \CrefAndHyperrefIfExist{definition:function_of_sets}{functions} $I \to A$. The finite self product of $A$ taken $n$ times is often denoted by \hl{$A^n$}.
For finitely many sets $A_1,\ldots,A_n$, their Cartesian product is denoted by \hl{$A_1 \times \cdots \times A_n$}. Elements of such a finite product may be written as $(a_1,\ldots,a_n)$.

\end{definition}
\begin{definition}[Product of Groups] \label{definition:product_of_groups}
Let $\{ G_i \}_{i \in I}$ be a family of groups indexed by a (possibly \CrefAndHyperrefIfExist{definition:countable_finite_uncountable_sets}{infinite} but \CrefAndHyperrefIfExist{definition:small_set}{small}) set $I$, each with group operation denoted multiplicatively and identity element $e_i$. 
The \hldef{(direct) product of the family $\{G_i\}_{i \in I}$}, denoted by \hl{$\prod_{i \in I} G_i$}, is defined as the set of all functions
$$ \prod_{i \in I} G_i := \{ (g_i)_{i \in I} \mid g_i \in G_i \text{ for all } i \in I \}, $$
equipped with the binary operation defined componentwise by
$$ (g_i)_{i \in I} \cdot (h_i)_{i \in I} := (g_i h_i)_{i \in I} $$
for all $(g_i)_{i \in I}, (h_i)_{i \in I} \in \prod_{i \in I} G_i$.  
Then $\prod_{i \in I} G_i$ is a group with the identity element $(e_i)_{i \in I}$ and inverses given by
$$ (g_i)_{i \in I}^{-1} = (g_i^{-1})_{i \in I}.  $$

The product $\prod_{i=1} G_i$ is the \CrefAndHyperrefIfExist{definition:product_and_coproduct_of_objects_in_a_category}{product} of the objects $G_i$ in the \CrefAndHyperrefIfExist{definition:group_homomorphism}{category of groups}.
As a set, note that $\prod_{i \in I} G_i$ coincides with the \CrefAndHyperrefIfExist{definition:product_of_sets}{product $\prod_{i \in I} G_i$} of the $G_i$ as sets. 

A self product of a group $G$ (indexed by a small set $I$), is often denoted by \hl{$G^I$}. A finite self product of a group $G$ taken $n$ times is often denoted by \hl{$G^n$}. In case that $G$ is abelian, these may be written as \hl{$G^{\oplus I}$} and \hl{$G^{\oplus n}$} respectively.

The product of finitely many groups $G_1,\ldots,G_n$ is often denoted by \hl{$G_1 \times \cdots \times G_n$}. 
\end{definition}
\begin{definition}[Product of Modules] \label{definition:product_of_modules_of_rings}
    Let $R$ and $S$ be \CrefAndHyperrefIfExist{definition:ring}{(not necessarily commutative) rings}, and let $\{ M_i \}_{i \in I}$ be a (possibly \CrefAndHyperrefIfExist{definition:countable_finite_uncountable_sets}{infinite} but \CrefAndHyperrefIfExist{definition:small_set}{small}) family of \CrefAndHyperrefIfExist{definition:module_of_a_ring}{$(R,S)$-bimodules}.  

\CrefAndHyperrefIfExist{definition:module_of_a_ring}{left $R$-modules, of right $R$-modules, of two-sided $R$-modules}, or of 

    The \hldef{(direct) product of the family $\{M_i\}_{i \in I}$} is defined, as a \CrefAndHyperrefIfExist{definition:group}{group}, as the \CrefAndHyperrefIfExist{definition:product_of_groups}{product of sets}:
    $$ \prod_{i \in I} M_i := \{ (m_i)_{i \in I} \mid m_i \in M_i \text{ for all } i \in I \}.  $$

    $\prod_{i \in I} M_i$ inherits a natural $R$-$S$ module structure defined componentwise by the following rules for all $(m_i)_{i \in I}, (n_i)_{i \in I} \in \prod_{i \in I} M_i$ and all scalars $r \in R$, $s \in S$:
    $$ (m_i)_{i \in I} + (n_i)_{i \in I} := (m_i + n_i)_{i \in I}, \qquad r \cdot (m_i)_{i \in I} \cdot s := (r \cdot m_i \cdot s)_{i \in I}.  $$

    % \begin{itemize}
    % \item If each $M_i$ is a left $R$-module:
    % $$
    % (m_i)_{i \in I} + (n_i)_{i \in I} := (m_i + n_i)_{i \in I}, \qquad
    % r (m_i)_{i \in I} := (r m_i)_{i \in I}.
    % $$

    % \item If each $M_i$ is a right $R$-module:
    % $$
    % (m_i)_{i \in I} + (n_i)_{i \in I} := (m_i + n_i)_{i \in I}, \qquad
    % (m_i)_{i \in I} r := (m_i r)_{i \in I}.
    % $$

    % \item If each $M_i$ is an $(R,S)$-bimodule:
    % $$
    % (r (m_i)_{i \in I}) s := (r m_i s)_{i \in I}, \qquad r \in R,\, s \in S.
    % $$

    The zero element of $\prod_{i \in I} M_i$ is the tuple $(0)_{i \in I}$, and additive inverses are given componentwise:
    $$
    -(m_i)_{i \in I} := (-m_i)_{i \in I}.
    $$
    % Thus, $\prod_{i \in I} M_i$ is a left/right/two-sided $R$-module or an $(R,S)$-bimodule under these operations.
    % \end{itemize}

    Note that we can define the product of a family $\{M_i\}_{i \in I}$ of left/right/two-sided $R$-modules by taking the \CrefAndHyperrefIfExist{definition:module_of_a_ring}{natural bimodule structure} of each module.

    \CrefAndHyperrefIfExist{definition:product_of_groups}{As usual}, $\prod_{i \in I} M_i$ is the \CrefAndHyperrefIfExist{definition:product_and_coproduct_of_objects_in_a_category}{categorical product} of the objects $M_i$ in the appropriate \CrefAndHyperrefIfExist{definition:category_of_modules_and_bimodules_over_rings}{category of modules}. Moreover, the product of finitely many modules $M_1,\ldots,M_n$ is often written as \hl{$M_1 \times \cdots \times M_n$}, which agrees with notation for the \CrefAndHyperrefIfExist{definition:product_of_groups}{product of finitely many groups}. We often write the self-product of a module $M$ indexed by a (small) set $I$ as \hl{$M^I$} or by \hl{$M^{\oplus I}$}. A finite self-product of a module $M$ taken $n$ times is often denoted by \hl{$M^n$} or \hl{$M^{\oplus n}$}; note that these all agree with the notations for abelian groups.
\end{definition}
\begin{definition}[Coproduct of Modules] \label{definition:coproduct_of_modules_of_rings}
    Let $R$ and $S$ be \CrefAndHyperrefIfExist{definition:ring}{(not necessarily commutative) rings}, and let $\{ M_i \}_{i \in I}$ be a (possibly infinite but small) family of $(R,S)$-bimodules.
    % \CrefAndHyperrefIfExist{definition:module_of_a_ring}{left $R$-modules, right $R$-modules, or $(R,S)$-bimodules}, respectively.

    The \hldef{coproduct (direct sum) of the family $\{M_i\}_{i \in I}$}, denoted by \hl{$\bigoplus_{i \in I} M_i$}, is constructed as
    $$ \bigoplus_{i \in I} M_i := \left\{ (m_i)_{i \in I} \in \prod_{i \in I} M_i \mid m_i = 0 \text{ for all but finitely many } i \in I \right\} $$
    \CrefIfExists{definition:product_of_modules_of_rings}
    consisting of all tuples with only finitely many nonzero entries.

    Addition and scalar multiplication in $\bigoplus_{i \in I} M_i$ are defined componentwise as in the \CrefAndHyperrefIfExist{definition:product_of_modules_of_rings}{direct product}:
    $$ (m_i)_{i \in I} + (n_i)_{i \in I} := (m_i + n_i)_{i \in I}, \qquad r \cdot (m_i)_{i \in I} \cdot s := (r \cdot m_i \cdot s)_{i \in I}, \qquad r \in R,\, s \in S.  $$
    In all cases, the zero element is $(0)_{i \in I}$, and additive inverses are given by $-(m_i)_{i \in I} := (-m_i)_{i \in I}$. 

    Note that we can define the coproduct of a family $\{M_i\}_{i \in I}$ of left/right/two-sided $R$-modules by taking the \CrefAndHyperrefIfExist{definition:module_of_a_ring}{natural bimodule structure} of each module.
    
    \TODO{submodule}
    Note that $\bigoplus_{i \in I} M_i$ is a submodule of $\prod_{i \in I} M_i$. Moreover, $\bigoplus_{i \in I} M_i$ is the \CrefAndHyperrefIfExist{definition:product_and_coproduct_of_objects_in_a_category}{coproduct} in the appropriate \CrefAndHyperrefIfExist{definition:category_of_modules_and_bimodules_over_rings}{category of modules}.

    For finitely many modules $M_1,\ldots,M_n$, the direct sum $\bigoplus_{j=1}^{n} M_j$, which may also be written as \hl{$M_1 \oplus \cdots \oplus M_n$}, is simply the usual \CrefAndHyperrefIfExist{definition:product_of_modules_of_rings}{Cartesian product $\prod_{j=1}^n M_j$} of the modules, as every tuple automatically has only finitely many nonzero entries.
\end{definition}


\subsubsection{General limits and colimits}

It is probably best to keep in mind that the definitions listed in this subsubsection are intended as precise definitions, rather than intuitively helpful ones.


\begin{definition}[Diagram in a category and category of diagrams] \label{definition:diagram_in_a_category_indexed_by_a_small_category}
Let $\mathcal{C}$ be a \hyperrefIfExists{definition:category}{(large) category}\CrefIfExists{definition:category}, and let $I$ be a \CrefAndHyperrefIfExist{definition:category}{(large) category}. 
    \begin{enumerate}
        \item 
        A \hldef{diagram of shape $I$ in $\mathcal{C}$} is a \hyperrefIfExists{definition:functor_between_categories}{functor}\CrefIfExists{definition:functor_between_categories} $D: I \to \mathcal{C}$.
        We often denote such a diagram by the family \hl{$\{ D(i) \}_{i \in \mathrm{Ob}(I)}$} with transition maps given by the functorial image of morphisms in $I$. 
        
        A diagram is also synonymously called a \hldef{system}. Moreover, the category $I$ is called the \hldef{index category} or the \hldef{indexing category of the diagram $D$}.

        \item Given two diagrams $D,E: I \to \mathcal{C}$, a \hldef{morphism of diagrams} is a simply a \hyperrefIfExists{definition:natural_transformation_between_functors_between_categories}{natural transformation}\CrefIfExists{definition:natural_transformation_between_functors_between_categories} $D \Rightarrow E$ of the functors $D$ and $E$. 

        \item The \hldef{category of $I$-shaped diagrams in $\mathcal{C}$} or simply \hldef{diagram category (of $I$-shaped diagrams in $\calC$)}, often denoted \hl{$\mathcal{C}^I$}, \hl{$[I, \calC]$}, or \hl{$\operatorname{Fun}(I, \calC)$},
        is the (large) category whose objects are functors $I \to \mathcal{C}$ (that is, diagrams of shape $I$ in $\mathcal{C}$) and whose morphisms are \CrefAndHyperrefIfExist{definition:natural_transformation_between_functors_between_categories}{natural transformations} between such functors. The category $\calC^I$ is also called the \hldef{functor category of functors $I \to \calC$}. \TextIfExists{definition:presheaf_on_a_category}{Equivalently, the functor category $\calC^I$ is the category \CrefAndHyperrefIfExist{definition:presheaf_on_a_category}{$\PreShv(I^{\op}, \calC)$ of presheaves} on $I^{\op}$ with values in $\calC$ and hence notations for presheaf categories are applicable as notations for functor categories.}

        If $\calC$ is \hyperrefIfExists{definition:locally_small_category}{locally small}\CrefIfExists{definition:locally_small_category} and $I$ is small, then $\calC^I$ is locally small by Lemma \ref{lemma:category_of_presheaves_on_a_small_category_of_locally_small_value_is_locally_small}.
    \end{enumerate}
\end{definition}


\begin{lemma} \label{lemma:category_of_presheaves_on_a_small_category_of_locally_small_value_is_locally_small}
    Let $\calC$ be a \hyperrefIfExists{definition:locally_small_category}{small category}\CrefIfExists{definition:locally_small_category} (resp. $U$-small category where $U$ is some \hyperrefIfExists{definition:grothendieck_universe}{universe}\CrefIfExists{definition:grothendieck_universe}) and let $\calA$ be a \CrefAndHyperrefIfExist{definition:locally_small_category}{locally small} category (resp. $U$-locally small category). The \hyperrefIfExists{definition:presheaf_on_a_category}{presheaf category $\PreShv(\calC, \calA)$}\CrefIfExists{definition:presheaf_on_a_category} is locally small (resp. $U$-locally small).
\end{lemma}
\begin{proof}
    A morphism $\calF \to \calG$ in $\PreShv(\calC, \calA)$ is a \hyperrefIfExists{definition:natural_transformation_between_functors_between_categories}{natural transformation}\CrefIfExists{definition:natural_transformation_between_functors_between_categories} of the functors $\calF, \calG: \calC^{\op} \to \calA$. Such a natural transformation is encoded by a family $(\eta_C)_C$ of morphisms (satisfying certain conditions) $\eta_C: \calF(C) \to \calG(C)$ in $\calA$ over objects $C$ of $\calC^{\op}$. The product $\prod_{C \in \Ob \calC^{\op}} \Hom_{\calA}(\calF(C), \calG(C))$ is a product of ($U$-small) sets indexed by a ($U$-small) set, and the collection of natural transformations is a subset of this set. Therefore, $\Hom_{\PreShv(\calC, \calA)}(\calF, \calG)$ is a ($U$-small) set.  
\end{proof}


\begin{definition}[Cones, limits and colimits in a category] \label{definition:limit_and_colimit_of_a_diagram_in_a_category}
Let $\mathcal{C}$ be a \CrefAndHyperrefIfExist{definition:category}{(large) category}, let $I$ be a (large) category, and let $D: I \to \mathcal{C}$ be a \CrefAndHyperrefIfExist{definition:diagram_in_a_category_indexed_by_a_small_category}{diagram}\CrefIfExists{definition:diagram_in_a_category_indexed_by_a_small_category}.

\begin{enumerate}
    \item A \hldef{cone to the diagram $D$} is an object $L \in \mathcal{C}$ together with a family of morphisms
    \[
    \{\pi_i: L \to D(i)\}_{i \in I}
    \]
    such that for every morphism $f: i \to j$ in $I$, the diagram
    \begin{center}
    \begin{tikzcd}[row sep=large, column sep=large]
        & L \arrow[dl, "\pi_i"'] \arrow[dr, "\pi_j"] & \\
        D(i) \arrow[rr, "D(f)"] & & D(j)
    \end{tikzcd}
    \end{center}
    commutes, i.e.  $D(f) \circ \pi_i = \pi_j$.
    


    \item A cone $(L, \{\pi_i\})$ is called a \hldef{limit of $D$} if it satisfies the following ``universal property'':
    for any cone $(C, \{ f_i \})$ over $D$, there exists a \textit{unique} morphism $u: C \to L$ such that
    \[
    \pi_i \circ u = f_i \quad \text{for all } i \in I.
    \]
    Visually, the following diagrams commute every morphism $f: i \to j$ in $I$:
    \begin{center}
    \begin{tikzcd}[row sep=large, column sep=large]
        & C \arrow[d, "\exists ! u", dashed] \arrow[ddl, "f_i"', bend right=20] \arrow[ddr, "f_j", bend left=20] & \\
        & L \arrow[dl, "\pi_i"] \arrow[dr, "\pi_j"'] & \\
        D(i) \arrow[rr, "D(f)"] & & D(j).
    \end{tikzcd}
    \end{center}
    If such a cone exists, then the object $L$ is necessarily unique up to unique isomorphism by the universal property. In this case, $L$ is denoted by \hl{$\lim_{i \in I} D$} or \hl{$\lim D$}.



    
    \item A \hldef{cocone from the diagram $D$} is an object $C \in \mathcal{C}$ together with a family of morphisms
    \[
    \{\iota_i: D(i) \to C\}_{i \in I}
    \]
    such that for every morphism $f: i \to j$ in $I$, the diagram
    \begin{center}
    \begin{tikzcd}[row sep=large, column sep=large]
        D(i) \arrow[rr, "D(f)"] \arrow[dr, "\iota_i"'] & & D(j) \arrow[dl, "\iota_j"] \\
        & C & 
    \end{tikzcd}
    \end{center}
    commutes, i.e. $\iota_j \circ D(f) = \iota_i$.

    \item A cocone $(L, \{\iota_i\})$ is called a \hldef{colimit of $D$} if it satisfies the following ``universal property'':
    for any cocone $(C, \{ g_i \})$ under $D$, there exists a \textit{unique} morphism $u: L \to C$ such that
    \[
    u \circ \iota_i = g_i \quad \text{for all } i \in I.
    \]
    Visually, the following diagrams commute every morphism $f: i \to j$ in $I$:
    \begin{center}
    \begin{tikzcd}[row sep=large, column sep=large]
        D(i) \arrow[rr, "D(f)"] \arrow[dr, "\iota_i"] \arrow[ddr, "g_i"', bend right=20] & & D(j) \arrow[dl, "\iota_j"'] \arrow[ddl, "g_j", bend left=20] \\
        & L \arrow[d, "\exists ! u", dashed] & \\
        & C &. 
    \end{tikzcd}
    \end{center}
    If such a cocone exists, then the object $L$ is necessarily unique up to unique isomorphism by the universal property. In this case, $L$ is denoted by \hl{$\colim_{i \in I} D$} or \hl{$\colim D$}.

\end{enumerate}

A limit/colimit is called \hldef{finite} (resp. \hldef{small}) if the diagram category $I$ is finite (resp. small).

Some authors use the terms \hldef{projective limit} or \hldef{inverse limit} to refer to what is defined here as a limit, Similarly, the terms \hldef{inductive limit} or \hldef{direct limit} are sometimes used to mean a colimit. However, these phrases can have more specific meanings to other authors: a \emph{projective} or \emph{inverse limit} may refer to a limit over a diagram indexed by a \hyperrefIfExists{definition:partially_ordered_set}{codirected poset}\CrefIfExists{definition:partially_ordered_set}. Likewise, an \emph{inductive} or \emph{direct limit} may refer to a colimit over a \hyperrefIfExists{definition:partially_ordered_set}{directed poset}\CrefIfExists{definition:partially_ordered_set}\TextIfExists{definition:projective_and_inductive_limits_in_categories}{ (see \Cref{definition:projective_and_inductive_limits_in_categories})}.

Thus, while the terms are sometimes used interchangeably with ``limit'' and ``colimit,'' they may also emphasize particular indexing shapes and directions, distinguishing them from general limits and colimits taken over arbitrary small categories.
\end{definition}

\begin{example}[Geometric and Algebraic Limits/Colimits]
Many fundamental constructions in topology and algebra are characterized by universal properties of limits and colimits.

\begin{enumerate}
    \item \textbf{The $p$-adic Integers (Inverse Limit):} 
    The ring of $p$-adic integers $\mathbb{Z}_p$ is the limit of the inverse system of finite cyclic rings $\mathbb{Z}/p^n\mathbb{Z}$ with quotient maps:
    \[ \mathbb{Z}_p \cong \varprojlim_{n} \mathbb{Z}/p^n\mathbb{Z} = \{ (a_n)_{n \in \mathbb{N}} \in \prod \mathbb{Z}/p^n\mathbb{Z} \mid a_{n+1} \equiv a_n \pmod{p^n} \} \]
    

    \item \textbf{Absolute Galois Groups (Inverse Limit):} 
    The absolute Galois group $G_K = \operatorname{Gal}(\overline{K}/K)$ of a field $K$ is the inverse limit of the Galois groups of all finite Galois extensions $L/K$ contained in $\overline{K}$:
    \[ \operatorname{Gal}(\overline{K}/K) \cong \varprojlim_{L/K \text{ finite}} \operatorname{Gal}(L/K) \]
    This gives $G_K$ the structure of a profinite group.

    \item \textbf{The Cone of a Space (Pushout):} 
    Think of $X \times [0,1]$ as a cylinder. The map $X \to X \times [0,1]$ picks out the "top lid." The map $X \to \{*\}$ sends that entire lid to a single point. 
    The **pushout** $CX$ is the result of taking the cylinder and pinching the entire top lid into a single vertex.
    \begin{center}
    \begin{tikzcd}
    X \arrow[r, "{(x,1)}"] \arrow[d] & X \times [0,1] \arrow[d] \\
    \{*\} \arrow[r] & CX = (X \times [0,1]) / (X \times \{1\})
    \end{tikzcd}
    \end{center}

\end{enumerate}
\end{example}

\TODO{examples of limits and colimits, including products, coproducts, equalizer and coequalizer}


\begin{definition} \label{definition:small_and_finite_limits_and_colimits_in_a_category}
    Let $\mathcal{C}$ be a \CrefAndHyperrefIfExist{definition:category}{(large) category}, let $I$ be a \hyperrefIfExists{definition:locally_small_category}{small category}\CrefIfExists{definition:locally_small_category}, and let $D: I \to \mathcal{C}$ be a \hyperrefIfExists{definition:diagram_in_a_category_indexed_by_a_small_category}{diagram}\CrefIfExists{definition:diagram_in_a_category_indexed_by_a_small_category}.

    A \CrefAndHyperrefIfExist{definition:limit_and_colimit_of_a_diagram_in_a_category}{limit or colimit} is called \hldef{finite} (resp. \hldef{small}) if the indexing category $I$ has finitely many objects and morphisms (resp. if $I$ is a \CrefAndHyperrefIfExist{definition:locally_small_category}{small category}).
\end{definition}


\begin{definition}[Complete and Cocomplete Category] \label{definition:complete_and_cocomplete_category}
Let $\mathcal{C}$ be a \CrefAndHyperrefIfExist{definition:category}{category}.  
\begin{itemize}
    \item The category $\mathcal{C}$ is called \hldef{complete} (resp. \hldef{finitely complete}) if all \CrefAndHyperrefIfExist{definition:small_and_finite_limits_and_colimits_in_a_category}{small limits} (resp. finite limits) exist in $\mathcal{C}$; that is, for every small diagram $D : J \to \mathcal{C}$ (with $J$ a small category), the limit $\lim D$ exists and is an object of $\mathcal{C}$.
    \item The category $\mathcal{C}$ is called \hldef{cocomplete} (resp. \hldef{finitely cocomplete}) if all \CrefAndHyperrefIfExist{definition:small_and_finite_limits_and_colimits_in_a_category}{small colimits} (resp. finite colimits) exist in $\mathcal{C}$; that is, for every small diagram $D : J \to \mathcal{C}$, the colimit $\mathrm{colim}\ D$ exists and is an object of $\mathcal{C}$.
\end{itemize}
\end{definition}

\begin{definition}[Filtered category] \label{definition:filtered_cofiltered_category}
    \begin{enumerate}
        \item 
        A \hldef{filtered category} is a (nonempty, large) category $\mathcal{I}$ satisfying the following conditions:

        \begin{itemize}
            \item For every finite collection of objects $i_1, i_2, \ldots, i_n$ in $\mathcal{I}$, there exists an object $j$ and morphisms
            \[
            \phi_k: i_k \to j, \quad \text{for each } k=1, \ldots, n.
            \]

            \item For every pair of morphisms $f,g: i \to j$ in $\mathcal{I}$, there exists an object $k$ and a morphism 
            \[
            h: j \to k
            \] 
            that satisfies 
            \[
            h \circ f = h \circ g.
            \]
        \end{itemize}

        \begin{figure}[h]
            \centering
            \begin{minipage}{0.45\textwidth}
                \centering
                % Diagram 1: "Joint Limit" (Upper Bound)
                \begin{tikzcd}[row sep=large, column sep=large]
                    i_1 \arrow[dr, "\phi_1", dashed] & \\
                    & j \\
                    i_2 \arrow[ur, "\phi_2"', dashed] & 
                \end{tikzcd}
                \caption*{Condition 1: Upper Bound}
            \end{minipage}
            \hfill
            \begin{minipage}{0.45\textwidth}
                \centering
                \begin{tikzcd}[row sep=large, column sep=large]
                    i \arrow[r, "f", shift left] \arrow[r, "g"', shift right] & 
                    j \arrow[r, "h", dashed] & 
                    k
                \end{tikzcd}
                \caption*{Condition 2: Coequalizing map}
            \end{minipage}
        \end{figure}
        In other words, $\mathcal{I}$ is nonempty, any finite diagram of objects admits a \CrefAndHyperrefIfExist{definition:limit_and_colimit_of_a_diagram_in_a_category}{cocone}, and any pair of parallel morphisms become equal after post-composition with an appropriate morphism.

    \item Dually, a \hldef{Cofiltered category} is a category whose \hyperrefIfExists{definition:opposite_category_of_a_category}{opposite category}\CrefIfExists{definition:opposite_category_of_a_category} is filtered. More explicitly, A cofiltered category is a (nonempty, large) category $\mathcal{I}$ satisfying the following conditions:

    \begin{itemize}
        \item For every finite collection of objects $i_1, i_2, \ldots, i_n$ in $\mathcal{I}$, there exists an object $j$ and morphisms
        \[
        \phi_k: j \to i_k, \quad \text{for each } k=1, \ldots, n.
        \]

        \item For every pair of morphisms $f,g: j \to i$ in $\mathcal{I}$, there exists an object $k$ and a morphism 
        \[
        h: k \to j
        \] 
        that satisfies
        \[
        f \circ h = g \circ h.
        \]
    \end{itemize}

    \begin{figure}[h]
        \centering
        \begin{minipage}{0.45\textwidth}
            \centering
            % Diagram 1: "Joint Limit" (Lower Bound)
            \begin{tikzcd}[row sep=large, column sep=large]
                i_1 & \\
                & j \arrow[ul, "\phi_1", dashed] \arrow[dl, "\phi_2"', dashed] \\
                i_2 & 
            \end{tikzcd}
            \caption*{Condition 1: Lower Bound}
        \end{minipage}
        \hfill
        \begin{minipage}{0.45\textwidth}
            \centering
            \begin{tikzcd}[row sep=large, column sep=large]
                k \arrow[r, "h", dashed] &
                j \arrow[r, "f", shift left] \arrow[r, "g"', shift right] & 
                i
            \end{tikzcd}
            \caption*{Condition 2: Equalizing map}
        \end{minipage}
    \end{figure}

    In other words, $\mathcal{I}$ is nonempty, any finite diagram of objects admits a cone, and any pair of parallel morphisms become equal after pre-composition with an appropriate morphism.

    \end{enumerate}

    
\end{definition}



\begin{definition}[Special cases of limits] \label{definition:projective_and_inductive_limits_in_categories}
Let $\mathcal{C}$ be a (large) category. Let $I$ be a (large) category. Let $I \to \mathcal{C}$ be a diagram/system. 
\begin{itemize}
    \item Suppose that the system is a \hyperrefIfExists{definition:system_in_a_category_indexed_by_a_directed_poset}{cofiltered system}\CrefIfExists{definition:system_in_a_category_indexed_by_a_directed_poset}, i.e. $I$ is a cofiltered category. A \hyperrefIfExists{definition:limit_and_colimit_of_a_diagram_in_a_category}{limit}\CrefIfExists{definition:limit_and_colimit_of_a_diagram_in_a_category} of this diagram is often denoted by 
    $$\hlin{ \varprojlim_{i\in I} D(i) }$$
    and may be called a \hldef{cofiltered (inverse/projective) limit}. In case that the system is more specifically an \hyperrefIfExists{definition:system_in_a_category_indexed_by_a_directed_poset}{inverse/projective system}\CrefIfExists{definition:system_in_a_category_indexed_by_a_directed_poset}, i.e. $I$ is a cofiltered poset, the preferred term for such a limit is \emph{inverse/projective limit}.

    \item Suppose that the system is a filtered system, i.e. $I$ is a filtered category. A colimit of this diagram is often denoted by 
    $$\hlin{ \varinjlim_{i\in I} D(i) }$$
    and may be called a \hldef{filtered colimit} or a \hldef{direct/inductive/injective limit}. In case that the system is more specifically a direct/inductive system, i.e. $I$ is a filtered poset, the preferred term for such a limit is \emph{direct/inductive limit}.

\end{itemize}
\end{definition}



% \TODO{example of ind/pro-limit}

One noteworthy fact is that if a category has all \CrefAndHyperrefIfExist{definition:product_and_coproduct_of_objects_in_a_category}{(co)products} and \CrefAndHyperrefIfExist{definition:equalizer_and_coequalizer_of_morphisms_in_a_category}{(co)equalizers}, then it has all \CrefAndHyperrefIfExist{definition:limit_and_colimit_of_a_diagram_in_a_category}{(co)limits} (\Cref{theorem:limits_colimits_are_equalizers_coequaulizers_of_products_coproducts}).
\begin{definition}[Equalizer in a category] \label{definition:equalizer_and_coequalizer_of_morphisms_in_a_category}
Let $\mathcal{C}$ be a \CrefAndHyperrefIfExist{definition:category}{(large) category} and let $f, g : X \to Y$ be morphisms in $\mathcal{C}$. 
\begin{enumerate}
    \item An \hldef{equalizer of $f$ and $g$} is an object $E$ together with a morphism
    $$e : E \to X$$
    such that
    $$f \circ e = g \circ e$$
    and for any object $Z$ with morphism $z : Z \to X$ satisfying
    $$f \circ z = g \circ z,$$
    there exists a unique morphism $u : Z \to E$ making the diagram commute:
    $$e \circ u = z.$$

    \begin{center}
        \begin{tikzcd}[column sep=large, row sep=large]
        Z \arrow[d, dashed, "\exists! u"] \arrow[dr, "z"] & & \\
        E \arrow[r, "e"] & X \arrow[r, shift left, "f"] \arrow[r, shift right, "g"'] & Y
        \end{tikzcd}
    \end{center}
    If such an equalizer of $f$ and $g$ exists, then we say that the following \hldef{equalizer diagram is exact}:
    \begin{center}
    \begin{tikzcd}[column sep=large, row sep=large]
    E \arrow[r, "e"] & X \arrow[r, shift left, "f"] \arrow[r, shift right, "g"'] & Y
    \end{tikzcd}
    \end{center}

    \item A \hldef{coequalizer of $f$ and $g$} is an object $Q$ together with a morphism
    $$q : Y \to Q$$
    such that
    $$q \circ f = q \circ g$$
    and for any object $Z$ with morphism $w : Y \to Z$ satisfying
    $$w \circ f = w \circ g,$$
    there exists a unique morphism $v : Q \to Z$ making the diagram commute:
    $$v \circ q = w.$$

    \begin{center}
        \begin{tikzcd}[column sep=large]
        X \arrow[r, shift left, "f"] \arrow[r, shift right, "g"'] & Y \arrow[r, "q"] \arrow[dr, "w"'] & Q \arrow[d, dashed, "\exists! v"] \\
        & & Z
        \end{tikzcd}
    \end{center}
    If such a coequalizer of $f$ and $g$ exists, then we say that the following \hldef{coequalizer diagram is exact}:
    \begin{center}
    \begin{tikzcd}[column sep=large, row sep=large]
        X \arrow[r, shift left, "f"] \arrow[r, shift right, "g"'] & Y \arrow[r, "q"] & Q 
    \end{tikzcd}
    \end{center}



\end{enumerate}
\end{definition}
\begin{theorem} \label{theorem:limits_colimits_are_equalizers_coequaulizers_of_products_coproducts}
    Let $\mathcal{C}$ be a \CrefAndHyperrefIfExist{definition:category}{category}. Let $F: \mathcal{J} \to \mathcal{C}$ be a \CrefAndHyperrefIfExist{definition:diagram_in_a_category_indexed_by_a_small_category}{diagram} where $\mathcal{J}$ is a \CrefAndHyperrefIfExist{definition:locally_small_category}{small category}. 

    \begin{enumerate}
        \item The \CrefAndHyperrefIfExist{definition:limit_and_colimit_of_a_diagram_in_a_category}{limit} of $F$ is constructed as the \CrefAndHyperrefIfExist{definition:equalizer_and_coequalizer_of_morphisms_in_a_category}{equalizer} of the pair of morphisms $(s, t)$: assuming that the \CrefAndHyperrefIfExist{definition:product_and_coproduct_of_objects_in_a_category}{products} and equalizers below exist in $\calC$, the limit $\lim F$ exists and 
        \[
        \lim F \cong \operatorname{eq} \left( 
        \prod_{j \in \operatorname{Ob}(\mathcal{J})} F(j) 
        \xrightarrow[\;t\;]{\;s\;} 
        \prod_{\alpha \in \operatorname{Mor}(\mathcal{J})} F(\operatorname{cod}(\alpha)) 
        \right)
        \]
        where $\operatorname{cod}(\alpha)$ stands for the codomain of the morphism $\alpha$, and the morphisms $s$ and $t$ are induced by the universal property of the product, such that for any morphism $\alpha: i \to k$ in $\mathcal{J}$, the projection to the factor indexed by $\alpha$ is:
        \begin{itemize}
            \item $\pi_{\alpha} \circ s = F(\alpha) \circ \pi_{i}$
            \item $\pi_{\alpha} \circ t = \pi_{k}$
        \end{itemize}

        \item The \CrefAndHyperrefIfExist{definition:limit_and_colimit_of_a_diagram_in_a_category}{colimit} of $F$ is constructed as the \CrefAndHyperrefIfExist{definition:equalizer_and_coequalizer_of_morphisms_in_a_category}{coequalizer} of the pair of morphisms $(s, t)$: assuming that the \CrefAndHyperrefIfExist{definition:product_and_coproduct_of_objects_in_a_category}{coproducts} and coequalizers below exist in $\calC$, the colimit $\operatorname{colim} F$ exists and 
        \[
        \operatorname{colim} F \cong \operatorname{coeq} \left( 
        \coprod_{\alpha \in \operatorname{Mor}(\mathcal{J})} F(\operatorname{dom}(\alpha)) 
        \xrightarrow[\;t\;]{\;s\;} 
        \coprod_{j \in \operatorname{Ob}(\mathcal{J})} F(j) 
        \right)
        \]
        where $\operatorname{dom}(\alpha)$ stands for the domain of the morphism $\alpha$, and the morphisms $s$ and $t$ are induced by the universal property of the coproduct, such that for any morphism $\alpha: i \to k$ in $\mathcal{J}$, the injection from the summand indexed by $\alpha$ is:
        \begin{itemize}
            \item $s \circ \iota_{\alpha} = \iota_{k} \circ F(\alpha)$
            \item $t \circ \iota_{\alpha} = \iota_{i}$
        \end{itemize}

    
    \end{enumerate}

    In particular,
    \begin{enumerate}
        \item If $\mathcal{C}$ has all nonempty finite (resp. small) products and equalizers, then $\mathcal{C}$ has all nonempty finite (resp. small) limits.
        \item If $\mathcal{C}$ has all nonempty finite (resp. small) coproducts and coequalizers, then $\mathcal{C}$ has all nonempty finite (resp. small) colimits.

        \item If $\mathcal{C}$ has all finite (resp. small) products and equalizers, then $\mathcal{C}$ has all finite (resp. small) limits.
        \item If $\mathcal{C}$ has all finite (resp. small) coproducts and coequalizers, then $\mathcal{C}$ has all finite (resp. small) colimits.

    \end{enumerate}


\end{theorem}



\subsection{Additive categories}

\subsubsection{Additive and abelian categories}

Given modules $M$ and $N$, notice that \CrefAndHyperrefIfExist{definition:hom_of_left_right_bi_modules_of_rings}{$\Hom(M,N)$} is not only a set, but also an abelian group. This, along with few other nice properties, makes the category of modules into an \CrefAndHyperrefIfExist{definition:additive_category_preadditive_category}{additive category}.


\begin{definition} \label{definition:initial_final_zero_objects_of_a_category}
Let $\mathcal{C}$ be a \CrefAndHyperrefIfExist{definition:category}{(large) category}.

\begin{enumerate}
    \item An object $I \in \mathcal{C}$ is called an \hldef{initial object} if for every object $X \in \mathcal{C}$ there exists a unique morphism
    $$I \to X.$$
    Equivalently, an initial object is a \CrefAndHyperrefIfExist{definition:limit_and_colimit_of_a_diagram_in_a_category}{limit} of the empty \CrefAndHyperrefIfExist{definition:diagram_in_a_category_indexed_by_a_small_category}{diagram}, if such a limit exists.

    \item An object $F \in \mathcal{C}$ is called a \hldef{final object} (or \hldef{terminal object}) if for every object $X \in \mathcal{C}$ there exists a unique morphism
    $$X \to F.$$
    Equivalently, a final object is a \CrefAndHyperrefIfExist{definition:limit_and_colimit_of_a_diagram_in_a_category}{colimit} of the empty \CrefAndHyperrefIfExist{definition:diagram_in_a_category_indexed_by_a_small_category}{diagram}, if such a colimit exists.

    \item An object $Z \in \mathcal{C}$ is called a \hldef{zero object} if $Z$ is both initial and final in $\mathcal{C}$. In particular, for every object $X \in \mathcal{C}$ there exist unique morphisms
    $$Z \to X \quad \text{and} \quad X \to Z.$$
\end{enumerate}
In particular, if initial/final/zero objects exist in a cateogry, then they are unique up to unique isomorphism.
\end{definition}

\begin{definition} \label{definition:pointed_category}
    A \hldef{pointed category} is a \CrefAndHyperrefIfExist{definition:category}{category} that has a \CrefAndHyperrefIfExist{definition:initial_final_zero_objects_of_a_category}{zero object}.
\end{definition}
\begin{definition}[Zero morphism] \label{definition:zero_morphism_in_a_pointed_category}
Let $\mathcal{C}$ be a \CrefAndHyperrefIfExist{definition:pointed_category}{pointed category}. Let $X$ and $Y$ be objects in $\mathcal{C}$, and let $0$ be a \CrefAndHyperrefIfExist{definition:initial_final_zero_objects_of_a_category}{zero object} of $\mathcal{C}$. 
The \hldef{zero morphism} from $X$ to $Y$ is the unique composite morphism
$$\hlin{0_{XY} : X \to 0 \to Y,}$$
where $X \to 0$ is the unique morphism into the terminal object $0$ and $0 \to Y$ is the unique morphism out of the initial object $0$. This morphism is independent of the choice of the zero object $0$.
\end{definition}


\input{../_definitions/definition_additive_category_preadditive_category.tex}
\begin{lemma} \label{lemma:finite_products_and_finite_coproducts_coincide_in_preadditive_categories}
    Let $\calA$ be a \CrefAndHyperrefIfExist{definition:additive_category}{preadditive category}. Finite \CrefAndHyperrefIfExist{definition:product_and_coproduct_of_objects_in_a_category}{products} in $\calA$ coincide with finite \CrefAndHyperrefIfExist{definition:product_and_coproduct_of_objects_in_a_category}{coproducts}. More precisely, if $\{A_i\}_{i = 1}^n$ is a finite collection of objects of $\calA$, then 
    \begin{enumerate}
        \item if $\prod_{i = 1}^n A_i$ exists, then so does $\coprod_{i=1}^n A_i$ and these are naturally isomorphic. 

        \item if $\coprod_{i=1}^n A_i$, then so does $\prod_{i=1}^n A_i$ and these are naturally isomorphic. 
    \end{enumerate}
\end{lemma}

\begin{proof}
    \TODO{}
\end{proof}

In fact, morphisms between modules have kernels, cokernels, and nice images and coimages


\begin{definition} \label{definition:kernel_and_cokernel_of_a_morphism_in_a_category}
Let $\mathcal{C}$ be a \CrefAndHyperrefIfExist{definition:category}{(large)} \CrefAndHyperrefIfExist{definition:pointed_category}{pointed category}, i.e. a category with a \CrefAndHyperrefIfExist{definition:initial_final_zero_objects_of_a_category}{zero object} $0$. Let $X,Y \in \mathrm{Ob}(\mathcal{C})$ be an object and let $f: X \to Y$ be a morphism. 

\begin{enumerate}
    \item A morphism $i: K \to X$ is called the \hldef{kernel of $f$} if:
    \begin{enumerate}
        \item $f \circ i = 0$, where $0$ is the \CrefAndHyperrefIfExist{definition:zero_morphism_in_a_pointed_category}{zero morphism} $K \to Y$,
        \item for any morphism $g: Z \to X$ such that $f \circ g = 0$, there exists a unique morphism $u: Z \to K$ such that $g = i \circ u$.
    \end{enumerate}
    The kernel, if it exists, is unique up to unique \CrefAndHyperrefIfExist{definition:isomorphism_in_a_category}{isomorphism}. \hl{$\ker(f)$} denotes the object $K$ determined (up to isomorphism) by a kernel of $f$.

    \TextIfExists{definition:equalizer_and_coequalizer_of_morphisms_in_a_category}{
        Equivalently, $\ker(f)$ is the \CrefAndHyperref{definition:equalizer_and_coequalizer_of_morphisms_in_a_category}{equalizer} of $f$ and the $0$ morphism $X \to Y$.
    }

    \item a morphism $p: Y \to Q$ is called the \hldef{cokernel of $f$} if:
    \begin{enumerate}
        \item $p \circ f = 0$, where $0$ is the \CrefAndHyperrefIfExist{definition:initial_final_zero_objects_of_a_category}{zero morphism} $X \to Q$,
        \item for any morphism $g: Y \to Z$ such that $g \circ f = 0$, there exists a unique morphism $v: Q \to Z$ such that $g = v \circ p$.
    \end{enumerate}
    The cokernel, if it exists, is unique up to unique isomorphism. \hl{$\operatorname{coker}(f)$} denotes the object $Q$ determined (up to isomorphism) by a cokernel of $f$.

    \TextIfExists{definition:equalizer_and_coequalizer_of_morphisms_in_a_category}{
        Equivalently, $\coker(f)$ is the \CrefAndHyperref{definition:equalizer_and_coequalizer_of_morphisms_in_a_category}{coequalizer} of $f$ and the $0$ morphism $X \to Y$.
    }

\end{enumerate}

\end{definition}


\begin{definition}[Monomorphism and Epimorphism in Categories] \label{definition:monomorphism_and_epimorphism_in_categories}
Let $\mathcal{C}$ be a \CrefAndHyperrefIfExist{definition:category}{category}. For objects $A, B \in \mathcal{C}$, let $f: A \to B$ be a morphism in $\mathcal{C}$.  
\begin{itemize}
    \item The morphism $f$ is called a \hldef{monomorphism} (or a \hldef{monic morphism}) if for every object $X$ and every pair of morphisms $g_1, g_2 : X \to A$, the equality $f \circ g_1 = f \circ g_2$ implies $g_1 = g_2$.  
    \item The morphism $f$ is called an \hldef{epimorphism} (or an \hldef{epic morphism}) if for every object $Y$ and every pair of morphisms $h_1, h_2: B \to Y$, the equality $h_1 \circ f = h_2 \circ f$ implies $h_1 = h_2$.  
\end{itemize}
\end{definition}

\begin{definition} \label{definition:image_coimage_of_a_morphism_in_a_category}
Let $\mathcal{C}$ be a \CrefAndHyperrefIfExist{definition:category}{category}, and let $f: A \to B$ be a morphism in $\mathcal{C}$. 
\begin{enumerate}
    \item 
    An \hldef{image of $f$} consists of an object $I \in \mathrm{Ob}(\mathcal{C})$ together with a factorization of $f$ into two morphisms
        \begin{align*}
        A \xrightarrow{e} I \xrightarrow{m} B,
        \end{align*}
    where $e$ is an \CrefAndHyperrefIfExist{definition:monomorphism_and_epimorphism_in_categories}{epimorphism} and $m$ is a \CrefAndHyperrefIfExist{definition:monomorphism_and_epimorphism_in_categories}{monomorphism}, such that for any other factorization
        \begin{align*}
        A \xrightarrow{e'} I' \xrightarrow{m'} B
        \end{align*}
    with $e'$ epi and $m'$ mono, there exists a unique isomorphism $\varphi: I \simeq I'$ satisfying $m = m'\varphi$ and $\varphi e = e'$.
    \begin{center}
    \begin{tikzcd}
        & I \arrow[dr, "m", hook] \arrow[dd, "\exists ! \varphi", dashed, "\sim"' {sloped} ] & \\
        A \arrow[ur, "e", two heads] \arrow[dr, "e'"', two heads] & & B \\
        & I' \arrow[ur, "m'"', hook] &
    \end{tikzcd}
    \end{center}
    The monomorphism $m: I \to B$ (or equivalently its subobject class) is called the \hldef{image of $f$ in $\mathcal{C}$}.

    \item
    Let $\mathcal{C}$ be a \CrefAndHyperrefIfExist{definition:category}{category}, and let $f: A \to B$ be a morphism in $\mathcal{C}$. A \hldef{coimage of $f$} consists of an object $C \in \mathrm{Ob}(\mathcal{C})$ together with a factorization of $f$ into two morphisms
        \begin{align*}
        A \xrightarrow{e} C \xrightarrow{m} B,
        \end{align*}
    where $e$ is an epimorphism and $m$ is a monomorphism, such that for any other factorization
        \begin{align*}
        A \xrightarrow{e'} C' \xrightarrow{m'} B
        \end{align*}
    with $e'$ epi and $m'$ mono, there exists a unique isomorphism $\varphi: C \simeq C'$ satisfying $m = m'\varphi$ and $\varphi e = e'$.
    \begin{center}
    \begin{tikzcd}
        & C \arrow[dr, "m", hook] \arrow[dd, "\exists ! \varphi", dashed, "\sim"' {sloped}] & \\
        A \arrow[ur, "e", two heads] \arrow[dr, "e'"', two heads] & & B \\
        & C' \arrow[ur, "m'"', hook] &
    \end{tikzcd}
    \end{center}
    The epimorphism $e: A \to C$ (or equivalently its quotient class) is called the \hldef{coimage of $f$ in $\mathcal{C}$}.


\end{enumerate}
\end{definition}

\begin{lemma} \label{lemma:kernel_cokernel_image_coimage_of_modules_over_rings_are_categorical}
    Let $R,S$ be \CrefAndHyperrefIfExist{definition:ring}{(not-necessarily commutative) rings with unity}, let $M,N$ be \CrefAndHyperrefIfExist{definition:module_of_a_ring}{$R$-$S$-bimodules}, and let $f: M \to N$ be a \CrefAndHyperrefIfExist{definition:homomorphism_of_modules_over_a_ring}{module homomorphism}.
    The \CrefAndHyperrefIfExist{definition:kernel_image_cokernel_coimage_of_a_module_homomorphism}{kerel, cokernel, image, and coimage} of $f$ are respectively the categorical \CrefAndHyperrefIfExist{definition:kernel_and_cokernel_of_a_morphism_in_a_category}{kernel, cokernel}, \CrefAndHyperrefIfExist{definition:image_coimage_of_a_morphism_in_a_category}{image, and coimage}. 
\end{lemma}

\begin{definition}[Abelian category] \label{definition:abelian_category}
Let $\mathcal{A}$ be a category. The category $\mathcal{A}$ is an \hldef{abelian category} if:
\begin{itemize}
    \item $\mathcal{A}$ is an \CrefAndHyperrefIfExist{definition:additive_category_preadditive_category}{additive category}.

    \item Every morphism $f: A \to B$ has a \CrefAndHyperrefIfExist{definition:kernel_and_cokernel_of_a_morphism_in_a_category}{kernel $\ker(f)$ and a cokernel $\operatorname{coker}(f)$}.

    \item For every morphism $f: A \to B$, the canonical morphism $\operatorname{coim}(f) \to \operatorname{im}(f)$ is an isomorphism, where
    $$
    \operatorname{coim}(f) = \operatorname{coker}(\ker(f) \to A),\quad \operatorname{im}(f) = \ker(B \to \operatorname{coker}(f)).
    $$
    \TODO{I think I need to re-check this defintion}
    \TODO{coimage}
\end{itemize}

\TextIfExists{definition:pre_abelian_category}{In particular, every abelian category is \Cref{definition:pre_abelian_category}{pre-abelian}}.

It is also worth considering Grothendieck's additional axioms for abelian categories\CrefIfExists{definition:grothendiecks_additional_axioms_for_abelian_categories}.

\end{definition}


\begin{proposition} \label{proposition:diagram_category_of_a_preadditive_additive_abelian_category_indexed_by_a_small_category_is_preadditive_additive_abelian}
    Let $\calA$ be a \CrefAndHyperrefIfExist{definition:additive_category}{preadditive} (resp. \CrefAndHyperrefIfExist{definition:additive_category}{additive}, \CrefAndHyperrefIfExist{definition:abelian_category}{abelian}) category and let $J$ be a \CrefAndHyperrefIfExist{definition:locally_small_category}{small} category. The \CrefAndHyperrefIfExist{definition:diagram_in_a_category_indexed_by_a_small_category}{diagram category} $\calA^J$ is preadditive (resp. additive, abelian).
\end{proposition}


\subsection{Additive functors between additive categories}

\begin{definition}[Additive functor] \label{definition:additive_functor_between_additive_categories}

    \begin{enumerate}
        \item Let $\mathcal{A}$ and $\mathcal{B}$ be \hyperrefIfExists{definition:additive_category_preadditive_category}{pre-additive categories}. A functor
        $$ F: \mathcal{A} \to \mathcal{B} $$
        is an \hldef{additive functor} if for every pair of objects $A, A' \in \mathcal{A}$, the induced map
        $$ F_{A,A'}: \operatorname{Hom}_{\mathcal{A}}(A, A') \to \operatorname{Hom}_{\mathcal{B}}(F(A), F(A')) $$
        is a group homomorphism of abelian groups, or equvialently if it is \CrefAndHyperrefIfExist{definition:category_enriched_in_a_monoidal_category}{enriched over the category $\Ab$ of abelian groups}.
        
        \item Let $\mathcal{A}$ and $\mathcal{B}$ be \hyperrefIfExists{definition:additive_category_preadditive_category}{additive categories}. A functor
        $$ F: \mathcal{A} \to \mathcal{B} $$
        is an \hldef{additive functor} if it an additive functor of pre-additive categories and satisfies the following:
        \begin{itemize}
            \item $F$ sends the zero object $0_{\mathcal{A}}$ of $\mathcal{A}$ to the zero object $0_{\mathcal{B}}$ of $\mathcal{B}$, i.e.,
            $$ F(0_{\mathcal{A}}) = 0_{\mathcal{B}}.  $$
            \item $F$ preserves finite direct sums: For any finite family of objects $\{A_i\}_{i=1}^n$ in $\mathcal{A}$,
            $$ F\left(\bigoplus_{i=1}^n A_i\right) \cong \bigoplus_{i=1}^n F(A_i) $$
            via the canonical isomorphism induced by $F$ applied to the canonical injections and projections.
        \end{itemize}
        In other words, $F$ is a functor that is compatible with the additive structures on $\mathcal{A}$ and $\mathcal{B}$.
    \end{enumerate}
\end{definition}


\TODO{examples of additive functors}

We note that Hom's and tensor products induce \CrefAndHyperrefIfExist{definition:n_ary_additive_functor_between_additive_categories}{bi-additive functors} on categories of modules.

\begin{definition}[n-ary Additive Functor] \label{definition:n_ary_additive_functor_between_additive_categories}
Let $I$ be a finite set with $|I| = n$. Let $\{\mathcal{A}_i\}_{i\in I}$ be \CrefAndHyperrefIfExist{definition:additive_category_preadditive_category}{additive categories} and let $\mathcal{B}$ be an additive category. An \hldef{n-ary additive functor} (or \hldef{multilinear functor})
$$F : \prod_{i\in I}\mathcal{A}_i \to \mathcal{B}$$
\CrefIfExists{definition:product_category_of_a_family_of_categories} is a functor such that for each fixed collection of all but one variable, the resulting functor in the remaining variable is \CrefAndHyperrefIfExist{definition:additive_functor_between_additive_categories}{additive}. Equivalently, for every $j\in I$ and objects $(A_i)_{i\in I}$ and morphisms $f_1,f_2:A_j\to A'_j$ in $\mathcal{A}_j$, we have
\begin{align*}
&F(A_1,\dots,A_{j-1}, f_1+f_2, A_{j+1},\dots,A_n)
\\
=& F(A_1,\dots,A_{j-1}, f_1, A_{j+1},\dots,A_n)  \\
& + F(A_1,\dots,A_{j-1}, f_2, A_{j+1},\dots,A_n),
\end{align*}
and $F$ preserves zero morphisms componentwise:
$$F(A_1,\dots,0_{A_j,A'_j},\dots,A_n) = 0_{F(A_1,\dots),F(A'_1,\dots)}.$$
A bifunctor that satisfies this property for $n=2$ is simply called a \hldef{biadditive functor}.
\end{definition}

\subsubsection{Exact functors between abelian categories}

One particularly nice kind of additive functor is an exact functor

\begin{definition} \label{definition:short_exact_sequence_in_an_additive_category}
Let $\mathcal{A}$ be an \CrefAndHyperrefIfExist{definition:additive_category_preadditive_category}{additive category}. A sequence 
$$0 \to A \xrightarrow{f} B \xrightarrow{g} C \to 0$$ 
of moprhisms in $\mathcal{A}$ is called a \hldef{short exact sequence} if the morphisms satisfy:
\begin{itemize}
  \item $f : A \to B$ is a \CrefAndHyperrefIfExist{definition:monomorphism_and_epimorphism_in_categories}{monomorphism} and is the \CrefAndHyperrefIfExist{definition:kernel_and_cokernel_of_a_morphism_in_a_category}{kernel of $g$},
  \item $g : B \to C$ is an \CrefAndHyperrefIfExist{definition:monomorphism_and_epimorphism_in_categories}{epimorphism} and is the \CrefAndHyperrefIfExist{definition:kernel_and_cokernel_of_a_morphism_in_a_category}{cokernel of $f$},
  \item the sequence is \CrefAndHyperrefIfExist{definition:acyclic_complex_of_objects_in_an_abelian_category}{exact at} $B$, meaning \CrefAndHyperrefIfExist{definition:image_coimage_of_a_morphism_in_a_category}{$\mathrm{Im}(f) = \mathrm{Ker}(g)$}.
\end{itemize}
This means the sequence starts and ends with the zero object and is exact everywhere.
\end{definition}
\begin{definition} \label{definition:left_right_exact_functor_between_categories}
Let $\mathcal{C}$ and $\mathcal{D}$ be categories.
\begin{enumerate}
    \item Assume $\mathcal{C}$ admits all finite \CrefAndHyperrefIfExist{definition:limit_and_colimit_of_a_diagram_in_a_category}{limits}. A functor $F: \mathcal{C} \to \mathcal{D}$ is called \hldef{left exact} if it preserves all finite limits. Explicitly, for every finite diagram $D: \mathcal{J} \to \mathcal{C}$, the canonical map
    $$ F(\varprojlim D) \xrightarrow{\cong} \varprojlim (F \circ D) $$
    is an isomorphism.
    
    \item Assume $\mathcal{C}$ admits all finite \CrefAndHyperrefIfExist{definition:limit_and_colimit_of_a_diagram_in_a_category}{colimits}. A functor $F: \mathcal{C} \to \mathcal{D}$ is called \hldef{right exact} if it preserves all finite colimits. Explicitly, for every finite diagram $D: \mathcal{J} \to \mathcal{C}$, the canonical map
    $$ \varinjlim (F \circ D) \xrightarrow{\cong} F(\varinjlim D) $$
    is an isomorphism.
    
    \item If $\mathcal{C}$ admits both finite limits and finite colimits, a functor $F$ is called \hldef{exact} if it is both left exact and right exact.
\end{enumerate}
\end{definition}

\begin{definition} \label{definition:exact_functor_between_abelian_categories}
    Let $F: \mathcal{A} \to \mathcal{B}$ be an \hyperrefIfExists{definition:additive_functor_between_additive_categories}{additive functor}\CrefIfExists{definition:additive_functor_between_additive_categories} between \hyperrefIfExists{definition:abelian_category}{abelian categories}\CrefIfExists{definition:abelian_category}.
    \begin{enumerate}

        \item $F$ is called \hldef{left exact} if it preserves all \CrefAndHyperrefIfExist{definition:limit_and_colimit_of_a_diagram_in_a_category}{finite limits}, or equivalently it preserves \CrefAndHyperrefIfExist{definition:kernel_and_cokernel_of_a_morphism_in_a_category}{kernels} and any finite limit diagrams. Equivalently, for every left exact sequence in $\mathcal{A}$
        \[
        0 \to A' \xrightarrow{f} A \xrightarrow{g} A''
        \]
        the sequence
        \[
        0 \to F(A') \xrightarrow{F(f)} F(A) \xrightarrow{F(g)} F(A'')
        \]
        is exact at $F(A')$ and $F(A)$ (i.e., $F$ preserves \CrefAndHyperrefIfExist{definition:monomorphism_and_epimorphism_in_categories}{monomorphisms} and exactness at the first two terms).

        \item Dually, $F$ is called \hldef{right exact} if it preserves all \CrefAndHyperrefIfExist{definition:limit_and_colimit_of_a_diagram_in_a_category}{finite colimits}, or equivalently it preserves \CrefAndHyperrefIfExist{definition:kernel_and_cokernel_of_a_morphism_in_a_category}{cokernels} and any finite colimit diagrams. Equivalently, for every right exact sequence in $\mathcal{A}$
        \[
        A' \xrightarrow{f} A \xrightarrow{g} A'' \to 0,
        \]
        the sequence
        \[
        F(A') \xrightarrow{F(f)} F(A) \xrightarrow{F(g)} F(A'') \to 0
        \]
        is exact at $F(A)$ and $F(A'')$ (i.e., $F$ preserves \CrefAndHyperrefIfExist{definition:monomorphism_and_epimorphism_in_categories}{epimorphisms} and exactness at the last two terms).

        \item $F$ is called \hldef{exact} if it is both left and right exact.
    \end{enumerate}

    \TextIfExists{definition:left_right_exact_functor_between_categories}{
        The additive functor $F$ is left/right exact if and only if it is \CrefAndHyperref{definition:left_right_exact_functor_between_categories}{left/right exact} in the more general sense, i.e. if it preserves all \CrefAndHyperrefIfExist{definition:small_and_finite_limits_and_colimits_in_a_category}{finite} \CrefAndHyperrefIfExist{definition:limit_and_colimit_of_a_diagram_in_a_category}{limits/colimits}

    }
\end{definition}

\begin{lemma} \label{lemma:additive_functor_between_abelian_categories_is_exact_if_and_only_if_it_preserves_kernels_and_cokernels}
    Let \[ F: \mathcal{A} \to \mathcal{B} \] be an \hyperrefIfExists{definition:additive_functor_between_additive_categories}{additive functor}\CrefIfExists{definition:additive_functor_between_additive_categories} between \hyperrefIfExists{definition:abelian_category}{abelian categories}\CrefIfExists{definition:abelian_category}. It is \CrefAndHyperrefIfExist{definition:exact_functor_between_abelian_categories}{exact} if and only if it preserves \CrefAndHyperrefIfExist{definition:kernel_and_cokernel_of_a_morphism_in_a_category}{kernels and cokernels}.
\end{lemma}

\begin{theorem}[Freyd-Mitchell Embedding Theorem] \label{theorem:freyd_mitchell_embedding_theorem_for_small_abelian_categories}
    Let $\mathcal{A}$ be a \CrefAndHyperrefIfExist{definition:locally_small_category}{small} \CrefAndHyperrefIfExist{definition:abelian_category}{abelian category}.
    There exists a \CrefAndHyperrefIfExist{definition:ring}{ring} $R$ (which may not be commutative) and a functor $F: \mathcal{A} \to \operatorname{Mod}_R$\CrefIfExists{theorem:category_of_modules_over_a_sheaf_of_rings_on_a_site_on_an_essentially_small_category_has_enough_injectives} such that:
    \TODO{Show thta exact functors preserve finite limits and colimits}
    \begin{enumerate}
        \item $F$ is \CrefAndHyperrefIfExist{definition:exact_functor_between_abelian_categories}{exact}, meaning it preserves all finite limits and colimits (in particular, kernels, cokernels, and exact sequences).
        \item $F$ is \CrefAndHyperrefIfExist{definition:full_and_faithful_functor_between_locally_small_categories}{fully faithful}.
    \end{enumerate}
    Consequently, any diagram-chasing argument valid for modules over a ring is also valid in any small abelian category, and by extension (using the fact that any exact diagram involves only a set of objects), in any abelian category.
\end{theorem}

% \begin{theorem}[Freyd-Mitchell Embedding Theorem] \label{theorem:freyd_mitchell_embedding_theorem}
%     Let $\calA$ be a \CrefAndHyperrefIfExist{definition:locally_small_category}{small} \CrefAndHyperrefIfExist{definition:abelian_category}{abelian category}.
%     There exists a \CrefAndHyperrefIfExist{definition:ring}{ring} $R$ and an \CrefAndHyperrefIfExist{definition:exact_functor_between_abelian_categories}{exact}, \CrefAndHyperrefIfExist{definition:full_and_faithful_functor_between_locally_small_categories}{fully faithful} functor from $\calA$ into the category \CrefAndHyperrefIfExist{definition:category_of_modules_and_bimodules_over_rings}{$R\mathrm{-mod}$} of \CrefAndHyperrefIfExist{definition:module_of_a_ring}{left $R$-modules}.
% \end{theorem}


The Snake lemma is an application to the \CrefAndHyperrefIfExist{theorem:freyd_mitchell_embedding_theorem_for_small_abelian_categories}{Freyd-Mitchell embedding theorem}. More specifically, while one can prove the snake lemma for general abelian categories directly, one can alternatively first prove the snake lemma for the category of $R$-modules, then make use of the theorem to prove it in general. 


\begin{definition}[Reflecting a type of morphism] \label{definition:reflects_a_type_of_morphism_for_a_functor_between_categories}
Let $F : \mathcal{C} \to \mathcal{D}$ be a \CrefAndHyperrefIfExist{definition:functor_between_categories}{functor between (large) categories}, and let $\mathcal{P}$ be a property of morphisms (or more generally a property of sequences or families of morphisms) that is stable under \CrefAndHyperrefIfExist{definition:isomorphism_in_a_category}{isomorphism} (e.g. \CrefAndHyperrefIfExist{definition:monomorphism_and_epimorphism_in_categories}{monomorphism, epimorphism}, isomorphism, etc.). We say that $F$ \hldef{reflects $\mathcal{P}$-morphisms} if for every morphism $f : x \to y$ in $\mathcal{C}$, whenever $F(f)$ has property $\mathcal{P}$ in $\mathcal{D}$, it follows that $f$ has property $\mathcal{P}$ in $\mathcal{C}$.
\end{definition}

\begin{proposition} \label{proposition:full_and_faithful_additive_functor_between_abelian_categories_reflects_exactness}
Let $F: \mathcal{A} \to \mathcal{B}$ be a \CrefAndHyperrefIfExist{definition:full_and_faithful_functor_between_locally_small_categories}{full and faithful} \CrefAndHyperrefIfExist{definition:additive_functor_between_additive_categories}{additive functor} between \CrefAndHyperrefIfExist{definition:abelian_category}{abelian categories}. Then $F$ \CrefAndHyperrefIfExist{definition:reflects_a_type_of_morphism_for_a_functor_between_categories}{reflects exactness}, i.e., given a sequence
\[
0 \to A' \xrightarrow{f} A \xrightarrow{g} A'' \to 0
\]
in $\mathcal{A}$, if the sequence
\[
0 \to F(A') \xrightarrow{F(f)} F(A) \xrightarrow{F(g)} F(A'') \to 0
\]
is exact in $\mathcal{B}$, then the original sequence is exact in $\mathcal{A}$. More generally, $F$ reflects any \CrefAndHyperrefIfExist{definition:acyclic_complex_of_objects_in_an_abelian_category}{exact sequence} and all limits and colimits that exist in $\calB$ for diagrams coming from $\calA$.
\end{proposition}
\begin{lemma} \label{lemma:an_additive_functors_between_additive_categories_is_faithful_exact_if_it_only_sends_the_zero_morphism_to_a_zero_morphism}
    Let $F: \calA \to \calB$ be an \CrefAndHyperrefIfExist{definition:additive_functor_between_additive_categories}{additive functor} between \CrefAndHyperrefIfExist{definition:additive_category}{additive categories}. The functor $F$ is \CrefAndHyperrefIfExist{definition:full_and_faithful_functor_between_locally_small_categories}{faithful} if and only if for any morphism $f$ in $\calA$, we have $F(f) = 0$ exactly when $f = 0$.
\end{lemma}

\begin{proof}
    To say that $F$ is faithful means that for every pair of objects $X$ and $Y$ in $\calA$, the induced \CrefAndHyperrefIfExist{definition:group}{abelian group} \CrefAndHyperrefIfExist{definition:group_homomorphism}{homomorphism}
    $$F_{X,Y}: \Hom_\calA(X,Y) \to \Hom_\calB(F(X), F(Y))$$
    on the Hom-sets is \CrefAndHyperrefIfExist{definition:injective_surjective_bijective_map_of_sets}{injective}. Equivalently, this means that for every morphism $f: X \to Y$ in $\calA$, we have $F(f) = 0$ if and only if $f = 0$.
\end{proof}

\begin{proposition} \label{proposition:exact_functor_between_abelian_categories_is_faithful_if_and_only_if_reflects_short_exact_sequences}
    Let $F: \calA \to \calB$ be an \CrefAndHyperrefIfExist{definition:exact_functor_between_abelian_categories}{exact functor} between \CrefAndHyperrefIfExist{definition:abelian_category}{abelian categories}. The functor $F$ is \CrefAndHyperrefIfExist{definition:full_and_faithful_functor_between_locally_small_categories}{faithful} if and only if it \CrefAndHyperrefIfExist{definition:reflects_a_type_of_morphism_for_a_functor_between_categories}{reflects} \CrefAndHyperrefIfExist{definition:short_exact_sequence_in_an_additive_category}{short exact sequences}.
\end{proposition}


\begin{proof}
    Suppose $F$ is faithful. Let
    \[
    A_1 \xrightarrow{f} A_2 \xrightarrow{g} A_3
    \]
    be a sequence in $\mathcal{A}$ such that
    \[
    F(A_1) \xrightarrow{F(f)} F(A_2) \xrightarrow{F(g)} F(A_3)
    \]
    is exact in $\mathcal{B}$. Because $F$ is exact, it preserves kernels and images, so
    \[
    F(\ker(g)) = \ker(F(g)), \quad F(\mathrm{im}(f)) = \mathrm{im}(F(f)).
    \]

    Exactness in $\mathcal{B}$ gives
    \[
    \ker(F(g)) = \mathrm{im}(F(f)),
    \]
    so
    \[
    F(\ker(g)) = F(\mathrm{im}(f)).
    \]

    Since $F$ is faithful, it \CrefAndHyperrefIfExist{definition:reflects_a_type_of_morphism_for_a_functor_between_categories}{reflects} \CrefAndHyperrefIfExist{definition:monomorphism_and_epimorphism_in_categories}{monomorphisms and epimorphisms} (\Cref{proposition:reflection_of_monomorphism_and_epimorphisms_by_faithful_functors}), and morphisms that become equal via $F$ must be equal in $\mathcal{A}$. Hence,
    \[
    \ker(g) = \mathrm{im}(f).
    \]
    Therefore, the sequence
    \[
    A_1 \xrightarrow{f} A_2 \xrightarrow{g} A_3
    \]
    is exact in $\mathcal{A}$, i.e., $F$ reflects short exact sequences.

    Conversely, suppose $F$ reflects short exact sequences. Let $f: A \to B$ be a morphism in $\mathcal{A}$ such that
    \[
    F(f) = 0.
    \]
    Consider the sequence
    \[
    0 \to A \xrightarrow{f} B.
    \]
    Since $F(f) = 0$, we have the sequence
    \[
    0 \to F(A) \xrightarrow{F(f)} F(B)
    \]
    is exact at $F(A)$. Because $F$ reflects short exact sequences, the original sequence must be exact at $A$, so $f$ is a monomorphism with zero image, hence $f=0$. Thus $F$ is faithful by \Cref{lemma:an_additive_functors_between_additive_categories_is_faithful_exact_if_it_only_sends_the_zero_morphism_to_a_zero_morphism}.
\end{proof}


\begin{lemma}[Snake lemma] \label{lemma:snake_lemma}
    Let $\calA$ be an \CrefAndHyperrefIfExist{definition:abelian_category}{abelian category}. Let
    Given the commutative diagram with exact rows:
    $$ \begin{array}{ccccccccc} & & A & \xrightarrow{f} & B & \xrightarrow{g} & C & \xrightarrow{} & 0 \\ & & \downarrow{a} & & \downarrow{b} & & \downarrow{c} & & \\ 0 & \xrightarrow{} & A' & \xrightarrow{f'} & B' & \xrightarrow{g'} & C' & & \\ \end{array} $$
    there exists an exact sequence connecting the kernels and cokernels:
    $$ \ker(a) \xrightarrow{\bar{f}} \ker(b) \xrightarrow{\bar{g}} \ker(c) \xrightarrow{d} \text{coker}(a) \xrightarrow{\bar{f}'} \text{coker}(b) \xrightarrow{\bar{g}'} \text{coker}(c) $$
    where $d$ is the connecting homomorphism. Furthermore, if $f$ is \CrefAndHyperrefIfExist{definition:monomorphism_and_epimorphism_in_categories}{monic}, then $\bar{f}$ is monic; if $g'$ is \CrefAndHyperrefIfExist{definition:monomorphism_and_epimorphism_in_categories}{epi}, then $\bar{g}'$ is epi.
\end{lemma}

\begin{proof}
    We first show this in the case that $\calA$ is the \CrefAndHyperrefIfExist{definition:category_of_modules_and_bimodules_over_rings}{category of $R$-modules} for any (not necessarily commutative) ring $R$. We perform a diagram chase on the elements of the modules.
    \begin{enumerate}
        \item \textbf{Construction of $d$}: Let $z \in \ker(c) \subseteq C$. Since $g$ is surjective, there exists $y \in B$ such that $g(y) = z$. By commutativity, $g'(b(y)) = c(g(y)) = c(z) = 0$. Thus $b(y) \in \ker(g')$. By exactness of the bottom row, there exists a unique $x' \in A'$ such that $f'(x') = b(y)$. Define $d(z) = [x'] \in A'/\text{im}(a) = \text{coker}(a)$. 

        We show that $[x']$ is independent of the choice of $y$. Suppose we choose another lift $y_1 \in B$ such that $g(y_1) = z$. Then $g(y - y_1) = g(y) - g(y_1) = z - z = 0$. By the exactness of the top row at $B$, there exists some $x \in A$ such that $f(x) = y - y_1$. Applying $b$, we have $b(y - y_1) = b(f(x))$. By the commutativity of the first square, $b(f(x)) = f'(a(x))$. Let $x'$ and $x'_1$ be the unique elements in $A'$ such that $f'(x') = b(y)$ and $f'(x'_1) = b(y_1)$. Then $f'(x' - x'_1) = b(y) - b(y_1) = b(y - y_1) = f'(a(x))$. Since $f'$ is injective (due to the exactness of the bottom row at $A'$), we must have $x' - x'_1 = a(x)$. This implies $x' \equiv x'_1 \pmod{\text{im}(a)}$. Therefore, $[x'] = [x'_1]$ in $\text{coker}(a)$, proving that the definition of $d(z)$ is independent of the choice of the preimage $y$.

        %Write $y_1 = y$ and say that $y_2 \in B$ is an element such that $g(y_2) = z$ as well. Note that $g(y_1-y_2) = z - z = 0$, so $y_1-y_2 \in \ker g = \operatorname{im} f$. Let $x \in A$ such that $f(x) = y_1-y_2$. When then have $f'(a(x)) = b(f(x)) = b(y_1-y_2)

        \item \textbf{Exactness at $\ker(b)$}: Clearly $\bar{g} \circ \bar{f} = 0$. If $y \in \ker(\bar{g})$, then $g(y) = 0$, so $y = f(x)$ for some $x \in A$. Then $f'(a(x)) = b(f(x)) = b(y) = 0$. Since $f'$ is injective, $a(x) = 0$, so $x \in \ker(a)$ and $y \in \text{im}(\bar{f})$.
        \item \textbf{Exactness at $\ker(c)$}: If $z = \bar{g}(y)$ for $y \in \ker(b)$, then in the construction of $d(z)$, we can choose this $y$. Since $b(y)=0$, we have $x'=0$, so $d(z)=0$. Conversely, if $d(z)=0$, then the lift $x'$ is in $\text{im}(a)$, say $x'=a(x)$. Then $b(f(x)) = f'(a(x)) = f'(x') = b(y)$. Thus $y - f(x) \in \ker(b)$ and $g(y - f(x)) = g(y) = z$.
        \item \textbf{Exactness at $\text{coker}(a)$}: Similar diagram chasing confirms $\bar{f}' \circ d = 0$ and the corresponding inclusion.

        We verify the exactness of the sequence at $\text{coker}(a)$, specifically the sequence $\ker(c) \xrightarrow{d} \text{coker}(a) \xrightarrow{\bar{f}'} \text{coker}(b)$. Let $z \in \ker(c)$. By the construction of the connecting homomorphism, $d(z) = [x']$ where $f'(x') = b(y)$ for some $y \in B$ with $g(y) = z$. Applying $\bar{f}'$, we get $\bar{f}'([x']) = [f'(x')] = [b(y)]$. Since $b(y) \in \text{im}(b)$, its class in $\text{coker}(b)$ is zero. Thus $\bar{f}' \circ d = 0$. \textbf{Inclusion $\ker(\bar{f}') \subseteq \text{im}(d)$}: Let $[x'] \in \text{coker}(a)$ such that $\bar{f}'([x']) = [0]$ in $\text{coker}(b)$. This implies $f'(x') \in \text{im}(b)$, so there exists $y \in B$ such that $b(y) = f'(x')$. We apply $g'$ to both sides: $g'(b(y)) = g'(f'(x'))$. Since the bottom row is exact, $g' \circ f' = 0$, so $g'(b(y)) = 0$. By commutativity of the second square, $c(g(y)) = g'(b(y)) = 0$. This means $g(y) \in \ker(c)$. Let $z = g(y)$. By the definition of the connecting homomorphism, $d(z)$ is found by lifting $z$ to $B$ (we choose $y$), applying $b$ (we get $b(y) = f'(x')$), and taking the preimage under $f'$ (which is $x'$). Thus, $d(z) = [x']$, which shows that $[x'] \in \text{im}(d)$.

        \item We verify the exactness of the sequence at $\text{coker}(b)$, specifically looking at the sequence $\text{coker}(a) \xrightarrow{\bar{f}'} \text{coker}(b) \xrightarrow{\bar{g}'} \text{coker}(c)$. Let $[x'] \in \text{coker}(a)$ where $x' \in A'$. Then $\bar{f}'([x']) = [f'(x')]$. Applying $\bar{g}'$, we have $\bar{g}'([f'(x')]) = [g'(f'(x'))]$. Since the bottom row is exact, $g' \circ f' = 0$, thus $[g'(f'(x'))] = [0]$, so $\bar{g}' \circ \bar{f}' = 0$. Let $[y'] \in \text{coker}(b)$ such that $\bar{g}'([y']) = [0]$ in $\text{coker}(c)$. This implies $g'(y') \in \text{im}(c)$, so there exists $z \in C$ such that $c(z) = g'(y')$. Since $g$ is surjective, there exists $y \in B$ such that $g(y) = z$. By commutativity, $g'(b(y)) = c(g(y)) = c(z) = g'(y')$. Thus $g'(y' - b(y)) = 0$, which means $y' - b(y) \in \ker(g')$. By exactness of the bottom row, there exists $x' \in A'$ such that $f'(x') = y' - b(y)$. Transitioning to the cokernel, $[y'] = [f'(x') + b(y)] = [f'(x')] + [b(y)]$. Since $b(y) \in \text{im}(b)$, its class $[b(y)] = 0$ in $\text{coker}(b)$. Therefore, $[y'] = [f'(x')] = \bar{f}'([x'])$, showing $[y'] \in \text{im}(\bar{f}')$.
    \end{enumerate}

    Let $\calB$ be the smallest \CrefAndHyperref{definition:full_subcategory_of_a_category}{full} abelian subcategory of $\calA$ containing the objects $A,B,C,A',B',C'$; it is a small category. Let choose a ring $R$ and a functor $F: \calB \to \Mod_R$ that is \CrefAndHyperrefIfExist{definition:exact_functor_between_abelian_categories}{exact} and \CrefAndHyperrefIfExist{definition:full_and_faithful_functor_between_locally_small_categories}{fully faithful} by the \CrefAndHyperrefIfExist{theorem:freyd_mitchell_embedding_theorem_for_small_abelian_categories}{Freyd-Mitchell Embedding Theorem}. Apply the functor $F$ to the diagram. Since $F$ is exact, it preserves kernels, cokernels, images, and the exactness of the rows. The result holds in $\Mod_R$ by the element-based proof above. Since $F$ is a full and faithful embedding, the morphism $d: \ker(c) \to \coker(a)$ is present in $\calB$. Moreover, by \Cref{proposition:full_and_faithful_additive_functor_between_abelian_categories_reflects_exactness}, $F$ \CrefAndHyperrefIfExist{definition:reflects_a_type_of_morphism_for_a_functor_between_categories}{reflects} exactness, so the exact sequence in $\Mod_R$ remains exact in $\calB$. In particular, the exact sequence exists and is exact in $\calA$.
\end{proof}


The five-lemma may also be proven using the Freyd-Mitchell embedding theorem

\begin{lemma} \label{lemma:five_lemma}
Consider the following commutative diagram in an \CrefAndHyperrefIfExist{definition:abelian_category}{abelian category} $\mathcal{A}$ with \CrefAndHyperrefIfExist{definition:short_exact_sequence_in_an_additive_category}{exact rows}:
$$
\begin{array}{ccccccccc}
A & \xrightarrow{f_1} & B & \xrightarrow{f_2} & C & \xrightarrow{f_3} & D & \xrightarrow{f_4} & E \\
\downarrow{\alpha} & & \downarrow{\beta} & & \downarrow{\gamma} & & \downarrow{\delta} & & \downarrow{\epsilon} \\
A' & \xrightarrow{g_1} & B' & \xrightarrow{g_2} & C' & \xrightarrow{g_3} & D' & \xrightarrow{g_4} & E' \\
\end{array}
$$
where $A, B, C, D, E, A', B', C', D', E'$ are objects and $\alpha, \beta, \gamma, \delta, \epsilon$ are morphisms.

In the commutative diagram defined above, the following statements hold:
\begin{enumerate}
    \item If $\beta$ and $\delta$ are monomorphisms and $\alpha$ is an epimorphism, then $\gamma$ is a monomorphism.
    \item If $\beta$ and $\delta$ are epimorphisms and $\epsilon$ is a monomorphism, then $\gamma$ is an epimorphism.
    \item If $\alpha, \beta, \delta, \epsilon$ are isomorphisms, then $\gamma$ is an isomorphism.
\end{enumerate}
\end{lemma}


\subsubsection{Grothendieck's additional axioms for abelian categories}

In fact, Grothendieck posed certain additional axioms that some abelian categories might enjoy.


\begin{definition}[Grothendieck's axioms for abelian categories (Ab1--Ab5)] \label{definition:grothendiecks_additional_axioms_for_abelian_categories}
Let $\mathcal{A}$ be an \CrefAndHyperrefIfExist{definition:abelian_category}{abelian category}.

Grothendieck introduced the following hierarchy of additional axioms to express stronger completeness and exactness properties in $\mathcal{A}$ --- we note that Ab1, Ab2, and Ab2\textsuperscript{*} are already satisfied for any abelian category:

\begin{itemize}

  \item \hldef{Ab1}: Every morphism in $\calA$ has a \CrefAndHyperrefIfExist{definition:kernel_and_cokernel_of_a_morphism_in_a_category}{kernel and a cokernel}.
  \item \hldef{Ab2}: Every \CrefAndHyperrefIfExist{definition:monomorphism_and_epimorphism_in_categories}{monic} in $\calA$ is the kernel of its cokernel. 
  \item \hldef{Ab2\textsuperscript{*}}: Every epi in $\calA$ is the cokernel of its kernel. 

  \item \hldef{AB3}: The category $\mathcal{A}$ is \CrefAndHyperrefIfExist{definition:complete_and_cocomplete_category}{cocomplete}.
  \begin{itemize}
    \item Since $\mathcal{A}$ is abelian (and hence \CrefAndHyperrefIfExist{lemma:equalizer_coequalizer_in_an_additive_category_are_given_by_kernel_and_cokernel}{admits} \CrefAndHyperrefIfExist{definition:equalizer_and_coequalizer_of_morphisms_in_a_category}{equalizers} as \CrefAndHyperrefIfExist{definition:kernel_image_cokernel_coimage_of_a_module_homomorphism}{kernels}), this is equivalent to requiring that $\mathcal{A}$ has all small \CrefAndHyperrefIfExist{definition:product_and_coproduct_of_objects_in_a_category}{coproducts} (direct sums).
  \end{itemize}

  \item \hldef{AB4}: The category $\mathcal{A}$ satisfies AB3, and coproducts are \emph{exact}.
  \begin{itemize}
    \item That is, the coproduct of a family of short exact sequences is a short exact sequence. Explicitly, for any family of short exact sequences $0 \to A_i \to B_i \to C_i \to 0$ indexed by a set $I$, the sequence
    \[ 0 \to \bigoplus_{i \in I} A_i \to \bigoplus_{i \in I} B_i \to \bigoplus_{i \in I} C_i \to 0 \]
    is exact in $\mathcal{A}$.
  \end{itemize}

  \item \hldef{AB5}: The category $\mathcal{A}$ satisfies AB3, and \CrefAndHyperrefIfExist{definition:projective_and_inductive_limits_in_categories}{filtered colimits} are \emph{exact}.
  \begin{itemize}
    \item Equivalently, for any \CrefAndHyperrefIfExist{definition:filtered_cofiltered_category}{filtered} index category $J$ and any \CrefAndHyperrefIfExist{definition:system_in_a_category_indexed_by_a_directed_poset}{directed system} of short exact sequences $0 \to A_j \to B_j \to C_j \to 0$, the colimit sequence
    \[ 0 \to \varinjlim A_j \to \varinjlim B_j \to \varinjlim C_j \to 0 \]
    is exact.
    \item Note: AB5 implies AB4. An abelian category satisfying AB5 and having a \CrefAndHyperrefIfExist{definition:generator_of_a_category}{generator} is called a \hldef{Grothendieck category}.
  \end{itemize}
  
  \item \hldef{AB6}: The category $\mathcal{A}$ satisfies AB3, and for any object $X$ and any family of filtered subobjects $\{F_i\}_{i \in I}$ of $X$ (where each $F_i$ is a filter of subobjects), the intersection commutes with the limit:
  \[ \bigcap_{i \in I} (\varinjlim_{j \in F_i} U_{i,j}) = \varinjlim_{(j_i) \in \prod F_i} (\bigcap_{i \in I} U_{i, j_i}). \]
  (This axiom is less commonly cited but appears in Grothendieck's Tohoku paper).

  \item \hldef{AB3\textsuperscript{*}}: The category \(\mathcal{A}\) is \CrefAndHyperrefIfExist{definition:complete_and_cocomplete_category}{complete} (i.e., has all small products).

  \item \hldef{AB4\textsuperscript{*}}: The category \(\mathcal{A}\) satisfies AB3\textsuperscript{*} and products are exact.
  \begin{itemize}
    \item Note: This is rarely satisfied for module categories (e.g., it fails for Abelian groups), but it is satisfied for the category of sheaves on a space.
  \end{itemize}

  \item \hldef{AB5\textsuperscript{*}}: The category \(\mathcal{A}\) satisfies AB3\textsuperscript{*} and filtered limits (inverse limits) are exact.
\end{itemize}

\textbf{Notes:}
\begin{itemize}
  \item AB5 implies AB4, and AB4 implies AB3.
  \item AB5\textsuperscript{*} implies AB4\textsuperscript{*}, and AB4\textsuperscript{*} implies AB3\textsuperscript{*}.
\end{itemize}
\end{definition}


% \begin{definition}[Grothendieck's axioms AB3--AB6] \label{definition:grothendiecks_ab_axioms}
% Let $\mathcal{A}$ be an \CrefAndHyperrefIfExist{definition:abelian_category}{abelian category}. (Recall that the axioms AB1 and AB2 refer to the existence of kernels/cokernels and the isomorphism between coimage and image, which are part of the definition of an abelian category).

% Grothendieck introduced the following hierarchy of additional axioms to express stronger completeness and exactness properties in $\mathcal{A}$:

% \begin{itemize}
%   \item \hldef{AB3}: The category $\mathcal{A}$ is \CrefAndHyperrefIfExist{definition:complete_and_cocomplete_category}{cocomplete}.
%   \begin{itemize}
%     \item Since $\mathcal{A}$ is abelian (hence has finite colimits), this is equivalent to requiring that $\mathcal{A}$ has all small \CrefAndHyperrefIfExist{definition:product_and_coproduct_of_objects_in_a_category}{coproducts} (direct sums).
%   \end{itemize}

%   \item \hldef{AB4}: The category $\mathcal{A}$ satisfies AB3, and coproducts are \emph{exact}.
%   \begin{itemize}
%     \item That is, the coproduct of a family of short exact sequences is a short exact sequence. Explicitly, for any family of short exact sequences $0 \to A_i \to B_i \to C_i \to 0$ indexed by a set $I$, the sequence
%     \[ 0 \to \bigoplus_{i \in I} A_i \to \bigoplus_{i \in I} B_i \to \bigoplus_{i \in I} C_i \to 0 \]
%     is exact in $\mathcal{A}$.
%   \end{itemize}

%   \item \hldef{AB5}: The category $\mathcal{A}$ satisfies AB3, and \CrefAndHyperrefIfExist{definition:projective_and_inductive_limits_in_categories}{filtered colimits} are \emph{exact}.
%   \begin{itemize}
%     \item Equivalently, for any \CrefAndHyperrefIfExist{definition:filtered_cofiltered_category}{filtered} index category $J$ and any \CrefAndHyperrefIfExist{definition:system_in_a_category_indexed_by_a_directed_poset}{directed system} of short exact sequences $0 \to A_j \to B_j \to C_j \to 0$, the colimit sequence
%     \[ 0 \to \varinjlim A_j \to \varinjlim B_j \to \varinjlim C_j \to 0 \]
%     is exact.
%     \item Note: AB5 implies AB4. An abelian category satisfying AB5 and having a \CrefAndHyperrefIfExist{definition:generator_of_a_category}{generator} is called a \hldef{Grothendieck category}.
%   \end{itemize}
  
%   \item \hldef{AB6}: The category $\mathcal{A}$ satisfies AB3, and for any object $X$ and any family of filtered subobjects $\{F_i\}_{i \in I}$ of $X$ (where each $F_i$ is a filter of subobjects), the intersection commutes with the limit:
%   \[ \bigcap_{i \in I} (\varinjlim_{j \in F_i} U_{i,j}) = \varinjlim_{(j_i) \in \prod F_i} (\bigcap_{i \in I} U_{i, j_i}). \]
%   (This axiom is less commonly cited but appears in Grothendieck's Tohoku paper).

%   \item \hldef{AB3\textsuperscript{*}}: The category \(\mathcal{A}\) is \CrefAndHyperrefIfExist{definition:complete_and_cocomplete_category}{complete} (i.e., has all small products).

%   \item \hldef{AB4\textsuperscript{*}}: The category \(\mathcal{A}\) satisfies AB3\textsuperscript{*} and products are exact.
%   \begin{itemize}
%     \item Note: This is rarely satisfied for module categories (e.g., it fails for Abelian groups), but it is satisfied for the category of sheaves on a space.
%   \end{itemize}

%   \item \hldef{AB5\textsuperscript{*}}: The category \(\mathcal{A}\) satisfies AB3\textsuperscript{*} and filtered limits (inverse limits) are exact.
% \end{itemize}

% \textbf{Notes:}
% \begin{itemize}
%   \item AB5 implies AB4, and AB4 implies AB3.
%   \item AB5\textsuperscript{*} implies AB4\textsuperscript{*}, and AB4\textsuperscript{*} implies AB3\textsuperscript{*}.
%   \item The condition you originally listed as "Ab2" (disjoint/universal sums) characterizes \emph{extensive categories} or \emph{toposes}, not abelian categories. In an abelian category, the coproduct is a biproduct and is never "disjoint" in the sense of set theory (unless $0=1$).
% \end{itemize}
% \end{definition}

\begin{definition}[Generator of a category] \label{definition:generator_of_a_category}
Let \(\mathcal{C}\) be a \CrefAndHyperrefIfExist{definition:category}{category}. 
\begin{enumerate}
    \item  An object \(G \in \mathcal{C}\) is called a \hldef{generator} (or \hldef{separator}) if for every pair of distinct morphisms \(f, g : X \to Y\) in \(\mathcal{C}\), there exists a morphism \(h : G \to X\) such that
    \[
    f \circ h \neq g \circ h.
    \]
    In case that $\calC$ is \CrefAndHyperrefIfExist{definition:locally_small_category}{locally small}, this is equivalent to the condition that the \CrefAndHyperrefIfExist{definition:representable_functor_on_a_category_enriched_in_a_monoidal_category}{representable functor}
    \[
    \mathrm{Hom}_{\mathcal{C}}(G, -) : \mathcal{C} \to \mathbf{Set}
    \]
    is \CrefAndHyperrefIfExist{definition:full_and_faithful_functor_between_locally_small_categories}{faithful}, 
    %
    % In other words, for every pair of distinct morphisms \(f, g : X \to Y\) in \(\mathcal{C}\), there exists a morphism \(h : G \to X\) such that
    % \[
    % f \circ h \neq g \circ h.
    % \]
    %
    which in turn is equivalent to the condition that for every object \(X \in \mathcal{C}\), there exists an epimorphism
    \[
    \bigoplus_{i \in I} G \twoheadrightarrow X
    \]
    for some indexing set \(I\), where \(\bigoplus\) denotes the \CrefAndHyperrefIfExist{definition:product_and_coproduct_of_objects_in_a_category}{coproduct} in \(\mathcal{C}\).

    \item A family \(\{G_i\}_{i \in I}\) is called a \hldef{generating family} if for every pair of distinct morphisms \(f, g : X \to Y\) in \(\mathcal{C}\), there exists some index \(i \in I\) and a morphism \(h : G_i \to X\) such that
    \[
    f \circ h \neq g \circ h.
    \]
    In case $\calC$ is locally small, this is equivalent to the condition that the collection of representable functors
    \[
    \{\mathrm{Hom}_{\mathcal{C}}(G_i, -) : \mathcal{C} \to \mathbf{Set}\}_{i \in I}
    \]
    is jointly faithful, which in turn is equivalent to the condition that for every object \(X \in \mathcal{C}\), there exists a family of objects \(\{G_i\}_{i \in J}\) from the generating set indexed by some set \(J\), and an epimorphism
    \[
    \bigoplus_{i \in J} G_i \twoheadrightarrow X.
    \]

\end{enumerate}
\end{definition}

\begin{theorem}[Examples of Grothendieck Categories] \label{theorem:examples_of_grothendieck_categories}
Examples of \CrefAndHyperrefIfExist{definition:grothendiecks_additional_axioms_for_abelian_categories}{Grothendieck categories} include:
\begin{itemize}
    \item The category of abelian groups,
    \item The category of $R$-$S$ bimodules where $R$,$S$ are \CrefAndHyperrefIfExist{definition:ring}{(not necessarily commutative) rings} (\Cref{theorem:the_category_of_R_S_bimodules_is_a_grothendieck_abelian_category_and_AB4_star})
    \item The category of \CrefAndHyperrefIfExist{definition:sheaf_on_a_site}{sheaves} of abelian groups on a \CrefAndHyperrefIfExist{definition:grothendieck_topology_on_a_category_site_covering_sieve_topologically_generating_family}{site} with a small \CrefAndHyperrefIfExist{definition:grothendieck_topology_on_a_category_site_covering_sieve_topologically_generating_family}{topologically generating family},
    \item The category of sheaves of \CrefAndHyperrefIfExist{definition:module_over_a_sheaf_of_rings_on_a_site}{$O_X$-modules} for any \CrefAndHyperrefIfExist{definition:ringed_space}{ringed space} $(X, O_X)$.
    \item The category of quasi-coherent sheaves on a scheme or algebraic stack. \TODO{quasi-coherent sheaves}
    \TODO{I need to figure out if for sheaves of abelian groups/sheaves of $\calO$-modules whether essentially smallness of the site is really necessary}
    \item The category of \CrefAndHyperrefIfExist{definition:sheaf_on_a_site}{sheaves} of abelian groups on an  \CrefAndHyperrefIfExist{definition:essentially_small_category}{essentially small} \CrefAndHyperrefIfExist{definition:grothendieck_topology_on_a_category_site_covering_sieve_topologically_generating_family}{site} $(C,J)$.

    \item (\cite[Expos\'e II, Proposition 6.7]{SGA4_I}) The category of sheaves of $\calO$-modules on an essentially small site (or an essentially $\mathscr{U}$-small site if a universe $\mathscr{U}$ is available) $(C,J)$. 
\end{itemize}
\end{theorem}
\begin{lemma} \label{lemma:construction_of_direct_limit_in_a_category_of_sets_with_finitary_operations}
    Let $\calC$ be a category of sets with finitary operations (e.g. the \CrefAndHyperrefIfExist{definition:category_of_sets}{category of sets}, the \CrefAndHyperrefIfExist{definition:category_of_modules_and_bimodules_over_rings}{category of modules over a ring}, the category of monoids, the category of semigroups, the category of groups, the category of Lie algebras, the category of Lie algebras, etc.) \TODO{define category of sets with finitary operations}.
    Let $F: I \to \calC$ be a small \CrefAndHyperrefIfExist{definition:projective_and_inductive_limits_in_categories}{direct system}. The \CrefAndHyperrefIfExist{definition:projective_and_inductive_limits_in_categories}{direct limit} $\varinjlim_{i \in I} F(i)$ can be constructed as the quotient of the \CrefAndHyperrefIfExist{definition:product_and_coproduct_of_objects_in_a_category}{coproduct} $\coprod_{i \in I} F(i)$ given by the following equivalence relation: for $i,j \in I$, the elements $x_i \in F(i)$ and $x_j \in F(j)$ are equivalent if there exists some $k$ with $i \leq k$ and $j \leq k$ such that $f_{ik}(x_i) = f_{jk}(x_j)$. In particular, the operations on the objects of $\calC$ induce natural operations on this set making it into an object of $\calC$.
\end{lemma}
\begin{theorem} \label{theorem:the_category_of_R_S_bimodules_is_a_grothendieck_abelian_category_and_AB4_star}
    Let $R,S$ be \CrefAndHyperrefIfExist{definition:ring}{(not necessarily commutative) rings}.
    The \CrefAndHyperrefIfExist{definition:category_of_modules_and_bimodules_over_rings}{category of} \CrefAndHyperrefIfExist{definition:module_of_a_ring}{$R$-$S$-bimodules} is a an \CrefAndHyperrefIfExist{definition:grothendiecks_additional_axioms_for_abelian_categories}{Grothendieck} category and an \CrefAndHyperrefIfExist{definition:grothendiecks_additional_axioms_for_abelian_categories}{$AB4*$} category.
\end{theorem}

\begin{proof}
    We handwave details.

    Given $R$-$S$-bimodules $M$ and $N$, the set $\operatorname{Hom}_{{}_R \mathrm{Mod}_S}(M,N)$ is an abelian group. Moreover, there is a $0$-object, namely the zero module in ${}_R \mathrm{Mod}_S$. Therefore, ${}_R \mathrm{Mod}_S$ is \CrefAndHyperrefIfExist{definition:additive_category}{preadditive}. The direct sum of finitely many $R$-$S$-bimodules is their \CrefAndHyperrefIfExist{definition:coproduct_of_modules_of_rings}{coproduct}. Therefore, ${}_R \mathrm{Mod}_S$ is \CrefAndHyperrefIfExist{definition:additive_category}{additive}.

    Given a \CrefAndHyperrefIfExist{definition:homomorphism_of_modules_over_a_ring}{morphism} $f: M \to N$ be $R$-$S$-bimodules, \CrefAndHyperrefIfExist{definition:kernel_image_cokernel_coimage_of_a_module_homomorphism}{$\ker f$ and $\operatorname{coker} f$} are the \CrefAndHyperrefIfExist{definition:kernel_and_cokernel_of_a_morphism_in_a_category}{categorical kernel and cokernel} of $f$ in \CrefAndHyperrefIfExist{definition:category_of_modules_and_bimodules_over_rings}{${}_R \mathrm{Mod}_S$}. Moreover, the \CrefAndHyperrefIfExist{definition:monomorphism_and_epimorphism_in_categories}{monomorphisms} in ${}_R \mathrm{Mod}_S$ are the injective module homomorphisms $f: M \to N$; such an $f$ is the kernel of its cokernel. In other words, ${}_R \mathrm{Mod}_S$ satisfies $AB1$ and $AB2$ and hence is an \CrefAndHyperrefIfExist{definition:abelian_category}{abelian category}.

    Moreover, small \CrefAndHyperrefIfExist{definition:product_and_coproduct_of_objects_in_a_category}{coproducts} exist in ${}_R \mathrm{Mod}_S$ (\Cref{definition:coproduct_of_modules_of_rings}), and it is easy to see that they are in fact exact, so ${}_R \mathrm{Mod}_S$ satisfies $AB3$ and $AB4$. To show that filtered colimits in ${}_R \mathrm{Mod}_S$ are exact, we first note that \CrefAndHyperrefIfExist{definition:small_and_finite_limits_and_colimits_in_a_category}{small} \CrefAndHyperrefIfExist{definition:projective_and_inductive_limits_in_categories}{colimits} \CrefAndHyperrefIfExist{theorem:limit_and_colimit_are_left_right_adjoint_to_diagonal_functor_for_locally_small_base_and_small_index}{are} \CrefAndHyperrefIfExist{definition:adjoint_functors_between_categories_unit_counit_of_adjoint_functors}{left adjoint} and \CrefAndHyperrefIfExist{proposition:left_right_adjoint_is_right_left_exact}{hence} is \CrefAndHyperrefIfExist{definition:exact_functor_between_abelian_categories}{right exact}; for any small index category $J$ and any system of short exact sequences $0 \to A_j \to B_j \to C_j \to 0$, the sequence $0 \to A \to B \to C \to 0$ of \CrefAndHyperrefIfExist{definition:diagram_in_a_category_indexed_by_a_small_category}{diagrams} is exact by \Cref{proposition:exact_sequence_in_a_diagram_category_of_an_abelian_category_indexed_by_a_small_category_is_given_by_componentwise_exact_sequences}, so applying the \CrefAndHyperrefIfExist{theorem:limits_and_colimits_as_functors_from_functor_category_to_value_category}{colimit functor} yields a right exact sequence 
    $$\colim_{j \in J} A_j \to \colim_{j \in J} B_j \to \colim_{j \in J} C_j \to 0.$$ 
    If $J$ is additionally \CrefAndHyperrefIfExist{definition:filtered_cofiltered_category}{filtered} so that the system is a \CrefAndHyperrefIfExist{definition:system_in_a_category_indexed_by_a_directed_poset}{directed} one, then we show that the sequence
    \begin{equation} \label{eq:right_exact_sequence_of_direct_system_of_exact_ring_modules}
    \varinjlim_{j \in J} A_j \to \varinjlim_{j \in J} B_j \to \varinjlim_{j \in J} C_j \to 0.
    \end{equation}
    is left exact as well. Take some element of the kernel of $\varinjlim_{j \in J} A_j \to \varinjlim_{j \in J} B_j$; such an element is represented by some element $a_j \in A_j$ for some $j \in J$. Since its image is zero in $\varinjlim{j \in J} B_j$, it must be zero as an element of $B_k$ for some $k \in J$. Since $J$ is filtered, there exists some $k' \in J$ so that there are arrows $j \to k'$ and $k \to k'$ and so that the image of $a_j$ in $B_{k'}$ is $0$. The image of $a_j$ in $A_{k'}$ is then $0$ due to the assumption that 
    $$0 \to A_{k'} \to B_{k'} \to C_{k'} \to 0$$
    is exact. Therefore, $a_j$ is $0$ in $\varinjlim_{j \in J} A_j$, so \eqref{eq:right_exact_sequence_of_direct_system_of_exact_ring_modules} is left exact as claimed and ${}_R \mathrm{Mod}_S$ is $AB5$.

    Similarly as how we argued that ${}_R \mathrm{Mod}_S$ is $AB3$ and $AB4$, one can argue that ${}_R \mathrm{Mod}_S$ is $AB3^*$ and $AB4^*$. Moreover, one can show that the $R$-$S$-bimodule $R \otimes_{\bbZ} S^{\op}$\CrefIfExists{definition:tensor_product_of_a_ring_and_an_algebra_over_a_ring}\CrefIfExists{definition:opposite_ring_of_a_ring} is a generator for ${}_R \mathrm{Mod}_S$.
\end{proof}




\subsection{Adjoint functors}

\CrefAndHyperrefIfExist{definition:adjoint_functors_between_categories_unit_counit_of_adjoint_functors}{Adjoint functors} enjoy nice properties, especially in homological algebra. 

\begin{definition} \label{definition:adjoint_functors_between_categories_unit_counit_of_adjoint_functors}
Let $\mathcal{C}$ and $\mathcal{D}$ be \CrefAndHyperrefIfExist{definition:category}{categories}. Let $F : \mathcal{C} \to \mathcal{D}$ and $G : \mathcal{D} \to \mathcal{C}$ be functors. 

An \hldef{adjunction between $F$ and $G$} consists of two \CrefAndHyperrefIfExist{definition:natural_transformation_between_functors_between_categories}{natural transformations}: $\eta : \mathrm{Id}_{\mathcal{C}} \implies GF$ (the \hldef{unit}), and  $\varepsilon : FG \implies \mathrm{Id}_{\mathcal{D}}$ (the \hldef{counit})

These must satisfy the triangle identities: For every object $X \in \mathcal{C}$ 
and $Y \in \mathcal{D}$, 
$$\varepsilon_{FX} \circ F(\eta_X) = \text{id}_{FX}$$
$$G(\varepsilon_Y) \circ \eta_{GY} = \text{id}_{GY}.$$
In diagrammatic form, the triangle identities assert that the following are commutative diagrams:
\begin{center}
\begin{tikzcd}
F(X) \arrow[r, "F(\eta_X)"] \arrow[rd, "\text{id}_{F(X)}"'] & FGF(X) \arrow[d, "\varepsilon_{F(X)}"] \\
& F(X)
\end{tikzcd}
\begin{tikzcd}
G(Y) \arrow[r, "\eta_{G(Y)}"] \arrow[rd, "\text{id}_{G(Y)}"'] & GFG(Y) \arrow[d, "G(\varepsilon_Y)"] \\
& G(Y)
\end{tikzcd}
\end{center}

We say that $F$ is a \hldef{left adjoint to $G$} and $G$ is a \hldef{right adjoint to $F$} (written \hl{$F \dashv G$}). 

% for every object $A$ in $\mathcal{C}$ and $B$ in $\mathcal{D}$ there is a \CrefAndHyperrefIfExist{definition:natural_transformation_between_functors_between_categories}{natural isomorphism}
% \begin{align*}
% \operatorname{Hom}_{\mathcal{D}}(F(A), B) \cong \operatorname{Hom}_{\mathcal{C}}(A, G(B))
% \end{align*}
% that is natural in both $A$ and $B$.


In the case that $\mathcal{C}$ and $\mathcal{D}$ are \CrefAndHyperrefIfExist{definition:locally_small_category}{locally small} categories (or $U$-locally small categories if a \CrefAndHyperrefIfExist{definition:grothendieck_universe}{universe} $U$ is available), we have an adjunction $F \dashv G$ if and only if for every object $X$ in $\mathcal{C}$ and $Y$ in $\mathcal{D}$ there is a \CrefAndHyperrefIfExist{definition:natural_transformation_between_functors_between_categories}{natural isomorphism}
\begin{align*}
\operatorname{Hom}_{\mathcal{D}}(F(X), Y) \cong \operatorname{Hom}_{\mathcal{C}}(X, G(Y))
\end{align*}
that is natural in both $X$ and $Y$. In this case, the \hldef{unit of the adjunction} is the natural transformation $\eta : \mathrm{Id}_{\mathcal{C}} \Rightarrow G F$ such that, 
\begin{enumerate}
    \item for every $X \in \calC$, the morphism $\eta_X: X \to GF(X)$ (each called a \hldef{unit morphism}) in $\calC$ is obtained as the image of $\id_{F(X)}$ via the adjoint isomorphism
    $$\Hom_\calD(F(X), F(X)) \cong \Hom_\calC(X, GF(X)). $$

    \item for every $Y \in \calD$, the morphism $\epsilon_Y: FG(Y) \to Y$ (each called a \hldef{counit morphism}) in $\calD$ is obtained as the image of $\id_{G(Y)}$ via the adjoint isomorphism 
    $$\Hom_\calC(G(Y), G(Y)) \cong \Hom_\calD(FG(Y), Y).$$

\end{enumerate}


% Let $F : \mathcal{C} \to \mathcal{D}$ and $G : \mathcal{D} \to \mathcal{C}$ be functors. 
% $F$ is a \hldef{left adjoint to $G$} and $G$ is a \hldef{right adjoint to $F$} (written \hl{$F \dashv G$}) if for every object $A$ in $\mathcal{C}$ and $B$ in $\mathcal{D}$ there is a \CrefAndHyperrefIfExist{definition:natural_transformation_between_functors_between_categories}{natural isomorphism}
% \begin{align*}
% \operatorname{Hom}_{\mathcal{D}}(F(A), B) \cong \operatorname{Hom}_{\mathcal{C}}(A, G(B))
% \end{align*}
% that is natural in both $A$ and $B$.
\end{definition}

\TODO{examples of adjoint functors}

One of the most essential adjunction of functors in algebra would be the tensor-hom adjunction:

\begin{theorem}[Tensor-Hom Adjunction for Bimodules] \label{theorem:tensor_hom_adjunction_for_bimodules_of_rings}
    \
    \begin{enumerate}
        \item Let $R,S,T$ be \CrefAndHyperrefIfExist{definition:ring}{(not necessarily commutative) rings}. Let $M$ be an \CrefAndHyperrefIfExist{definition:module_of_a_ring}{$R$-$S$ bimodule}, let $N$ be an $S$-$T$ bimodule, and let $P$ be an $R$-$T$ bimodule. Then there is a natural isomorphism of abelian groups
        $$ \operatorname{Hom}_{R\text{-}T}(M \otimes_S N,\, P) \;\;\cong\;\; \operatorname{Hom}_{S\text{-}T}(N,\, \operatorname{Hom}_R(M, P)) $$
        \CrefIfExists{definition:tensor_product_of_bimodules_of_rings} \CrefIfExists{definition:hom_of_left_right_bi_modules_of_rings}.
        Note that $\operatorname{Hom}_{R\text{-}T}$ is the abelian group of $R$-$T$ bimodule homomorphisms, $\operatorname{Hom}_{S\text{-}T}$ is the abelian group of $S$-$T$ bimodule homomorphisms, and $\operatorname{Hom}_R(M,P)$ is endowed with the structure of an $S$-$T$ bimodule via
        \begin{align*}
        (s \cdot f)(m) &= f(m \cdot s), \\
        (f \cdot t)(m) &= f(m) \cdot t,
        \end{align*}
        for all $s \in S$, $t \in T$, $f \in \operatorname{Hom}_R(M,P)$, $m \in M$.
        \TextIfExists{definition:adjoint_functors_between_categories_unit_counit_of_adjoint_functors}{Intuitively, this expresses that $M \otimes_S -$ is \CrefAndHyperrefIfExist{definition:adjoint_functors_between_categories_unit_counit_of_adjoint_functors}{left adjoint} to $\operatorname{Hom}_R(M,-)$ in the category of bimodules.}

        \item Let $R,S,T$ be \CrefAndHyperrefIfExist{definition:ring}{(not necessarily commutative) rings}. Let $M$ be an \CrefAndHyperrefIfExist{definition:module_of_a_ring}{$R$-$S$ bimodule}, let $N$ be an $S$-$T$ bimodule, and let $P$ be an $R$-$T$ bimodule. Then there is a natural isomorphism of abelian groups
        $$ \operatorname{Hom}_{R\text{-}T}(M \otimes_S N,\, P) \;\;\cong\;\; \operatorname{Hom}_{R\text{-}S}(M,\, \operatorname{Hom}_T(N, P)) $$
        \CrefIfExists{definition:tensor_product_of_bimodules_of_rings} \CrefIfExists{definition:hom_of_left_right_bi_modules_of_rings}.

        Note that $\operatorname{Hom}_{R\text{-}T}$ is the abelian group of $R$-$T$ bimodule homomorphisms, $\operatorname{Hom}_{R\text{-}S}$ is the abelian group of $R$-$S$ bimodule homomorphisms, and $\operatorname{Hom}_T(N,P)$ is endowed with the structure of an $R$-$S$ bimodule via
        \begin{align*}
        (r \cdot f)(n) &= r \cdot f(n), \\
        (f \cdot s)(n) &= f(n \cdot s),
        \end{align*}
        for all $r \in R$, $s \in S$, $f \in \operatorname{Hom}_T(N,P)$, and $n \in N$.

        \TextIfExists{definition:adjoint_functors_between_categories_unit_counit_of_adjoint_functors}{Intuitively, this expresses that $-\otimes N$ is \CrefAndHyperrefIfExist{definition:adjoint_functors_between_categories_unit_counit_of_adjoint_functors}{left adjoint} to $\operatorname{Hom}_T(N,-)$ in the category of bimodules.}

    \end{enumerate}
\end{theorem}


\TODO{}

\begin{proposition} \label{proposition:left_right_adjoint_is_right_left_exact}
    Let $\calA, \calB$ be \CrefAndHyperrefIfExist{definition:abelian_category}{abelian categories} and let $F: \calA \to \calB$ and $G: \calB \to \calA$ be \CrefAndHyperrefIfExist{definition:adjoint_functors_between_categories_unit_counit_of_adjoint_functors}{adjoint} \CrefAndHyperrefIfExist{definition:additive_functor_between_additive_categories}{additive functors}, say with $F \dashv G$ (i.e. $F$ is left adjoint to $G$). The left adjoint functor $F$ is \CrefAndHyperrefIfExist{definition:exact_functor_between_abelian_categories}{right exact} and and the right adjoint functor $G$ is \CrefAndHyperrefIfExist{definition:exact_functor_between_abelian_categories}{left exact}
\end{proposition}
\begin{proposition} \label{proposition:tensor_product_of_modules_is_right_exact_hom_of_modules_is_left_exact}
    Let $R,S,T$ be \CrefAndHyperrefIfExist{definition:ring}{(not necessarily commutative) rings}. Recall that categories of modules are \CrefAndHyperrefIfExist{definition:abelian_category}{abelian}\CrefIfExists{theorem:the_category_of_R_S_bimodules_is_a_grothendieck_abelian_category_and_AB4_star}.
    
    \begin{enumerate}
        \item Let $M$ be an \CrefAndHyperrefIfExist{definition:module_of_a_ring}{$R$-$S$-bimodule}. The functor \CrefAndHyperrefIfExist{definition:tensor_product_of_bimodules_of_rings}{$M \otimes_S -: {}_{S}\mathbf{Mod}_T \to {}_{R}\mathbf{Mod}_T$} is a \CrefAndHyperrefIfExist{definition:exact_functor_between_abelian_categories}{right exact functor}.

        \item Let $N$ be an $S$-$T$-bimodule. The functor $- \otimes_S N: {}_{R}\mathbf{Mod}_S \to {}_{R}\mathbf{Mod}_T$ is a \CrefAndHyperrefIfExist{definition:exact_functor_between_abelian_categories}{right exact functor}.

        \item Let $M$ be an $R$-$S$-bimodule. The functor
        $$\Hom(M,-): {}_{R} \mathbf{Mod}_{T} \to {}_{S} \mathbf{Mod}_{T}$$
        \CrefIfExists{definition:hom_of_left_right_bi_modules_of_rings} is a \CrefAndHyperrefIfExist{definition:exact_functor_between_abelian_categories}{left exact functor}.

        Now let $M$ be an $S$-$R$-bimodule. The functor
        $$\Hom(M,-): {}_{T} \mathbf{Mod}_{R} \to {}_{S} \mathbf{Mod}_{T}$$
        is a \CrefAndHyperrefIfExist{definition:exact_functor_between_abelian_categories}{left exact functor}.

        \item Let $N$ be an $R$-$T$-bimodule. The functor
        $$\Hom(-,N): {}_{R} \mathbf{Mod}_{S}^{\op} \to {}_{S} \mathbf{Mod}_{T}$$
        \CrefIfExists{definition:opposite_category_of_a_category} is a \CrefAndHyperrefIfExist{definition:exact_functor_between_abelian_categories}{left exact functor}.

        Now let $N$ be an $T$-$R$-bimodule. The functor
        $$\Hom(-,N): {}_{S} \mathbf{Mod}_{R}^{\op} \to {}_{S} \mathbf{Mod}_{T}$$
        is a \CrefAndHyperrefIfExist{definition:exact_functor_between_abelian_categories}{left exact functor}.

    \end{enumerate}
\end{proposition}
\begin{proof}
    The right exactness of the tensor functors and the left exactness of $\Hom(M,-)$ follow from \Cref{theorem:tensor_hom_adjunction_for_bimodules_of_rings} and \Cref{proposition:left_right_adjoint_is_right_left_exact}.
    \TODO{prove left exactness of $\Hom(-,N)$}.
\end{proof}



\section{Chain complexes of objects in additive categories}

\CrefAndHyperrefIfExist{definition:chain_complex_of_objects_in_an_additive_category}{Chain complexes} are made of a sequence of morphisms in an additive category such that the composition of any two conseuctive morphisms is $0$. The first kind of chain complexes one should keep in mind are chain complexes of \CrefAndHyperrefIfExist{definition:module_of_a_ring}{modules} over a ring.

% \begin{definition}[Chain complex in an additive category] \label{definition:chain_complex_of_objects_in_an_additive_category}
% Let $\mathcal{A}$ be an \hyperrefIfExists{definition:additive_category_preadditive_category}{preadditive category} and let $I$ be a totally ordered set (typically $\mathbb{Z}$, but $I \subseteq \mathbb{Z}$ is also allowed). 
% \begin{enumerate}
%     \item A \hldef{chain complex} $(K^\bullet, d^\bullet)$ in $\mathcal{A}$ indexed by $I$ consists of:
%     \begin{itemize}
%         \item Objects $\{ K^i \}_{i \in I}$ in $\mathcal{A}$, called the \hldef{terms in degree $i$},
%         \item Morphisms $d^i: K^i \to K^{i+1}$ in $\mathcal{A}$, called the \hldef{differentials in degree $i$},
%     \end{itemize}
%     such that for every $i \in I$, $d^{i+1} \circ d^i = 0$. That is,
%     $$ K^\bullet: \cdots \xrightarrow{d^{i-2}} K^{i-1} \xrightarrow{d^{i-1}} K^i \xrightarrow{d^i} K^{i+1} \xrightarrow{d^{i+1}} \cdots $$
%     with $d^{i+1}d^i = 0$ for all $i$. We might typically use notation such as \hl{$K^\bullet = (K^i, d^i)_{i \in I}$} to denote a chain complex in $\mathcal{A}$.

%     A cochain complex can be defined similarly/dually.

%     \item Let $K^\bullet = (K^i, d_K^i)$ and $L^\bullet = (L^i, d_L^i)$ be \CrefAndHyperrefIfExist{definition:chain_complex_of_objects_in_an_additive_category}{chain complexes} in $\mathcal{A}$ indexed by the same set $I$. 
%     A \hldef{morphism of chain complexes} (or \hldef{chain map})
%     $$ f^\bullet: K^\bullet \to L^\bullet $$
%     consists of morphisms $f^i: K^i \to L^i$ for all $i \in I$, such that for every $i \in I$,
%     $$ d_L^i \circ f^i = f^{i+1} \circ d_K^i, $$
%     i.e., the following diagram commutes for all $i$:

%     $$ \begin{array}{ccc} K^i & \xrightarrow{d_K^i} & K^{i+1} \\ \downarrow{f^i} && \downarrow{f^{i+1}} \\ L^i & \xrightarrow{d_L^i} & L^{i+1} \end{array}.$$

% \end{enumerate}

% There is then a category, often denoted by \hl{$\mathrm{Ch}(\mathcal{A})$} or \hl{$\mathbf{Ch}(\mathcal{A})$}, whose objects are chain complexes in $\calA$ and whose morphisms are morphisms of chain complexes. In particular, we may denote by 
% $$\hlin{\operatorname{Hom}(K^\bullet, L^\bullet)=  \operatorname{Hom}_{\mathrm{Ch}(\mathcal{A})}(K^\bullet, L^\bullet)}$$
% the set of chain maps $K^\bullet \to L^\bullet$; it is in fact an abelian group.

% A \hldef{morphism of cochain complexes} is defined similarly, and we similarly denote by \hl{$\mathrm{Ch}(\mathcal{A})$} or \hl{$\mathbf{Ch}(\mathcal{A})$} the caetgory of cochain complexes in $\calA$. 


% \TextIfExists{definition:dg_category_over_a_ring}{
% If $k$ is a \CrefAndHyperrefIfExist{definition:commutative_ring}{commutative ring} such that $\Hom_\calA(X,Y)$ is \CrefAndHyperrefIfExist{definition:category_enriched_in_a_monoidal_category}{enriched in} the category of \CrefAndHyperrefIfExist{definition:module_of_a_ring}{$k$-modules}, then $\mathrm{Ch}(\calA)$ \CrefAndHyperref{definition:category_of_chain_complexes_of_objects_in_an_additive_category_as_a_dg_category}{can be equipped with} the structure of a \CrefAndHyperrefIfExist{definition:dg_category_over_a_ring}{dg-category over $k$}.
% }


% \end{definition}

\begin{definition}[Chain complex in a preadditive category] \label{definition:chain_complex_of_objects_in_an_additive_category}
Let $\mathcal{A}$ be a \hyperrefIfExists{definition:additive_category_preadditive_category}{preadditive category} and let $I$ be a totally ordered set (typically $\mathbb{Z}$, but $I \subseteq \mathbb{Z}$ is also allowed). 
\begin{enumerate}
    \item A \hldef{chain complex} $(K_\bullet, d_\bullet)$ in $\mathcal{A}$ indexed by $I$ is the homological convention for sequences with decreasing degrees. It consists of:
    \begin{itemize}
        \item Objects $\{ K_i \}_{i \in I}$ in $\mathcal{A}$, called the \hldef{terms in degree $i$},
        \item Morphisms $d_i: K_i \to K_{i-1}$ in $\mathcal{A}$, called the \hldef{boundary maps} or \hldef{differentials in degree $i$},
    \end{itemize}
    such that for every $i \in I$, $d_{i-1} \circ d_i = 0$. That is,
    $$ K_\bullet: \cdots \xrightarrow{d_{i+1}} K_i \xrightarrow{d_i} K_{i-1} \xrightarrow{d_{i-1}} K_{i-2} \xrightarrow{} \cdots $$
    with $d_{i-1}d_i = 0$ for all $i$. We typically use the notation \hl{$K_\bullet = (K_i, d_i)_{i \in I}$}.



    \item Dually, a \hldef{cochain complex} $(K^\bullet, d^\bullet)$ in $\mathcal{A}$ follows the \hldef{cohomological convention} with increasing degrees. It consists of objects $\{ K^i \}_{i \in I}$ and \hldef{coboundary maps} $d^i: K^i \to K^{i+1}$ such that $d^{i+1} \circ d^i = 0$:
    $$ K^\bullet: \cdots \xrightarrow{d^{i-1}} K^i \xrightarrow{d^i} K^{i+1} \xrightarrow{d^{i+1}} K^{i+2} \xrightarrow{} \cdots $$
    We typically use the notation \hl{$K^\bullet = (K^i, d^i)_{i \in I}$}.

    \item Let $K_\bullet = (K_i, d_i^K)$ and $L_\bullet = (L_i, d_i^L)$ be \CrefAndHyperrefIfExist{definition:chain_complex_of_objects_in_an_additive_category}{chain complexes} in $\mathcal{A}$ indexed by the same set $I$. A \hldef{morphism of chain complexes} (or \hldef{chain map})
    $$ f_\bullet: K_\bullet \to L_\bullet $$
    consists of morphisms $f_i: K_i \to L_i$ for all $i \in I$, such that for every $i \in I$, the following diagram commutes:
    $$ \begin{array}{ccc} K_i & \xrightarrow{d_i^K} & K_{i-1} \\ \downarrow{f_i} && \downarrow{f_{i-1}} \\ L_i & \xrightarrow{d_i^L} & L_{i-1} \end{array} $$
    i.e., $d_i^L \circ f_i = f_{i-1} \circ d_i^K$. 



    A \hldef{morphism of cochain complexes} $f^\bullet: K^\bullet \to L^\bullet$ is defined similarly, satisfying the commutativity condition $d_L^i \circ f^i = f^{i+1} \circ d_K^i$.
\end{enumerate}

The collection of these objects and morphisms forms a category. Notation for these categories is as follows:
\begin{itemize}
    \item \hl{$\mathrm{Ch}(\mathcal{A})$} or \hl{$\mathbf{Ch}(\mathcal{A})$} is often used as a general term.
    \item To be explicit about the indexing convention, one uses \hl{$\mathrm{Ch}_\bullet(\mathcal{A})$} for chain complexes and \hl{$\mathrm{Ch}^\bullet(\mathcal{A})$} (or sometimes $\mathrm{CoCh}(\mathcal{A})$) for cochain complexes.
    \item The set of chain maps between two complexes is denoted by $\hlin{\operatorname{Hom}_{\mathrm{Ch}(\mathcal{A})}(K_\bullet, L_\bullet)}$; it is an abelian group under pointwise addition $(f+g)_i = f_i + g_i$.
\end{itemize}

\TextIfExists{definition:dg_category_over_a_ring}{
If $k$ is a \CrefAndHyperrefIfExist{definition:commutative_ring}{commutative ring} such that $\Hom_\calA(X,Y)$ is \CrefAndHyperrefIfExist{definition:category_enriched_in_a_monoidal_category}{enriched in} the category of \CrefAndHyperrefIfExist{definition:module_of_a_ring}{$k$-modules}, then $\mathrm{Ch}(\calA)$ \CrefAndHyperref{definition:category_of_chain_complexes_of_objects_in_an_additive_category_as_a_dg_category}{can be equipped with} the structure of a \CrefAndHyperrefIfExist{definition:dg_category_over_a_ring}{dg-category over $k$}.
}
\end{definition}

% 
\begin{remark} \label{remark:cohomological_vs_homological_conventions}
    The convention used to define chain complexes in \Cref{definition:chain_complex_of_objects_in_an_additive_category} is a \emph{cohomological one} --- note that indices are written as superscripts and increase when ``following the arrows''. Such a chain complex may also be referred to as a \hldef{cochain complex} or a \hldef{cohomological chain complex} to emphasize an adoption of a cohomological convention. 

    The dual convention would be a \emph{homological one}, in which indices are written as subscripts and decrease when ``following the arrow''. As such, one may speak of a \hldef{(homological) chain complex} $(K_\bullet, d_\bullet)$ indexed by $I$ as consisting of:

    \begin{itemize}
    \item Objects $\{ K_i \}_{i \in I}$ in $\mathcal{A}$, called the \hldef{terms in degree $i$},
    \item Morphisms $d_i: K_i \to K_{i-1}$ in $\mathcal{A}$, called the \hldef{differentials in degree $i$},
    \end{itemize}
    such that for every $i \in I$, $d_{i-1} \circ d_i = 0$. That is,
    $$ 
    K_\bullet: \cdots \xrightarrow{d_{i+1}} K_i \xrightarrow{d_i} K_{i-1} \xrightarrow{d_{i-1}} K_{i-2} \xrightarrow{d_{i-2}} \cdots
    $$
    with $d_{i-1} d_i = 0$ for all $i$. We might typically use notation such as \hl{$K_\bullet = (K_i, d_i)_{i \in I}$} to denote a chain complex in $\mathcal{A}$.

    The differences between the conventions persist --- for example, cohomological objects are usually written with superscript indicees whereas homological objects are usually written with subscript indicees.
\end{remark}
%\begin{convention} \label{convention:homological_algebra_is_discussed_in_cohomological_terms}
    When discussing homological algebra in abstract terms, we may often adopt the homological convention in some discussions and the cohomological convention in others \CrefIfExists{remark:cohomological_vs_homological_conventions}.
    %; for instance, indices are written as superscripts and increase along the direction of the arrows in chain complexes. 
\end{convention}

% \begin{definition}[Morphisms of chain complexes] \label{definition:chain_complex_of_objects_in_an_additive_category}
Let $\mathcal{A}$ be an \CrefAndHyperrefIfExist{definition:additive_category}{additive category}, and let $K^\bullet = (K^i, d_K^i)$ and $L^\bullet = (L^i, d_L^i)$ be \CrefAndHyperrefIfExist{definition:chain_complex_of_objects_in_an_additive_category}{chain complexes} in $\mathcal{A}$ indexed by the same set $I$. 
A \hldef{morphism of chain complexes} (or \hldef{chain map})
$$ f^\bullet: K^\bullet \to L^\bullet $$
consists of morphisms $f^i: K^i \to L^i$ for all $i \in I$, such that for every $i \in I$,
$$ d_L^i \circ f^i = f^{i+1} \circ d_K^i, $$
i.e., the following diagram commutes for all $i$:

$$ \begin{array}{ccc} K^i & \xrightarrow{d_K^i} & K^{i+1} \\ \downarrow{f^i} && \downarrow{f^{i+1}} \\ L^i & \xrightarrow{d_L^i} & L^{i+1} \end{array}.$$

There is then a category, often denoted by \hl{$\mathrm{Ch}(\mathcal{A})$} or \hl{$\mathbf{Ch}(\mathcal{A})$}, whose objects are chain complexes in $\calA$ and whose morphisms are morphisms of chain complexes. In particular, we may denote by 
$$\hlin{\operatorname{Hom}(K^\bullet, L^\bullet)=  \operatorname{Hom}_{\mathrm{Ch}(\mathcal{A})}(K^\bullet, L^\bullet)}$$
the set of chain maps $K^\bullet \to L^\bullet$; it is in fact an abelian group.

A \hldef{morphism of cochain complexes} is defined similarly, and we similarly denote by \hl{$\mathrm{Ch}(\mathcal{A})$} or \hl{$\mathbf{Ch}(\mathcal{A})$} the caetgory of cochain complexes in $\calA$. 
\end{definition}

% See Also
% 
\begin{proposition} \label{proposition:category_of_chain_complexes_in_an_additive_category_is_additive}
Let $\mathcal{A}$ be an \hyperrefIfExists{definition:additive_category}{additive category}. 
\begin{enumerate}
    \item The category \hyperrefIfExists{definition:chain_complex_of_objects_in_an_additive_category}{$\mathrm{Ch}(\calA)$} of chain complexes is itself and additive category.

    \item If $\calA$ is an \hyperrefIfExists{definition:abelian_category}{abelian category}, then $\mathrm{Ch}(\calA)$ is an abelian category.

    \item If $\calA$ is an \hyperrefIfExists{definition:abelian_category}{abelian category} satisfying Grothendieck's axiom \CrefAndHyperrefIfExist{definition:grothendiecks_additional_axioms_for_abelian_categories}{AB$n$ (resp. AB$n^*$)} for $n \in \{3,4,5,6\}$, then $\mathrm{Ch}(\calA)$ also satisfies AB$n$ (resp. AB$n^*$). If $\calA$ is a \CrefAndHyperrefIfExist{definition:grothendiecks_additional_axioms_for_abelian_categories}{Grothendieck abelian category}, then so is $\mathrm{Ch}(\calA)$
\end{enumerate}
\end{proposition}
\begin{proof}
    Combine \Cref{proposition:category_of_chain_complexes_of_objects_in_a_preadditive_category_is_equivalent_to_the_category_of_additive_functors_from_the_walking_chain_complex_category} and \Cref{lemma:additive_functor_category_from_small_preadditive_categories_preserves}.
\end{proof}

% \begin{corollary}
% Let $\calB$ be a \CrefAndHyperrefIfExist{definition:additive_category}{preadditive category}.
% \begin{enumerate}
%     \item The \CrefAndHyperrefIfExist{definition:chain_complex_of_objects_in_an_additive_category}{(co)chain complex category} $\text{Ch}(\calB)$ is preadditive. If $\calB$ is additionally \CrefAndHyperrefIfExist{definition:additive_category}{additive}/\CrefAndHyperrefIfExist{definition:abelian_category}{abelian}, then so is $\text{Ch}(\calB)$.

%     \item If $\calB$ is an abelian category with property \CrefAndHyperrefIfExist{definition:grothendiecks_additional_axioms_for_abelian_categories}{$ABn$ for $n = 3,4,5,6$ or $ABn^*$ for $n = 3,4,5$}, then $\text{Add}(\calA, \calB)$ possesses the same property.
% \end{enumerate}

% \end{corollary}

\begin{example}
    \TODO{simple examples of chain complexes}
\end{example}

\begin{example}
    Some examples of chain complexes include  \TODO{}
    \begin{enumerate}
        \item The \CrefAndHyperrefIfExist{definition:singular_chain_complex_of_a_topological_space_with_coefficients_in_a_commutative_ring}{singular chain complex} of a topological space.
    \end{enumerate}
\end{example}




\subsection{Category of chain complexes as a functor category}

\begin{definition} \label{definition:quiver}
A \hldef{quiver} is a quadruple $Q = (Q_0, Q_1, s, t)$, where:
\begin{itemize}
    \item $Q_0$ is a collection of \hldef{vertices}.
    \item $Q_1$ is a collection of \hldef{arrows}.
    \item $s, t: Q_1 \to Q_0$ are functions assigning to each arrow $\alpha \in Q_1$ its \hldef{source} $s(\alpha)$ and its \hldef{target} $t(\alpha)$.
\end{itemize}
\end{definition}
\begin{definition} \label{definition:path_category_generated_by_a_quiver}
Let $Q$ be a \CrefAndHyperrefIfExist{definition:quiver}{quiver}.
The \hldef{path category generated by $Q$}, denoted \hl{$\mathcal{F}(Q)$}, is the \CrefAndHyperrefIfExist{definition:category}{category} defined as follows:
\begin{itemize}
    \item The objects of $\mathcal{F}(Q)$ are the vertices $Q_0$.
    \item For any two objects $x, y \in Q_0$, the set of morphisms $\text{Hom}_{\mathcal{F}(Q)}(x, y)$ consists of all paths from $x$ to $y$ --- A \hldef{path of length $n \ge 1$ from $x$ to $y$} is a sequence of arrows $\alpha_n \dots \alpha_1$ such that $s(\alpha_1) = x$, $t(\alpha_n) = y$, and $s(\alpha_{i+1}) = t(\alpha_i)$ for all $1 \le i < n$. Additionally, for each vertex $x$, there is a path $e_x$ of length $0$, which serves as the identity morphism.
    \item Composition of morphisms is defined by the concatenation of paths.
\end{itemize}
\end{definition}
\begin{definition} \label{definition:preadditive_category_generated_by_a_quiver}
Let $Q$ be a \CrefAndHyperrefIfExist{definition:quiver}{quiver} whose collection of arrows is small.

The \hldef{preadditive category generated by $Q$}, denoted \hl{$\mathbb{Z}Q$}, is the \CrefAndHyperrefIfExist{definition:additive_category_preadditive_category}{preadditive category}, i.e. the \CrefAndHyperrefIfExist{definition:category_enriched_in_a_monoidal_category}{category enriched over} the \CrefAndHyperrefIfExist{definition:category_of_groups_of_abelian_groups}{category of abelian groups} defined as follows:
\begin{itemize}
    \item The objects of $\mathbb{Z}Q$ are the vertices $Q_0$.
    \item For any objects $x, y \in Q_0$, the morphism set $\text{Hom}_{\mathbb{Z}Q}(x, y)$ is the free abelian group generated by the set of all paths from $x$ to $y$ in $Q$.
    \item Composition is the unique bilinear extension of the path concatenation in $\mathcal{F}(Q)$. That is, for paths $u, v, w$ where concatenation is defined, composition satisfies $(u + v) \circ w = u \circ w + v \circ w$ and $w \circ (u + v) = w \circ u + w \circ v$.
\end{itemize}
\end{definition}
\begin{definition} \label{definition:walking_chain_complex_category}
Let $Q_{\text{chain}}$ be the \CrefAndHyperrefIfExist{definition:quiver}{quiver} with vertex set $Q_0 = \mathbb{Z}$ and arrow set $Q_1 = \{ d_n : n \to n-1 \mid n \in \mathbb{Z} \}$. 
\TODO{quotient of a category}
The \hldef{walking chain complex category}, denoted \hl{$\mathcal{I}$}, is the quotient of the \CrefAndHyperrefIfExist{definition:additive_category}{preadditive category} $\mathbb{Z}Q_{\text{chain}}$ by the ideal generated by the relations $d_{n-1} \circ d_n = 0$ for all $n \in \mathbb{Z}$. Explicitly:
\begin{itemize}
    \item Objects are the integers $\mathbb{Z}$.
    \item Morphisms are $\mathbb{Z}$-linear combinations of paths, subject to the relation that any path containing a subsegment $d_{n-1}d_n$ is identified with the zero morphism.
\end{itemize}
\end{definition}
\begin{definition} \label{definition:additive_functor_category_between_preadditive_categories}
Let $\mathcal{A}$ and $\mathcal{B}$ be \CrefAndHyperrefIfExist{definition:additive_category_preadditive_category}{preadditive categories} (\CrefAndHyperrefIfExist{definition:category_enriched_in_a_monoidal_category}{categories enriched over} the \CrefAndHyperrefIfExist{definition:category_of_groups_of_abelian_groups}{category of abelian groups}). The \hldef{additive functor category} \hl{$\text{Add}(\mathcal{A}, \mathcal{B})$} is the functor category where:
\begin{itemize}
    \item Objects are {additive functors} $F: \mathcal{A} \to \mathcal{B}$. An \CrefAndHyperrefIfExist{definition:additive_functor_between_additive_categories}{additive functor} is a functor such that for any $x, y \in \text{Ob}(\mathcal{A})$, the map $F: \text{Hom}_{\mathcal{A}}(x, y) \to \text{Hom}_{\mathcal{B}}(F(x), F(y))$ is a group homomorphism.
    \item Morphisms are \CrefAndHyperrefIfExist{definition:natural_transformation_between_functors_between_categories}{natural transformations} between \CrefAndHyperrefIfExist{definition:additive_functor_between_additive_categories}{additive functors}.
\end{itemize}
\end{definition}
\begin{proposition} \label{proposition:cat_of_chain_complexes_in_a_preadd_cat_is_eq_to_the_cat_of_add_functors_from_the_walking_chain_complex_category}
Let $\mathcal{B}$ be a \CrefAndHyperrefIfExist{definition:additive_category_preadditive_category}{preadditive category}.
The category $\text{Ch}(\mathcal{B})$ of \CrefAndHyperrefIfExist{definition:chain_complex_of_objects_in_an_additive_category}{chain complexes} in $\mathcal{B}$ is isomorphic to the category \CrefAndHyperrefIfExist{definition:additive_functor_category_between_preadditive_categories}{$\text{Add}(\mathcal{I}, \mathcal{B})$}(\Cref{definition:walking_chain_complex_category}). 

An additive functor $F: \mathcal{I} \to \mathcal{B}$ corresponds to the chain complex defined by $C_n = F(n)$ and differentials $\partial_n = F(d_n)$.
\end{proposition}
\begin{lemma} \label{lemma:additive_functor_category_from_small_preadditive_categories_preserves}
Let $\calA, \calB$ be \CrefAndHyperrefIfExist{definition:additive_category}{preadditive categories} with $\calA$ \CrefAndHyperrefIfExist{definition:locally_small_category}{small}.
\begin{enumerate}
    \item The \CrefAndHyperrefIfExist{definition:additive_functor_category_between_preadditive_categories}{additive functor category} $\text{Add}(\calA, \calB)$ is preadditive. If $\calB$ is additionally \CrefAndHyperrefIfExist{definition:additive_category}{additive}/\CrefAndHyperrefIfExist{definition:abelian_category}{abelian}, then so is $\text{Add}(\calA, \calB)$.

    \item If $\calB$ is an abelian category with property \CrefAndHyperrefIfExist{definition:grothendiecks_additional_axioms_for_abelian_categories}{$ABn$ for $n = 3,4,5,6$ or $ABn^*$ for $n = 3,4,5$}, then $\text{Add}(\calA, \calB)$ possesses the same property.
\end{enumerate}
\end{lemma}

\begin{proposition} \label{proposition:category_of_chain_complexes_in_an_additive_category_is_additive}
Let $\mathcal{A}$ be an \hyperrefIfExists{definition:additive_category}{additive category}. 
\begin{enumerate}
    \item The category \hyperrefIfExists{definition:chain_complex_of_objects_in_an_additive_category}{$\mathrm{Ch}(\calA)$} of chain complexes is itself and additive category.

    \item If $\calA$ is an \hyperrefIfExists{definition:abelian_category}{abelian category}, then $\mathrm{Ch}(\calA)$ is an abelian category.

    \item If $\calA$ is an \hyperrefIfExists{definition:abelian_category}{abelian category} satisfying Grothendieck's axiom \CrefAndHyperrefIfExist{definition:grothendiecks_additional_axioms_for_abelian_categories}{AB$n$ (resp. AB$n^*$)} for $n \in \{3,4,5,6\}$, then $\mathrm{Ch}(\calA)$ also satisfies AB$n$ (resp. AB$n^*$). If $\calA$ is a \CrefAndHyperrefIfExist{definition:grothendiecks_additional_axioms_for_abelian_categories}{Grothendieck abelian category}, then so is $\mathrm{Ch}(\calA)$
\end{enumerate}
\end{proposition}
\begin{proof}
    Combine \Cref{proposition:category_of_chain_complexes_of_objects_in_a_preadditive_category_is_equivalent_to_the_category_of_additive_functors_from_the_walking_chain_complex_category} and \Cref{lemma:additive_functor_category_from_small_preadditive_categories_preserves}.
\end{proof}

% \begin{corollary}
% Let $\calB$ be a \CrefAndHyperrefIfExist{definition:additive_category}{preadditive category}.
% \begin{enumerate}
%     \item The \CrefAndHyperrefIfExist{definition:chain_complex_of_objects_in_an_additive_category}{(co)chain complex category} $\text{Ch}(\calB)$ is preadditive. If $\calB$ is additionally \CrefAndHyperrefIfExist{definition:additive_category}{additive}/\CrefAndHyperrefIfExist{definition:abelian_category}{abelian}, then so is $\text{Ch}(\calB)$.

%     \item If $\calB$ is an abelian category with property \CrefAndHyperrefIfExist{definition:grothendiecks_additional_axioms_for_abelian_categories}{$ABn$ for $n = 3,4,5,6$ or $ABn^*$ for $n = 3,4,5$}, then $\text{Add}(\calA, \calB)$ possesses the same property.
% \end{enumerate}

% \end{corollary}

\subsection{Homology and cohomology of a chain complex}

\begin{definition} \label{definition:boundary_cycle_coboundary_cocyble_of_a_chain_cochain_complex}
    Let $\mathcal{A}$ be an \CrefAndHyperrefIfExist{definition:abelian_category}{abelian category}.
    \begin{enumerate}
        \item Let $C_\bullet = (\dots \to C_{n+1} \xrightarrow{d_{n+1}} C_n \xrightarrow{d_n} C_{n-1} \to \dots)$ be a \CrefAndHyperrefIfExist{definition:chain_complex_of_objects_in_an_additive_category}{chain complex} in $\mathcal{A}$.
        For each integer $n$, we define:
        \begin{itemize}
            \item The object of \hldef{$n$-cycles}, denoted \hl{$Z_n(C)$}, is the \CrefAndHyperrefIfExist{definition:kernel_and_cokernel_of_a_morphism_in_a_category}{kernel} of the differential leaving $C_n$:
            $$Z_n(C) := \ker(d_n: C_n \to C_{n-1}).$$
            \item The object of \hldef{$n$-boundaries}, denoted \hl{$B_n(C)$}, is the \CrefAndHyperrefIfExist{definition:image_coimage_of_a_morphism_in_a_category}{image} of the differential entering $C_n$:
            $$B_n(C) := \text{im}(d_{n+1}: C_{n+1} \to C_n).$$
        \end{itemize}
        Since $d_n \circ d_{n+1} = 0$, there is a canonical \CrefAndHyperrefIfExist{definition:monomorphism_and_epimorphism_in_categories}{monomorphism} $B_n(C) \hookrightarrow Z_n(C)$.

        \item Let $C^\bullet = (\dots \to C^{n-1} \xrightarrow{d^{n-1}} C^n \xrightarrow{d^n} C^{n+1} \to \dots)$ be a \hldef{cochain complex in $\mathcal{A}$}.
        For each integer $n$, we define:
        \begin{itemize}
            \item The object of \hldef{$n$-cocycles}, denoted \hl{$Z^n(C)$}, is the \CrefAndHyperrefIfExist{definition:kernel_and_cokernel_of_a_morphism_in_a_category}{kernel} of the differential leaving $C^n$:
            $$Z^n(C) := \ker(d^n: C^n \to C^{n+1}).$$
            \item The object of \hldef{$n$-coboundaries}, denoted \hl{$B^n(C)$}, is the \CrefAndHyperrefIfExist{definition:image_coimage_of_a_morphism_in_a_category}{image>} of the differential entering $C^n$:
            $$B^n(C) := \text{im}(d^{n-1}: C^{n-1} \to C^n).$$
        \end{itemize}
        Since $d^n \circ d^{n-1} = 0$, there is a canonical \CrefAndHyperrefIfExist{definition:monomorphism_and_epimorphism_in_categories}{monomorphism} $B^n(C) \hookrightarrow Z^n(C)$.
    \end{enumerate}
\end{definition}

\begin{definition}[Acyclic complex] \label{definition:acyclic_complex_of_objects_in_an_abelian_category}
Let $\mathcal{A}$ be an \CrefAndHyperrefIfExist{definition:additive_category}{additive category}, and let $(C_\bullet, d_\bullet)$ be a \CrefAndHyperrefIfExist{definition:chain_complex_of_objects_in_an_additive_category}{complex} in $\mathcal{A}$:
\[
\cdots \xrightarrow{d_{n+1}} C_n \xrightarrow{d_n} C_{n-1} \xrightarrow{d_{n-1}} \cdots.
\]

The complex $(C_\bullet, d_\bullet)$ is called \hldef{acyclic at $C_n$} (or sometimes synonymously \hldef{exact at $C_n$})  if we have $\ker d_n \cong \mathrm{im} d_{n+1}$ \CrefIfExists{definition:kernel_and_cokernel_of_a_morphism_in_a_category}  \CrefIfExists{definition:image_coimage_of_a_morphism_in_a_category}. 

If $\calA$ is an \CrefAndHyperrefIfExist{definition:abelian_category}{abelian category}, then this is equivalent to the condition that the \CrefAndHyperrefIfExist{definition:homology_and_cohomology_objects_for_a_chain_complex_in_an_additive_category}{(co)homology objects} $H^n(C_\bullet) := \ker d_n / \operatorname{im} d_{n+1}$ are zero in $\mathcal{A}$.

We furthermore say that the complex $(C_\bullet, d_\bullet)$ is \hldef{acyclic} or \hldef{exact} if it is acyclic/exact everywhere.
\end{definition}

\begin{proposition} \label{proposition:exact_sequence_in_a_diagram_category_of_an_abelian_category_indexed_by_a_small_category_is_given_by_componentwise_exact_sequences}
    Let $\calA$ be a \CrefAndHyperrefIfExist{definition:abelian_category}{abelian category} and let $J$ be a \CrefAndHyperrefIfExist{definition:locally_small_category}{small category}. 

    Given a sequence $A \to B \to C$ of objects in the \CrefAndHyperrefIfExist{definition:diagram_in_a_category_indexed_by_a_small_category}{diagram category} $\calA^J$, the sequence is \CrefAndHyperrefIfExist{definition:acyclic_complex_of_objects_in_an_abelian_category}{exact} at $B$ if and only if all the sequences $A(j) \to B(j) \to C(j)$ are exact at $B(j)$ for every $j \in \Ob(J)$. 
\end{proposition}

% \begin{definition}[Chain complexes and their (co)homology objects] \label{definition:homology_and_cohomology_objects_for_a_chain_complex_in_an_additive_category}
%     Let $\mathcal{A}$ be an \hyperrefIfExists{definition:abelian_category}{abelian category}.
    
%     \begin{itemize}
%         \item For a cohomologically indexed chain complex $K^\bullet$, its \hldef{cohomology object in degree $i$} is defined by
%         $$ \hlin{H^i(K^\bullet) := \ker(d^i) / \operatorname{im}(d^{i-1})} $$
%         where the kernel and image are taken in $\mathcal{A}$.

%         \item For a homologically indexed chain complex $K_\bullet$, its \hldef{homology object in degree $i$} is defined by
%         $$ \hlin{H_i(K_\bullet) := \ker(d_i) / \operatorname{im}(d_{i+1})} $$
%         where the kernel and image are taken in $\mathcal{A}$.
%     \end{itemize}
% \end{definition}

\begin{definition}[Chain complexes and their (co)homology objects] \label{definition:homology_and_cohomology_objects_for_a_chain_complex_in_an_additive_category}
    Let $\mathcal{A}$ be an \hyperrefIfExists{definition:abelian_category}{abelian category}.
    
    \begin{itemize}
        \item For a \CrefAndHyperrefIfExist{definition:cochain_complex}{cochain complex} $K^\bullet$, its \hldef{cohomology object in degree $i$} is defined as the quotient of the object of $i$-cocycles by the object of $i$-coboundaries:
        $$ \hlin{H^i(K^\bullet) := Z^i(K) / B^i(K) = \ker(d^i) / \operatorname{im}(d^{i-1}).} $$

        \item For a \CrefAndHyperrefIfExist{definition:chain_complex}{chain complex} $K_\bullet$, its \hldef{homology object in degree $i$} is defined as the quotient of the object of $i$-cycles by the object of $i$-boundaries:
        $$ \hlin{H_i(K_\bullet) := Z_i(K) / B_i(K). = \ker(d_i) / \operatorname{im}(d_{i+1}).} $$
    \end{itemize}
\end{definition}


\begin{proposition} \label{proposition:homology_and_cohomology_of_complexes_in_an_abelian_category_are_functorial}
    \begin{enumerate}
        \item Let $f_\bullet: C_\bullet \to D_\bullet$ be a \CrefAndHyperrefIfExist{definition:chain_complex_of_objects_in_an_additive_category}{morphism of chain complexes} in an \CrefAndHyperrefIfExist{definition:abelian_category}{abelian category}. This morphism sends \CrefAndHyperrefIfExist{definition:boundary_cycle_coboundary_cocyble_of_a_chain_cochain_complex}{boundaries to boundaries} and \CrefAndHyperrefIfExist{definition:boundary_cycle_coboundary_cocyble_of_a_chain_cochain_complex}{cycles to cycles}. More preicsely, for all $i \in \bbZ$, $f_i: C_i \to D_i$ restricts to morphisms $Z_i(C_\bullet) \to Z_i(D_\bullet)$ and $B_i(C_\bullet) \to B_i(D_\bullet)$. Therefore, there are induced morphisms $H_i(f_\bullet): H_i(C_\bullet) \to H_i(D_\bullet)$ which are in fact functorial, i.e. respect composition.

        \item Let $f^\bullet: C^\bullet \to D^\bullet$ be a \CrefAndHyperrefIfExist{definition:chain_complex_of_objects_in_an_additive_category}{morphism of cochain complexes} in an \CrefAndHyperrefIfExist{definition:abelian_category}{abelian category}. This morphism sends \CrefAndHyperrefIfExist{definition:boundary_cycle_coboundary_cocyble_of_a_chain_cochain_complex}{coboundaries to coboundaries} and \CrefAndHyperrefIfExist{definition:boundary_cycle_coboundary_cocyble_of_a_chain_cochain_complex}{cocycles to cocycles}. More precisely, for all $i \in \bbZ$, $f^i: C^i \to D^i$ restricts to morphisms $Z^i(C^\bullet) \to Z^i(D^\bullet)$ and $B^i(C^\bullet) \to B^i(D^\bullet)$. Therefore, there are induced morphisms \hl{$H^i(f^\bullet): H^i(C^\bullet) \to H^i(D^\bullet)$} which are functorial, i.e. respect composition.
    \end{enumerate}
\end{proposition}


\subsection{Singular homology groups of a topological space}

\begin{definition} \label{definition:geometric_simplex_of_independent_points_in_a_real_vector_space}
Let $V$ be a real vector space of finite dimension.  
A \hldef{$k$-simplex in topology} (or a \hldef{geometric $k$-simplex}) is the convex hull of $k+1$ affinely independent points $v_0, v_1, \dots, v_k \in V$, and is denoted by
$$\hlin{[v_0, v_1, \dots, v_k] := \left\{ \sum_{i=0}^k t_i v_i \ \middle| \ t_i \ge 0, \ \sum_{i=0}^k t_i = 1 \right\}.}$$


It is also standard to talk of the \hldef{standard topological $n$-simplex} ---  the topological space \hl{$|\Delta^n|$} defined as the subset of Euclidean space $\mathbb{R}^{n+1}$ given by
$$ |\Delta^n| = \Big\{ (t_0, t_1, \ldots, t_n) \in \mathbb{R}^{n+1} : \sum_{i=0}^n t_i = 1, \text{ and } t_i \geq 0 \text{ for all } i \Big\} $$
equipped with the induced topology from the usual Euclidean topology on $\mathbb{R}^{n+1}$.

\TODO{comment on how $|\Delta^n|$ makes sense via a geometric realization}
% Eqiuvalently, $|\Delta^n|$ 

\end{definition}
\TODO{comment on functoriality}
\begin{definition}[Singular simplices] \label{definition:singular_simplex_of_a_topological_space}
Let $X$ be a \CrefAndHyperrefIfExist{definition:topological_space}{topological space}. For each integer $n \ge 0$, A \hldef{singular $n$-simplex in $X$} is a \CrefAndHyperrefIfExist{definition:continuous_map_of_topological_spaces}{continuous map}
$$\sigma : \Delta^n \to X.$$
where \CrefAndHyperrefIfExist{definition:geometric_simplex_of_independent_points_in_a_real_vector_space}{$\Delta^n$ is the standard topological $n$-simplex}. 
 The set of all singular $n$-simplices in $X$ is denoted by \hl{$S_n(X)$}.
\end{definition}
\begin{definition}[Singular chain group with coefficients] \label{definition:singular_chain_group_of_a_topological_space_with_ceofficients_in_a_commutative_ring_with_unity}
Let $X$ be a \CrefAndHyperrefIfExist{definition:topological_space}{topological space}, let $S_n(X)$ be the set of \CrefAndHyperrefIfExist{definition:singular_simplex_of_a_topological_space}{singular $n$-simplices in $X$}, and let $R$ be a \CrefAndHyperrefIfExist{definition:commutative_ring}{commutative ring} with unity.  

\begin{enumerate}
    \item The \hldef{singular $n$-chain group of $X$ with coefficients in $R$} is the free $R$-module \hl{$C_n(X;R)$} whose elements are finite formal linear combinations
    $$\sum_i r_i\, \sigma_i, \quad \text{with } \sigma_i \in S_n(X), \ r_i \in R.$$
    Elements of $C_n(X;R)$ are called \hldef{singular $n$-chains in $X$ with coefficients in $R$}.

    \item If $A \subseteq X$ is a subspace, the quotient groups
    $$\hlin{C_n(X,A;R) = C_n(X;R)/C_n(A;R)}$$
    may be referred to as the \hldef{relative singular $n$-chain groups of $X$ in $A$ with coefficients in $R$} and elements of this group may be referred to as \hldef{relative singular $n$-chains in $X$ relative to $A$ with coefficients in $R$}.
\end{enumerate}
In either case, when $R = \bbZ$, the ring $R$ may be suppressed from notation, so we may write \hl{$C_n(X)$} and \hl{$C_n(X,A)$} for $C_n(X,\bbZ)$ and $C_n(X,A,\bbZ)$ respectively.
\end{definition}
\begin{definition}[Singular chain complex with coefficients] \label{definition:singular_chain_complex_of_a_topological_space_with_coefficients_in_a_commutative_ring}
Let $X$ be a \CrefAndHyperrefIfExist{definition:topological_space}{topological space}, and let $R$ be a \CrefAndHyperrefIfExist{definition:commutative_ring}{commutative ring} with unity.

For each $n \ge 1$, define the $R$-linear \hldef{boundary operator}
$$\partial_n : C_n(X;R) \to C_{n-1}(X;R)$$
\CrefIfExists{definition:singular_chain_group_of_a_topological_space_with_ceofficients_in_a_commutative_ring_with_unity} by
$$\partial_n(\sigma) = \sum_{i=0}^n (-1)^i\, \sigma \circ \delta_i,$$
where $\delta_i : \Delta^{n-1} \to \Delta^n$ is the $i$-th face inclusion.
Extend $\partial_n$ to $C_n(X;R)$ by $R$-linearity. Then $(C_n(X;R), \partial_n)$ forms a \CrefAndHyperrefIfExist{definition:chain_complex_of_objects_in_an_additive_category}{chain complex} in the abelian category of $R$-modules. This chain complex is called the \hldef{singular chain complex of $X$ with coefficients in $R$}.

 
If $A \subseteq X$ is a subspace, then the boundary maps above induce maps 
$$C_n(X,A;R) \to C_{n-1}(X,A;R)$$
on the relative chain groups $C_n(X,A;R)$, yielding a chain complex of $R$-modules; this chain complex may be called the \hldef{relative singular chain complex of the pair $(X,A)$ with coefficients in $R$}.
\end{definition}

\begin{definition}[Singular homology with coefficients] \label{definition:singular_homology_and_relative_singular_homology_of_topological_space_with_coefficients_in_a_commutative_unital_ring}
Let $X$ be a \CrefAndHyperrefIfExist{definition:topological_space}{topological space} and $R$ a \CrefAndHyperrefIfExist{definition:commutative_ring}{commutative ring} with $1$.  
The \hldef{$n$-th singular homology group of $X$ with coefficients in $R$} is the \CrefAndHyperrefIfExist{definition:homology_and_cohomology_objects_for_a_chain_complex_in_an_additive_category}{homology group}
$$\hlin{H_n(X;R) = H_n(C_*(X;R))}$$
where $C_*(X;R)$ is the \CrefAndHyperrefIfExist{definition:singular_chain_complex_of_a_topological_space_with_coefficients_in_a_commutative_ring}{singular chain complex} of $X$ with coefficients in $R$.

Given a \CrefAndHyperrefIfExist{definition:subspace_of_a_topological_space}{subspace} $A \subseteq X$, the \hldef{$n$-th relative singular homology group of $(X,A)$ with coefficients in $R$} is defined as
$$\hlin{H_n(X,A;R) = H_n(C_*(X,A;R))}$$
where $C_*(X,A;R)$ is the \CrefAndHyperrefIfExist{definition:singular_chain_complex_of_a_topological_space_with_coefficients_in_a_commutative_ring}{relative singular chain complex} of $(X,A)$ with coefficients in $R$.
% In other words, these singular homology groups are the \CrefAndHyperrefIfExist{definition:homology_and_cohomology_objects_for_a_chain_complex_in_an_additive_category}{homology $R$-modules} of the \CrefAndHyperrefIfExist{definition:singular_chain_complex_of_a_topological_space_with_coefficients_in_a_commutative_ring}{singular chain complex $C_*(X;R)$ and the relative singular chain complex $C_*(X,A;R)$} respectively.

We may denote $H_n(X;\bbZ)$ and $H_n(X,A;\bbZ)$ by \hl{$H_n(X)$} and \hl{$H_n(X,A)$} respectively.
\end{definition}

\subsection{Homotopy between morphisms of topological spaces}
\begin{definition}[Homotopy of maps of topological spaces] \label{definition:homotopy_of_maps_of_topological_spaces_relative_to_a_subset}

    Let $X$ and $Y$ be topological spaces and let $K \subseteq X$ be a subset. Let $C(X,Y)$ denote the set of all continuous maps $f : X \to Y$.  

    \begin{enumerate}
        \item A \hldef{homotopy between two maps $f,g \in C(X,Y)$ relative to $K$ } is a continuous map
        $$H : X \times [0,1] \to Y$$
        such that for all $x \in X$,
        $$H(x,0) = f(x), \quad H(x,1) = g(x),$$
        and for all $x \in K$ and $t \in [0,1]$,
        $$H(x,t) = f(x) = g(x).$$

        If such an $H$ exists, we say $f$ and $g$ are \hldef{homotopic relative to $K$}, and we write \hl{$f \simeq g \text{ rel } K$}; this is an equivalence relation.

        A \hldef{homotopy between two maps $f,g \in C(X,Y)$} is simply a homotopy relative to $\emptyset$. We write we write \hl{$f \simeq g$} if a homotopy between them exists.

        \item Let $(X, x_0)$ and $(Y, y_0)$ be \CrefAndHyperrefIfExist{definition:pointed_topological_space}{pointed topological spaces} and let $K \subseteq X$ be a subset with $x_0 \in K$. Let $C_*(X,Y)$ denote the set of all continuous based maps $f : X \to Y$ satisfying $f(x_0) = y_0$.

        A \hldef{homotopy of based maps $f,g \in C_*(X,Y)$ relative to $K$} is a continuous map
        $$H : X \times [0,1] \to Y$$
        such that for all $x \in X$,
        $$H(x,0) = f(x), \quad H(x,1) = g(x),$$
        and for all $k \in K$ and $t \in [0,1]$,
        $$H(k,t) = f(k) = g(k),$$
        in particular fixing the basepoint throughout,
        $$H(x_0, t) = y_0 \quad \text{for all } t \in [0,1].$$

        If such an $H$ exists, we say $f$ and $g$ are \hldef{based homotopic relative to $K$}, and we write \hl{$f \simeq g \text{ rel } K$}. This is an equivalence relation.

        A \hldef{homotopy of based maps $f,g \in C_*(X,Y)$} without relative condition is the special case $K = \{x_0\}$ and is called a \hldef{homotopy of based maps} or \hldef{based homotopy}. We write \hl{$f \simeq g$} if such a homotopy exists.

    \end{enumerate}
\end{definition}
\TODO{}

\subsection{Chain homotopy between morphisms of chain complexes}


\begin{definition} \label{definition:chain_homotopy_between_chain_maps_between_complexes}
Let $\mathcal{A}$ be an \hyperrefIfExists{definition:additive_category_preadditive_category}{additive category}\CrefIfExists{definition:additive_category_preadditive_category}. 
\begin{enumerate}

    \item Let $f_\bullet, g_\bullet: C_\bullet \to D_\bullet$ be \hyperrefIfExists{definition:chain_complex_of_objects_in_an_additive_category}{chain maps between complexes}\CrefIfExists{definition:chain_complex_of_objects_in_an_additive_category} in $\mathcal{A}$. A \hldef{chain homotopy from $f_\bullet$ to $g_\bullet$} is a collection of morphisms $\{ s_n : C_n \to D_{n+1} \}$ such that for all $n$,
    \[
    f_n - g_n = d_{n+1}^D \circ s_n + s_{n-1} \circ d_n^C.
    \]
    If such an $s_\bullet$ exists, we say that $f_\bullet$ and $g_\bullet$ are \hldef{chain homotopic} and write \hl{$f_\bullet \simeq g_\bullet$}.

    \item Let $f_\bullet: C_\bullet \to D_\bullet$ be a \hyperrefIfExists{definition:chain_complex_of_objects_in_an_additive_category}{chain map between complexes}\CrefIfExists{definition:chain_complex_of_objects_in_an_additive_category} in $\mathcal{A}$. A \hldef{chain contraction} is a chain homotopy from $f_\bullet$ to the zero complex. The chain map $f_\bullet$ is said to be \hldef{null homotopic} if a chain contraction of $f_\bullet$ exists, i.e. $f_\bullet$ is chain homotopic to the $0$ chain complex. 

    \item Let $f_\bullet: C_\bullet \to D_\bullet$ be a chain map between complexes. We say that $f_\bullet$ is a \hldef{chain homotopy equivalence} if there exists a chain map and $h_\bullet: D_\bullet \to C_\bullet$ such that
    $$fg \simeq \id_{D_\bullet} \quad \text{and} \quad gf \simeq \id_{C_\bullet}.$$
    In this case, it is appropriate to call $f$ and $g$ \hldef{chain homotopy inverses of each other}.
\end{enumerate}
One similarly defines the above notions for cochain complexes and their morphisms.
\end{definition}


\subsection{Mapping cones of morphisms between topological spaces}
\TODO{}

\subsection{Mapping cones of morphisms between chain complexes}

\begin{definition} \label{definition:mapping_cone_of_a_map_of_chain_cochain_complexes}
    \begin{enumerate}
        \item Let $f : (C_\bullet, d^C_\bullet) \to (D_\bullet, d^D_\bullet)$ be a \CrefAndHyperrefIfExist{definition:chain_complex_of_objects_in_an_additive_category}{morphism of chain complexes} in an \CrefAndHyperrefIfExist{definition:additive_category_preadditive_category}{additive category $\mathcal{A}$}.

        The \hldef{mapping cone of $f$}, denoted \hl{$\operatorname{Cone}(f)$}, is the chain complex defined by:
        \begin{itemize}
        \item Objects: For each $n$, 
        $$\operatorname{Cone}(f)_n = D_n \oplus C_{n-1}.$$
        \item Differential: For each $n$, define
        $$d^{\operatorname{Cone}(f)}_n : \operatorname{Cone}(f)_n \to \operatorname{Cone}(f)_{n-1}$$
        by the matrix morphism
        $$d^{\operatorname{Cone}(f)}_n = \begin{pmatrix} d^D_n & f_{n-1} \\ 0 & -d^C_{n-1} \end{pmatrix} : D_n \oplus C_{n-1} \to D_{n-1} \oplus C_{n-2}.$$
        \end{itemize}

        This construction defines a chain complex, i.e., $d^{\operatorname{Cone}(f)}_{n-1} \circ d^{\operatorname{Cone}(f)}_n = 0$.

        \item Dually, let $g : (C^\bullet, d_C^\bullet) \to (D^\bullet, d_D^\bullet)$ be a \CrefAndHyperrefIfExist{definition:chain_complex_of_objects_in_an_additive_category}{morphism of cochain complexes} in $\mathcal{A}$. 

        The \hldef{mapping cone of $g$}, denoted \hl{$\operatorname{Cone}(g)$}, is the \CrefAndHyperrefIfExist{definition:chain_complex_of_objects_in_an_additive_category}{cochain complex} defined by:
        \begin{itemize}
        \item Objects: For each $n$, 
        $$\operatorname{Cone}(g)^n = D^n \oplus C^{n+1}.$$
        \item Differential: For each $n$, define
        $$d_{\operatorname{Cone}(g)}^n : \operatorname{Cone}(g)^n \to \operatorname{Cone}(g)^{n+1}$$
        by the matrix morphism
        $$d_{\operatorname{Cone}(g)}^n = \begin{pmatrix} d_D^n & g^{n+1} \\ 0 & -d_C^{n+1} \end{pmatrix} : D^n \oplus C^{n+1} \to D^{n+1} \oplus C^{n+2}.$$
        \end{itemize}

        This construction defines a cochain complex, i.e., $d_{\operatorname{Cone}(g)}^{n+1} \circ d_{\operatorname{Cone}(g)}^n = 0$.
            \end{enumerate}
\end{definition}

\section{Derived functors}

This section roughly follows \cite[Section 2]{weibel}.

\subsection{Abelian categories with enough objects of a class and resolutions}
\begin{definition} \label{definition:has_enough_objects_of_a_class_on_the_left_right_for_an_abelian_category}
    Let $\mathcal{A}$ be an \CrefAndHyperrefIfExist{definition:abelian_category}{abelian category} and let $\mathcal{X}$ be a class of objects in $\mathcal{A}$.

    We say that $\calA$ \hldef{has enough objects of class $\calX$ on the left (resp. on the right)} if for any object $M \in \calA$, there exists an object $X$ of the class $\calX$ and an \CrefAndHyperrefIfExist{definition:monomorphism_and_epimorphism_in_categories}{epimorphism} $X \twoheadrightarrow M$ (resp. a monomorphism $M \hookrightarrow X$). 
\end{definition}
\begin{definition} \label{definition:has_enough_injectives_or_projectives_for_an_abelian_category}
Let $\mathcal{A}$ be an \CrefAndHyperrefIfExist{definition:abelian_category}{abelian category}.
\begin{enumerate}
    \item $\mathcal{A}$ is said to \hldef{have enough injectives} if for every object $A$ in $\calA$, there is an \CrefAndHyperrefIfExist{definition:monomorphism_and_epimorphism_in_categories}{monomorphism} $A \to I$ with $I$ an \CrefAndHyperrefIfExist{definition:injective_and_projective_objects_in_a_category}{injective object} of $\calA$. \TextIfExistsElse{definition:has_enough_objects_of_a_class_on_the_left_right_for_an_abelian_category}{Equivalently, $\calA$ has enough injectives if it has enough objects of the class of injectives on the right (\Cref{definition:has_enough_objects_of_a_class_on_the_left_right_for_an_abelian_category})}

    \item $\mathcal{A}$ is said to \hldef{have enough projectives} if for every object $A$ in $\calA$, there is a \CrefAndHyperrefIfExist{definition:monomorphism_and_epimorphism_in_categories}{epimorphism} $P \to A$ with $P$ a \CrefAndHyperrefIfExist{definition:injective_and_projective_objects_in_a_category}{projective object} of $\calA$. \TextIfExistsElse{definition:has_enough_objects_of_a_class_on_the_left_right_for_an_abelian_category}{Equivalently, $\calA$ has enough projectives if it has enough objects of the class of projectives on the left (\Cref{definition:has_enough_objects_of_a_class_on_the_left_right_for_an_abelian_category})}

\end{enumerate}
\end{definition}

% See Also
% definition:left_right_derived_functors_of_a_right_left_exact_functor_between_abelian_categories_where_source_has_enough_projectives_injectives 

\begin{theorem} \label{theorem:examples_of_abelian_categories_with_enough_injectives_or_projectives}
\begin{enumerate}
    \item Examples of abelian categories with enough injectives include:
    \begin{itemize}
        \item The category of abelian groups.
        \item The category of modules over a ring.
        \item The category of sheaves of abelian groups on a ringed space or on an essentially small site.
    \end{itemize}

    \item Examples of abelian categories with enough projectives include:
    \begin{itemize}
        \item The category of modules over a ring with enough projectives (e.g., rings with unity and suitable properties). \TODO{make this more precise}
        \item The category of finitely generated modules over a semisimple ring.
    \end{itemize}
\end{enumerate}
% These conditions ensure the existence of derived functors such as Ext and Tor.
\end{theorem}
\begin{definition} \label{definition:left_right_resolution_of_a_class_of_objects_in_an_abelian_category}
Let $\mathcal{A}$ be an \CrefAndHyperrefIfExist{definition:abelian_category}{abelian category} and let $\mathcal{X}$ be a class of objects in $\mathcal{A}$. Let $M$ be an object of $\calA$.

\begin{enumerate}
    \item A \hldef{right resolution of $M$} is a \CrefAndHyperrefIfExist{definition:chain_complex_of_objects_in_an_additive_category}{cochain complex} $I^\bullet$ with $I^i = 0$ for $i < 0$ and a map $M \to I^0$ such that the augmented complex
    $$0 \to M \to I^0 \to I^1 \to I^2 \to \cdots$$
    is \CrefAndHyperrefIfExist{definition:acyclic_complex_of_objects_in_an_abelian_category}{exact}.

    % \item An \hldef{injective resolution of $M$} is a right resolution $I^\bullet$ for which the objects $I^i$ are all \CrefAndHyperrefIfExist{definition:injective_and_projective_objects_in_a_category}{injective}.

    \item A \hldef{left resolution of $M$} is a \CrefAndHyperrefIfExist{definition:chain_complex_of_objects_in_an_additive_category}{chain complex} $P_\bullet$ with $P_i = 0$ for $i < 0$ and a map $P_0 \to M$ such that the augmented complex
    $$\cdots P_2 \to P_1 \to P_0 \to M \to 0$$
    is \CrefAndHyperrefIfExist{definition:acyclic_complex_of_objects_in_an_abelian_category}{exact}.
    
    \item An \hldef{$\mathcal{X}$-left resolution} of an object $M \in \mathcal{A}$ a \CrefAndHyperrefIfExist{definition:left_right_resolution_of_a_class_of_objects_in_an_abelian_category}{left resolution} by objects of $X$, i.e. an exact complex
    $$ \cdots \to X_2 \to X_1 \to X_0 \to M \to 0 $$
    with each $X_i \in \mathcal{X}$.

    \item An \hldef{$\mathcal{X}$-right resolution} of an object $M \in \mathcal{A}$ a \CrefAndHyperrefIfExist{definition:left_right_resolution_of_a_class_of_objects_in_an_abelian_category}{right resolution} by objects of $X$, i.e. an exact complex
    $$ 0 \to M \to X^0 \to X^1 \to X^2 \to \cdots $$
    with each $X_i \in \mathcal{X}$.

    \item A \hldef{projective resolution of $M$} is a left resolution $P^\bullet$ for which the objects $P^i$ are all \CrefAndHyperrefIfExist{definition:injective_and_projective_objects_in_a_category}{projective}.

    \item An \hldef{injective resolution of $M$} is a right resolution $I^\bullet$ for which the objects $I^i$ are all \CrefAndHyperrefIfExist{definition:injective_and_projective_objects_in_a_category}{injective}.
\end{enumerate}

\end{definition}
\begin{lemma} \label{lemma:projective_injective_objects_in_an_abelian_category_always_have_a_projective_injctive_resolution}
    Let $\calA$ be an \CrefAndHyperrefIfExist{definition:abelian_category}{abelian category}.
    \begin{enumerate}
        \item A \CrefAndHyperrefIfExist{definition:injective_and_projective_objects_in_a_category}{projective object} $\calP$ always has a \CrefAndHyperrefIfExist{definition:left_right_resolution_of_a_class_of_objects_in_an_abelian_category}{projective resolution} given by 
        $$\cdots \to 0 \to \calP \xrightarrow{\id} \calP \to 0.$$
        \item A \CrefAndHyperrefIfExist{definition:injective_and_projective_objects_in_a_category}{injective object} $\calI$ always has a \CrefAndHyperrefIfExist{definition:left_right_resolution_of_a_class_of_objects_in_an_abelian_category}{injective resolution} given by 
        $$0 \to \calI \xrightarrow{\id} \calI \to 0 \to \cdots.$$
    \end{enumerate}
\end{lemma}

\begin{proof}
    This is clear.
\end{proof}

\begin{lemma} [cf. {\cite[Lemma 2.2.5, Lemma 2.3.6]{weibel}}] \label{lemma:an_object_of_abelian_category_with_enough_objects_of_a_class_on_the_right_left_has_right_left_resolution_by_the_class}
Let $\mathcal{A}$ be an \CrefAndHyperrefIfExist{definition:abelian_category}{abelian category} and let $\calX$ be a class of objects in $\calA$.

\begin{enumerate}
    \item If $\mathcal{A}$ \CrefAndHyperrefIfExist{definition:has_enough_objects_of_a_class_on_the_left_right_for_an_abelian_category}{has enough objects of class $\calX$ on the right}, then for every object $A \in \mathcal{A}$ there exists an \CrefAndHyperrefIfExist{definition:left_right_resolution_of_a_class_of_objects_in_an_abelian_category}{$\calX$-right resolution of $A$}.

    \item If $\mathcal{A}$ \CrefAndHyperrefIfExist{definition:has_enough_objects_of_a_class_on_the_left_right_for_an_abelian_category}{has enough objects of class $\calX$ on the left}, then for every object $A \in \mathcal{A}$ there exists an \CrefAndHyperrefIfExist{definition:left_right_resolution_of_a_class_of_objects_in_an_abelian_category}{$\calX$-left resolution of $A$}.

\end{enumerate}

Note that this is a special case of \Cref{proposition:abelian_category_with_enough_objets_of_a_class_on_the_right_left_has_resolutions_of_complexes} obtained by letting the complex $M^\bullet$ be the complex such that
$$M^i = \begin{cases} A &\text{if } i = 0 \\ 0 &\text{otherwise} \end{cases}.$$


In particular,
\begin{itemize}
    \item If $\mathcal{A}$ \CrefAndHyperrefIfExist{definition:has_enough_injectives_or_projectives_for_an_abelian_category}{has enough injective objects}, then for every object $A \in \mathcal{A}$ there exists an \CrefAndHyperrefIfExist{definition:left_right_resolution_of_a_class_of_objects_in_an_abelian_category}{injective resolution of $A$}.
    \item If $\mathcal{A}$ \CrefAndHyperrefIfExist{definition:has_enough_injectives_or_projectives_for_an_abelian_category}{has enough projective objects}, then for every object $A \in \mathcal{A}$ there exists a \CrefAndHyperrefIfExist{definition:left_right_resolution_of_a_class_of_objects_in_an_abelian_category}{projective resolution of $A$}.
    \item If $F: \calA \to \calB$ is a \CrefAndHyperrefIfExist{definition:exact_functor_between_abelian_categories}{left (resp. right) exact functor} between abelian categories and $\calA$ has enough $F$-acyclic objects on the right (resp. left), then for every object $A \in \calA$, there exists an \CrefAndHyperrefIfExist{definition:F_acyclic_resolution_for_a_right_left_exact_functor_between_abelian_categories}{right (resp. left) $F$-acyclic resolution} of $A$.
\end{itemize}

\end{lemma}
\begin{proof}
    \begin{enumerate}
        \item Let $A \in \calA$ be an object. Since $\calA$ has enough objects of class $\calX$ of the right, there is an object $X_0$ of $\calX$ and a monomorphism $\varepsilon_0: A \to X_0$. Let \CrefAndHyperrefIfExist{definition:kernel_and_cokernel_of_a_morphism_in_a_category}{$A_0 = \operatorname{coker} \varepsilon_0$}. Inductively, given an object $A_{n-1}$ of $\calA$, choose an object $X_n$ of $\calX$ and a monomorphism $\varepsilon_{n}: A_{n-1} \hookrightarrow X_n$. Let $A_n = \operatorname{coker} \varepsilon_n$. In particular, there is a surjection $X_n \twoheadrightarrow A_n$. Let $d_n$ be the composition
        $$X_{n-1} \twoheadrightarrow A_{n-1} \xrightarrow{\varepsilon_n} X_n.$$
        The chain complex
        $$0 \to A \xrightarrow{\varepsilon_0} X_0 \xrightarrow{d_0} X_1 \xrightarrow{d_1} \cdots$$
        is thus an $\calX$-right resolution of $A$.

        \item This is simply the dual statement of the next statement.

    \end{enumerate} 
\end{proof}

In fact, a generalization is possible: if the abelian category $\calA$ has enough objects in a class $\calX$ (on the right/left) \emph{complex}, then the complex has a ``resolution'' by objects of $\calX$. 

\begin{lemma} \label{lemma:abelian_categories_are_finitely_complete_and_finitely_cocomplete}
    \CrefAndHyperrefIfExist{definition:abelian_category}{Abelian categories} are \CrefAndHyperrefIfExist{definition:complete_and_cocomplete_category}{finitely complete and finitely cocomplete}.
\end{lemma}
\begin{proof}
    Abelian categories have finite products and finite coproducts (in the form of direct sums), empty products and coproducts (in the form of the zero object), and \CrefAndHyperrefIfExist{definition:equalizer_and_coequalizer_of_morphisms_in_a_category}{equalizers and coequalizers} (in the form of kernels and cokernels of morphisms), so Theorem \ref{theorem:limits_colimits_are_equalizers_coequaulizers_of_products_coproducts} applies.
\end{proof}


\begin{proposition} \label{proposition:abelian_category_with_enough_objets_of_a_class_on_the_right_left_has_resolutions_of_complexes}
    Let $\mathcal{A}$ be an \CrefAndHyperrefIfExist{definition:abelian_category}{abelian category} and let $\calX$ be a class of objects in $\calA$.
    \begin{enumerate}
        \item If $\mathcal{A}$ \CrefAndHyperrefIfExist{definition:has_enough_objects_of_a_class_on_the_left_right_for_an_abelian_category}{has enough objects of class $\calX$ on the right}, then for every \CrefAndHyperrefIfExist{definition:bounded_complexes_on_an_additive_category_and_homologically_bounded_objects_on_an_abelian_category}{bounded below} complex $M^\bullet$ of objects in $\calA$, there exists a bounded below complex $I^\bullet$ of objects in $\calX$ and a \CrefAndHyperrefIfExist{definition:quasi_isomorphism_of_chain_complexes_of_objects_in_an_abelian_category}{quasi-isomorphism} $M^\bullet \to I^\bullet$.
        
        \item If $\mathcal{A}$ \CrefAndHyperrefIfExist{definition:has_enough_objects_of_a_class_on_the_left_right_for_an_abelian_category}{has enough objects of class $\calX$ on the left}, then for every \CrefAndHyperrefIfExist{definition:bounded_complexes_on_an_additive_category_and_homologically_bounded_objects_on_an_abelian_category}{bounded above} complex $M^\bullet$ of objects in $\calA$, there exists a bounded above complex $P^\bullet$ of objects in $\calX$ and a \CrefAndHyperrefIfExist{definition:quasi_isomorphism_of_chain_complexes_of_objects_in_an_abelian_category}{quasi-isomorphism} $P^\bullet \to M^\bullet$.
    \end{enumerate}
\end{proposition}
\begin{proof}
    We prove that if $\calA$ has enough objects of class $\calX$ on the left, then there exists a complex $P^\bullet$ of objects in $\calX$ and a quasi-isomorphism $P^\bullet \to M^\bullet$. The other statement can be proven basically symmetrically.

    First suppose that $M^\bullet$ is \CrefAndHyperrefIfExist{definition:bounded_complexes_on_an_additive_category_and_homologically_bounded_objects_on_an_abelian_category}{bounded above}; say that $M^i = 0$ for all $i > n$. We inductively construct $P^\bullet$ and the quasi-isomoprhism to $M^\bullet$. Choose an object $P^n$ from $\calX$ and a surjective morphism $\epsilon_n: P^n \twoheadrightarrow M^n$, and let $d: P^n \to P^{n+1}$ be the zero map. Assume inductively that we have constructed the complex $P^\bullet$ and maps $\epsilon_i: P^i \to M^i$ for $i = k+1,k+2,\ldots,n$. We want to construct $P^k$, the differential $d:P^k \to P^{k+1}$ and the map $\epsilon_k: P^k \to M^k$. 

    Let $L_k = Z^{k+1}(P) \times_{Z^{k+1}(M)} M^k$
    \begin{center}
    \begin{tikzcd}
        L_k \ar[r] \ar[d] & Z^{k+1}(P) \ar[d, "\epsilon_{k+1}"] \\
        M^k \ar[r, "d" ] & Z^{k+1}(M)
    \end{tikzcd}
    \end{center}
    
    where $Z^{i}$ denotes the $i$th \CrefAndHyperrefIfExist{definition:boundary_cycle_coboundary_cocyble_of_a_chain_cochain_complex}{cycle} of a complex \CrefIfExists{definition:homology_and_cohomology_objects_for_a_chain_complex_in_an_additive_category}; recall that abelian categories have finite limits by \Cref{lemma:abelian_categories_are_finitely_complete_and_finitely_cocomplete}, so \CrefAndHyperrefIfExist{definition:cartesian_product_of_two_objects_in_a_category_over_an_object}{fiber products} exist. Choose an object $P^k$ from $\calX$ and a surjective moprhism $\pi: P^k \twoheadrightarrow L_k$. Set the differential $d: P^k \to P^{k+1}$ to be $\operatorname{proj}_{Z^{k+1}(P)} \circ \pi$ and the map $\epsilon_k: P^k \to M^k$ to be $\operatorname{proj}_{M^k} \circ \pi$. 
    
    We verify that the square
    \begin{center}
        \begin{tikzcd}
            P^k \ar[r, "d"] \ar[d, "\epsilon_k"]  & P^{k+1} \ar[d, "\epsilon_{k+1}"] \\
            M^k \ar[r, "d"] & M^{k+1} 
        \end{tikzcd}
    \end{center}
    commutes, i.e. that $\epsilon_{k+1} \circ d = d \circ \varepsilon_k$. The left is $\epsilon_{k+1} \circ \operatorname{proj}_{Z^{k+1}(P)} \circ \pi$ and the right is $d \circ \operatorname{proj}_{M^k} \circ \pi$, which do indeed coincide. 

    We verify that the chain map $\epsilon$ induces isomorphisms $H^{*} (P) \xrightarrow{\sim} H^{*}(M)$ on cohomology objects. We first show that the induced maps on cohomology are epimorphisms. Let $u:Z^k(M) \to L_k = Z^{k+1}(P) \times_{Z^{k+1}(M)} M^k$ be the unique morphism corresponding to the morphisms $0: Z^k(M) \to Z^{k+1}(P)$ and $Z^k(M) \hookrightarrow M^k$. Let $Y$ be the pullback of $\pi$ along $u$:
    \begin{center}
    \begin{tikzcd}
        Y \ar[r, "\tilde{\pi}"] \ar[d, "\tilde{u}"] & Z^k(M) \ar[d, "u"] \\
        P^k \ar[r, "\pi"] & L_k.
    \end{tikzcd}
    \end{center}
    Note that $\operatorname{im} \tilde{u}$ is a subobject of $Z^k(P) = \ker(d: P^k \to P^{k+1})$ because
    $$d \circ \tilde{u} = \operatorname{proj}_{Z^{k+1}(P)} \circ \pi \circ \tilde{u} = \operatorname{proj}_{Z^{k+1}(P)} \circ u \circ \tilde{\pi} = 0.$$
    Therefore, $\tilde{u}$ factors through $Z^k(P)$. Writing $[\tilde{u}]$ for the composition $Y \xrightarrow{\tilde{u}} Z^k(P) \twoheadrightarrow H^k(P)$, note that 
    $$H^k(\epsilon) \circ [\tilde{u}] = [\epsilon_k \circ \tilde{u}] = [\operatorname{proj}_{M^k} \circ \pi \circ \tilde{u}] = [\operatorname{proj}_{M^k} \circ u \circ \tilde{\pi}] = [(\id: Z^k(M) \to Z^k(M)) \circ \tilde{\pi}].$$
    The right most expression is the composition 
    $$Y \xrightarrow{\tilde{\pi}} Z^k(M) \twoheadrightarrow H^k(M).$$
    Since $\pi$ is an epimorphism, $\tilde{\pi}$ is an epimorphism, so the above composition is an epimorphism. We have thus shown that $H^k(\epsilon) \circ [\tilde{u}]$ is an epimorphism, so $H^k(\epsilon)$ is an epimorphism.

    We now show that $H^{k+1}(\epsilon): H^{k+1}(P) \to H^{k+1}(M)$ is a monomorphism. Let $K$ be the kernel of $Z^{k+1}(P) \xrightarrow{\epsilon_{k+1}} Z^{k+1}(M) \twoheadrightarrow H^{k+1}(M)$; this kernel coincides with ``the (k+1)-cycles of $P$ mapping to (k+1)-boundaries of $M$''. More precisely, $K$ can be regarded as the fiber product
    \begin{center}
    \begin{tikzcd}
        K \ar[r] \ar[d] & Z^{k+1}(P) \ar[d, "\epsilon_{k+1}"] \\ 
        B^{k+1}(M) \ar[r, hookrightarrow] & Z^{k+1}(M),
    \end{tikzcd}
    \end{center}
    and note that this Cartesian diagram displays $K$ as a subobject of $Z^{k+1}(P)$. Furthe note that the morphism $d: M^k \to B^{k+1}(M)$ naturally induces a morphism $L_k \to K$; in fact, $K$ is then the image of the projection map $\operatorname{proj}_{Z^{k+1}(P)}: L_k \to Z^{k+1}(P)$. On the other hand, by definition, 
    $$B^{k+1}(P) = \operatorname{im}(d: P^k \to Z^{k+1}(P)) = \operatorname{im}(\operatorname{proj}_{Z^{k+1}(P)} \circ \pi).$$
    Since $\pi$ is an epimorphism, this image in turn equals $\operatorname{im}(\operatorname{proj}_{Z^{k+1}(P)})$, which equals $K$ as we have seen. Therefore, $K$ coincides with $B^{k+1}(P)$, which means that the map $Z^{k+1}(P) \to H^{k+1}(M)$, whose kernel is $K$ by definition, naturally induces a monomorphism $H^{k+1}(P) \to H^{k+1}(M)$ as desired.

\end{proof}

\begin{lemma}[cf. {\cite[Porism 2.2.7]{weibel}}] \label{lemma:projective_injective_complex_with_map_to_from_object_with_left_right_resolution_lifts_uniquely_up_to_chain_homotopy}
    Let $\calA$ be an \CrefAndHyperrefIfExist{definition:abelian_category}{abelian category}.
    \begin{enumerate}
        \item Let
        $$\cdots \to P_2 \to P_1 \to P_0 \to M \to 0$$
        be a \CrefAndHyperrefIfExist{definition:chain_complex_of_objects_in_an_additive_category}{chain complex} with $P_i$ \CrefAndHyperrefIfExist{definition:injective_and_projective_objects_in_a_category}{projective}. For every \CrefAndHyperrefIfExist{definition:left_right_resolution_of_a_class_of_objects_in_an_abelian_category}{left resolution} $Q_\bullet \to N$ of an object $N$, every map $M \to N$ lifts to a \CrefAndHyperrefIfExist{definition:chain_complex_of_objects_in_an_additive_category}{complex map} $P_\bullet \to Q_\bullet$ unique up to \CrefAndHyperrefIfExist{definition:chain_homotopy_between_chain_maps_between_complexes}{chain homotopy}.

        \item Let
        $$0 \to M \to I^0 \to I^1 \to I^2 \to \cdots$$
        be a \CrefAndHyperrefIfExist{definition:chain_complex_of_objects_in_an_additive_category}{(co)chain complex} with $I^i$ \CrefAndHyperrefIfExist{definition:injective_and_projective_objects_in_a_category}{injective}. For every \CrefAndHyperrefIfExist{definition:left_right_resolution_of_a_class_of_objects_in_an_abelian_category}{right resolution} $N \to Q^\bullet$ of an object $N$, every map $N \to M$ lifts to a \CrefAndHyperrefIfExist{definition:chain_complex_of_objects_in_an_additive_category}{complex map} $Q^\bullet \to I^\bullet$ unique up to \CrefAndHyperrefIfExist{definition:chain_homotopy_between_chain_maps_between_complexes}{chain homotopy}.
    \end{enumerate}
\end{lemma}

\begin{proof}
    \begin{enumerate}
        \item The map $P_0 \to M \to N$ lifts to a map $P_0 \to Q_0$ because $P_0$ is projective and $Q_0 \to N$ is an epimorphism. 
        Inductively suppose that there are morphisms $P_i \to Q_i$ for $0 \leq i \leq n$, where $n \geq 0$ that make 
        \begin{center}
        \begin{tikzcd}
            P_n \ar[r] \ar[d] & P_{n-1} \ar[r] \ar[d] & \cdots \ar[r] & P_0 \ar[r] \ar[d] & M \ar[r] \ar[d] & 0 \\
            Q_n \ar[r] & Q_{n-1} \ar[r] & \cdots \ar[r] & Q_0 \ar[r] & N \ar[r] & 0 \\
        \end{tikzcd}
        \end{center}
        into a commuting diagram are established. The morphism $Q_{n} \to Q_{n-1}$ (where we let $Q_{-1} = N$ and $P_{-1} = M$ here in case that $n = 0$) acts as $0$ when restricted to \CrefAndHyperrefIfExist{definition:image_coimage_of_a_morphism_in_a_category}{$\mathfrak{I} \coloneq \operatorname{im} (P_{n+1} \to P_n \to Q_n)$} because the composition 
        $$P_{n+1} \to P_n \to Q_n \to Q_{n-1}$$
        equals the composition 
        $$P_{n+1} \to P_n \to P_{n-1} \to Q_{n-1}.$$
        In other words, $\mathfrak{I}$ is a \CrefAndHyperrefIfExist{definition:subobject_of_an_object_of_an_additive_category}{subobject} of \CrefAndHyperrefIfExist{definition:kernel_and_cokernel_of_a_morphism_in_a_category}{$\ker(Q_n \to Q_{n-1})$}, which is isomorphic to $\operatorname{im}(Q_{n+1} \to Q_n)$ by the acyclicity of the sequence of the $Q_i$'s. Therefore, we have a map $P_{n+1} \twoheadrightarrow \mathfrak{I}\hookrightarrow \operatorname{im}(Q_{n+1} \to Q_n)$ along with an epimorphism $Q_{n+1} \twoheadrightarrow \operatorname{im}(Q_{n+1} \to Q_n)$. Since $P_{n+1}$ is projective, the former map lifts to a map $P_{n+1} \to Q_{n+1}$ in a way that is compatible with the latter, i.e. the following commutes:
        \begin{center}
        \begin{tikzcd}
            P_{n+1} \ar[rd] \ar[d,dotted] & \\
            Q_{n+1} \ar[r] & \operatorname{im}(Q_{n+1} \to Q_n).
        \end{tikzcd}
        \end{center}
        By induction, this shows thta $M \to N$ lifts to a morphism $P_\bullet \to Q_\bullet$ of complexes.

        We show that the morphism of complexes is unique up to chain homotopy, i.e. if $f_1, f_2: P_\bullet \to Q_\bullet$ are two morphisms of complexes, then $h \coloneq f_1 - f_2$ is null homotopic. We construct a \CrefAndHyperrefIfExist{definition:chain_homotopy_between_chain_maps_between_complexes}{chain contraction} $\{s_n: P_n \to Q_{n+1}\}$ of $h$ by induction on $n$. If $n < 0$, then set $s_n = 0$. If $n = 0$, note that the composition $P_0 \xrightarrow{h_0} Q_0 \to N$ equals the composition $P_0 \to M \xrightarrow{0} N$, so $\operatorname{im}(h_0)$ is a subobject of $\ker(Q_0 \to N) \cong \operatorname{im}(Q_1 \to Q_0)$. The projectivitiy of $P_0$ thus yields a lift $s_0: P_0 \to Q_1$ such that $h_0$ equals the composition $P_0 \xrightarrow{s_0} X_1 \xrightarrow{d} Q_0$:
        \begin{center}
        \begin{tikzcd}
           &  P_0 \ar[dl, "s_0", dotted] \ar[d, "h_0"] \\
           Q_1 \ar[r, "d"] & Q_0 
        \end{tikzcd}
        \end{center}
        Note moreover that $h_0 = ds_0 + s_{-1} d$ because $s_{-1} = 0$. Inductively suppose that we have maps $s_i$ for $i \leq n$  such that $h_n = d s_{n} + s_{n-1} d$ or equivalently that $ds_{n} = h_n - s_{n-1} d$. Consider the map $h_{n+1} - s_{n} d: P_{n+1} \to Q_{n+1}$. Compute
        $$d(h_{n+1} - s_{n} d) = dh_{n+1} - d s_{n} d = dh_{n+1} - (h_n - s_{n-1}d)d = (dh_{n+1} - h_n d) + s_{n-1} d d = 0$$
        Therefore, $\operatorname{im} (h_{n+1} - s_n d)$ is a subobject of $\ker(Q_{n+1} \to Q_{n}) \cong \operatorname{im}(Q_{n+2} \to Q_{n+1})$, which is in turn a quotient of $Q_{n+2}$. Since $P_{n+1}$ is projective, there is a morphism $s_{n+1}: P_{n+1} \to Q_{n+2}$ such that $d s_{n+1} = h_{n+1} - s_{n} d$. 
        \begin{center}
        \begin{tikzcd}
           &  P_{n+1} \ar[dl, dotted, "s_{n+1}"] \ar[d, "h_n - s_{n-1} d = ds_{n}"] \\
           Q_{n+2} \ar[r, "d"]  &\operatorname{im}(Q_{n+2} \to Q_{n+1}) \cong \ker(Q_{n+1} \to Q_{n})
        \end{tikzcd}
        \end{center}
        The $s_n$ thus form a chain contraction as needed.

        \item This is simply dual to the previous part.
    \end{enumerate}
\end{proof}

\subsection{Derived functors of right or left exact functors between abelian categories where the source category has enough projectives or injectives}



\begin{definition} \label{definition:left_right_derived_functors_of_a_right_left_exact_functor_between_abelian_categories_where_source_has_enough_projectives_injectives}
    \TODO{I think that the definition of derived categories might be doable for more general kinds of resolutions? Perhaps it is that if I have a right exact functor $F$, then $L^i F$ can be computed with resolutions of $F$-acyclic objects? \CrefIfExists{definition:F_acyclic_object_for_a_left_or_right_functor_between_abelian_categories}}
    \TODO{Apparently, left/right derived functors may be defined for functors that are additive and preserve finite coproducts, and not necessarily right/left exact; the exactness condition ensures that the zeroth derived functor agrees with $F$.}
Let $\mathcal{A}$ and $\mathcal{B}$ be \CrefAndHyperrefIfExist{definition:abelian_category}{abelian categories}, and let 
$$F: \mathcal{A} \to \mathcal{B}$$ 
be an \CrefAndHyperrefIfExist{definition:additive_functor_between_additive_categories}{additive functor}.

\begin{enumerate}
    \item Suppose that the functor $F$ is \CrefAndHyperrefIfExist{definition:exact_functor_between_abelian_categories}{right exact} and suppose that $A \in \calA$ is an object for which a \CrefAndHyperrefIfExist{definition:left_right_resolution_of_a_class_of_objects_in_an_abelian_category}{projective resolution}
    $$\cdots \to P_2 \to P_1 \to P_0 \to A \to 0$$
    exists in $\calA$. We define the \hldef{left derived object} \hl{$L_n F A \in \calB$} by applying $F$ to obtain a complex
    $$\cdots \to F(P_2) \to F(P_1) \to F(P_0) \to 0$$
    and letting $L_n F(A)$ be the \CrefAndHyperrefIfExist{definition:homology_and_cohomology_objects_for_a_chain_complex_in_an_additive_category}{$n$-th homology object} of this complex in $\mathcal{B}$:
    $$L_n F(A) := H_n(F(P_\bullet)).$$
    The object $L_n F(A)$ is independent of the choice of projective resolution up to natural isomorphism (\Cref{proposition:left_right_derived_objects_for_a_right_left_exact_functor_between_abelian_categories_are_well_defined}). 

    By convention, set $L_n F = 0$ for $n < 0$.

    The \hldef{higher left derived objects} refer to the object $L_n F(A)$ for $n > 0$. 

    \item  Suppose that the functor $F$ is \CrefAndHyperrefIfExist{definition:exact_functor_between_abelian_categories}{right exact} and that $\calA$ \CrefAndHyperrefIfExist{definition:has_enough_injectives_or_projectives_for_an_abelian_category}{has enough projectives}. The \hldef{left derived functors} refer to the family of functors
    $$\hlin{L_n F : \mathcal{A} \to \mathcal{B}, \quad A \mapsto L_n F(A).}$$
    The \hldef{higher left derived functors} refer to the functors $L_n F$ for $n > 0$. 

    \item Suppose that the functor $F$ is \CrefAndHyperrefIfExist{definition:exact_functor_between_abelian_categories}{right exact} and suppose that $A \in \calA$ is an object for which a \CrefAndHyperrefIfExist{definition:left_right_resolution_of_a_class_of_objects_in_an_abelian_category}{injective resolution}
    $$0 \to A \to I^0 \to I^1 \to I^2 \to \cdots$$
    exists in $\calA$. We define the \hldef{right derived object} \hl{$R_n F A \in \calB$}, also often denoted by \hl{$R^n FA$}, by applying $F$ to obtain a complex
    $$0 \to F(I^0) \to F(I^1) \to F(I^2) \to \cdots.$$
    and letting $R_n F(A)$ be the \CrefAndHyperrefIfExist{definition:homology_and_cohomology_objects_for_a_chain_complex_in_an_additive_category}{$n$-th cohomology object} of this complex in $\mathcal{B}$:
    $$R_n F(A) := H^n(F(I_\bullet)).$$
    The object $R_n F(A)$ is independent of the choice of injective resolution up to natural isomorphism (\Cref{proposition:left_right_derived_objects_for_a_right_left_exact_functor_between_abelian_categories_are_well_defined}). 

    By convention, set $R_n F = 0$ for $n < 0$.

    The \hldef{higher right derived objects} refer to the object $R_n F(A)$ for $n > 0$. 

    \item  Suppose that the functor $F$ is \CrefAndHyperrefIfExist{definition:exact_functor_between_abelian_categories}{right exact} and that $\calA$ \CrefAndHyperrefIfExist{definition:has_enough_injectives_or_projectives_for_an_abelian_category}{has enough injectives}. The \hldef{right derived functors} refer to the family of functors
    $$\hlin{R_n F : \mathcal{A} \to \mathcal{B}, \quad A \mapsto R_n F(A).}$$
    The right derived functors are also often denoted by \hl{$R^n F$}.
    The \hldef{higher right derived functors} refer to the functors $R_n F$ for $n > 0$. 

    
    % If the functor $F$ is right exact and $\calA$ \CrefAndHyperrefIfExist{definition:has_enough_injectives_or_projectives_for_an_abelian_category}{has enough projectives}, then its \hldef{left derived functors} are a family of functors
    % $$\hlin{L_n F : \mathcal{A} \to \mathcal{B}, \quad n \geq 0,}$$
    % which are defined for each object $A$ in $\mathcal{A}$ by choosing (\Cref{lemma:an_object_of_abelian_category_with_enough_objects_of_a_class_on_the_right_left_has_right_left_resolution_by_the_class}) a \CrefAndHyperrefIfExist{definition:left_right_resolution_of_a_class_of_objects_in_an_abelian_category}{projective resolution}
    % $$\cdots \to P_2 \to P_1 \to P_0 \to A \to 0$$
    % in $\mathcal{A}$ and applying $F$ to obtain a complex
    % $$\cdots \to F(P_2) \to F(P_1) \to F(P_0) \to 0.$$
    % Then $L_n F(A)$ is defined to be the \CrefAndHyperrefIfExist{definition:homology_and_cohomology_objects_for_a_chain_complex_in_an_additive_category}{$n$-th homology object} of this complex in $\mathcal{B}$:
    % $$L_n F(A) := H_n(F(P_\bullet)).$$
    % The functors $L_n F$ are independent of the choice of projective resolution up to natural isomorphism. 

    % By convention, set $L_n F = 0$ for $n < 0$.

    % \item If the functor $F$ is \CrefAndHyperrefIfExist{definition:exact_functor_between_abelian_categories}{left exact} and $\calA$ \CrefAndHyperrefIfExist{definition:has_enough_injectives_or_projectives_for_an_abelian_category}{has enough injectives}, then its \hldef{right derived functors} are a family of functors
    % $$\hlin{R^n F : \mathcal{A} \to \mathcal{B}, \quad n \geq 0,}$$
    % which are defined for each object $A$ in $\mathcal{A}$ by choosing (\Cref{lemma:an_object_of_abelian_category_with_enough_objects_of_a_class_on_the_right_left_has_right_left_resolution_by_the_class}) an \CrefAndHyperrefIfExist{definition:left_right_resolution_of_a_class_of_objects_in_an_abelian_category}{injective resolution}
    % $$0 \to A \to I^0 \to I^1 \to I^2 \to \cdots$$
    % in $\mathcal{A}$ and applying $F$ to obtain a complex
    % $$0 \to F(I^0) \to F(I^1) \to F(I^2) \to \cdots.$$
    % Then $R^n F(A)$ is defined to be the \CrefAndHyperrefIfExist{definition:homology_and_cohomology_objects_for_a_chain_complex_in_an_additive_category}{$n$-th cohomology object} of this complex in $\mathcal{B}$:
    % $$R^n F(A) := H^n(F(I^\bullet)).$$
    % The functors $R^n F$ are independent of the choice of injective resolution up to natural isomorphism.

    % By convention, set $R^n F = 0$ for $n < 0$.
\end{enumerate}
\end{definition}
\begin{lemma}[Horseshoe lemma, cf. {\cite[Horsehoe Lemma 2.2.8]{weibel}}] \label{lemma:horseshoe_lemma_of_projective_injective_resolutions_in_abelian_categories}
    Let $\calA$ be an \CrefAndHyperrefIfExist{definition:abelian_category}{abelian category}. 
    \begin{enumerate}
        \item Suppose that
            $$0 \to A' \xrightarrow{i_A} A \xrightarrow{\pi_A} A'' \to 0$$
            is a short exact sequence in $\calA$, and that $\varepsilon': P'_\bullet \to A'$ and $\varepsilon'': P''_{\bullet} \to A''$ are respectively \CrefAndHyperrefIfExist{lemma:flat_resolution_lemma_of_tor_objects_of_abelian_categories_with_a_right_exact_bifunctor_assuming_that_category_has_enough_projectives_or_flats}{projective resolutions}. 
            \begin{center}
                \begin{tikzcd}
                & & & 0 \ar[d] & \\
                \cdots P_2' \ar[r] & P_1' \ar[r] & P_0' \ar[r, "\varepsilon'"] & A' \ar[r] \ar[d, "i_A"] & 0 \\
                & & & A \ar[d, "\pi_A"] & \\
                \cdots P_2'' \ar[r] & P_1'' \ar[r] & P_0'' \ar[r, "\varepsilon''"] & A'' \ar[r] \ar[d] & 0 \\ 
                & & & 0 & 
                \end{tikzcd}
            \end{center}
            Let $P_\bullet = P'_\bullet \oplus P''_\bullet$. The complex $P_\bullet$ is a projective resolution of $A$, and the short exact sequence lifts to an exact esquence of complexes
            $$0 \to P' \xrightarrow{i} P \xrightarrow{\pi} P'' \to 0$$
            where $i_n: P_n' \to P_n$ and $\pi_n: P_n \to P_n''$ are the natural inclusion and projection respectively.
        \item Suppose that
        $$0 \to A' \xrightarrow{i_A} A \xrightarrow{\pi_A} A'' \to 0$$
        is a short exact sequence in $\calA$, and that $\eta': A' \to I'^\bullet$ and $\eta'': A'' \to I''^\bullet$ are respectively \CrefAndHyperrefIfExist{lemma:injective_resolution_lemma}{injective resolutions}. 

        \begin{center}
            \begin{tikzcd} 
            & 0 \ar[d] & & & \\
            0 \ar[r] & A' \ar[r, "\eta'"] \ar[d, "i_A"] & {I'}^{0} \ar[r] & {I'}^{1} \ar[r] & {I'}^{2} \cdots \\
            & A \ar[d, "\pi_A"] & & & \\
            0 \ar[r] & A' \ar[r, "\eta''"] \ar[d] & {I''}^{0} \ar[r] & {I''}^{1} \ar[r] & {I''}^{2} \cdots \\
            & 0 & & & 
            \end{tikzcd}
        \end{center}
        Let $I^\bullet = I'^\bullet \oplus I''^\bullet$. The complex $I^\bullet$ is an injective resolution of $A$, and the short exact sequence lifts to a short exact sequence of complexes
        $$0 \to I'^\bullet \xrightarrow{i} I^\bullet \xrightarrow{\pi} I''^\bullet \to 0,$$
        where $i^n: I'^n \to I^n$ and $\pi^n: I^n \to I''^n$ are the natural inclusion and projection at each degree \(n\).
    \end{enumerate}
\end{lemma}

% \TODO{delete the following; it seems we genuinely need the projectivity/injectivity}
% \begin{lemma}
%     Let $\calA$ be an \CrefAndHyperrefIfExist{definition:abelian_category}{abelian category}. 
%     Let $\calX$ be a class of objects in $\calA$ such that $\calX$ is closed under \CrefAndHyperrefIfExist{definition:additive_category_preadditive_category}{direct sums}. 
%     \begin{enumerate}
%         \item Suppose that
%             $$0 \to A' \xrightarrow{i_A} A \xrightarrow{\pi_A} A'' \to 0$$
%             is a short exact sequence in $\calA$, and that $\varepsilon': P'_\bullet \to A'$ and $\varepsilon'': P''_{\bullet} \to A''$ are respectively \CrefAndHyperrefIfExist{definition:left_right_resolution_of_a_class_of_objects_in_an_abelian_category}{$\calX$-left resolutions}. 
%             \begin{center}
%                 \begin{tikzcd}
%                 & & & 0 \ar[d] & \\
%                 \cdots P_2' \ar[r] & P_1' \ar[r] & P_0' \ar[r, "\varepsilon'"] & A' \ar[r] \ar[d, "i_A"] & 0 \\
%                 & & & A \ar[d, "\pi_A"] & \\
%                 \cdots P_2'' \ar[r] & P_1'' \ar[r] & P_0'' \ar[r, "\varepsilon''"] & A'' \ar[r] \ar[d] & 0 \\ 
%                 & & & 0 & 
%                 \end{tikzcd}
%             \end{center}
%             Let $P_\bullet = P'_\bullet \oplus P''_\bullet$. The complex $P_\bullet$ is an $\calX$-left resolution of $A$, and the short exact sequence lifts to an exact esquence of complexes
%             $$0 \to P' \xrightarrow{i} P \xrightarrow{\pi} P'' \to 0$$
%             where $i_n: P_n' \to P_n$ and $\pi_n: P_n \to P_n''$ are the natural inclusion and projection respectively.
%         \item Suppose that
%         $$0 \to A' \xrightarrow{i_A} A \xrightarrow{\pi_A} A'' \to 0$$
%         is a short exact sequence in $\calA$, and that $\eta': A' \to I'^\bullet$ and $\eta'': A'' \to I''^\bullet$ are respectively \CrefAndHyperrefIfExist{definition:left_right_resolution_of_a_class_of_objects_in_an_abelian_category}{$\calX$-right resolutions}. 

%         \begin{center}
%             \begin{tikzcd} 
%             & 0 \ar[d] & & & \\
%             0 \ar[r] & A' \ar[r, "\eta'"] \ar[d, "i_A"] & {I'}^{0} \ar[r] & {I'}^{1} \ar[r] & {I'}^{2} \cdots \\
%             & A \ar[d, "\pi_A"] & & & \\
%             0 \ar[r] & A' \ar[r, "\eta''"] \ar[d] & {I''}^{0} \ar[r] & {I''}^{1} \ar[r] & {I''}^{2} \cdots \\
%             & 0 & & & 
%             \end{tikzcd}
%         \end{center}
%         Let $I^\bullet = I'^\bullet \oplus I''^\bullet$. The complex $I^\bullet$ is an $\calX$-right resolution of $A$, and the short exact sequence lifts to a short exact sequence of complexes
%         $$0 \to I'^\bullet \xrightarrow{i} I^\bullet \xrightarrow{\pi} I''^\bullet \to 0,$$
%         where $i^n: I'^n \to I^n$ and $\pi^n: I^n \to I''^n$ are the natural inclusion and projection at each degree \(n\).
%     \end{enumerate}

% \end{lemma}

\begin{proof}
    \TODO{}
\end{proof}
\begin{proposition}[cf.{\cite[Lemma 2.4.1]{weibel}}] \label{proposition:left_right_derived_objects_for_a_right_left_exact_functor_between_abelian_categories_are_well_defined}
    Let $F: \calA \to \calB$ be an \CrefAndHyperrefIfExist{definition:additive_functor_between_additive_categories}{additive functor} between \CrefAndHyperrefIfExist{definition:abelian_category}{abelian categories}. Let $A$ be an object of $\calA$. 
    \begin{enumerate}
        \item Suppose that $F$ is \CrefAndHyperrefIfExist{definition:exact_functor_between_abelian_categories}{right exact}, and suppose that a \CrefAndHyperrefIfExist{definition:left_right_resolution_of_a_class_of_objects_in_an_abelian_category}{projective resolution}
        $$\cdots \to P_2 \to P_1 \to P_0 \to A \to 0$$
        of $A$ exists in $\calA$. Let 
        $$\cdots \to Q_2 \to Q_1 \to Q_0 \to A \to 0$$
        be any projective resolution of $A$ in $\calA$. For all $n$, there are natural isomorphisms
        $$H_n(F(P_\bullet)) \cong H_n(F(Q_\bullet)).$$
        In other words, the \CrefAndHyperrefIfExist{definition:left_right_derived_functors_of_a_right_left_exact_functor_between_abelian_categories_where_source_has_enough_projectives_injectives}{left derived objects $L_n F(A)$} is well defined.

        \item Suppose that $F$ is \CrefAndHyperrefIfExist{definition:exact_functor_between_abelian_categories}{left exact}, and suppose that a \CrefAndHyperrefIfExist{definition:left_right_resolution_of_a_class_of_objects_in_an_abelian_category}{injective resolution}
        $$0 \to A \to I^0 \to I^1 \to I^2 \to \cdots$$
        of $A$ exists in $\calA$. Let 
        $$0 \to A \to Q^0 \to Q^1 \to Q^2 \to \cdots$$
        be any injective resolution of $A$ in $\calA$. For all $n$, there are natural isomorphisms
        $$H_n(F(I^\bullet)) \cong H_n(F(Q^\bullet)).$$
        In other words, the \CrefAndHyperrefIfExist{definition:left_right_derived_functors_of_a_right_left_exact_functor_between_abelian_categories_where_source_has_enough_projectives_injectives}{right derived objects $R_n F(A)$} is well defined.
    \end{enumerate}
\end{proposition}

\begin{proof}
    \begin{enumerate}
        \item By \Cref{lemma:projective_injective_complex_with_map_to_from_object_with_left_right_resolution_lifts_uniquely_up_to_chain_homotopy}, there is a lift $f: P_\bullet \to Q_\bullet$ of the identity map $A \to A$ unique up to chain homotopy. There are then induced natural maps $H_n(F(f)): H_n(F(P_\bullet)) \to H_n(F(Q_\bullet))$. There is also a lift $f': Q_\bullet \to P_\bullet$ of the identity map $A \to A$ unique up to chain homotopy, and this also induces natural maps $H_n(F(f')): H_n(F(Q_\bullet)) \to H_n(F(P_\bullet))$. The chain maps $f$ and $f'$ are in fact \CrefAndHyperrefIfExist{definition:chain_homotopy_between_chain_maps_between_complexes}{chain homotopy inverses} because \Cref{lemma:projective_injective_complex_with_map_to_from_object_with_left_right_resolution_lifts_uniquely_up_to_chain_homotopy} also implies that any lifts $P_\bullet \to P_\bullet$ and $Q_\bullet \to Q_\bullet$ of the identity map $A \to A$ are chain homotopic to the identity chain maps. Therefore, $H_n(F(f))$ and $H_n(F(f'))$ are inverses of each other as morphisms in $\calB$.  \TODO{prove basic facts about the fucntoriality of homology/cohomology of chain complexes}

        \item This is dual to the previous part.
    \end{enumerate}
\end{proof}


\subsection{Tor and Ext}


Recall by \Cref{proposition:left_right_adjoint_is_right_left_exact} that left/right adjoint functors between abelian categories are right/left exact. Also recall the \CrefAndHyperrefIfExist{theorem:tensor_hom_adjunction_for_bimodules_of_rings}{tensor-hom adjunction} of modules. We can thus define derived functors as follows:

\begin{definition} \label{definition:tor_functors_of_bimodules_of_rings}
    Let $R,S,T$ be \CrefAndHyperrefIfExist{definition:ring}{(not necessarily commutative) rings}, let $M$ be an \CrefAndHyperrefIfExist{definition:module_of_a_ring}{$R$-$S$ bimodule}, and let $N$ be an $S$-$T$ bimodule. Let $n \geq 0$ be an integer.
    The \hldef{Tor of $M$ and $N$}, denoted by \hl{$\Tor_n(M,N)$} or \hl{$\Tor_n^S(M,N)$} (or other notations such as \hl{$\Tor_n^S(_{R} M_S, {}_S N_T))$} or \hl{$_{R}(\Tor_n^S(M,N))_{T}$} to further emphasize the module structures), is defined in the following ways:
    \begin{enumerate}
        \item as the \CrefAndHyperrefIfExist{definition:left_right_derived_functors_of_a_right_left_exact_functor_between_abelian_categories_where_source_has_enough_projectives_injectives}{$n$th left derived functor} $L_n(M \otimes_S -)(N)$ of the \CrefAndHyperrefIfExist{definition:exact_functor_between_abelian_categories}{right exact functor} $M \otimes_S - : _{S} \mathbf{Mod}_{T} \to  _{R} \mathbf{Mod}_{T}$ \CrefIfExists{definition:category_of_modules_and_bimodules_over_rings}\CrefIfExists{definition:tensor_product_of_bimodules_of_rings} (\Cref{theorem:tensor_hom_adjunction_for_bimodules_of_rings}) (\Cref{proposition:left_right_adjoint_is_right_left_exact})

        \item as the \CrefAndHyperrefIfExist{definition:left_right_derived_functors_of_a_right_left_exact_functor_between_abelian_categories_where_source_has_enough_projectives_injectives}{$n$th left derived functor} $L_n(- \otimes_S N)(M)$ of the \CrefAndHyperrefIfExist{definition:exact_functor_between_abelian_categories}{right exact functor} $- \otimes_S N : {}_R \mathbf{Mod}_S \to {}_R \mathbf{Mod}_T$ \CrefIfExists{definition:category_of_modules_and_bimodules_over_rings}\CrefIfExists{definition:tensor_product_of_bimodules_of_rings} (\Cref{theorem:tensor_hom_adjunction_for_bimodules_of_rings}) (\Cref{proposition:left_right_adjoint_is_right_left_exact}).
    \end{enumerate}
    The two definitions turn out to coincide up to natural isomorphism. \TODO{ref}
    \TODO{discuss how it can be computed by flat's or projectives}
\end{definition}

\begin{definition} \label{definition:ext_functors_of_bimodules_of_rings}
    Let $R,S,T$ be \CrefAndHyperrefIfExist{definition:ring}{(not necessarily commutative) rings}, let $M$ be an \CrefAndHyperrefIfExist{definition:module_of_a_ring}{$R$-$S$ bimodule}, and let $N$ be an $R$-$T$ bimodule. Let $n \geq 0$ be an integer.
    The \hldef{Ext of $M$ and $N$}, denoted by \hl{$\Ext^n(M,N)$} or \hl{$\Ext_R^n(M,N)$} (or other notations such as \hl{$\Ext_R^n({}_{R} M_S, {}_{R} N_T)$} or \hl{${}_{S}(\Ext_R^n(M,N))_{T}$} to further emphasize the module structures), is defined in the following ways:
    \begin{enumerate}
        \item as the \CrefAndHyperrefIfExist{definition:left_right_derived_functors_of_a_right_left_exact_functor_between_abelian_categories_where_source_has_enough_projectives_injectives}{$n$th right derived functor} $R^n \Hom_R(M, -)(N)$ of the \CrefAndHyperrefIfExist{definition:exact_functor_between_abelian_categories}{left exact functor} $\Hom_R(M, -) : {}_{R} \mathbf{Mod}_{T} \to {}_{S} \mathbf{Mod}_{T}$ \CrefIfExists{definition:category_of_modules_and_bimodules_over_rings}\CrefIfExists{definition:hom_functor_of_bimodules_of_rings} (\Cref{theorem:tensor_hom_adjunction_for_bimodules_of_rings}) (\Cref{proposition:left_right_adjoint_is_right_left_exact})

        \item as the \CrefAndHyperrefIfExist{definition:left_right_derived_functors_of_a_right_left_exact_functor_between_abelian_categories_where_source_has_enough_projectives_injectives}{$n$th right derived functor} $R^n \Hom_R(-, N)(M)$ of the \CrefAndHyperrefIfExist{definition:exact_functor_between_abelian_categories}{left exact functor} $\Hom_R(-, N) : ({}_{R} \mathbf{Mod}_{S})^{\text{op}} \to {}_{S} \mathbf{Mod}_{T}$ \CrefIfExists{definition:category_of_modules_and_bimodules_over_rings}\CrefIfExists{definition:opposite_category_of_a_category}\CrefIfExists{definition:hom_functor_of_bimodules_of_rings} (\Cref{theorem:tensor_hom_adjunction_for_bimodules_of_rings}) (\Cref{proposition:left_right_adjoint_is_right_left_exact}).
    \end{enumerate}
    The two definitions turn out to coincide up to natural isomorphism. \TODO{ref}
\end{definition}


\subsection{Balancing Ext and Tor}

The goal is to show that the two definitions of \CrefAndHyperrefIfExist{definition:ext_functors_of_bimodules_of_rings}{$\Ext$} agree with each other and similarly for \CrefAndHyperrefIfExist{definition:tor_functors_of_bimodules_of_rings}{$\Tor$}\TODO{ref}. 

\subsubsection{Double complexes}

\begin{definition} \label{definition:double_complex_of_objects_in_an_additive_category}
Let $\mathcal{A}$ be an \CrefAndHyperrefIfExist{definition:additive_category}{additive category}. A \hldef{double complex} (also called a \hldef{bicomplex}) in $\mathcal{A}$ is a collection of objects
$$ \{ A^{p,q} \}_{(p,q)\in\mathbb{Z}^2} $$
together with morphisms
$$ d'_A{}^{p,q} : A^{p,q} \to A^{p+1,q}, \quad d''_A{}^{p,q} : A^{p,q} \to A^{p,q+1}, $$
satisfying the identities
$$ (d'_A)^{p+1,q} \circ (d'_A)^{p,q} = 0, \qquad (d''_A)^{p,q+1} \circ (d''_A)^{p,q} = 0, $$
and the \hldef{anti-commutativity relation}
$$ (d''_A)^{p+1,q} \circ (d'_A)^{p,q} + (d'_A)^{p,q+1} \circ (d''_A)^{p,q} = 0.  $$

An alternative (equivalent) convention defines a double complex $(A^{p,q},d'_A,d''_A)$ so that
$$ (d''_A)^{p+1,q} \circ (d'_A)^{p,q} - (d'_A)^{p,q+1} \circ (d''_A)^{p,q} = 0.  $$
This convention differs by a sign and corresponds to replacing $d''_A$ by $-d''_A$. Thus, both conventions are equivalent up to the natural isomorphism $A^{p,q}\mapsto A^{p,q}$, $d'_A\mapsto d'_A$, $d''_A\mapsto -d''_A$.
\end{definition}
\begin{definition} \label{definition:morphism_of_double_complex_of_objects_in_an_additive_category}
Let $\mathcal{A}$ be an \CrefAndHyperrefIfExist{definition:additive_category_preadditive_category}{additive category}, and let
$$(A^{p,q}, d'_A, d''_A)_{(p,q)\in \mathbb{Z}^2}, \quad (B^{p,q}, d'_B, d''_B)_{(p,q)\in \mathbb{Z}^2}$$
be \CrefAndHyperrefIfExist{definition:double_complex_of_objects_in_an_additive_category}{double complexes} in $\mathcal{A}$.

A \hldef{morphism of double complexes} $f : A \to B$ is a collection of morphisms
$$ f^{p,q} : A^{p,q} \to B^{p,q}, $$
for all $(p,q) \in \mathbb{Z}^2$, such that the following diagrams commute:

    \hlalign{
    \begin{align*}
    d'_B{}^{p,q} \circ f^{p,q} &= f^{p+1,q} \circ d'_A{}^{p,q}, \\ 
    d''_B{}^{p,q} \circ f^{p,q} &= f^{p,q+1} \circ d''_A{}^{p,q}.
    \end{align*}
    }

In other words, $f$ respects both horizontal and vertical differentials of the double complexes.

The double complexes and their morphisms form a category, sometimes denoted by \hl{$\mathbf{DC}(\mathcal{A})$}.
\end{definition}
\begin{definition} \label{definition:bounded_double_complex_of_objects_in_an_additive_category}
Let $(A^{p,q},d'_A,d''_A)$ be a double complex in an additive category $\mathcal{A}$.
\begin{itemize}
  \item The double complex is called \hldef{bounded above} if there exist integers $p_0,q_0$ such that $A^{p,q}=0$ whenever $p>p_0$ or $q>q_0$.
  \item The double complex is called \hldef{bounded below} if there exist integers $p_0,q_0$ such that $A^{p,q}=0$ whenever $p<p_0$ or $q<q_0$.

  \item The double complex is called \hldef{bounded} if it is both bounded above and below. 
  \item The double complex is said to be in the \hldef{first quadrant} (also called \emph{first-quadrant double complex}) if $A^{p,q}=0$ whenever $p<0$ or $q<0$. In particular, any first quadrant double complex is bounded below.
  \item The double complex is said to be in the \hldef{third quadrant} (also called \emph{third-quadrant double complex}) if $A^{p,q}=0$ whenever $p>0$ or $q>0$. In particular, any third quadrant double complex is bounded above.

  \item Let us say that the double complex is \hldef{locally finite along diagonals} or \hldef{locally bounded along diagonals}\footnote{These do not seem to be standard teminology.} if for each integer $n$, there exist at most finitely many pairs $(p,q)$ with $p+q = n$ such thta $A^{p,q} \neq 0$. 
  
  \item Let us say that the double complex is \hldef{bounded in total degree}\footnote{This does not seem to be standard teminology.} if there exist integers $m$ and $M$ such that $A^{p,q} = 0$ whenever $m \leq p+q \leq M$. 
\end{itemize}
\end{definition}
\begin{definition} \label{definition:total_complexes_of_a_double_complex_of_objects_in_an_additive_category}
Let $(A^{p,q},d'_A,d''_A)$ be a \CrefAndHyperrefIfExist{definition:double_complex_of_objects_in_an_additive_category}{double complex} in an \CrefAndHyperrefIfExist{definition:additive_category_preadditive_category}{additive category} $\mathcal{A}$. For each integer $n$, define:

    \hlalign{
    \begin{align*}
    \Tot_{\oplus}^n(A) &= \bigoplus_{p+q=n} A^{p,q}, \\
    \Tot_{\Pi}^n(A) &= \prod_{p+q=n} A^{p,q}.
    \end{align*}
    }

assuming that the direct sum $\bigoplus_{p+q=n} A^{p,q}$ and the product $\prod_{p+q=n} A^{p,q}$ respectively exist.
These are called the \hldef{direct-sum total complex} and the \hldef{product total complex}, respectively.

Define the differentials categorically as follows:

\begin{itemize}
  \item For the direct sum total complex, the differential $d^n : \Tot_{\oplus}^n(A) \to \Tot_{\oplus}^{n+1}(A)$ is the unique morphism such that, for each $(p,q)$ with $p+q=n$, we have
    \[
    d^n \circ \iota_{p,q} =
      \iota_{p+1,q} \circ d'_A {}^{p,q}
      + (-1)^p\, \iota_{p,q+1} \circ d''_A {}^{p,q},
    \]
    where $\iota_{p,q} : A^{p,q} \to \Tot_{\oplus}^n(A)$ is the canonical inclusion.

  \item For the product total complex, the differential $d^n : \Tot_{\Pi}^n(A) \to \Tot_{\Pi}^{n+1}(A)$ is the unique morphism such that, for each $(p,q)$ with $p+q=n+1$, we have
    \[
    \pi_{p,q} \circ d^n =
      d'_A {}^{p-1,q} \circ \pi_{p-1,q}
      + (-1)^{p-1}\, d''_A {}^{p,q-1} \circ \pi_{p,q-1},
    \]
    where $\pi_{p,q} : \Tot_{\Pi}^{n+1}(A) \to A^{p,q}$ is the canonical projection.
\end{itemize}

Then $(\Tot_{\oplus}^\bullet(A), d)$ and $(\Tot_{\Pi}^\bullet(A), d)$ are 
\CrefAndHyperrefIfExist{definition:chain_complex_of_objects_in_an_additive_category}{chain complexes} in~$\mathcal{A}$, whenever the corresponding sums or products exist. 
The complexes $\Tot_{\oplus}^\bullet(A)$ and $\Tot_{\Pi}^\bullet(A)$ are also denoted by \hl{$\Tot^{\oplus}(A)$} and \hl{$\Tot^{\Pi}(A)$}.
\end{definition}
\begin{definition}[Double Complex associated to biadditive functor and chain complexes] \label{definition:double_complex_associated_to_biadditive_functor_and_chain_complexes}
    Let $\mathcal{A}$, $\mathcal{B}$, and $\calC$ be \CrefAndHyperrefIfExist{definition:additive_category}{additive categories}, and suppose that
    $$F : \mathcal{A} \times \mathcal{B} \to \mathcal{C} $$
    is a \CrefAndHyperrefIfExist{definition:n_ary_additive_functor_between_additive_categories}{biadditive functor}. Let $X_\bullet$ and $Y_\bullet$ be \CrefAndHyperrefIfExist{definition:chain_complex_of_objects_in_an_additive_category}{chain complexes} of objects in $\calA$ and $\calB$ respectively.

    Construct the \CrefAndHyperrefIfExist{definition:double_complex_of_objects_in_an_additive_category}{double complex} \(Z_{\bullet,\bullet} = (Z_{n,m}, d^h_{n,m}, d^v_{n,m})\) in \(C\) associated to \(F\), \(X_\bullet\), and \(Y_\bullet\) as follows:
    \[ Z_{n,m} := F(X_n, Y_m), \]
    with horizontal differentials
    \[ d^h_{n,m} := F(d^X_n, \mathrm{id}_{Y_m}): Z_{n,m} \to Z_{n-1,m}, \]
    and vertical differentials
    \[ d^v_{n,m} := (-1)^n F(\mathrm{id}_{X_n}, d^Y_m): Z_{n,m} \to Z_{n,m-1}.  \]
    These differentials indeed satisfy the double complex conditions:
    \[ d^h \circ d^h = 0, \quad d^v \circ d^v = 0, \quad \text{and} \quad d^h \circ d^v + d^v \circ d^h = 0.  \]
    We may often denote the double complex $Z_{\bullet,\bullet}$ by \hl{$F(X_\bullet, Y_\bullet)$}. In particular, $F$ induces a bifunctor  
    $$F: \mathbf{Ch}(\calA) \times \mathbf{Ch}(\calB) \to \mathbf{DC}(\calC)$$
    (\Cref{definition:morphism_of_double_complex_of_objects_in_an_additive_category})
    that is in fact a biadditive functor of additive categories (see \Cref{proposition:category_of_chain_complexes_in_an_additive_category_is_additive}) \TODO{verify that we indeed get a biadditive functor}

    In particular, we may speak of the \CrefAndHyperrefIfExist{definition:total_complexes_of_a_double_complex_of_objects_in_an_additive_category}{total complexes} $\Tot^{\oplus}(F(X_\bullet, Y_\bullet))$ and $\Tot^{\prod}(F(X_\bullet, Y_\bullet))$, and these specify biadditive functors
    $$\Tot^{\oplus}(F(-,-)), \Tot^{\prod}(F(-,-)): \mathbf{Ch}(\calA) \times \mathbf{Ch}(\calB) \to \mathbf{Ch}(\calC).$$

\end{definition}

\subsubsection{Acyclic assembly lemma}

\begin{lemma} \label{lemma:total_complexes_of_double_complex_with_bounded_diagonals_are_canonically_isomorphic}
    Let $C$ be a \CrefAndHyperrefIfExist{definition:double_complex_of_objects_in_an_additive_category}{double complex} of objects in an \CrefAndHyperrefIfExist{definition:additive_category}{additive category} $\calA$. If $C$ is \CrefAndHyperrefIfExist{definition:bounded_double_complex_of_objects_in_an_additive_category}{locally bounded along diagonals},
    % for each $n$, the total number of pairs $(p,q)$ such that $p+q = n$ and $C^{p,q} \neq 0$ is finite, 
    then the complexes \CrefAndHyperrefIfExist{definition:total_complexes_of_a_double_complex_of_objects_in_an_additive_category}{$\Tot^{\bigoplus}(A)$ and $\Tot^{\prod}(A)$} are naturally isomorphic.
\end{lemma}

\begin{proof}
    Since $C$ is locally bounded along diagonals, the degreee $n$ components
    \begin{align*}
    \left(\Tot^{\oplus}\right)^n(A) &= \bigoplus_{p+q=n} A^{p,q}, \\
    \left(\Tot^{\Pi}\right)^n(A) &= \prod_{p+q=n} A^{p,q}.
    \end{align*}
    are finite direct sums and finite products respectively and hence are naturally isomorphic (\Cref{lemma:finite_products_and_finite_coproducts_coincide_in_preadditive_categories}). The differential maps of the two total complexes also naturally coincide.
\end{proof}
\begin{lemma}[cf. {\cite[Acyclic Assembly Lemma 2.7.3]{weibel}}] \label{lemma:acyclic_assembly_lemma_for_bounded_double_complexes_with_exact_rows_or_columns}
    \TODO{It may be the case that this is generalizable beyond first quadrant double complexes, but I don't have a slick way to show this. See the commented out code for the statements; also, it may be necessary to assume something like AB4$*$ for such statements}
    Let $\calA$ be an \CrefAndHyperrefIfExist{definition:abelian_category}{abelian category} for which (small) \CrefAndHyperrefIfExist{definition:projective_and_inductive_limits_in_categories}{filtered colimits} which exist are exact (e.g. which holds if $\calA$ satisfies \CrefAndHyperrefIfExist{definition:grothendiecks_additional_axioms_for_abelian_categories}{Ab5}). Let $C$ be a \CrefAndHyperrefIfExist{definition:double_complex_of_objects_in_an_additive_category}{double complex} in $\calA$.

    % Let $C$ be a \CrefAndHyperrefIfExist{definition:double_complex_of_objects_in_an_additive_category}{double complex} in an \CrefAndHyperrefIfExist{definition:abelian_category}{abelian category} $\calA$ for which (small) filtered colimits which exist are exact (e.g. which holds if $\calC$ satisfies \CrefAndHyperrefIfExist{definition:grothendiecks_additional_axioms_for_abelian_categories}{Ab5}). 

    If $C$ has exact columns or has exact rows and $C$ is a \CrefAndHyperrefIfExist{definition:bounded_double_complex_of_objects_in_an_additive_category}{bounded below or bounded above double complex}, then $\Tot^{\Pi}(C)$ is an \CrefAndHyperrefIfExist{definition:acyclic_complex_of_objects_in_an_abelian_category}{acyclic chain complex}.



    % \begin{enumerate}
    %     \item 
    %     %Assuming that $\calA$ has arbitrary (small) products (in particular, the \CrefAndHyperrefIfExist{definition:total_complexes_of_a_double_complex_of_objects_in_an_additive_category}{total complex $\Tot^{\Pi}(C)$} exists), and products are exact (e.g. by virtue of $\calA$ satisfying \CrefAndHyperrefIfExist{definition:grothendiecks_additional_axioms_for_abelian_categories}{Ab$4^*$}),
    %     \begin{enumerate}
    %         \item if $C$ has exact columns or has exact rows and $C$ is a \CrefAndHyperrefIfExist{definition:bounded_double_complex_of_objects_in_an_additive_category}{bounded below or bounded above double complex}, then $\Tot^{\Pi}(C)$ is an \CrefAndHyperrefIfExist{definition:acyclic_complex_of_objects_in_an_abelian_category}{acyclic chain complex}.
    % \TODO{Fix the statements below to correctly specify what kind of boundedness of the double complex is needed. }
    %         \item if $C$ has exact rows and $C$ is a \CrefAndHyperrefIfExist{definition:bounded_double_complex_of_objects_in_an_additive_category}{bounded double complex}, then $\Tot^{\Pi}(C)$ is an \CrefAndHyperrefIfExist{definition:acyclic_complex_of_objects_in_an_abelian_category}{acyclic chain complex}.
    %     \end{enumerate}

    %     \item 
    %     %Assuming that $\calA$ has arbitrary (small) coproducts (in particular, the \CrefAndHyperrefIfExist{definition:total_complexes_of_a_double_complex_of_objects_in_an_additive_category}{total complex $\Tot^{\oplus}(C)$} exists), and (not necessarily finite) coproducts are exact (e.g. by virtue of $\calA$ satisfying \CrefAndHyperrefIfExist{definition:grothendiecks_additional_axioms_for_abelian_categories}{Ab4})
    %     \begin{enumerate}
    %         \item if $C$ has exact rows and $C$ is a \CrefAndHyperrefIfExist{definition:bounded_double_complex_of_objects_in_an_additive_category}{bounded double complex}, then $\Tot^{\oplus}(C)$ is an \CrefAndHyperrefIfExist{definition:acyclic_complex_of_objects_in_an_abelian_category}{acyclic chain complex}.
    %         \item if $C$ has exact columns and $C$ is a \CrefAndHyperrefIfExist{definition:bounded_double_complex_of_objects_in_an_additive_category}{bounded double complex}, then $\Tot^{\oplus}(C)$ is an \CrefAndHyperrefIfExist{definition:acyclic_complex_of_objects_in_an_abelian_category}{acyclic chain complex}.
    %     \end{enumerate}
    % \end{enumerate}

    % \begin{enumerate}
    %     \item Assuming that $\calA$ has arbitrary (small) products (in particular, the \CrefAndHyperrefIfExist{definition:total_complexes_of_a_double_complex_of_objects_in_an_additive_category}{total complex $\Tot^{\Pi}(C)$} exists), and products are exact,
    %     \begin{enumerate}
    %         \item if $C$ has exact columns and every diagonal of $C$ is bounded on the lower right (e.g. $C$ is an upper half-plane complex), then $\Tot^{\Pi}(C)$ is an \CrefAndHyperrefIfExist{definition:acyclic_complex_of_objects_in_an_abelian_category}{acyclic chain complex}.
    %         \item if $C$ has exact rows and every diagonal of $C$ is bounded on the upper left (e.g. $C$ is an right half-plane complex), then $\Tot^{\Pi}(C)$ is an \CrefAndHyperrefIfExist{definition:acyclic_complex_of_objects_in_an_abelian_category}{acyclic chain complex}.
    %     \end{enumerate}

    %     \item Assuming that the \CrefAndHyperrefIfExist{definition:total_complexes_of_a_double_complex_of_objects_in_an_additive_category}{total complex $\Tot^{\oplus}(C)$} exists (which holds e.g. when $\calA$ has all finite direct sums), 
    %     \begin{enumerate}
    %         \item if $C$ has exact rows and every diagonal of $C$ is bounded on the lower right (e.g. $C$ is an upper half-plane complex), then $\Tot^{\oplus}(C)$ is an \CrefAndHyperrefIfExist{definition:acyclic_complex_of_objects_in_an_abelian_category}{acyclic chain complex}.
    %         \item if $C$ has exact columns and every diagonal of $C$ is bounded on the upper left (e.g. $C$ is an right half-plane complex), then $\Tot^{\oplus}(C)$ is an \CrefAndHyperrefIfExist{definition:acyclic_complex_of_objects_in_an_abelian_category}{acyclic chain complex}.
    %     \end{enumerate}
    % \end{enumerate}
    
\end{lemma}

\begin{proof}

    % Assume that $\calA$ has arbitrary (small) products, and that products are exact in $\calA$.
    We show that if $C$ has exact columns and 
    $C$ is bounded below, 
    % every diagonal of $C$ is bounded on the lower right,
    then $\Tot^{\Pi}(C)$ is an \CrefAndHyperrefIfExist{definition:acyclic_complex_of_objects_in_an_abelian_category}{acyclic chain complex}; it can then be argued symmetrically that if $C$ has exact columns and $C$ is bounded above, then $\Tot^{\Pi}(C)$ is acyclic. Moreover, the case of exact rows can be deduced by reflecting the rows and columns of double complexes. 

    Note that since $C$ is assumed to be bounded below and hence is \CrefAndHyperrefIfExist{definition:bounded_double_complex_of_objects_in_an_additive_category}{locally bounded along diagonals}, $\Tot^{\Pi}(C)$ and $\Tot^{\oplus}(C)$ exist, are constructed by finite products (which are also finite coproducts), and are naturally isomorphic by Lemma \ref{lemma:total_complexes_of_double_complex_with_bounded_diagonals_are_canonically_isomorphic}. Further recall that finite coproducts in an abelian category are exact. 

    Define the sub-double complexes $F^k C$ of $C$ by 
    \begin{align*}
    (F^k C)^{p,q} = \begin{cases} C^{p,q} &\text{if } p \leq k \\ 0 &\text{otherwise} \end{cases}.
    \end{align*}
    This yields a filtration 
    $$\cdots \subseteq F^{k-1} C \subseteq F^k C \subseteq F^{k+1} \subseteq \cdots \subseteq C.$$
    Moreover, for each $n$,
    $$\left( \Tot^\Pi(F^k C) \right)^n = \prod_{p \leq k, p+q = n} C^{p,q}.$$
    For each $n$, the above stabilizes as $k \to \infty$ to $\Tot^{\Pi}(C)^n$. Now let 
    $$D^k = F^k C / F^{k-1} C = \begin{cases} C^{p,q} &\text{if } p = k \\ 0 &\text{otherwise}.\end{cases}$$
    Since each column $C^{k,*}$ is exact by assumption, the total complex $\Tot^{\Pi}(D^k)$ is acyclic. Note that we have short exact sequences
    $$0 \to F^{k-1} C \to F^k C \to D^k \to 0$$
    of double complexes. The totalization functor $\Tot^{\Pi}(-)$ in this case is exact because all of the double complexes are \CrefAndHyperrefIfExist{definition:bounded_double_complex_of_objects_in_an_additive_category}{locally bounded along diagonals}.
    % $\Tot^{\Pi}(-): \mathrm{Ch}(\mathrm{Ch}(\calA)) \to \mathrm{Ch}(\calA)$ (cf. \Cref{theorem:category_of_double_complexes_of_objects_of_an_additive_category_is_naturally_isomorphic_to_the_category_of_chain_complexes_of_chain_complexes}) is exact 
    We hence have a short exact sequence
    $$0 \to \Tot^{\Pi}(F^{k-1} C) \to \Tot^{\Pi}(F^k C) \to \Tot^{\Pi}(D^k) \to 0.$$
    Since $\Tot^{\Pi}(D^k)$ is \CrefAndHyperrefIfExist{definition:acyclic_complex_of_objects_in_an_abelian_category}{acyclic}, the \CrefAndHyperrefIfExist{theorem:long_exact_sequence_of_left_right_derived_functors_and_homological_delta_functors}{long exact cohomology sequences} yield isomorphisms
    $$H^n(\Tot^{\Pi}(F^{k-1} C)) \cong H^n(\Tot^{\Pi}(F^{k} C)).$$
    Since $C$ is assumed to be bounded, the subcomplex $F^k C$ is zero for sufficiently negative $k$, in which case $\Tot^{\Pi}(F^k C)$ is acyclic. By induction on $k$, $\Tot^{\Pi}(F^k C)$ remains acyclic for all $k$. Moroever, the filtered colimit $\varinjlim_k \Tot^{\Pi}(F^k C)$ is $\Tot^{\Pi}(C)$. The assumed exactness of filtered colimits in $\calA$ concludes that $\Tot^{\Pi}(C)$ is acyclic.

    By symmetry, if $C$ instead has exact rows, then $\Tot^{\Pi}(C)$ is an acyclic chain complex.
\end{proof}

\begin{lemma} \label{lemma:applying_biadd_func_to_res_is_defined_up_to_quasi_iso_and_agrees_with_the_tot_complex_of_the_double_complex_of_the_biadd_func_on_the_two_res}
    Let $F: \calA \times \calB \to \calC$ be a \CrefAndHyperrefIfExist{definition:n_ary_additive_functor_between_additive_categories}{biadditive functor} of \CrefAndHyperrefIfExist{definition:abelian_category}{abelian categories}. Assume that (small) filtered colimits which exist in $\calC$ are exact (e.g. which holds if $\calC$ satisfies \CrefAndHyperrefIfExist{definition:grothendiecks_additional_axioms_for_abelian_categories}{Ab5}).

    Let $A \in \calA$ and $B \in \calB$ be objects. 
    \begin{enumerate}
        \item Suppose that \CrefAndHyperrefIfExist{definition:left_right_resolution_of_a_class_of_objects_in_an_abelian_category}{left resolutions} $P_{A,\bullet} \to A$ and $P_{B,\bullet} \to B$ exist such that $P_{A,i}$ and $P_{B,i}$ are \CrefAndHyperrefIfExist{definition:flat_object_in_an_abelian_category_with_respect_to_a_right_exact_monoidal_product_functor}{flat} with respect to $F$ on the left and right respectively, i.e. $F(P_{A,i}, -): \calB \to \calC$ and $F(-, P_{B,i}): \calA \to \calC$ are exact for all $i$. 

        The complexes $F(P_{A,\bullet}, B)$ and $F(A, P_{B,\bullet})$ are \CrefAndHyperrefIfExist{definition:quasi_isomorphism_of_chain_complexes_of_objects_in_an_abelian_category}{quasi-isomorphic} to the complex $\Tot(F(P_{A,\bullet}, P_{B,\bullet}))$\CrefIfExists{definition:total_complexes_of_a_double_complex_of_objects_in_an_additive_category}\CrefIfExists{definition:double_complex_associated_to_biadditive_functor_and_chain_complexes}.

        \item Suppose that \CrefAndHyperrefIfExist{definition:left_right_resolution_of_a_class_of_objects_in_an_abelian_category}{right resolutions} $A \to I^{A,\bullet}$ and $B \to I^{B,\bullet}$ exist such that $I^{A,i}$ and $I^{B,i}$ are \CrefAndHyperrefIfExist{definition:flat_object_in_an_abelian_category_with_respect_to_a_right_exact_monoidal_product_functor}{flat} with respect to $F$ on the left and right respectively, $F(I^{A,i}, -): \calB \to \calC$ and $F(-, I^{B,i}): \calA \to \calC$ are exact for all $i$. 

        The complexes $F(I^{A,\bullet}, B)$ and $F(A, I^{B,\bullet})$ are \CrefAndHyperrefIfExist{definition:quasi_isomorphism_of_chain_complexes_of_objects_in_an_abelian_category}{quasi-isomorphic} to the complex $\Tot(F(I^{A,\bullet}, I^{B,\bullet}))$\CrefIfExists{definition:total_complexes_of_a_double_complex_of_objects_in_an_additive_category}\CrefIfExists{definition:double_complex_associated_to_biadditive_functor_and_chain_complexes}.
    \end{enumerate}
\end{lemma}

\begin{proof}
    We prove 1. The other part is the dual statement.

    Choose resolutions $P_{A,\bullet} \xrightarrow{\varepsilon} A$ and $P_{B,\bullet} \xrightarrow{\eta} B$ such that $F(P_{A,i}, -): \calB \to \calC$ and $F(-, P_{B,i}): \calA \to \calC$ are exact for all $i$. 
    Identifying $A$ and $B$ with complexes concentrated in degree $0$, we can \CrefAndHyperrefIfExist{definition:double_complex_associated_to_biadditive_functor_and_chain_complexes}{form} the three \CrefAndHyperrefIfExist{definition:double_complex_of_objects_in_an_additive_category}{double complexes} $F(P_{A,\bullet}, P_{B,\bullet})$, $F(A, P_{B,\bullet})$, and $F(P_{A,\bullet} , B)$. Note that the augmentation morphisms $\varepsilon$ and $\eta$ induce morphisms $P_{A,\bullet} \otimes P_{B,\bullet} \to A \otimes P_{B,\bullet}, P_{A,\bullet} \otimes B$.

    Let $C$ be the double complex of objects in $\calC$ obtained from $F(P_{A,\bullet}, P_{B,\bullet})$ by adding $F(A, P_{B,\bullet}[-1])$ in the column $p = -1$. One can show that the translate $\Tot(C)[1]$ is the \CrefAndHyperrefIfExist{definition:mapping_cone_of_a_map_of_chain_cochain_complexes}{mapping cone} of the map 
    $$\Tot(F(P_{A,\bullet}, P_{B,\bullet})) \xrightarrow{\varepsilon \otimes \id} \Tot(F(A, P_{B,\bullet})) = F(A, P_{B,\bullet}).$$
    Moreover, since each $F(-, P_{B,i})$ is an exact functor, every row of $C$ is exact, so $\Tot(C)$ is exact by \Cref{lemma:acyclic_assembly_lemma_for_bounded_double_complexes_with_exact_rows_or_columns}. Therefore, $F(\varepsilon, \id)$ is a quasi-isomorphism and hence 
    $$H_*(\Tot(F(P_{A,\bullet}, P_{B,\bullet}))) \xrightarrow{H_*(F(\varepsilon, P_{B,\bullet}))} H_*(F(A, P_{B,\bullet}))$$
    is a natural isomorphism. 

    By symmetry, there is a natural isomorphism $H_*(\Tot(F(P_{A,\bullet} P_{B,\bullet}))) \xrightarrow{} H_*(F(P_{A,\bullet}, B))$.
\end{proof}

\begin{theorem}[Balancing generalized derived functors of a biadditive functor of abelian categories computed via flat resolutions] \label{theorem:balancing_generalized_derived_functors_of_a_biadditive_functor_of_abelian_categories_computed_via_flat_resolutions}
    Let $F: \calA \times \calB \to \calC$ be a \CrefAndHyperrefIfExist{definition:n_ary_additive_functor_between_additive_categories}{biadditive functor} of \CrefAndHyperrefIfExist{definition:abelian_category}{abelian categories}. Assume that (small) filtered colimits which exist in $\calC$ are exact (e.g. which holds if $\calC$ satisfies \CrefAndHyperrefIfExist{definition:grothendiecks_additional_axioms_for_abelian_categories}{Ab5}).

    Let $A \in \calA$ and $B \in \calB$ be objects. 
    \begin{enumerate}
        \item Suppose that \CrefAndHyperrefIfExist{definition:left_right_resolution_of_a_class_of_objects_in_an_abelian_category}{left resolutions} $P_{A,\bullet} \to A$ and $P_{B,\bullet} \to B$ exist such that $F(P_{A,i}, -): \calB \to \calC$ and $F(-, P_{B,i}): \calA \to \calC$ are exact for all $i$, i.e. $P_{A,\bullet}$ and $P_{B,\bullet}$ are \CrefAndHyperrefIfExist{definition:flat_resolution_of_an_object_in_an_abelian_category_with_respect_to_a_right_exact_monoidal_product_functor}{flat resolutions} of $A$ and $B$ respectively. 
        \begin{enumerate}
            \item The objects \CrefAndHyperrefIfExist{definition:generalized_derived_functors_for_biadditive_functors_of_abelian_categories_via_resolutions_by_flat_objects}{$L_n^{I}F(A,B)$} and $L_n^{II}F(A,B)$ are naturally isomorphic.
            \item The objects $L_n^{I}F(A,B)$ and $L_n^{II}F(A,B)$ are well defined (up to natural isomorphism), i.e. do not depend on the choice of left resolutions of $A$ and $B$ respectively.
        \end{enumerate}

        \item Suppose that \CrefAndHyperrefIfExist{definition:left_right_resolution_of_a_class_of_objects_in_an_abelian_category}{right resolutions} $A \to I^{A,\bullet}$ and $B \to I^{B,\bullet}$ exist such that $F(I^{A,i}, -): \calB \to \calC$ and $F(-, I^{B,i}): \calA \to \calC$ are exact for all $i$. 
        \begin{enumerate}
            \item The objects \CrefAndHyperrefIfExist{definition:generalized_derived_functors_for_biadditive_functors_of_abelian_categories_via_resolutions_by_flat_objects}{$R^n_{I}F(A,B)$} and $R^n_{II}F(A,B)$ are naturally isomorphic.
            \item The objects $R^n_{I}F(A,B)$ and $R^n_{II}F(A,B)$ are well defined (up to natural isomorphism), i.e. do not depend on the choice of left resolutions of $A$ and $B$ respectively.
        \end{enumerate}
    \end{enumerate}
\end{theorem}

\begin{proof}

    We prove 1. The other part is the dual statement.

    Choose resolutions $P_{A,\bullet} \xrightarrow{\varepsilon} A$ and $P_{B,\bullet} \xrightarrow{\eta} B$ such that $F(P_{A,i}, -): \calB \to \calC$ and $F(-, P_{B,i}): \calA \to \calC$ are exact for all $i$. As per \Cref{lemma:applying_biadd_func_to_res_is_defined_up_to_quasi_iso_and_agrees_with_the_tot_complex_of_the_double_complex_of_the_biadd_func_on_the_two_res}, $F(\varepsilon, \id)$ is a quasi-isomorphism and hence $H_*(\Tot(F(P_{A,\bullet}, P_{B,\bullet}))) \xrightarrow{H_*(F(\varepsilon, P_{B,\bullet}))} H_*(F(A, P_{B,\bullet}))$ is a natural isomorphism.  By symmetry, there is a natural isomorphism $H_*(\Tot(F(P_{A,\bullet} P_{B,\bullet}))) \xrightarrow{} H_*(F(P_{A,\bullet}, B))$. Therefore, $L_n^I(A,B)$ and $L_n^{II}(A,B)$ are naturally isomorphic as claimed. In particular, $L_n^{I}(A,B)$ and $L_n^{II}(A,B)$ are independent of the choice of resolution of $A$ and $B$ respectively.

\end{proof}
\begin{theorem}[Balancing of Tor, cf. {\cite[Theorem 2.7.2]{weibel}}] \label{theorem:balancing_of_tor_on_abelian_categories_with_right_exact_bifunctor}
    Let $\mathcal{A}, \calB, \calC$ be \CrefAndHyperrefIfExist{definition:abelian_category}{abelian categories}, and let $\otimes : \mathcal{A} \times \mathcal{B} \to \mathcal{C}$ be a \CrefAndHyperrefIfExist{definition:n_ary_additive_functor_between_additive_categories}{biadditive functor}
    that is \CrefAndHyperrefIfExist{definition:exact_functor_between_abelian_categories}{right exact} in each variable. 
    Assume that (small) filtered colimits which exist in $\calC$ are exact (e.g. which holds if $\calC$ satisfies \CrefAndHyperrefIfExist{definition:grothendiecks_additional_axioms_for_abelian_categories}{Ab5}).

    Given $A \in \calA$ and $B \in \calB$ for which \CrefAndHyperrefIfExist{definition:flat_resolution_of_an_object_in_an_abelian_category_with_respect_to_a_right_exact_monoidal_product_functor}{flat resolutions exist}, let $\Tor_n^{I}(A,B)$ and $\Tor_n^{II}(A,B)$ respectively be the \CrefAndHyperrefIfExist{definition:tor_functors_of_a_right_exact_monoidal_functor_on_an_abelian_category}{Tor objects $\mathrm{Tor}_n^{\calA}(A,B)$}\CrefIfExists{definition:generalized_derived_functors_for_biadditive_functors_of_abelian_categories_via_resolutions_by_flat_objects} computed via \CrefAndHyperrefIfExist{definition:flat_resolution_of_an_object_in_an_abelian_category_with_respect_to_a_right_exact_monoidal_product_functor}{flat resolutions} of $A$ in $\calA$ and of $B$ in $\calB$.
    \begin{enumerate}
        \item $\Tor_n^{I}(A,B)$ and $\Tor_n^{II}(A,B)$ are naturally isomorphic.
        \item $\Tor_n^{I}(A,B)$ and $\Tor_n^{II}(A,B)$ are independent of the choice of flat resolution of $A$ and $B$ respectively.
    \end{enumerate}
    In particular, we may identify the objects $\Tor_n^{I}(A,B)$ and $\Tor_n^{II}(A,B)$ and simply write $\Tor_n(A,B)$ for either.
\end{theorem}

\begin{proof}
    This follows from \Cref{theorem:balancing_generalized_derived_functors_of_a_biadditive_functor_of_abelian_categories_computed_via_flat_resolutions}.
    % Choose flat resolutions $P_\bullet \xrightarrow{\varepsilon} A$ and $Q_\bullet \xrightarrow{\eta} B$. Identifying $A$ and $B$ with complexes concentrated in degree $0$, we can \CrefAndHyperrefIfExist{definition:double_complex_associated_to_biadditive_functor_and_chain_complexes}{form} the three tensor product \CrefAndHyperrefIfExist{definition:double_complex_of_objects_in_an_additive_category}{double complexes} $P_\bullet \otimes Q_\bullet$, $A \otimes Q_\bullet$, and $P_\bullet \otimes B$. Note that the augmentation morphisms $\varepsilon$ and $\eta$ induced morphisms $P_\bullet \otimes Q_\bullet \to A \otimes Q_\bullet, P_\bullet \otimes B$.

    % Let $C$ be the double complex of objects in $\calC$ obtained from $P_\bullet \otimes Q_\bullet$ by adding $A \otimes Q_{\bullet}[-1]$ in the column $p = -1$. The translate $\Tot(C)[1]$ is the mapping cone of the map 
    % $$\Tot(P_\bullet \otimes Q_\bullet) \xrightarrow{\varepsilon \otimes Q} \Tot(A \otimes Q_\bullet) = A \otimes Q_\bullet.$$
    % \TODO{why is the translate the mapping cone of $\varepsilon \otimes Q$?} Moreover, since each $- \otimes Q_q$ is an exact functor, every row of $C$ is exact, so $\Tot(C)$ is exact by \Cref{lemma:acyclic_assembly_lemma_for_bounded_double_complexes_with_exact_rows_or_columns}. Therefore, $\varepsilon \otimes Q$ is a quasi-isomorphism and hence $H_*(\Tot(P_\bullet \otimes Q_\bullet)) \xrightarrow{H_*(\varepsilon \otimes Q_\bullet)} H_*(A \otimes Q_\bullet)$ is a natural isomorphism. 

    % By symmetry, there is a natural isomorphism $H_*(\Tot(P_\bullet \otimes Q_\bullet)) \xrightarrow{} H_*(P_\bullet \otimes B)$. Therefore, $\Tor_n^{I}(A,B)$ and $\Tor_n^{II}(A,B)$ are naturally isomorphic as claimed. In particular, $\Tor_n^{I}(A,B)$ and $\Tor_n^{II}(A,B)$ are independent of the choice of flat resolution of $A$ and $B$ respectively.
\end{proof}


\TODO{state the balancing theorems of Ext and Tor explicitly}
 


\section{Ext and Tor}




\section{Sheaves on topological spaces}

Sheaves (say on a topological space) of abelian groups form an abelian category with a rich theory.


\subsubsection{Presheaves and sheaves on topological spaces}

\begin{definition}[Presheaf on a category] \label{definition:presheaf_on_a_category}
    Let $C$ and $\mathcal{A}$ be \hyperrefIfExists{definition:category}{(large) categories}\CrefIfExists{definition:category}. 
    \begin{enumerate}
        \item A \hldef{presheaf $\mathcal{F}$ on $C$ with values in $\mathcal{A}$} is a functor
        \[
        \mathcal{F}: C^{\mathrm{op}} \to \mathcal{A}.
        \]
        In other words, a presheaf $\calF$ on $C$ with values in $\calA$ is simply a \CrefAndHyperrefIfExist{definition:functor_between_categories}{contravariant functor} from $C$ to $\calA$. 
        Explicitly, for every object $U$ in $C$, one has an object $\mathcal{F}(U)$ in $\mathcal{A}$ (called the \hldef{$U$-valued sections/sections evaluated at $U$ of $\calF$}\TextIfExists{definition:sections_of_a_presheaf_on_a_category_valued_in_a_category}{, cf. \Cref{definition:sections_of_a_presheaf_on_a_category_valued_in_a_category}}), and for every morphism $f: V \to U$ in $C$, one has a morphism (called the \hldef{restriction map})
        \[
        \mathcal{F}(f): \mathcal{F}(U) \to \mathcal{F}(V)
        \]
        in $\mathcal{A}$, such that for all composable morphisms $W \xrightarrow{g} V \xrightarrow{f} U$ in $C$, the following diagram in $\mathcal{A}$ commutes:
        \[
        \begin{tikzcd}
        \mathcal{F}(U) \arrow[r, "\mathcal{F}(f)"] \arrow[rr, bend left, "\mathcal{F}(f \circ g)"] & \mathcal{F}(V) \arrow[r, "\mathcal{F}(g)"] & \mathcal{F}(W)
        \end{tikzcd}
        \]
        That is,
        \[
        \mathcal{F}(g) \circ \mathcal{F}(f) = \mathcal{F}(f \circ g),
        \]
        and for every object $U$ in $C$, $\mathcal{F}(\mathrm{id}_U) = \mathrm{id}_{\mathcal{F}(U)}$.


        \item 
        Let $\mathcal{F},\mathcal{G}: C^{\mathrm{op}} \to \mathcal{A}$ be two presheaves on $C$ with values in $\mathcal{A}$. A \hldef{morphism of presheaves}
        \[
        \varphi: \mathcal{F} \to \mathcal{G}
        \]
        is a \hyperrefIfExists{definition:natural_transformation_between_functors_between_categories}{natural transformation of functors}\CrefIfExists{definition:natural_transformation_between_functors_between_categories}: for each object $U$ of $C$, one has a morphism
        \[
        \varphi_U: \mathcal{F}(U) \to \mathcal{G}(U)
        \]
        in $\mathcal{A}$, such that for every morphism $f: V \to U$ in $C$, the diagram
        \[
        \begin{tikzcd}
        \mathcal{F}(U) \arrow[r, "\mathcal{F}(f)"] \arrow[d, "\varphi_U"'] & \mathcal{F}(V) \arrow[d, "\varphi_V"] \\
        \mathcal{G}(U) \arrow[r, "\mathcal{G}(f)"'] & \mathcal{G}(V)
        \end{tikzcd}
        \]
        commutes, i.e.,
        \[
        \varphi_V \circ \mathcal{F}(f) = \mathcal{G}(f) \circ \varphi_U
        \]
        for all objects and morphisms in $C$.

        \item Given a \hyperrefIfExists{definition:grothendieck_universe}{universe}\CrefIfExists{definition:grothendieck_universe} $U$, a \hldef{$U$-presheaf on $\calC$} typically refers to a presheaf of $U$-sets on $C$.

        \item The \hldef{presheaf category/category of $\calA$-valued presheaves on $\calC$} is the (large) category whose objects are the presheaves on $C$ with values in $\calA$ and whose morphisms are the presheaf morphisms. Common notations for the presheaf category include, but are not limited to: \hl{$\calA^{\calC^{\op}}$}, \hl{$\PreShv(\calC, \calA)$}, \hl{$[\calC^{\op}, \calA]$}. If the value category $\calA$ is clear from context, then notations such as \hl{$\PreShv(\calC)$} are also common. \TextIfExists{definition:diagram_in_a_category_indexed_by_a_small_category}{Note that the presheaf category $\PreShv(\calC, \calA)$ is equivalent to the \CrefAndHyperrefIfExist{definition:diagram_in_a_category_indexed_by_a_small_category}{category of functors} $\calC^{\op} \to \calA$ and hence notations for the functor categories are applicable as notations for presheaf categories.}

    \end{enumerate}
\end{definition}


\begin{definition}[Category of opens of a topological space] \label{definition:category_of_opens_of_a_topological_space}
    Let $X$ be a \CrefAndHyperrefIfExist{definition:topological_space}{topological space}. The \hldef{category of opens of $X$}, sometimes denoted \hl{$\mathbf{Open}(X)$} (or \hl{$\mathrm{Open}(X)$} or \hl{$\mathrm{Ouv}(X)$} (for the French word ``ouvert'', meaning open), etc.), is the \CrefAndHyperrefIfExist{definition:locally_small_category}{small} \CrefAndHyperrefIfExist{definition:category}{category} defined as follows:
    \begin{itemize}
        \item The objects are the open subsets $U \subseteq X$.
        \item For two open sets $U, V \subseteq X$, the morphism set is
        $$\mathrm{Hom}_{\mathbf{Open}(X)}(U,V) =
        \begin{cases}
        \{\iota_{U,V}\}, & \text{if } U \subseteq V, \\
        \varnothing, & \text{otherwise},
        \end{cases}$$
        where $\iota_{U,V}$ denotes the inclusion morphism $U \hookrightarrow V$.
        \item Composition of morphisms is given by composition of set-theoretic inclusions, i.e.
        $$\iota_{V,W} \circ \iota_{U,V} = \iota_{U,W} \quad \text{whenever } U \subseteq V \subseteq W.$$
        \item The identity morphism on an object $U$ is the inclusion $\iota_{U,U} = \mathrm{id}_U$.
    \end{itemize}
\end{definition}

\begin{definition} \label{definition:continuous_functor_between_sites_of_opens_on_topological_spaces_induced_by_continuous_map}
    Let $(X, \tau_X)$ and $(Y, \tau_Y)$ be \CrefAndHyperrefIfExist{definition:topological_space}{topological spaces}, and let $f : X \to Y$ be a \CrefAndHyperrefIfExist{definition:continuous_map_of_topological_spaces}{continuous map}.
    Let $\operatorname{Open}(X)$ and $\operatorname{Open}(Y)$ be their respective \CrefAndHyperrefIfExist{definition:category_of_opens_of_a_topological_space}{categories of open sets} with inclusion morphisms, equipped with the \CrefAndHyperrefIfExist{definition:site_of_opens_on_a_topological_space}{canonical} \CrefAndHyperrefIfExist{definition:grothendieck_topology_on_a_category_site_covering_sieve_topologically_generating_family}{Grothendieck topologies} given by open coverings.

    Define the functor 
    $$\hlin{f^{-1} : \operatorname{Open}(Y) \to \operatorname{Open}(X), \quad U \mapsto f^{-1}(U).}$$
    It is a \CrefAndHyperrefIfExist{definition:continuous_functor_of_sites}{continuous functor of sites} from $\operatorname{Open}(Y)$ to $\operatorname{Open}(Y)$ which induces a \CrefAndHyperrefIfExist{definition:morphism_of_sites}{site morphism} 
    $$f: (\operatorname{Open}(X), \text{can}) \to (\operatorname{Open}(Y), \text{can})$$.
\end{definition}

\begin{definition}[Presheaf on a topological space] \label{definition:presheaf_on_a_topological_space}
    Let $X$ be a \CrefAndHyperrefIfExist{definition:topological_space}{topological space}. Let $\calD$ be a category.
    
    A \hldef{presheaf (of objects of $\calD$/valued in $\calD$) on $X$} is a rule $\mathcal{F}$ that assigns:
    \begin{itemize}
        \item to each open set $U \subseteq X$, an object $\mathcal{F}(U) \in \Ob \calD$, called the \hldef{sections of $\mathcal{F}$ over $U$},
        \item to each inclusion of open sets $V \subseteq U$, a morphism
        $$\rho^U_V: \mathcal{F}(U) \to \mathcal{F}(V), \quad s \mapsto s|_V,$$
    \end{itemize}
        in the category $\calD$ called the \hldef{restriction map} such that the following conditions hold:
    \begin{itemize}
        \item (Identity) For every open set $U \subseteq X$, the restriction map $\rho^U_U$ is the identity on $\mathcal{F}(U)$.
        \item (Transitivity) For inclusions $W \subseteq V \subseteq U$ of open sets, one has
        $$\rho^U_W = \rho^V_W \circ \rho^U_V.$$
    \end{itemize}
    For instance, we may speak of a \hldef{presheaf of sets/groups/rings/etc. on the topological space $X$}.


    Equivalently, a presheaf on $X$ (of objects in a category $\calD$) is a \CrefAndHyperrefIfExist{definition:functor_between_categories}{functor}
    $$\mathbf{Open}(X)^{\op} \to \calD$$
    from the opposite of the category \CrefAndHyperrefIfExist{definition:category_of_opens_of_a_topological_space}{$\mathbf{Open}(X)$} of open subsets of $X$\TextIfExists{definition:presheaf_on_a_category}{ (see also \Cref{definition:presheaf_on_a_category})}.

    \TextIfExists{definition:presheaf_on_a_category}{Equivalently, a presheaf on $X$ is a presheaf on the category $\mathbf{Open}(X)$ in the sense of \Cref{definition:presheaf_on_a_category}}.


    The sections object $\calF(U)$ is also denoted by \hl{$\Gamma(U, \calF)$}\TextIfExists{definition:sections_of_a_presheaf_on_a_category_valued_in_a_category}{ (see \Cref{definition:sections_of_a_presheaf_on_a_category_valued_in_a_category})}.
    Moreover, the object $\calF(X) = \Gamma(X, \calF)$ is called the \hldef{global sections object of $\calF$}. \TextIfExists{definition:sections_of_a_presheaf_on_a_category_valued_in_a_category}{This agrees with the notion of global sections as discussed in \Cref{definition:sections_of_a_presheaf_on_a_category_valued_in_a_category}.}

\end{definition}

\begin{definition}[Sheaf on a topological space] \label{definition:sheaf_on_a_topological_space_valued_in_a_category_with_a_terminal_object}

    Let $X$ be a \CrefAndHyperrefIfExist{definition:topological_space}{topological space}, let $\calD$ be a \CrefAndHyperrefIfExist{definition:category}{category} with a \CrefAndHyperrefIfExist{lemma:initial_or_final_object_in_a_category_that_is_also_in_a_full_subcategory_is_initial_or_final_in_the_subcategory}{terminal object}, and let $\mathcal{F}$ be a \CrefAndHyperrefIfExist{definition:presheaf_on_a_topological_space}{presheaf valued in $\calD$ on $X$}.  
    Then $\mathcal{F}$ is a \hldef{sheaf} if it satisfies the following additional condition (known as the \hldef{sheaf axioms}):

    For every open set $U \subseteq X$ and every \CrefAndHyperrefIfExist{definition:open_covering_of_a_topological_space}{open cover} $\{U_i\}_{i \in I}$ of $U$, let $\calJ$ be the \CrefAndHyperrefIfExist{definition:diagram_in_a_category_indexed_by_a_small_category}{diagram} in the \CrefAndHyperrefIfExist{definition:category_of_opens_of_a_topological_space}{category of opens} of $U$ consisting of the inclusions $U_i \cap U_j \hookrightarrow U_i$ for all $i,j \in I$. Then $\calF$ is a sheaf if the \CrefAndHyperrefIfExist{definition:limit_and_colimit_of_a_diagram_in_a_category}{limit} of the diagram $\calF \circ \calJ$ exists in $\calD$ and the natural morphism
    $$\calF(U) \to \lim_{j \in \calJ} \calF(j)$$
    is an isomorphism. More precisely, $\calJ:J \to \operatorname{Open}(U)$ should be the functor whose index category $J$ consists of
    \begin{enumerate}
        \item An object $i$ for every $i \in I$ and an object $(i,j)$ for every pair $i,j \in I$,
        \item Morphisms $p_1: (i,j) \to i$ and $p_2: (i,j) \to j$ for every pair $i,j \in I$
    \end{enumerate}
    and which sends the objects and morphisms as follows:
    \begin{enumerate}
        \item $\calJ(i) = U_i$
        \item $\calJ(i,j) = U_i \cap U_j$
        \item $\calJ(p_1): U_i \cap U_j \hookrightarrow U_i$
        \item $\calJ(p_2): U_i \cap U_j \hookrightarrow U_j$.
    \end{enumerate}
    In particular, taking $U = \emptyset$ and taking the empty open cover of the empty set, $\calF(\emptyset)$ must be the \CrefAndHyperrefIfExist{lemma:initial_or_final_object_in_a_category_that_is_also_in_a_full_subcategory_is_initial_or_final_in_the_subcategory}{terminal object} of $\calD$

    In the case that $\calD$ admits all \CrefAndHyperrefIfExist{definition:small_and_finite_limits_and_colimits_in_a_category}{small limits}, the sheaf condition is equivalent to the following: For every open set $U \subset X$ and every open cover $\{U_i\}_{i \in I}$ of $U$, the following \CrefAndHyperrefIfExist{definition:equalizer_and_coequalizer_of_morphisms_in_a_category}{equalizer diagram is exact}:
    $$\calF(U) \to \prod_{i \in I} \calF(U_i) \rightrightarrows \prod_{i,j \in I} \calF(U_i \cap U_j).$$
    Here, the morphism $\calF(U) \to \prod_{i \in I} \calF(U_i)$ and the two morphisms $\prod_{i \in I} \calF(U_i) \rightrightarrows \prod_{i,j \in I} \calF(U_i \cap U_j)$ are induced by the \CrefAndHyperrefIfExist{definition:presheaf_on_a_topological_space}{restriction maps} $\calF(U) \to \calF(U_i)$ and $\calF(U_i) \to \calF(U_i \cap U_j)$.

    In the case that $\calD$ is some \CrefAndHyperrefIfExist{definition:subcategory_of_a_category}{subcategory} of the \CrefAndHyperrefIfExist{definition:category_of_sets}{category of sets}, the sheaf condition is equivalent to the following: For every open set $U \subseteq X$ and every \CrefAndHyperrefIfExist{definition:open_covering_of_a_topological_space}{open cover} $\{U_i\}_{i \in I}$ of $U$, 
    \begin{itemize}
        \item (Locality) If $s, t \in \mathcal{F}(U)$ are such that $s|_{U_i} = t|_{U_i}$ for all $i$, then $s = t$.
        \item (Gluing) If for each $i$ there is $s_i \in \mathcal{F}(U_i)$ such that for all $i,j$ one has $s_i|_{U_i \cap U_j} = s_j|_{U_i \cap U_j}$, then there exists a unique $s \in \mathcal{F}(U)$ such that $s|_{U_i} = s_i$ for all $i$.
    \end{itemize}

    \TextIfExists{definition:sheaf_on_a_site}{
        Equivalently, a sheaf on a topological space $X$ may be defined as a \CrefAndHyperrefIfExist{definition:sheaf_on_a_site}{sheaf on} the \CrefAndHyperrefIfExist{definition:grothendieck_topology_on_a_category_site_covering_sieve_topologically_generating_family}{site} \CrefAndHyperrefIfExist{definition:site_of_opens_on_a_topological_space}{of opens on $X$}. 
        % whose objects are open subsets $U$ of $X$ and whose morphisms are inclusions $U_1 \hookrightarrow U_2$ of open subsets of $X$, and whose Grothendieck topology is the in which a cover of an open subset $U \subseteq X$ is given by a collection $\{U_i \to U\}_{i \in I}$ in which $\bigcup U_i = U$. 
    }

\end{definition}


\TODO{examples of presheaves and sheaves on topological spaces}

Note that on a topological space $X$, it is natural to talk about open sets $\{U_i\}$ which cover an open set $U$ of $X$. One can express more general notions of ``coverings'' of objects in a category through the notion of a \CrefAndHyperrefIfExist{definition:grothendieck_topology_on_a_category_site_covering_sieve_topologically_generating_family}{``site''}, which consists of a category and a \CrefAndHyperrefIfExist{definition:grothendieck_topology_on_a_category_site_covering_sieve_topologically_generating_family}{``Grothendieck topology''} on the category. The notion of \CrefAndHyperrefIfExist{definition:sheaf_on_a_topological_space_valued_in_a_category_with_a_terminal_object}{``sheaf on a topological space''} above is a specialization of this more general notion.

\subsubsection{Sheafification of presheaves}


\begin{definition} \label{definition:categories_of_presheaves_and_sheaves_on_a_topological_space_valued_in_a_category}
    \TODO{Move these notations to the defintions of presheaves and sheaves on topological spaces}
    Let $X$ be a \CrefAndHyperrefIfExist{definition:topological_space}{topological space}, and let $\calD$ be a \CrefAndHyperrefIfExist{definition:category}{category} with a \CrefAndHyperrefIfExist{lemma:initial_or_final_object_in_a_category_that_is_also_in_a_full_subcategory_is_initial_or_final_in_the_subcategory}{terminal object}.

    The \CrefAndHyperrefIfExist{definition:presheaf_on_a_topological_space}{presheaves on $X$ valued in $\calD$}, along with the \CrefAndHyperrefIfExist{definition:morphism_of_presheaves_on_a_topological_space}{morphisms} thereof, form a (in general large) \CrefAndHyperrefIfExist{definition:category}{category} often denoted by notations such as \hl{$\PreShv(X, \calD)$} \TODO{include more notations} (or \hl{$\PreShv(X)$} if the category $\calD$ is clear). If $\calD$ is \CrefAndHyperrefIfExist{definition:locally_small_category}{locally small}, then so is $\PreShv(X, \calD)$.


    Similarly, the \CrefAndHyperrefIfExist{definition:sheaf_on_a_topological_space_valued_in_a_category_with_a_terminal_object}{sheaves on $X$ valued in $\calD$}, along with the \CrefAndHyperrefIfExist{definition:morphism_of_presheaves_on_a_topological_space}{morphisms} thereof, form a (in general large) \CrefAndHyperrefIfExist{definition:category}{category} often denoted by notations such as \hl{$\Shv(X, \calD)$} \TODO{include more notations} (or \hl{$\Shv(X)$} if the category $\calD$ is clear). The category $\Shv(X, \calD)$ is a \CrefAndHyperrefIfExist{definition:full_subcategory_of_a_category}{full subcategory} of $\PreShv(X, \calD)$.
    
    \TextIfExists{definition:sheaf_on_a_site}{
    Equivalently, the categories of presheaves and sheaves are the categories $\PreShv(\mathbf{Open}(X), \calD)$ and $\Shv(\mathbf{Open}(X), \calD)$ of \CrefAndHyperrefIfExist{definition:presheaf_on_a_category}{presheaves} and \CrefAndHyperrefIfExist{definition:sheaf_on_a_site}{sheaves} where \CrefAndHyperrefIfExist{definition:category_of_opens_of_a_topological_space}{$\mathbf{Open}(X)$} is the category of open subsets of $X$ equipped with its \CrefAndHyperrefIfExist{definition:site_of_opens_on_a_topological_space}{usual} \CrefAndHyperrefIfExist{definition:basis_and_grothendieck_pretopology_for_a_grothendieck_topology_on_a_category}{Grothendieck pretopology}.
    }

\end{definition}


\begin{definition}[Sheaf associated to a presheaf] \label{definition:sheafification_of_a_presheaf_on_a_topological_space_valued_in_a_category_admitting_direct_colimits}
    Let $X$ be a topological space, and let $\mathcal{D}$ be a \CrefAndHyperrefIfExist{definition:category}{category} admitting \CrefAndHyperrefIfExist{definition:projective_and_inductive_limits_in_categories}{direct colimits} (e.g. the category of sets, groups, abelian groups, modules over rings, or vector spaces over fields). Let $\mathcal{P}$ be a \CrefAndHyperrefIfExist{definition:presheaf_on_a_topological_space}{presheaf on $X$ with values in $\mathcal{D}$}.  

    The \hldef{sheaf associated to the presheaf $\mathcal{P}$} or the \hldef{sheaffification of the presheaf $\calP$}, denoted \hl{$\mathcal{P}^+$} or sometimes by \hl{$a\calP$}, is a sheaf on $X$ together with a morphism of presheaves
    $$\eta: \mathcal{P} \to \mathcal{P}^+,$$
    satisfying the following universal property:  
    for every sheaf $\mathcal{F}$ on $X$ (valued in $\mathcal{D}$), any morphism of presheaves
    $$\varphi: \mathcal{P} \to \mathcal{F}$$
    factors uniquely through $\eta$, i.e., there exists a unique morphism of sheaves
    $$\widetilde{\varphi}: \mathcal{P}^+ \to \mathcal{F}$$
    such that
    $$\varphi = \widetilde{\varphi} \circ \eta.$$

    Concretely, $\mathcal{P}^+$ can be constructed by assigning to each open set $U \subseteq X$ the set (or object in $\mathcal{D}$)
        \[
        \mathcal{P}^+(U) := \left\{ 
        s = (s_x)_{x \in U} \in \prod_{x \in U} \mathcal{P}_x \ 
        \middle|\ 
        \begin{aligned}
        &\forall x \in U, \\
        &\exists \text{ an open } V \subseteq U \text{ with } x \in V, \\
        &\exists t \in \mathcal{P}(V) \text{ such that } \\
        &\quad \forall y \in V, s_y = t_y
        \end{aligned}
        \right\}.
        \]
    where $\mathcal{P}_x$ is the \CrefAndHyperrefIfExist{definition:stalk_of_a_presheaf_on_a_topological_space_at_a_point}{stalk} of $\mathcal{P}$ at $x$, and $t_y$ is the \CrefAndHyperrefIfExist{definition:stalk_of_a_presheaf_on_a_topological_space_at_a_point}{germ} of $t$ at $y$. In particular, $\calP^+$ exists.

    It is noteworthy that the assignment $\calP \mapsto \calP^+$ is a functor
    $$\PreShv(X, \calD) \to \Shv(X, \calD).$$
    (\Cref{definition:categories_of_presheaves_and_sheaves_on_a_topological_space_valued_in_a_category}) and that this functor is left adjoint to the inclusion functor
    $$\Shv(X, \calD) \hookrightarrow \PreShv(X, \calD)$$
    \TextIfExists{definition:sheafification_functor_on_a_site}{Equivalently, the assignment $\calP \mapsto \calP^+$ is the sheafification functor as defined in \Cref{definition:sheafification_functor_on_a_site}.}
\end{definition}

% See Also
% theorem:sheafification_of_a_presheaf_of_sets_on_a_small_enough_site 

\subsubsection{Sheaf of rings, ringed spaces, and sheaves of modules}


\begin{definition}[Ringed space] \label{definition:ringed_space}
    A \hldef{ringed space} is a pair $(X, \mathcal{O}_X)$ where
    \begin{itemize}
        \item $X$ is a \CrefAndHyperrefIfExist{definition:topological_space}{topological space}, and
        \item $\mathcal{O}_X$ is a \CrefAndHyperrefIfExist{definition:sheaf_on_a_topological_space_valued_in_a_category_with_a_terminal_object}{sheaf of} \CrefAndHyperrefIfExist{definition:commutative_ring}{commutative rings} on $X$.
    \end{itemize}
    Equivalently, a ringed space is a \CrefAndHyperrefIfExist{definition:ringed_site}{ringed site} where the site is the \CrefAndHyperrefIfExist{definition:site_of_opens_on_a_topological_space}{site of opens} of the topological space $X$
    The sheaf $\calO_X$ may be suppressed from the notation and only $X$ may be used to denote a ringed space. The sheaf \hl{$\mathcal{O}_X$}, also commonly denoted by $\mathscr{O}_X$, is called the \hldef{structure sheaf of $X$}.
\end{definition}


\TODO{define module sheaf on a topological space}
% \begin{definition} \label{definition:module_over_a_sheaf_of_rings_on_a_site}

    \begin{enumerate}
        \item 
        Let $\mathcal{C}$ be a \CrefAndHyperrefIfExist{definition:grothendieck_topology_on_a_category_site_covering_sieve_topologically_generating_family}{site}, and let $\mathcal{A}$ and $\mathcal{B}$ be \CrefAndHyperrefIfExist{definition:sheaf_on_a_site}{sheaves} of (not necessarily commutative) \CrefAndHyperrefIfExist{definition:ring}{rings} on $\mathcal{C}$. 
        
        \begin{enumerate}
            \item 
            An \hldef{$(\mathcal{A}, \mathcal{B})$-bimodule} (or a \hldef{bimodule over $(\mathcal{A}, \mathcal{B})$}) is a \CrefAndHyperrefIfExist{definition:sheaf_on_a_site}{sheaf} $\mathcal{M}$ of abelian groups on $\mathcal{C}$ equipped with a left $\mathcal{A}$-module structure given by a \CrefAndHyperrefIfExist{definition:sheaf_on_a_site}{morphism of sheaves} of sets
            $$ \lambda: \mathcal{A} \times \mathcal{M} \longrightarrow \mathcal{M}, $$
            and a right $\mathcal{B}$-module structure given by a morphism of sheaves of sets
            $$ \rho: \mathcal{M} \times \mathcal{B} \longrightarrow \mathcal{M}, $$
            such that the actions are compatible. Specifically, for every object $U$ in $\mathcal{C}$, every section $m \in \mathcal{M}(U)$, every $a \in \mathcal{A}(U)$, and every $b \in \mathcal{B}(U)$, the equality
            $$ \lambda_U(a, \rho_U(m, b)) = \rho_U(\lambda_U(a, m), b) $$
            holds in $\mathcal{M}(U)$. In standard multiplicative notation where $\lambda(a,m)$ is denoted $a \cdot m$ and $\rho(m,b)$ is denoted $m \cdot b$, this condition is the associativity axiom
            $$ (a \cdot m) \cdot b = a \cdot (m \cdot b). $$

            In particular, for every object $U \in \calC$, the abelian group $\calM(U)$ has the structure of an \CrefAndHyperrefIfExist{definition:module_of_a_ring}{$\calA(U)-\calB(U)$-bimodule}.

            \item Let $\mathcal{M}$ and $\mathcal{N}$ be $(\mathcal{A}, \mathcal{B})$-bimodules. A \hldef{homomorphism of $(\mathcal{A}, \mathcal{B})$-bimodules} (or an \hldef{$(\mathcal{A}, \mathcal{B})$-linear morphism}) is a morphism of sheaves of abelian groups $f: \mathcal{M} \to \mathcal{N}$ such that for every object $U$ of $\mathcal{C}$, every section $m \in \mathcal{M}(U)$, every $a \in \mathcal{A}(U)$, and every $b \in \mathcal{B}(U)$, the following compatibility conditions hold:
            $$ f_U(a \cdot m) = a \cdot f_U(m) \quad \text{and} \quad f_U(m \cdot b) = f_U(m) \cdot b. $$


        \end{enumerate}

        \noindent We denote the category of $(\mathcal{A}, \mathcal{B})$-bimodules, with morphisms being morphisms of sheaves of abelian groups compatible with both the left $\mathcal{A}$-action and the right $\mathcal{B}$-action, by
        \hl{$ \mathcal{A}\text{-}\mathcal{B}\text{-}\mathsf{Mod} $}
        or sometimes by
        \hl{$ {}_{\mathcal{A}}\mathsf{Mod}_{\mathcal{B}} $}
        \TODO{talk about how bimodules can be identifies with left/right modules}

        \item 

        Let $(\mathcal{C}, J)$ be a \CrefAndHyperrefIfExist{definition:grothendieck_topology_on_a_category_site_covering_sieve_topologically_generating_family}{site}. Let $\mathcal{O}$ be a \CrefAndHyperrefIfExist{definition:sheaf_on_a_site}{sheaf of (not necessarily commutative) rings on $(\mathcal{C}, J)$}, i.e. $((\calC, J), \calO)$ is a \CrefAndHyperrefIfExist{definition:ringed_site}{ringed site}.  

        \begin{enumerate}
            \item An \hldef{(left/right/two-sided) $\mathcal{O}$-module} consists of the following data:
            \begin{itemize}
                \item A sheaf $\mathcal{F}$ of abelian groups on $(\mathcal{C}, J)$,
            \item for every object $U \in \mathcal{C}$, the structure of an (left/right/two-sided) $\mathcal{O}(U)$-module on $\mathcal{F}(U)$,
            \end{itemize}
            such that for every morphism $f: V \to U$ in $\mathcal{C}$, the restriction map 
            $$\rho_{U,V}: \mathcal{F}(U) \to \mathcal{F}(V)$$ 
            is $\mathcal{O}(U)$-linear when the $\mathcal{O}(U)$-action on $\mathcal{F}(V)$ is defined via the natural ring homomorphism 
            $$\mathcal{O}(U) \to \mathcal{O}(V)$$
            induced by $f$.


            \item Let $\mathcal{F}$ and $\mathcal{G}$ be \CrefAndHyperrefIfExist{definition:module_over_a_sheaf_of_rings_on_a_site}{$\mathcal{O}$-modules}.

            A \hldef{morphism of $\mathcal{O}$-modules} $\varphi: \mathcal{F} \to \mathcal{G}$ is a \CrefAndHyperrefIfExist{definition:sheaf_on_a_site}{morphism of sheaves} of abelian groups such that, for every object $U \in \mathcal{C}$, the component map
            $$\varphi_U : \mathcal{F}(U) \to \mathcal{G}(U)$$
            is $\mathcal{O}(U)$-linear, i.e. it satisfies
            $$\varphi_U(r \cdot s) = r \cdot \varphi_U(s) \quad \text{for all } r \in \mathcal{O}(U), \, s \in \mathcal{F}(U).$$

            The collection of all $\mathcal{O}$-modules together with their morphisms of $\mathcal{O}$-modules forms the \hldef{category of $\mathcal{O}$-modules}, denoted \hl{$\mathbf{Mod}(\mathcal{O})$}.

            \TextIfExists{definition:algebra_over_a_sheaf_of_rings_on_a_site}{See also \Cref{definition:algebra_over_a_sheaf_of_rings_on_a_site}.}
        \end{enumerate}

        \noindent In case that a \CrefAndHyperrefIfExist{definition:sheafification_functor_on_a_site}{sheafification functor} 
        $$\PreShv(\calC, \mathbf{Rings}) \to \Shv(\calC, \mathbf{Rings})$$ 
        exists, a left, right, two-sided $\calO$-module (and morphisms thereof) is equivalent to a $(\calO,\bbZ)$-bimodule, $(\bbZ,\calO)$-bimodule, and $(\calO, \calO)$-bimodule (and morphisms thereof) respectively, where $\bbZ$ is the \CrefAndHyperrefIfExist{definition:constant_sheaf_on_a_site_with_sheafification}{constant sheaf} of the integer ring $\bbZ$.

\end{enumerate}


\end{definition}


% See Also
% theorem:category_of_modules_over_a_sheaf_of_rings_on_a_site_on_an_essentially_small_category_has_enough_injectives

\TODO{define sheaf hom, sheaf tensor product}


\section{Left/right derived functors of right/left exact functors}

\section{}

\newpage

\section{Assignment 1: Due Friday, Jan 23}

\begin{problem}
    In the following categories, prove whether there are \CrefAndHyperrefIfExist{definition:initial_final_zero_objects_of_a_category}{initial/final objects} and describe what they are.
    \begin{enumerate}
        \item The \CrefAndHyperrefIfExist{definition:category_of_sets}{category of sets}.
        \item The \CrefAndHyperrefIfExist{definition:category_of_groups_of_abelian_groups}{category of groups}.
    \end{enumerate}
\end{problem}

% \TODO{Hom(A,-) preserves limits, Hom(-,A) preserves colimits}


Read the definition of a \CrefAndHyperref{definition:group_object_in_a_category_with_a_final_object}{group object}.

\begin{problem}
    Let $\calC$ be a \CrefAndHyperrefIfExist{definition:locally_small_category}{locally small category} with a \CrefAndHyperrefIfExist{definition:initial_final_zero_objects_of_a_category}{final object} and let $G$ be a group object of $\calC$. Prove that the \CrefAndHyperrefIfExist{definition:representable_functor_on_a_locally_small_category}{representable functor} $h_G: \calC \to \mathbf{Sets}$ in fact factors through \CrefAndHyperrefIfExist{definition:category_of_groups_of_abelian_groups}{$\mathbf{Grp}$}. In other words, $h_G$ can be regarded as a functor $\calC \to \mathbf{Grp}$, and composing this functor with the forgetful functor $\mathbf{Grp} \to \mathbf{Sets}$ recovers the original functor $h_G: \calC \to \mathbf{Sets}$.
\end{problem}

\begin{problem}
    Given a \CrefAndHyperrefIfExist{definition:commutative_ring}{commutative ring} $R$, one can construct a \CrefAndHyperrefIfExist{definition:topological_space}{topological space} $\Spec R$, see \CrefIfExists{definition:affine_scheme} (focus on the construction of the topological space, and ignore the discussion on the structure sheaf).
    \begin{enumerate}
        \item Show that there is a functor $\Spec: \mathbf{CommRing}^{\op} \to \Top$ given by
        \begin{itemize}
            \item sending a commutative ring $R$ to $\Spec R$, and 
            \item sending a ring homomorphism $\varphi: R_1 \to R_2$ to the map $\varphi^*: \Spec R_2 \to \Spec R_1$ given by $\mathfrak{p} \mapsto \varphi^{-1}(\mathfrak{p})$.
            %(in particular, show that $\varphi^*$ is a \CrefAndHyperrefIfExist{definition:continuous_map_of_topological_spaces}{continuous map}).
        \end{itemize}
        \item Show that the above functor $\Spec$ is not \CrefAndHyperrefIfExist{definition:full_and_faithful_functor_between_locally_small_categories}{faithful}. (Hint: one should be able to find examples involving finite rings)
    \end{enumerate}
\end{problem}


Read the definition of a \CrefAndHyperrefIfExist{definition:cartesian_product_of_two_objects_in_a_category_over_an_object}{fiber product} of two objects in a category.

\begin{problem}
    Let $f:X \to Z$ and $g: Y \to Z$ be morphisms in the \CrefAndHyperrefIfExist{definition:category_of_sets}{category of sets}. Prove that the \CrefAndHyperrefIfExist{definition:cartesian_product_of_two_objects_in_a_category_over_an_object}{fiber product} $X \times_Z Y$ exists by explicitly constructing it (along with canonical morphisms from $X \times_Z Y$ to $X,Y,Z$), and verifying that your construction possesses the appropriate universal property.
\end{problem}


\begin{problem}[The Pullback Lemma]
    Consider a commutative diagram in a category $\mathcal{C}$:
    \[
    \begin{tikzcd}
        A \arrow[r, "u"] \arrow[d, "f"] & B \arrow[r, "v"] \arrow[d, "g"] & C \arrow[d, "h"] \\
        X \arrow[r, "p"] & Y \arrow[r, "q"] & Z
    \end{tikzcd}
    \]
    Assume that the right-hand square (with corners $B, C, Y, Z$) is a \CrefAndHyperrefIfExist{definition:cartesian_product_of_two_objects_in_a_category_over_an_object}{Cartesian square}.
    Prove that the left-hand square (with corners $A, B, X, Y$) is a Cartesian square if and only if the outer rectangle (with corners $A, C, X, Z$) is a Cartesian square.
\end{problem}

Read the definition of the \CrefAndHyperrefIfExist{definition:category_of_opens_of_a_topological_space}{category of opens} of a topological space.

\begin{problem}
    Let $\bbR(x)$ denote the field of rational functions of $x$ with coefficients in $\bbR$. Given an open subset $U$ of $\bbR$, let $\calO(U)$ denote 
    $$\calO(U) = \{f \in \bbR(x): f \text{ is defined at all points } x \in U\}.$$
    Note that it is a commutative ring under pointwise addition and multiplication: for $f,g \in \bbR(U)$, we have $(f+g)(x) = f(x) + g(x)$ and $(f \cdot g)(x) = f(x) \cdot g(x)$. 

    Given an inclusion $U \subseteq V$ of open subsets, note that there is an injective ring homomorphism $\calO(V) \hookrightarrow \calO(U)$ and note that this describes a functor/\CrefAndHyperrefIfExist{definition:diagram_in_a_category_indexed_by_a_small_category}{diagram} $\calO: \mathbf{Open}(\bbR)^{\op} \to \mathbf{CommRings}$\CrefIfExists{definition:category_of_opens_of_a_topological_space}\CrefIfExists{definition:opposite_category_of_a_category}\CrefIfExists{category_of_rings} (you do not need to prove this). 
    
    Fix a point $x \in \bbR$. There is a \CrefAndHyperrefIfExist{definition:full_subcategory_of_a_category}{(full) subcategory} $N_x$ of $\mathbf{Open}(\bbR)$ whose objects are open neighborhoods of $x$, so there is an induced diagram $N_x^{\op} \to \mathbf{CommRings}$. Show that the \CrefAndHyperrefIfExist{definition:projective_and_inductive_limits_in_categories}{direct limit}
    $$\varinjlim_{U \in N_x^{\op}} \calO(U)$$
    of this diagram is isomorphic to the following ring:
    $$\calO_x = \{f \in \bbR(x): f \text{ is defined at } x \}.$$
\end{problem}


\newpage

\section{Assignment 2: Due Feb 6 (Problems not yet complete)}

\TODO{some problems on tensor products}

\begin{problem}
    Let $R$ and $S$ be \CrefAndHyperrefIfExist{definition:ring}{not necessarily commutative rings}. Let $\{M_i\}_{i \in I}$ be a small family of \CrefAndHyperrefIfExist{definition:module_of_a_ring}{$R$-$S$-bimodules}. Prove that $\prod_{i \in I} M_i$ and $\bigoplus_{i \in I} M_i$ as constructed in \Cref{definition:product_of_modules_of_rings} \Cref{definition:coproduct_of_modules_of_rings} are respectively the categorical \CrefAndHyperrefIfExist{definition:product_and_coproduct_of_objects_in_a_category}{product and coproduct} in the \CrefAndHyperrefIfExist{definition:category_of_modules_and_bimodules_over_rings}{category of $R$-$S$-bimodule}.
\end{problem}

\begin{problem}
    Let $R$ and $S$ be \CrefAndHyperrefIfExist{definition:ring}{not necessarily commutative rings}. Show that the following categories are \CrefAndHyperrefIfExist{definition:equivalence_of_categories}{equivalent} by producing an explicit equivalence of categories and showing that it is indeed an equivalence of categories:
    \begin{itemize}
        \item The category \CrefAndHyperrefIfExist{definition:category_of_modules_and_bimodules_over_rings}{${}_R \mathbf{Mod}_{S}$}
        \item The category $(R \otimes_\mathbb{Z} S^{\op})-\mathbf{Mod}$ (\Cref{definition:opposite_ring_of_a_ring}) (\Cref{definition:tensor_product_of_a_ring_and_an_algebra_over_a_ring}).
    \end{itemize}
\end{problem}

\begin{problem} \label{problem:yoneda_phliosophy}
    Let $\mathcal{C}$ be a \CrefAndHyperrefIfExist{definition:locally_small_category}{locally small category}. Let $X$ and $Y$ be objects of $\calC$. 
    Show that $X$ and $Y$ are isomorphic if and only if \CrefAndHyperrefIfExist{definition:representable_functor_on_a_locally_small_category}{$h_X$ and $h_Y$} are \CrefAndHyperrefIfExist{definition:natural_transformation_between_functors_between_categories}{naturally isomorphic} functors (note that, dually, we can also say that $X$ and $Y$ are isomorphic if and only if $h^X$ and $h^Y$ are naturally isomorphic functors).
\end{problem}

\begin{problem}
    Let $R,S,T,U$ be \CrefAndHyperrefIfExist{definition:ring}{not necessarily commutative rings}. Let $M$ be an $R-S$-bidmodule, let $N$ be an $S-T$-bimodule, and let $P$ be an $T-U$-bimodule. Show that the tensor products $(M \otimes_S N) \otimes_T P$ and $M \otimes_S (N \otimes_T P)$ are isomorphic using \Cref{problem:yoneda_phliosophy} and the \CrefAndHyperrefIfExist{theorem:tensor_hom_adjunction_for_bimodules_of_rings}{tensor-Hom adjunction}\footnote{In fact, the tensor products are naturally isomorphic, and you should think about what that means precisely, but I am not telling you to prove that the isomorphism is naturally isomorphic}. In particular, do not write out an explicit isomorphism between the tensor products.
\end{problem}

\begin{problem}
    Prove the following, called the short five lemma. Do not assume the general five lemma. 
    % You may use the snake lemma.

    % The Short Five Lemma in a General Abelian Category

    Let $\mathcal{A}$ be an \CrefAndHyperrefIfExist{definition:abelian_category}{abelian category}. Consider the following commutative diagram in $\mathcal{A}$ where the rows are \CrefAndHyperrefIfExist{definition:short_exact_sequence_in_an_additive_category}{short exact}:

    \[
    \begin{tikzcd}
    0 \arrow[r] & A \arrow[r, "f"] \arrow[d, "\alpha"] & B \arrow[r, "g"] \arrow[d, "\beta"] & C \arrow[r] \arrow[d, "\gamma"] & 0 \\
    0 \arrow[r] & A' \arrow[r, "f'"] & B' \arrow[r, "g'"] & C' \arrow[r] & 0
    \end{tikzcd}
    \]
    \begin{enumerate}
        \item If $\alpha$ and $\gamma$ are monomorphisms, then $\beta$ is a monomorphism.
        \item If $\alpha$ and $\gamma$ are epimorphisms, then $\beta$ is an epimorphism.
        \item If $\alpha$ and $\gamma$ are isomorphisms, then $\beta$ is an isomorphism.
    \end{enumerate}
    


\end{problem}



\appendix

\section{}

\begin{definition} \label{definition:representable_functor_on_a_category_enriched_in_a_monoidal_category}
    Let $C$ be a \CrefAndHyperrefIfExist{definition:category_enriched_in_a_monoidal_category}{category enriched in a monidal category} $\mathcal{V}$. Given an object $X$ of $C$, the \hldef{functor of points} \hl{$h_X$} is the \CrefAndHyperrefIfExist{definition:functor_between_categories}{functor}/\CrefAndHyperrefIfExist{definition:presheaf_on_a_category}{presheaf} $C^{\op} \to \mathcal{V}$ given by $T \mapsto \Hom_C(T, X)$. A functor $C^{\op} \to \mathcal{V}$ (or equivalently, a presheaf on $C$ valued in $\mathcal{V}$) is said to be \hldef{representable} if it is \CrefAndHyperrefIfExist{definition:natural_transformation_between_functors_between_categories}{naturally isomorphic} to some functor $h_X$ of points for an object $X$ of $C$.

    Dually, a functor $C \to \calV$ is called \hldef{co-representable} if it is naturally isomorphic to a functor $T \mapsto \Hom_C(X, T)$ for an object $X$ in $C$. 

    For instance, we may speak of these notions when $\calV$ is the monoidal category $\Sets$, i.e. $C$ is a \CrefAndHyperrefIfExist{definition:locally_small_category}{locally small category}.
\end{definition}
\begin{definition} \label{definition:symmetric_monoidal_category}
A \hldef{symmetric monoidal category} is a \hyperrefIfExists{definition:monoidal_category}{monoidal category} $(\mathcal{C}, \otimes, \mathbb{I})$ together with a natural isomorphism (symmetry)
$$\hlin{\gamma_{X,Y}: X \otimes Y \xrightarrow{\cong} Y \otimes X}$$
for all $X, Y \in \mathcal{C}$, such that for all $X, Y, Z \in \mathcal{C}$ the following holds:
\begin{itemize}
    \item $\gamma_{Y,X} \circ \gamma_{X,Y} = \mathrm{id}_{X \otimes Y}$ (involutivity);
    % % \TODO{TODO: add the hexagon and symmetry coherence diagrams}
    % \item the hexagon and symmetry coherence diagrams commute.
        \item the \textbf{hexagon coherence diagrams} commute:
        \[
        \begin{tikzcd}[column sep=large]
        (X \otimes Y) \otimes Z \arrow[r, "\alpha_{X,Y,Z}"] \arrow[d, "\gamma_{X,Y} \otimes \mathrm{id}_Z"'] & X \otimes (Y \otimes Z) \arrow[r, "\gamma_{X, Y \otimes Z}"] & (Y \otimes Z) \otimes X \\
        (Y \otimes X) \otimes Z \arrow[rr, "\alpha_{Y,X,Z}"'] & & Y \otimes (X \otimes Z) \arrow[u, "\mathrm{id}_Y \otimes \gamma_{X,Z}"']
        \end{tikzcd}
        \]
        and the analogous hexagon with inverse braiding:
        \[
        \begin{tikzcd}[column sep=large]
        X \otimes (Y \otimes Z) \arrow[r, "\alpha^{-1}_{X,Y,Z}"] \arrow[d, "\mathrm{id}_X \otimes \gamma_{Y,Z}"'] & (X \otimes Y) \otimes Z \arrow[r, "\gamma_{X \otimes Y, Z}"] & Z \otimes (X \otimes Y) \\
        X \otimes (Z \otimes Y) \arrow[rr, "\alpha^{-1}_{X,Z,Y}"'] & & (X \otimes Z) \otimes Y \arrow[u, "\gamma_{X,Z} \otimes \mathrm{id}_Y"']
        \end{tikzcd}
        \]
    \item the \textbf{symmetry coherence diagram} commutes:
        \[
        \begin{tikzcd}
        X \otimes Y \arrow[r, "\gamma_{X,Y}"] \arrow[dr, swap, "\mathrm{id}_{X \otimes Y}"] & Y \otimes X \arrow[d, "\gamma_{Y,X}"] \\
        & X \otimes Y
        \end{tikzcd}
        \]
\end{itemize}
A \hldef{closed symmetric monoidal category} usually refers to a symmetric monoidal category that is \hyperrefIfExists{definition:closed_monoidal_category}{closed as a monoidal category}. 
\end{definition}
\begin{definition}[Grothendieck Universe] \label{definition:grothendieck_universe}
    Let $U$ be a set. We say $U$ is a \hldef{Grothendieck universe} (or just a \hldef{universe}) if the following conditions hold:
    \begin{enumerate}
        \item If $x \in U$ and $y \in x$, then $y \in U$ (transitivity).
        \item If $x,y \in U$, then $\{x,y\} \in U$ (closed under pair formation).
        \item If $x \in U$, then the power set $\mathcal{P}(x) \in U$.
        \item If $I \in U$ and $(x_\alpha)_{\alpha \in I}$ is a family with each $x_\alpha \in U$, then $\bigcup_{\alpha \in I} x_\alpha \in U$.
    \end{enumerate}
    A set $X$ is called \hldef{$U$-small} or a \hldef{$U$-set} if $X \in U$.
\end{definition}

\begin{definition} \label{definition:tarski_grothendieck_set_theory}
\hldef{Tarski--Grothendieck set theory}, denoted by \hl{$\mathsf{TG}$}, is the theory consisting of the axioms of \CrefAndHyperrefIfExist{definition:zermelo_fraenkel_set_theory}{$\mathsf{ZFC}$} together with \hldef{Tarski's Axiom of Universes}, which asserts that for every set $x$ there exists a \CrefAndHyperrefIfExist{definition:grothendieck_universe}{universe} $U$ such that $x \in U$:
$$ \hlin{ \forall x \, \exists U \, (U \text{ is a Grothendieck universe} \land x \in U) } $$

\end{definition}
\begin{definition} \label{definition:hereditarily_finite_sets}
A set $x$ is a \hldef{hereditarily finite set} if its \CrefAndHyperrefIfExist{definition:transitive_closure_of_a_set}{transitive closure} $\text{tc}(x)$ is a finite set. 

The collection of all hereditarily finite sets is denoted by \hl{$V_\omega$}. \CrefAndHyperref{proposition:collection_of_hereditarily_finite_sets_is_a_set}{It is a set}.
\end{definition}
\begin{definition} \label{definition:hierarchy_of_grothendieck_universes_assuming_tarski_grothendieck_set_theory}
    Given an axiom system containing the \CrefAndHyperrefIfExist{definition:tarski_grothendieck_set_theory}{Tarski-Grothendieck set theory} axioms, the \hldef{hierarchy of Grothendieck universes} consists of the \CrefAndHyperrefIfExist{definition:grothendieck_universe}{Grothendieck universes} $U_n$ for $n \in \mathbb{N} \cup \{0\}$ constructed as follows:
    \begin{enumerate}
        \item Base case: $U_0 \coloneq V_\omega$, the set of all \CrefAndHyperrefIfExist{definition:hereditarily_finite_sets}{hereditarily finite sets}.
        \item Successor step: For each $n \geq 0$, $U_{n+1}$ is the unique minimal Grothendieck universe such that $U_n \in U_{n+1}$.
    \end{enumerate}
    In particular, $U_n \in U_{n+1}$ and $U_n \subset U_{n+1}$ for all $n$.
\end{definition}

\begin{definition} \label{definition:category_of_categories_relative_to_a_grothendieck_universe}
Let $\mathcal{U}$ be a fixed \CrefAndHyperrefIfExist{definition:grothendieck_universe}{Grothendieck universe}. The \hldef{category of categories} (relative to $\mathcal{U}$) is the category defined by:
\begin{itemize}
    \item The objects are all \CrefAndHyperrefIfExist{definition:category}{categories} $\mathcal{C}$ such that $\text{Ob}(\mathcal{C}) \in \mathcal{U}$ (often called $\mathcal{U}$-categories).
    \item The morphisms are \CrefAndHyperrefIfExist{definition:functor_between_categories}{functors} between such categories.
\end{itemize}
This category is denoted by
\hl{$\mathbf{CAT}$}
In this context, $\mathbf{Cat}$ (the category of categories belonging to $\mathcal{U}$) is an object of $\mathbf{CAT}$.
\end{definition}

\begin{definition} \label{definition:concrete_category_over_a_category}
Let $\mathcal{X}$ be a category. A \hldef{concrete category over $\mathcal{X}$} is a pair $(\mathcal{C}, U)$ consisting of a category $\mathcal{C}$ and a \CrefAndHyperrefIfExist{definition:full_and_faithful_functor_between_locally_small_categories}{faithful functor}
$U \colon \mathcal{C} \to \mathcal{X}$
In this context, $U$ is called the \hldef{underlying functor} (or \hldef{forgetful functor}) of the concrete category.

When $\mathcal{X} = \mathbf{Set}$ (the category of sets), the pair $(\mathcal{C}, U)$ is simply referred to as a \hldef{concrete category}. For any object $A$ in $\mathcal{C}$, the set $U(A)$ is called the \hldef{underlying set} of $A$, and for any morphism $f: A \to B$, the function $U(f): U(A) \to U(B)$ is called the \hldef{underlying function} of $f$.
\end{definition}

\section{Grothendieck topologies}

\begin{definition}[{\cite[Expos\'e I D\'efinition 4.1]{SGA4_I}}] \label{definition:sieve_on_an_object_in_a_category}
Let $C$ be a \CrefAndHyperrefIfExist{definition:category}{(large) category}. 

\begin{enumerate}
    \item A \hldef{sieve $S$ on the category $C$} is a \CrefAndHyperrefIfExist{definition:full_subcategory_of_a_category}{full subcategory} $D$ of $C$ such that for any object $U$ of $C$ there exists an object $V$ of \TODO{correctly parse the definiton}
    \item A \hldef{sieve $S$ on an object $U \in \operatorname{Ob}(C)$} is a collection of morphisms in $C$ with codomain $U$ that is closed under precomposition by any compatible morphism in $C$. In other words, $S$ is a sieve if for every $f : V \to U$ in $S$ and morphism $g : W \to V$ in $C$, the composition $f \circ g : W \to U$ is also in $S$. 

    Given a morphism $f: V \to U$ in a sieve $S$, we also say that \hldef{$f$ factors through $U$}.
\end{enumerate}
\end{definition}

% \begin{definition}[Generated sieve] \label{definition:sieve_on_an_object_of_a_category_generated_by_a_family_of_morphisms}
%     Let $\mathcal{C}$ be a (large) category, $X \in \mathcal{C}$ an object, and $S$ a \CrefAndHyperrefIfExist{definition:sieve_on_an_object_in_a_category}{sieve} on $X$. A sieve $S$ is said to be \hldef{generated} by a family of morphisms $\{f_i : U_i \to X\}_{i \in I}$ if $S$ is the smallest sieve on $X$ containing all the morphisms $f_i$, i.e., $S$ consists precisely of all morphisms $g : Y \to X$ such that $g$ factors through some $f_i$.
% \end{definition}

\begin{definition} \label{definition:sieve_on_an_object_of_a_category_generated_by_a_family_of_morphisms}
Let $\mathcal{C}$ be a \CrefAndHyperrefIfExist{definition:category}{category} and $U \in \mathcal{C}$ an object. Let $\mathcal{S} = \{f_i: U_i \to U\}_{i \in I}$ be a family of morphisms with codomain $U$. 

The \hldef{sieve generated by $\mathcal{S}$}, denoted \hl{$(\mathcal{S})$} or \hl{$\langle \mathcal{S} \rangle$}, is the smallest \CrefAndHyperrefIfExist{definition:sieve_on_an_object_in_a_category}{sieve on $U$} containing all the morphisms in $\mathcal{S}$.

Explicitly, a morphism $h: V \to U$ belongs to the generated sieve if and only if $h$ factors through some morphism in $\mathcal{S}$. That is, there exists an index $i \in I$ and a morphism $g: V \to U_i$ such that
$$ h = f_i \circ g. $$
\end{definition}

\begin{definition} \label{definition:pullback_sieve_of_an_object_in_a_category_via_a_morphism_to_the_object}
Let $C$ be a category, let $U \in \operatorname{Ob}(C)$, and let $S$ be a \CrefAndHyperrefIfExist{definition:sieve_on_an_object_in_a_category}{sieve on $U$}.
For a morphism $f : V \to U$ in $C$, the \hldef{pullback sieve} \hl{$f^*S$} (or \hldef{basechange sieve} \hl{$S \times_U V$}) on $V$ is defined by
\[
f^*S = \{ g : W \to V \mid f \circ g \in S \}.
\]
In other words, $f^*S$ consists of all morphisms into $V$ whose composite with $f$ belongs to the sieve $S$ on $U$.
\end{definition}
% \begin{definition}[Grothendieck topology] \label{definition:grothendieck_topology_on_a_category_site_covering_sieve_topologically_generating_family}
%     Let $\mathscr{U}$ be a \hyperrefIfExists{definition:grothendieck_universe}{universe}\CrefIfExists{definition:grothendieck_universe} and let $\calC$ be a \hyperrefIfExists{definition:locally_small_category}{locally small category}\CrefIfExists{definition:locally_small_category}.

%     \begin{enumerate}
%         \item \textbf{(Grothendieck Topology via Sieves)}
%         A \hldef{Grothendieck topology} $J$ on $\calC$ is an assignment to each object $U \in \calC$ of a collection $J(U)$ of \CrefAndHyperrefIfExist{definition:sieve_on_an_object_in_a_category}{sieves} on $U$, called \hldef{covering sieves}, satisfying:
%         \begin{enumerate}
%             \item (Maximality) The maximal \CrefAndHyperrefIfExist{definition:sieve_on_an_object_in_a_category}{sieve} $\{ f : V \to U \mid V \in \calC \}$ is in $J(U)$.
%             \item (Stability) If $S \in J(U)$ and $f : V \to U$ is any morphism, then the \CrefAndHyperrefIfExist{definition:pullback_sieve_of_an_object_in_a_category_via_a_morphism_to_the_object}{pullback sieve} $f^{*}S$ is in $J(V)$.
%             \item (Transitivity/Local Character) If $S$ is a sieve on $U$ and there exists a covering sieve $R \in J(U)$ such that for every morphism $f : V \to U$ in $R$, the pullback sieve $f^{*}S$ is in $J(V)$, then $S \in J(U)$.
%         \end{enumerate}

%         % \item \textbf{(Grothendieck Pretopology / Basis)}
%         % If $\calC$ admits fiber products, one can define a topology via \hldef{covering families}. A \hldef{Grothendieck pretopology} (or basis) is a collection $K(U)$ of families $\{U_i \to U\}_{i \in I}$ for each object $U$, satisfying:
%         % \begin{itemize}
%         %     \item (Isomorphism) $\{U' \xrightarrow{\sim} U\} \in K(U)$ for any isomorphism.
%         %     \item (Stability) If $\{U_i \to U\} \in K(U)$ and $V \to U$ is a morphism, then $\{U_i \times_U V \to V\} \in K(V)$.
%         %     \item (Composition) If $\{U_i \to U\} \in K(U)$ and for each $i$, $\{V_{ij} \to U_i\} \in K(U_i)$, then the composite family $\{V_{ij} \to U\} \in K(U)$.
%         % \end{itemize}
%         % Every pretopology generates a unique Grothendieck topology $J$, where $S \in J(U)$ iff $S$ contains a covering family from the pretopology.

%         \item A \hldef{site} is a pair $(\calC, J)$ consisting of a category $\calC$ and a Grothendieck topology $J$.

%         \item A family of objects $\mathcal{G} = \{G_\alpha\}$ in a site $(\calC, J)$ is called a \hldef{topologically generating family} if for every object $X \in \calC$, there exists a covering sieve $S \in J(X)$ \CrefAndHyperrefIfExist{definition:sieve_on_an_object_of_a_category_generated_by_a_family_of_morphisms}{generated by} morphisms with domains in $\mathcal{G}$. Equivalently, every object $X$ admits a cover $\{U_i \to X\}$ where each $U_i \in \mathcal{G}$.

%         \item A \hldef{$\mathscr{U}$-site} is a site whose underlying category is $\mathscr{U}$-locally small and which admits a $\mathscr{U}$-small topologically generating family.
%     \end{enumerate}
% \end{definition}

\begin{definition}[Grothendieck topology] \label{definition:grothendieck_topology_on_a_category_site_covering_sieve_topologically_generating_family}
    Let $\mathscr{U}$ be a \hyperrefIfExists{definition:grothendieck_universe}{universe}\CrefIfExists{definition:grothendieck_universe}.
    \begin{enumerate}
        % \item Let $C$ be a \hyperrefIfExists{definition:locally_small_category}{locally small category}\CrefIfExists{definition:locally_small_category}. A \hldef{Grothendieck topology on $C$} assigns to each object $U$ of $C$ a collection of families of morphisms $\{U_i \to U\}_{i \in I}$, called \hldef{coverings of $U$}, satisfying:
        % \begin{itemize}
        %     \item (Isomorphism) If $f: V \to U$ is an isomorphism in $C$, then $\{f: V \to U\}$ is a covering of $U$.
        %     \item (Stability under base change) If $\{U_i \to U\}_{i \in I}$ is a covering of $U$ and $V \to U$ is any morphism, then the family $\{ U_i \times_U V \to V \}_{i \in I}$ is a covering of $V$.
        %     \item (Transitivity) If $\{U_i \to U\}_{i \in I}$ is a covering of $U$ and for each $i$, $\{V_{ij} \to U_i\}_{j \in J_i}$ is a covering of $U_i$, then the family $\{ V_{ij} \to U \}_{i \in I,\, j \in J_i}$ is a covering of $U$.
        % \end{itemize}

        \item (See \cite[Expos\'e II, D\'efinition 1.1]{SGA4_I}) Let $\calC$ be a \CrefAndHyperrefIfExist{definition:category}{category}. A \hldef{Grothendieck topology on $\calC$} assigns to each object $U$ of $\calC$ a collection \hl{$J(U)$} of \CrefAndHyperrefIfExist{definition:sieve_on_an_object_in_a_category}{sieves} $\{U_i \to U\}_{i \in I}$, each called a \hldef{covering sieve of $U$}, satisfying:
        \begin{enumerate}
            \item (Stability under ``base change''): If $S \in J(U)$ is a covering sieve of an object $U$, and $f: V \to U$ is any morphism in $\calC$, then the \CrefAndHyperrefIfExist{definition:pullback_sieve_of_an_object_in_a_category_via_a_morphism_to_the_object}{pullback sieve} $f^* S$ is a covering sieve of $U$.
            % \item (Local character condition) If $F$ is a sieve on $U$ such that the sieve $\bigcup_...$ \TODO{}
            \item (Local character condition) If $S$ is a sieve on $U$, and if there exists a covering sieve $R \in J(U)$ such that for all $f: V \to U$ in $R$ the \CrefAndHyperrefIfExist{definition:pullback_sieve_of_an_object_in_a_category_via_a_morphism_to_the_object}{pullback sieve} $f^* S$ is in $J(V)$, then $S \in J(U)$. 
            
            \item The \CrefAndHyperrefIfExist{definition:maximal_sieve_on_an_object_in_a_category}{maximal sieve} is a covering sieve.
        \end{enumerate}


        % Equivalently, a Grothendieck topology $J$ on a category $C$ is an assignment of a collection $J(U)$ of \CrefAndHyperrefIfExist{definition:sieve_on_an_object_in_a_category}{sieves} on each object $U \in \operatorname{Ob}(C)$ such that:
        % \begin{enumerate}
        %     \item the maximal \CrefAndHyperrefIfExist{definition:sieve_on_an_object_in_a_category}{sieve} $\{ f : V \to U \mid f \in \operatorname{Mor}(C) \}$ belongs to $J(U)$,
        %     \item if $S \in J(U)$ and $f : V \to U$, then the \CrefAndHyperrefIfExist{definition:pullback_sieve_of_an_object_in_a_category_via_a_morphism_to_the_object}{pullback sieve $f^{*}S$} on $V$ belongs to $J(V)$,
        %     \item if $S$ is a sieve on $U$, and if there exists $R \in J(U)$ such that for all $f : V \to U$ in $R$ the \CrefAndHyperrefIfExist{definition:pullback_sieve_of_an_object_in_a_category_via_a_morphism_to_the_object}{pullback sieve $f^{*}S$} is in $J(V)$, then $S \in J(U)$.
        % \end{enumerate}

        Some will refer to a Grothendieck topology as simply a \hldef{topology}, not to be confused with the related, but less general, notion of a \CrefAndHyperrefIfExist{definition:topological_space}{topology on a set}.


        \item (See \cite[Expos\'e II, 1.1.5]{SGA4_I}) A \hldef{site} is a category $\calC$ equipped with a Grothendieck topology.

        When we are working with a \CrefAndHyperref{definition:basis_and_grothendieck_pretopology_for_a_grothendieck_topology_on_a_category}{Grothendieck pretopology} $K$ on a category $\calC$, we may regard $\calC$ as a site by equipping it with the \CrefAndHyperref{definition:grothendieck_topology_generated_by_a_pretopology}{Grothendieck topology generated by} $K$. 

        \item (See \cite[Expos\'e II, D\'efinition 1.2]{SGA4_I}) Let $(\calC, J)$ be a site. A family of morphisms $(U_i \to U)_{i \in I}$ is called a \hldef{covering family of $U$ (with respect to the site/topology)} or a \hldef{cover of $U$ (with respect to the site/topology)} if the \CrefAndHyperrefIfExist{definition:sieve_on_an_object_of_a_category_generated_by_a_family_of_morphisms}{sieve generated by} the family is a covering sieve of $U$. 

        \item (See \cite[Expos\'e II, D\'efinition 3.0.1]{SGA4_I}) Let $(\calC, J)$ be a \CrefAndHyperrefIfExist{definition:grothendieck_topology_on_a_category_site_covering_sieve_topologically_generating_family}{site}, where $J$ is a Grothendieck topology on $\calC$.

        A family $G$ of objects $\calC$ is called a \hldef{topologically generating family of the site/topology} or a \hldef{generating family/collection of the site/topology} if for every object $X \in \calC$, there is a covering family $\{X_\alpha \to X\}_{\alpha \in A}$ of $X$ such that every $X_\alpha$ is a member of $G$.  

        Equivalently, the Grothendieck topology $J$ is the smallest Grothendieck topology containing all covers of the $U_i$. Also equivalently, for any $S \in J(X)$, the sieve $S$ contains a covering family $\{V_i \to X\}$ such that each morphism $V_i \to X$ factors through some member of $G$. \TODO{Verify that these claimed equivalences are indeed equivalences}
        
        % A family of objects $\{U_i\}_{i \in I}$ in $\calC$ is called a \hldef{topologically generating family} if for every object $X \in \calC$ and every covering sieve $S \in J(X)$, the sieve $S$ is \CrefAndHyperrefIfExist{definition:sieve_on_an_object_of_a_category_generated_by_a_family_of_morphisms}{generated by} pullbacks of covering families from the family $\{U_i\}$.

        % More precisely, this means that for any $S \in J(X)$, the sieve $S$ contains a covering family $\{V_j \to X\}$ such that each morphism $V_j \to X$ factors through some $U_i$, and the covering families of the $U_i$ generate the topology $J$. 
        % Equivalently, the Grothendieck topology $J$ is the smallest Grothendieck topology containing all coverings of the $U_i$.

        % When one speaks of a \hldef{generating family/collection} of a site, one usually refers to the above notion of a topologically generating family.

        \item (See \cite[Expos\'e II, D\'efinition 3.0.2]{SGA4_I}) A \hldef {$\mathscr{U}$-site} is a site whose underlying category $\calC$ is \hyperrefIfExists{definition:locally_small_category}{$\mathscr{U}$-locally small}\CrefIfExists{definition:locally_small_category} and which has a $\mathscr{U}$-small topologically generating family. A $\mathscr{U}$-site is called \hldef{$\mathscr{U}$-small} if its underlying category is $\mathscr{U}$-small. Similarly, a \hldef{small site} is a site whose underlying category is a set and a \hldef{locally small site} is a site whose underlying category is \CrefAndHyperrefIfExist{definition:locally_small_category}{locally small}.
    \end{enumerate}
\end{definition}

\begin{definition} \label{definition:site_of_opens_on_a_topological_space}
    Let $(X, \tau_X)$ be a topological space. The \hldef{small site associated to $X$} or \hldef{the site of open covers of $X$} or \hldef{the canonical site on $\operatorname{Open} X$} is the \CrefAndHyperrefIfExist{definition:category_of_opens_of_a_topological_space}{category $\operatorname{Open}(X)$ of open subsets} of $X$ with inclusion morphisms, equipped with the canonical \CrefAndHyperrefIfExist{definition:grothendieck_topology_on_a_category_site_covering_sieve_topologically_generating_family}{Grothendieck topology} \CrefAndHyperrefIfExist{definition:grothendieck_topology_generated_by_a_pretopology}{generated by} the \CrefAndHyperrefIfExist{definition:basis_and_grothendieck_pretopology_for_a_grothendieck_topology_on_a_category}{Grothendieck pretopology} whose covering families $\{U_i \to U\}_{i \in I}$, for $U \in \operatorname{Open}(X)$ are families of morphisms in $\operatorname{Open}(X)$ such that $\bigcup_{i \in I} U_i = U$. In other words, $\{U_i \to U\}_{i \in I}$ is a covering for the pretopology if it is an \CrefAndHyperrefIfExist{definition:open_covering_of_a_topological_space}{open coverings}.
\end{definition}
\begin{definition}[Sheaf on a site] \label{definition:sheaf_on_a_site}

% \TODO{There might be some need to say that $\calA$ is a category for which sheaves on the site ``can be defined''}
% \TODO{go through statements using the notion of sheaves and make sure that the value categories have small products and that the categories have small generating families.}

Let $(\calC, J)$ be a \CrefAndHyperrefIfExist{definition:grothendieck_topology_on_a_category_site_covering_sieve_topologically_generating_family}{site}. Let $\calA$ be a \CrefAndHyperrefIfExist{definition:category}{(large) category}.
\begin{enumerate}
    \item A \CrefAndHyperrefIfExist{definition:presheaf_on_a_category}{presheaf} $\calF: \calC^{\op} \to \calA$\CrefIfExists{definition:opposite_category_of_a_category} is called a \hldef{sheaf on the site $(\calC, J)$ valued in $\calA$} if, for every object $U$ of $\calC$ and every \CrefAndHyperrefIfExist{definition:grothendieck_topology_on_a_category_site_covering_sieve_topologically_generating_family}{covering sieve} $S \in J(U)$, the \CrefAndHyperrefIfExist{definition:limit_and_colimit_of_a_diagram_in_a_category}{limit}
    $$\varprojlim_{(V \to U) \in (\calD_S)^{\op}} \calF|_{\calD_S}(V),$$
    exists and the canonical natural morphism
    $$\calF(U) \to \varprojlim_{(V \to U) \in (\calD_S)^{\op}} \calF|_{\calD_S}(V)$$
    is an isomorphism. Here, $\calD_S \hookrightarrow \calC/U$\CrefIfExists{definition:category_of_objects_over_under_a_fixed_object_in_a_category} is the full \CrefAndHyperrefIfExist{definition:downward_upward_closed_subcategory_of_a_category}{downward-closed subcategory} such that $\operatorname{Ob}(\calD_S) = \{(f: V \to U): f \in S(V)\}$,

    In particular, when we are working with a \CrefAndHyperref{definition:basis_and_grothendieck_pretopology_for_a_grothendieck_topology_on_a_category}{Grothendieck pretopology} $K$ on a category $\calC$, we may speak of sheaves on the site whose Grothendieck topology is the \CrefAndHyperref{definition:grothendieck_topology_generated_by_a_pretopology}{one generated by} $K$.

    \item Given sheaves $\calF, \calG: \calC^{\op} \to \calA$ on the site $(\calC, J)$, a \hldef{morphism between the sheaves} is a \CrefAndHyperrefIfExist{definition:presheaf_on_a_category}{morphism} between $\calF$ and $\calG$ as presheaves.


    \item Let $U$ be a \hyperrefIfExists{definition:grothendieck_universe}{universe}\CrefIfExists{definition:grothendieck_universe}. A \hldef{$U$-sheaf} typically refers to a $U$-presheaf that is a sheaf for a $U$-site. In other words, a $U$-sheaf is a sheaf on a site whose underlying category is \hyperrefIfExists{definition:locally_small_category}{$U$-locally small}\CrefIfExists{definition:locally_small_category} and which has a $U$-small topologically generating family such that the sheaf is valued in $U$-sets.

    \item The \hldef{sheaf category/category of $\calA$-valued sheaves on $\calC$} is the (large) category defined as the full subcategory of $\PreShv(\calC, \calA)$ whose objects are the sheaves on $\calC$ with values in $\calA$. Common notations for the sheaf category include \hl{$\Shv(\calC, \calA)$}, \hl{$\Shv(\calC, J, \calA)$}, \hl{$\Sh(\calC, \calA)$}, \hl{$\Sh(\calC, J, \calA)$}. If the value category $\calA$ is clear from context, then notations such as \hl{$\Shv(\calC)$}, \hl{$\Shv(\calC, J)$}, \hl{$\Sh(\calC)$}, \hl{$\Sh(\calC, J)$} are also common.

\end{enumerate}

% Let $(\calC, J)$ be a \CrefAndHyperrefIfExist{definition:grothendieck_topology_on_a_category_site_covering_sieve_topologically_generating_family}{site} with a small \CrefAndHyperrefIfExist{definition:grothendieck_topology_on_a_category_site_covering_sieve_topologically_generating_family}{topological generating family} (or a $U$-small topologically generating family if a \CrefAndHyperrefIfExist{definition:grothendieck_universe}{universe} $U$ is available) and let $\mathcal{A}$ be a \CrefAndHyperrefIfExist{definition:category}{(large) category} that has all \CrefAndHyperrefIfExist{definition:locally_small_category}{small} \CrefAndHyperrefIfExist{definition:product_and_coproduct_of_objects_in_a_category}{products} (Some common examples of categories that have small products and thus play the role of $\calA$ here include $\mathcal{A} = \text{Set}$, $\text{Ab}$, $R\mathbf{-mod}$ for a fixed ring $R$, $\text{rings}$). 
% \begin{enumerate}

%     \item For any object $U$ of $\calC$ and every covering $\{U_i \to U\}_{i \in I}$ in $J$, note that there are morphisms $U_i \times_U U_j \to U_i$ for every $i,j \in I$. 
%     % Consider the subcategory of $C$ consisting of the objects $U_i$ and $U_i \times_U U_j$, together with these morphisms.
%     Given any presheaf $\calF: C^{\op} \to \calA$, there is a \CrefAndHyperrefIfExist{definition:diagram_in_a_category_indexed_by_a_small_category}{diagram} in $\calA$ consisting of objects $\calF(U_i)$ and $\calF(U_i \times_U U_j)$ and morphisms $\calF(U_i) \to \calF(U_i \times_U U_j)$. The presheaf $\calF$ is called a \hldef{sheaf on the site $(\calC, J)$ valued in $\calA$} if, for every object $U$ of $\calC$ and every covering $\{U_i \to U\}_{i \in I}$ in $J$, the sections object $\calF(U)$ is the \CrefAndHyperrefIfExist{definition:limit_and_colimit_of_a_diagram_in_a_category}{limit} of the aforementioned diagram:
    
%     % A \hyperrefIfExists{definition:presheaf_on_a_category}{presheaf}\CrefIfExists{definition:presheaf_on_a_category} $\mathcal{F}: C^{\mathrm{op}} \to \mathcal{A}$ is a \hldef{sheaf on the site $(\calC,J)$ valued in $\calA$} if, for every object $U$ of $\calC$ and every covering $\{U_i \to U\}_{i \in I}$ in $J$, the sections object $\calF(U)$ is the \CrefAndHyperrefIfExist{definition:limit_and_colimit_of_a_diagram_in_a_category}{limit} of the sections objects $\calF(U_i)$:
%     % $$\calF(U) \cong \varprojlim_{}$$
    
%     % following sequence is an \CrefAndHyperrefIfExist{definition:equalizer_and_coequalizer_of_morphisms_in_a_category}{equalizer} in $\mathcal{A}$:
%     % \[
%     % \mathcal{F}(U) \to \prod_{i} \mathcal{F}(U_i) \rightrightarrows \prod_{i, j} \mathcal{F}(U_i \times_U U_j)
%     % \]
%     % where the first map sends $s$ to $(\mathcal{F}(U_i \to U)(s))_i$ and the arrows to $(\mathcal{F}(U_i \times_U U_j \to U_i)(s_i))_{i,j}$ and $(\mathcal{F}(U_i \times_U U_j \to U_j)(s_j))_{i,j}$, respectively.

%     % \item A \hyperrefIfExists{definition:presheaf_on_a_category}{presheaf}\CrefIfExists{definition:presheaf_on_a_category} $\mathcal{F}: C^{\mathrm{op}} \to \mathcal{A}$ is a \hldef{sheaf on the site $(\calC,J)$ valued in $\calA$} if, for every object $U$ of $\calC$ and every covering $\{U_i \to U\}_{i \in I}$ in $J$, the following sequence is an \CrefAndHyperrefIfExist{definition:equalizer_and_coequalizer_of_morphisms_in_a_category}{equalizer} in $\mathcal{A}$:
%     % \[
%     % \mathcal{F}(U) \to \prod_{i} \mathcal{F}(U_i) \rightrightarrows \prod_{i, j} \mathcal{F}(U_i \times_U U_j)
%     % \]
%     % where the first map sends $s$ to $(\mathcal{F}(U_i \to U)(s))_i$ and the arrows to $(\mathcal{F}(U_i \times_U U_j \to U_i)(s_i))_{i,j}$ and $(\mathcal{F}(U_i \times_U U_j \to U_j)(s_j))_{i,j}$, respectively.

%     \item A \hldef{morphism of sheaves} $\calF: \calC^{\op} \to \calA$ is a \hyperrefIfExists{definition:presheaf_on_a_category}{morphism as presheaves}\CrefIfExists{definition:presheaf_on_a_category}. 


%     \item Let $U$ be a \hyperrefIfExists{definition:grothendieck_universe}{universe}\CrefIfExists{definition:grothendieck_universe}. A \hldef{$U$-sheaf} typically refers to a $U$-presheaf that is a sheaf for a $U$-site. In other words, a $U$-sheaf is a sheaf on a site whose underlying category is \hyperrefIfExists{definition:locally_small_category}{$U$-locally small}\CrefIfExists{definition:locally_small_category} and which has a $U$-small topologically generating family such that the sheaf is valued in $U$-sets.

%     \item The \hldef{sheaf category/category of $\calA$-valued sheaves on $\calC$} is the (large) category defined as the full subcategory of $\PreShv(\calC, \calA)$ whose objects are the sheaves on $C$ with values in $\calA$. Common notations for the sheaf category include \hl{$\Shv(\calC, \calA)$}, \hl{$\Shv(\calC, J, \calA)$}, \hl{$\Sh(\calC, \calA)$}, \hl{$\Sh(\calC, J, \calA)$}. If the value category $\calA$ is clear from context, then notations such as \hl{$\Shv(\calC)$}, \hl{$\Shv(\calC, J)$}, \hl{$\Sh(\calC)$}, \hl{$\Sh(\calC, J)$} are also common.

% \end{enumerate}
\end{definition}



\begin{definition} \label{definition:sheafification_functor_on_a_site}
    Let $\calC$ be a \CrefAndHyperrefIfExist{definition:grothendieck_topology_on_a_category_site_covering_sieve_topologically_generating_family}{site} and let $\calA$ be a \CrefAndHyperrefIfExist{definition:category}{(large) category}.

    Assuming that the \CrefAndHyperrefIfExist{definition:presheaf_on_a_category}{presheaf} category $\PreShv(\calC, \calA)$ (and hence the \CrefAndHyperrefIfExist{definition:sheaf_on_a_site}{sheaf} category $\Shv(\calC, \calA)$) is \CrefAndHyperrefIfExist{definition:locally_small_category}{locally small} (or $U$-locally small if a \CrefAndHyperrefIfExist{definition:grothendieck_universe}{Grothendieck universe} $U$ is available), a \hldef{sheafification functor} refers to a functor
    $$a: \PreShv(\calC, \calA) \to \Shv(\calC, \calA) $$
    that is \CrefAndHyperrefIfExist{definition:adjoint_functors_between_categories_unit_counit_of_adjoint_functors}{left adjoint} to the inclusion functor 
    $$i:\Shv(\calC, \calA) \hookrightarrow \PreShv(\calC, \calA)  .$$
    If such a sheafification functor exists, then it is unique up to unique natural isomorphism. Given a presheaf $P$, the sheafification $a(P)$ is also sometimes called the \hldef{sheaf associated to $P$}.
    \TextIfExists{theorem:sheafification_of_a_presheaf_of_sets_on_a_small_enough_site}{See \Cref{theorem:sheafification_of_a_presheaf_of_sets_on_a_small_enough_site} for common conditions under which sheafification exists.} 
\end{definition}

% See Also
%theorem:sheafification_of_a_presheaf_of_sets_on_a_small_enough_site
\begin{theorem}{cf. {\cite[Expos\'e II, Th\'eor\`eme 3.4]{SGA4_I}}} \label{theorem:sheafification_of_a_presheaf_of_sets_on_a_small_enough_site}
    \begin{enumerate}
        \item Let $U$ be a universe. Let $\calC$ be a \hyperrefIfExists{definition:grothendieck_topology_on_a_category_site_covering_sieve_topologically_generating_family}{$U$-site}\CrefIfExists{definition:grothendieck_topology_on_a_category_site_covering_sieve_topologically_generating_family}. A \CrefAndHyperrefIfExist{definition:sheafification_of_a_presheaf_on_a_topological_space_valued_in_a_category_admitting_direct_colimits}{sheafification functor}
        $$a: \Shv(\calC, \USets) \to \PreShv(\calC, \USets).$$
        exists. 
        % The inclusion functor 
        % $$i: \PreShv(\calC, \USets) \hookrightarrow \Shv(\calC, \USets)$$
        % has a \hyperrefIfExists{definition:adjoint_functors_between_categories_unit_counit_of_adjoint_functors}{left adjoint functor}\CrefIfExists{definition:adjoint_functors_between_categories_unit_counit_of_adjoint_functors}

        \item Let $\calC$ be a site whose underlying category is \CrefAndHyperrefIfExist{definition:locally_small_category}{locally small} and which has a \CrefAndHyperrefIfExist{definition:grothendieck_topology_on_a_category_site_covering_sieve_topologically_generating_family}{topologically generating family} that is a set (rather than a proper class). A sheafification functor 
        $$a: \Shv(\calC, \Sets) \to \PreShv(\calC, \Sets)$$
        exists.

        \item (see e.g. {\cite[3]{nlab:sheafification}}) Let $(\calC, J)$ be a \CrefAndHyperrefIfExist{definition:grothendieck_topology_on_a_category_site_covering_sieve_topologically_generating_family}{site} on an \CrefAndHyperrefIfExist{definition:essentially_small_category}{essentially small category} $\calC$. Suppose that the category $\calA$ is \CrefAndHyperrefIfExist{definition:complete_and_cocomplete_category}{complete, cocomplete}, that small \CrefAndHyperrefIfExist{definition:projective_and_inductive_limits_in_categories}{filtered colimits} in $\calA$ are exact, and that $\calA$ satisfies the IPC-property. A \CrefAndHyperrefIfExist{definition:sheafification_functor_on_a_site}{sheafification functor} 
        $$a: \PreShv(\calC, \calA) \to \Shv(\calC, \calA) $$
        exists.
        \TODO{IPC-property, exactess in this context.}

        \TODO{state as a fact that these categories are complete, cocomplete, with small filtered colimits that are exact}
        This is true for instance of $\calA = \mathbf{Set}, \mathbf{Grp}$, $k-\mathbf{Alg}$ for a field $k$, or $\mathbf{Mod}_R$ for a \CrefAndHyperrefIfExist{definition:ring}{(not necessarily commutative unital) ring $R$}.
    \end{enumerate}
\end{theorem}
\begin{remark}
    If the presheaf is valued in nice ``algebraic category'', e.g. groups, abelian groups, rings, modules over a ring, etc., then the sheafification is again valued in that category. \TODO{Make this more precise.}
\end{remark}
\begin{definition} \label{definition:ringed_site}
    \TODO{there are places where sites and sheaves of rings on them are used, but it would be better to just have them be ringed sites.}

    A \hldef{ringed site} is a \CrefAndHyperrefIfExist{definition:grothendieck_topology_on_a_category_site_covering_sieve_topologically_generating_family}{site} $(\mathcal{C}, J)$ with a small \CrefAndHyperrefIfExist{definition:grothendieck_topology_on_a_category_site_covering_sieve_topologically_generating_family}{topological generating family} equipped with a \CrefAndHyperrefIfExist{definition:sheaf_on_a_site}{sheaf} of (not necessarily commutative) rings $\mathcal{O}$. If the Grothendieck topology $J$ is clear in context, one may even write that $(C, \calO)$ is a ringed site.

    A \hldef{morphism of ringed sites}
    $$ ((\mathcal{C},J),\mathcal{O}) \to ((\mathcal{C}',J'),\mathcal{O}') $$
    consists of a \CrefAndHyperrefIfExist{definition:morphism_of_sites}{morphism of sites} $f : (\mathcal{C},J) \to (\mathcal{C}',J')$ and a \CrefAndHyperrefIfExist{definition:sheaf_on_a_site}{morphism of sheaves} of rings $f^\# : \mathcal{O}' \to f_*\mathcal{O}$ \CrefIfExists{definition:inverse_image_of_a_sheaf_under_a_continuous_functor_of_sites_or_a_site_morphism}.
\end{definition}
\begin{definition} \label{definition:module_over_a_sheaf_of_rings_on_a_site}

    \begin{enumerate}
        \item 
        Let $\mathcal{C}$ be a \CrefAndHyperrefIfExist{definition:grothendieck_topology_on_a_category_site_covering_sieve_topologically_generating_family}{site}, and let $\mathcal{A}$ and $\mathcal{B}$ be \CrefAndHyperrefIfExist{definition:sheaf_on_a_site}{sheaves} of (not necessarily commutative) \CrefAndHyperrefIfExist{definition:ring}{rings} on $\mathcal{C}$. 
        
        \begin{enumerate}
            \item 
            An \hldef{$(\mathcal{A}, \mathcal{B})$-bimodule} (or a \hldef{bimodule over $(\mathcal{A}, \mathcal{B})$}) is a \CrefAndHyperrefIfExist{definition:sheaf_on_a_site}{sheaf} $\mathcal{M}$ of abelian groups on $\mathcal{C}$ equipped with a left $\mathcal{A}$-module structure given by a \CrefAndHyperrefIfExist{definition:sheaf_on_a_site}{morphism of sheaves} of sets
            $$ \lambda: \mathcal{A} \times \mathcal{M} \longrightarrow \mathcal{M}, $$
            and a right $\mathcal{B}$-module structure given by a morphism of sheaves of sets
            $$ \rho: \mathcal{M} \times \mathcal{B} \longrightarrow \mathcal{M}, $$
            such that the actions are compatible. Specifically, for every object $U$ in $\mathcal{C}$, every section $m \in \mathcal{M}(U)$, every $a \in \mathcal{A}(U)$, and every $b \in \mathcal{B}(U)$, the equality
            $$ \lambda_U(a, \rho_U(m, b)) = \rho_U(\lambda_U(a, m), b) $$
            holds in $\mathcal{M}(U)$. In standard multiplicative notation where $\lambda(a,m)$ is denoted $a \cdot m$ and $\rho(m,b)$ is denoted $m \cdot b$, this condition is the associativity axiom
            $$ (a \cdot m) \cdot b = a \cdot (m \cdot b). $$

            In particular, for every object $U \in \calC$, the abelian group $\calM(U)$ has the structure of an \CrefAndHyperrefIfExist{definition:module_of_a_ring}{$\calA(U)-\calB(U)$-bimodule}.

            \item Let $\mathcal{M}$ and $\mathcal{N}$ be $(\mathcal{A}, \mathcal{B})$-bimodules. A \hldef{homomorphism of $(\mathcal{A}, \mathcal{B})$-bimodules} (or an \hldef{$(\mathcal{A}, \mathcal{B})$-linear morphism}) is a morphism of sheaves of abelian groups $f: \mathcal{M} \to \mathcal{N}$ such that for every object $U$ of $\mathcal{C}$, every section $m \in \mathcal{M}(U)$, every $a \in \mathcal{A}(U)$, and every $b \in \mathcal{B}(U)$, the following compatibility conditions hold:
            $$ f_U(a \cdot m) = a \cdot f_U(m) \quad \text{and} \quad f_U(m \cdot b) = f_U(m) \cdot b. $$


        \end{enumerate}

        \noindent We denote the category of $(\mathcal{A}, \mathcal{B})$-bimodules, with morphisms being morphisms of sheaves of abelian groups compatible with both the left $\mathcal{A}$-action and the right $\mathcal{B}$-action, by
        \hl{$ \mathcal{A}\text{-}\mathcal{B}\text{-}\mathsf{Mod} $}
        or sometimes by
        \hl{$ {}_{\mathcal{A}}\mathsf{Mod}_{\mathcal{B}} $}
        \TODO{talk about how bimodules can be identifies with left/right modules}

        \item 

        Let $(\mathcal{C}, J)$ be a \CrefAndHyperrefIfExist{definition:grothendieck_topology_on_a_category_site_covering_sieve_topologically_generating_family}{site}. Let $\mathcal{O}$ be a \CrefAndHyperrefIfExist{definition:sheaf_on_a_site}{sheaf of (not necessarily commutative) rings on $(\mathcal{C}, J)$}, i.e. $((\calC, J), \calO)$ is a \CrefAndHyperrefIfExist{definition:ringed_site}{ringed site}.  

        \begin{enumerate}
            \item An \hldef{(left/right/two-sided) $\mathcal{O}$-module} consists of the following data:
            \begin{itemize}
                \item A sheaf $\mathcal{F}$ of abelian groups on $(\mathcal{C}, J)$,
            \item for every object $U \in \mathcal{C}$, the structure of an (left/right/two-sided) $\mathcal{O}(U)$-module on $\mathcal{F}(U)$,
            \end{itemize}
            such that for every morphism $f: V \to U$ in $\mathcal{C}$, the restriction map 
            $$\rho_{U,V}: \mathcal{F}(U) \to \mathcal{F}(V)$$ 
            is $\mathcal{O}(U)$-linear when the $\mathcal{O}(U)$-action on $\mathcal{F}(V)$ is defined via the natural ring homomorphism 
            $$\mathcal{O}(U) \to \mathcal{O}(V)$$
            induced by $f$.


            \item Let $\mathcal{F}$ and $\mathcal{G}$ be \CrefAndHyperrefIfExist{definition:module_over_a_sheaf_of_rings_on_a_site}{$\mathcal{O}$-modules}.

            A \hldef{morphism of $\mathcal{O}$-modules} $\varphi: \mathcal{F} \to \mathcal{G}$ is a \CrefAndHyperrefIfExist{definition:sheaf_on_a_site}{morphism of sheaves} of abelian groups such that, for every object $U \in \mathcal{C}$, the component map
            $$\varphi_U : \mathcal{F}(U) \to \mathcal{G}(U)$$
            is $\mathcal{O}(U)$-linear, i.e. it satisfies
            $$\varphi_U(r \cdot s) = r \cdot \varphi_U(s) \quad \text{for all } r \in \mathcal{O}(U), \, s \in \mathcal{F}(U).$$

            The collection of all $\mathcal{O}$-modules together with their morphisms of $\mathcal{O}$-modules forms the \hldef{category of $\mathcal{O}$-modules}, denoted \hl{$\mathbf{Mod}(\mathcal{O})$}.

            \TextIfExists{definition:algebra_over_a_sheaf_of_rings_on_a_site}{See also \Cref{definition:algebra_over_a_sheaf_of_rings_on_a_site}.}
        \end{enumerate}

        \noindent In case that a \CrefAndHyperrefIfExist{definition:sheafification_functor_on_a_site}{sheafification functor} 
        $$\PreShv(\calC, \mathbf{Rings}) \to \Shv(\calC, \mathbf{Rings})$$ 
        exists, a left, right, two-sided $\calO$-module (and morphisms thereof) is equivalent to a $(\calO,\bbZ)$-bimodule, $(\bbZ,\calO)$-bimodule, and $(\calO, \calO)$-bimodule (and morphisms thereof) respectively, where $\bbZ$ is the \CrefAndHyperrefIfExist{definition:constant_sheaf_on_a_site_with_sheafification}{constant sheaf} of the integer ring $\bbZ$.

\end{enumerate}


\end{definition}


% See Also
% theorem:category_of_modules_over_a_sheaf_of_rings_on_a_site_on_an_essentially_small_category_has_enough_injectives

\section{Miscellaneous definitions}

\begin{definition} \label{definition:function_of_sets}
Let $X$ and $Y$ be sets. A \hldef{map} (or \hldef{function}) from $X$ to $Y$ is a rule $f$ assigning to each element $x \in X$ exactly one element $f(x) \in Y$. We write \hl{$f : X \to Y$}.

We say that $X$ is the \hldef{domain} and that $Y$ is the \hldef{codomain of $f$}.

% Given any sets $X$ and $Y$, the collection of maps $X \to Y$ is a set; the collection of all sets along with maps between them form a \CrefAndHyperrefIfExist{definition:locally_small_category}{locally small} \CrefAndHyperrefIfExist{definition:category}{category}, usually called the \hldef{category of sets}, and often denoted by notations such as \hl{$\mathrm{Set}$}, \hl{$\mathbf{Set}$}, \hl{$\mathrm{Sets}$}, \hl{$\mathbf{Sets}$}, \hl{$(\mathrm{Set})$}, \hl{$(\mathbf{Set})$}, \hl{$(\mathrm{Sets})$}, \hl{$(\mathbf{Sets})$}.

\end{definition}


\begin{definition}[Monoid] \label{definition:monoid}
A \hldef{monoid} is a \CrefAndHyperrefIfExist{definition:semigroup}{semigroup} $(M,\cdot)$ such that there exists an element $e \in M$, called the \hldef{identity element}, with the property:
$$ e \cdot a = a \cdot e = a \quad \text{for all } a \in M. $$

\TextIfExists{definition:monoid_object_in_a_category_with_a_final_object}{Equivalently, a monoid is a \CrefAndHyperrefIfExist{definition:monoid_object_in_a_category_with_a_final_object}{monoid object} in the \CrefAndHyperrefIfExist{definition:category_of_sets}{category of sets}.}
\end{definition}


\begin{definition}[Groups] \label{definition:group}
A \hldef{group} is a pair $(G,\cdot)$ where $G$ is a set and $\cdot : G \times G \to G$ is a binary operation, subject to the following conditions:

1. (Associativity) For all $g,h,k \in G$ one has 
$$ (g \cdot h) \cdot k = g \cdot (h \cdot k). $$

2. (Identity element) There exists an element \hl{$e \in G$} such that for all $g \in G$, 
$$ e \cdot g = g \cdot e = g. $$

3. (Inverse element) For all $g \in G$ there exists an element \hl{$g^{-1} \in G$} such that 
$$ g \cdot g^{-1} = g^{-1} \cdot g = e. $$

The element $e$ is called the \hldef{identity element of $G$}, and $g^{-1}$ is called the \hldef{inverse of $g$}.

Equivalently, a group is a \CrefAndHyperrefIfExist{definition:monoid}{monoid} with inverse elements.

\TextIfExists{definition:group_object_in_a_category_with_a_final_object}{Equivalently, a group is a \CrefAndHyperrefIfExist{definition:group_object_in_a_category_with_a_final_object}{group object} in the \CrefAndHyperrefIfExist{definition:category_of_sets}{category of sets}.}

A group $(G, \cdot)$ is often simply written as $G$, when the notation for the binary operation $\cdot$ is clear. 

An \hldef{abelian group} or synonymously, a \hldef{commutative group}, is a group $(G,\cdot)$ whose binary operation $\cdot$ is \CrefAndHyperrefIfExist{definition:commutative_binary_operation}{\hldef{abelian} or \hldef{commutative}}, i.e. satisfies 
$$g \cdot h = h \cdot g$$
for all $g,h \in G$. 


\TextIfExists{definition:module_of_a_ring}{An abelian group is equivalent to a $\bbZ$-module.}

\end{definition}



\begin{definition}[Group homomorphism] \label{definition:group_homomorphism}
Let $(G,\cdot)$ and $(H,\ast)$ be \CrefAndHyperrefIfExist{definition:group}{groups}. A map $f : G \to H$ is called a \hldef{group homomorphism} if for all $g_{1},g_{2} \in G$ one has
$$ f(g_{1} \cdot g_{2}) = f(g_{1}) \ast f(g_{2}). $$

The collection of all groups with the group homomorphisms forms a \CrefAndHyperrefIfExist{definition:locally_small_category}{locally small} \CrefAndHyperrefIfExist{definition:category}{category}, called the \hldef{category of groups}.

If $f$ is \CrefAndHyperrefIfExist{definition:injective_surjective_bijective_map_of_sets}{bijective}, then $f$ is called a \hldef{group isomorphism}. 
\TextIfExists{definition:isomorphism_in_a_category}{Equivalently, a group isomorphism is an \CrefAndHyperrefIfExist{definition:isomorphism_in_a_category}{isomorphism} in the category of groups}.
\end{definition}

\begin{definition}[Topology] \label{definition:topological_space}
Let $X$ be a set. A \hldef{topology on $X$} is a collection $\mathcal{T}$ of subsets of $X$ such that:
\begin{enumerate}
    \item $\emptyset \in \mathcal{T}$ and $X \in \mathcal{T}$,
    \item For any collection $\{ U_i \}_{i \in I} \subseteq \mathcal{T}$ (with $I$ arbitrary), the union $\bigcup_{i \in I} U_i \in \mathcal{T}$,
    \item For any finite collection $\{ U_1, \ldots, U_n \} \subseteq \mathcal{T}$, the intersection $U_1 \cap \cdots \cap U_n \in \mathcal{T}$.
\end{enumerate}
If $\mathcal{T}$ is a topology on $X$, the pair $(X, \mathcal{T})$ is called a \hldef{topological space}. Members of $\mathcal{T}$ are called \hldef{open sets}. 

A subset $C \subseteq X$ is \hldef{closed} if its complement $X \setminus C$ is an open set in $\mathcal{T}$

One very often refers to $X$ as a topological spcae, omitting the notation of the topology $\mathcal{T}$. 

The collection of all topologies on a set $X$ may be denoted by notations such as \hl{$\mathrm{Top}(X)$}, \hl{$\mathbf{Top}(X)$}, or \hl{$\mathsf{Top}(X)$}.
\end{definition}





\begin{definition} \label{definition:continuous_map_of_topological_spaces}
Let $(X,\mathcal{T}_X)$ and $(Y,\mathcal{T}_Y)$ be \CrefAndHyperrefIfExist{definition:topological_space}{topological spaces}. A map $f : X \to Y$ is called \hldef{continuous} if for every open set $V \in \mathcal{T}_Y$, the preimage $f^{-1}(V)$ is an open set in $X$, that is,
$$\hlin{\forall V \in \mathcal{T}_Y, \; f^{-1}(V) \in \mathcal{T}_X.}$$
Equivalently, $f$ is continuous if and only if for every closed set $C \subseteq Y$, the preimage $f^{-1}(C)$ is closed in $X$. 


A \hldef{map of topological spaces} usually refers to a continuous map between the topological spaces.

The set of continuous maps from $X$ to $Y$ is sometimes denoted by \hl{$C(X,Y)$}. Other standard notation include \hl{$\operatorname{Hom}_{\mathrm{Top}}(X,Y)$} or \hl{$\operatorname{Top}(X,Y)$} coming from more general notation for morphisms between objects in a \CrefAndHyperrefIfExist{definition:category}{category}.

% The collection of topological spaces along with continuous maps form a \CrefAndHyperrefIfExist{definition:locally_small_category}{locally small} \CrefAndHyperrefIfExist{definition:category}{category}, usually called the \hldef{category of topological spaces} and often denoted by notations such as $\mathrm{Top}$, $\mathbf{Top}$, etc. 

% The set of continuous maps from $X$ to $Y$ is sometimes denoted by \hl{$C(X,Y)$}. Other standard notation include \hl{$\operatorname{Hom}_{\mathrm{Top}}(X,Y)$} or \hl{$\operatorname{Top}(X,Y)$} coming from more general notation for morphisms between objects in a category.
\end{definition}


\begin{definition}\label{definition:ring}
    A \hldef{ring} is a triple $(R, +, \cdot)$ where 
    \begin{enumerate}
        \item $(R,+)$ is a \CrefAndHyperrefIfExist{definition:group}{commutative group}, and
        \item $(R, \cdot)$ is a \CrefAndHyperrefIfExist{definition:monoid}{monoid}. 
        \item $\cdot$ is distributive over $+$, i.e. for all $a,b,c \in R$, we have
        $$a \cdot (b+c) = a \cdot b + a \cdot c \quad \text{and} \quad (a+b) \cdot c = a \cdot c + b \cdot c.$$
    \end{enumerate}

    Equivalently, a ring is a triple $(R,+,\cdot)$ where $+,\cdot: R \times R \to R$ are binary operations satisfying
    \begin{enumerate}
        \item $(a+b)+c = a+(b+c)$ and $(ab)c = a(bc)$ for all $a,b,c \in R$
        \item There exists an element \hl{$0 \in R$} such that $a+0 = a = 0 + a$ for all $a \in R$.
        \item For every $a \in R$, there exists an element \hl{$-a \in R$} such that $a+(-a) = 0 = (-a) + a$ for all $a \in R$.
        \item There exists an element \hl{$1 \in R$} such that $a \cdot 1 = a = 1 \cdot a$ for all $a \in R$.
        \item For all $a,b,c \in R$, we have
        $$a \cdot (b+c) = a \cdot b + a \cdot c \quad \text{and} \quad (a+b) \cdot c = a \cdot c + b \cdot c.$$
    \end{enumerate} 

    The operation $+$ is often called \hldef{addition} and the operation $\cdot$ is often called \hldef{multiplication}. Accordingly, the identity element $0$ of $+$ is often called the \hldef{additive identity} and the identity element $1$ of $\cdot$ is often called the \hldef{multiplicative identity}.

    % If $\cdot$ is additionally a \CrefAndHyperrefIfExist{definition:commutative_binary_operation}{commutative operation}, i.e. $a \cdot b = b \cdot a$ for all $a,b \in R$, then we call the ring \hldef{commutative}.  


\end{definition}
\begin{remark}
    Some writers might not require a ring to have a multiplicative identity element, i.e. would define a ring so that $(R,+)$ is a commutative group, $(R, \cdot)$ is a semigroup, and $\cdot$ is distributive over $+$. Such writers would call the notion of ring in \Cref{definition:ring} a \hldef{unitary ring} to emphasize the existence of the multiplicative identity $1$. 
\end{remark}


\begin{definition} \label{definition:commutative_ring}
   A \hldef{commutative (unital) ring} is a \CrefAndHyperrefIfExist{definition:ring}{ring} $(R, +, \cdot)$ such that $\cdot$ is a \CrefAndHyperrefIfExist{definition:commutative_binary_operation}{commutative operation}, i.e. $a \cdot b = b \cdot a$. 

   For many writers (e.g. ``commutative'' algebraists or number theorists), a \hldef{ring} refers to a commutative ring as above.
\end{definition}


\begin{definition} \label{definition:unit_of_a_ring}
    Let $(R,+,\cdot)$ be a \CrefAndHyperrefIfExist{definition:ring}{not-necessarily commutative ring}. A \hldef{unit} or \hldef{invertible element of $R$} is an element $u \in R$ such that there exist an element $v \in R$ such that 
    $$uv = 1 = vu.$$ 
    Such an element $v$ is called the \hldef{multiplicative inverse of $u$} and is often denoted by \hl{$u^{-1}$}. If it exists, then it is unique.
    
    The set of units of $R$ forms a \CrefAndHyperrefIfExist{definition:group}{group}, often denoted by \hl{$R^\times$} or \hl{$R^*$}, under the multiplication operation $\cdot$. It is called the \hldef{group of units} or \hldef{unit group} of $R$.
\end{definition}


\begin{definition} \label{definition:division_ring}
    Let $(R,+,\cdot)$ be a \CrefAndHyperrefIfExist{definition:ring}{not-necessarily commutative ring}. It is called a \hldef{division ring}, a \hldef{skew field}, or an \hldef{sfield}, if it is a \CrefAndHyperrefIfExist{definition:zero_ring}{nontrivial ring} in which every nonzero element $a \in R$ is a \CrefAndHyperrefIfExist{definition:unit_of_a_ring}{unit}. 
\end{definition}

\begin{definition}[Field] \label{definition:field}
A \hldef{field} is commutative \CrefAndHyperrefIfExist{definition:division_ring}{division} \CrefAndHyperrefIfExist{definition:commutative_ring}{ring}. In other words, a field is a commutative ring for which all nonzero elements have a \CrefAndHyperrefIfExist{definition:unit_of_a_ring}{multiplicative inverse}.
\end{definition}




\begin{definition} \label{definition:ring_homomorphism}
Let $(R,+,\cdot)$ and $(S,+,\cdot)$ be \CrefAndHyperrefIfExist{definition:ring}{rings}, not assumed to be commutative. A function $f: R \to S$ is called a \hldef{ring homomorphism} if for all $r_1,r_2 \in R$ the following properties hold:
\begin{enumerate}
    \item $f(r_1 + r_2) = f(r_1) + f(r_2)$,
    \item $f(r_1 r_2) = f(r_1) f(r_2)$,
    \item $f(1_R) = 1_S$ where $1_R$ and $1_S$ denote the multiplicative identities in $R$ and $S$, respectively.
\end{enumerate}
A ring homomorphism is said to be a \hldef{ring isomorphism} if it is invertible as a map of sets.

An \hldef{$R$-ring} refers to a ring $S$ equipped with a ring homomorphism $f: R \to S$. 

We note that a ring homomorphism $f: R \to S$ yields a natural \CrefAndHyperrefIfExist{definition:module_of_a_ring}{left $R$-module} structure on $S$ and a natural right $R$-module structure on $S$ respectively as follows for $r \in R$ and $s \in S$:
$$r \cdot s = f(r) \cdot s$$
$$s \cdot r = s \cdot f(r).$$
However, these left and right module structures need not yield a two-sided $R$--module structure.
% A ring $S$ equipped with a ring homomorphism $f: R \to S$ is called an \hldef{$R$-algebra}.
\end{definition}


\begin{definition}[Vector space over a field] \label{definition:vector_space_over_a_field}
    Let $(k,+,\cdot)$ be a \CrefAndHyperrefIfExist{definition:field}{field}. A \hldef{vector space over $k$} or a \hldef{$k$-vector space} is a triple $(V,+,\cdot)$\footnote{Note that $+$ and $\cdot$ are abuse of notation here as these are already used for the addition and multiplication of $\cdot$} where 
    \begin{enumerate}
        \item $(V,+)$ is an abelian group, and
        \item $\cdot$ is a map $k \times V \to V$, called \hldef{scalar multiplication}
    \end{enumerate}
    such that the following axioms hold for all $a,b \in k$ and all $u,v \in V$:

    1. (Compatibility with field multiplication)  
    $$ (ab)\cdot v = a \cdot (b \cdot v). $$

    2. (Identity scalar)  
    $$ 1 \cdot v = v. $$

    3. (Distributivity over vector addition)  
    $$ a \cdot (u+v) = a \cdot u + a \cdot v. $$

    4. (Distributivity over scalar addition)  
    $$ (a+b) \cdot v = a \cdot v + b \cdot v. $$
\end{definition}

\begin{definition} \label{definition:morphism_of_vector_spaces}
Let $F$ be a \CrefAndHyperrefIfExist{definition:field}{field}, and let $V$ and $W$ be \CrefAndHyperrefIfExist{definition:vector_space_over_a_field}{$F$-vector spaces}.
A function $T : V \to W$ is called a \hldef{(homo)morphism of vector spaces over $F$}, or an \hldef{$F$-linear map}, if for all $u,v \in V$ and all $a,b \in F$, we have
$$ T(au + bv) = aT(u) + bT(v).  $$
The set of all such morphisms from $V$ to $W$ is denoted by
$$\hlin{\operatorname{Hom}_F(V,W)}.$$
\end{definition}
\begin{definition} \label{definition:basis_of_a_vector_space_over_a_field}
Let $F$ be a \CrefAndHyperrefIfExist{definition:field}{field}, and let $V$ be an \CrefAndHyperrefIfExist{definition:vector_space_over_a_field}{$F$-vector space}.
A subset $B \subseteq V$ is called a \hldef{basis of $V$} if: (i) $B$ is \CrefAndHyperrefIfExist{definition:linearly_independent_elements_of_a_module_over_a_ring}{linearly independent} over $F$, and (ii) $B$ \CrefAndHyperrefIfExist{definition:span_a_module_over_a_ring_for_elements_of_the_module}{spans} $V$.

If $B$ is a basis, we define the \hldef{dimension of $V$ over $F$} (or \hldef{rank of $V$ over $F$}), denoted by
$$ \hlin{\dim_F(V)}, $$
\TODO{cardinality}
to be the cardinality of $B$.
This value is uniquely determined by $V$ and $F$.
\end{definition}

\begin{definition}[Partially ordered set] \label{definition:partially_ordered_set}
    \begin{enumerate}
        \item 
        A \hldef{partially ordered set} (or \hldef{poset}), or \hldef{ordered set} is a pair $(P, \leq)$ where $P$ is a set and 
        \[
        \leq : P \times P \to \{\text{true}, \text{false}\}
        \]
        is a binary relation on $P$ satisfying the following axioms for all $a,b,c \in P$:
        \begin{itemize}
            \item \hldef{Reflexivity:} $a \leq a$,
            \item \hldef{Antisymmetry:} if $a \leq b$ and $b \leq a$, then $a = b$,
            \item \hldef{Transitivity:} if $a \leq b$ and $b \leq c$, then $a \leq c$.
        \end{itemize}
        The relation $\leq$ is called an \hldef{order} or a \hldef{partial order}

        \item A partially ordered set $(P, \leq)$ is called a \hldef{directed partially ordered set} if for every pair $a,b \in P$, there exists $c \in P$ such that
        \[
        a \leq c \quad \text{and} \quad b \leq c.
        \]

        \item A partially ordered set $(P, \leq)$ is called a \hldef{codirected partially ordered set} (or \hldef{downward directed poset}) if for every pair $a,b \in P$, there exists $d \in P$ such that
        \[
        d \leq a \quad \text{and} \quad d \leq b.
        \]
        \end{enumerate}
\end{definition}

\begin{lemma} \label{lemma:posets_correspond_to_small_filtered_thin_categories}
    Let $(P, \leq)$ be a nonempty \hyperrefIfExists{definition:partially_ordered_set}{poset}\CrefIfExists{definition:partially_ordered_set}. 
    \begin{enumerate}
        \item Regarding $P$ as a category whose objects are the elements of $P$ and such that there is a unique arrow $a \to b$ if and only if $a \leq b$, the category is filtered. 
        \item Every nonempty \hyperrefIfExists{definition:locally_small_category}{small}\CrefIfExists{definition:locally_small_category} \hyperrefIfExists{definition:thin_category}{thin}\CrefIfExists{definition:thin_category} \hyperrefIfExists{definition:filtered_cofiltered_category}{filtered category}\CrefIfExists{definition:filtered_cofiltered_category} corresponds to a poset in this way.
        \item Moreover, the poset $P$ is \hyperrefIfExists{definition:partially_ordered_set}{directed}\CrefIfExists{definition:partially_ordered_set} if and only if the category is filtered. The poset $P$ is \hyperrefIfExists{definition:partially_ordered_set}{codirected}\CrefIfExists{definition:partially_ordered_set} if and only if the category is cofiltered.
    \end{enumerate}
\end{lemma}


\begin{definition} \label{definition:free_group_generated_by_a_set}
Let $S$ be a set. The \hldef{free group generated by $S$} is a pair $(F(S), \iota)$ consisting of a \CrefAndHyperrefIfExist{definition:group}{group} $F(S)$ and a function $\iota: S \to F(S)$, satisfying the following universal property: for any group $G$ and any function $f: S \to G$, there exists a unique group homomorphism $\varphi: F(S) \to G$ such that the diagram commutes (i.e., $\varphi \circ \iota = f$).
The standard notation for the free group on $S$ is
\hl{$F(S)$}
or sometimes
\hl{$\langle S \rangle$}
Elements of $F(S)$ are uniquely represented as reduced words in the alphabet $S \cup \{s^{-1} \mid s \in S\}$.
\end{definition}

\begin{definition} \label{definition:free_abelian_group_generated_by_a_set}
Let $S$ be a set. The \hldef{free abelian group generated by $S$} is the abelian group consisting of all formal linear combinations of elements of $S$ with integer coefficients, such that only finitely many coefficients are nonzero.
This group is denoted by
\hl{$\mathbb{Z}[S]$}
or alternatively as the direct sum
$$\hlin{\mathbb{Z}^{(S)} := \bigoplus_{s \in S} \mathbb{Z} s}$$
It satisfies the universal property that for any abelian group $A$ and any function $f: S \to A$, there exists a unique group homomorphism $\psi: \mathbb{Z}[S] \to A$ extending $f$.
\end{definition}

\begin{definition}[Homotopy groups] \label{definition:homotopy_groups_of_a_pointed_topological_space}
For any \CrefAndHyperrefIfExist{definition:pointed_topological_space}{pointed topological space} $(X,x_0)$ and integer $n \ge 0$, the \hldef{$n$-th homotopy group of $X$ at $x_0$}, denoted \hl{$\pi_n(X,x_0)$}, is defined as the set of all \CrefAndHyperrefIfExist{definition:homotopy_class_of_maps_of_topological_spaces_relative_to_a_subset}{homotopy classes (rel.\ $\partial I^n$)} of based maps
$$f : (I^n, \partial I^n) \to (X,x_0),$$
where $I^n = [0,1]^n$. For $n \ge 1$, $\pi_n(X,x_0)$ is a group under concatenation of based maps, and for $n \ge 2$, it is abelian.

The \hldef{fundamental group of $(X,x_0)$} refers to $\pi_1(X,x_0)$. Equivalently, it is the group of homotopy classes (rel.\ endpoints) of \CrefAndHyperrefIfExist{definition:path_and_loop_in_a_topological_space}{loops} $\gamma : [0,1] \to X$ satisfying $\gamma(0)=\gamma(1)=x_0$. 
\end{definition}



\begin{definition} \label{definition:prime_and_maximal_ideal_of_a_ring}
Let $R$ be a \CrefAndHyperrefIfExist{definition:ring}{(not necessarily commutative) ring}. A \CrefAndHyperrefIfExist{definition:ideal_of_a_ring}{proper two-sided ideal $P \unlhd R$} is called a \hldef{prime ideal} if the following equivalent conditions holds:
\begin{enumerate}
    \item If $I,J$ are left ideals and \CrefAndHyperrefIfExist{definition:product_of_ideals_of_a_ring}{$IJ \subset P$}, then $I \subset P$ or $J \subset P$.
    \item If $I,J$ are right ideals and $IJ \subset P$, then $I \subset P$ or $J \subset P$.
    \item If $I,J$ are two-sided idaels and $IJ \subset P$, then $I \subset P$ or $J \subset P$.
    \item If $x,y \in R$ with $xRy \subset \mathfrak{p}$, then $x \in \mathfrak{p}$ or $y \in \mathfrak{p}$.
\end{enumerate}

A proper left/right/two-sided ideal $M \subsetneq R$ is called \hldef{maximal} if there exists no other left/right/two-sided ideal $J \unlhd R$ such that $M \subsetneq J \subsetneq R$. Equivalently, 
\begin{itemize}
    \item a left/right ideal $M$ of $R$ is maximal if and only if the \CrefAndHyperrefIfExist{definition:quotient_module_of_a_module_by_a_module_of_a_ring}{quotient module $R/M$} is a \CrefAndHyperrefIfExist{definition:simple_module_of_a_ring}{simple} left/right $R$-module.
    \item a two-sided ideal $M$ of $R$ is maximal if and only if the \CrefAndHyperrefIfExist{definition:quotient_ring_of_a_ring_by_a_two_sided_ideal}{quotient ring $R/M$} is a \CrefAndHyperrefIfExist{definition:simple_ring}{simple ring}. 
\end{itemize}
\end{definition}

\begin{definition}[Affine scheme] \label{definition:affine_scheme}
Let $A$ be a \CrefAndHyperrefIfExist{definition:commutative_ring}{commutative ring with unity}. Define the set \hl{$\mathrm{Spec}(A)$}
to be the set of all \CrefAndHyperrefIfExist{definition:prime_and_maximal_ideal_of_a_ring}{prime ideals} of $A$. Equip it with the \hldef{Zariski topology}, which is the \CrefAndHyperrefIfExist{definition:topological_space}{topology} whose closed sets are given by \hldef{vanishing loci}
$$\hlin{V(I) = \{\mathfrak{p} \in \mathrm{Spec}(A) : I \subseteq \mathfrak{p}\}}$$
for ideals $I \subseteq A$.  
Define the sheaf \hl{$\mathcal{O}_{\mathrm{Spec}(A)}$}, called the \hldef{structure sheaf of $\Spec A$}, by 
$$\mathcal{O}_{\mathrm{Spec}(A)}(U) = \{ \ \text{locally defined fractions of elements of $A$ on $U$} \ \},$$
for each open set $U \subseteq \mathrm{Spec}(A)$. It is the case that the stalk at $\mathfrak{p} \in \mathrm{Spec}(A)$ is canonically the \CrefAndHyperrefIfExist{definition:localization_of_a_commutative_ring_by_a_multiplicative_subset}{localization $A_{\mathfrak{p}}$}.  
Then $(\mathrm{Spec}(A), \mathcal{O}_{\mathrm{Spec}(A)})$ is a \CrefAndHyperrefIfExist{definition:locally_ringed_space_on_a_topological_space}{locally ringed space}, called the \hldef{affine scheme associated to $A$}.

Moreover, given $f \in A$, we define the locus \hl{$D(f)$} by 
$$D(f) = \Spec A \setminus V((f)) = \{ \mathfrak{p} \in \operatorname{Spec} A : f \notin \mathfrak{p} \}$$
\end{definition}

\begin{definition} \label{definition:cartesian_product_of_two_objects_in_a_category_over_an_object}
    Let $\mathcal{C}$ be a \CrefAndHyperrefIfExist{definition:category}{category}, let $Z$ be an object, and let $X, Y$ be objects of $\mathcal{C}$ \CrefAndHyperrefIfExist{definition:category_of_objects_over_under_a_fixed_object_in_a_category}{over} $Z$, i.e. morphisms $X \to Z$ and $Y \to Z$ are fixed. A \hldef{cartesian product of $X$ and $Y$ over $Z$ in $\mathcal{C}$} (or \hldef{fiber product} or \hldef{pullback diagram}) is an object, often denoted by \hl{$X \times_Z Y$}, with \hldef{projection morphisms} $X \times_Z Y \to X$ and $X \times_Z Y \to Y$ that are universal. 
    More precisely, for any object $T$ of $\mathcal{C}$ and morphisms $f_X : T \to X$, $f_Y : T \to Y$, there exists a unique morphism $u : T \to X \times_Z Y$ such that the following diagram commutes:
        \begin{center}
        \begin{tikzcd}
            T \ar[rd, dotted, "u" ] \ar[rrd, "f_X", bend left] \ar[ddr, "f_Y", bend right] & & \\
            & X \times_Z Y \ar[r] \ar[d] &  \ar[d] X \\
            & Y \ar[r] & Z
        \end{tikzcd}
        \end{center}
        Equivalently, $X \times_Z Y$ is the \CrefAndHyperrefIfExist{definition:limit_and_colimit_of_a_diagram_in_a_category}{limit} of the \CrefAndHyperrefIfExist{definition:diagram_in_a_category_indexed_by_a_small_category}{diagram}
        \begin{center}
            \begin{tikzcd}
            & X \ar[d] \\
            Y \ar[r] & Z
            \end{tikzcd}
        \end{center}
        in $\calC$. 

        The commutative diagram 
        \begin{center}
        \begin{tikzcd}
        X \times_Z Y \ar[r] \ar[d] & X \ar[d] \\
        Y \ar[r] & Z
        \end{tikzcd} 
        \end{center}
        may be referred to as a \hldef{cartesian square}.

\end{definition}

\begin{definition} \label{definition:semigroup_object_in_a_category}
    Let $\mathcal{C}$ be a \CrefAndHyperrefIfExist{definition:category}{(large) category}. 
    A \hldef{semigroup object in $\mathcal{C}$} is an object $A \in \mathcal{C}$ such that the \CrefAndHyperrefIfExist{definition:product_and_coproduct_of_objects_in_a_category}{product} $A \times A$ exists in $\calC$ together with a morphism
    $$ \mu : A \times A \to A, $$

    called the \hldef{multiplication morphism} such that the associativity diagram
    \begin{center}
    \begin{tikzcd}
    A \times A \times A
    \arrow[r, "\mu \times \mathrm{id}_A"]
    \arrow[d, "\mathrm{id}_A \times \mu"']
    & A \times A
    \arrow[d, "\mu"]
    \\
    A \times A
    \arrow[r, "\mu"']
    & A
    \end{tikzcd}
    \end{center}
    commutes.

    The semigroup object structure $(A,\mu,\eta, \iota)$ is said to be \hldef{abelian} or \hldef{commutative} if the morphisms $\mu: A \times A \to A$ and $\mu \circ \tau_{A,A}: A \times A \to A$ coincide, where $\tau_{A,A}: A \times A \to A \times A$ is the symmetry morphism swapping the two factors.
\end{definition}
\begin{definition} \label{definition:monoid_object_in_a_category_with_a_final_object}
    Let $\mathcal{C}$ be a \CrefAndHyperrefIfExist{definition:category}{(large) category} with a \CrefAndHyperrefIfExist{definition:initial_final_zero_objects_of_a_category}{final object}. 
    A \hldef{monoid object in $\mathcal{C}$} is a \CrefAndHyperrefIfExist{definition:semigroup_object_in_a_category}{semigroup  object} $(A,\mu)$ together with a \hldef{unit morphism} 
    $$
    \eta : 1 \to A
    $$
    such that the \CrefAndHyperrefIfExist{definition:product_and_coproduct_of_objects_in_a_category}{products} $1 \times A$ and $A \times 1$ exist and the unit diagrams
    $$
    \begin{tikzcd}
    1 \times A
    \arrow[rr, "\eta \times \mathrm{id}_A"]
    \arrow[dr, "\mathrm{pr}_2"']
    && A \times A
    \arrow[dl, "\mu"]
    \\
    & A &
    \end{tikzcd}
    $$

    $$
    \begin{tikzcd}
    A \times 1
    \arrow[rr, "\mathrm{id}_A \times \eta"]
    \arrow[dr, "\mathrm{pr}_1"']
    && A \times A
    \arrow[dl, "\mu"]
    \\
    & A &
    \end{tikzcd}
    $$

    commute.
\end{definition}
\begin{definition} \label{definition:group_object_in_a_category_with_a_final_object}
    Let $\mathcal{C}$ be a \CrefAndHyperrefIfExist{definition:category}{(large) category} with a \CrefAndHyperrefIfExist{definition:initial_final_zero_objects_of_a_category}{final object}. 
    A \hldef{group object in $\mathcal{C}$} is a \CrefAndHyperrefIfExist{definition:monoid_object_in_a_category_with_a_final_object}{monoid object} $(A,\mu, \eta)$ together with a \hldef{inverse morphism} 
    $$
    \iota : A \to A
    $$
    such that the diagrams
    $$
    \begin{tikzcd}
    A
    \arrow[r, "\Delta"]
    \arrow[dr, "\eta \circ {!}_A"']
    & A \times A
    \arrow[d, "\mu \circ (\mathrm{id}_A \times \iota)"]
    \\
    & A
    \end{tikzcd}
    $$

    $$
    \begin{tikzcd}
    A
    \arrow[r, "\Delta"]
    \arrow[dr, "\eta \circ {!}_A"']
    & A \times A
    \arrow[d, "\mu \circ (\iota \times \mathrm{id}_A)"]
    \\
    & A
    \end{tikzcd}
    $$

    commute, where $\Delta : A \to A \times A$ is the diagonal and ${!}_A : A \to 1$ is the unique morphism.


\end{definition}

\begin{definition}[Power Set] \label{definition:power_set_of_a_set}
Let $A$ be a set. The \hldef{power set of $A$}, denoted by \hl{$\mathcal{P}(A)$}, is the set of all subsets of $A$:
$$ \mathcal{P}(A) := \{\, B \mid B \subseteq A \,\}.  $$
Equivalently, every element of $\mathcal{P}(A)$ is itself a set $B$ satisfying $B \subseteq A$. Under the \CrefAndHyperrefIfExist{definition:zermelo_fraenkel_set_theory}{axiom of power set}, note that the $\calP(A)$ exists.
\end{definition}

\begin{definition} \label{definition:algebra_of_a_ring}
Let $R$ be a \CrefAndHyperrefIfExist{definition:ring}{(not-necessarily commutative) ring with unity}. An \hldef{$R$-algebra} is a ring $A$ together with a \CrefAndHyperrefIfExist{definition:ring_homomorphism}{ring homomorphism} 
$$\varphi : R \to A$$ 
into the \CrefAndHyperrefIfExist{definition:center_of_a_ring}{center $Z(A)$} of $A$ (so that $\varphi(r)$ commutes with every element of $A$ for all $r \in R$), such that $\varphi(1_R) = 1_A$. The ring homomorphism $\varphi$ is called the \hldef{structure map} of the algebra.

Equivalently, an $R$-algebra consists of a ring $A$ endowed with a \CrefAndHyperrefIfExist{definition:module_of_a_ring}{two-sided $R$-module} structure for which the scalar multiplication satisfies
$$ r \cdot (ab) = (r \cdot a) b = a (r \cdot b) \quad \text{for all } r \in R, \, a,b \in A. $$

In particular, any ring homomorphism between \CrefAndHyperrefIfExist{definition:commutative_ring}{commutative rings} specifies an algebra structure.
\end{definition}


\begin{definition}[Homotopy of maps of topological spaces] \label{definition:homotopy_of_maps_of_topological_spaces_relative_to_a_subset}

    Let $X$ and $Y$ be topological spaces and let $K \subseteq X$ be a subset. Let $C(X,Y)$ denote the set of all continuous maps $f : X \to Y$.  

    \begin{enumerate}
        \item A \hldef{homotopy between two maps $f,g \in C(X,Y)$ relative to $K$ } is a continuous map
        $$H : X \times [0,1] \to Y$$
        such that for all $x \in X$,
        $$H(x,0) = f(x), \quad H(x,1) = g(x),$$
        and for all $x \in K$ and $t \in [0,1]$,
        $$H(x,t) = f(x) = g(x).$$

        If such an $H$ exists, we say $f$ and $g$ are \hldef{homotopic relative to $K$}, and we write \hl{$f \simeq g \text{ rel } K$}; this is an equivalence relation.

        A \hldef{homotopy between two maps $f,g \in C(X,Y)$} is simply a homotopy relative to $\emptyset$. We write we write \hl{$f \simeq g$} if a homotopy between them exists.

        \item Let $(X, x_0)$ and $(Y, y_0)$ be \CrefAndHyperrefIfExist{definition:pointed_topological_space}{pointed topological spaces} and let $K \subseteq X$ be a subset with $x_0 \in K$. Let $C_*(X,Y)$ denote the set of all continuous based maps $f : X \to Y$ satisfying $f(x_0) = y_0$.

        A \hldef{homotopy of based maps $f,g \in C_*(X,Y)$ relative to $K$} is a continuous map
        $$H : X \times [0,1] \to Y$$
        such that for all $x \in X$,
        $$H(x,0) = f(x), \quad H(x,1) = g(x),$$
        and for all $k \in K$ and $t \in [0,1]$,
        $$H(k,t) = f(k) = g(k),$$
        in particular fixing the basepoint throughout,
        $$H(x_0, t) = y_0 \quad \text{for all } t \in [0,1].$$

        If such an $H$ exists, we say $f$ and $g$ are \hldef{based homotopic relative to $K$}, and we write \hl{$f \simeq g \text{ rel } K$}. This is an equivalence relation.

        A \hldef{homotopy of based maps $f,g \in C_*(X,Y)$} without relative condition is the special case $K = \{x_0\}$ and is called a \hldef{homotopy of based maps} or \hldef{based homotopy}. We write \hl{$f \simeq g$} if such a homotopy exists.

    \end{enumerate}
\end{definition}
\begin{definition}[Homotopy class of maps relative to a subset] \label{definition:homotopy_class_of_maps_of_topological_spaces_relative_to_a_subset}
    Let $X$ and $Y$ be \CrefAndHyperrefIfExist{definition:topological_space}{topological spaces} and let $K \subseteq X$. Let $C(X,Y)$ denote the set of all \CrefAndHyperrefIfExist{definition:continuous_map_of_topological_spaces}{continuous maps} $f : X \to Y$.  

    \begin{enumerate}
        \item Two maps $f, g \in C(X,Y)$ are said to be in the same \hldef{homotopy class relative to $K$} if there exists a \CrefAndHyperrefIfExist{definition:homotopy_of_maps_of_topological_spaces_relative_to_a_subset}{homotopy relative to $K$}
        $$H : X \times [0,1] \to Y$$
        such that
        $$H(x,0) = f(x), \quad H(x,1) = g(x),$$
        and
        $$H(k,t) = f(k) = g(k) \quad \text{for all } k \in K, t \in [0,1].$$

        The \hldef{homotopy class of maps relative to $K$} containing a map $f: X \to Y$ is denoted by \hl{$[f]_K$}.

        Two maps $f, g \in C(X,Y)$ are said to be in the same \hldef{homotopy class} if they are in the same homotopy class relative to $\emptyset$.

        The \hldef{homotopy class of maps} containing a map $f: X \to Y$ is denoted by \hl{$[f]$}.

        The set of homotopy classes of maps may often be denoted by \hl{$[X,Y]$}.

        \item 
        Let $(X, x_0)$ and $(Y, y_0)$ be \CrefAndHyperrefIfExist{definition:pointed_topological_space}{pointed topological spaces} and let $K \subseteq X$ be a subset containing $x_0$. Let $C_*(X,Y)$ denote the set of all continuous based maps $f : X \to Y$ with $f(x_0) = y_0$.

        Two based maps $f, g \in C_*(X,Y)$ are said to be in the same \hldef{homotopy class relative to $K$} if there exists a homotopy of based maps relative to $K$
        $$H : X \times [0,1] \to Y$$
        such that for all $x \in X$,
        $$H(x,0) = f(x), \quad H(x,1) = g(x),$$
        and for all $k \in K$ and $t \in [0,1]$,
        $$H(k,t) = f(k) = g(k),$$
        particularly ensuring the basepoint is fixed throughout,
        $$H(x_0,t) = y_0 \quad \text{for all } t \in [0,1].$$

        The \hldef{homotopy class relative to $K$} containing $f: (X,x_0) \to (Y,y_0)$ is denoted by \hl{$[f]_K$}.

        Two based maps $f, g \in C_*(X,Y)$ are said to be in the same \hldef{homotopy class} if they are in the same homotopy class relative to $\{x_0\}$.

        The \hldef{homotopy class} containing a map $f:(X,x_0) \to (Y,y_0)$ is denoted by \hl{$[f]$}.

        The set of homotopy classes of pointed maps $(X,x_0) \to (Y,y_0)$ may often be denoted by \hl{$[(X,x_0),(Y,y_0)]$} or by \hl{$[X,Y]$} if the base points are clear.
    \end{enumerate}
\end{definition}
\begin{definition} \label{definition:homotopy_category_of_topological_spaces}
The \hldef{homotopy category of topological spaces}, denoted \hl{$\text{hTop}$}, is the category whose objects are \CrefAndHyperrefIfExist{definition:topological_space}{topological spaces} and whose morphisms are \CrefAndHyperrefIfExist{definition:homotopy_class_of_maps_of_topological_spaces_relative_to_a_subset}{homotopy classes} of \CrefAndHyperrefIfExist{definition:continuous_map_of_topological_spaces}{continuous maps}. In other words, for objects $X$ and $Y$, the set of morphisms is defined as $\text{Hom}_{\text{hTop}}(X, Y) = [X, Y] = C(X, Y) / \simeq$.
\end{definition}
\begin{proposition} \label{proposition:homotopy_classes_of_maps_between_topological_spaces_are_preserved_by_composition}
Composition in the \CrefAndHyperrefIfExist{definition:homotopy_category_of_topological_spaces}{homotopy category of topological spaces} is well-defined. If $f_1, f_2: X \to Y$ are homotopic and $g_1, g_2: Y \to Z$ are homotopic, then the compositions $g_1 \circ f_1$ and $g_2 \circ f_2$ are homotopic as maps from $X$ to $Z$. That is, $[g] \circ [f] = [g \circ f]$ is independent of the choice of representatives.
\end{proposition}
\begin{theorem} \label{theorem:functor_from_category_of_topological_spaces_to_homotopy_category_of_topological_spaces}
There exists a canonical functor $Q: \text{Top} \to \text{hTop}$ which is the identity on objects and maps each continuous map $f$ to its homotopy class $[f]$. This functor is full and essentially surjective.
\end{theorem}

\begin{definition} \label{definition:C_k_morphism_between_C_k_manifolds}
% Let $k \in \mathbb{N}_0 \cup \{\infty\}$ be fixed. Let $(M, \mathcal{A}_M)$ and $(N, \mathcal{A}_N)$ be $C^k$-manifolds of dimensions $n$ and $m$, respectively, where $M,N$ are topological manifolds and $\calA_M$ and $\calA_N$ are $C^k$-atlases.

% A \hldef{$C^k$-morphism} (or \hldef{$C^k$-map}) between $M$ and $N$ is a \CrefAndHyperrefIfExist{definition:continuous_map_of_topological_spaces}{continuous map}
% $$ f : M \to N $$
% such that for every $p \in M$ there exist \hyperrefIfExists{definition:chart_on_a_topological_manifold}{charts} $(U, \varphi) \in \mathcal{A}_M$ with $p \in U$ and $(V, \psi) \in \mathcal{A}_N$ with $f(p) \in V$ satisfying
% $$ \psi \circ f \circ \varphi^{-1} : \varphi(U \cap f^{-1}(V)) \to \psi(V) $$
% is a \CrefAndHyperrefIfExist{definition:C_k_map_between_open_subsets_of_closed_half_spaces_of_Rns}{$C^k$-map} between open subsets of Euclidean spaces $\mathbb{R}^n$ and $\mathbb{R}^m$, i.e.,
% $$ \psi \circ f \circ \varphi^{-1} \in C^k(\varphi(U \cap f^{-1}(V)), \psi(V)).  $$
% If $f$ is a homeomorphism and its inverse $f^{-1} : N \to M$ is also a $C^k$-morphism, then $f$ is called a \hldef{$C^k$-diffeomorphism}.
% We let \hl{$C^k(M,N)$} denote the space of $C^k$-maps $M \to N$.  We let \hl{$C^k(M)$} denote the space of \hldef{$C^k$-functions}, i.e. the $C^k$-maps $M \to \bbR$.

Let $k \in \mathbb{N}_0 \cup \{\infty\}$ be fixed. Let $(M, \mathcal{A}_M)$ and $(N, \mathcal{A}_N)$ be \CrefAndHyperrefIfExist{definition:C_k_manifold}{$C^k$-manifolds with boundary} of dimensions $n$ and $m$, respectively, where $M,N$ are \CrefAndHyperrefIfExist{definition:topological_manifold}{topological manifolds with boundary} and $\calA_M$ and $\calA_N$ are \CrefAndHyperrefIfExist{definition:C_k_atlas_on_a_topological_manifold}{$C^k$-atlases} whose charts map to open subsets of the \CrefAndHyperrefIfExist{definition:closed_half_space_in_euclidean_space}{closed half-spaces} $\mathbb{H}^n$ and $\mathbb{H}^m$.

A \hldef{$C^k$-morphism} (or \hldef{$C^k$-map}) between $M$ and $N$ is a \CrefAndHyperrefIfExist{definition:continuous_map_of_topological_spaces}{continuous map}
$$ f : M \to N $$
such that for every $p \in M$ there exist \hyperrefIfExists{definition:chart_on_a_topological_manifold}{charts} $(U, \varphi) \in \mathcal{A}_M$ with $p \in U$ and $(V, \psi) \in \mathcal{A}_N$ with $f(p) \in V$ satisfying
$$ \psi \circ f \circ \varphi^{-1} : \varphi(U \cap f^{-1}(V)) \to \psi(V) $$
is a \CrefAndHyperrefIfExist{definition:C_k_map_between_open_subsets_of_closed_half_spaces_of_Rns}{$C^k$-map} between open subsets of the closed half-spaces $\mathbb{H}^n$ and $\mathbb{H}^m$, i.e.,
$$ \psi \circ f \circ \varphi^{-1} \in C^k(\varphi(U \cap f^{-1}(V)), \psi(V)).  $$
If $f$ is a homeomorphism and its inverse $f^{-1} : N \to M$ is also a $C^k$-morphism, then $f$ is called a \hldef{$C^k$-diffeomorphism}.
We let \hl{$C^k(M,N)$} denote the space of $C^k$-maps $M \to N$. We let \hl{$C^k(M)$} denote the space of \hldef{$C^k$-functions}, i.e., the $C^k$-maps $M \to \mathbb{R}$.

In particular, we may speak of these notions when $M$ and $N$ are \CrefAndHyperrefIfExist{definition:C_k_manifold}{$C^k$-manifolds without boundary}.

\end{definition}

\begin{remark}
    The notations $C^k(M,N)$ (and $C^k(M)$) agrees with the usual notations $C^k(M,N)$ and $C^k(M)$ in the case that $M$ is an open subset of $\bbR^n$\CrefIfExists{definition:C_k_map_between_open_subsets_of_closed_half_spaces_of_Rns}.
\end{remark}

\begin{definition} \label{definition:C_k_manifold}
Let $k \in \mathbb{N}_0 \cup \{\infty\}$ be fixed. An \hldef{$n$-dimensional $C^k$/$k$-differentiable-(real)manifold with boundary (resp. without boundary)} is a pair $(M, \mathcal{A})$, where $M$ is a \CrefAndHyperrefIfExist{definition:topological_manifold}{topological manifold with boundary (resp. without boundary)} of dimension $n$ and $\mathcal{A}$ is a \CrefAndHyperrefIfExist{definition:C_k_atlas_on_a_topological_manifold}{$C^k$-atlas} on $M$.

The atlas $\mathcal{A}$ is usually taken to be \CrefAndHyperrefIfExist{definition:atlas_on_a_topological_manifold}{maximal} with respect to \CrefAndHyperrefIfExist{definition:C_k_compatible_charts_on_a_topological_manifold}{$C^k$-compatibility}, meaning it contains every $C^k$-chart compatible with all charts in $\mathcal{A}$.

Note that a $C^0$-manifold is simply a \CrefAndHyperrefIfExist{definition:topological_manifold}{topological manifold} and that a $C^\infty$-manifold is synonymously referred to as a \hldef{smooth/differentiable (real) manifold}.
% See convention:C_k_for_k_equal_0_means_continuous_whereas_C_k_for_k_equal_infty_means_smooth
\end{definition}

\begin{definition}[Opposite ring] \label{definition:opposite_ring_of_a_ring}
Let $R = (R, +, \cdot, 0, 1)$ be a \CrefAndHyperrefIfExist{definition:ring}{ring} with addition $+$, multiplication $\cdot$, additive identity $0$, and multiplicative identity $1$ (not necessarily commutative).

The \hldef{opposite ring of $R$}, denoted $R^{\mathrm{op}}$, is the ring with the same underlying set $R$ and the same addition $+$ and additive identity $0$, but with multiplication defined by
$$ r \star s := s \cdot r $$
for all $r, s \in R$.

That is, multiplication in $R^\mathrm{op}$ is the multiplication of $R$ reversed in order.

If $R$ is \CrefAndHyperrefIfExist{definition:commutative_ring}{commutative}, then $R$ and $R^{\op}$ are naturally isomorphic to each other.
\end{definition}
