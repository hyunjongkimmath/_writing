%% Delete this \nocite command invocation to make the references section only list out the bibitems that are actually cited.
\nocite{*}

% \input{../_definitions/}

\section{Basic definitions}

\subsection{Topological manifolds}



\begin{definition}[Topological Manifold]
    \label{definition:topological_manifold}
    \begin{enumerate}
        \item A \hldef{topological manifold of dimension $n$} is a Hausdorff, second-countable \CrefAndHyperrefIfExist{definition:topological_manifold}{topological space} $M$ such that each point $p \in M$ has an \CrefAndHyperrefIfExist{definition:neighborhood_and_neighborhood_basis_of_a_point_in_a_topological_space}{open neighborhood} $U_p \subseteq M$ homeomorphic to an open subset of $\mathbb{R}^n$.  
        That is, there exists a homeomorphism (a \CrefAndHyperrefIfExist{definition:chart_on_a_topological_manifold}{chart})
        $$ \varphi_p : U_p \to V_p \subseteq \mathbb{R}^n, $$
        where $V_p$ is open in $\mathbb{R}^n$.
        Some synonyms of the notion of ``topological manifold'' include: \hldef{manifold} or \hldef{real manifold}.

        \item 
        A \hldef{topological manifold with boundary of dimension $n$} is a \CrefAndHyperrefIfExist{definition:separation_axioms_of_topology}{Hausdorff}, \CrefAndHyperrefIfExist{definition:first_and_second_countable_topological_space}{second-countable} \CrefAndHyperrefIfExist{definition:topological_space}{topological space} $M$ such that each point $p \in M$ has an open neighborhood $U_p \subseteq M$ \CrefAndHyperrefIfExist{definition:homeomorphism_of_topological_spaces}{homeomorphic} to an open subset of either $\mathbb{R}^n$ or the \CrefAndHyperrefIfExist{definition:closed_half_space_in_euclidean_space}{closed half-space}
        $$ \mathbb{H}^n = \{x \in \mathbb{R}^n : x_n \ge 0\}.  $$
        That is, there exists a homeomorphism (chart)
        $$ \varphi_p : U_p \to V_p, $$
        where $V_p$ is open in $\mathbb{R}^n$ or $\mathbb{H}^n$.  
        A point $p \in M$ is called a \hldef{boundary point} if it corresponds under such a chart to a point in $\{x \in \mathbb{H}^n : x_n = 0\}$, and an \hldef{interior point} otherwise.  
        The set of all boundary points of $M$ is denoted by \hl{$\partial M$} and is called the \hldef{boundary of $M$}. Boundary points and an interior points of $M$ coincide with \CrefAndHyperrefIfExist{definition:interior_and_boundary_of_topological_space}{boundary points and interior points} of $M$ as a topological space. Some synonyms of the notion of ``topological manifold with boundary'' include: \hldef{manifold with boundary} or \hldef{real manifold with boundary}
    \end{enumerate}
    By a \hldef{topological manifold without boundary}, we mean a topological manifold. In particular, we may speak of a \hldef{topological manifold with or without boundary}; many properties and attributes can be spoken for both topological manifolds and topological manifolds without boundary. 

    Note that by the above standard definition, every topological manifold is technically a topological manifold with boundary (but not vice versa). Thus, in principal, definitions concerning topological manifolds with boundary should be applicable to topological manifolds without boundary. 
\end{definition}

\begin{definition}[Chart]
    \label{definition:chart_on_a_topological_manifold}
    \begin{enumerate}
        \item A \hldef{(coordinate) chart on a topological manifold $M$ of dimension $n$} is a pair $(U, \varphi)$ where
        \begin{itemize}
            \item $U \subseteq M$ is an open subset;
            \item $\varphi : U \to V \subseteq \mathbb{R}^n$ is a \CrefAndHyperrefIfExist{definition:homeomorphism_of_topological_spaces}{homeomorphism} onto an open subset $V$ of $\mathbb{R}^n$.
        \end{itemize}
        The map $\varphi$ is called a \hldef{coordinate map}, and the image $\varphi(p)$ of a point $p \in U$ gives the \hldef{coordinates of $p$} in this chart.

        \item A \hldef{(coordinate) chart on a topological manifold with boundary $M$ of dimension $n$} is a pair $(U, \varphi)$ where
        \begin{itemize}
            \item $U \subseteq M$ is an open subset;
            \item $\varphi : U \to V \subseteq \mathbb{R}^n \text{ or } \mathbb{H}^n$ is a homeomorphism onto an open subset $V$ of either $\mathbb{R}^n$ or the \CrefAndHyperrefIfExist{definition:closed_half_space_in_euclidean_space}{closed half-space $\mathbb{H}^n = \{x \in \mathbb{R}^n : x_n \ge 0\}$}; equivalently, we may just specify $\varphi$ to be a homeomorphism onto an open subet of $\bbH^n$.
        \end{itemize}
        The map $\varphi$ is called a \hldef{coordinate map}, and the image $\varphi(p)$ of a point $p \in U$ gives the \hldef{coordinates of $p$} in this chart.  
        A point $p \in M$ is called a \hldef{boundary point} if for some chart $(U, \varphi)$ containing $p$, the coordinates satisfy $\varphi(p)_n = 0$; otherwise, $p$ is an \hldef{interior point}. Boundary points and an interior points of $M$ coincide with \CrefAndHyperrefIfExist{definition:interior_and_boundary_of_topological_space}{boundary points and interior points} of $M$ as a topological space.
    \end{enumerate}
\end{definition}

\begin{definition}[Atlas]
    \label{definition:atlas_on_a_topological_manifold}
An \hldef{atlas} on a \CrefAndHyperrefIfExist{definition:topological_manifold}{topological manifold $M$ with or without boundary} of dimension $n$ is a collection $\mathcal{A} = \{ (U_\alpha, \varphi_\alpha) \}_{\alpha \in A}$ of charts such that the sets $U_\alpha$ cover $M$, i.e. 
    $$ \bigcup_{\alpha \in A} U_\alpha = M. $$
    Two atlases are said to be \hldef{compatible} if their union is also an atlas.
    An atlas is said to be \hldef{maximal} if it is not properly contained in any larger atlas.
\end{definition}
\begin{definition}[Chart Transition Map]
    \label{definition:transition_map_between_charts_on_a_topological_manifold}

    Given two \hyperrefIfExists{definition:chart_on_a_topological_manifold}{charts} $(U, \varphi)$ and $(V, \psi)$ on a \CrefAndHyperrefIfExist{definition:topological_manifold}{topological manifold with boundary} $M$ such that $U \cap V \neq \emptyset$, the \hldef{chart transition map} or \hldef{change of coordinates map} (from $U$ to $V$) is the map
    $$ \psi \circ \varphi^{-1} : \varphi(U \cap V) \to \psi(U \cap V), $$
    which is a \CrefAndHyperrefIfExist{definition:homeomorphism_of_topological_spaces}{homeomorphism} between open subsets of $\mathbb{H}^n = \{x \in \mathbb{R}^n : x_n \ge 0\}$.

    In particular, we may speak of this notion when $M$ is a \CrefAndHyperrefIfExist{definition:topological_manifold}{topological manifold without boundary}.
\end{definition}


\subsection{\texorpdfstring{$C^k$}{Ck} manifolds}

$C^k$-manifolds are manifolds whose transition maps are $C^k$-maps between open subsets of Euclidean spaces.

\begin{definition} \label{definition:C_k_map_between_open_subsets_of_closed_half_spaces_of_Rns}

    Let $U \subseteq \mathbb{H}^n$ and $V \subseteq \mathbb{H}^m$ be open subsets of the closed half-spaces 
    $\mathbb{H}^n = \{ x \in \mathbb{R}^n : x_n \geq 0 \}$ and $\mathbb{H}^m = \{ y \in \mathbb{R}^m : y_m \geq 0 \}$ respectively, 
    and let $k \in \mathbb{N}_0 \cup \{\infty\}$ be fixed.
    \begin{enumerate}
        \item A \hldef{$C^k$-morphism} (or \hldef{$C^k$-map}) from $U$ to $V$ is a function $f : U \to V$
        such that $f$ extends to a $k$-times continuously differentiable function on an open neighborhood of $U$ in $\mathbb{R}^n$.
        Equivalently, all partial derivatives
        $$ D^\alpha f : U \to \mathbb{R}^m, \quad |\alpha| \leq k, $$ 
        exist and are continuous up to the boundary on $U$.
        We denote the set of all such maps by \hl{$C^k(U, V)$}.

        \item A \hldef{$C^k$-function on $U$} is a $C^k$-map from $U$ to $\bbR$. We let \hl{$C^k(U)$} denote the space of $C^k$-functions on $U$.

        \item A \hldef{$C^k$-diffeomorphism} $f: U \to V$ is a $C^k$-morphism that is a bijection whose inverse $f^{-1}$ is also a $C^k$-morphism between open subsets of half-spaces.

        \item A \hldef{smooth morphism/map/function/diffeomorphism} is one that is $C^\infty$ in the above sense.
    \end{enumerate}
    In fact, when $U$ and $V$ are open subsets of $\mathbb{R}^n$ without boundary points, the notions of $C^k$-morphisms coincide with the classical ones of $k$-times continuously differentiable maps between open subsets of $\mathbb{R}^n$. In this case, the extension condition is trivially satisfied by restricting to an open neighborhood in $\mathbb{R}^n$. For open subsets with boundary points in $\mathbb{H}^n$, the extension requirement ensures differentiability up to the boundary.

    % In fact, we may discuss all of these notions in the case that $U$ and $V$ are open subsets of $\bbR^n$ by identifying $U$ and $V$ with open subsets of $\bbH^n$ and $\bbH^m$ contained entirely in the \CrefAndHyperrefIfExist{definition:interior_and_boundary_of_topological_space}{interiors} of $\bbH^n$ and $\bbH^m$ via a diffeomorphism \TODO{state a statement for this.} Moreover, a function $f: U \to V$ for such $U$ and $V$ is a $C^k$-morphism if $f$ itself is $k$-times continuous differentiable, without the need to refer to an extension of $f$. 

    % \begin{enumerate}
    %     \item Let $U \subseteq \mathbb{R}^n$ and $V \subseteq \mathbb{R}^m$ be open subsets, and let $k \in \mathbb{N}_0 \cup \{\infty\}$ be fixed.
    %     \begin{enumerate}
    %         \item A \hldef{$C^k$-morphism} (or \hldef{$C^k$-map}) from $U$ to $V$ is a function $f : U \to V$
    %         which is \hldef{$k$-times continuously differentiable}, that is: \TODO{TODO: define the partial derivative maps} All partial derivatives
    %         $$ D^\alpha f : U \to \mathbb{R}^m, \quad |\alpha| \le k, $$ 
    %         of $f$ of order up to $k$ exist and are \hyperrefIfExists{definition:continuous_map_between_euclidean_spaces}{continuous} on $U$,
    %         We denote the set of all such maps by \hl{$C^k(U, V)$}.

    %         \TODO{This notation is intrusive because we're not talking about manifolds right now.}
    %         \item A \hldef{$C^k$-function on $U$} is a $C^k$-map from $U$ to $\bbR$. We let \hl{$C^k(U)$} denote the space of $C^k$-functions on $U$.

    %         \item A \hldef{$C^k$-diffeomorphism} $f: U \to V$ is a $C^k$-morphism that is a bijection whose inverse $f^{-1}$ is also a $C^k$-morphism. 

    %         \item A \hldef{smooth morphism/map/function/diffeomorphism} is one that is $C^\infty$.
    %     \end{enumerate}

    %     \item Let $U \subseteq \mathbb{H}^n$ and $V \subseteq \mathbb{H}^m$ be open subsets of the closed half-spaces 
    %     $\mathbb{H}^n = \{ x \in \mathbb{R}^n : x_n \geq 0 \}$ and $\mathbb{H}^m = \{ y \in \mathbb{R}^m : y_m \geq 0 \}$ respectively, 
    %     and let $k \in \mathbb{N}_0 \cup \{\infty\}$ be fixed.
    %     \begin{enumerate}
    %         \item A \hldef{$C^k$-morphism} (or \hldef{$C^k$-map}) from $U$ to $V$ is a function $f : U \to V$
    %         such that $f$ extends to a $k$-times continuously differentiable function on an open neighborhood of $U$ in $\mathbb{R}^n$.
    %         Equivalently, all partial derivatives
    %         $$ D^\alpha f : U \to \mathbb{R}^m, \quad |\alpha| \leq k, $$ 
    %         exist and are continuous up to the boundary on $U$.
    %         We denote the set of all such maps by \hl{$C^k(U, V)$}.

    %         \item A \hldef{$C^k$-function on $U$} is a $C^k$-map from $U$ to $\bbR$. We let \hl{$C^k(U)$} denote the space of $C^k$-functions on $U$.

    %         \item A \hldef{$C^k$-diffeomorphism} $f: U \to V$ is a $C^k$-morphism that is a bijection whose inverse $f^{-1}$ is also a $C^k$-morphism between open subsets of half-spaces.

    %         \item A \hldef{smooth morphism/map/function/diffeomorphism} is one that is $C^\infty$ in the above sense.
    %     \end{enumerate}

    % \end{enumerate}

\end{definition}



\begin{definition} \label{definition:C_k_compatible_charts_on_a_topological_manifold}
    Let $k \in \mathbb{N}_0 \cup \{\infty\}$.  Let $M$ be a \CrefAndHyperrefIfExist{definition:topological_manifold}{topological manifold} (with or without boundary) of dimension $n$, and let $k \in \mathbb{N}_0 \cup \{\infty\}$. 
        % A \hldef{chart on $M$} is a pair $(U, \varphi)$, where $U \subseteq M$ is open and 
%         $\varphi : U \to \varphi(U) \subseteq \mathbb{H}^n$ is a \CrefAndHyperrefIfExist{definition:homeomorphism_of_topological_spaces}{homeomorphism} onto an open subset of the closed half-space $\mathbb{H}^n = \{x \in \mathbb{R}^n : x_n \ge 0\}$.
% % 
        Two \CrefAndHyperrefIfExist{definition:C_k_compatible_charts_on_a_topological_manifold}{charts} $(U, \varphi)$ and $(V, \psi)$ on $M$ are said to be \hldef{$C^k$-compatible} (or \hldef{smoothly compatible} if $k = \infty$) if the transition maps
        $$ \psi \circ \varphi^{-1} : \varphi(U \cap V) \to \psi(U \cap V) $$
        $$ \varphi \circ \psi^{-1} : \psi(U \cap V) \to \varphi(U \cap V) $$
        are \CrefAndHyperrefIfExist{definition:C_k_map_between_open_subsets_of_closed_half_spaces_of_Rns}{$C^k$-maps between open subsets of $\mathbb{H}^n$}; in particular, both transition maps are \CrefAndHyperrefIfExist{definition:C_k_map_between_open_subsets_of_closed_half_spaces_of_Rns}{$C^k$-diffeomorphisms}.

\end{definition}


\begin{definition} \label{definition:C_k_atlas_on_a_topological_manifold}

Let $k \in \mathbb{N}_0 \cup \{\infty\}$ be fixed. Let $M$ be a \CrefAndHyperrefIfExist{definition:topological_manifold}{topological manifold with or without boundary} of dimension $n$. 

A \hldef{$C^k$-atlas} (or \hldef{smooth atlas} if $k = \infty$) on $M$ is an \CrefAndHyperrefIfExist{definition:atlas_on_a_topological_manifold}{atlas}
$$ \mathcal{A} = \{(U_\alpha, \varphi_\alpha)\}_{\alpha \in A} $$
such that for every pair $\alpha, \beta \in A$ with $U_\alpha \cap U_\beta \neq \emptyset$, the charts $(U_\alpha, \varphi_\alpha)$ and $(U_\beta, \varphi_\beta)$ are \CrefAndHyperrefIfExist{definition:C_k_compatible_charts_on_a_topological_manifold}{$C^k$-compatible}.
% the \CrefAndHyperrefIfExist{definition:transition_map_between_charts_on_a_topological_manifold}{transition maps}
% $$ \varphi_\beta \circ \varphi_\alpha^{-1} : \varphi_\alpha(U_\alpha \cap U_\beta) \to \varphi_\beta(U_\alpha \cap U_\beta) $$
% are \CrefAndHyperrefIfExist{definition:C_k_map_between_open_subsets_of_half_spaces}{$C^k$-diffeomorphisms} between open subsets of the \CrefAndHyperrefIfExist{definition:closed_half_space_in_euclidean_space}{closed half-space $\mathbb{H}^n = \{ x \in \mathbb{R}^n : x_n \ge 0 \}$}.

% Equivalently, for all overlapping charts, these transition functions belong to $C^k(\varphi_\alpha(U_\alpha \cap U_\beta), \varphi_\beta(U_\alpha \cap U_\beta))$ and have $C^k$ inverses.

% In particular, we may speak of these notions when $M$ is a \CrefAndHyperrefIfExist{definition:topological_manifold}{topological manifold without boundary}.

% Let $k \in \mathbb{N}_0 \cup \{\infty\}$ be fixed.

% Let $M$ be a topological manifold of dimension $n$. A \hldef{$C^k$-atlas} (or \hldef{smooth atlas} if $k = \infty$) on $M$ is an \hyperrefIfExists{definition:atlas_on_a_topological_manifold}{atlas}
% $$ \mathcal{A} = \{(U_\alpha, \varphi_\alpha)\}_{\alpha \in A} $$
% such that for every pair $\alpha, \beta \in A$ with $U_\alpha \cap U_\beta \neq \emptyset$, the \hyperrefIfExists{definition:transition_map_between_charts_on_a_topological_manifold}{transition maps}
% $$ \varphi_\beta \circ \varphi_\alpha^{-1} : \varphi_\alpha(U_\alpha \cap U_\beta) \to \varphi_\beta(U_\alpha \cap U_\beta) $$
% are \hyperrefIfExists{definition:C_k_map_between_open_subsets_of_closed_half_spaces_of_Rns}{$C^k$-diffeomorphisms} between open subsets of $\mathbb{R}^n$.

% Equivalently, for all overlapping charts, these transition functions belong to $C^k(\varphi_\alpha(U_\alpha \cap U_\beta), \varphi_\beta(U_\alpha \cap U_\beta))$ and have $C^k$ inverses.
\end{definition}



\begin{definition} \label{definition:C_k_manifold}
Let $k \in \mathbb{N}_0 \cup \{\infty\}$ be fixed. An \hldef{$n$-dimensional $C^k$/$k$-differentiable-(real)manifold with boundary (resp. without boundary)} is a pair $(M, \mathcal{A})$, where $M$ is a \CrefAndHyperrefIfExist{definition:topological_manifold}{topological manifold with boundary (resp. without boundary)} of dimension $n$ and $\mathcal{A}$ is a \CrefAndHyperrefIfExist{definition:C_k_atlas_on_a_topological_manifold}{$C^k$-atlas} on $M$.

The atlas $\mathcal{A}$ is usually taken to be \CrefAndHyperrefIfExist{definition:atlas_on_a_topological_manifold}{maximal} with respect to \CrefAndHyperrefIfExist{definition:C_k_compatible_charts_on_a_topological_manifold}{$C^k$-compatibility}, meaning it contains every $C^k$-chart compatible with all charts in $\mathcal{A}$.

Note that a $C^0$-manifold is simply a \CrefAndHyperrefIfExist{definition:topological_manifold}{topological manifold} and that a $C^\infty$-manifold is synonymously referred to as a \hldef{smooth/differentiable (real) manifold}.
% See convention:C_k_for_k_equal_0_means_continuous_whereas_C_k_for_k_equal_infty_means_smooth
\end{definition}

\begin{convention} \label{convention:C_k_for_k_equal_0_means_continuous_whereas_C_k_for_k_equal_infty_means_smooth}
In manifold theory, there are many notions describable or definable via $C^k$, i.e. $k$-differentiability. In the case of $k = 0$, the adjective/adverb of $C^0$ is omitted. In the case of $k = \infty$, one can synonymously describe that notion as ``smooth'' or simply ``differentiable''. In other cases, one can say ``$k$-differentiable'' instead of ``$C^k$''. For example, a $C^0$\hyperrefIfExists{definition:C_k_manifold}{-manifold} is simply a \hyperrefIfExists{definition:topological_manifold}{(real) topological manifold}, a $k$-differentiable manifold refers to a $C^k$-manifold, and a smooth/differentiable (real) manifold refers to a $C^\infty$-manifold.
\end{convention}
\begin{definition} \label{definition:C_k_morphism_between_C_k_manifolds}
% Let $k \in \mathbb{N}_0 \cup \{\infty\}$ be fixed. Let $(M, \mathcal{A}_M)$ and $(N, \mathcal{A}_N)$ be $C^k$-manifolds of dimensions $n$ and $m$, respectively, where $M,N$ are topological manifolds and $\calA_M$ and $\calA_N$ are $C^k$-atlases.

% A \hldef{$C^k$-morphism} (or \hldef{$C^k$-map}) between $M$ and $N$ is a \CrefAndHyperrefIfExist{definition:continuous_map_of_topological_spaces}{continuous map}
% $$ f : M \to N $$
% such that for every $p \in M$ there exist \hyperrefIfExists{definition:chart_on_a_topological_manifold}{charts} $(U, \varphi) \in \mathcal{A}_M$ with $p \in U$ and $(V, \psi) \in \mathcal{A}_N$ with $f(p) \in V$ satisfying
% $$ \psi \circ f \circ \varphi^{-1} : \varphi(U \cap f^{-1}(V)) \to \psi(V) $$
% is a \CrefAndHyperrefIfExist{definition:C_k_map_between_open_subsets_of_closed_half_spaces_of_Rns}{$C^k$-map} between open subsets of Euclidean spaces $\mathbb{R}^n$ and $\mathbb{R}^m$, i.e.,
% $$ \psi \circ f \circ \varphi^{-1} \in C^k(\varphi(U \cap f^{-1}(V)), \psi(V)).  $$
% If $f$ is a homeomorphism and its inverse $f^{-1} : N \to M$ is also a $C^k$-morphism, then $f$ is called a \hldef{$C^k$-diffeomorphism}.
% We let \hl{$C^k(M,N)$} denote the space of $C^k$-maps $M \to N$.  We let \hl{$C^k(M)$} denote the space of \hldef{$C^k$-functions}, i.e. the $C^k$-maps $M \to \bbR$.

Let $k \in \mathbb{N}_0 \cup \{\infty\}$ be fixed. Let $(M, \mathcal{A}_M)$ and $(N, \mathcal{A}_N)$ be \CrefAndHyperrefIfExist{definition:C_k_manifold}{$C^k$-manifolds with boundary} of dimensions $n$ and $m$, respectively, where $M,N$ are \CrefAndHyperrefIfExist{definition:topological_manifold}{topological manifolds with boundary} and $\calA_M$ and $\calA_N$ are \CrefAndHyperrefIfExist{definition:C_k_atlas_on_a_topological_manifold}{$C^k$-atlases} whose charts map to open subsets of the \CrefAndHyperrefIfExist{definition:closed_half_space_in_euclidean_space}{closed half-spaces} $\mathbb{H}^n$ and $\mathbb{H}^m$.

A \hldef{$C^k$-morphism} (or \hldef{$C^k$-map}) between $M$ and $N$ is a \CrefAndHyperrefIfExist{definition:continuous_map_of_topological_spaces}{continuous map}
$$ f : M \to N $$
such that for every $p \in M$ there exist \hyperrefIfExists{definition:chart_on_a_topological_manifold}{charts} $(U, \varphi) \in \mathcal{A}_M$ with $p \in U$ and $(V, \psi) \in \mathcal{A}_N$ with $f(p) \in V$ satisfying
$$ \psi \circ f \circ \varphi^{-1} : \varphi(U \cap f^{-1}(V)) \to \psi(V) $$
is a \CrefAndHyperrefIfExist{definition:C_k_map_between_open_subsets_of_closed_half_spaces_of_Rns}{$C^k$-map} between open subsets of the closed half-spaces $\mathbb{H}^n$ and $\mathbb{H}^m$, i.e.,
$$ \psi \circ f \circ \varphi^{-1} \in C^k(\varphi(U \cap f^{-1}(V)), \psi(V)).  $$
If $f$ is a homeomorphism and its inverse $f^{-1} : N \to M$ is also a $C^k$-morphism, then $f$ is called a \hldef{$C^k$-diffeomorphism}.
We let \hl{$C^k(M,N)$} denote the space of $C^k$-maps $M \to N$. We let \hl{$C^k(M)$} denote the space of \hldef{$C^k$-functions}, i.e., the $C^k$-maps $M \to \mathbb{R}$.

In particular, we may speak of these notions when $M$ and $N$ are \CrefAndHyperrefIfExist{definition:C_k_manifold}{$C^k$-manifolds without boundary}.

\end{definition}

\begin{remark}
    The notations $C^k(M,N)$ (and $C^k(M)$) agrees with the usual notations $C^k(M,N)$ and $C^k(M)$ in the case that $M$ is an open subset of $\bbR^n$\CrefIfExists{definition:C_k_map_between_open_subsets_of_closed_half_spaces_of_Rns}.
\end{remark}

\subsection{(Real) \texorpdfstring{$C^k$}{Ck}-vector bundles on (real) \texorpdfstring{$C^k$}{Ck}-manifolds}


\begin{definition} \label{definition:C_k_vector_bundle_on_C_k_manifold}
  Let $k \in \bbZ_{\geq 0} \cup \{\infty\}$, and let $M$ be a \CrefAndHyperrefIfExist{definition:C_k_manifold}{$C^k$ manifold with or without boundary} of dimension $m$. A \hldef{$C^k$ vector bundle of rank $r$ over $M$} is a triple $(E, \pi, M)$ where:
  \begin{itemize}
    \item $E$ is a \CrefAndHyperrefIfExist{definition:topological_space}{topological space} called the \hldef{total space},
    \item $\pi : E \to M$ is a \CrefAndHyperrefIfExist{definition:continuous_map_of_topological_spaces}{continuous} \CrefAndHyperrefIfExist{definition:injective_surjective_bijective_map_of_sets}{surjection} called the \hldef{projection map},
    \item For each $p \in M$, the \CrefAndHyperrefIfExist{definition:fiber_of_a_map_of_topological_spaces_over_a_point}{fiber} \hl{$E_p := \pi^{-1}(\{p\})$} is endowed with the structure of a \CrefAndHyperrefIfExist{definition:vector_space_over_a_field}{vector space} over $\mathbb{R}$ of dimension $r$,
    \item There exists an \CrefAndHyperrefIfExist{definition:open_covering_of_a_topological_space}{open cover} $\{ U_\alpha \}_{\alpha \in A}$ of $M$ by open sets, and \CrefAndHyperrefIfExist{definition:homeomorphism_of_topological_spaces}{homeomorphisms} (called \hldef{local trivializations})
    $$ \phi_\alpha : \pi^{-1}(U_\alpha) \to U_\alpha \times \mathbb{R}^r $$
    such that:
    \begin{itemize}
      \item Each $\phi_\alpha$ is a \CrefAndHyperrefIfExist{definition:C_k_map_between_open_subsets_of_closed_half_spaces_of_Rns}{$C^k$ diffeomorphism} onto its image, where $U_\alpha$ is identified with an open subset of $\mathbb{R}^m_+$,
      \item For every $p \in U_\alpha$, the restriction
      $$ \phi_\alpha|_{E_p} : E_p \to \{p\} \times \mathbb{R}^r \cong \mathbb{R}^r $$
      is a \CrefAndHyperrefIfExist{definition:morphism_of_vector_spaces}{vector space isomorphism},
    \end{itemize}
    \item For all indices $\alpha, \beta$, define the \hldef{transition functions}
    $$ \hlin{t_{\alpha \beta} : U_\alpha \cap U_\beta \to \mathrm{GL}(r, \mathbb{R})} $$
    uniquely by the relation
    $$
    \phi_\alpha \circ \phi_\beta^{-1} (p, v) = (p, t_{\alpha \beta}(p) v) \quad \text{for } p \in U_\alpha \cap U_\beta, \; v \in \mathbb{R}^r.
    $$
    Each $t_{\alpha \beta}$ is a $C^k$ map respecting the boundary structure.
  \end{itemize}

  The total space $E$ then in fact has a canonical structure as a $C^k$-manifold (without boundary if $M$ is a $C^k$ manifold without boundary) \CrefIfExists{theorem:total_space_of_a_C_k_vector_bundle_is_a_C_k_manifold}

  Let $E$ be a $C^k$ vector bundle over a $C^k$ manifold with boundary $M$. A \hldef{$C^k$-section of $E$ over an open subset $U \subseteq M$} (where $U$ may intersect the boundary) is a \CrefAndHyperrefIfExist{definition:C_k_morphism_between_C_k_manifolds}{$C^k$ map} $s: U \to E$ such that $\pi \circ s = \id_{U}$.
  We might denote by 
  $$\hlin{\Gamma^{C^k}(U, E) = \Gamma^{C^k}(U, E;\bbR) = E^{C^k}(U) = E^{C^k}(U;\bbR)}$$
  the space of $C^k$ sections of $E$ (as a vector space of $M$). It is a real vector space. When $k$ is self-apparent, this space may also be without the superscript of $C^k$, i.e. by 
  $$\hlin{\Gamma(U, E) = \Gamma(U, E;\bbR) = E(U) = E(U;\bbR)}.$$
  A $C^k$-section of $E$ over $M$ itself may be referred to as a \hldef{global $C^k$-section of $E$}; the space of such sections may be shorthand-notated as \hl{$\Gamma_k(E)$}, \hl{$\Gamma(E)$}, or \hl{$\Gamma_k(E;\bbR)$}.

  A $C^0$-section of $E$ is simply called a \hldef{(continuous) section of $E$}, and a $C^\infty$-section of $E$ is called a \hldef{smooth section of $E$}.

\end{definition}


% \begin{definition}[$C^k$ vector bundle of rank $r$ over a manifold with boundary]
% Let $k \in \mathbb{N}_0 \cup \{\infty\}$, and let $M$ be a $C^k$ manifold with boundary of dimension $m$. A \hldef{$C^k$ vector bundle of rank $r$ over $M$} is a triple $(E, \pi, M)$ where:
% \begin{itemize}
%   \item $E$ is a topological space called the \hldef{total space},
%   \item $\pi : E \to M$ is a continuous surjection called the \hldef{projection map},
%   \item For each $p \in M$, the fiber $E_p := \pi^{-1}(\{p\})$ is endowed with the structure of a vector space over $\mathbb{R}$ of dimension $r$,
%   \item There exists an open cover $\{ U_\alpha \}_{\alpha \in A}$ of $M$ by open sets in the manifold-with-boundary topology, and homeomorphisms (called \hldef{local trivializations})
%   $$
%   \phi_\alpha : \pi^{-1}(U_\alpha) \to U_\alpha \times \mathbb{R}^r
%   $$
%   such that:
%   \begin{itemize}
%     \item Each $\phi_\alpha$ is a $C^k$ diffeomorphism onto its image, where $U_\alpha$ is identified with an open subset of $\mathbb{R}^m_+$,
%     \item For every $p \in U_\alpha$, the restriction
%     $$
%     \phi_\alpha|_{E_p} : E_p \to \{p\} \times \mathbb{R}^r \cong \mathbb{R}^r
%     $$
%     is a vector space isomorphism,
%   \end{itemize}
%   \item For all indices $\alpha, \beta$, define the \hldef{transition functions}
%   $$
%   \hlin{t_{\alpha \beta} : U_\alpha \cap U_\beta \to \mathrm{GL}(r, \mathbb{R})}
%   $$
%   uniquely by the relation
%   $$
%   \phi_\alpha \circ \phi_\beta^{-1} (p, v) = (p, t_{\alpha \beta}(p) v) \quad \text{for } p \in U_\alpha \cap U_\beta, \; v \in \mathbb{R}^r.
%   $$
%   Each $t_{\alpha \beta}$ is a $C^k$ map respecting the boundary structure.
% \end{itemize}

% Let $E$ be a $C^k$ vector bundle over a $C^k$ manifold with boundary $M$. A \hldef{$C^k$-section of $E$ over an open subset $U \subseteq M$} (where $U$ may intersect the boundary) is a \CrefAndHyperrefIfExist{definition:C_k_morphism_between_C_k_manifolds}{$C^k$ map} $s: U \to E$ such that $\pi \circ s = \id_{U}$.
% We might denote by 
% $$\hlin{\Gamma^{C^k}(U, E) = \Gamma^{C^k}(U, E;\bbR) = E^{C^k}(U) = E^{C^k}(U;\bbR)}$$
% the space of $C^k$ sections of $E$ (as a vector space of $M$). It is a real vector space. When $k$ is self-apparent, this space may also be without the superscript of $C^k$, i.e. by 
% $$\hlin{\Gamma(U, E) = \Gamma(U, E;\bbR) = E(U) = E(U;\bbR)}.$$
% A $C^k$-section of $E$ over $M$ itself may be referred to as a \hldef{global $C^k$-section of $E$}; the space of such sections may be shorthand-notated as \hl{$\Gamma_k(E)$}, \hl{$\Gamma(E)$}, or \hl{$\Gamma_k(E;\bbR)$}.

% A $C^0$-section of $E$ is simply called a \hldef{(continuous) section of $E$}, and a $C^\infty$-section of $E$ is called a \hldef{smooth section of $E$}.

% \end{definition}



\begin{theorem}[Total space of a $C^k$ vector bundle as a $C^k$ manifold] \label{theorem:total_space_of_a_C_k_vector_bundle_is_a_C_k_manifold}

Let $k \in \mathbb{N} \cup \{\infty\}$, let $M$ be a $C^k$ manifold with or without boundary of dimension $n$, and let
$$
\pi : E \to M
$$
be a $C^k$ vector bundle of rank $r$ over $M$. 
Let $\{U_\alpha\}_{\alpha \in A}$ be an open cover of $M$ together with \CrefAndHyperrefIfExist{definition:C_k_vector_bundle_on_C_k_manifold}{local trivializations}
$$ \Phi_\alpha : \pi^{-1}(U_\alpha) \to U_\alpha \times \mathbb{R}^r.$$

Fix a \CrefAndHyperrefIfExist{definition:C_k_atlas_on_a_topological_manifold}{$C^k$ atlas} $\{(U_\alpha,\varphi_\alpha)\}_{\alpha \in A}$ on $M$, where each
$$ \varphi_\alpha : U_\alpha \xrightarrow{\ \cong\ } V_\alpha \subseteq \mathbb{R}^n $$
is a homeomorphism onto an open subset $V_\alpha$.

Define \CrefAndHyperrefIfExist{definition:C_k_compatible_charts_on_a_topological_manifold}{charts} on $E$ by
$$
\Psi_\alpha : \pi^{-1}(U_\alpha) \longrightarrow V_\alpha \times \mathbb{R}^r, \qquad
\Psi_\alpha(e) := \bigl(\varphi_\alpha(\pi(e)),\, v_\alpha(e)\bigr),
$$
where $v_\alpha(e) \in \mathbb{R}^r$ is determined by the identity
$$
\Phi_\alpha(e) = \bigl(\pi(e),\, v_\alpha(e)\bigr).
$$

Then the following hold.
\begin{enumerate}
  \item Each $\Psi_\alpha$ is a \CrefAndHyperrefIfExist{definition:homeomorphism_of_topological_spaces}{homeomorphism} from $\pi^{-1}(U_\alpha)$ onto the open subset $V_\alpha \times \mathbb{R}^r$ of $\mathbb{R}^{n+r}$.
  \item For any $\alpha,\beta \in A$ with $U_\alpha \cap U_\beta \neq \varnothing$, the \CrefAndHyperrefIfExist{definition:transition_map_between_charts_on_a_topological_manifold}{chart transition map}
  $$ \Psi_\beta \circ \Psi_\alpha^{-1} : \bigl(V_\alpha \cap \varphi_\alpha(U_\alpha \cap U_\beta)\bigr) \times \mathbb{R}^r \longrightarrow \bigl(V_\beta \cap \varphi_\beta(U_\alpha \cap U_\beta)\bigr) \times \mathbb{R}^r $$
  is given by
  $$ (u,v) \longmapsto \Bigl( \varphi_\beta \circ \varphi_\alpha^{-1}(u),\, g_{\beta\alpha}\bigl(\varphi_\alpha^{-1}(u)\bigr)\, v \Bigr), $$
  and is of \CrefAndHyperrefIfExist{definition:C_k_map_between_open_subsets_of_closed_half_spaces_of_Rns}{class $C^k$}, since $\varphi_\beta \circ \varphi_\alpha^{-1}$ and $g_{\beta\alpha}$ are $C^k$.

  \item Hence the collection of charts
  $$ \bigl\{ \bigl(\pi^{-1}(U_\alpha), \Psi_\alpha\bigr) \bigr\}_{\alpha \in A} $$
  is a \CrefAndHyperrefIfExist{definition:C_k_atlas_on_a_topological_manifold}{$C^k$ atlas} on $E$, making $E$ into a \CrefAndHyperrefIfExist{definition:C_k_manifold}{$C^k$ manifold} (without boundary if $M$ is without boundary) of dimension $n + r$.
  \item With this $C^k$ manifold structure on $E$, the projection
  $$ \pi : E \to M $$
  \TODO{submersion}
  is a $C^k$ submersion.
\end{enumerate}
In particular, the \CrefAndHyperrefIfExist{definition:C_k_vector_bundle_on_C_k_manifold}{total space} of a $C^k$ vector bundle of rank $r$ over an $n$-dimensional $C^k$ manifold is canonically a $C^k$ manifold of dimension $n + r$.
\end{theorem}






\begin{definition}[Morphism of $C^k$ vector bundles over a $C^k$ manifold with or without boundary] \label{definition:morphism_of_C_k_vector_bundles_over_a_C_k_manifold_with_or_without_boundary}
Let $k \in \mathbb{N} \cup \{\infty\}$, let $M$ be a \CrefAndHyperrefIfExist{definition:C_k_manifold}{$C^k$ manifold with or without boundary}, and let
$$ \pi_E : E \to M, \qquad \pi_F : F \to M $$
be \CrefAndHyperrefIfExist{definition:C_k_vector_bundle_on_C_k_manifold}{$C^k$ vector bundles} of ranks $r_E$ and $r_F$ over $M$.
A \hldef{morphism of $C^k$ vector bundles over $M$} (or a \hldef{$C^k$ vector bundle morphism covering the identity of $M$}) is a map $\Phi : E \to F$ satisfying:
\begin{enumerate}
  \item $\Phi$ is of class $C^k$ as a \CrefAndHyperrefIfExist{definition:C_k_morphism_between_C_k_manifolds}{map} between $C^k$ manifolds (with or without boundary); recall that $E$ has the structure of a $C^k$-manifold (\Cref{theorem:total_space_of_a_C_k_vector_bundle_is_a_C_k_manifold}),
  \item $\pi_F \circ \Phi = \pi_E$ (i.e.\ $\Phi$ is fiber-preserving over the identity on $M$),
  \item for each $x \in M$, the restriction
  $$ \Phi_x := \Phi \big|_{E_x} : E_x \longrightarrow F_x $$
  \CrefIfExists{definition:C_k_vector_bundle_on_C_k_manifold}
  is a \CrefAndHyperrefIfExist{definition:morphism_of_vector_spaces}{linear map} of real vector spaces.
\end{enumerate}
In this situation $\Phi$ is also called a \hldef{bundle map} (of class $C^k$) from $E$ to $F$ over $M$.
\end{definition}




% \begin{definition} \label{definition:dual_bundle_of_a_C_k_vector_bundle}

%   \TODO{do this definition for manifolds with boundary}
% Let $k \in \mathbb{N}_0 \cup \{\infty\}$, let $M$ be a $C^k$-manifold, and let $(E, \pi, M)$ be a $C^k$ \hyperrefIfExists{definition:C_k_vector_bundle_on_C_k_manifold}{vector bundle} of rank $r$ over $M$.

% The \hldef{dual bundle of $E$} is the $C^k$ vector bundle \hl{$(E^*, \pi_{E^*}, M)$} where:
% \begin{itemize}
%   \item The \hyperrefIfExists{definition:C_k_vector_bundle_on_C_k_manifold}{total space} 
%   $$ E^* := \bigsqcup_{p \in M} E_p^* $$
%   is the disjoint union of the dual vector spaces $E_p^* := \mathrm{Hom}_{\mathbb{R}}(E_p, \mathbb{R})$ of the fibers $E_p$,
%   \item The \hyperrefIfExists{definition:C_k_vector_bundle_on_C_k_manifold}{projection map} 
%   $$ \pi_{E^*} : E^* \to M $$
%   sends each $\varphi \in E_p^*$ to its base point $p \in M$,
%   \item If $\{(U_\alpha, \phi_\alpha)\}_{\alpha \in A}$ is a system of \hyperrefIfExists{definition:C_k_vector_bundle_on_C_k_manifold}{local trivializations of $E$},
%   $$ \phi_\alpha : \pi^{-1}(U_\alpha) \to U_\alpha \times \mathbb{R}^r, $$
%   then the \hldef{local trivializations of $E^*$} induced by $\phi_\alpha$ are given by
%   $$
%   \phi_\alpha^* : \pi_{E^*}^{-1}(U_\alpha) \to U_\alpha \times (\mathbb{R}^r)^*, 
%   $$
%   defined fiberwise by dualizing the vector space isomorphism 
%   $$
%   \phi_\alpha|_{E_p} : E_p \to \{p\} \times \mathbb{R}^r,
%   $$
%   \TODO{TODO: define dual vector space}
%   where $(\mathbb{R}^r)^* \cong \mathbb{R}^r$ is the standard dual vector space,
%   \item The \hyperrefIfExists{definition:C_k_vector_bundle_on_C_k_manifold}{transition functions of $E^*$ with respect to these trivializations} are given by the inverse transpose of those of $E$, i.e.,
%   $$ t_{\alpha \beta}^*(p) = (t_{\alpha \beta}(p)^T)^{-1} \quad \text{for } p \in U_\alpha \cap U_\beta, $$
%   where $t_{\alpha \beta} : U_\alpha \cap U_\beta \to \mathrm{GL}(r, \mathbb{R})$ are the $C^k$ transition functions of $E$.
% \end{itemize}
% The dual bundle $E^*$ is thus a $C^k$ vector bundle of rank $r$ over $M$.
% \end{definition}



\begin{definition} \label{definition:dual_bundle_of_a_C_k_vector_bundle}
Let $k \in \mathbb{N}_0 \cup \{\infty\}$, let $M$ be a \CrefAndHyperrefIfExist{definition:C_k_manifold}{$C^k$-manifold with or without boundary}, and let $(E, \pi, M)$ be a $C^k$ \hyperrefIfExists{definition:C_k_vector_bundle_on_C_k_manifold_with_or_without_boundary}{vector bundle} of rank $r$ over $M$.
The \hldef{dual bundle of $E$} is the $C^k$ vector bundle \hl{$(E^*, \pi_{E^*}, M)$} defined of rank $r$ as follows:
\begin{itemize}
  \item The \CrefAndHyperrefIfExist{definition:C_k_vector_bundle_on_C_k_manifold}{total space}
  $$ E^* := \bigsqcup_{p \in M} E_p^* $$
  is the disjoint union of the \CrefAndHyperrefIfExist{definition:dual_of_a_left_right_two_sided_module}{dual vector spaces $E_p^* := \mathrm{Hom}_\mathbb{R}(E_p, \mathbb{R})$} of the \CrefAndHyperrefIfExist{definition:fiber_of_a_map_of_topological_spaces_over_a_point}{fibers} $E_p$.
  
  \item The \CrefAndHyperrefIfExist{definition:C_k_vector_bundle_on_C_k_manifold}{projection map}
  $$ \pi_{E^*} : E^* \to M $$
  sends each $\varphi \in E_p^*$ to its base point $p \in M$.
  
  \item If $\{(U_\alpha, \phi_\alpha)\}_{\alpha \in A}$ is a system of \CrefAndHyperrefIfExist{definition:C_k_vector_bundle_on_C_k_manifold}{local trivializations} of $E$ over \CrefAndHyperrefIfExist{definition:chart_on_a_topological_manifold}{coordinate charts} $(U_\alpha, \psi_\alpha)$ compatible with the manifold structure (allowing charts modeled on open subsets of $\mathbb{R}^n_+$ near $\partial M$), where
  $$ \phi_\alpha : \pi^{-1}(U_\alpha) \to U_\alpha \times \mathbb{R}^r, $$
  then the \hldef{induced local trivializations of $E^*$} are defined by
  $$
  \phi_\alpha^* : \pi_{E^*}^{-1}(U_\alpha) \to U_\alpha \times (\mathbb{R}^r)^*,
  $$
  determined fiberwise by dualizing the linear isomorphisms
  $$
  (\phi_\alpha)|_{E_p} : E_p \to \{p\} \times \mathbb{R}^r,
  $$
  and hence $(\mathbb{R}^r)^* \cong \mathbb{R}^r$.
  
  \item The \CrefAndHyperrefIfExist{definition:C_k_vector_bundle_on_C_k_manifold}{transition functions} of $E^*$ relative to these trivializations are given by the inverse transpose of those of $E$:
  $$
  t_{\alpha\beta}^*(p) = (t_{\alpha\beta}(p)^T)^{-1}, \quad p \in U_\alpha \cap U_\beta,
  $$
  where $t_{\alpha\beta} : U_\alpha \cap U_\beta \to \mathrm{GL}(r, \mathbb{R})$ are the $C^k$ transition functions of $E$.
\end{itemize}
\end{definition}




\begin{definition} \label{definition:support_of_a_section_of_a_C_k_vector_bundle_over_a_C_k_manifold}
Let $M$ be a \CrefAndHyperrefIfExist{definition:C_k_manifold}{$C^k$-manifold with or without boundary} and let $E \to M$ be a \hyperrefIfExists{definition:C_k_vector_bundle_on_C_k_manifold}{$C^k$ vector bundle}. 
\begin{enumerate}
    \item For any \CrefAndHyperrefIfExist{definition:C_k_vector_bundle_on_C_k_manifold}{($C^k$-)section} $s \in \Gamma(E)$, define its \hldef{support}  
    $$ \hlin{\operatorname{supp}(s) = \overline{ \left\{ x \in M : s(x) \neq 0 \right\} } }$$
    where $0$ denotes the zero vector in the \CrefAndHyperrefIfExist{definition:fiber_of_a_map_of_topological_spaces_over_a_point}{fiber} $E_x$ and the \CrefAndHyperrefIfExist{definition:closure_of_a_subspace_of_a_topological_space}{closure} is taken in $M$.

    \item A section $s \in \Gamma(E)$ is said to be \hldef{compactly supported} if its support is \CrefAndHyperrefIfExist{definition:compact_topological_space}{compact}.
    
    \item Let \hl{$\Gamma_c(E) = \Gamma_c(M) = \Gamma_c(M;\bbR)$} denote the space of compactly supported sections, i.e. .
    $$\Gamma_c(E) = \{s \in \Gamma(E): \operatorname{supp}(s) \text{ is compact}\}$$
\end{enumerate}
\end{definition}


\begin{lemma}
    Let $M$ be a \hyperrefIfExists{definition:C_k_manifold}{$C^k$-manifold with or without boundary} and let $E \to M$ be a \hyperrefIfExists{definition:C_k_vector_bundle_on_C_k_manifold}{$C^k$ vector bundle}. 
    If $M$ is \CrefAndHyperrefIfExist{definition:compact_topological_space}{compact}, then \CrefAndHyperrefIfExist{definition:support_of_a_section_of_a_C_k_vector_bundle_over_a_C_k_manifold}{$\Gamma_c(E) = \Gamma(E)$}. 
\end{lemma}

\subsection{Tangent and cotangent bundles of (real) \texorpdfstring{$C_k$}{Ck} manifolds}

In general, we may speak of tangent/cotangent bundles of $C^k$-manifolds; these have natural structures as $C^{k-1}$-manifolds. We are most interested in the case that $k = \infty$.


\begin{definition} \label{definition:tangent_space_of_a_topological_manifold_at_a_point}
  Let $M$ be a \CrefAndHyperrefIfExist{definition:topological_manifold}{topological manifold of dimension $n$ with or without boundary}, and let $p \in M$ be a point such that there is an open neighborhood of $M$ that is a \CrefAndHyperrefIfExist{definition:C_k_manifold}{$C^1$-manifold} as a submanifold of $M$.
  
  \begin{enumerate}
    \item The \hldef{tangent space of $M$ at the point $p$}, denoted \hl{$T_p M$}, is defined as follows:
    \TODO{TODO: define the derivative/jacobian matrix of a self map of $R^n$ }
    \TODO{TODO: justify why having the derivative of the transition map between charts implies that $T_p M$ is well defined and independent of the choice of chart.}
    Choose a \CrefAndHyperrefIfExist{definition:chart_on_a_topological_manifold}{chart} $(U, \varphi)$ around $p$ with $p \in U \subseteq M$, where $\varphi : U \to V$ is a \CrefAndHyperrefIfExist{definition:homeomorphism_of_topological_spaces}{homeomorphism} onto an open subset $V \subseteq \mathbb{R}^n$ or the \CrefAndHyperrefIfExist{definition:closed_half_space_in_euclidean_space}{upper half-space $\bbH$} (when $p$ is a boundary point).
    
    We identify $T_p M$ with the vector space $\mathbb{R}^n$ via the differential of $\varphi$ at $p$. More precisely,
    $$
    T_p M := \{\,(U, \varphi), v \mid v \in \mathbb{R}^n \,\} / \sim
    $$
    where two pairs $((U, \varphi), v)$ and $((U', \varphi'), v')$ are equivalent if $p \in U \cap U'$ and
    $$
    v' = d(\varphi' \circ \varphi^{-1})_{\varphi(p)} (v),
    $$
    with $d(\varphi' \circ \varphi^{-1})_{\varphi(p)} : \mathbb{R}^n \to \mathbb{R}^n$ the derivative (Jacobian matrix) of the transition map at $\varphi(p)$.

    A \hldef{tangent vector of $M$ at $p$} is then an element of $T_p M$. 

    \item The \hldef{cotangent space of $M$ at $p$}, denoted \hl{$T_p^* M$}, is the \CrefAndHyperrefIfExist{definition:dual_of_a_left_right_two_sided_module}{dual space $(T_p M)^* = \Hom_{\mathbb{R}}(T_p M, \mathbb{R})$}.

    A \hldef{cotangent vector of $M$ at $p$} is then an element of $T_p^* M$. 
  \end{enumerate}
\end{definition}


% \begin{definition} \label{definition:tangent_space_of_a_topological_manifold_at_a_point}
%   \TODO{do this definition for manifolds with boundary}
% Let $M$ be a \CrefAndHyperrefIfExist{definition:topological_manifold}{topological manifold} of dimension $n$ and let $p \in M$ be a point.

% \begin{enumerate}
%     \item The \hldef{tangent space of $M$ at the point $p$}, denoted \hl{$T_p M$}, is defined as follows:
%     \TODO{TODO: define the derivative/jacobian matrix of a self map of $R^n$ }
%     \TODO{TODO: justify why having the derivative of the transition map between charts implies that $T_p M$ is well defined and independent of the choice of chart.}

%     Choose a chart $(U, \varphi)$ around $p$ with $p \in U \subseteq M$, where $\varphi : U \to V$ is a homeomorphism onto an open subset $V \subseteq \mathbb{R}^n$.

%     Identify $T_p M$ with the vector space $\mathbb{R}^n$ via the differential of $\varphi$ at $p$. More precisely, 
%     $$
%     T_p M := \{\, (U, \varphi), v \mid v \in \mathbb{R}^n \,\} / \sim
%     $$
%     where two pairs $((U, \varphi), v)$ and $((U', \varphi'), v')$ are equivalent if $p \in U \cap U'$ and
%     $$
%     v' = d(\varphi' \circ \varphi^{-1})_{\varphi(p)} (v),
%     $$
%     with $d(\varphi' \circ \varphi^{-1})_{\varphi(p)} : \mathbb{R}^n \to \mathbb{R}^n$ being the derivative (Jacobian matrix) of the transition map between charts at $\varphi(p)$.

%     Thus, $T_p M$ is a well-defined $n$-dimensional real vector space that does not depend on the choice of chart.

%     \TODO{TODO: define dual space of a vector space}
%     \item The \hldef{cotangent space of $M$ at the point $p$}, denoted \hl{$T_p^* M$}, is the dual space $(T_p M)^* = \Hom_{\bbR-\VectorSpaces}(T_p M, \bbR)$. 
% \end{enumerate}
% \end{definition}




% \begin{definition} \label{definition:tangent_bundle_of_a_smooth_manifold}
% Let $M$ be a \hyperrefIfExists{definition:C_k_manifold}{smooth $n$-dimensional manifold}.

% The \hldef{tangent bundle of $M$} is the \hyperrefIfExists{definition:C_k_vector_bundle_on_C_k_manifold}{vector bundle}
% $$ (TM, \pi, M) $$
% where:
% \begin{itemize}
%   \item The total space $TM := \bigsqcup_{p \in M} T_p M$ is the disjoint union of \hyperrefIfExists{definition:tangent_space_of_a_topological_manifold_at_a_point}{tangent spaces} of $M$ at all points,
%   \item The projection map $\pi : TM \to M$ sends each tangent vector to its base point,
%   \item Locally, for each chart $(U, \varphi)$ on $M$ with $\varphi : U \to \mathbb{R}^n$, the tangent bundle trivializes as
%   $$ \pi^{-1}(U) \cong U \times \mathbb{R}^n, $$
%   reflecting identification $T_p M \cong \mathbb{R}^n$ via the differential of the chart.
% \end{itemize}
% \end{definition}

\begin{definition} \label{definition:tangent_bundle_of_a_C_k_manifold}
  Let $k \in \bbZ_{\geq 1} \cup \{\infty\}$, and let $M$ be a \CrefAndHyperrefIfExist{definition:C_k_manifold}{$C^k$ $n$-dimensional manifold with or without boundary}.

  The \hldef{tangent bundle of $M$} is the \CrefAndHyperrefIfExist{definition:C_k_vector_bundle_on_C_k_manifold}{vector bundle}
  $$\hlin{(TM, \pi, M)}$$
  where:
  \begin{itemize}
    \item The \CrefAndHyperrefIfExist{definition:C_k_vector_bundle_on_C_k_manifold}{total space} 
    $$ TM := \bigsqcup_{p \in M} T_p M $$
    is the disjoint union of \CrefAndHyperrefIfExist{definition:tangent_space_of_a_topological_manifold_at_a_point}{tangent spaces} of $M$ at all points, defined via equivalence classes of $C^k$-compatible curves or derivations of $C^k$ functions at $p$, including points at the boundary.
    
    \item The projection map 
    $$ \pi : TM \to M $$
    sends each tangent vector to its base point.
    
    \item Locally, for any \CrefAndHyperrefIfExist{definition:chart_on_a_topological_manifold}{chart} $(U, \varphi)$ on $M$ with 
    $$ \varphi : U \to V \subset \mathbb{R}^n \text{ or } \varphi : U \to V \subset \bbH \text{ if } U \text{ contians boundary points}, $$
    the tangent bundle \CrefAndHyperrefIfExist{definition:tangent_bundle_of_a_C_k_manifold}{trivializes} as
    $$ \pi^{-1}(U) \cong U \times \mathbb{R}^n.  $$
    This reflects the identification 
    $$ T_p M \cong \mathbb{R}^n $$
    via the differential (pushforward) of the $C^k$ chart $\varphi$.
  \end{itemize}
  
  The total space $TM$ carries a natural $C^{k-1}$ vector  bundle structure over $M$ and \CrefAndHyperref{theorem:total_space_of_a_C_k_vector_bundle_is_a_C_k_manifold}{hence} a $C^{k-1}$ manifold structure (without boundary if $M$ is without boundary), and the projection $\pi$ is a \CrefAndHyperrefIfExist{definition:C_k_morphism_between_C_k_manifolds}{$C^{k-1}$ map}.
\end{definition}


% \begin{definition} \label{definition:tangent_bundle_of_a_C_k_manifold}
%   \TODO{do this definition for manifolds with boundary}
%   Let $k \in \bbZ_{\geq 1} \cup \{\infty\}$, and let $M$ be a \hyperrefIfExists{definition:C_k_manifold}{$C^k$ $n$-dimensional manifold}.

%   The \hldef{tangent bundle of $M$} is the \hyperrefIfExists{definition:C_k_vector_bundle_on_C_k_manifold}{vector bundle}
%   $$ (TM, \pi, M) $$
%   where:
%   \begin{itemize}
%     \item The total space $TM := \bigsqcup_{p \in M} T_p M$ is the disjoint union of \hyperrefIfExists{definition:tangent_space_of_a_C_k_manifold_at_a_point}{tangent spaces} of $M$ at all points, each defined via equivalence classes of $C^k$-compatible curves or derivations of $C^k$ functions at $p$,
%     \item The projection map $\pi : TM \to M$ sends each tangent vector to its base point,
%     \item Locally, for each chart $(U, \varphi)$ on $M$ with $\varphi : U \to \mathbb{R}^n$, the tangent bundle trivializes as
%     $$ \pi^{-1}(U) \cong U \times \mathbb{R}^n, $$
%     reflecting the identification $T_p M \cong \mathbb{R}^n$ via the differential (pushforward) of the $C^k$ chart $\varphi$.
%   \end{itemize}
%   The total space $TM$ carries a natural structure of a $C^{k-1}$ manifold, and the projection $\pi$ is a $C^{k-1}$ map.
% \end{definition}



\begin{definition} \label{definition:cotangent_bundle_of_a_smooth_manifold}
Let $k \in \bbZ_{\geq 1} \cup \{\infty\}$, and let $M$ be a \hyperrefIfExists{definition:C_k_manifold}{$C^k$ $n$-dimensional manifold}.

The \hldef{cotangent bundle of $M$} is the \hyperrefIfExists{definition:dual_bundle_of_a_C_k_vector_bundle}{dual vector bundle} 
$$\hlin{(T^*M, \pi_{T^*M}, M)}$$
of the \hyperrefIfExists{definition:tangent_bundle_of_a_C_k_manifold}{tangent bundle $TM$}. It is also denoted by \hl{$\Omega M$}. In particular, $\Omega M$ carries a carries a natural $C^{k-1}$ vector  bundle structure over $M$ and \CrefAndHyperref{theorem:total_space_of_a_C_k_vector_bundle_is_a_C_k_manifold}{hence} a $C^{k-1}$ manifold structure (without boundary if $M$ is without boundary), cf. \Cref{definition:tangent_bundle_of_a_C_k_manifold}.

% \hyperrefIfExists{definition:C_k_vector_bundle_on_C_k_manifold}{vector bundle}
% (also denoted by \hl{$\Omega M$}) where:
% \begin{itemize}
%   \item The total space $T^*M := \bigsqcup_{p \in M} T_p^* M$ is the disjoint union of cotangent spaces of $M$ at all points,
%   \item The projection map $\pi^* : T^*M \to M$ sends each covector to its base point,
%   \item Locally, on a chart $(U, \varphi)$, the cotangent bundle trivializes as
%   $$ (\pi^*)^{-1}(U) \cong U \times (\mathbb{R}^n)^* \cong U \times \mathbb{R}^n, $$
%   where $(\mathbb{R}^n)^*$ is the dual space of $\mathbb{R}^n$.
% \end{itemize}

% The \hldef{cotangent bundle of $M$} is the \hyperrefIfExists{definition:C_k_vector_bundle_on_C_k_manifold}{vector bundle}
% $$ (T^*M, \pi^*, M) $$
% (also denoted by \hl{$\Omega M$}) where:
% \begin{itemize}
%   \item The total space $T^*M := \bigsqcup_{p \in M} T_p^* M$ is the disjoint union of cotangent spaces of $M$ at all points,
%   \item The projection map $\pi^* : T^*M \to M$ sends each covector to its base point,
%   \item Locally, on a chart $(U, \varphi)$, the cotangent bundle trivializes as
%   $$ (\pi^*)^{-1}(U) \cong U \times (\mathbb{R}^n)^* \cong U \times \mathbb{R}^n, $$
%   where $(\mathbb{R}^n)^*$ is the dual space of $\mathbb{R}^n$.
% \end{itemize}
\end{definition}


\subsection{Exterior powers of real vector bundles}

In general, one can define exterior powers of (real) vector bundles on real manifolds. A case of interest would be the exterior powers of the (co)tangent bundle of a smooth manifold.

\begin{definition} \label{definition:symmetric_exterior_power_of_a_two_sided_module_over_a_ring}
    Let $R$ be a \CrefAndHyperrefIfExist{definition:ring}{(not necessarily commutative) ring}, and let $M$ an \CrefAndHyperrefIfExist{definition:module_of_a_ring}{two-sided $R$-module}.
    \begin{enumerate}
        \item The \hldef{symmetric power of $M$ of degree $n$}, denoted by \hl{$S_R^n(M)$} or \hl{$\Sym^n(M) = \Sym_R^n(M)$}, is the \CrefAndHyperrefIfExist{definition:quotient_module_of_a_module_by_a_module_of_a_ring}{quotient two-sided module}
        $$S_R^n(M) := M^{\otimes_R n} / I_{\mathrm{sym}},$$
        \CrefIfExists{definition:tensor_product_of_bimodules_of_rings} where $I_{\mathrm{sym}}$ is the \CrefAndHyperrefIfExist{definition:module_of_a_ring}{two-sided} \CrefAndHyperrefIfExist{definition:submodule_of_a_module_over_a_ring}{submodule} of $M^{\otimes_R n}$ \CrefAndHyperrefIfExist{definition:submodule_of_a_module_generated_by_elements}{generated by} all elements of the form
        $$x_1 \otimes \cdots \otimes x_n - (x_{\sigma(1)} \otimes \cdots \otimes x_{\sigma(n)}), \qquad x_i \in M, \; \sigma \in \mathfrak{S}_n.$$
        \CrefIfExists{definition:symmetric_group_on_a_set} 
        \CrefIfExists{definition:action_of_symmetric_group_on_tensor_power_on_a_two_sided_over_a_ring}
        

        \item The \hldef{exterior power of $M$ of degree $n$}, denoted by $\Lambda_R^n(M)$, is the \CrefAndHyperrefIfExist{definition:quotient_module_of_a_module_by_a_module_of_a_ring}{quotient two-sided module}
        $$\Lambda_R^n(M) := M^{\otimes_R n} / I_{\mathrm{alt}},$$
        where $I_{\mathrm{alt}}$ is two-sided \CrefAndHyperrefIfExist{definition:submodule_of_a_module_over_a_ring}{submodule} of $M^{\otimes_R n}$ \CrefAndHyperrefIfExist{definition:submodule_of_a_module_generated_by_elements}{generated by} all elements of the form
        $$x_1 \otimes \cdots \otimes x_n - \mathrm{sgn}(\sigma)\, (x_{\sigma(1)} \otimes \cdots \otimes x_{\sigma(n)}), \qquad x_i \in M, \; \sigma \in \mathfrak{S}_n.$$
        \CrefIfExists{definition:symmetric_group_on_a_set} 
        \CrefIfExists{definition:signature_of_an_element_of_the_symmetric_group_on_a_finite_set}
    \end{enumerate}

    In particular, we often speak of symmetric powers of exterior powers of modules over \CrefAndHyperrefIfExist{definition:commutative_ring}{commutative rings} and even \CrefAndHyperrefIfExist{definition:vector_space_over_a_field}{vector spaces} over \CrefAndHyperrefIfExist{definition:field}{fields}.
\end{definition}


\begin{definition}\label{definition:symmetric_power_of_a_vector_bundle_on_a_topological_manifold}
    Let $k \in \bbZ_{\geq 0} \cup \{\infty\}$. Let $\pi : E \to M$ be a \hyperrefIfExists{definition:C_k_vector_bundle_on_C_k_manifold}{real $C^k$-vector bundle} of rank $r$ over a \CrefAndHyperrefIfExist{definition:C_k_manifold}{$C^k$-topological manifold} $M$ with or without boundary, and let $l \in \{0,1,\ldots,r\}.$

    The \hldef{$l$th symmetric power of $E$}, denoted by \hl{$\Sym^l E$}, is the vector bundle over $M$...
    \TODO{complete}
\end{definition}


\begin{definition} \label{definition:kth_exterior_power_of_vector_bundle}
  \TODO{do this definition for manifolds with boundary}
Let $k \in \bbZ_{\geq 0} \cup \{\infty\}$. Let $\pi : E \to M$ be a \hyperrefIfExists{definition:C_k_vector_bundle_on_C_k_manifold}{real $C^k$-vector bundle} of rank $r$ over a \CrefAndHyperrefIfExist{definition:C_k_manifold}{$C^k$-topological manifold} $M$ with or without boundary, and let $l \in \{0,1,\ldots,r\}.$

\TODO{TODO: describe how to define a vector bundle fiberwise}
The \hldef{$l$th exterior power of $E$}, denoted by \hl{$\bigwedge^l E$}, is the \hyperrefIfExists{definition:C_k_vector_bundle_on_C_k_manifold}{vector bundle over} $M$ defined fiberwise by
$$ (\bigwedge^l E)_p := \bigwedge^l (E_p) $$
for each $p \in M$, where $E_p = \pi^{-1}(\{p\})$ is the fiber of $E$ at $p$ and $\bigwedge^l (E_p)$ is the \CrefAndHyperrefIfExist{definition:symmetric_exterior_power_of_a_two_sided_module_over_a_ring}{$l$th exterior power of the vector space} $E_p$.

The total space of $\bigwedge^l E$ is given the unique vector bundle structure for which local trivializations of $E$ induce local trivializations
$$ \bigwedge^l \Phi_U : \pi^{-1}(U) \to U \times \bigwedge^l(\mathbb{R}^r) $$
making $\bigwedge^l E$ a vector bundle of rank $\binom{r}{l}$ over $M$.
\end{definition}

\begin{proposition}
    Let $M$ be a $n$-dimensional (real) $C^k$-manifold for $k \in \bbN_0 \cup \{\infty\}$ and let $E$ be a $C^k$-vector bundle of rank $r$ on $M$. The \hyperrefIfExists{definition:kth_exterior_power_of_vector_space_over_a_field}{$l$-th exterior power} $\bigwedge^l E$ is a $C^k$-vector bundle of rank ${r\choose k}$.
\end{proposition}

\subsection{\texorpdfstring{$k$}{k}-forms on a smooth real manifold}


\begin{definition}\label{definition:bundle_of_k_forms_n_smooth_manifold}
Let $k \in \bbZ_{\geq 1} \cup \{\infty\}$, let $M$ be a $n$-dimensional \hyperrefIfExists{definition:C_k_manifold}{$C^k$-manifold} with or without boundary, and let $l \in \{0,\ldots,n\}$. Recall that $TM$ and $T^*M$ carry $C^{k-1}$-vector bundle structures over $M$ and hence are $C^{k-1}$-manifolds themselves.

The \hyperrefIfExists{definition:kth_exterior_power_of_vector_bundle}{$l$th exterior power} $\bigwedge^l \Omega M$ of the \hyperrefIfExists{definition:cotangent_bundle_of_a_smooth_manifold}{cotangent bundle} $T^* M = \Omega M$ is often denoted by \hl{$\Omega^l M$} and referred to as the \hldef{bundle of $l$-forms on $M$}; if $k = \infty$, then we might refer to such a $l$-form as a \hldef{smooth/differentiable $l$-form}.

We use notation such as \hl{$\Omega^l(M)$} and \hl{$\Omega^l(M;\bbR)$} to denote a \hyperrefIfExists{definition:C_k_vector_bundle_on_C_k_manifold}{space of sections} over $M$, i.e. $\Gamma(M, \Omega^l M)$, and call sections of such a space a \hldef{$l$-form on $M$}. --- unless otherwise specified, we take $\Omega^l(M)$ be the space $\Gamma^{C^{k-1}}(M, \Omega^l M)$ of $C^{k-1}$ $l$-forms.

In other words, a $l$-form on $M$ is a $C^{k-1}$-section $\omega$ of $\Omega^l M$, i.e. a $C^{k-1}$ assignment $p \mapsto \omega_p$ where $\omega_p:(T_p M)^l \to \mathbb{R}$ is an \CrefAndHyperrefIfExist{definition:symmetric_antisymmetric_alternating_multilienar_forms_over_rings}{alternating} \CrefAndHyperrefIfExist{definition:multilinear_map_of_modules_over_rings}{multilinear form}.

% that assigns to each point $p \in M$ an \CrefAndHyperrefIfExist{definition:symmetric_antisymmetric_alternating_multilienar_forms_over_rings}{alternating} \CrefAndHyperrefIfExist{definition:multilinear_map_of_modules_over_rings}{multilinear form}
% $$ \omega_p : (T_p M)^l \to \mathbb{R}.  $$
By convention, we let $\Omega^0(M)$ equal the space \hyperrefIfExists{definition:C_k_morphism_between_C_k_manifolds}{$C^{k-1}(M)$} of \hyperrefIfExists{definition:C_k_morphism_between_C_k_manifolds}{$C^{k-1}$ real-valued functions} on $M$.

Let \hl{$\Omega_c^l(M) = \Omega_c^l(M; \bbR) \subseteq \Omega^l(M)$} be the space of \CrefAndHyperrefIfExist{definition:support_of_a_section_of_a_C_k_vector_bundle_over_a_C_k_manifold}{compactly supported} ($C^{k-1}$ ) $l$-forms on $M$, i.e. 
$$ \Omega_c^l(M) = \Omega_c^l(M; \bbR) := \Gamma_c\left(M, \Omega^l M \right). $$
\end{definition}

% 
\begin{definition} \label{definition:k_form_on_a_smooth_manifold}
Let $M$ be a $n$-dimensional \hyperrefIfExists{definition:C_k_manifold}{$C^k$-manifold}. 

The set \hl{$\Omega^k(M)$} is the space of \hldef{(smooth differential) $k$-forms on $M$}, defined as
$$ \Omega^k(M) = \Omega^k(M; \bbR) := \Gamma\left(M, \Omega^k M \right), $$
i.e. the space of \hyperrefIfExists{definition:C_k_vector_bundle_on_C_k_manifold}{smooth sections} of the \hyperrefIfExists{definition:kth_exterior_power_of_vector_bundle}{$k$-th exterior power} of the \hyperrefIfExists{definition:cotangent_bundle_of_a_smooth_manifold}{cotangent bundle}.

\end{definition}

\TODO{TODO: read the following four definitions}

\begin{definition}\label{definition:exterior_derivative_on_smooth_k_forms_on_a_smooth_manifold}
Let $k \in \bbZ_{\geq 2} \cup \{\infty\}$, let $M$ be a $n$-dimensional \hyperrefIfExists{definition:C_k_manifold}{$C^k$ manifold} with or without boundary, and let $U \subseteq M$ be open.

For each $0 \leq m$, the \hldef{exterior derivative} is a map
$$\hlin{d: \Omega^m(U) \to \Omega^{m+1}(U)}$$
with $\Omega^m(U)$ and $\Omega^{m+1}(U)$ consisting of \CrefAndHyperrefIfExist{definition:C_k_vector_bundle_on_C_k_manifold}{$C^{k-1}$-sections} of $\Omega^m M$ and $\Omega^{m+1} M$ over $U$ respectively\CrefIfExists{definition:bundle_of_k_forms_n_smooth_manifold}; we note however, that the range of $d$ consists of $C^{k-2}$-sections of $\Omega^{m+1} M$. The exterior derivative is the unique $\mathbb{R}$-linear map satisfying:
\begin{itemize}
    \item For any function $f \in C^{k-1}(U) = \Omega^0(U)$, the exterior derivative $df \in \Omega^1(U)$ is the usual differential of $f$: for every smooth vector field $X$ on $U$,
    \[ df(X) = X(f), \]
    and, in local coordinates $(x^1, \ldots, x^n)$,
    \TODO{partial derivatives}
    \[ df = \frac{\partial f}{\partial x^1} dx^1 + \cdots + \frac{\partial f}{\partial x^n} dx^n.  \]

    \item For all $\omega \in \Omega^m(U)$, $\eta \in \Omega^n(U)$,
    \[ d(\omega \wedge \eta) = d\omega \wedge \eta + (-1)^m \omega \wedge d\eta \]
    (graded Leibniz rule).

    \item Assuming that $k \geq 3$, for all $\omega \in \Omega^m(U)$,
    \[ d(d\omega) = 0.  \]
\end{itemize}
Moreover, if $\omega \in \Omega_c^m(U)$, then $d\omega \in \Omega_c^{m+1}(U)$, i.e., the exterior derivative preserves \CrefAndHyperrefIfExist{definition:support_of_a_section_of_a_C_k_vector_bundle_over_a_C_k_manifold}{compactly supported forms}.
\end{definition}


\begin{definition}
Let $k \in \bbZ_{\geq 2} \cup \{\infty\}$, let $M$ be a $C^k$ manifold with or without boundary, and $U \subset M$ open. 

\begin{enumerate}
    \item The \hldef{de Rham complex} on $U$ is the sequence of $\mathbb{R}$-vector spaces and maps:
    $$ 0 \to \Omega^0(U) \xrightarrow{d} \Omega^1(U) \xrightarrow{d} \Omega^2(U) \xrightarrow{d} \cdots \xrightarrow{d} \Omega^{\min(\dim M, k-1)}(U) \to 0 $$
    where each $d$ is the exterior derivative.

    \item The \hldef{de Rham complex of compactly supported differential forms} on $U$ is the sequence of $\mathbb{R}$-vector spaces and maps:
    $$ 0 \to \Omega_c^0(U) \xrightarrow{d} \Omega_c^1(U) \xrightarrow{d} \Omega_c^2(U) \xrightarrow{d} \cdots \xrightarrow{d} \Omega_c^{\min(\dim M, k-1)}(U) \to 0 $$
    where each $d$ is the exterior derivative as in the previous notation, and the image of $d$ is contained in forms with compact support because $d$ preserves compact support.
\end{enumerate}
Both are \CrefAndHyperrefIfExist{definition:chain_complex_of_objects_in_an_additive_category}{chain complexes} of $\bbR$-vector spaces.

\end{definition}


\section{Complex Manifolds}

\begin{definition}[Holomorphic function from a subset of $\mathbb{C}^n$ to $\mathbb{C}^m$] \label{definition:holomorphic_function_from_a_subset_of_Cn_to_Cm}
Let $n,m \geq 1$ be integers, and consider $\mathbb{C}^n$ and $\mathbb{C}^m$ as complex vector spaces endowed with their standard Euclidean topologies. Let $U \subseteq \mathbb{C}^n$ be an open subset, and let $f : U \to \mathbb{C}^m$ be a function.
  \begin{enumerate}
    \item We say that $f$ is \hldef{holomorphic/(complex) analytic/complex differentiable function at a point $a \in U$} if there exists a $\mathbb{C}$-linear map
    $$ L_a : \mathbb{C}^n \to \mathbb{C}^m $$
    \TODO{norms on $\bbC^n$ are equivalent to one another}
    such that, for some (and hence any) norm $\|\cdot\|$ on $\mathbb{C}^n$ and $\mathbb{C}^m$, one has
    $$ \lim_{h \to 0} \frac{\| f(a+h) - f(a) - L_a(h) \|}{\|h\|} = 0, $$
    where the \CrefAndHyperrefIfExist{definition:limit_of_a_function_between_extended_metric_spaces_at_an_accumulation_point}{limit} is taken over $h \in \mathbb{C}^n$ with $a+h \in U$.
    In this case, the map $L_a$ is uniquely determined and is called the \hldef{complex derivative} (or \hldef{complex Fréchet derivative}) of $f$ at $a$. 
    %The \hldef{complex derivative of $f$ at $a$} may also refer to the value $L_a(0)$. 

    In the case of $n,m = 1$, we $f$ is holomorphic at $a$ if and only if the \CrefAndHyperrefIfExist{definition:limit_of_a_function_between_extended_metric_spaces_at_an_accumulation_point}{limit}
    \[
    \lim_{z \to z_0} \frac{f(z) - f(z_0)}{z - z_0}
    \]
    exists in $\mathbb{C}$. In that case the limit is denoted \hl{$f'(z_0)$} and is called the \hldef{complex derivative of $f$ at $z_0$}.
    
    the value $L_a(0)$ complex derivative 

    \item The function $f$ is called \hldef{holomorphic/analytic on $U$} if it is holomorphic at every point $a \in U$.

    \item A holomorphic function $\bbC^n \to \bbC^m$ is called \hldef{entire}.

  \end{enumerate}
\end{definition}
\begin{definition}[Biholomorphic map of subsets of $\mathbb{C}$] \label{definition:biholomorphic_map_of_open_subsets_of_complex_space}
Let $U\subseteq \bbC^n$ and $V \subseteq \bbC^m$ be open subsets for integer $n,m \geq 1$.
A map $f : U \to V$ is called a \hldef{biholomorphic map of subsets of $\mathbb{C}$}, or simply a \hldef{biholomorphism between $U$ and $V$}, if:
\begin{enumerate}
  \item $f$ is \CrefAndHyperrefIfExist{definition:injective_surjective_bijective_map_of_sets}{bijective} as a map of sets,
  \item $f$ and \CrefAndHyperrefIfExist{definition:injective_surjective_bijective_map_of_sets}{$f^{-1}$} are \CrefAndHyperrefIfExist{definition:holomorphic_function_from_a_subset_of_Cn_to_Cm}{holomorphic} on $U$.
\end{enumerate}
In this case, we say that $U$ and $V$ are \hldef{(complex-)analytically isomorphic as open subsets of $\mathbb{C}$}. Moreover, $n$ and $m$ must necessarily be equal in this case.
\end{definition}
\begin{definition}[Complex Manifold] \label{definition:complex_analytic_manifold} \

    \begin{enumerate}
        \item A \hldef{complex (analytic) manifold $M$ of complex dimension $n$ (without boundary)} is a \CrefAndHyperrefIfExist{definition:topological_manifold}{topological manifold} of dimension $2n$ equipped with an \CrefAndHyperrefIfExist{definition:atlas_on_a_topological_manifold}{atlas} of \CrefAndHyperrefIfExist{definition:chart_on_a_topological_manifold}{charts} \(\{(U_\alpha, \varphi_\alpha)\}\) where:
        \begin{itemize}
            % \item Each \(U_\alpha \subseteq M\) is open and covers \(M\),
            \item Each chart \(\varphi_\alpha : U_\alpha \to \Omega_\alpha \subseteq \bbR^{2n}\) is regarded as a \CrefAndHyperrefIfExist{definition:homeomorphism_of_topological_spaces}{homeomorphism} onto an open subset of \(\mathbb{C}^n\) by homeomorphically identifying $\bbC^n$ with $\bbR^{2n}$,
            \item The \CrefAndHyperrefIfExist{definition:transition_map_between_charts_on_a_topological_manifold}{transition maps} \(\varphi_\beta \circ \varphi_\alpha^{-1} : \varphi_\alpha(U_\alpha \cap U_\beta) \to \varphi_\beta(U_\alpha \cap U_\beta)\) are \CrefAndHyperrefIfExist{definition:biholomorphic_map_of_open_subsets_of_complex_space}{biholomorphic maps} between open subsets of \(\mathbb{C}^n\).

        \end{itemize}
        An equivalent definition of a complex manifold $M$ of complex dimension $n$ is a \CrefAndHyperrefIfExist{definition:separation_axioms_of_topology}{Hausdorff} \CrefAndHyperrefIfExist{definition:first_and_second_countable_topological_space}{second countable} topological space equipped with an atlas of charts \(\{(U_\alpha, \varphi_\alpha)\}\) where:
        \begin{itemize}
            \item Each \(U_\alpha \subseteq M\) is open and covers \(M\),
            \item Each chart \(\varphi_\alpha : U_\alpha \to \Omega_\alpha \subseteq \mathbb{C}^n\) is a \CrefAndHyperrefIfExist{definition:homeomorphism_of_topological_spaces}{homeomorphism} onto an open subset of \(\mathbb{C}^n\),
            \item The transition maps \(\varphi_\beta \circ \varphi_\alpha^{-1} : \varphi_\alpha(U_\alpha \cap U_\beta) \to \varphi_\beta(U_\alpha \cap U_\beta)\) are \CrefAndHyperrefIfExist{definition:biholomorphic_map_of_open_subsets_of_complex_space}{biholomorphic maps} between open subsets of \(\mathbb{C}^n\).
        \end{itemize}
        Equivalently, \(M\) is a locally ringed space locally isomorphic to \((\Omega, \mathcal{O}_\Omega)\) where \(\Omega \subseteq \mathbb{C}^n\) is open and \(\mathcal{O}_\Omega\) is the sheaf of holomorphic functions.

        \item A \hldef{complex manifold $M$ of complex dimension $n$ with boundary} is a \CrefAndHyperrefIfExist{definition:topological_manifold}{topological manifold with boundary} of dimension $2n$ eqiupped with an \CrefAndHyperrefIfExist{definition:atlas_on_a_topological_manifold}{atlas} of \CrefAndHyperrefIfExist{definition:chart_on_a_topological_manifold}{charts} $\{(U_\alpha, \varphi_\alpha)\}$ where

        \begin{itemize}
            \item Each chart \(\varphi_\alpha : U_\alpha \to \Omega_\alpha \subseteq \bbR^{2n}_+\), which is a map into the $2n$-dimensional \CrefAndHyperrefIfExist{definition:closed_half_space_in_euclidean_space}{Euclidean closed half-space} $\bbR^{2n}_+$,  is regarded as a \CrefAndHyperrefIfExist{definition:homeomorphism_of_topological_spaces}{homeomorphism} onto an open subset of the complex closed upper half-space \CrefAndHyperrefIfExist{definition:closed_upper_half_space_in_C_n}{$\mathbb{H}^n$} by homeomorphically identifying $\bbH^n$ with $\bbR^{2n}_+$,
            \item The \CrefAndHyperrefIfExist{definition:transition_map_between_charts_on_a_topological_manifold}{transition maps} \(\varphi_\beta \circ \varphi_\alpha^{-1} : \varphi_\alpha(U_\alpha \cap U_\beta) \to \varphi_\beta(U_\alpha \cap U_\beta)\) are extendable to \CrefAndHyperrefIfExist{definition:biholomorphic_map_of_open_subsets_of_complex_space}{biholomorphic maps} between open subsets of \(\mathbb{C}^n\). More precisely, there exist open subsets $\Omega_\alpha, \Omega_\beta$ of $\bbC^n$ containing $\varphi_\alpha(U_\alpha \cap U_\beta)$ and $\varphi_\beta(U_\alpha \cap U_\beta)$ respectively and a biholomorphic map $f_{\alpha\beta}: \Omega_\alpha \to \Omega_\beta$ such that $f_{\alpha\beta}$ restricts to $\varphi_\beta \circ \varphi_\alpha^{-1}$ on  $\varphi_\alpha(U_\alpha \cap U_\beta)$.
        \end{itemize}
        % An equivalent definition of a complex manifold $M$ of complex dimension $n$ is a \CrefAndHyperrefIfExist{definition:separation_axioms_of_topology}{Hausdorff} \CrefAndHyperrefIfExist{definition:first_and_second_countable_topological_space}{second countable} topological space equipped with an atlas of charts \(\{(U_\alpha, \varphi_\alpha)\}\) where:
        
    \end{enumerate}


\end{definition}

\begin{definition} \label{definition:almost_complex_structure_on_a_smooth_manifold_of_even_real_dimension}
    Let $M$ be a \hyperrefIfExists{definition:C_k_manifold}{smooth manifold} of real dimension $2n$ with or without boundary. An \hldef{almost complex structure on $M$} is a smooth \CrefAndHyperrefIfExist{definition:morphism_of_C_k_vector_bundles_over_a_C_k_manifold_with_or_without_boundary}{vector bundle endomorphism}
    $$ J : TM \to TM $$
    of the \CrefAndHyperrefIfExist{definition:tangent_bundle_of_a_C_k_manifold}{tangent bundle $TM$} of $M$ such that
    $$ J^2 = - \mathrm{id}_{TM}.  $$
\end{definition}

\begin{definition} \label{definition:integrable_almost_complex_structure_on_a_smooth_manifold_of_even_real_dimension}
    \TODO{TODO: Lie bracket of vector fields}
    Let $M$ be a \hyperrefIfExists{definition:C_k_manifold}{smooth manifold} of real dimension $2n$ with or without boundary equipped with an \hyperrefIfExists{definition:almost_complex_structure_on_a_smooth_manifold_of_even_real_dimension}{almost complex structure}
    $$
    J : TM \to TM, \quad J^2 = - \mathrm{id}_{TM}.
    $$
    We say that $J$ is \hldef{integrable} if the Nijenhuis tensor
    $$ \hlin{N_J (X, Y) := [X, Y] + J([JX, Y]) + J([X, JY]) - [JX, JY]}$$
    vanishes identically for all smooth vector fields $X, Y \in \Gamma(TM)$,
    where $[ \cdot , \cdot ]$ is the Lie bracket of vector fields.
\end{definition}

\begin{proposition} \label{proposition:complex_manifold_is_equivalent_to_a_smooth_manifold_with_an_integrable_almost_complex_structure}
    Let $M$ be a \CrefAndHyperrefIfExist{definition:topological_manifold}{topological manifold without boundary}. The following are equivalent:
    \begin{enumerate}
        \item $M$ is a \CrefAndHyperrefIfExist{definition:complex_analytic_manifold}{complex manifold}.
        \item $M$ is a \CrefAndHyperrefIfExist{definition:C_k_manifold}{smooth manifold} equipped with an \CrefAndHyperrefIfExist{definition:integrable_almost_complex_structure_on_a_smooth_manifold_of_even_real_dimension}{integrable} \CrefAndHyperrefIfExist{definition:almost_complex_structure_on_a_smooth_manifold_of_even_real_dimension}{almost complex structure}. 
    \end{enumerate}
\end{proposition}

\section{Vector fields}

\begin{definition}[Local and global vector fields on a $C^k$ manifold] \label{definition:local_and_global_vector_field_on_a_C_k_manifold}
Let $k \in \bbZ_{\geq 1} \cup \{\infty\}$, let $M$ be a \CrefAndHyperrefIfExist{definition:C_k_manifold}{$C^k$ manifold (possibly with boundary)}, and let
$$ \pi : TM \to M $$
denote its \CrefAndHyperrefIfExist{definition:tangent_bundle_of_a_C_k_manifold}{tangent bundle}, which is a $C^{k-1}$-manifold.

\begin{enumerate}
  \item Let $U \subseteq M$ be an open subset.
  A \hldef{local $C^{k-1}$ vector field on $U$} is a \CrefAndHyperrefIfExist{definition:C_k_vector_bundle_on_C_k_manifold}{$C^{k-1}$-section} over $U$, i.e. a \CrefAndHyperrefIfExist{definition:C_k_morphism_between_C_k_manifolds}{$C^{k-1}$ map}
  $$ X : U \to TM $$
  of $C^{k-1}$-manifolds such that $\pi \circ X = \iota_U$, where $\iota_U : U \hookrightarrow M$ is the inclusion map.
  Equivalently, $X$ is a $C^k$ section of the restricted bundle $TM|_U \to U$.

  \item A \hldef{global $C^{k-1}$ vector field on $M$} (or simply a \hldef{$C^{k-1}$ vector field on $M$}) is a local $C^k$ vector field on $U = M$, i.e. a \CrefAndHyperrefIfExist{definition:C_k_vector_bundle_on_C_k_manifold}{global $C^{k-1}$-section}.
\end{enumerate}
\end{definition}


\begin{definition}[Local and global frames of the tangent bundle of a differentiable manifold] \label{definition:local_and_global_frames_of_the_tangent_bundle_of_a_differentiable_manifold}
Let $k \in \mathbb{Z}_{\geq 1} \cup \{\infty\}$, let $M$ be a $C^k$ manifold of (pure) dimension $n$, and let \CrefAndHyperrefIfExist{definition:tangent_bundle_of_a_C_k_manifold}{$TM$ be its tangent bundle}.

\begin{enumerate}
  \item Let $U \subseteq M$ be a nonempty open subset.
  A \hldef{local $C^k$ frame of $TM$ over $U$} (or simply a \hldef{local frame over $U$}) is an ordered $n$-tuple of \CrefAndHyperrefIfExist{definition:local_and_global_vector_field_on_a_C_k_manifold}{local $C^k$ vector fields}
  $$ (X_1,\dots,X_n) \in \Gamma^k(TU)^n $$
  \CrefIfExists{definition:C_k_vector_bundle_on_C_k_manifold} such that for every point $p \in U$ the $n$-tuple
  $$ (X_1(p),\dots,X_n(p)) $$
  is a \CrefAndHyperrefIfExist{definition:basis_of_a_vector_space_over_a_field}{basis} of the $n$-dimensional real vector space $T_pM$.

  \item A \hldef{global $C^k$ frame of $TM$} (or simply a \hldef{global frame}) is a local $C^k$ frame over $U = M$, i.e.\ an ordered $n$-tuple of global $C^k$ vector fields
  $$
  (X_1,\dots,X_n) \in \Gamma^k(TM)^n
  $$
  such that $(X_1(p),\dots,X_n(p))$ is a basis of $T_pM$ for every $p \in M$.
\end{enumerate}
\end{definition}



\section{Orientation on manifolds}

\begin{definition}[Orientation of a real vector space] \label{definition:orientation_of_a_real_vector_space}
Let $V$ be a finite-dimensional real vector space of dimension $n \geq 1$.
An \hldef{orientation of $V$} is an equivalence class of ordered bases of $V$ under the following equivalence relation:
two ordered bases
$$ (v_1,\dots,v_n) \quad \text{and} \quad (w_1,\dots,w_n) $$
of $V$ are declared equivalent if the unique linear automorphism
$$ T : V \to V $$
\TODO{determinant}
satisfying $T(v_i) = w_i$ for all $i = 1,\dots,n$ has positive determinant with respect to (equivalently, in any) choice of identification of $V$ with $\mathbb{R}^n$.

Equivalently, fix any ordered basis $(e_1,\dots,e_n)$ of $V$, and declare that an ordered basis $(v_1,\dots,v_n)$ of $V$ is \hldef{positively oriented with respect to $(e_1,\ldots,e_n)$} if the determinant of the change-of-basis matrix from $(e_1,\dots,e_n)$ to $(v_1,\dots,v_n)$ is positive.
This defines an equivalence relation on the set of ordered bases of $V$ with exactly two equivalence classes, called the \hldef{orientations of $V$}.
A choice of one of these two classes is an orientation of $V$.

A \hldef{oriented real vector space} is a pair $(V,o)$ where $V$ is a finite-dimensional real vector space and $o$ is a chosen orientation of $V$ in the above sense.
\end{definition}



\TODO{TODO: overall defining orientations is still very confusing to me, so I will have to fix definitions}

\begin{definition}[Standard Orientation on $\mathbb{R}^n$] \label{definition:standard_orientation_on_R_n}
  Let $n \in \mathbb{N}$ and consider the Euclidean space $\mathbb{R}^n$ equipped with the standard ordered basis
  \[
    \mathcal{E} = (e_1, e_2, \ldots, e_n),
  \]
  where each $e_i$ is the vector with a $1$ in the $i$th coordinate and $0$ elsewhere.

  The \hldef{standard orientation on $\mathbb{R}^n$} is the \CrefAndHyperrefIfExist{definition:orientation_of_a_real_vector_space}{orientation} defined by declaring the ordered basis $\mathcal{E}$ to be positively oriented. More precisely, since for $n$-dimensional vector spaces the orientation is given by equivalence classes of ordered bases modulo the sign of the determinant of the change of basis matrix, an ordered basis $(v_1, \ldots, v_n)$ of $\mathbb{R}^n$ is said to be \hldef{positively oriented} if and only if
  $$
    \det([v_1 \; v_2 \; \cdots \; v_n]) > 0,
  $$
  where $[v_1 \; v_2 \; \cdots \; v_n]$ is the matrix whose columns are the vectors $v_i$ written in the standard basis $\mathcal{E}$.

  Thus, the standard orientation on $\mathbb{R}^n$ is the equivalence class of ordered bases that yield a positive determinant with respect to the standard basis $\mathcal{E}$.
\end{definition}



\begin{definition}[Pointwise orientation of a manifold] \label{definition:orientation_of_a_differentiable_manifold_with_or_without_boundary}
Let $n \geq 0$ be an integer, let $k \in \bbZ_{\geq 1} \cup \{\infty\}$, and let $M$ be a \CrefAndHyperrefIfExist{definition:C_k_manifold}{$C^k$-manifold} of (pure) dimension $n$ with or without boundary. 
\begin{enumerate}
    \item An \hldef{orientation on $M$ at $p$} is a choice of \CrefAndHyperrefIfExist{definition:orientation_of_a_real_vector_space}{orientation} of the \CrefAndHyperrefIfExist{definition:tangent_space_of_a_topological_manifold_at_a_point}{tangent space $T_p M$}.
    \item A \hldef{pointwise orientation of $M$} is a choice of orientation of $M$ at each $p \in M$. 
    \item Say that $M$ is equipped with a pointwise orientation. A \CrefAndHyperrefIfExist{definition:local_and_global_frames_of_the_tangent_bundle_of_a_differentiable_manifold}{local frame} $(E_i)$ for $TM$ is said to be 
    \begin{enumerate}
        \item \hldef{(positively) oriented} if $(E_1|_p,\ldots,E_n|_p)$ is \CrefAndHyperrefIfExist{definition:orientation_of_a_real_vector_space}{positively oriented} basis for $T_p M$ at each $p \in U$ with respect to the pointwise orientation. 
        \item \hldef{negative oriented} if $(E_1|_p,\ldots,E_n|_p)$ is \CrefAndHyperrefIfExist{definition:orientation_of_a_real_vector_space}{negatively oriented} basis for $T_p M$ at each $p \in U$ with respect to the pointwise orientation. 
    \end{enumerate}
    \item A pointwise orientation of $M$ is said to be \hldef{continuous} if every point of $M$ is in the domain of an oriented local frame. 
    \item An \hldef{orientation of $M$} is a continuous pointwise orientation.

    \item $M$ is said to be \hldef{orientable} if there exists an orientation of $M$. Otherwise, $M$ is said to be \hldef{nonorientable}

    \item An \hldef{oriented manifold (with or without boundary)} is an ordered pair $(M,\calO)$ of an orientable $C^k$ manifold and $\calO$ is a choice of orientation for $M$.
\end{enumerate}
% A \hldef{}

% Fix a point $p \in M$.
% The local homology group
% $$
% H_n(M, M \setminus \{p\}; \mathbb{Z})
% $$
% is canonically isomorphic to $\mathbb{Z}$.
% A \hldef{pointwise orientation of $M$ at $p$} (also called a \hldef{local orientation of $M$ at $p$}) is a choice of a generator of this rank-one $\mathbb{Z}$-module.
% Equivalently, a pointwise orientation at $p$ is a choice of one of the two elements of
% $$
% H_n(M, M \setminus \{p\}; \mathbb{Z}) \setminus \{0\},
% $$
% modulo multiplication by $-1$.
\end{definition}



% \begin{definition} \label{definition:canonical_orientation_of_a_chart_on_a_topological_manifold}
% Let $M$ be an $n$-dimensional topological manifold, and let $(U, \varphi)$ be a \CrefAndHyperrefIfExist{definition:chart_on_a_topological_manifold}{chart on $M$}.

% % The \hldef{orientation of the chart $(U,\varhpi)$} is the orientation on $U$ obtained by pulling back the standard

% An \hldef{orientation for the chart $(U, \varphi)$} is a choice of equivalence class of ordered bases of $\mathbb{R}^n$ at each point of $V$ consistent under orientation-preserving homeomorphisms of $V$.

% More concretely, such an orientation for $(U, \varphi)$ is an equivalence class determined by the \hyperrefIfExists{definition:standard_orientation_on_R_n}{standard orientation of $\mathbb{R}^n$}. The chart $(U, \varphi)$ is called \hldef{oriented} if $\varphi$ is orientation-preserving, i.e., its transition maps with other charts respect the chosen orientation.
% \end{definition}

% \begin{definition}
% Let $M$ be an $n$-dimensional topological manifold. An \hldef{oriented atlas} on $M$ is a collection $\mathcal{A} = \{ (U_\alpha, \varphi_\alpha) \}_{\alpha \in A}$ of oriented charts covering $M$ such that for every pair of charts $(U_\alpha, \varphi_\alpha)$ and $(U_\beta, \varphi_\beta)$ with $U_\alpha \cap U_\beta \neq \emptyset$, the transition function 
% $$
% \varphi_\beta \circ \varphi_\alpha^{-1} : \varphi_\alpha(U_\alpha \cap U_\beta) \to \varphi_\beta(U_\alpha \cap U_\beta)
% $$
% is an orientation-preserving homeomorphism of open subsets of $\mathbb{R}^n$.

% The atlas $\mathcal{A}$ is called \emph{maximal oriented} if it is maximal with respect to inclusion among oriented atlases on $M$, i.e., it contains every oriented chart compatible with all charts in $\mathcal{A}$.
% \end{definition}

% \begin{definition}
% Let $M$ be an $n$-dimensional topological manifold. An \hldef{orientation on $M$} is a choice of a maximal oriented atlas on $M$.

% The pair $(M, \mathcal{A})$, where $\mathcal{A}$ is a maximal oriented atlas, is called an \hldef{oriented topological manifold}.

% By definition, $M$ is \emph{orientable} if there exists at least one orientation on $M$, i.e., at least one maximal oriented atlas exists. The orientation fixes a consistent "choice of orientation" throughout the manifold.
% \end{definition}



\section{Poincar\'e duality}

\TODO{read the following formulations of Poincare duality}
\TODO{give formulations of poincare duality on a non-compact manifold}

\begin{theorem}[Classical Poincaré Duality]
    \TODO{define oriented manifold, cap product, fundamental class, singular homology, cohomology}
Let $M$ be a connected, closed (i.e., compact without boundary), oriented \CrefAndHyperref{definition:topological_manifold}{topological manifold} of dimension $n$, and let $R$ be a commutative ring with unity. Then for each $k \in \{0, \ldots, n\}$, there is an isomorphism
$$ H^k(M;R) \cong H_{n-k}(M;R) $$
induced by the cap product with the fundamental class $[M] \in H_n(M;R)$.
\end{theorem}

\begin{theorem}[Perfect Pairing]
    \TODO{define $\smile$, $\langle, \rangle$, singular homology, cohomology}
Under the hypotheses of the previous theorem, for each $k \in \{0, \ldots, n\}$, the bilinear pairing
$
H^k(M;R) \times H^{n-k}(M;R) \to R, \quad (\alpha, \beta) \mapsto \langle \alpha \smile \beta, [M] \rangle
$
is nondegenerate (i.e., perfect).
\end{theorem}

\begin{proposition}[Relative Poincaré Duality]
    \TODO{define relative cohomology, relative fundamental class}
Let $M$ be a compact, oriented $n$-manifold with (possibly nonempty) boundary $\partial M$, and let $R$ be a commutative ring with unity. For each $k \in \{0, \ldots, n\}$, there is an isomorphism
$
H^k(M, \partial M; R) \cong H_{n - k}(M; R)
$
again induced via cap product with the relative fundamental class $[M, \partial M] \in H_n(M, \partial M; R)$.
\end{proposition}

\begin{proposition}[Poincaré Duality for Compact Supports]
Let $X$ be a connected, oriented $n$-dimensional manifold (not necessarily compact) and $R$ a commutative ring with unity. Then for each $k \in \{0,\ldots,n\}$, the cap product with the fundamental class in Borel–Moore homology induces an isomorphism
$
H_c^k(X; R) \cong H_{n - k}^{BM}(X; R).
$
\end{proposition}

\begin{proposition}[Poincaré Duality Groups]
Let $G$ be a group of type FP (i.e., admits a finite projective resolution as a $\mathbb{Z}[G]$-module), and let $n$ be an integer. $G$ is said to be an $n$-dimensional Poincaré duality group over a commutative ring $R$ with dualizing module $D$ if there exists a class $\mu \in H^n(G; D)$ such that cap product with $\mu$ induces isomorphisms
$
H^k(G; M) \cong H_{n - k}(G; D \otimes_R M)
$
for all (left) $R[G]$-modules $M$ and all $k \geq 0$.
\end{proposition}

\begin{corollary}[Betti Number Symmetry]
Let $M$ be a closed, connected, oriented $n$-manifold. Then the $k$-th Betti number equals the $(n - k)$-th Betti number:
$
\operatorname{rank}_R H^k(M; R) = \operatorname{rank}_R H^{n-k}(M; R).
$
\end{corollary}

% Intuitive statement (for exposition, not inside a formal environment):
% Intuitively, Poincaré duality expresses the deep geometric fact that $k$-dimensional cycles and $(n-k)$-dimensional cocycles are in bijection via intersection, with the fundamental class acting as the mediator. This duality underlies the symmetry of Betti numbers and the perfectness of cohomological pairings in oriented manifolds.



% \begin{definition}
% Let $k \in \bbZ_{\geq 1} \cup \{\infty\}$, and let $M$ be a $C^k$ manifold. 

% The assignment $U \mapsto \Omega_c^m(U)$ for open $U\subseteq M$ defines a presheaf on $M$ taking values in $\mathbb{R}$-vector spaces, by restriction.
% Taking the direct sum over $m$, and $d$ as above, gives a sequence of sheaves of $\mathbb{R}$-vector spaces:
% $$
% 0 \to \mathcal{F}^0 \xrightarrow{d} \mathcal{F}^1 \xrightarrow{d} \cdots \xrightarrow{d} \mathcal{F}^{\dim M} \to 0
% $$
% where $\mathcal{F}^m$ denotes the sheaf sending $U$ to $\Omega_c^m(U)$. This is called the \hldef{de Rham complex of sheaves of compactly supported differential forms}.
% \end{definition}


\appendix
\section{Set Theory}
\begin{definition} \label{definition:injective_surjective_bijective_map_of_sets}
Let $X$ and $Y$ be sets and let $f: X \to Y$ be a function.
\begin{itemize}
  \item The function $f$ is said to be \hldef{injective} (or \hldef{one-to-one}) if for all $x_1, x_2 \in X$, $f(x_1) = f(x_2)$ implies $x_1 = x_2$.
  \item The function $f$ is said to be \hldef{surjective} (or \hldef{onto}) if for every $y \in Y$ there exists $x \in X$ such that $f(x) = y$.

  \item The map $f$ is \hldef{bijective} if it is both injective and surjective. In this case, there exists a unique \hldef{inverse map} \hl{$f^{-1} : Y \to X$} such that for all $x \in X$ and $y \in Y$,
    $$f^{-1}(f(x)) = x \text{ and } f(f^{-1}(y)) = y.$$
\end{itemize}
\end{definition}

\section{Linear Algebra}
\begin{definition}[Vector space over a field] \label{definition:vector_space_over_a_field}
    Let $(k,+,\cdot)$ be a \CrefAndHyperrefIfExist{definition:field}{field}. A \hldef{vector space over $k$} or a \hldef{$k$-vector space} is a triple $(V,+,\cdot)$\footnote{Note that $+$ and $\cdot$ are abuse of notation here as these are already used for the addition and multiplication of $\cdot$} where 
    \begin{enumerate}
        \item $(V,+)$ is an abelian group, and
        \item $\cdot$ is a map $k \times V \to V$, called \hldef{scalar multiplication}
    \end{enumerate}
    such that the following axioms hold for all $a,b \in k$ and all $u,v \in V$:

    1. (Compatibility with field multiplication)  
    $$ (ab)\cdot v = a \cdot (b \cdot v). $$

    2. (Identity scalar)  
    $$ 1 \cdot v = v. $$

    3. (Distributivity over vector addition)  
    $$ a \cdot (u+v) = a \cdot u + a \cdot v. $$

    4. (Distributivity over scalar addition)  
    $$ (a+b) \cdot v = a \cdot v + b \cdot v. $$
\end{definition}

\begin{definition} \label{definition:basis_of_a_vector_space_over_a_field}
Let $F$ be a \CrefAndHyperrefIfExist{definition:field}{field}, and let $V$ be an \CrefAndHyperrefIfExist{definition:vector_space_over_a_field}{$F$-vector space}.
A subset $B \subseteq V$ is called a \hldef{basis of $V$} if: (i) $B$ is \CrefAndHyperrefIfExist{definition:linearly_independent_elements_of_a_module_over_a_ring}{linearly independent} over $F$, and (ii) $B$ \CrefAndHyperrefIfExist{definition:span_a_module_over_a_ring_for_elements_of_the_module}{spans} $V$.

If $B$ is a basis, we define the \hldef{dimension of $V$ over $F$} (or \hldef{rank of $V$ over $F$}), denoted by
$$ \hlin{\dim_F(V)}, $$
\TODO{cardinality}
to be the cardinality of $B$.
This value is uniquely determined by $V$ and $F$.
\end{definition}
\begin{definition} \label{definition:morphism_of_vector_spaces}
Let $F$ be a \CrefAndHyperrefIfExist{definition:field}{field}, and let $V$ and $W$ be \CrefAndHyperrefIfExist{definition:vector_space_over_a_field}{$F$-vector spaces}.
A function $T : V \to W$ is called a \hldef{(homo)morphism of vector spaces over $F$}, or an \hldef{$F$-linear map}, if for all $u,v \in V$ and all $a,b \in F$, we have
$$ T(au + bv) = aT(u) + bT(v).  $$
The set of all such morphisms from $V$ to $W$ is denoted by
$$\hlin{\operatorname{Hom}_F(V,W)}.$$
\end{definition}

\section{Abstract algebra}


\begin{definition}\label{definition:ring}
    A \hldef{ring} is a triple $(R, +, \cdot)$ where 
    \begin{enumerate}
        \item $(R,+)$ is a \CrefAndHyperrefIfExist{definition:group}{commutative group}, and
        \item $(R, \cdot)$ is a \CrefAndHyperrefIfExist{definition:monoid}{monoid}. 
        \item $\cdot$ is distributive over $+$, i.e. for all $a,b,c \in R$, we have
        $$a \cdot (b+c) = a \cdot b + a \cdot c \quad \text{and} \quad (a+b) \cdot c = a \cdot c + b \cdot c.$$
    \end{enumerate}

    Equivalently, a ring is a triple $(R,+,\cdot)$ where $+,\cdot: R \times R \to R$ are binary operations satisfying
    \begin{enumerate}
        \item $(a+b)+c = a+(b+c)$ and $(ab)c = a(bc)$ for all $a,b,c \in R$
        \item There exists an element \hl{$0 \in R$} such that $a+0 = a = 0 + a$ for all $a \in R$.
        \item For every $a \in R$, there exists an element \hl{$-a \in R$} such that $a+(-a) = 0 = (-a) + a$ for all $a \in R$.
        \item There exists an element \hl{$1 \in R$} such that $a \cdot 1 = a = 1 \cdot a$ for all $a \in R$.
        \item For all $a,b,c \in R$, we have
        $$a \cdot (b+c) = a \cdot b + a \cdot c \quad \text{and} \quad (a+b) \cdot c = a \cdot c + b \cdot c.$$
    \end{enumerate} 

    The operation $+$ is often called \hldef{addition} and the operation $\cdot$ is often called \hldef{multiplication}. Accordingly, the identity element $0$ of $+$ is often called the \hldef{additive identity} and the identity element $1$ of $\cdot$ is often called the \hldef{multiplicative identity}.

    % If $\cdot$ is additionally a \CrefAndHyperrefIfExist{definition:commutative_binary_operation}{commutative operation}, i.e. $a \cdot b = b \cdot a$ for all $a,b \in R$, then we call the ring \hldef{commutative}.  


\end{definition}
\begin{remark}
    Some writers might not require a ring to have a multiplicative identity element, i.e. would define a ring so that $(R,+)$ is a commutative group, $(R, \cdot)$ is a semigroup, and $\cdot$ is distributive over $+$. Such writers would call the notion of ring in \Cref{definition:ring} a \hldef{unitary ring} to emphasize the existence of the multiplicative identity $1$. 
\end{remark}


\begin{definition} \label{definition:commutative_ring}
   A \hldef{commutative (unital) ring} is a \CrefAndHyperrefIfExist{definition:ring}{ring} $(R, +, \cdot)$ such that $\cdot$ is a \CrefAndHyperrefIfExist{definition:commutative_binary_operation}{commutative operation}, i.e. $a \cdot b = b \cdot a$. 

   For many writers (e.g. ``commutative'' algebraists or number theorists), a \hldef{ring} refers to a commutative ring as above.
\end{definition}


\begin{definition}[Field] \label{definition:field}
A \hldef{field} is commutative \CrefAndHyperrefIfExist{definition:division_ring}{division} \CrefAndHyperrefIfExist{definition:commutative_ring}{ring}. In other words, a field is a commutative ring for which all nonzero elements have a \CrefAndHyperrefIfExist{definition:unit_of_a_ring}{multiplicative inverse}.
\end{definition}



\begin{definition} \label{definition:module_of_a_ring}
Let $R$ be a \CrefAndHyperrefIfExist{definition:ring}{not-necessarily commutative ring}. 
\begin{enumerate}
    \item A \hldef{left $R$-module} is an abelian group $(M,+)$ together with an operation $R \times M \to M$, denoted $(r,m) \mapsto rm$, such that for all $r,s \in R$ and $m,n \in M$:
    \begin{itemize}
        \item $r(m+n) = rm + rn$,
        \item $(r+s)m = rm + sm$,
        \item $(rs)m = r(sm)$,
        \item $1_R m = m$ where $1_R$ is the multiplicative identity of $R$.
    \end{itemize}

    \item A \hldef{right $R$-module} is defined similarly as an abelian group $(M,+)$ with an operation $M \times R \to M$, denoted $(m,r) \mapsto mr$, such that for all $r,s \in R$ and $m,n \in M$:
    \begin{itemize}
        \item $(m+n)r = mr + nr$,
        \item $m(r+s) = mr + ms$,
        \item $m(rs) = (mr)s$,
        \item $m 1_R = m$.
    \end{itemize}

    \item Let $R$ and $S$ be  (not necessarily commutative) \CrefAndHyperrefIfExist{definition:ring}{rings}.

    An \hldef{$R$-$S$-bimodule} (or an \hldef{$R$-$S$-module} or an $(R,S)$-module, etc.)is an \CrefAndHyperrefIfExist{definition:group}{abelian group} $(M,+)$ equipped with
    \begin{enumerate}
        \item a left action of $R$:
        $$\hlin{R \times M \to M, \quad (r,m) \mapsto r \cdot m},$$
        making $M$ a \CrefAndHyperrefIfExist{definition:module_of_a_ring}{left $R$-module},
        \item a right action of $S$:
        $$\hlin{M \times S \to M, \quad (m,s) \mapsto m \cdot s},$$
        making $M$ a right $S$-module,
    \end{enumerate}
    such that the left and right actions commute; that is, for all $r \in R$, $s \in S$, and $m \in M$,
    $$ r \cdot (m \cdot s) = (r \cdot m) \cdot s.  $$

    \item A \hldef{two-sided $R$-module} (or \hldef{$R$-bimodule}) is an $R$-$R$-bimodule.
    
    % an abelian group $(M,+)$ which is simultaneously a left $R$-module and a right $R$-module, such that $(rm)s = r(ms)$ for all $r,s \in R$, $m \in M$. Equivalently, a two-sided $R$-module is an \hldef{$R$-$R$-bimodule}\CrefIfExists{definition:module_of_a_ring}


\end{enumerate}
If $R$ is a \CrefAndHyperrefIfExist{definition:commutative_ring}{commutative ring}, then a left/right $R$-module can automatically be regarded as a two-sided $R$-module. As such, we simply talk about \hldef{$R$-modules} in this case. 

Any abelian group is equivalent to a two-sided $\bbZ$-module. Moreover, any left $R$-module is equivalent to an \CrefAndHyperrefIfExist{definition:module_of_a_ring}{$R-\bbZ$-bimodule} and any right $R$-module is equivalent to an \CrefAndHyperrefIfExist{definition:module_of_a_ring}{$\bbZ-R$-bimodule}. Given a left/right/two-sided $R$-module, its \hldef{natural bimodule structure} will refer to its structure as a $R$-$\bbZ$/$\bbZ$-$R$/$R$-$R$ bimodule. In this way, many definitions associated with the notions of left/right/two-sided $R$-modules can be defined as special cases for definitions for $R$-$S$-bimodules.
\end{definition}

\begin{definition} \label{definition:multilinear_map_of_modules_over_rings}
    \begin{enumerate}
        \item 
        Let $R_0,\ldots,R_k$ be \CrefAndHyperrefIfExist{definition:ring}{(not necessarily commutative) rings}. Let $M_i$ be a \CrefAndHyperrefIfExist{definition:module_of_a_ring}{$R_{i-1}-R_i$-bimodule} for $i = 1,\ldots,k$, and let $N$ be an $R_0-R_k$-bimodule. A function $\Phi: M_1 \times \cdots \times M_k \to N$ is called a \hldef{multilinear map} (or \emph{$R_0-R_k$-multilinear}) if \label{item:multilinear_map_of_modules_over_rings_distinct}
        \begin{itemize}
            \item for each $j=1,\ldots,k$ and fixed $m_i \in M_i$ for $i \neq j$, the map $M_j \to N$ given by $m_j \mapsto \Phi(m_1,\ldots,m_j,\ldots,m_k)$ is a \CrefAndHyperrefIfExist{definition:group_homomorphism}{group homomorphism} and
            \item for all $m_i \in M_i$ for $i = 1,\ldots, k$ and $r_j \in R_j$ where $j \in \{1,\ldots,k-1\}$, we have 
            $$\Phi(m_1,\ldots,m_j r_j, m_{j+1}, \ldots,m_k) = \Phi(m_1,\ldots,m_j r_j, r_j m_{j+1},\ldots,m_k).$$
            \item $\Phi$ is \CrefAndHyperrefIfExist{definition:homomorphism_of_modules_over_a_ring}{left $R_0$-linear} in the first argument and right $R_k$-linear in the $k$th argument, i.e. for all $r_0 \in R_0$ and $r_k \in r_k$ we have
            $$\Phi(r_0m_1,m_2,\ldots,m_{k-1}, m_kr_k) = r_0 \cdot \Phi(m_1,\ldots,m_k) \cdot r_k.$$
        \end{itemize}

        \item Let $R$ be a (not necessarily commutative) ring and let $M$ be a \CrefAndHyperrefIfExist{definition:homomorphism_of_modules_over_a_ring}{two-sided $R$-module}. A \hldef{multilinear form} is a multilinear map $M^r \to R$ (where $M^r$ here is the \CrefAndHyperrefIfExist{definition:product_of_sets}{set theoretic Cartesian product}, rather than a product of groups or modules) for some $r \geq 0$. 

    \end{enumerate}

    In particular, when $R$ be a \CrefAndHyperrefIfExist{definition:commutative_ring}{commutative ring}, and $M_i$ for $i = 1,\ldots,k$ and $N$ are $R$-modules, we may speak of a multilinear map $\Phi: M_1 \times \cdots \times M_k \to N$. We may thus also speak of multilinear maps $M^r \to R$ for $r \geq 0$

    % \item Let $R$ be a \CrefAndHyperrefIfExist{definition:commutative_ring}{commutative ring}. Let $M_i$ be $R$-modules for $i = 1,\ldots, k$, and let $N$ be an $R$-module. A function $\Phi: M_1 \times \cdots \times M_k \to N$ usually said to be \hldef{multilinear map} if it is multilinear in the sense of \ref{item:multilinear_map_of_modules_over_rings_distinct} and is \CrefAndHyperrefIfExist{definition:homomorphism_of_modules_over_a_ring}{$R$-linear} in each variable, i.e. satisfies \label{item:multilinear_map_of_modules_over_rings_commutative}
    % $$\Phi(m_1,\ldots,m_{j-1}, r_jm_j,m_{j+1},\ldots,m_k) = r_j\Phi(m_1,\ldots,m_{j-1},m_j,m_{j+1},\ldots,m_k)$$
    % for all $m_i \in M_i$ for $i =1,\ldots,k$, all $j = 1,\ldots,k$, and all $r_j \in R$. In particular, a \emph{multilinear map} where the modules involved are all modules of a commutative ring will most usually refer to a multilinear map in this sense rather than in the sense of \ref{item:multilinear_map_of_modules_over_rings_distinct}.

    % \TODO{$M^k$}
    % \item Let $R$ be a commutative ring and let $M$ be an $R$-module. A \hldef{multilinear form} is a multilinear map $M^k \to R$ in the sense of \ref{item:multilinear_map_of_modules_over_rings_commutative}.

    Additionally, we may speak of \hldef{bilinear maps/forms}, \hldef{trilinear maps/forms}, etc.
\end{definition}
\begin{definition} \label{definition:symmetric_antisymmetric_alternating_multilienar_forms_over_rings}
    Let $R$ be a \CrefAndHyperrefIfExist{definition:ring}{(not necessarily commutative) ring}, $M$ an \CrefAndHyperrefIfExist{definition:homomorphism_of_modules_over_a_ring}{two-sided $R$-module}, $k \in \bbN$ and $\Phi: M^k \to R$ a \CrefAndHyperrefIfExist{definition:multilinear_map_of_modules_over_rings}{multilinear form}
\begin{enumerate}
    \item $\Phi$ is \hldef{symmetric} if for every \CrefAndHyperrefIfExist{definition:symmetric_group_on_a_set}{permutation} $\sigma \in S_k$ and all $x_1,\dots,x_k \in M$,
    $$
    \Phi(x_1, ..., x_k) = \Phi(x_{\sigma(1)}, ..., x_{\sigma(k)}).
    $$

    \item $\Phi$ is \hldef{antisymmetric} if for all $\sigma \in S_k$ and all $x_1,\dots,x_k \in M$,
    $$ \Phi(x_{\sigma(1)}, ..., x_{\sigma(k)}) = \operatorname{sgn}(\sigma) \Phi(x_1, ..., x_k), $$
    where $\operatorname{sgn}(\sigma)$ is the \CrefAndHyperrefIfExist{definition:signature_of_an_element_of_the_symmetric_group_on_a_finite_set}{sign} of the permutation $\sigma$.
    
    \item $\Phi$ is \hldef{alternating} if whenever $x_i = x_j$ for some $i \neq j$, $\Phi(x_1, ..., x_k) = 0$. Every alternating form is antisymmetric, since if $x_i = x_j$ and we swap coordinates, both terms are zero.
\end{enumerate}

For $k=2$, these definitions specialize to \hldef{bilinear forms}; in particular:
\begin{itemize}
    \item $\Phi$ is symmetric if $\Phi(x, y) = \Phi(y, x)$.
    \item $\Phi$ is antisymmetric if $\Phi(x, y) = -\Phi(y, x)$.
    \item $\Phi$ is alternating if $\Phi(x, x) = 0$ for all $x$.
\end{itemize}
\end{definition}
\begin{definition}[Hom of left/right/bi-modules] \label{definition:hom_of_left_right_bi_modules_of_rings}
Let $R,S,T$ be \CrefAndHyperrefIfExist{definition:ring}{(not necessarily commutative) rings}.
\begin{enumerate}
    \item Let $M$ and $N$ be \CrefAndHyperrefIfExist{definition:module_of_a_ring}{left $R$-modules}. The \hldef{homomorphism group of left $R$-modules from $M$ to $N$} is the abelian group
    $$\hlin{\Hom(M,N) = \mathrm{Hom}_R(M, N) := \{ f : M \to N \mid f \text{ is a left $R$-module homomorphism} \}.} $$
    \CrefIfExists{definition:homomorphism_of_modules_over_a_ring}

    \item Let $M$ and $N$ be \CrefAndHyperrefIfExist{definition:module_of_a_ring}{right $R$-modules}. The \hldef{homomorphism group of right $R$-modules from $M$ to $N$} is the abelian group
    $$\hlin{ \Hom(M,N) = \mathrm{Hom}_R(M, N) := \{ f : M \to N \mid f \text{ is a right $R$-module homomorphism} \}.}$$

    \item Let $S$ be a (not necessarily commutative ring) and let $M$ and $N$ be \CrefAndHyperrefIfExist{definition:module_of_a_ring}{$R-S$-bimodules}. The \hldef{homomorphism group of $R$-$S$-bimodules from $M$ to $N$} is the abelian group 
    $$\hlin{\Hom(M,N) = \Hom_{R-S}(M,N) \coloneq \{f: M \to N | f \text{ is a } R-S\text{-bimodule homomorphism} \}}$$
    \CrefIfExists{definition:hom_between_bimodules}
    \end{enumerate}

    In each case, $\Hom(M,N)$ has a natural structure of an \hldef{abelian group} given by \hldef{pointwise addition}: for $f, g \in \mathrm{Hom}(M, N)$,
    $$ (f + g)(m) := f(m) + g(m), $$
    and the zero morphism \hl{$0$} given by $0(m) := 0_N$ acts as the identity element.
    The additive inverse \hl{$-f$} is defined by $(-f)(m) := -f(m)$. Moreover, depending on bi-module structures that $M$ and $N$ may be carrying, $\Hom(M,N)$ may itself carry additional module structures:
    \begin{itemize}
        \item In case that $M$ is a \CrefAndHyperrefIfExist{definition:hom_between_bimodules}{$R-S$-bimodule} and $N$ is a $R-T$-bimodule, $\mathrm{Hom}_R(M, N)$, the group of left $R$-module homomorphisms, is an $S-T$-bimodule as follows:
        $$(s\cdot f\cdot t)(m) = f(m \cdot s) \cdot t \quad f \in \Hom_R(M,N), s \in S, t \in T.$$

        \item Dually, in case that $M$ is a \CrefAndHyperrefIfExist{definition:hom_between_bimodules}{$S-R$-bimodule} and $N$ is a $T-R$-bimodule, $\mathrm{Hom}_R(M, N)$, the group of right $R$-module homomorphisms, is an $S-T$-bimodule as follows:
        $$(s\cdot f\cdot t)(m) = f(s \cdot m) \cdot t \quad f \in \Hom_R(M,N), s \in S, t \in T.$$
    \end{itemize}
    Some cases of interest may be when $R$, $S$, or $T$ is in fact $\bbZ$ --- these allow us to see module structures on $\Hom(M,N)$ even when $M$ and $N$ are one-sided modules.

    \TODO{state this as a theorem}
    We furthermore note that $\Hom_R(-,-)$ yields \CrefAndHyperrefIfExist{definition:n_ary_additive_functor_between_additive_categories}{biadditive functors}
    $$\Hom_R(-,-): {}_R \mathbf{Mod}_{S}^{\op} \times {}_{R} \mathbf{Mod}_{T} \to {}_{S} \mathbf{Mod}_{T}$$
    $$\Hom_R(-,-): {}_S \mathbf{Mod}_{R}^{\op} \times {}_{T} \mathbf{Mod}_{R} \to {}_{S} \mathbf{Mod}_{T}.$$
    \CrefIfExists{definition:opposite_category_of_a_category}\CrefIfExists{definition:category_of_modules_and_bimodules_over_rings} \CrefIfExists{theorem:the_category_of_R_S_bimodules_is_a_grothendieck_abelian_category_and_AB4_star}
\end{definition}
\begin{definition} \label{definition:dual_of_a_left_right_two_sided_module}
Let $R$ be a \CrefAndHyperrefIfExist{definition:ring}{(not necessarily commutative) ring}. Depending on the module structure of $M$, we define its dual module as follows:
\begin{enumerate}
    \item If $M$ is a \CrefAndHyperrefIfExist{definition:module_of_a_ring}{left $R$-module}, then the \hldef{(right) dual module of $M$} is 
    $$\hlin{M^* = M^\vee \coloneq \operatorname{Hom}_R(M, R)}.$$
    \CrefIfExists{definition:hom_of_left_right_bi_modules_of_rings}
    Note that it is a right $R$-module, as $M$ is a $R-\bbZ$-bimodule and $R$ is an $R-R$-bimodule. 

    \item If $M$ is a \CrefAndHyperrefIfExist{definition:module_of_a_ring}{right $R$-module}, then the \hldef{(left) dual module of $M$} is 
    $$\hlin{{}^* M = {}^\vee M \coloneq \operatorname{Hom}_R(M, R)}.$$
    \CrefIfExists{definition:hom_of_left_right_bi_modules_of_rings}
    Note that it is a left $R$-module, as $M$ is a $\bbZ-R$-bimodule and $R$ is an $R-R$-bimodule. 

    % \item If $M$ is a \CrefAndHyperrefIfExist{definition:module_of_a_ring}{right $R$-module}, the set of all left $R$-linear maps from $M$ to $R$ is denoted by
    % $$\hlin{\,{}^*M = \operatorname{Hom}_R(M, R)},$$
    % and is called the \hldef{left dual module} of $M$.

    \item If $M$ is a \CrefAndHyperrefIfExist{definition:two_sided_module_of_a_ring}{two-sided $R$-module}, then the \hldef{dual of $M$} usually refers to either the right or the left dual as above.
    % then $M$ admits both left and right $R$-actions, and we may define two duals:
    % \begin{align*}
    %     M^* &= \operatorname{Hom}_{R\text{-left}}(M, R) \quad \text{(right dual module)},\\
    %     {}^*M &= \operatorname{Hom}_{\text{right-}R}(M, R) \quad \text{(left dual module)}.
    % \end{align*}
    % In particular, $M^*$ becomes a right $R$-module, and ${}^*M$ becomes a left $R$-module, via pointwise scalar multiplication:
    % $$ (f \cdot r)(m) = f(m)r, \quad (r \cdot f)(m) = rf(m). $$
\end{enumerate}
In any case, the functor $M \mapsto M^\vee$ is a \CrefAndHyperrefIfExist{definition:functor_between_categories}{contravariant functor} from the appropriate \CrefAndHyperrefIfExist{definition:category_of_modules_and_bimodules_over_rings}{category of modules} to itself.

If $R$ is a \CrefAndHyperrefIfExist{definition:field}{field} $F$ and $V$ is an \CrefAndHyperrefIfExist{definition:vector_space_over_a_field}{$F$-vector space}, then the dual module
$$\hlin{V^* = V^\vee \coloneq  \operatorname{Hom}_F(V, F)}$$
is called the \hldef{dual vector space of $V$}.
\end{definition}


\section{Topology}


\begin{definition}[Topology] \label{definition:topological_space}
Let $X$ be a set. A \hldef{topology on $X$} is a collection $\mathcal{T}$ of subsets of $X$ such that:
\begin{enumerate}
    \item $\emptyset \in \mathcal{T}$ and $X \in \mathcal{T}$,
    \item For any collection $\{ U_i \}_{i \in I} \subseteq \mathcal{T}$ (with $I$ arbitrary), the union $\bigcup_{i \in I} U_i \in \mathcal{T}$,
    \item For any finite collection $\{ U_1, \ldots, U_n \} \subseteq \mathcal{T}$, the intersection $U_1 \cap \cdots \cap U_n \in \mathcal{T}$.
\end{enumerate}
If $\mathcal{T}$ is a topology on $X$, the pair $(X, \mathcal{T})$ is called a \hldef{topological space}. Members of $\mathcal{T}$ are called \hldef{open sets}. 

A subset $C \subseteq X$ is \hldef{closed} if its complement $X \setminus C$ is an open set in $\mathcal{T}$

One very often refers to $X$ as a topological spcae, omitting the notation of the topology $\mathcal{T}$. 

The collection of all topologies on a set $X$ may be denoted by notations such as \hl{$\mathrm{Top}(X)$}, \hl{$\mathbf{Top}(X)$}, or \hl{$\mathsf{Top}(X)$}.
\end{definition}





\begin{definition} \label{definition:euclidean_n_space}
For a positive integer $n$, let \hl{$\mathbb{R}^n$} denote the $n$-fold Cartesian product of the real line $\mathbb{R}$ with itself:
$$\hlin{\mathbb{R}^n = \{ (x_1, x_2, \dots, x_n) : x_i \in \mathbb{R} \text{ for all } i=1,\dots,n \}}.$$ 
The set $\mathbb{R}^n$ is called \hl{Euclidean n-space}. A point $x = (x_1, \dots, x_n) \in \mathbb{R}^n$ is associated with the \hl{Euclidean norm}
$$\hlin{\|x\| = \sqrt{x_1^2 + x_2^2 + \dots + x_n^2}}.$$ 
The corresponding metric $d : \mathbb{R}^n \times \mathbb{R}^n \to [0,\infty)$ is given by
$$\hlin{d(x,y) = \|x - y\|}.$$ 
This metric induces the standard topology on $\mathbb{R}^n$, called the \hldef{Euclidean topology}.
\end{definition}
\begin{definition} \label{definition:closed_half_space_in_euclidean_space}
    The \hldef{closed half-space} (in \CrefAndHyperrefIfExist{definition:euclidean_n_space}{$\bbR^n$}) refers to the \CrefAndHyperrefIfExist{definition:topological_space}{topological space} \hl{$\mathbb{H}^n \subset \bbR^n$} defined by
    $$\mathbb{H}^n = \{x \in \mathbb{R}^n : x_n \ge 0\}$$
    Other common notation for $\bbH^n$ include \hl{$\bbR_+^n$} and \hl{$\bbR_{\geq 0}^n$}.
    \TextIfExists{definition:closed_upper_half_space_in_C_n}{
    The closed upper half space $\bbH^n$ in $\bbR^n$ should not be mistaken for the closely related notion of \CrefAndHyperrefIfExist{definition:closed_upper_half_space_in_C_n}{closed half space $\bbH^n$} in $\bbC^n$.
    }
\end{definition}
\begin{definition} \label{definition:neighborhood_and_neighborhood_basis_of_a_point_in_a_topological_space}
Let $(X,\mathcal{T})$ be a \CrefAndHyperrefIfExist{definition:topological_space}{topological space} and let $x \in X$.
\begin{itemize}
    \item An \hldef{open neighborhood of $x$} is any open set $U \in \mathcal{T}$ such that $x \in U$.
    \item A \hldef{neighborhood of $x$} is a set $N \subseteq X$ for which there exists an \CrefAndHyperrefIfExist{definition:open_neighborhood_of_a_point_of_a_topological_space}{open neighborhood $U \in \mathcal{T}$} of $x$ such that $U \subseteq N$.
    \item A \hldef{neighborhood basis} (or \hldef{local base}) at $x$ is a nonempty collection $\mathcal{B}_x$ of neighborhoods of $x$ such that for every neighborhood $N$ of $x$, there exists $B \in \mathcal{B}_x$ with $B \subseteq N$.

    The elements of $\mathcal{B}_x$ are said to \hldef{form a base of neighborhoods} at $x$.
\end{itemize}
\end{definition}
\begin{definition} \label{definition:continuous_map_of_topological_spaces}
Let $(X,\mathcal{T}_X)$ and $(Y,\mathcal{T}_Y)$ be \CrefAndHyperrefIfExist{definition:topological_space}{topological spaces}. A map $f : X \to Y$ is called \hldef{continuous} if for every open set $V \in \mathcal{T}_Y$, the preimage $f^{-1}(V)$ is an open set in $X$, that is,
$$\hlin{\forall V \in \mathcal{T}_Y, \; f^{-1}(V) \in \mathcal{T}_X.}$$
Equivalently, $f$ is continuous if and only if for every closed set $C \subseteq Y$, the preimage $f^{-1}(C)$ is closed in $X$. 


A \hldef{map of topological spaces} usually refers to a continuous map between the topological spaces.

The set of continuous maps from $X$ to $Y$ is sometimes denoted by \hl{$C(X,Y)$}. Other standard notation include \hl{$\operatorname{Hom}_{\mathrm{Top}}(X,Y)$} or \hl{$\operatorname{Top}(X,Y)$} coming from more general notation for morphisms between objects in a \CrefAndHyperrefIfExist{definition:category}{category}.

% The collection of topological spaces along with continuous maps form a \CrefAndHyperrefIfExist{definition:locally_small_category}{locally small} \CrefAndHyperrefIfExist{definition:category}{category}, usually called the \hldef{category of topological spaces} and often denoted by notations such as $\mathrm{Top}$, $\mathbf{Top}$, etc. 

% The set of continuous maps from $X$ to $Y$ is sometimes denoted by \hl{$C(X,Y)$}. Other standard notation include \hl{$\operatorname{Hom}_{\mathrm{Top}}(X,Y)$} or \hl{$\operatorname{Top}(X,Y)$} coming from more general notation for morphisms between objects in a category.
\end{definition}
\begin{definition} \label{definition:homeomorphism_of_topological_spaces}
Let $(X,\mathcal{T}_X)$ and $(Y,\mathcal{T}_Y)$ be \CrefAndHyperrefIfExist{definition:topological_space}{topological spaces}. A function $f : X \to Y$ is called a \hldef{homeomorphism} if it satisfies all of the following:
\begin{enumerate}
  \item $f$ is \CrefAndHyperrefIfExist{definition:injective_surjective_bijective_map_of_sets}{bijective};
  \item $f$ is \CrefAndHyperrefIfExist{definition:continuous_map_of_topological_spaces}{continuous} with respect to $\mathcal{T}_X$ and $\mathcal{T}_Y$; and
  \item the \CrefAndHyperrefIfExist{definition:injective_surjective_bijective_map_of_sets}{inverse map $f^{-1} : Y \to X$} is also continuous.
\end{enumerate}
If such a function exists, the spaces $X$ and $Y$ are said to be \hldef{homeomorphic}.
\end{definition}
\begin{definition}[Separation axioms] \label{definition:separation_axioms_of_topology}
Let $(X,\mathcal{T})$ be a \CrefAndHyperrefIfExist{definition:topological_space}{topological space}.
\begin{itemize}
  \item $(X,\mathcal{T})$ is \hldef{T$_0$} (\hldef{Kolmogorov}) if for every pair of distinct points $x, y \in X$, there exists an open set $U \in \mathcal{T}$ such that, without loss of generality, $x \in U$ and $y \notin U$.
  \item $(X,\mathcal{T})$ is \hldef{T$_1$} (\hldef{Fréchet}) if for every pair of distinct points $x, y \in X$, there exist open sets $U, V \in \mathcal{T}$ such that $x \in U$, $y \notin U$, and $y \in V$, $x \notin V$.
  \item $(X,\mathcal{T})$ is \hldef{T$_2$} or \hldef{Hausdorff} if for every pair of distinct points $x, y \in X$, there exist disjoint open sets $U, V \in \mathcal{T}$ such that $x \in U$ and $y \in V$.
  \item $(X,\mathcal{T})$ is \hldef{regular} if it is T$_1$ and for each point $x \in X$ and closed set $F \subseteq X$ with $x \notin F$, there exist disjoint open sets $U, V \in \mathcal{T}$ such that $x \in U$ and $F \subseteq V$.
  \item $(X,\mathcal{T})$ is \hldef{T$_3$} (regular Hausdorff) if it is T$_1$ and regular.
  \item $(X,\mathcal{T})$ is \hldef{completely regular} if for each closed set $F \subseteq X$ and $x \notin F$, there exists a continuous function $f : X \to [0,1]$ such that $f(x) = 0$ and $f|_F = 1$.
  \item $(X,\mathcal{T})$ is \hldef{T$_{3\frac{1}{2}}$} (completely regular Hausdorff) if it is T$_1$ and completely regular.
  \item $(X,\mathcal{T})$ is \hldef{normal} if it is T$_1$ and for each pair of disjoint closed sets $A, B \subseteq X$, there exist disjoint open sets $U, V \in \mathcal{T}$ such that $A \subseteq U$ and $B \subseteq V$.
  \item $(X,\mathcal{T})$ is \hldef{T$_4$} (normal Hausdorff) if it is T$_1$ and normal.
   \item $(X,\mathcal{T})$ is \hldef{T$_5$} (completely normal Hausdorff) if it is T$1$ and completely normal.
    \item $(X,\mathcal{T})$ is \hldef{perfectly normal} if every closed set is a \CrefAndHyperrefIfExist{definition:G_delta_set_and_F_sigma_set_of_a_topological_space}{$G\delta$} (\CrefAndHyperrefIfExist{definition:countable_finite_uncountable_sets}{countable} intersection of open sets) and the space is normal.
    \item $(X,\mathcal{T})$ is \hldef{T$_6$} (perfectly normal Hausdorff) if it is T$_1$ and perfectly normal.
\end{itemize}
\end{definition}
\begin{definition} \label{definition:first_and_second_countable_topological_space}
Let $(X,\mathcal{T})$ be a \CrefAndHyperrefIfExist{definition:topological_space}{topological space}.
\begin{itemize}
  \item The space $(X,\mathcal{T})$ is said to be \hldef{first countable} if every point $x \in X$ has a \CrefAndHyperrefIfExist{definition:countable_finite_uncountable_sets}{countable} \CrefAndHyperrefIfExist{definition:neighborhood_and_neighborhood_basis_of_a_point_in_a_topological_space}{neighborhood basis}, i.e., there exists a countable collection $\{U_n\}_{n \in \mathbb{N}}$ of open neighborhoods of $x$ such that for any open neighborhood $U$ of $x$, there exists $n \in \mathbb{N}$ with $U_n \subseteq U$.
  \item The space $(X,\mathcal{T})$ is said to be \hldef{second countable} if there exists a countable \CrefAndHyperrefIfExist{definition:basis_for_a_topology}{basis} $\mathcal{B} \subseteq \mathcal{T}$ for the topology $\mathcal{T}$, i.e., every open set $U \in \mathcal{T}$ can be written as a union of elements of $\mathcal{B}$.
\end{itemize}
\end{definition}
\begin{definition} \label{definition:interior_and_boundary_of_topological_space}
Let $(X,\mathcal{T})$ be a topological space and let $A \subseteq X$.
\begin{itemize}
  \item The \hldef{interior of $A$}, denoted by \hl{$\operatorname{int}(A)$}, is the union of all open sets contained in $A$:
  $$\hlin{\operatorname{int}(A) = \bigcup \{ U \in \mathcal{T} : U \subseteq A \}}.$$
  \item The \hldef{boundary of $A$}, denoted by \hl{$\partial A$}, is defined as
  $$\hlin{\partial A = \overline{A} \setminus \operatorname{int}(A)}.$$ 
  \CrefIfExists{definition:closure_of_a_subspace_of_a_topological_space}
  Equivalently, a point $x \in X$ belongs to $\partial A$ if every open neighborhood of $x$ intersects both $A$ and its complement $X \setminus A$.
\end{itemize}
\end{definition}

\begin{definition} \label{definition:open_covering_of_a_topological_space}
Let $(X, \tau)$ be a \CrefAndHyperrefIfExist{definition:topological_space}{topological space}. An \hldef{open covering of $X$} is a family of open sets 
\[ \mathcal{U} = \{ U_i \}_{i \in I} \] 
such that 
\[
\bigcup_{i \in I} U_i = X.
\]
Here, each $U_i \in \tau$ is an open subset of $X$ indexed by a set $I$, which can be finite or infinite.
\end{definition}
\begin{definition}[Closure of a subset] \label{definition:closure_of_a_subspace_of_a_topological_space}
Let $(X, \mathcal{T})$ be a topological space and $A \subseteq X$ a subset. The \hldef{closure of $A$ in $X$}, denoted by \hl{$\overline{A}$}, is defined as the intersection of all closed sets containing $A$, i.e.,
\[
\overline{A} \;:=\; \bigcap \{ C \subseteq X : C \text{ is closed and } A \subseteq C \}.
\]
Equivalently, $\overline{A}$ is the smallest closed set containing $A$.
\end{definition}


\begin{definition}[Compact topological space] \label{definition:compact_topological_space}
A topological space $(X, \mathcal{T})$ is \hldef{compact} if every open cover of $X$ admits a finite subcover; that is, for every collection $\{ U_i \}_{i \in I}$ of open sets in $\mathcal{T}$ such that $X = \bigcup_{i \in I} U_i$, there exists a finite subcollection $\{ U_{i_j} \}_{j=1}^n$ such that $X = \bigcup_{j=1}^n U_{i_j}$.

Some mathematicians, e.g. algebraic geometers, would refer to this property as \hldef{quasi-compactness}.
\end{definition}

\begin{definition}[Fiber of a map of topological spaces] \label{definition:fiber_of_a_map_of_topological_spaces_over_a_point}
Let $X$ and $Y$ be \CrefAndHyperrefIfExist{definition:topological_space}{topological spaces} and let $f : X \to Y$ be a \CrefAndHyperrefIfExist{definition:continuous_map_of_topological_spaces}{continuous map}.
For a point $y \in Y$, the \hldef{fiber of $f$ over $y$} is the \CrefAndHyperrefIfExist{definition:preimage_of_a_subset_of_the_codomain_of_a_set_map}{inverse image} $f^{-1}(y) = f^{-1}(\{y\})$ endowed with the \CrefAndHyperrefIfExist{definition:subspace_of_a_topological_space}{subspace topology} induced from $X$. The fiber is also denoted by notations such as \hl{$\mathrm{Fib}_f(y)$} or \hl{$X_y$}.
\end{definition}
\begin{definition}[Submodule generated by elements in an $(R,S)$-bimodule] \label{definition:submodule_of_a_module_generated_by_elements}
    Let $R$ and $S$ be \CrefAndHyperrefIfExist{definition:ring}{(not necessarily commutative) rings}. 
    \begin{enumerate}
        \item 
        Let $M$ be an \CrefAndHyperrefIfExist{definition:module_of_a_ring}{$(R,S)$-bimodule}.

        Given a subset $X \subseteq M$, the \hldef{sub-bimodule of $M$ generated by $X$} is the smallest $(R,S)$-sub-bimodule of $M$ containing $X$. It is often denoted by notations such as \hl{$\langle X \rangle = \langle X \rangle_{R,S}$} and is more explicitly the intersection
        $$\langle X \rangle_{R,S} = \bigcap_{X \subseteq T \subseteq M, T \text{ is a } (R,S)\text{-submodule of } M} T$$
        of al $(R,S)$-submodules of $M$ containing $X$. 
        
        Equivalently, $\langle X \rangle_{R,S}$ consists of all \CrefAndHyperrefIfExist{definition:linear_combination_of_elements_in_a_module}{linear combinations} of $X$. 

        \item If $M$ is a left/right/two-sided $R$-module and given a subset $X \subseteq M$, the \hldef{submodule of $M$ generated by $X$} is the submodule of the \CrefAndHyperrefIfExist{definition:module_of_a_ring}{natural bimodule} of $M$ generated by $X$. It is denoted by notations such as $\langle X \rangle = \langle X \rangle_R$. 
    \end{enumerate}
    % Explicitly, this sub-bimodule consists of all finite sums of elements of the form
    % \[
    % \hl{$r \cdot x \cdot s$}
    % \]
    % where $r \in R$, $s \in S$, and $x \in X$. That is,
    % \[
    % \hl{$\langle X \rangle_{R,S} = \left\{ \sum_{i=1}^n r_i x_i s_i \mid n \in \mathbb{N}, r_i \in R, s_i \in S, x_i \in X \right\}$}.
    % \]
    % This sub-bimodule is the smallest $(R,S)$-bimodule containing $X$, closed under the actions of $R$ and $S$ and addition.
\end{definition}
\begin{definition}[Tensor product of bimodules] \label{definition:tensor_product_of_bimodules_of_rings}
Let $R,S,T$ be \CrefAndHyperrefIfExist{definition:ring}{(not necessarily commutative) rings}, let $M$ be an \CrefAndHyperrefIfExist{definition:module_of_a_ring}{$R$-$S$ bimodule}, and let $N$ be an $S$-$T$ bimodule. In the \CrefAndHyperrefIfExist{definition:free_abelian_group_generated_by_a_set}{free abelian group} $\bbZ[M \times N]$ generated by the \CrefAndHyperrefIfExist{definition:product_of_sets}{Cartesian product $M \times N$}, let $U$ be the subgroup generated by elements of the form
\TODO{subgroup generated}
\begin{align*}
&(m+m',n) - (m,n) - (m',n),\\
&(m,n+n') - (m,n) - (m,n'),\\
&(m \cdot s, n) - (m, s \cdot n),
\end{align*}
for all $m,m' \in M$, $n,n' \in N$, and $s \in S$. The \hldef{tensor product of $M$ and $N$ over $S$} is the \CrefAndHyperrefIfExist{definition:quotient_of_a_group_by_a_normal_subgroup}{quotient} abelian group
$$M \otimes_S N := \mathbb{Z}[M \times N] / U.$$
The image of an element of the form $(m,n) \in M \times N$ in $M \otimes_S N$ is denoted \hl{$m \otimes n$} and called a \hldef{pure tensor}. In general, the elements of $M \otimes_S N$ are finite sums 
$$\sum_{i=1}^n m_i \otimes n_i \quad m_i \in M, n_i \in N$$
of pure tensors. Thus, the pure tensors satisfy the following relations:
\begin{align*}
    (m + m') \otimes n &= m \otimes n + m' \otimes n \\ 
    m \otimes (n + n') &= m \otimes n + m \otimes n' \\
    (m \cdot s) \otimes n &= m \otimes (s \cdot n)
\end{align*}

This tensor product becomes naturally an $R$-$T$ bimodule with left action and right action defined by
\begin{align*}
r \cdot (m \otimes n) &= (r \cdot m) \otimes n, \\
(m \otimes n) \cdot t &= m \otimes (n \cdot t),
\end{align*}
for all $r \in R$, $t \in T$, $m \in M$, and $n \in N$.

Inductively, given rings $R_0,\ldots,R_k$ and $R_{i-1}-R_i$-bimodules $M_i$ for $i = 1,\ldots,k$, we may speak of the tensor product
$$M_0 \otimes_{R_1} M_1 \otimes_{R_2} \cdots \otimes_{R_{k-1}} M_k;$$
tensor products are associative\TODO{}, so parentheses are not strictly needed to notate them. Its \hldef{pure tensors} are elements of the form $m_0 \otimes m_1 \otimes \cdots \otimes m_k$ for $m_i \in M_i$, and its general elements are finite sums
$$\sum_{j=1}^n m_{0j} \otimes m_{1j} \otimes \cdots m_{kj} \quad m_{ij} \in M_i.$$
of pure tensors. It also has a natural $R_0-R_k$-bimodule structure.

\TextIfExists{definition:n_ary_additive_functor_between_additive_categories}{In general, $(M_0,\ldots,M_k) \mapsto M_0 \otimes_{R_1} M_1 \otimes_{R_2} \cdots \otimes_{R_{k-1}} M_k$ defines a \CrefAndHyperrefIfExist{definition:n_ary_additive_functor_between_additive_categories}{$(k+1)$-ary additive functor}
$${}_{R_0}\mathbf{Mod}_{R_1} \times \cdots \times {}_{R_{k-1}}\mathbf{Mod}_{R_k} \to {}_{R_0} \mathbf{Mod}_{R_k}$$
(\Cref{theorem:the_category_of_R_S_bimodules_is_a_grothendieck_abelian_category_and_AB4_star}).}


Given a ring $R$ and a two-sided $R$-module $M$, we may also speak of the \hldef{$n$-fold tensor product} \hl{$M^{\otimes n} = M^{\otimes_R n}$}

\end{definition}
\begin{definition} \label{definition:quotient_module_of_a_module_by_a_module_of_a_ring}
\TODO{define coset, kernel of R-module homomorphism}
Let $R,S$ be \CrefAndHyperrefIfExist{definition:ring}{(not necessarily commutative) rings}. 
\begin{enumerate}
    \item Let $M$ be an \CrefAndHyperrefIfExist{definition:module_of_a_ring}{$R$-$S$-bimodule}. Let $N \subseteq M$ be a \CrefAndHyperrefIfExist{definition:submodule_of_a_module_over_a_ring}{submodule of $M$}.  

    The \CrefAndHyperrefIfExist{definition:quotient_of_a_group_by_a_normal_subgroup}{quotient group $M/N$}, which is well defined as $M$ is an \CrefAndHyperrefIfExist{definition:group}{abelian group} and hence $N$ is a \CrefAndHyperrefIfExist{definition:normal_subgroup_of_a_group}{normal subgroup}\CrefIfExists{proposition:subgroup_of_a_group_is_normal_subgroup_of_normalizer}, has the structure of an $R$-$S$-bimodule --- the (abelian) group structure is simply the group structure of $M/N$, whereas the $R$-$S$-bimodule structure is given as follows: for $m \in M$, $r \in R$, $s \in S$, we have
    $$r \cdot (m + N) \cdot s = r \cdot m \cdot s + N.$$

    This $R$-$S$-bidmodule structure on $M/N$ is called the \hldef{quotient $R$-$S$-bidmodule of $M$ by $N$} and is also denoted as \hl{$M/N$}.
    
    The canonical projection map
    $$\pi: M \to M/N, \quad m \mapsto m+N,$$
    is a surjective \CrefAndHyperrefIfExist{definition:homomorphism_of_modules_over_a_ring}{$R$-module homomorphism} with kernel $N$.


    \item Let $M$ be a left/right/two-sided $R$-module. Let $N \subseteq M$ be a submodule of $M$. The \hldef{quotient $R$-module} \hl{$M/N$} is the quotient of $M$ by $N$ for their respective \CrefAndHyperrefIfExist{definition:module_of_a_ring}{natural bimodule structures}.
\end{enumerate}

    \TextIfExists{definition:quotient_object_of_an_object_of_an_abelian_category_by_a_subobject}{The quotient $M/N$ is the \CrefAndHyperrefIfExist{definition:quotient_object_of_an_object_of_an_abelian_category_by_a_subobject}{categorical quotient object} of $M$ by the \CrefAndHyperrefIfExist{definition:subobject_of_an_object_of_an_additive_category}{subobject} $N$ in the appropriate category of modules}

\end{definition}

\begin{definition} \label{definition:multilinear_map_of_modules_over_rings}
    \begin{enumerate}
        \item 
        Let $R_0,\ldots,R_k$ be \CrefAndHyperrefIfExist{definition:ring}{(not necessarily commutative) rings}. Let $M_i$ be a \CrefAndHyperrefIfExist{definition:module_of_a_ring}{$R_{i-1}-R_i$-bimodule} for $i = 1,\ldots,k$, and let $N$ be an $R_0-R_k$-bimodule. A function $\Phi: M_1 \times \cdots \times M_k \to N$ is called a \hldef{multilinear map} (or \emph{$R_0-R_k$-multilinear}) if \label{item:multilinear_map_of_modules_over_rings_distinct}
        \begin{itemize}
            \item for each $j=1,\ldots,k$ and fixed $m_i \in M_i$ for $i \neq j$, the map $M_j \to N$ given by $m_j \mapsto \Phi(m_1,\ldots,m_j,\ldots,m_k)$ is a \CrefAndHyperrefIfExist{definition:group_homomorphism}{group homomorphism} and
            \item for all $m_i \in M_i$ for $i = 1,\ldots, k$ and $r_j \in R_j$ where $j \in \{1,\ldots,k-1\}$, we have 
            $$\Phi(m_1,\ldots,m_j r_j, m_{j+1}, \ldots,m_k) = \Phi(m_1,\ldots,m_j r_j, r_j m_{j+1},\ldots,m_k).$$
            \item $\Phi$ is \CrefAndHyperrefIfExist{definition:homomorphism_of_modules_over_a_ring}{left $R_0$-linear} in the first argument and right $R_k$-linear in the $k$th argument, i.e. for all $r_0 \in R_0$ and $r_k \in r_k$ we have
            $$\Phi(r_0m_1,m_2,\ldots,m_{k-1}, m_kr_k) = r_0 \cdot \Phi(m_1,\ldots,m_k) \cdot r_k.$$
        \end{itemize}

        \item Let $R$ be a (not necessarily commutative) ring and let $M$ be a \CrefAndHyperrefIfExist{definition:homomorphism_of_modules_over_a_ring}{two-sided $R$-module}. A \hldef{multilinear form} is a multilinear map $M^r \to R$ (where $M^r$ here is the \CrefAndHyperrefIfExist{definition:product_of_sets}{set theoretic Cartesian product}, rather than a product of groups or modules) for some $r \geq 0$. 

    \end{enumerate}

    In particular, when $R$ be a \CrefAndHyperrefIfExist{definition:commutative_ring}{commutative ring}, and $M_i$ for $i = 1,\ldots,k$ and $N$ are $R$-modules, we may speak of a multilinear map $\Phi: M_1 \times \cdots \times M_k \to N$. We may thus also speak of multilinear maps $M^r \to R$ for $r \geq 0$

    % \item Let $R$ be a \CrefAndHyperrefIfExist{definition:commutative_ring}{commutative ring}. Let $M_i$ be $R$-modules for $i = 1,\ldots, k$, and let $N$ be an $R$-module. A function $\Phi: M_1 \times \cdots \times M_k \to N$ usually said to be \hldef{multilinear map} if it is multilinear in the sense of \ref{item:multilinear_map_of_modules_over_rings_distinct} and is \CrefAndHyperrefIfExist{definition:homomorphism_of_modules_over_a_ring}{$R$-linear} in each variable, i.e. satisfies \label{item:multilinear_map_of_modules_over_rings_commutative}
    % $$\Phi(m_1,\ldots,m_{j-1}, r_jm_j,m_{j+1},\ldots,m_k) = r_j\Phi(m_1,\ldots,m_{j-1},m_j,m_{j+1},\ldots,m_k)$$
    % for all $m_i \in M_i$ for $i =1,\ldots,k$, all $j = 1,\ldots,k$, and all $r_j \in R$. In particular, a \emph{multilinear map} where the modules involved are all modules of a commutative ring will most usually refer to a multilinear map in this sense rather than in the sense of \ref{item:multilinear_map_of_modules_over_rings_distinct}.

    % \TODO{$M^k$}
    % \item Let $R$ be a commutative ring and let $M$ be an $R$-module. A \hldef{multilinear form} is a multilinear map $M^k \to R$ in the sense of \ref{item:multilinear_map_of_modules_over_rings_commutative}.

    Additionally, we may speak of \hldef{bilinear maps/forms}, \hldef{trilinear maps/forms}, etc.
\end{definition}
\begin{definition} \label{definition:symmetric_antisymmetric_alternating_multilienar_forms_over_rings}
    Let $R$ be a \CrefAndHyperrefIfExist{definition:ring}{(not necessarily commutative) ring}, $M$ an \CrefAndHyperrefIfExist{definition:homomorphism_of_modules_over_a_ring}{two-sided $R$-module}, $k \in \bbN$ and $\Phi: M^k \to R$ a \CrefAndHyperrefIfExist{definition:multilinear_map_of_modules_over_rings}{multilinear form}
\begin{enumerate}
    \item $\Phi$ is \hldef{symmetric} if for every \CrefAndHyperrefIfExist{definition:symmetric_group_on_a_set}{permutation} $\sigma \in S_k$ and all $x_1,\dots,x_k \in M$,
    $$
    \Phi(x_1, ..., x_k) = \Phi(x_{\sigma(1)}, ..., x_{\sigma(k)}).
    $$

    \item $\Phi$ is \hldef{antisymmetric} if for all $\sigma \in S_k$ and all $x_1,\dots,x_k \in M$,
    $$ \Phi(x_{\sigma(1)}, ..., x_{\sigma(k)}) = \operatorname{sgn}(\sigma) \Phi(x_1, ..., x_k), $$
    where $\operatorname{sgn}(\sigma)$ is the \CrefAndHyperrefIfExist{definition:signature_of_an_element_of_the_symmetric_group_on_a_finite_set}{sign} of the permutation $\sigma$.
    
    \item $\Phi$ is \hldef{alternating} if whenever $x_i = x_j$ for some $i \neq j$, $\Phi(x_1, ..., x_k) = 0$. Every alternating form is antisymmetric, since if $x_i = x_j$ and we swap coordinates, both terms are zero.
\end{enumerate}

For $k=2$, these definitions specialize to \hldef{bilinear forms}; in particular:
\begin{itemize}
    \item $\Phi$ is symmetric if $\Phi(x, y) = \Phi(y, x)$.
    \item $\Phi$ is antisymmetric if $\Phi(x, y) = -\Phi(y, x)$.
    \item $\Phi$ is alternating if $\Phi(x, x) = 0$ for all $x$.
\end{itemize}
\end{definition}
% \begin{definition}[Chain complex in an additive category] \label{definition:chain_complex_of_objects_in_an_additive_category}
% Let $\mathcal{A}$ be an \hyperrefIfExists{definition:additive_category_preadditive_category}{preadditive category} and let $I$ be a totally ordered set (typically $\mathbb{Z}$, but $I \subseteq \mathbb{Z}$ is also allowed). 
% \begin{enumerate}
%     \item A \hldef{chain complex} $(K^\bullet, d^\bullet)$ in $\mathcal{A}$ indexed by $I$ consists of:
%     \begin{itemize}
%         \item Objects $\{ K^i \}_{i \in I}$ in $\mathcal{A}$, called the \hldef{terms in degree $i$},
%         \item Morphisms $d^i: K^i \to K^{i+1}$ in $\mathcal{A}$, called the \hldef{differentials in degree $i$},
%     \end{itemize}
%     such that for every $i \in I$, $d^{i+1} \circ d^i = 0$. That is,
%     $$ K^\bullet: \cdots \xrightarrow{d^{i-2}} K^{i-1} \xrightarrow{d^{i-1}} K^i \xrightarrow{d^i} K^{i+1} \xrightarrow{d^{i+1}} \cdots $$
%     with $d^{i+1}d^i = 0$ for all $i$. We might typically use notation such as \hl{$K^\bullet = (K^i, d^i)_{i \in I}$} to denote a chain complex in $\mathcal{A}$.

%     A cochain complex can be defined similarly/dually.

%     \item Let $K^\bullet = (K^i, d_K^i)$ and $L^\bullet = (L^i, d_L^i)$ be \CrefAndHyperrefIfExist{definition:chain_complex_of_objects_in_an_additive_category}{chain complexes} in $\mathcal{A}$ indexed by the same set $I$. 
%     A \hldef{morphism of chain complexes} (or \hldef{chain map})
%     $$ f^\bullet: K^\bullet \to L^\bullet $$
%     consists of morphisms $f^i: K^i \to L^i$ for all $i \in I$, such that for every $i \in I$,
%     $$ d_L^i \circ f^i = f^{i+1} \circ d_K^i, $$
%     i.e., the following diagram commutes for all $i$:

%     $$ \begin{array}{ccc} K^i & \xrightarrow{d_K^i} & K^{i+1} \\ \downarrow{f^i} && \downarrow{f^{i+1}} \\ L^i & \xrightarrow{d_L^i} & L^{i+1} \end{array}.$$

% \end{enumerate}

% There is then a category, often denoted by \hl{$\mathrm{Ch}(\mathcal{A})$} or \hl{$\mathbf{Ch}(\mathcal{A})$}, whose objects are chain complexes in $\calA$ and whose morphisms are morphisms of chain complexes. In particular, we may denote by 
% $$\hlin{\operatorname{Hom}(K^\bullet, L^\bullet)=  \operatorname{Hom}_{\mathrm{Ch}(\mathcal{A})}(K^\bullet, L^\bullet)}$$
% the set of chain maps $K^\bullet \to L^\bullet$; it is in fact an abelian group.

% A \hldef{morphism of cochain complexes} is defined similarly, and we similarly denote by \hl{$\mathrm{Ch}(\mathcal{A})$} or \hl{$\mathbf{Ch}(\mathcal{A})$} the caetgory of cochain complexes in $\calA$. 


% \TextIfExists{definition:dg_category_over_a_ring}{
% If $k$ is a \CrefAndHyperrefIfExist{definition:commutative_ring}{commutative ring} such that $\Hom_\calA(X,Y)$ is \CrefAndHyperrefIfExist{definition:category_enriched_in_a_monoidal_category}{enriched in} the category of \CrefAndHyperrefIfExist{definition:module_of_a_ring}{$k$-modules}, then $\mathrm{Ch}(\calA)$ \CrefAndHyperref{definition:category_of_chain_complexes_of_objects_in_an_additive_category_as_a_dg_category}{can be equipped with} the structure of a \CrefAndHyperrefIfExist{definition:dg_category_over_a_ring}{dg-category over $k$}.
% }


% \end{definition}

\begin{definition}[Chain complex in a preadditive category] \label{definition:chain_complex_of_objects_in_an_additive_category}
Let $\mathcal{A}$ be a \hyperrefIfExists{definition:additive_category_preadditive_category}{preadditive category} and let $I$ be a totally ordered set (typically $\mathbb{Z}$, but $I \subseteq \mathbb{Z}$ is also allowed). 
\begin{enumerate}
    \item A \hldef{chain complex} $(K_\bullet, d_\bullet)$ in $\mathcal{A}$ indexed by $I$ is the homological convention for sequences with decreasing degrees. It consists of:
    \begin{itemize}
        \item Objects $\{ K_i \}_{i \in I}$ in $\mathcal{A}$, called the \hldef{terms in degree $i$},
        \item Morphisms $d_i: K_i \to K_{i-1}$ in $\mathcal{A}$, called the \hldef{boundary maps} or \hldef{differentials in degree $i$},
    \end{itemize}
    such that for every $i \in I$, $d_{i-1} \circ d_i = 0$. That is,
    $$ K_\bullet: \cdots \xrightarrow{d_{i+1}} K_i \xrightarrow{d_i} K_{i-1} \xrightarrow{d_{i-1}} K_{i-2} \xrightarrow{} \cdots $$
    with $d_{i-1}d_i = 0$ for all $i$. We typically use the notation \hl{$K_\bullet = (K_i, d_i)_{i \in I}$}.



    \item Dually, a \hldef{cochain complex} $(K^\bullet, d^\bullet)$ in $\mathcal{A}$ follows the \hldef{cohomological convention} with increasing degrees. It consists of objects $\{ K^i \}_{i \in I}$ and \hldef{coboundary maps} $d^i: K^i \to K^{i+1}$ such that $d^{i+1} \circ d^i = 0$:
    $$ K^\bullet: \cdots \xrightarrow{d^{i-1}} K^i \xrightarrow{d^i} K^{i+1} \xrightarrow{d^{i+1}} K^{i+2} \xrightarrow{} \cdots $$
    We typically use the notation \hl{$K^\bullet = (K^i, d^i)_{i \in I}$}.

    \item Let $K_\bullet = (K_i, d_i^K)$ and $L_\bullet = (L_i, d_i^L)$ be \CrefAndHyperrefIfExist{definition:chain_complex_of_objects_in_an_additive_category}{chain complexes} in $\mathcal{A}$ indexed by the same set $I$. A \hldef{morphism of chain complexes} (or \hldef{chain map})
    $$ f_\bullet: K_\bullet \to L_\bullet $$
    consists of morphisms $f_i: K_i \to L_i$ for all $i \in I$, such that for every $i \in I$, the following diagram commutes:
    $$ \begin{array}{ccc} K_i & \xrightarrow{d_i^K} & K_{i-1} \\ \downarrow{f_i} && \downarrow{f_{i-1}} \\ L_i & \xrightarrow{d_i^L} & L_{i-1} \end{array} $$
    i.e., $d_i^L \circ f_i = f_{i-1} \circ d_i^K$. 



    A \hldef{morphism of cochain complexes} $f^\bullet: K^\bullet \to L^\bullet$ is defined similarly, satisfying the commutativity condition $d_L^i \circ f^i = f^{i+1} \circ d_K^i$.
\end{enumerate}

The collection of these objects and morphisms forms a category. Notation for these categories is as follows:
\begin{itemize}
    \item \hl{$\mathrm{Ch}(\mathcal{A})$} or \hl{$\mathbf{Ch}(\mathcal{A})$} is often used as a general term.
    \item To be explicit about the indexing convention, one uses \hl{$\mathrm{Ch}_\bullet(\mathcal{A})$} for chain complexes and \hl{$\mathrm{Ch}^\bullet(\mathcal{A})$} (or sometimes $\mathrm{CoCh}(\mathcal{A})$) for cochain complexes.
    \item The set of chain maps between two complexes is denoted by $\hlin{\operatorname{Hom}_{\mathrm{Ch}(\mathcal{A})}(K_\bullet, L_\bullet)}$; it is an abelian group under pointwise addition $(f+g)_i = f_i + g_i$.
\end{itemize}

\TextIfExists{definition:dg_category_over_a_ring}{
If $k$ is a \CrefAndHyperrefIfExist{definition:commutative_ring}{commutative ring} such that $\Hom_\calA(X,Y)$ is \CrefAndHyperrefIfExist{definition:category_enriched_in_a_monoidal_category}{enriched in} the category of \CrefAndHyperrefIfExist{definition:module_of_a_ring}{$k$-modules}, then $\mathrm{Ch}(\calA)$ \CrefAndHyperref{definition:category_of_chain_complexes_of_objects_in_an_additive_category_as_a_dg_category}{can be equipped with} the structure of a \CrefAndHyperrefIfExist{definition:dg_category_over_a_ring}{dg-category over $k$}.
}
\end{definition}

% 
\begin{remark} \label{remark:cohomological_vs_homological_conventions}
    The convention used to define chain complexes in \Cref{definition:chain_complex_of_objects_in_an_additive_category} is a \emph{cohomological one} --- note that indices are written as superscripts and increase when ``following the arrows''. Such a chain complex may also be referred to as a \hldef{cochain complex} or a \hldef{cohomological chain complex} to emphasize an adoption of a cohomological convention. 

    The dual convention would be a \emph{homological one}, in which indices are written as subscripts and decrease when ``following the arrow''. As such, one may speak of a \hldef{(homological) chain complex} $(K_\bullet, d_\bullet)$ indexed by $I$ as consisting of:

    \begin{itemize}
    \item Objects $\{ K_i \}_{i \in I}$ in $\mathcal{A}$, called the \hldef{terms in degree $i$},
    \item Morphisms $d_i: K_i \to K_{i-1}$ in $\mathcal{A}$, called the \hldef{differentials in degree $i$},
    \end{itemize}
    such that for every $i \in I$, $d_{i-1} \circ d_i = 0$. That is,
    $$ 
    K_\bullet: \cdots \xrightarrow{d_{i+1}} K_i \xrightarrow{d_i} K_{i-1} \xrightarrow{d_{i-1}} K_{i-2} \xrightarrow{d_{i-2}} \cdots
    $$
    with $d_{i-1} d_i = 0$ for all $i$. We might typically use notation such as \hl{$K_\bullet = (K_i, d_i)_{i \in I}$} to denote a chain complex in $\mathcal{A}$.

    The differences between the conventions persist --- for example, cohomological objects are usually written with superscript indicees whereas homological objects are usually written with subscript indicees.
\end{remark}
%\begin{convention} \label{convention:homological_algebra_is_discussed_in_cohomological_terms}
    When discussing homological algebra in abstract terms, we may often adopt the homological convention in some discussions and the cohomological convention in others \CrefIfExists{remark:cohomological_vs_homological_conventions}.
    %; for instance, indices are written as superscripts and increase along the direction of the arrows in chain complexes. 
\end{convention}

% \begin{definition}[Morphisms of chain complexes] \label{definition:chain_complex_of_objects_in_an_additive_category}
Let $\mathcal{A}$ be an \CrefAndHyperrefIfExist{definition:additive_category}{additive category}, and let $K^\bullet = (K^i, d_K^i)$ and $L^\bullet = (L^i, d_L^i)$ be \CrefAndHyperrefIfExist{definition:chain_complex_of_objects_in_an_additive_category}{chain complexes} in $\mathcal{A}$ indexed by the same set $I$. 
A \hldef{morphism of chain complexes} (or \hldef{chain map})
$$ f^\bullet: K^\bullet \to L^\bullet $$
consists of morphisms $f^i: K^i \to L^i$ for all $i \in I$, such that for every $i \in I$,
$$ d_L^i \circ f^i = f^{i+1} \circ d_K^i, $$
i.e., the following diagram commutes for all $i$:

$$ \begin{array}{ccc} K^i & \xrightarrow{d_K^i} & K^{i+1} \\ \downarrow{f^i} && \downarrow{f^{i+1}} \\ L^i & \xrightarrow{d_L^i} & L^{i+1} \end{array}.$$

There is then a category, often denoted by \hl{$\mathrm{Ch}(\mathcal{A})$} or \hl{$\mathbf{Ch}(\mathcal{A})$}, whose objects are chain complexes in $\calA$ and whose morphisms are morphisms of chain complexes. In particular, we may denote by 
$$\hlin{\operatorname{Hom}(K^\bullet, L^\bullet)=  \operatorname{Hom}_{\mathrm{Ch}(\mathcal{A})}(K^\bullet, L^\bullet)}$$
the set of chain maps $K^\bullet \to L^\bullet$; it is in fact an abelian group.

A \hldef{morphism of cochain complexes} is defined similarly, and we similarly denote by \hl{$\mathrm{Ch}(\mathcal{A})$} or \hl{$\mathbf{Ch}(\mathcal{A})$} the caetgory of cochain complexes in $\calA$. 
\end{definition}

% See Also
% 
\begin{proposition} \label{proposition:category_of_chain_complexes_in_an_additive_category_is_additive}
Let $\mathcal{A}$ be an \hyperrefIfExists{definition:additive_category}{additive category}. 
\begin{enumerate}
    \item The category \hyperrefIfExists{definition:chain_complex_of_objects_in_an_additive_category}{$\mathrm{Ch}(\calA)$} of chain complexes is itself and additive category.

    \item If $\calA$ is an \hyperrefIfExists{definition:abelian_category}{abelian category}, then $\mathrm{Ch}(\calA)$ is an abelian category.

    \item If $\calA$ is an \hyperrefIfExists{definition:abelian_category}{abelian category} satisfying Grothendieck's axiom \CrefAndHyperrefIfExist{definition:grothendiecks_additional_axioms_for_abelian_categories}{AB$n$ (resp. AB$n^*$)} for $n \in \{3,4,5,6\}$, then $\mathrm{Ch}(\calA)$ also satisfies AB$n$ (resp. AB$n^*$). If $\calA$ is a \CrefAndHyperrefIfExist{definition:grothendiecks_additional_axioms_for_abelian_categories}{Grothendieck abelian category}, then so is $\mathrm{Ch}(\calA)$
\end{enumerate}
\end{proposition}
\begin{proof}
    Combine \Cref{proposition:category_of_chain_complexes_of_objects_in_a_preadditive_category_is_equivalent_to_the_category_of_additive_functors_from_the_walking_chain_complex_category} and \Cref{lemma:additive_functor_category_from_small_preadditive_categories_preserves}.
\end{proof}

% \begin{corollary}
% Let $\calB$ be a \CrefAndHyperrefIfExist{definition:additive_category}{preadditive category}.
% \begin{enumerate}
%     \item The \CrefAndHyperrefIfExist{definition:chain_complex_of_objects_in_an_additive_category}{(co)chain complex category} $\text{Ch}(\calB)$ is preadditive. If $\calB$ is additionally \CrefAndHyperrefIfExist{definition:additive_category}{additive}/\CrefAndHyperrefIfExist{definition:abelian_category}{abelian}, then so is $\text{Ch}(\calB)$.

%     \item If $\calB$ is an abelian category with property \CrefAndHyperrefIfExist{definition:grothendiecks_additional_axioms_for_abelian_categories}{$ABn$ for $n = 3,4,5,6$ or $ABn^*$ for $n = 3,4,5$}, then $\text{Add}(\calA, \calB)$ possesses the same property.
% \end{enumerate}

% \end{corollary}