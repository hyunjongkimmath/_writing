% packages
\documentclass[12pt]{amsart}
\usepackage{amsmath}
\usepackage{bm}
\usepackage{amssymb}
\usepackage{amscd}
\usepackage{amsthm}
\usepackage{aliascnt}  % Add this to make it so that Cref can print out the appropriate environment name, instead of the parent environment name. (load after amsthm, before cleveref is fine)
\usepackage{hyperref}
\usepackage{cleveref} % cleveref must be loaded after amsthm, but before the theorem environments are defined.
\usepackage{fullpage,enumitem}
\usepackage{units}
\usepackage[all]{xy}
\usepackage{xcolor} % for \color
\usepackage{mathrsfs} % for \mathscr
% \usepackage{cleveref}
%\xyoption{curve}
\usepackage{url}
\usepackage{tikz-cd}
\usepackage{mathtools}
\usepackage{stmaryrd} % for \mapsfrom

\usepackage[OT2,T1]{fontenc}
\DeclareSymbolFont{cyrletters}{OT2}{wncyr}{m}{n}
\DeclareMathSymbol{\Sha}{\mathalpha}{cyrletters}{"58}


\usepackage{import}
\usepackage{dsfont} %For mathds{1}

% ----- Define theorem environments -----

% The definitions of the theorem environments below are set up so that they use cref and aliascnt in tandem --- aliascnt will make sure that cref prints out the correct name of each environment, instead of the parent environment name.

% Main theorem environment, counter resets per section
\theoremstyle{definition}
\newtheorem{theorem}{Theorem}[section]
\newtheorem*{theorem*}{Theorem}
\numberwithin{theorem}{subsection} % optional; usually fine with \newtheorem[section]

% Define all theorem-like environments that share 'theorem' counter using aliascnt:

% Lemma
\newaliascnt{lemma}{theorem}
\newtheorem{lemma}[lemma]{Lemma}
\aliascntresetthe{lemma}

% Corollary
\newaliascnt{corollary}{theorem}
\newtheorem{corollary}[corollary]{Corollary}
\aliascntresetthe{corollary}

% Proposition
\newaliascnt{proposition}{theorem}
\newtheorem{proposition}[proposition]{Proposition}
\aliascntresetthe{proposition}

% Claim
\newaliascnt{claim}{theorem}
\newtheorem{claim}[claim]{Claim}
\aliascntresetthe{claim}

% Claim
\newaliascnt{goal}{theorem}
\newtheorem{goal}[goal]{Goal}
\newtheorem*{goal*}{Goal}
\aliascntresetthe{goal}

% Remark
\newaliascnt{remark}{theorem}
\newtheorem{remark}[remark]{Remark}
\aliascntresetthe{remark}

% Switch to remark style for these environments
% \theoremstyle{remark}

% Definition
\newaliascnt{definition}{theorem}
\newtheorem{definition}[definition]{Definition}
\aliascntresetthe{definition}

% Notation
\newaliascnt{notation}{theorem}
\newtheorem{notation}[notation]{Notation}
\aliascntresetthe{notation}

% Convention
\newaliascnt{convention}{theorem}
\newtheorem{convention}[convention]{Convention}
\aliascntresetthe{convention}

% Context
\newaliascnt{context}{theorem}
\newtheorem{context}[context]{Context}
\aliascntresetthe{context}

% Question
\newaliascnt{question}{theorem}
\newtheorem{question}[question]{Question}
\aliascntresetthe{question}

%Problem
\newaliascnt{problem}{theorem}
\newtheorem{problem}[problem]{Problem}
\newtheorem*{problem*}{Problem}
\aliascntresetthe{problem}

% Example
\newaliascnt{example}{theorem}
\newtheorem{example}[example]{Example}
\aliascntresetthe{example}

% Conjecture
\newaliascnt{conjecture}{theorem}
\newtheorem{conjecture}[conjecture]{Conjecture}
\aliascntresetthe{conjecture}

% ----- Tell cleveref how to reference these environments -----

\crefname{theorem}{theorem}{theorems}
\Crefname{theorem}{Theorem}{Theorems}

\crefname{lemma}{lemma}{lemmas}
\Crefname{lemma}{Lemma}{Lemmas}

\crefname{corollary}{corollary}{corollaries}
\Crefname{corollary}{Corollary}{Corollaries}

\crefname{proposition}{proposition}{propositions}
\Crefname{proposition}{Proposition}{Propositions}

\crefname{claim}{claim}{claims}
\Crefname{claim}{Claim}{Claims}

\crefname{goal}{goal}{goals}
\Crefname{goal}{Goal}{Goals}

\crefname{remark}{remark}{remarks}
\Crefname{remark}{Remark}{Remarks}

\crefname{definition}{definition}{definitions}
\Crefname{definition}{Definition}{Definitions}

\crefname{notation}{notation}{notations}
\Crefname{notation}{Notation}{Notations}

\crefname{convention}{convention}{conventions}
\Crefname{convention}{Convention}{Conventions}

\crefname{context}{context}{contexts}
\Crefname{context}{Context}{Contexts}

\crefname{question}{question}{questions}
\Crefname{question}{Question}{Questions}

\crefname{problem}{problem}{problems}
\Crefname{problem}{Problem}{Problems}

\crefname{example}{example}{examples}
\Crefname{example}{Example}{Examples}

\crefname{conjecture}{conjecture}{conjectures}
\Crefname{conjecture}{Conjecture}{Conjectures}


%% theorem style environments

% \theoremstyle{plain}
% \newtheorem{theorem}{Theorem}[section]
% \numberwithin{theorem}{section}
% % \numberwithin{equation}{section}
% \newtheorem{lemma}[theorem]{Lemma}
% % \numberwithin{lemma}[theorem]{Lemma}
% \newtheorem{corollary}[theorem]{Corollary}
% \newtheorem{proposition}[theorem]{Proposition}
% \newtheorem{claim}[theorem]{Claim}
% \newtheorem{remark}[theorem]{Remark}

% \theoremstyle{remark}
% \newtheorem{definition}[theorem]{Definition}
% \newtheorem{notation}[theorem]{Notation}
% \newtheorem{convention}[theorem]{Convention}
% \newtheorem{question}[theorem]{Question}
% \newtheorem{example}[theorem]{Example}
% \newtheorem{conjecture}[theorem]{Conjecture}


% % To make sure that \Cref prints out the correct environment name
% % For standard theorem environments (usually cleveref knows these, but specifying doesn't hurt)
% \crefname{theorem}{theorem}{theorems}
% \Crefname{theorem}{Theorem}{Theorems}
% \crefname{lemma}{lemma}{lemmas}
% \Crefname{lemma}{Lemma}{Lemmas}
% \crefname{corollary}{corollary}{corollaries}
% \Crefname{corollary}{Corollary}{Corollaries}
% \crefname{proposition}{proposition}{propositions}
% \Crefname{proposition}{Proposition}{Propositions}
% \crefname{claim}{claim}{claims}
% \Crefname{claim}{Claim}{Claims}
% \crefname{remark}{remark}{remarks}
% \Crefname{remark}{Remark}{Remarks}
% % For remark-style environments
% \crefname{definition}{definition}{definitions}
% \Crefname{definition}{Definition}{Definitions}
% \crefname{notation}{notation}{notations}
% \Crefname{notation}{Notation}{Notations}
% \crefname{convention}{convention}{conventions}
% \Crefname{convention}{Convention}{Conventions}
% \crefname{question}{question}{questions}
% \Crefname{question}{Question}{Questions}
% \crefname{example}{example}{examples}
% \crefname{conjecture}{conjecture}{conjectures}
% \Crefname{conjecture}{Conjecture}{Conjectures}


%% Commenting command
\newcommand{\TODO}[1]{{\color{orange} ($\spadesuit$ TODO: #1)}}
\newcommand{\TODOdef}[1]{{\color{blue} ($\spadesuit$ TODO: #1)}}
\newcommand{\TODOpriority}[1]{{\color{red} ($\spadesuit$ TODO: #1)}}

\setlength{\parskip}{0.125in}
\setlength{\parindent}{0in}

%% Custom commands/mathoperators

\newcommand{\tickar}{\mathrel{\ooalign{\hss$\mapstochar\mkern 5mu$\hss\cr$\to$}}}
\newcommand\seq{\subseteq}
\newcommand\ssm{\smallsetminus}
\newcommand\sneq{\subsetneq}
\newcommand\sub\subset
\newcommand\longto\longrightarrow

\DeclareMathOperator\genus{genus}
\DeclareMathOperator\Jac{Jac}
\DeclareMathOperator\tor{tor}
\DeclareMathOperator\Div{Div}
\DeclareMathOperator\Pic{Pic}
\DeclareMathOperator\Cl{Cl}
\DeclareMathOperator\Prin{Prin}
\DeclareMathOperator\codim{codim}
\DeclareMathOperator\Sym{Sym}
\DeclareMathOperator\Aut{Aut}
\DeclareMathOperator\Gal{Gal}
\DeclareMathOperator\GL{GL}
\DeclareMathOperator\PGL{PGL}
\DeclareMathOperator\PSL{PSL}
\DeclareMathOperator\SL{SL}
\DeclareMathOperator\Hom{Hom}
\DeclareMathOperator\End{End}
\DeclareMathOperator\Mod{Mod}
\DeclareMathOperator\Ext{Ext}
\DeclareMathOperator\id{id}
\DeclareMathOperator\inv{inv}
\DeclareMathOperator\Core{Core}
\DeclareMathOperator\RHom{RHom}
\DeclareMathOperator\ad{ad}
\DeclareMathOperator\Spec{Spec}
\DeclareMathOperator\Supp{Supp}
\DeclareMathOperator\Perv{Perv}
\DeclareMathOperator\Ob{Ob}
\DeclareMathOperator\Add{Add}
\DeclareMathOperator\Ab{Ab}
\DeclareMathOperator\Sm{Sm}
\newcommand{\Sets}{\mathbf{Sets}}
\newcommand{\USets}{U\mathbf{-Sets}}
\DeclareMathOperator\Sch{Sch}
\DeclareMathOperator\et{\mathrm{\acute{e}t}}
\DeclareMathOperator\proet{\mathrm{pro\acute{e}t}}
\DeclareMathOperator\Shv{Shv}
\DeclareMathOperator\PreShv{PreShv}
\DeclareMathOperator\Sh{Sh}
\DeclareMathOperator\Spc{Spc}
\DeclareMathOperator\pt{pt}
\DeclareMathOperator\Nis{Nis}
\DeclareMathOperator\Char{Char}
\DeclareMathOperator\Frob{Frob}
\DeclareMathOperator\geom{geom}
\DeclareMathOperator\Fun{Fun}
\DeclareMathOperator\op{op}
\DeclareMathOperator\Tot{Tot}
\DeclareMathOperator\Tor{Tor}
\DeclareMathOperator\CoInd{CoInd}
\DeclareMathOperator\Ind{Ind}
\DeclareMathOperator\coker{coker}

\DeclareMathOperator\Map{Map}
\DeclareMathOperator\colim{colim}
\DeclareMathOperator\SeqSpec{SeqSpec}
\DeclareMathOperator\Top{\mathbf{Top}}
\newcommand{\hospectra}{\operatorname{Ho}(\operatorname{Spectra})}
\newcommand{\hocofib}{\operatorname{hocofib}}
% \newcommand\Map{\mathrm{Map}}
% \newcommand{\colim}{\operatorname{colim}}
% \newcommand{\SeqSpec}{\operatorname{SeqSpec}}
% \newcommand{\Top}{\operatorname{Top}}

\DeclareMathOperator\Conn{\operatorname{Conn}}
\DeclareMathOperator\Connflat{\operatorname{Conn}^{\mathrm{flat}}}
\DeclareMathOperator\Connholflat{\operatorname{Conn}_{\mathrm{hol}}^{\mathrm{flat}}}
\DeclareMathOperator\Connregflat{\operatorname{Conn}_{\mathrm{reg}}^{\mathrm{flat}}}
\DeclareMathOperator\Rep{\operatorname{Rep}}
\DeclareMathOperator\VectorSpaces{\operatorname{Vec}}
\DeclareMathOperator\LocSys{\operatorname{LocSys}}

\DeclareMathOperator\Bun{\operatorname{Bun}}
\DeclareMathOperator\BunG{\operatorname{Bun}_G}

\newcommand\defi[1]{{\sl #1}}

\newcommand\A{\ensuremath{\mathbb{A}}}
\newcommand\C{\ensuremath{\mathbb{C}}}
\newcommand\F{\ensuremath{\mathbb{F}}}
\newcommand\N{\ensuremath{\mathbb{N}}}
\newcommand\Q{\ensuremath{\mathbb{Q}}}
\newcommand\R{\ensuremath{\mathbb{R}}}
\newcommand\Z{\ensuremath{\mathbb{Z}}}

\DeclareMathOperator\Prob{\operatorname{Prob}}
\DeclareMathOperator\rk{\operatorname{rk}}
\DeclareMathOperator\Sel{\operatorname{Sel}}

% \renewcommand{\P}{\ensuremath{\mathbb{P}}}

% \newcommand\bmu{{\boldsymbol\mu}}

% \newcommand{\p}{\mathfrak{p}}

% \newcommand{\gp}[1]{\langle\,#1\,\rangle}

% \newcommand\squig{\ensuremath{\quad\rightsquigarrow\quad}}
% \newcommand\thus{\ensuremath{\quad\Rightarrow\quad}}
% \newcommand\plus{\quad -- \quad}
% \newcommand\point{\ensuremath{\tiny \bullet\quad}}
% \newcommand\qthus{\quad\thus}

% % \newcommand{\lto}{\longrightarrow}

\newcommand{\np}{\newpage}
\newcommand{\bs}{\bigskip}
\newcommand{\ms}{\medskip}

% \newenvironment{myenum}{%
%         \begin{enumerate}\setlength{\itemsep}{0.075in}}
% {\end{enumerate}}

% % provide problem set and subset lists
% %  - use [resume] option to starting numbering where previous list ended
% \newlist{probset}{enumerate}{10}
% \setlist[probset]{itemsep=0.2in,labelsep=0.125in,label=\arabic*.}

% \newlist{subprobset}{enumerate}{10}
% \setlist[subprobset]{itemsep=0.125in,labelsep=0.125in,label=\roman*)}

% redefine labels with extra spacing
\renewcommand{\labelenumi}{\arabic{enumi}.}
\renewcommand{\theequation}{\Alph{equation}}

\newcommand\bbA{\mathbb{A}}
\newcommand\bbB{\mathbb{B}}
\newcommand\bbC{\mathbb{C}}
\newcommand\bbD{\mathbb{D}}
\newcommand\bbE{\mathbb{E}}
\newcommand\bbF{\mathbb{F}}
\newcommand\bbG{\mathbb{G}}
\newcommand\bbH{\mathbb{H}}
\newcommand\bbI{\mathbb{I}}
\newcommand\bbJ{\mathbb{J}}
\newcommand\bbK{\mathbb{K}}
\newcommand\bbL{\mathbb{L}}
\newcommand\bbM{\mathbb{M}}
\newcommand\bbN{\mathbb{N}}
\newcommand\bbO{\mathbb{O}}
\newcommand\bbP{\mathbb{P}}
\newcommand\bbQ{\mathbb{Q}}
\newcommand\bbR{\mathbb{R}}
\newcommand\bbS{\mathbb{S}}
\newcommand\bbT{\mathbb{T}}
\newcommand\bbU{\mathbb{U}}
\newcommand\bbV{\mathbb{V}}
\newcommand\bbW{\mathbb{W}}
\newcommand\bbX{\mathbb{X}}
\newcommand\bbY{\mathbb{Y}}
\newcommand\bbZ{\mathbb{Z}}

\newcommand\bfA{\mathbf{A}}
\newcommand\bfB{\mathbf{B}}
\newcommand\bfC{\mathbf{C}}
\newcommand\bfD{\mathbf{D}}
\newcommand\bfE{\mathbf{E}}
\newcommand\bfF{\mathbf{F}}
\newcommand\bfG{\mathbf{G}}
\newcommand\bfH{\mathbf{H}}
\newcommand\bfI{\mathbf{I}}
\newcommand\bfJ{\mathbf{J}}
\newcommand\bfK{\mathbf{K}}
\newcommand\bfL{\mathbf{L}}
\newcommand\bfM{\mathbf{M}}
\newcommand\bfN{\mathbf{N}}
\newcommand\bfO{\mathbf{O}}
\newcommand\bfP{\mathbf{P}}
\newcommand\bfQ{\mathbf{Q}}
\newcommand\bfR{\mathbf{R}}
\newcommand\bfS{\mathbf{S}}
\newcommand\bfT{\mathbf{T}}
\newcommand\bfU{\mathbf{U}}
\newcommand\bfV{\mathbf{V}}
\newcommand\bfW{\mathbf{W}}
\newcommand\bfX{\mathbf{X}}
\newcommand\bfY{\mathbf{Y}}
\newcommand\bfZ{\mathbf{Z}}

\newcommand\calA{\mathcal{A}}
\newcommand\calB{\mathcal{B}}
\newcommand\calC{\mathcal{C}}
\newcommand\calD{\mathcal{D}}
\newcommand\calE{\mathcal{E}}
\newcommand\calF{\mathcal{F}}
\newcommand\calG{\mathcal{G}}
\newcommand\calH{\mathcal{H}}
\newcommand\calI{\mathcal{I}}
\newcommand\calJ{\mathcal{J}}
\newcommand\calK{\mathcal{K}}
\newcommand\calL{\mathcal{L}}
\newcommand\calM{\mathcal{M}}
\newcommand\calN{\mathcal{N}}
\newcommand\calO{\mathcal{O}}
\newcommand\calP{\mathcal{P}}
\newcommand\calQ{\mathcal{Q}}
\newcommand\calR{\mathcal{R}}
\newcommand\calS{\mathcal{S}}
\newcommand\calT{\mathcal{T}}
\newcommand\calU{\mathcal{U}}
\newcommand\calV{\mathcal{V}}
\newcommand\calW{\mathcal{W}}
\newcommand\calX{\mathcal{X}}
\newcommand\calY{\mathcal{Y}}
\newcommand\calZ{\mathcal{Z}}
\newcommand\calAb{\mathcal{A}b}

\newcommand{\sphere}[1]{S^{#1}}
\newcommand{\Sone}{\sphere{1}}
\newcommand{\Sn}{\sphere{n}}
\newcommand{\Topcg}{\operatorname{Top}_{\mathrm{cg}}}
\newcommand{\strict}{\mathrm{strict}}
\newcommand{\stable}{\mathrm{stable}}
\newcommand{\SH}{\mathcal{SH}}

\newcommand\Dbc{D_{c}^{b}}
\newcommand\scrH{\mathscr{H}}
\newcommand\scrO{\mathscr{O}}
\newcommand\scrU{\mathscr{U}}
\newcommand\pscrH{{}^p\mathscr{H}}
\newcommand\Qellbar{\overline{\bbQ}_\ell}
\newcommand\Qell{\bbQ_\ell}
\newcommand\bark{\overline{k}}
\newcommand\barK{\overline{K}}
\newcommand\barx{\overline{x}}
\newcommand\bars{\overline{s}}

\newcommand{\Dleqn}[1][n]{D^{\leq {#1}}}
\newcommand{\Dgeqn}[1][n]{D^{\geq {#1}}}
\newcommand{\Dleqzero}{\Dleqn[0]}
\newcommand{\Dgeqzero}{\Dgeqn[0]}

\newcommand{\stDleqn}[1][n]{{}^{\text{st}}D^{\leq {#1}}}
\newcommand{\stDgeqn}[1][n]{{}^{\text{st}}D^{\geq {#1}}}
\newcommand{\stDleqzero}{\stDleqn[0]}
\newcommand{\stDgeqzero}{\stDgeqn[0]}

\newcommand{\pDleqn}[1][n]{{}^{p}D^{\leq {#1}}}
\newcommand{\pDgeqn}[1][n]{{}^{p}D^{\geq {#1}}}
\newcommand{\pDleqzero}{\pDleqn[0]}
\newcommand{\pDgeqzero}{\pDgeqn[0]}

\newcommand{\tauleqn}[1][n]{\tau_{\leq {#1}}}
\newcommand{\taugeqn}[1][n]{\tau_{\geq {#1}}}
\newcommand{\tauleqzero}{\tauleqn[0]}
\newcommand{\taugeqzero}{\taugeqn[0]}
\newcommand{\sttauleqn}[1][n]{{}^{\text{st}}\tauleqn[#1]}
\newcommand{\sttaugeqn}[1][n]{{}^{\text{st}}\taugeqn[#1]}
\newcommand{\sttauleqzero}{\sttauleqn[0]}
\newcommand{\sttaugeqzero}{\sttaugeqn[0]}
\newcommand{\ptauleqn}[1][n]{{}^{p}\tauleqn[#1]}
\newcommand{\ptaugeqn}[1][n]{{}^{p}\taugeqn[#1]}
\newcommand{\ptauleqzero}{\ptauleqn[0]}
\newcommand{\ptaugeqzero}{\ptaugeqn[0]}


\newcommand{\Deltaop}{\Delta^{\mathrm{op}}}
\newcommand{\SmNis}[1][S]{(\Sm/#1)_{\mathrm{Nis}}}
\newcommand{\SmSNis}{\SmNis}
\newcommand{\SmkNis}{\SmNis[k]}
\newcommand{\ShvNisSmS}{\ShvNisSm[S]}
\newcommand{\ShvNisSm}[1][S]{\Shv(\SmNis[#1])}
\newcommand{\simpsheaves}{\Deltaop\Shv}
\newcommand{\simpnissheaves}[1][S]{\Deltaop\ShvNisSm[#1]}
\newcommand{\pointedsimpnissheaves}{\Deltaop\ShvNisSmS_\bullet}
\newcommand{\pinAone}[1][n]{\pi_{#1}^{\bbA^1}}
\newcommand{\HnAone}[1][n]{H_{#1}^{\bbA^1}}
\newcommand{\tildeHnAone}[1][n]{\widetilde{H}_{#1}^{\bbA^1}}
% \author{Hyun Jong Kim}
\newcommand{\MWKtheory}[1][*]{K_{#1}^{MW}}
\newcommand{\MKtheory}[1][*]{K_{#1}^{M}}
\newcommand{\GWring}{\operatorname{GW}}
\newcommand{\MWKtheorysheaf}[1][*]{\underline{\mathbf{K}}_{#1}^{MW}}
\newcommand{\DAoneloc}{D_{\bbA^1\text{-loc}}}
\newcommand{\DAone}{D_{\bbA^1}}
\newcommand{\AoneQis}{\bbA^1\text{-Qis}}
\newcommand{\LAoneab}{L_{\bbA^1}^{\text{ab}}}
\newcommand{\CstarAone}{C_*^{\bbA^1}}

\newcommand{\Fq}{\bbF_q}
\newcommand{\Fqn}{\bbF_{q^n}}
\newcommand{\Fp}{\bbF_p}
\newcommand{\Fqbar}{\overline{\bbF}_q}
\newcommand{\Lotimes}{\stackrel{\mathrm{L}}{\otimes}}

\newcommand{\pioneet}{\pi_{1}^{\mathrm{\acute{e}t}}}

% macros_highlighting.tex
% %% command for highlighting definitions and notations
% The markings are done with the `\hl` command of the `soul` package, a custom `\mathcolorbox` command
% defined using the `\colorbox` command of the `xcolor` package, and the `\colorbox` command in tandem with the `\begin{{minipage}}`
% command. These markings may cause syntax errors when this autogenerated document
% is compiled. If such syntax errors occur, one may manually correct such syntax errors; if all
% attempts at correcting such a syntax error fails, it is recommended to remove the \hl or \colorbox
% command appropriately.
\usepackage{soul}
\usepackage{xcolor}
\sethlcolor{lightgray}
\newcommand{\mathcolorbox}[2]{\colorbox{#1}{$\displaystyle #2$}}

%Custom helper commands for the highlighting
\newcommand{\hldef}[1]{\sethlcolor{lightgray}\hl{\em #1}}
\newcommand{\hlin}[1]{\mathcolorbox{lightgray}{#1}}
\newcommand{\hlalign}[1]{
    \colorbox{lightgray}{
    \begin{minipage}{\dimexpr\textwidth-2\fboxsep}
        #1
    \end{minipage}
    }
}


%% command for highlighting definitions and notations
% The markings are done with the `\hl` command of the `soul` package, a custom `\mathcolorbox` command
% defined using the `\colorbox` command of the `xcolor` package, and the `\colorbox` command in tandem with the `\begin{{minipage}}`
% command. These markings may cause syntax errors when this autogenerated document
% is compiled. If such syntax errors occur, one may manually correct such syntax errors; if all
% attempts at correcting such a syntax error fails, it is recommended to remove the \hl or \colorbox
% command appropriately.
\usepackage{soul}
\usepackage{xcolor}
\sethlcolor{lightgray}
\newcommand{\mathcolorbox}[2]{\colorbox{#1}{$\displaystyle #2$}}

%Custom helper commands for the highlighting
\newcommand{\hldef}[1]{\sethlcolor{lightgray}\hl{\em #1}}
\newcommand{\hlin}[1]{\mathcolorbox{lightgray}{#1}}
\newcommand{\hlalign}[1]{
    \colorbox{lightgray}{
    \begin{minipage}{\dimexpr\textwidth-2\fboxsep}
        #1
    \end{minipage}
    }
}

%Custom helper commands for glossary
\newcommand{\defin}[3]{\newglossaryentry{#2}{#3}\hldef{#1}\glsadd[format=textbf]{#2}} % #1 is the "display text", #2 is the label of the glossary entry and #3 is the inputs to pass to \newglossaryentry

% \newcommand{\defin}[3]{\hldef{#1}} % Use this when not using the glossasry

\newcommand{\notat}[3]{\newglossaryentry{#2}{#3}\hl{$#1$}\glsadd[format=textbf]{#2}}
\newcommand{\notatin}[3]{\newglossaryentry{#2}{#3}$$\hlin{#1}$$\glsadd[format=textbf]{#2}}
% \newcommand{\notat}[2]{\hldef{#1}}

% \colorbox{lightgray}{
% \begin{minipage}{\dimexpr\textwidth-2\fboxsep}
% \begin{align*}
%     \Dleqn &= \Dleqzero[-n] \\ 
%     \Dgeqn &= \Dgeqzero[-n]
% \end{align*}
% \end{minipage}
% }

\makeatletter %For printing refs only if the labels exist
\newcommand\CrefIfExists[1]{%
  \@ifundefined{r@#1}{%
    % label undefined: print nothing or something else
  }{%
    % label defined; print Definition and the reference number
    \space(\Cref{#1})%
  }%
}
\makeatother

\makeatletter
\newcommand{\hyperrefIfExists}[2]{%
  \@ifundefined{r@#1}{%
    % Label undefined: just print the text (no link)
    #2%
  }{%
    % Label defined: print the text with hyperlink using \hyperref
    \hyperref[#1]{#2}%
  }%
}
\makeatother

\newcommand{\CrefAndHyperrefIfExist}[2]{% #1: The reference #2 The text
\hyperrefIfExists{#1}{#2}\CrefIfExists{#1}}

\newcommand{\CrefAndHyperref}[2]{% #1: The reference #2 The text
\hyperref[#1]{#2}\CrefIfExists{#1}}



\makeatletter %For printing refs only if the labels exist
\newcommand\TextIfExists[2]{%
  \@ifundefined{r@#1}{%
    % label undefined: print nothing or something else
  }{%
    % label defined; print text 
    #2 %
  }%
}
\makeatother

\makeatletter %For printing refs only if the labels exist
\newcommand\TextIfExistsElse[3]{%
  \@ifundefined{r@#1}{%
    #3 % label undefined: print other text
  }{%
    % label defined; print text 
    #2 %
  }%
}
\makeatother