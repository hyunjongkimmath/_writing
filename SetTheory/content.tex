%% Delete this \nocite command invocation to make the references section only list out the bibitems that are actually cited.
\nocite{*}

\section{ZFC Axioms}

% Definition: Language and structure of set theory
\TODO{language of set theory}
\begin{notation}
Let the language of set theory consist of the binary relation symbol $\in$, intended to denote *membership*. The objects of discourse, called \hlin{sets}, are elements of a nonempty universe $V$, and all variables range over the elements of $V$.
\end{notation}

% Definition: Axioms of Zermelo–Fraenkel set theory (ZF)
\begin{definition} \label{definition:zermelo_fraenkel_set_theory}
    \TODO{first order language}
The \hldef{axioms of Zermelo–Fraenkel set theory (ZF)} are the following statements, formulated in the first-order language with symbol $\in$:
\begin{enumerate}

  \item \textbf{Axiom of Extensionality:} Two sets are equal if and only if they have the same elements.
  $$\forall A\, \forall B\, (\forall x\, (x \in A \leftrightarrow x \in B) \rightarrow A = B).$$

  \item \textbf{Axiom of Pairing:} For any sets $x, y$, there exists a set containing exactly $x$ and $y$ as elements.
  $$\forall x\, \forall y\, \exists A\, \forall z\, [z \in A \leftrightarrow (z = x \vee z = y)].$$

  \item \textbf{Axiom of Union:} For any set $A$, there exists a set that is the union of the elements of $A$.
  $$\forall A\, \exists U\, \forall x\, [x \in U \leftrightarrow \exists B\, (x \in B \wedge B \in A)].$$

  \item \textbf{Axiom of Power Set:} For any set $A$, there exists a set $\mathcal{P}(A)$ containing all subsets of $A$.
  $$\forall A\, \exists P\, \forall B\, [B \in P \leftrightarrow B \subseteq A].$$

  \item \textbf{Axiom of Infinity:} There exists an inductive set, that is, a set containing the empty set and closed under the successor operation.
  $$\exists I\, [\varnothing \in I \wedge \forall x\, (x \in I \rightarrow x \cup \{x\} \in I)].$$

  \item \textbf{Axiom Schema of Separation:} For any property $\varphi(x)$ expressible in the language of set theory and any set $A$, there is a subset of $A$ consisting of the elements of $A$ satisfying $\varphi(x)$.
  $$\forall A\, \exists B\, \forall x\, [x \in B \leftrightarrow (x \in A \wedge \varphi(x))].$$

  \item \textbf{Axiom Schema of Replacement:} For any definable function given by a formula $\varphi(x,y)$ ensuring that for each $x$ there exists a unique $y$ with $\varphi(x,y)$, the image of any set under this function is also a set.
  $$\forall A\, [\forall x \in A\, \exists! y\, \varphi(x,y)] \rightarrow \exists B\, \forall y\, (y \in B \leftrightarrow \exists x \in A\, \varphi(x,y)).$$

  \item \textbf{Axiom of Foundation (Regularity):} Every nonempty set $A$ contains an element $x$ that is disjoint from $A$.
  $$\forall A\, [A \neq \varnothing \rightarrow \exists x\, (x \in A \wedge x \cap A = \varnothing)].$$

\end{enumerate}
The \hldef{Axiom of Choice (AC)} asserts that for every set $X$ of nonempty sets, there exists a function $f : X \to \bigcup X$ such that $f(A) \in A$ for all $A \in X$. Formally:
$$\hlin{\forall X\, [(\forall A \in X\, A \neq \varnothing) \rightarrow \exists f\, (\operatorname{dom}(f) = X \wedge \forall A \in X\, f(A) \in A)]}.$$
When ZF is augmented by the Axiom of Choice, the resulting system is called \hldef{Zermelo–Fraenkel set theory with Choice (ZFC)}.
\end{definition}

\begin{definition}[Small Set] \label{definition:small_set}
    A \hldef{small set} refers to a set; the adjective of \emph{small} is used to emphasize that the set is a set, rather than a proper class.

    Alternatively, a \hldef{small set} is any object that belongs to a fixed universe of sets $\mathcal{U}$, called the \hldef{universe of small sets}.  
    In other words, a set $A$ is said to be \hldef{small} if and only if $A \in \mathcal{U}$.  
    Elements of $\mathcal{U}$ may themselves be sets, but $\mathcal{U}$ is not assumed to be a set; it may be a proper class.
\end{definition}

\begin{definition}[Subset] \label{definition:subset_of_a_set}
Let $A$ and $B$ be sets.
The set $A$ is said to be a \hldef{subset of $B$}, written as \hl{$A \subseteq B$} or \hl{$A \subset B$}, if every element of $A$ is also an element of $B$, that is,
$$ A \subseteq B \iff \forall x \,(x \in A \Rightarrow x \in B).  $$
If $A \subseteq B$ and $A \ne B$, then $A$ is called a \hldef{proper subset of $B$}; this is commonly denoted by \hl{$A \subsetneq B$}.
\end{definition}

\begin{lemma} \label{lemma:sets_are_equal_if_and_only_if_they_are_mutual_subsets}
    Let $A$ and $B$ be sets. They are equal if and only if \CrefAndHyperrefIfExist{definition:subset_of_a_set}{$A \subseteq B$} and $B \subseteq A$. 
\end{lemma}
\begin{proof}
    This is by the \CrefAndHyperrefIfExist{definition:zermelo_fraenkel_set_theory}{axiom of extensionality}.
\end{proof}

\begin{definition}[Power Set] \label{definition:power_set_of_a_set}
Let $A$ be a set. The \hldef{power set of $A$}, denoted by \hl{$\mathcal{P}(A)$}, is the set of all subsets of $A$:
$$ \mathcal{P}(A) := \{\, B \mid B \subseteq A \,\}.  $$
Equivalently, every element of $\mathcal{P}(A)$ is itself a set $B$ satisfying $B \subseteq A$. Under the \CrefAndHyperrefIfExist{definition:zermelo_fraenkel_set_theory}{axiom of power set}, note that the $\calP(A)$ exists.
\end{definition}

\begin{definition}[Set Complement] \label{definition:complement_of_a_set_by_a_subset}
Let $U$ be a fixed set and let $A \subseteq U$ be a \CrefAndHyperrefIfExist{definition:subset_of_a_set}{subset} of $U$.  
The \hldef{complement of $A$ (with respect to $U$)}, denoted by \hl{$A^{c}$} or \hl{$U \setminus A$}, is defined as
$$ A^{c} := \{ x \in U \mid x \notin A \}.  $$
Equivalently, for any $x \in U$, one has
$$ x \in A^{c} \iff x \notin A.  $$
When the set $U$ is clear from context, the notation $A^{c}$ may be used without explicit reference to $U$.
\end{definition}


\begin{definition}[Union of Sets] \label{definition:union_of_sets}
Let $I$ be a (possibly \CrefAndHyperrefIfExist{definition:countable_finite_uncountable_sets}{infinite} but \CrefAndHyperrefIfExist{definition:small_set}{small}) index set and let $\{A_i\}_{i \in I}$ be a family of sets indexed by $I$.  
The \hldef{union of the family $\{A_i\}_{i \in I}$}, denoted by \hl{$\bigcup_{i \in I} A_i$}, is defined as
$$ \bigcup_{i \in I} A_i := \{ x \mid \exists i \in I \text{ such that } x \in A_i \}.  $$
Under the \CrefAndHyperrefIfExist{definition:zermelo_fraenkel_set_theory}{axiom of union}, note that $\bigcup_{i \in I} A_i$ exists.
For finitely many sets $A_1, A_2, \dots, A_n$, their union is denoted by \hl{$A_1 \cup A_2 \cup \cdots \cup A_n$}.
\end{definition}

\begin{definition}[Intersection of Sets] \label{definition:intersection_of_sets}
Let $I$ be a (possibly \CrefAndHyperrefIfExist{definition:countable_finite_uncountable_sets}{infinite} but \CrefAndHyperrefIfExist{definition:small_set}{small}) index set and let $\{A_i\}_{i \in I}$ be a family of sets indexed by $I$.  
The \hldef{intersection of the family $\{A_i\}_{i \in I}$}, denoted by \hl{$\bigcap_{i \in I} A_i$}, is defined as
$$ \bigcap_{i \in I} A_i := \{ x \mid \forall i \in I,\, x \in A_i \}.  $$
For a finite family of sets \(A_1, A_2, \ldots, A_n\), their intersection is denoted by \hl{$A_1 \cap \cdots \cap A_n$}
\end{definition}


\section{Maps of sets}

% Definition: Function and injectivity/surjectivity preliminaries
\begin{definition} \label{definition:function_of_sets}
Let $X$ and $Y$ be sets. A \hldef{map} (or \hldef{function}) from $X$ to $Y$ is a rule $f$ assigning to each element $x \in X$ exactly one element $f(x) \in Y$. We write \hl{$f : X \to Y$}.

We say that $X$ is the \hldef{domain} and that $Y$ is the \hldef{codomain of $f$}.

% Given any sets $X$ and $Y$, the collection of maps $X \to Y$ is a set; the collection of all sets along with maps between them form a \CrefAndHyperrefIfExist{definition:locally_small_category}{locally small} \CrefAndHyperrefIfExist{definition:category}{category}, usually called the \hldef{category of sets}, and often denoted by notations such as \hl{$\mathrm{Set}$}, \hl{$\mathbf{Set}$}, \hl{$\mathrm{Sets}$}, \hl{$\mathbf{Sets}$}, \hl{$(\mathrm{Set})$}, \hl{$(\mathbf{Set})$}, \hl{$(\mathrm{Sets})$}, \hl{$(\mathbf{Sets})$}.

\end{definition}
\begin{definition} \label{definition:category_of_sets}
    The category of sets is the \CrefAndHyperrefIfExist{definition:locally_small_category}{(locally small)} \CrefAndHyperrefIfExist{definition:category}{category} 
    \begin{itemize}
        \item whose objects are \CrefAndHyperrefIfExist{definition:zermelo_fraenkel_set_theory}{sets}, and 
        \item whose morphisms $X \to Y$ are \CrefAndHyperrefIfExist{definition:function_of_sets}{set functions} $X \to Y$. 
    \end{itemize}

    The category of sets is often denoted by notations such as \hl{$\mathrm{Set}$}, \hl{$\mathbf{Set}$}, \hl{$\mathrm{Sets}$}, \hl{$\mathbf{Sets}$}, \hl{$(\mathrm{Set})$}, \hl{$(\mathbf{Set})$}, \hl{$(\mathrm{Sets})$}, \hl{$(\mathbf{Sets})$}.
\end{definition}

\begin{definition} \label{definition:identity_function_on_a_set}
Let $X$ be a set. The \hldef{identity function on $X$}, denoted by \hl{$\mathrm{id}_X$}, is the \CrefAndHyperrefIfExist{definition:function_of_sets}{function} $\mathrm{id}_X : X \to X$ defined by
$$
\mathrm{id}_X(x) = x \quad \text{for all } x \in X.
$$
It is the unique function on $X$ satisfying $f \circ \mathrm{id}_X = f = \mathrm{id}_X \circ f$ for every function $f : X \to Y$ and every function $f : Y \to X$.
\TextIfExists{definition:category_of_sets}{
The identity function is the \CrefAndHyperrefIfExist{definition:category}{identity map} on the objects $X$ in the \CrefAndHyperrefIfExist{definition:category_of_sets}{category of sets}.
}
\end{definition}

\begin{definition}[Preimage of a subset under a map of sets] \label{definition:preimage_of_a_subset_of_the_codomain_of_a_set_map}
Let $X$ and $Y$ be \CrefAndHyperrefIfExist{definition:zermelo_fraenkel_set_theory}{sets} and let $f : X \to Y$ be a \CrefAndHyperrefIfExist{definition:function_of_sets}{function}.
Let $B \subseteq Y$ be a \CrefAndHyperrefIfExist{definition:subset_of_a_set}{subset} of the codomain $Y$.
The \hldef{preimage of $B$ under $f$} (also called the \hldef{inverse image of $B$ under $f$}) is the subset of $X$ defined by
$$\hlin{f^{-1}(B) := \{ x \in X : f(x) \in B \} \subseteq X.} $$
If $B$ is a singleton set $\{y\}$, then we often denote $f^{-1}(B)$ by \hl{$f^{-1}(y)$}.
\end{definition}




\begin{definition} \label{definition:injective_surjective_bijective_map_of_sets}
Let $X$ and $Y$ be sets and let $f: X \to Y$ be a function.
\begin{itemize}
  \item The function $f$ is said to be \hldef{injective} (or \hldef{one-to-one}) if for all $x_1, x_2 \in X$, $f(x_1) = f(x_2)$ implies $x_1 = x_2$.
  \item The function $f$ is said to be \hldef{surjective} (or \hldef{onto}) if for every $y \in Y$ there exists $x \in X$ such that $f(x) = y$.

  \item The map $f$ is \hldef{bijective} if it is both injective and surjective. In this case, there exists a unique \hldef{inverse map} \hl{$f^{-1} : Y \to X$} such that for all $x \in X$ and $y \in Y$,
    $$f^{-1}(f(x)) = x \text{ and } f(f^{-1}(y)) = y.$$
\end{itemize}
\end{definition}


\begin{definition}[Product of Sets] \label{definition:product_of_sets}
Let $I$ be a (possibly \CrefAndHyperrefIfExist{definition:countable_finite_uncountable_sets}{infinite} but \CrefAndHyperrefIfExist{definition:small_set}{small}) index set and let $\{A_i\}_{i \in I}$ be a family of sets indexed by $I$.  
The \hldef{Cartesian product of the family $\{A_i\}_{i \in I}$}, denoted by \hl{$\prod_{i \in I} A_i$}, is defined as the set of all tuples/\CrefAndHyperrefIfExist{definition:function_of_sets}{functions}
$$ \prod_{i \in I} A_i := \{ (a_i)_{i \in I} \mid a_i \in A_i \text{ for all } i \in I \}, $$
where $(a_i)_{i \in I}$ denotes a function from $I$ to $\bigcup_{i \in I} A_i$ such that $(a_i)_{i \in I}(i) = a_i \in A_i$ for each $i \in I$.  

\TextIfExists{definition:category_of_sets}{
The Cartesian product $\prod_{i \in I} A_i$ is the \CrefAndHyperrefIfExist{definition:product_and_coproduct_of_objects_in_a_category}{product} of the objects $A_i$ in the \CrefAndHyperrefIfExist{definition:category_of_sets}{category of sets}.
}

The self product of a set $A$ indexed by $I$ is often denoted by \hl{$A^I$}. Note that elements of $A^I$ can be identified with \CrefAndHyperrefIfExist{definition:function_of_sets}{functions} $I \to A$. The finite self product of $A$ taken $n$ times is often denoted by \hl{$A^n$}.
For finitely many sets $A_1,\ldots,A_n$, their Cartesian product is denoted by \hl{$A_1 \times \cdots \times A_n$}. Elements of such a finite product may be written as $(a_1,\ldots,a_n)$.

\end{definition}


\section{Relations}

\begin{definition} \label{definition:n_ary_relation_on_sets}
    Let $n$ be a positive integer. Let $X_1, X_2, \dots, X_n$ be \CrefAndHyperrefIfExist{definition:set}{sets}. An \hldef{$n$-ary relation} on these sets is a subset $R \subseteq X_1 \times X_2 \times \dots \times X_n$. The integer $n$ is called the \hldef{arity} (or \hldef{degree}) of the relation. The sets $X_1, \dots, X_n$ are called the \hldef{domains of the relation}.

    If $(x_1, x_2, \dots, x_n) \in R$, we say that the elements $x_1, \dots, x_n$ are \hldef{related} by $R$.

    In the special case where $X_1 = X_2 = \dots = X_n = X$, we say that $R$ is an \hldef{$n$-ary relation on the set $X$}. In this case, $R \subseteq X^n$.

    Specific arities have standard names:
    \begin{itemize}
        \item A \hldef{unary relation} on $X$ is a subset of $X$ (arity $n=1$).
        \item A \hldef{binary relation} from $X$ to $Y$ is a subset of $X \times Y$ (arity $n=2$).
        \item A \hldef{ternary relation} is a subset of $X \times Y \times Z$ (arity $n=3$).
    \end{itemize}
\end{definition}

\begin{definition} \label{definition:related_to_for_a_binary_relation}
    Let $X$ and $Y$ be \CrefAndHyperrefIfExist{definition:set}{sets}. Let $R \subseteq X \times Y$ be a \CrefAndHyperrefIfExist{definition:n_ary_relation_on_sets}{binary relation} from $X$ to $Y$.
    If the pair $(x, y)$ is an element of $R$, we say that $x$ is \hldef{related to $y$ by $R$}. This is denoted by the notation:
    $$ \hlin{ x R y \quad \text{or} \quad x \sim_R y. } $$
\end{definition}

\begin{definition} \label{definition:domain_and_range_of_a_binary_relation}
    Let $R \subseteq X \times Y$ be a \CrefAndHyperrefIfExist{definition:binary_relation_special_case}{binary relation}.
    The \hldef{domain of $R$} is the set of all first coordinates in the relation:
    $$ \hlin{ \operatorname{dom}(R) = \{ x \in X \mid \exists y \in Y, x R y \}. } $$
    The \hldef{range} (or \hldef{image}) of $R$ is the set of all second coordinates in the relation:
    $$ \hlin{ \operatorname{ran}(R) = \{ y \in Y \mid \exists x \in X, x R y \}. } $$
\end{definition}

\begin{definition} \label{definition:inverse_relation}
    Let $R \subseteq X \times Y$ be a \CrefAndHyperrefIfExist{definition:n_ary_relation_on_sets}{binary relation}.
    The \hldef{inverse relation} (or \hldef{converse relation}) of $R$, denoted by \hl{$R^{-1}$} or \hl{$R^{\operatorname{op}}$}, is the binary relation from $Y$ to $X$ defined by reversing the pairs in $R$:
    $$ \hlin{ R^{-1} = \{ (y, x) \in Y \times X \mid (x, y) \in R \}. } $$
    Equivalently, $y R^{-1} x$ if and only if $x R y$.
\end{definition}

\begin{definition} \label{definition:composition_of_binary_relations_on_sets}
    Let $R \subseteq X \times Y$ and $S \subseteq Y \times Z$ be \CrefAndHyperrefIfExist{definition:n_ary_relation_on_sets}{binary relations}.
    The \hldef{composition} of $S$ with $R$, denoted by \hl{$S \circ R$}, is the binary relation from $X$ to $Z$ defined by:
    $$ \hlin{ S \circ R = \{ (x, z) \in X \times Z \mid \exists y \in Y \text{ such that } x R y \text{ and } y S z \}. } $$
    In infix notation, $x (S \circ R) z$ if and only if there exists an intermediate element $y \in Y$ such that $x R y$ and $y S z$.
\end{definition}



\begin{definition} \label{definition:reflexive_symmetric_antisymmetric_transitive_total_binary_relations_on_a_set}
Let $R$ be a \CrefAndHyperrefIfExist{definition:n_ary_relation_on_sets}{binary relation} on a set $X$.
\begin{enumerate}
    \item $R$ is \hldef{reflexive} if for all $x \in X$, $x R x$.
    \item $R$ is \hldef{symmetric} if for all $x, y \in X$, $x R y$ implies $y R x$.
    \item $R$ is \hldef{antisymmetric} if for all $x, y \in X$, ($x R y$ and $y R x$) implies $x = y$.
    \item $R$ is \hldef{transitive} if for all $x, y, z \in X$, ($x R y$ and $y R z$) implies $x R z$.
    \item $R$ is \hldef{total} (or \hldef{connected}) if for all distinct $x, y \in X$, either $x R y$ or $y R x$.
\end{enumerate}
\end{definition}

\begin{definition} \label{definition:equivalence_relation_on_a_set}
An \hldef{equivalence relation on a set $X$} is a \CrefAndHyperrefIfExist{definition:n_ary_relation_on_sets}{binary relation} on $X$ that is \CrefAndHyperrefIfExist{definition:reflexive_symmetric_antisymmetric_transitive_total_binary_relations_on_a_set}{reflexive, symmetric, and transitive}. 

\end{definition}

\begin{definition} \label{definition:equivalence_class_of_an_element_of_a_set_under_an_equivalence_relation_and_quotient_of_set_by_equivalence_relation}
If $\sim$ is an \CrefAndHyperrefIfExist{definition:equivalence_relation}{equivalence relation} on a set $X$ and $x \in X$, the \hldef{equivalence class of $x$}, denoted by \hl{$[x]$} or \hl{$[x]_{\sim}$}, is the set defined by
$$ \hlin{ [x] = \{ y \in X \mid x \sim y \}. } $$
The set of all equivalence classes is called the \hldef{quotient set of $X$ by $\sim$}, denoted by \hl{$X / {\sim}$}.
\end{definition}

\TODO{}
\begin{definition} 
Let $R$ be a \CrefAndHyperrefIfExist{definition:relation_on_sets}{relation on a set} $X$.
\begin{enumerate}
    \item $R$ is a \hldef{partial order} if it is reflexive, antisymmetric, and transitive. A set $X$ equipped with a partial order is called a \hldef{partially ordered set} or \hldef{poset}.
    \item $R$ is a \hldef{strict partial order} if it is irreflexive (for all $x \in X$, not $x R x$) and transitive.
    \item $R$ is a \hldef{total order} (or \hldef{linear order}) if it is a partial order that is total.
    \item $R$ is a \hldef{well-order} if it is a total order such that every non-empty subset of $X$ has a least element with respect to $R$.
\end{enumerate}
\end{definition}



\section{Ordinals and Cardinals}


\begin{notation} \label{notation:natural_numbers}
Let \hl{$\mathbb{N}$} denote the set of natural numbers, typically taken as $\mathbb{N} = \{1, 2, 3, \dots\}$. Elements of $\mathbb{N}$ will serve as indices for sequences whenever cardinality is compared to that of countably infinite sets.
\end{notation}

\begin{definition} \label{definition:countable_finite_uncountable_sets}
Let $A$ be a set.
\begin{itemize}
  \item The set $A$ is said to be \hldef{countably infinite} (or simply \hldef{countable}) if there exists a bijection $f : \mathbb{N} \to A$.
  \item The set $A$ is said to be \hldef{finite} if there exists some $n \in \mathbb{N}$ and a bijection $g : \{1,2,\dots,n\} \to A$.
  \item The set $A$ is said to be \hldef{at most countable} if it is either finite or countably infinite.
  \item The set $A$ is said to be \hldef{uncountable} if it is not at most countable.
\end{itemize}
\end{definition}

\begin{definition} \label{definition:ordinal_number}
A set $\alpha$ is an \hldef{ordinal number} (or simply an \hldef{ordinal}) if it satisfies the following two conditions:
\begin{enumerate}
    \item $\alpha$ is \hldef{transitive}, meaning that every element of $\alpha$ is also a subset of $\alpha$. That is, if $x \in \alpha$, then $x \subseteq \alpha$.
    \item $\alpha$ is \hldef{strictly well-ordered} by the membership relation $\in$. That is, the relation $<$ on $\alpha$ defined by $x < y \iff x \in y$ is a strict total ordering, and every non-empty subset of $\alpha$ has a least element under this ordering.
\end{enumerate}
The class of all ordinal numbers is denoted by \hl{$\operatorname{Ord}$} or \hl{$\mathbf{ON}$}. For ordinals $\alpha$ and $\beta$, we write \hl{$\alpha < \beta$} when $\alpha \in \beta$, and we write and \hl{$\alpha \le \beta$}  when $\alpha \in \beta$ or $\alpha = \beta$.
\end{definition}

\begin{definition} \label{definition:successor_ordinal_of_an_ordinal}
Let $\alpha$ be an \CrefAndHyperrefIfExist{definition:ordinal_number}{ordinal number}. The ordinal number $\alpha$ is called a \hldef{successor ordinal} if there exists an ordinal $\beta$ such that $\alpha = \beta \cup \{\beta\}$. In this case, $\alpha$ is the \hldef{immediate successor of $\beta$} and is denoted by \hl{$\beta + 1$}.
\end{definition}

\begin{definition} \label{definition:limit_ordinal}
An \CrefAndHyperrefIfExist{definition:ordinal_number}{ordinal number} $\alpha$ is a \hldef{limit ordinal} if $\alpha \ne \emptyset$ and $\alpha$ is not a \CrefAndHyperrefIfExist{definition:successor_ordinal_of_an_ordinal}{successor ordinal}. Equivalently, $\alpha$ is a limit ordinal if for every $\beta \in \alpha$, the successor $\beta + 1$ is also in $\alpha$. In this case, $\alpha$ is equal to the union of all smaller ordinals:
$$ \alpha = \bigcup_{\beta < \alpha} \beta. $$
\end{definition}


\begin{definition} \label{definition:finite_ordinal}
An \CrefAndHyperrefIfExist{definition:ordinal_number}{ordinal number} $\alpha$ is a \hldef{finite ordinal} (or a \hldef{natural number}) if $\alpha = \emptyset$ or if $\alpha$ is a \CrefAndHyperrefIfExist{definition:successor_ordinal}{successor ordinal} and every element $\beta \in \alpha$ is either $\emptyset$ or a successor ordinal. Ordinals that are not finite are called \hldef{infinite ordinals} or \hldef{transfinite ordinals}.
\end{definition}

\begin{definition} \label{definition:omega_ordinal}
    The set of all \CrefAndHyperrefIfExist{definition:finite_ordinal}{finite ordinals} is denoted by \hl{$\omega$} or \hl{$\mathbb{N}$}. It is the smallest \CrefAndHyperrefIfExist{definition:limit_ordinal}{limit ordinal}.
\end{definition}


\begin{definition} \label{definition:cardinal_number_and_cardinality_of_a_set}
An \CrefAndHyperrefIfExist{definition:ordinal_number}{ordinal number} $\kappa$ is a \hldef{cardinal number} (or simply a \hldef{cardinal}) if for every ordinal $\alpha < \kappa$, there is no \CrefAndHyperrefIfExist{definition:injective_surjective_bijective_map_of_sets}{bijection} between $\alpha$ and $\kappa$. Equivalently, a cardinal is an initial ordinal—an ordinal that is not equinumerous with any smaller ordinal.

The \hldef{cardinality of an arbitrary set $X$}, denoted by \hl{$|X|$}, \hl{$\operatorname{card}(X)$}, or \hl{$\# X$}, is the unique cardinal number $\kappa$ such that there exists a bijection between $X$ and $\kappa$. (The existence of such a $\kappa$ for every set requires the \CrefAndHyperrefIfExist{definition:zermelo_fraenkel_set_theory}{Axiom of Choice}).
\end{definition}

\begin{definition} \label{definition:successor_cardinal_of_a_cardinal_and_limit_cardinal}
Let $\kappa$ be a \CrefAndHyperrefIfExist{definition:cardinal_number_and_cardinality_of_a_set}{cardinal number}.
\begin{itemize}
    \item $\kappa$ is a \hldef{successor cardinal} if there exists a cardinal $\lambda$ such that $\kappa$ is the smallest cardinal strictly greater than $\lambda$. This successor is denoted by \hl{$\lambda^+$}.
    \item $\kappa$ is a \hldef{limit cardinal} if $\kappa \ne 0$ and $\kappa$ is not a successor cardinal.
\end{itemize}
\end{definition}

\begin{definition} \label{definition:cofinal_subset_and_cofinality_of_a_limit_ordinal}
Let $\alpha$ be a \CrefAndHyperrefIfExist{definition:limit_ordinal}{limit ordinal}. A subset $A \subseteq \alpha$ is \hldef{cofinal in $\alpha$} if for every $\beta \in \alpha$, there exists $\gamma \in A$ such that $\beta \le \gamma$.
The \hldef{cofinality of $\alpha$}, denoted by \hl{$\operatorname{cf}(\alpha)$}, is the least ordinal which is the order type of a cofinal subset of $\alpha$. Equivalently, $\operatorname{cf}(\alpha)$ is the smallest cardinality of a cofinal subset of $\alpha$.
\end{definition}



\begin{definition} \label{definition:regular_and_singular_cardinals}
Let $\kappa$ be a \CrefAndHyperrefIfExist{definition:cardinal_number_and_cardinality_of_a_set}{cardinal number}.
\begin{itemize}
    \item $\kappa$ is a \hldef{singular cardinal} if it can be written as a union of fewer than $\kappa$ sets, each of cardinality less than $\kappa$. Formally, $\kappa$ is singular if its \hldef{cofinality} is strictly less than $\kappa$, i.e., $\operatorname{cf}(\kappa) < \kappa$.
    \item $\kappa$ is a \hldef{regular cardinal} if it is not singular. Formally, $\kappa$ is regular if $\operatorname{cf}(\kappa) = \kappa$.
\end{itemize}
\end{definition}

\begin{definition} \label{definition:aleph_numbers}
The \hldef{aleph numbers} are the infinite \CrefAndHyperrefIfExist{definition:cardinal_number_and_cardinality_of_a_set}{cardinals} indexed by \CrefAndHyperrefIfExist{definition:ordinal_number}{ordinals}. They are defined recursively:
\begin{align*}
    \aleph_0 &= \omega, \\
    \aleph_{\alpha+1} &= \aleph_\alpha^+ \quad (\text{the successor cardinal of } \aleph_\alpha), \\
    \aleph_\lambda &= \bigcup_{\beta < \lambda} \aleph_\beta \quad \text{if } \lambda \text{ is a limit ordinal}.
\end{align*}
\CrefIfExists{definition:omega_ordinal}\CrefIfExists{definition:successor_cardinal_of_a_cardinal_and_limit_cardinal}\CrefIfExists{definition:limit_ordinal}
A cardinal $\kappa$ is an \hldef{uncountable cardinal} if $\kappa > \aleph_0$.
\end{definition}


\section{Category Theorey}
\begin{definition}[Category] \label{definition:category}
    A 
    \defin{category}{category}{
        name={Category},
        description={A nice enough collection of objects and morphisms (\Cref{definition:category})},
    }
    \hldef{category} $\mathcal{C}$ consists of the following data:
    \begin{itemize}
        \item A class of \defin{objects}{object_of_a_category}{
            name={Object of a category},
            description={\Cref{definition:category}},
        }
        denoted \notat{\operatorname{Ob}(\mathcal{C})}{class_of_objects_of_a_category}{
            name={$\operatorname{Ob}(\mathcal{C})$},
            description={Class of objects of a category $\calC$ \Cref{definition:category}},
            sort={Ob},
        }.
        % \hl{$\operatorname{Ob}(\mathcal{C})$}.
        \item For each pair of objects $X, Y \in \operatorname{Ob}(\mathcal{C})$, a class
        \notatin{\operatorname{Hom}_{\mathcal{C}}(X,Y)}{class_of_morphisms_between_two_objects_of_a_category}
        {
            name={$\operatorname{Hom}_{\mathcal{C}}(X,Y)$},
            description={Class of morphisms between objects $X$ and $Y$ of the category $\calC$ (\Cref{definition:category})},
            sort={Hom},
        }
        % $$\hlin{\operatorname{Hom}_{\mathcal{C}}(X,Y)}$$
        of \defin{morphisms}{morphism_between_objects_of_a_category}{
            name={Morphism between objects of a category},
            description={(\Cref{definition:category})},
        }
        (also called 
        \defin{arrows}{arrow_between_objects_of_a_category}{
            name={Arrow between objects of a category},
            description={Synonym for morphism (\Cref{definition:category})},
        }
        or
        \defin{homs}{hom_between_objects_of_a_category}{
            name={Hom between objects of a category},
            description={Synonym for morphism (\Cref{definition:category})},
        }). If the category $\calC$ is clear, then this \hldef{hom-class} is also denoted by \hl{$\operatorname{Hom}(X,Y)$}. It may also be denoted by \hl{$\operatorname{hom}_{\mathcal{C}}(X,Y)$} or \hl{$\operatorname{hom}(X,Y)$}, especially to distinguish from other types of hom's (e.g. \hyperrefIfExists{definition:internal_hom_object_in_a_category}{internal hom's})
        \item For each triple of objects $X,Y,Z$, a composition law
        $$ \circ : \operatorname{Hom}_{\mathcal{C}}(Y,Z) \times \operatorname{Hom}_{\mathcal{C}}(X,Y) \to \operatorname{Hom}_{\mathcal{C}}(X,Z), $$
        denoted \hl{$(g,f) \mapsto g \circ f$}.
        \item For each object $X$, an \hldef{identity morphism}
        $$\hlin{\operatorname{id}_X \in \operatorname{Hom}_{\mathcal{C}}(X,X).}$$
    \end{itemize}
    These data satisfy the following axioms:
    \begin{itemize}
        \item (Associativity) For all morphisms $f \in \operatorname{Hom}_{\mathcal{C}}(X,Y)$, $g \in \operatorname{Hom}_{\mathcal{C}}(Y,Z)$, and $h \in \operatorname{Hom}_{\mathcal{C}}(Z,W)$, 
        $$
        h \circ (g \circ f) = (h \circ g) \circ f.
        $$
        \item (Identity) For all $f \in \operatorname{Hom}_{\mathcal{C}}(X,Y)$,
        $$
        \operatorname{id}_Y \circ f = f = f \circ \operatorname{id}_X.
        $$
    \end{itemize}
    One often writes \hl{$X \in \calC$} synonymously with $X \in \Ob(\calC)$, i.e. to denote that $X$ is an object of of $\calC$. 

    We may call a category as above an \hldef{ordinary category} to distinguish this notion from the notions of \hyperrefIfExists{definition:category_enriched_in_a_monoidal_category}{\emph{categories enriched in monoidal categories}} or higher/$n$-categories.
    \TODO{TODO: define $n$-categories}

    A category as defined above may be called called a \hldef{large category} or a \hldef{class category} to emphasize that the hom-classes may be proper classes rather than sets (note, however, that the possibility that hom-classes are sets is not excluded for large categories). Accordingly, a \hldef{category} may often refer to a \hyperrefIfExists{definition:locally_small_category}{locally small category}\CrefIfExists{definition:locally_small_category}, which is a category whose hom-classes are all sets.
\end{definition}

% Later on, we refer to the \gls{category} again.

\begin{definition}[Locally small category] \label{definition:locally_small_category}
A \hyperrefIfExists{definition:category}{(large) category}\CrefIfExists{definition:category} $\mathcal{C}$ is called a \hldef{locally small category} if for every pair of objects $X, Y \in \operatorname{Ob}(\mathcal{C})$, the collection $\operatorname{Hom}_{\mathcal{C}}(X,Y)$ of morphisms between them is a (\CrefAndHyperrefIfExist{definition:small_set}{small}) \emph{set} (as opposed to a proper class). In other words, each hom-class is a set and may even be called a \hldef{hom-set}.

In some contexts, a locally small category may simply be called a \hldef{category}, especially when genuinely large categories are not considered.

A category $\mathcal{C}$ is called a \hldef{small category} if it is a locally small category and the class $\operatorname{Ob}(\mathcal{C})$ of objects is a set.

\TextIfExists{definition:grothendieck_universe}{
Given a \hyperrefIfExists{definition:grothendieck_universe}{universe}\CrefIfExists{definition:grothendieck_universe} $U$, we can define the notion of a \hldef{$U$-locally small category} and of a \hldef{$U$-small category} similarly. More explicitly, 
\begin{enumerate}
    \item a $U$-locally small category is a category such that for every pair of objects $X, Y \in \operatorname{Ob}(\mathcal{C})$, the collection $\operatorname{Hom}_{\mathcal{C}}(X,Y)$ of morphisms between them is a $U$-set.
    \item a $U$-small category is a category such that $\operatorname{Ob}(\mathcal{C})$ is a $U$-set and for every pair of objects $X, Y \in \operatorname{Ob}(\mathcal{C})$, the collection $\operatorname{Hom}_{\mathcal{C}}(X,Y)$ of morphisms between them is a $U$-set; in particular the collection of all objects and morhpisms in a $U$-small category is a $U$-set.
\end{enumerate}
}
\end{definition}

\begin{remark}
    Many ``concrete'' categories considered in ``classical mathematics'' or outside of more ``abstract'' category theory tend to be locally small. For example, the categories of sets, groups, $R$-modules, vector spaces, topological spaces, schemes, manifolds, sheaves on ``small enough'' sites are all locally small.
\end{remark}
\begin{definition} \label{definition:functor_between_categories}
Let $\mathcal{C}$ and $\mathcal{D}$ be \CrefAndHyperrefIfExist{definition:category}{(large) categories}. 
\begin{enumerate}
  \item A \hldef{functor $F: \calC \to \calD$ (from $\mathcal{C}$ to $\mathcal{D}$)} consists of :
  \begin{itemize}
    \item For each object $X$ in $\mathcal{C}$, an object $F(X)$ in $\mathcal{D}$.
    \item For each morphism $f: X \to Y$ in $\mathcal{C}$, a morphism $F(f): F(X) \to F(Y)$ in $\mathcal{D}$,
  \end{itemize}
  such that:
  \begin{align*}
    F(\mathrm{id}_X) &= \mathrm{id}_{F(X)} \quad \text{for all objects } X \text{ in } \mathcal{C}, \\
    F(g \circ f) &= F(g) \circ F(f) \quad \text{for all } X,Y,Z \in \Ob(\calC) \text{ and all } f: X \to Y, g: Y \to Z \text{ in } \mathcal{C}.
  \end{align*}

  Functors as defined above are also referred to as \hldef{covariant functors} to distinguish them from contravariant functors

  \item A \hldef{contravariant functor from $\calC$ to $\calD$} refers to a covariant functor $F:\calC^{\op} \to \calD$. Equivalently, such a functor consists of 
  \begin{itemize}
    \item For each object $X$ in $\mathcal{C}$, an object $F(X)$ in $\mathcal{D}$.
    \item For each morphism $f: X \to Y$ in $\mathcal{C}$, a morphism $F(f): F(Y) \to F(X)$ in $\mathcal{D}$,
  \end{itemize}
  such that:
  \begin{align*}
    F(\mathrm{id}_X) &= \mathrm{id}_{F(X)} \quad \text{for all objects } X \text{ in } \mathcal{C}, \\
    F(g \circ f) &= F(f) \circ F(g) \quad \text{for all } X,Y,Z \in \Ob(\calC) \text{ and all } f: X \to Y, g: Y \to Z \text{ in } \mathcal{C}.
  \end{align*}
  \TextIfExists{definition:presheaf_on_a_category}{A synonym for a ``contravariant functor from $\calC$ to $\calD$'' is a ``\CrefAndHyperrefIfExist{definition:presheaf_on_a_category}{presheaf on $\calC$ with values in $\calD$}''.}
  
\end{enumerate}
Note that declarations such as ``Let $F: \calC^{\op} \to \calD$ be a contravariant functor'' can be common; such declarations usually mean ``Let $F$ be a contravariant functor from $\calC$ to $\calD$'' as opposed to ``Let $F$ be a contravariant functor from $\calC^{\op}$ to $\calD$''. further note that a contravariant functor from $\calC$ to $\calD$ is equivalent to a covariant functor from $\calC^{\op}$ to $\calD$.
\end{definition}



\begin{definition}[Diagram in a category and category of diagrams] \label{definition:diagram_in_a_category_indexed_by_a_small_category}
Let $\mathcal{C}$ be a \hyperrefIfExists{definition:category}{(large) category}\CrefIfExists{definition:category}, and let $I$ be a \CrefAndHyperrefIfExist{definition:category}{(large) category}. 
    \begin{enumerate}
        \item 
        A \hldef{diagram of shape $I$ in $\mathcal{C}$} is a \hyperrefIfExists{definition:functor_between_categories}{functor}\CrefIfExists{definition:functor_between_categories} $D: I \to \mathcal{C}$.
        We often denote such a diagram by the family \hl{$\{ D(i) \}_{i \in \mathrm{Ob}(I)}$} with transition maps given by the functorial image of morphisms in $I$. 
        
        A diagram is also synonymously called a \hldef{system}. Moreover, the category $I$ is called the \hldef{index category} or the \hldef{indexing category of the diagram $D$}.

        \item Given two diagrams $D,E: I \to \mathcal{C}$, a \hldef{morphism of diagrams} is a simply a \hyperrefIfExists{definition:natural_transformation_between_functors_between_categories}{natural transformation}\CrefIfExists{definition:natural_transformation_between_functors_between_categories} $D \Rightarrow E$ of the functors $D$ and $E$. 

        \item The \hldef{category of $I$-shaped diagrams in $\mathcal{C}$} or simply \hldef{diagram category (of $I$-shaped diagrams in $\calC$)}, often denoted \hl{$\mathcal{C}^I$}, \hl{$[I, \calC]$}, or \hl{$\operatorname{Fun}(I, \calC)$},
        is the (large) category whose objects are functors $I \to \mathcal{C}$ (that is, diagrams of shape $I$ in $\mathcal{C}$) and whose morphisms are \CrefAndHyperrefIfExist{definition:natural_transformation_between_functors_between_categories}{natural transformations} between such functors. The category $\calC^I$ is also called the \hldef{functor category of functors $I \to \calC$}. \TextIfExists{definition:presheaf_on_a_category}{Equivalently, the functor category $\calC^I$ is the category \CrefAndHyperrefIfExist{definition:presheaf_on_a_category}{$\PreShv(I^{\op}, \calC)$ of presheaves} on $I^{\op}$ with values in $\calC$ and hence notations for presheaf categories are applicable as notations for functor categories.}

        If $\calC$ is \hyperrefIfExists{definition:locally_small_category}{locally small}\CrefIfExists{definition:locally_small_category} and $I$ is small, then $\calC^I$ is locally small by Lemma \ref{lemma:category_of_presheaves_on_a_small_category_of_locally_small_value_is_locally_small}.
    \end{enumerate}
\end{definition}

\begin{lemma} \label{lemma:category_of_presheaves_on_a_small_category_of_locally_small_value_is_locally_small}
    Let $\calC$ be a \hyperrefIfExists{definition:locally_small_category}{small category}\CrefIfExists{definition:locally_small_category} (resp. $U$-small category where $U$ is some \hyperrefIfExists{definition:grothendieck_universe}{universe}\CrefIfExists{definition:grothendieck_universe}) and let $\calA$ be a \CrefAndHyperrefIfExist{definition:locally_small_category}{locally small} category (resp. $U$-locally small category). The \hyperrefIfExists{definition:presheaf_on_a_category}{presheaf category $\PreShv(\calC, \calA)$}\CrefIfExists{definition:presheaf_on_a_category} is locally small (resp. $U$-locally small).
\end{lemma}
\begin{proof}
    A morphism $\calF \to \calG$ in $\PreShv(\calC, \calA)$ is a \hyperrefIfExists{definition:natural_transformation_between_functors_between_categories}{natural transformation}\CrefIfExists{definition:natural_transformation_between_functors_between_categories} of the functors $\calF, \calG: \calC^{\op} \to \calA$. Such a natural transformation is encoded by a family $(\eta_C)_C$ of morphisms (satisfying certain conditions) $\eta_C: \calF(C) \to \calG(C)$ in $\calA$ over objects $C$ of $\calC^{\op}$. The product $\prod_{C \in \Ob \calC^{\op}} \Hom_{\calA}(\calF(C), \calG(C))$ is a product of ($U$-small) sets indexed by a ($U$-small) set, and the collection of natural transformations is a subset of this set. Therefore, $\Hom_{\PreShv(\calC, \calA)}(\calF, \calG)$ is a ($U$-small) set.  
\end{proof}

\begin{definition}[Cones, limits and colimits in a category] \label{definition:limit_and_colimit_of_a_diagram_in_a_category}
Let $\mathcal{C}$ be a \CrefAndHyperrefIfExist{definition:category}{(large) category}, let $I$ be a (large) category, and let $D: I \to \mathcal{C}$ be a \CrefAndHyperrefIfExist{definition:diagram_in_a_category_indexed_by_a_small_category}{diagram}\CrefIfExists{definition:diagram_in_a_category_indexed_by_a_small_category}.

\begin{enumerate}
    \item A \hldef{cone to the diagram $D$} is an object $L \in \mathcal{C}$ together with a family of morphisms
    \[
    \{\pi_i: L \to D(i)\}_{i \in I}
    \]
    such that for every morphism $f: i \to j$ in $I$, the diagram
    \begin{center}
    \begin{tikzcd}[row sep=large, column sep=large]
        & L \arrow[dl, "\pi_i"'] \arrow[dr, "\pi_j"] & \\
        D(i) \arrow[rr, "D(f)"] & & D(j)
    \end{tikzcd}
    \end{center}
    commutes, i.e.  $D(f) \circ \pi_i = \pi_j$.
    


    \item A cone $(L, \{\pi_i\})$ is called a \hldef{limit of $D$} if it satisfies the following ``universal property'':
    for any cone $(C, \{ f_i \})$ over $D$, there exists a \textit{unique} morphism $u: C \to L$ such that
    \[
    \pi_i \circ u = f_i \quad \text{for all } i \in I.
    \]
    Visually, the following diagrams commute every morphism $f: i \to j$ in $I$:
    \begin{center}
    \begin{tikzcd}[row sep=large, column sep=large]
        & C \arrow[d, "\exists ! u", dashed] \arrow[ddl, "f_i"', bend right=20] \arrow[ddr, "f_j", bend left=20] & \\
        & L \arrow[dl, "\pi_i"] \arrow[dr, "\pi_j"'] & \\
        D(i) \arrow[rr, "D(f)"] & & D(j).
    \end{tikzcd}
    \end{center}
    If such a cone exists, then the object $L$ is necessarily unique up to unique isomorphism by the universal property. In this case, $L$ is denoted by \hl{$\lim_{i \in I} D$} or \hl{$\lim D$}.



    
    \item A \hldef{cocone from the diagram $D$} is an object $C \in \mathcal{C}$ together with a family of morphisms
    \[
    \{\iota_i: D(i) \to C\}_{i \in I}
    \]
    such that for every morphism $f: i \to j$ in $I$, the diagram
    \begin{center}
    \begin{tikzcd}[row sep=large, column sep=large]
        D(i) \arrow[rr, "D(f)"] \arrow[dr, "\iota_i"'] & & D(j) \arrow[dl, "\iota_j"] \\
        & C & 
    \end{tikzcd}
    \end{center}
    commutes, i.e. $\iota_j \circ D(f) = \iota_i$.

    \item A cocone $(L, \{\iota_i\})$ is called a \hldef{colimit of $D$} if it satisfies the following ``universal property'':
    for any cocone $(C, \{ g_i \})$ under $D$, there exists a \textit{unique} morphism $u: L \to C$ such that
    \[
    u \circ \iota_i = g_i \quad \text{for all } i \in I.
    \]
    Visually, the following diagrams commute every morphism $f: i \to j$ in $I$:
    \begin{center}
    \begin{tikzcd}[row sep=large, column sep=large]
        D(i) \arrow[rr, "D(f)"] \arrow[dr, "\iota_i"] \arrow[ddr, "g_i"', bend right=20] & & D(j) \arrow[dl, "\iota_j"'] \arrow[ddl, "g_j", bend left=20] \\
        & L \arrow[d, "\exists ! u", dashed] & \\
        & C &. 
    \end{tikzcd}
    \end{center}
    If such a cocone exists, then the object $L$ is necessarily unique up to unique isomorphism by the universal property. In this case, $L$ is denoted by \hl{$\colim_{i \in I} D$} or \hl{$\colim D$}.

\end{enumerate}

A limit/colimit is called \hldef{finite} (resp. \hldef{small}) if the diagram category $I$ is finite (resp. small).

Some authors use the terms \hldef{projective limit} or \hldef{inverse limit} to refer to what is defined here as a limit, Similarly, the terms \hldef{inductive limit} or \hldef{direct limit} are sometimes used to mean a colimit. However, these phrases can have more specific meanings to other authors: a \emph{projective} or \emph{inverse limit} may refer to a limit over a diagram indexed by a \hyperrefIfExists{definition:partially_ordered_set}{codirected poset}\CrefIfExists{definition:partially_ordered_set}. Likewise, an \emph{inductive} or \emph{direct limit} may refer to a colimit over a \hyperrefIfExists{definition:partially_ordered_set}{directed poset}\CrefIfExists{definition:partially_ordered_set}\TextIfExists{definition:projective_and_inductive_limits_in_categories}{ (see \Cref{definition:projective_and_inductive_limits_in_categories})}.

Thus, while the terms are sometimes used interchangeably with ``limit'' and ``colimit,'' they may also emphasize particular indexing shapes and directions, distinguishing them from general limits and colimits taken over arbitrary small categories.
\end{definition}
\begin{definition}[Product in a category] \label{definition:product_and_coproduct_of_objects_in_a_category}
Let $\mathcal{C}$ be a category and let $\{X_i\}_{i \in I}$ be a family of objects in $\mathcal{C}$ indexed by a class $I$. 
\begin{enumerate}
    \item A \hldef{product of the family $\{X_i\}$} is an object $P$ of $\mathcal{C}$ together with a ``universal'' family of morphisms
    $$\pi_i : P \to X_i, \quad \text{for each } i \in I. $$
    More precisely, for any object $Y$ and any family of morphisms $\{f_i : Y \to X_i\}_{i \in I}$, there exists a unique morphism
    $$f : Y \to P$$
    making the following diagram commute for all $i \in I$, i.e. $\pi_i \circ f = f_i$:
    \begin{center}
    \begin{tikzcd}[row sep=large, column sep=large]
        Y \arrow[d, "\exists ! f", dashed] \arrow[dr, "f_i"] & \\
        \prod X_i \arrow[r, "\pi_i"'] & X_i
    \end{tikzcd}
    \end{center}
    Such a product is often denoted by \hl{$\prod_{i \in I} X_i$}. If $\prod_{i \in I} X_i$ exists in $\calC$, then it is unique up to unique isomorphism by the universal property described above.
    
    Equivalently, the product $\prod_{i \in I} X_i$ is the \CrefAndHyperrefIfExist{definition:limit_and_colimit_of_a_diagram_in_a_category}{limit} of the \CrefAndHyperrefIfExist{definition:diagram_in_a_category_indexed_by_a_small_category}{diagram} $I \to \calC, i \mapsto X_i$, where $I$ is made into a category whose objects are the members of $I$ and whose morphisms are just the identity morphisms.


    \item A \hldef{coproduct} (or synonymously \hldef{direct sum}) of the family $\{X_i\}$ is an object $C$ of $\mathcal{C}$ together with a ``universal'' family of morphisms
    $$\iota_i : X_i \to C, \quad \text{for each } i \in I.$$
    More precisely, for any object $Y$ and any family of morphisms $\{g_i : X_i \to Y\}_{i \in I}$, there exists a unique morphism
    $$g : C \to Y$$
    making the following diagram commute for all $i \in I$, i.e. $g \circ \iota_i = g_i$:
    \begin{center}
    \begin{tikzcd}[row sep=large, column sep=large]
        X_i \arrow[r, "\iota_i"] \arrow[dr, "g_i"'] & \coprod X_i \arrow[d, "\exists ! g", dashed] \\
        & Y
    \end{tikzcd}
    \end{center}
    Such a coproduct is often denoted by \hl{$\coprod_{i \in I} X_i$} or \hl{$\oplus_{i \in I} X_i$}. If $\coprod_{i \in I} X_i$ exists in $\calC$, then it is unique up to unique isomorphism by the universal property described above.

    Equivalently, the coproduct $\coprod_{i \in I} X_i$ is the \CrefAndHyperrefIfExist{definition:limit_and_colimit_of_a_diagram_in_a_category}{colimit} of the \CrefAndHyperrefIfExist{definition:diagram_in_a_category_indexed_by_a_small_category}{diagram} $I \to \calC, i \mapsto X_i$, where $I$ is made into a category whose objects are the members of $I$ and whose morphisms are just the identity morphisms.
\end{enumerate}
\end{definition}

\section{Other set theories}

\subsection{First order logic}

\subsection{}

\begin{definition}[Grothendieck Universe] \label{definition:grothendieck_universe}
    Let $U$ be a set. We say $U$ is a \hldef{Grothendieck universe} (or just a \hldef{universe}) if the following conditions hold:
    \begin{enumerate}
        \item If $x \in U$ and $y \in x$, then $y \in U$ (transitivity).
        \item If $x,y \in U$, then $\{x,y\} \in U$ (closed under pair formation).
        \item If $x \in U$, then the power set $\mathcal{P}(x) \in U$.
        \item If $I \in U$ and $(x_\alpha)_{\alpha \in I}$ is a family with each $x_\alpha \in U$, then $\bigcup_{\alpha \in I} x_\alpha \in U$.
    \end{enumerate}
    A set $X$ is called \hldef{$U$-small} or a \hldef{$U$-set} if $X \in U$.
\end{definition}


\begin{definition} \label{definition:tarski_grothendieck_set_theory}
\hldef{Tarski--Grothendieck set theory}, denoted by \hl{$\mathsf{TG}$}, is the theory consisting of the axioms of \CrefAndHyperrefIfExist{definition:zermelo_fraenkel_set_theory}{$\mathsf{ZFC}$} together with \hldef{Tarski's Axiom of Universes}, which asserts that for every set $x$ there exists a \CrefAndHyperrefIfExist{definition:grothendieck_universe}{universe} $U$ such that $x \in U$:
$$ \hlin{ \forall x \, \exists U \, (U \text{ is a Grothendieck universe} \land x \in U) } $$

\end{definition}

\begin{definition} \label{definition:proper_class_in_a_two_sorted_first_order_language_with_variables_for_sets_and_classes}
    Let $\mathcal{L}_{\text{Class}}$ be a two-sorted first-order language with variables for sets (lowercase $x, y, z, \dots$) and variables for classes (uppercase $X, Y, Z, \dots$). We define a class $X$ to be \hldef{proper} if it is not equal to any set (i.e., $\neg \exists x (X = x)$, treating sets as classes that are elements of other classes).
\end{definition}

\begin{definition} \label{definition:von_neumann_bernays_godel_set_theory}
    \TODO{two-sorted first-order language}
Let $\mathcal{L}_{\text{Class}}$ be a two-sorted first-order language with variables for sets (lowercase $x, y, z, \dots$) and variables for classes (uppercase $X, Y, Z, \dots$).

\hldef{von Neumann--Bernays--Gödel set theory}, denoted \hl{$\mathsf{NBG}$}, is the theory in $\mathcal{L}_{\text{Class}}$ comprised of the following axioms:
\TODO{theory}
\begin{enumerate}
    \item \textbf{Extensionality:} Classes with the same elements are equal.
    \item \textbf{Foundation:} Every non-empty class of sets has an $\in$-minimal element.
    \item \textbf{Class Comprehension (Predicative):} For any formula $\phi$ in which quantification occurs only over sets, there exists a class $X$ such that:
    $$  \forall x \, (x \in X \iff \phi(x)) $$
    \item \textbf{Limitation of Size:} A class $X$ is a set if and only if there is no bijection between $X$ and the class of all sets $V$:
    $$ \exists x (X = x) \iff \neg (X \cong V) $$
    \item \textbf{Set Axioms:} The standard \CrefAndHyperrefIfExist{definition:zermelo_fraenkel_set_theory}{$\mathsf{ZFC}$} axioms of Pairing, Power Set, Union, Infinity, and Choice, relativized to sets.
\end{enumerate}
\end{definition}

\begin{definition} \label{definition:morse_kelley_set_theory}

Let $\mathcal{L}_{\text{Class}}$ be a two-sorted first-order language with variables for sets (lowercase $x, y, z, \dots$) and variables for classes (uppercase $X, Y, Z, \dots$).

\hldef{Morse--Kelley set theory}, denoted \hl{$\mathsf{MK}$}, consists of the axioms of \CrefAndHyperrefIfExist{definition:von_neumann_bernays_godel_set_theory}{$\mathsf{NBG}$} but with the \hldef{Impredicative Class Comprehension Schema}. That is, for \textit{any} formula $\phi$ in $\mathcal{L}_{\text{Class}}$ (allowing quantification over classes), there exists a class $X$ such that:
$$ \forall x \, (x \in X \iff \phi(x)) $$
\end{definition}


\begin{definition} \label{defintion:stratification_of_variables_in_a_formula_in_a_first_order_language_with_a_binary_relation}
    \TODO{formula, first-order language}
    Let $\mathcal{L}_{\in}$ be a first-order language with a single binary relation symbol $\in$.
    Let $\phi$ be a formula in the language $\mathcal{L}_{\in}$. A function $\sigma: \text{Vars}(\phi) \to \mathbb{N}$ mapping variables in $\phi$ to natural numbers is called a \hldef{stratification} if for every atomic formula $x \in y$ occurring in $\phi$, we have $\sigma(y) = \sigma(x) + 1$, and for every atomic formula $x = y$, we have $\sigma(x) = \sigma(y)$.

    A formula $\phi$ is called \hldef{stratified} if it admits a stratification.
\end{definition}

\begin{definition} \label{definition:new_foundations_with_urelements}
    \TODO{formula, first-order language}
    Let $\mathcal{L}_{\in}$ be a first-order language with a single binary relation symbol $\in$.
\hldef{New Foundations with Urelements}, denoted \hl{$\mathsf{NFU}$}, is a theory allowing for \hldef{urelements} (objects distinct from the empty set that have no elements) and includes the following axioms:
\begin{enumerate}
    \item \textbf{Weak Extensionality:} Non-empty sets with the same elements are equal.
    \item \textbf{Stratified Comprehension:} For any stratified formula $\phi(x)$ where $y$ does not occur free, the set of all $x$ satisfying $\phi$ exists:
    $$ \hlin{ \exists y \forall x \, (x \in y \iff \phi(x)) } $$
\end{enumerate}
Unlike $\mathsf{ZFC}$, $\mathsf{NFU}$ allows for a universal set $V$ such that $V \in V$, as the formula $x = x$ is stratified.
\end{definition}

\begin{definition} \label{definition:constructive_zermelo_fraenkel_set_theory}
\hldef{Constructive Zermelo--Fraenkel set theory}, denoted \hl{$\mathsf{CZF}$}, is a theory based on \hldef{intuitionistic logic} (rejecting the Law of Excluded Middle) comprising the following modifications to standard $\mathsf{ZFC}$:
\begin{enumerate}
    \item \textbf{Restricted Separation:} The Separation schema is restricted to bounded ($\Delta_0$) formulas.
    \item \textbf{Subset Collection:} Replaces the Power Set axiom (which is too strong constructively) with a schema stating that for any sets $A, B$, there exists a set $C$ of "multivalued functions" covering all total relations from $A$ to $B$.
    \item \textbf{Strong Collection:} A constructive replacement for the Replacement and Collection schemas.
\end{enumerate}
\end{definition}


