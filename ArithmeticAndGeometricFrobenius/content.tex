The purpose of this document is to list contexts in which arithmetic and geometric Frobenius maps arise, define them precisely, and discuss how these maps are related.

The point is (Remark \ref{remark:geometric_frobenii_should_be_maps_induced_by_base_change_by_frobenius_endomorhism_of_the_base_ring}) that geometric Frobenii should be mappings induced by base changes via $p^n$th power maps on the base ring (of prime characteristic $p$).

%% Delete this \nocite command invocation to make the references section only list out the bibitems that are actually cited.
\section{Frobenius endomorphisms of rings of prime characteristic}


\begin{convention} \label{convention:characteristic_of_base_ring}
Unless otherwise specified, all base rings (and in particular fields) $R$ for which $R$-algebras and $R$-schemes are considered are of prime characteristic \hl{$p$}.
%  \newglossaryentry{lalala}{name={$p$}, description={the prime number characteristic of a base ring/field $R$. See Convention \ref{convention:characteristic_of_base_ring}}}
\end{convention}

\begin{definition} \label{definition:absolute_frobenius_endomorphism_of_a_ring}
    Let $R$ be a commutative ring of prime characteristic $p$. Let $n \geq 0$. The
    \defin{absolute $p^n$th power Frobenius endomorphism of $R$}{absolute_p_nth_power_frobenius_endomorphism_of_a_ring_of_prime_characteristic}{
        name={Absolute $p^n$th power Frobenius endomorphism of a ring of prime characteristic},
        description={The ring endomorphism $R \to R$ given by $a \mapsto a^{p^n}$ where $R$ is a commutative ring of prime characteristic $p$ and $n \geq 0$},
        sort={Frobenius},
        }
    is the ring endomorphism $R \to R$ given by $a \mapsto a^{p^n}$. The \defin{absolute Frobenius endomorphism of $R$}{absolute_frobenius_endomorphism_of_a_ring_of_prime_characteristic}{
        name={Absolute Frobenius endomorphism of $R$},
        description={The ring endomorphism $R \to R$ given by $a \mapsto a^{p}$ where $R$ is a commutative ring of prime characteristic $p$},
        sort={Frobenius},
    }
    (or
    \defin{Frobenius endomorphism of $R$}{frobenius_endomorphism_of_a_ring_of_prime_characteristic}{
        name={Frobenius endomorphism of $R$},
        description={Synonym for the absolute Frobenius endomorphism of a commutative ring of prime characteristic},
        sort={Frobenius},
    }
    )
    simply refers to the $p$th power Frobenius endomorphism of $R$. Denote by \hl{$R^p$} the image of the $p$th power Frobenius endomorphism of $R$.

\end{definition}


\begin{remark}
    Note that the Frobenius endomorphism of a ring $R$ of prime characteristic is not necessarily surjective; in particular, it may not be an automorphism.
\end{remark}

\begin{remark} \label{remark:geometric_frobenii_should_be_maps_induced_by_base_change_by_frobenius_endomorhism_of_the_base_ring}
    In general, a \hldef{geometric Frobenius} should be a mapping created by base changes via $\varphi$ or equivalently by fiber products with $\varphi^*$ where $\varphi: R \to R$ is some $p^n$th power Frobenius endomorphism of the commutative base ring $R$ of prime characteristic $p$. 
\end{remark}


\section{Frobenius automorphisms of algebraic extensions of finite fields}

\begin{notation}
    For a prime power $q$, let \hl{$\Fq$} be the finite field of $q$ elements.
\end{notation}

\begin{claim}
    All fields of finite cardinality are isomorphic to $\Fq$ for some prime power $q$. 
\end{claim}

\begin{definition} \label{definition:arithmetic_and_geometric_frobenius_automorphisms_of_algebraic_extensions_of_finite_fields}
    Let $k$ be either a finite field $\Fq$ of $q$ elements or the algebraic closure $\Fqbar$ thereof and let $p = \Char k$. 
    \begin{enumerate}

        \item For $n \geq 0$, the \hldef{$p^n$th power Frobenius automorphism on $k$} is the \hyperrefIfExists{definition:absolute_frobenius_endomorphism_of_a_ring}{absolute $p^n$th Frobenius endomorphism of $k$}\CrefIfExists{definition:absolute_frobenius_endomorphism_of_a_ring}; it is a field automorphism.

        \item When $k$ is a finite field and $l$ is an algebraic extension of $k$, the \hldef{arithmetic Frobenius automorphism on $l$ (over the base field $k$)} usually refers to the $|k|$th power Frobenius automorphism on $l$; it is an element of $\Gal(l/k)$ that we usually denote by \hl{$\Frob_{l/k}$}. We often simply let the finite field $k$ be the base field and also denote $\Frob_{l/k}$ by \hl{$\Frob_{l}$}. 
        \item the \hldef{geometric Frobenius automorphism on $l$} refers to the element $\Frob_{l/k}^{-1} \in \Gal(l/k)$.
    \end{enumerate}
    A corresponding automorphism $\Spec l \to \Spec l$ may also be referred to as an \hldef{arithmetic/geometric Frobenius automorphism} as appropriate.
\end{definition}

% \begin{lemma}
%     For a finite field $\Fq$ or an algebraic closure thereof, the absolute Frobenius endomorphisms coincide with the arithmetic Frobenius automorphisms.
% \end{lemma}
% \begin{proof}
%     This is immediate from the definitions.
% \end{proof}

\begin{claim}
    \TODO{TODO: define topologically generating/generating}
    Let $k$ be a finite field and let $l$ be an algebraic extension of $k$. The \hyperrefIfExists{definition:arithmetic_and_geometric_frobenius_automorphisms_of_algebraic_extensions_of_finite_fields}{arithmetic Frobenius element $\Frob_{l} = \Frob_{l/k}$}\CrefIfExists{definition:arithmetic_and_geometric_frobenius_automorphisms_of_algebraic_extensions_of_finite_fields} generates (resp. topologically generates) the Galois group $\Gal(l/k)$ if $l/k$ is a finite (resp. profinite) extension.
    % \begin{enumerate}
    %     \item If $l/k$ is a finite extension, then the element $\Frob_{l} = \Frob_{l/k}$ generates the finite group $\Gal(l/k)$\CrefIfExists{definition:arithmetic_and_geometric_frobenius_automorphisms_of_algebraic_extensions_of_finite_fields}.
    %     \item If $l/k$ is an infinite extension, then the element $\Frob_{l} = \Frob_{l/k}$ topologically generates the profinite group $\Gal(l/k)$. 
    % \end{enumerate}
\end{claim}

\section{Frobenius morphisms of schemes over rings of prime characteristic} \label{section:frobenius_morphisms_of_schemes_wikipedia}

For a scheme of prime characteristic, there are various notions of Frobenius maps/objects. We mostly follow the presentation in Wikipedia \cite{wiki:frobenius_endomorphisms} for the names of these Frobenii. Note, however, that we define the below notions for general powers $p^n$ of the prime $p$, whereas loc.~cit.~ defines the notions only as $p$th powers.

\begin{definition} \label{definition:characteristic_a_prime_for_a_scheme}
    Let $p$ be a prime number. A scheme $X$ is said to be of \hldef{characteristic $p$} if $p \cdot \scrO_X = 0$, i.e. for all open $U \subseteq X$, the ring $\Gamma(U, \scrO_X)$ of sections has characteristic $p$.
\end{definition}

\begin{lemma}
    Let $p$ be a prime number. A scheme $X$ is of \hyperrefIfExists{definition:characteristic_a_prime_for_a_scheme}{characteristic $p$} if and only if the structure morphism $X \to \Spec \bbZ$ factors through $X \to \Spec \Fp$. 
\end{lemma}

\subsection{Absolute Frobenius morphisms on a scheme of prime characteristic}

Below defines the notion of an absolute Frobenius morphism of a scheme of prime characteristic, which does not depend on any choice of base scheme.


\begin{definition} \label{definition:absolute_frobenius_morphism_on_a_scheme}
    Let $p$ be a prime number. Let $X$ be a scheme of characteristic $p$. Let $n \geq 0$. 
    \begin{enumerate}
        \item The \hldef{$p^n$th power absolute Frobenius morphism on $X$} is the scheme endomorphism $X \to X$, denoted commonly by notations such as \hl{$F_{X,p^n}$} and \hl{$\Frob_{X,p^n}$} unless otherwise specified, given by the identity on the underlying topological space and affine locally by morphisms $\Spec A \to \Spec A$ whose corresponding ring morphism $A \to A$ is the \hyperrefIfExists{definition:absolute_frobenius_endomorphism_of_a_ring}{$p^n$th power absolute Frobenius endomorphism} \CrefIfExists{definition:absolute_frobenius_endomorphism_of_a_ring}. 

        In the case that $n = 1$, this is referred to as the \hldef{absolute Frobenius morphism on $X$} and notated as \hl{$F_{X}$}.

        \item Assuming that $F_{X,p^n}$ is invertible, we may write \hl{$F_{X, 1/p^n}$} or \hl{$\Frob_{X,1/p^n}$} to denote its inverse. We might also sometimes refer to this as the \hldef{$1/p^n$th power absolute Frobenius morphism on $X$}.
    \end{enumerate}    
\end{definition}

\begin{definition} \label{definition:restriction_of_scalars_by_Frobenius_for_a_scheme}
    Let $S$ be a scheme of prime characteristic $p$. Let $X$ be an $S$-scheme and say that $\varphi: X \to S$ is the structure morphism. Let $n \in \bbZ$.

    The \hldef{restriction of scalars for $X/S$ by $p^n$th power Frobenius} is the $S$-scheme denoted commonly by notations such as \hl{$X/S_{F, p^n}$}, \hl{$X/S_{\Frob, p^n}$} or \hl{$X_{F, p^n}$}, \hl{$X_{\Frob, p^n}$} if the base scheme $S$ is clear, whose underlying scheme is $X$ and whose structure morphism is given by the composition
    $$X \xrightarrow{\varphi} S \xrightarrow{F_{S,p^n}}S$$
    where \hyperrefIfExists{definition:absolute_frobenius_morphism_on_a_scheme}{$F_{S,p^n}$} is the \hyperrefIfExists{definition:absolute_frobenius_morphism_on_a_scheme}{$p^n$th absolute Frobenius morphism on $S$}\CrefIfExists{definition:absolute_frobenius_morphism_on_a_scheme}. For $n < 0$, note that this definition is only well defined under the assumption that $F_{S,p^{|n|}}$ is invertible.  In the case that $n = 1$, this is referred to as the \hldef{restriction of scalars for $X$ by Frobenius} and commonly denoted by notations such as \hl{$X_F$} and \hl{$X_{\Frob}$}. 
\end{definition}

\begin{claim}
    Let $S$ be a scheme of prime characteristic $p$. For any $n \geq 0$, the assignment \hyperrefIfExists{definition:restriction_of_scalars_by_Frobenius_for_a_scheme}{$X \mapsto X_{F, p^n}$} is functorial.
\end{claim}

\subsection{Restrict and Extension of scalars for a scheme over a scheme of prime characteristic}

\begin{definition} \label{definition:extension_of_scalars_of_a_scheme_of_prime_characteristic_by_frobenius}
    Let $S$ be a scheme of prime characteristic $p$. Let $X$ be an $S$-scheme. Let $n \in \bbZ$.
    The \hldef{extension of scalars of $X$ by $p^n$th power Frobenius} is the base change
    $$\hlin{X^{(p^n)/S} = X^{(p^n)} \coloneq X \times_S S_{\Frob, p^n}}$$
    where \hyperrefIfExists{definition:restriction_of_scalars_by_Frobenius_for_a_scheme}{$S_{\Frob, p^n} \to S$} denotes the \hyperrefIfExists{definition:restriction_of_scalars_by_Frobenius_for_a_scheme}{restriction of scalars of $S$} (as an $S$-scheme) by $p^n$th power Frobenius \CrefIfExists{definition:restriction_of_scalars_by_Frobenius_for_a_scheme}; in other words, $S_{F, p^n} \to S$ is simply $S \xrightarrow{F_{S,p^n}} S$ where $F_{S,p^n}$ is the \hyperrefIfExists{definition:absolute_frobenius_morphism_on_a_scheme}{$p^n$th absolute Frobenius morphism on $S$} \CrefIfExists{definition:absolute_frobenius_morphism_on_a_scheme}. Alternatively, we have the following Cartesian diagram:
    \begin{center}
        \begin{tikzcd}
        X^{(p^n)} \ar[r] \ar[d] & X \ar[d] \\
        S = S_{\Frob, p^n} \ar[r, "\Frob_{S, p^n}"] & S
        \end{tikzcd}
    \end{center}
    For $n < 0$, note that this definition is only well defined under the assumption that $F_{S,p^{|n|}}$ is invertible. In the case that $n = 1$, this is referred to as the \hldef{extension of scalars of $X$ by Frobenius}. In any case a \hldef{twist of $X$ by Frobenius} is a synonym for ``extension of scalars of $X$ by Frobenius''. 

        % \item Assume that the absolute Frobenius morphism $F_{S}$ (Definition \ref{definition:absolute_frobenius_morphism_on_a_scheme}) is invertible. \TODO{TODO: hyperref to $S_{F, 1/p^n}$}
        % Let \hl{$X^{(1/p^n)/S} = X^{(1/p^n)}$} denote $X \times_S S_{F,1/p^n}$. 

\end{definition}

\subsection{Relative, arithmetic, and geometric Frobenius morphisms of schemes over schemes of prime characteristic}

We now can define three kinds of Frobenius morphisms (\Cref{definition:relative_frobenius_morphism_of_a_scheme_over_a_scheme_of_prime_characteristic}, \Cref{definition:arithmetic_frobenius_morphism_of_a_scheme_over_a_scheme_of_prime_characteristic}, \Cref{definition:geometric_frobenius_of_a_scheme_over_a_scheme_of_prime_characteristic_with_invertible_absolute_frobenius}), which unlike the \hyperrefIfExists{definition:absolute_frobenius_morphism_on_a_scheme}{absolute morphism of a scheme}, are all ``relative'' and thus depend on the base scheme $S$ over which the scheme $X$ is defined.


\begin{definition} \label{definition:relative_frobenius_morphism_of_a_scheme_over_a_scheme_of_prime_characteristic}
    Let $S$ be a scheme of prime characteristic $p$. Let $X$ be an $S$-scheme and let $\varphi: X \to S$ be a structure morphism. Let $n \in \bbZ$.
    The \hldef{relative $p^n$th power Frobenius morphism of $X/S$} is the morphism
    $$(F_{X,p^n}, \varphi): X \to  X \times_S S_{F,p^n} = X^{(p^n)/S}$$
    where
    \begin{itemize}
        \item $X^{(p^n)/S}$ denotes the \hyperrefIfExists{definition:extension_of_scalars_of_a_scheme_of_prime_characteristic_by_frobenius}{extension of scalars of $X/S$ by the $p^n$th power Frobenius}\CrefIfExists{definition:extension_of_scalars_of_a_scheme_of_prime_characteristic_by_frobenius}
        \item $S_{F,p^n}$ denotes the \hyperrefIfExists{definition:restriction_of_scalars_by_Frobenius_for_a_scheme}{restriction of scalars of $S$} (as an $S$-scheme)\CrefIfExists{definition:restriction_of_scalars_by_Frobenius_for_a_scheme}, and 
        \item $\varphi$ is regarded as a morphism $X \to S_{F,p^n}$ by regarding $S_{F,p^n}$ as a copy of $S$. 
    \end{itemize}
    The relative Frobenius morphism above may commonly be denoted by notations such as \hl{$F_{X/S, p^n}$}, \hl{$F_{X|S, p^n}$}, \hl{$Frob_{X/S, p^n}$}, and \hl{$Frob_{X|S, p^n}$}. For $n < 0$, note that this definition is only well defined under the assumption that $F_{S,p^{|n|}}$ is invertible. 


    In the case that $n = 1$, this is referred to as the \hldef{relative Frobenius morphism of $X/S$} and may commonly be denoted by notations such as \hl{$F_{X/S}$}, \hl{$F_{X|S}$}, \hl{$Frob_{X/S}$}, and \hl{$Frob_{X|S}$}.
\end{definition}

\begin{remark} \label{remark:relative_frobenius_morphisms_of_schemes_over_characteristic_p_fields_are_very_often_considered}
In case $S$ is the spectrum of a field $k$ in characteristic $p$, the relative $p^n$th power Frobenius morphism $X \to X^{(p^n)}$ is very often considered. 
If a text considers a morphism $X \to X^{(p^n)}$ of a variety $X$ over such a field $k$ and simply describes the morphism by stating that it sends a point in $X$ with coordinates $(x_i)$ to the point in $x^{(p^n)}$ with coordinates $(x_i^{p^n})$, then the text is most likely considered the relative Frobenius morphism of $X$ as a $k$-scheme. 

For instance, the ($p^n$th power) Frobenius isogeny $E \to E^{(p^n)}$ of an elliptic curve $E/k$ is simply the relative $p^n$th power Frobenius morphism when $X = E/k$.
\end{remark}

\begin{definition} \label{definition:arithmetic_frobenius_morphism_of_a_scheme_over_a_scheme_of_prime_characteristic}
    Let $S$ be a scheme of prime characteristic $p$. Let $X$ be an $S$-scheme. Let $n \geq 1$ be an integer. The \hldef{arithmetic $p^n$th power Frobenius morphism of $X/S$} is the morphism 
    $$(\id_X, F_{S,p^n}): X^{(p^n)/S} \cong X \times_S S \to X \times_S S \cong X$$
    where $X^{(p^n)/S}$ denotes the \hyperrefIfExists{definition:extension_of_scalars_of_a_scheme_of_prime_characteristic_by_frobenius}{extension of scalars of $X/S$ by the $p^n$th power Frobenius}\CrefIfExists{definition:extension_of_scalars_of_a_scheme_of_prime_characteristic_by_frobenius} and $F_{S,p^n}: S \to S$ is the \hyperrefIfExists{definition:absolute_frobenius_morphism_on_a_scheme}{$p^n$th power absolute Frobenius morphism}\CrefIfExists{definition:absolute_frobenius_morphism_on_a_scheme}.  

    In the case that $n = 1$, this is referred to as the \hldef{arithmetic Frobenius morphism of $X/S$}. In the case that the \hyperrefIfExists{definition:absolute_frobenius_morphism_on_a_scheme}{absolute Frobenius morphism $F_S$} on $S$ is invertible, the arithmetic Frobenius morphism of $X/S$ is sometimes referred to as an \hldef{Frobenius automorphism of $S$} (see e.g. \cite[Section I.1]{kiehl_weissauer_wcps}).
    
\end{definition}


Unlike the arithmetic Frobenius morphism of $X/S$, the geometric Frobenius morphism of $X/S$ requires that the absolute Frobenius morphism $F_{S}$ on the base scheme $S$ be invertible. For instance, this happens if $S$ is $\Spec \Fq$ or $\Spec \Fqbar$.  

\begin{definition} \label{definition:geometric_frobenius_of_a_scheme_over_a_scheme_of_prime_characteristic_with_invertible_absolute_frobenius}
    Let $S$ be a scheme of prime characteristic $p$. Assume that the absolute Frobenius morphism $F_{S}$ (Definition \ref{definition:absolute_frobenius_morphism_on_a_scheme}) is invertible. Let $n \geq 1$ be an integer.
    The \hldef{geometric $p^n$th Frobenius morphism of $X/S$} is the morphism
    $$(\id_X, F_{X,1/p^n}):  X^{(1/p^n)} = X \times_S S_{F,1/p^n} \to X \times_S S \cong X$$
    where $X^{(1/p^n)}$ is the \hyperrefIfExists{definition:extension_of_scalars_of_a_scheme_of_prime_characteristic_by_frobenius}{extension of scalars of $X/S$ by the $1/p^n$th power Frobenius morphism}\CrefIfExists{definition:extension_of_scalars_of_a_scheme_of_prime_characteristic_by_frobenius} and $S_{F,1/p^n}$ is the \hyperrefIfExists{definition:restriction_of_scalars_by_Frobenius_for_a_scheme}{restriction of scalars for $S$ (over $S$) by the $p^n$th power Frobenius}\CrefIfExists{definition:restriction_of_scalars_by_Frobenius_for_a_scheme}.

    In the case that $n = 1$, this is referred to as the \hldef{geometric Frobenius morphism of $X/S$}.
\end{definition}

\begin{remark}
The \hyperrefIfExists{definition:arithmetic_frobenius_morphism_of_a_scheme_over_a_scheme_of_prime_characteristic}{arithmetic} and \hyperrefIfExists{definition:geometric_frobenius_of_a_scheme_over_a_scheme_of_prime_characteristic_with_invertible_absolute_frobenius}{geometric Frobenii} of a scheme $X/S$ act inversely on rational points\CrefIfExists{proposition:arithmetic_and_geometric_frobenius_details}.
\end{remark}

\section{Interpretations of the arithmetic and geometric Frobenii of schemes over finite fields}

\begin{proposition} \label{proposition:twist_of_scheme_is_isomorphic_to_scheme_if_absolute_frobenius_on_base_is_identity}
    Let $X$ be a scheme over an $S$-scheme of prime characteristic $p$. For $n \geq 0$, assume that the \hyperrefIfExists{definition:absolute_frobenius_morphism_on_a_scheme}{$p^n$th power absolute Frobenius morphism $F_{S, p^n}: S \to S$} is the identity morphism on $S$. The \hyperrefIfExists{definition:extension_of_scalars_of_a_scheme_of_prime_characteristic_by_frobenius}{extension of scalars/twist $X^{(p^n)}$} is naturally isomorphic to $X$ as an $S$-scheme. 
\end{proposition}
\begin{proof}
    By definition, $X^{(p^n)}$ fits in the following Cartesian diagram:
    \begin{center}
    \begin{tikzcd}
        X^{(p^n)} \ar[r] \ar[d] & S_{F,p^n} \ar[d, "F_{S,p^n} \circ \id"] \\
        X \ar[r] & S
    \end{tikzcd}
    \end{center}
    Under the assumption that $F_{S,p^n}$ is the identity map, $S_{F,p^n} \to S$ is the identity map $S \to S$ and hence $X^{(p^n)} \to S_{F,p^n}$ is naturally isomorphic to $X \to S$.
\end{proof}

\begin{corollary}
    Let $X$ be a scheme over $\Fq$. For all $n \geq 0$, the \hyperrefIfExists{definition:extension_of_scalars_of_a_scheme_of_prime_characteristic_by_frobenius}{extension of scalars/twist $X^{(q^n)}$} is naturally isomorphic to $X$ as an $\Fq$-scheme.
\end{corollary}

\begin{proposition} \label{proposition:arithmetic_and_geometric_frobenius_details}
    Let $S$ be a scheme of prime characteristic $p$. Let $X$ be a scheme over $S$; say that the structure morphism $X \to S$ is affine locally given by $\Spec A[X_i]_{i \in I}/(f_j)_{j \in J} \to \Spec A$ where $A[X] = A[X_i]_{i \in I}$ is the polynomial ring over $A$ in some set of generators $\{X_i\}_{i \in I}$ and $f_j \in A[X_i]_{i \in I}$ are some polynomials in the variables $X_i$\footnote{Note that each $f_j$ depends only on finitely many of the variables $X_i$.}. For $n \in \bbZ$ and for $f \in A[X_i]_{i \in I}$, let $f^{(p^n)}$ be the polynomial whose coefficients are $p^n$th powers of those of $f$ assuming that such $p^n$th powers exist, i.e. writing $f$ as a finite sum $f = \sum_K a_K X^K$ with $a_K \in A$, we have $f^{(p^n)} = \sum_K a_K^{p^n} X^K$.

    \begin{enumerate}
        \item For $n \geq 0$, the \hyperrefIfExists{definition:arithmetic_frobenius_morphism_of_a_scheme_over_a_scheme_of_prime_characteristic}{arithmetic $p^n$th power Frobenius morphism} $X^{(p^n)} \to X$\CrefIfExists{definition:extension_of_scalars_of_a_scheme_of_prime_characteristic_by_frobenius} is given affine locally by \label{part:arithmetic_frobenius_details}  
        $$\Spec A[Y_i]_{i \in I} / (f^{(p^n)}_j)_{j \in J} \to \Spec A[X_i]_{i \in I} / (f_j)_{j \in J}.$$
        The corresponding ring morphism is $A[X_i]_{i \in I} / (f_j)_{j \in J} \to A[Y_i]_{i \in I} / (f^{(p^n)}_j)_{j \in J}$ given by $\sum_K a_K X^K \mapsto \sum_K a_K^{p^n} Y^K$. 
        Furthermore, a point $(y_i)_{i \in I}$ of $X^{(p^n)}(A)$ is sent to the point $(y_i^{p^n})_{i \in I}$ of $X(A)$.  

        \item Assume that the \hyperrefIfExists{definition:absolute_frobenius_morphism_on_a_scheme}{absolute Frobenius morphism on $S$} is invertible. For $n \geq 0$, the \hyperrefIfExists{definition:geometric_frobenius_of_a_scheme_over_a_scheme_of_prime_characteristic_with_invertible_absolute_frobenius}{geometric $p^n$th power Frobenius morphism} $X^{(1/p^n)} \to X$ \label{part:geometric_frobenius_details}
        is given affine locally by 
        $$\Spec A[Y_i]_{i \in I} / (f^{(1/p^n)}_j)_{j \in J} \to \Spec A[X_i]_{i \in I} / (f_j)_{j \in J}.$$
        The corresponding ring morphism is $A[X_i]_{i \in I} / (f_j)_{j \in J} \to A[Y_i]_{i \in I} / (f^{(1/p^n)}_j)_{j \in J}$ given by $\sum_K a_K X^K \mapsto \sum_K a_K^{1/p^n} Y^K$. 
        Furthermore, a point $(y_i)_{i \in I}$ of $X^{(1/p^n)}(A)$ is sent to the point $(y_i^{1/p^n})_{i \in I}$ of $X(A)$.  


    \end{enumerate}
\end{proposition}

\begin{proof}
    We prove \ref{part:arithmetic_frobenius_details}; the proof of \ref{part:geometric_frobenius_details} is nearly identical. Affine locally, the arithmetic $p^n$th power Frobenius morphism is given by 
    $$(\id, F_{\Spec A, p^n}): \Spec A[Y] / (f) \times_{\Spec A} (\Spec A)_{F,p^n} \to \Spec A[X] / (f) \times_{\Spec A} \Spec A$$
    where 
    $$\id: \Spec A[Y] / (f) \to \Spec A[X] / (f)$$
    is the identity map where the source and target are written with different coordinates for distinction, and where $F_{\Spec A, p^n}:(\Spec A)_{F,p^n} \to \Spec A$ is the $p^n$th power absolute Frobenius morphism (Definition \ref{definition:absolute_frobenius_morphism_on_a_scheme}) on $\Spec A$. 

    Further write 
    $$\Spec A[Y] / (f) \times_{\Spec A} (\Spec A)_{F,p^n}$$
    as
    $$\Spec A_1[Y] / (f) \times_{\Spec A_1} (\Spec A_2)_{F,p^n}$$
    where $A_1$ and $A_2$ are copies of $A$ and the morphism $\Spec A_2 \to \Spec A_1$ that specifies this fibered product is $F_{\Spec A, p^n}$. In other words, the element $a \in A_1$ becomes $a^{p^n} \in A_2$ under the corresponding ring homomorphism $A_1 \to A_2$. We thus have the isomorphism
    $$\Spec A_1[Y] / (f) \times_{\Spec A_1} (\Spec A_2)_{F,p^n} \cong \Spec A_2[Y] / (f^{(p^n)})$$
    in such a way that arithmetic Frobnenius morphism is affine locally given by 
    $$\Spec A[Y] / (f^{(p^n)}) \to \Spec A[X]/(f)$$
    whose corresponding ring morphism sends $\sum_K a_K X^K$ to $\sum_k a_K^{p^n} Y^K$ as claimed. Note moreover that the $A$-valued point $(y_i)_{i \in I}$ of the domain is sent to $(y_i^p)_{i \in I}$ of the codomain.

\end{proof}

\section{Other, possibly conflicting, definitions of Frobenii on schemes}

There are different and even genuinely conflicting definitions given for various Frobenii on schemes other than those listed in Section \ref{section:frobenius_morphisms_of_schemes_wikipedia}. 

\begin{remark}
    Let $X/\Fq$ be a scheme. The notions of arithmetic and geometric Frobenii are often alternatively defined in the following manner:    

    \begin{enumerate}

        \item The arithmetic Frobenius may be defined as the morphism $\id \times F_{\Spec \Fqbar, q}: X \times_{\Spec \Fq} \Spec \Fqbar \to X \times_{\Spec \Fq} \Spec \Fqbar$; recall that $F_{\Spec \Fqbar, q}$ denotes the $q$th power absolute Frobenius morphism on $\Spec \Fqbar$ (Definition \ref{definition:absolute_frobenius_morphism_on_a_scheme}). Note that this arithmetic Frobenius morphism is only defined over $\Fqbar$ and not over $\Fqbar$ and that this morphism is always invertible. Furthermore, this notion of arithmetic Frobenius is a specialization of that of Definition \ref{definition:arithmetic_frobenius_morphism_of_a_scheme_over_a_scheme_of_prime_characteristic} in the case that $S = \Spec \Fq$.

        \item The geometric Frobenius may be simply defined (see e.g. \cite[Tag 03SQ]{stacks-project}) as the ($q$th power) absolute Frobenius morphism on $X$ (Definition \ref{definition:absolute_frobenius_morphism_on_a_scheme}); note that this is an $\Fq$-morphism and that this is not necessarily invertible. However, this does not coincide with the notion of the ($q$th power) geometric Frobenius defined in Definition \ref{definition:geometric_frobenius_of_a_scheme_over_a_scheme_of_prime_characteristic_with_invertible_absolute_frobenius}. We will continue to let a ``geometric Frobenius morphism of a scheme'' refer to the notion defined Definition \ref{definition:geometric_frobenius_of_a_scheme_over_a_scheme_of_prime_characteristic_with_invertible_absolute_frobenius}.

    \end{enumerate}
It seems appropriate to call the notions defined in Definitions \ref{definition:arithmetic_frobenius_morphism_of_a_scheme_over_a_scheme_of_prime_characteristic} and \ref{definition:geometric_frobenius_of_a_scheme_over_a_scheme_of_prime_characteristic_with_invertible_absolute_frobenius} respectively as arithmetic and geometric Frobenii because they are obtained by base changes via arithmetic/geometric Frobenii (Definition \ref{definition:arithmetic_and_geometric_frobenius_automorphisms_of_algebraic_extensions_of_finite_fields}) of the base field.

\end{remark}

\begin{definition}[c.f. {\cite[I.1]{kiehl_weissauer_wcps}}] \label{definition:frobenius_endomorphism_on_a_scheme_over_a_scheme_with_invertible_finite_order_absolute_frobenius_morphism}
    \TODO{TODO: how does this notion of frobenius relate to the relative Frobenius morphism?}
    Let $S$ be a scheme of prime characteristic $p$. Assume that the \hyperrefIfExists{definition:absolute_frobenius_morphism_on_a_scheme}{absolute Frobenius morphism $F_{S}$} (\CrefIfExists{definition:absolute_frobenius_morphism_on_a_scheme}) on $S$ is invertible and of order $n \geq 0$, i.e. $F_{S,p^n} : S \to S$ is the identity morphism on $S$. Let $X/S$ be a scheme. The \hldef{$p^n$th power Frobenius endomorphism on $X$} is the composition
    $$X \xrightarrow{F_{X,p^n}} X \xrightarrow{\Frob_{X/S, p^n, \geom}^{-1}} X^{(1/p^n)} \cong X,$$
    where $\Frob_{X/S, p^n, \geom}$ is the \hyperrefIfExists{definition:geometric_frobenius_of_a_scheme_over_a_scheme_of_prime_characteristic_with_invertible_absolute_frobenius}{$p^n$th power geometric Frobenius morphism of $X/S$} \CrefIfExists{definition:geometric_frobenius_of_a_scheme_over_a_scheme_of_prime_characteristic_with_invertible_absolute_frobenius}. Moreover, note that there is a natural isomorphism $X^{(1/p^n)} \cong X$ because $F_{S, p^n}$ is assumed to be the identity morphism on $S$\CrefIfExists{proposition:twist_of_scheme_is_isomorphic_to_scheme_if_absolute_frobenius_on_base_is_identity}. Unless otherwise specified, we denote this morphism by \hl{$\Frob_{X/S,p^n}$}, \hl{$\Frob_{X,p^n}$}, \hl{$\Frob_{X/S}$}, or \hl{$\Frob_{X}$} (depending on whether the base $S$ and/or the power $p^n$ is clear).
\end{definition}

\begin{remark}
    % The case of $S = \Fq$ and $p^n = q$ is of interest; it is also of interest to consider the $p^n$th power Frobenius endomorphism on $X \times_{\Fq} \Fqbar$. 
    The case of interest is when $X/\Fqbar$ is definable over $\Fq$ where $p^n = q$.
    % $S = \Fqbar$ and $p^n = q$ is of interest, assuming that $X$ is definable over $\Fq$. 
\end{remark}

\begin{remark}
    If $X$ is defined over $\Fq$, recall that $F_{x,q}$ is by definition the identity on the points. Further recall that if $(Y_i)_{i \in I}$ are affine coordinates of $X^{(1/p^n)}$, then $\Frob_{X/S, p^n, \geom}$ sends a point $(y_i)_{i \in I}$ on $X^{(1/p^n)}$ to the point $(y_i^{1/p^n})_{i \in I}$ on $X$ by \Cref{definition:absolute_frobenius_morphism_on_a_scheme}. Therefore, the $p^n$th power Frobenius endomorphism on $X$ should send a point $(x_i)_{i \in I}$ on $X$ to the point $(x_i^{p^n})_{i \in I}$ on $X$.
\end{remark}

\section{Frobenius morphisms on sheaves induced by scheme Frobenii}

\subsection{Geometric Frobenius action on a sheaf or derived object on a scheme of prime characteristic at a geometrically cloesd point}

\TODO{TODO: six functor formalism}
Given a (finite type, separated) scheme over a finite field $\Fq$ and given a constructible sheaf of $\Qellbar$ coefficients (or more generally a derived object thereof), we describe how to construct the Frobenius morphisms at the stalks. In turn, this will allow us to define trace functions for said Frobenii\CrefIfExists{definition:frobenius_trace_of_a_cohomologically_bounded_complex_of_constructible_ell_adic_sheaves_on_a_finite_type_separated_scheme_over_a_finite_field}.

\TODO{TODO: cite SGA 5 Chapter XV, Freitag-Kiehl, Kiehl-Weissauer}

cf. \cite[XV]{sga5} \cite[10.3]{fu_ect}


\begin{definition} \label{definition:geometric_frobenius_action_of_a_derived_object_on_a_scheme_of_characteristic_p_at_a_stalk}
    Let $X$ be a scheme of characteristic $p$. Let $x \in |X|$ be a closed point whose residue field is a finite field. Let $\barx$ be a geometric point over $X$. Let $K \in D(X)$ be an object in the derived category of sheaves of abelian groups on $X$. 
 %Let $n \geq 1$ be an integer.
    
    The \hyperrefIfExists{definition:arithmetic_and_geometric_frobenius_automorphisms_of_algebraic_extensions_of_finite_fields}{geometric Frobenius automorphism} in $\Gal(\overline{\kappa(x)} / \kappa(x))$ can be identified as an automorphism 
    $$\Spec \overline{\kappa(x)} \to \Spec \overline{\kappa(x)}$$
    over $\Spec \kappa(x)$ and hence induces an action 
    $$\hlin{\Frob_{\barx}: K_{\barx} \to K_{\barx}}$$
    on the stalk. Up to isomorphism this action only depends on the closed point $x$ and not the choice of the geometric point $\barx$ above $x$.

    This automorphism may be called the \hldef{geometric Frobenius action on $K$ at the stalk at $\barx$}. In case that the size of the residue field $\kappa(x)$ is $q$ (or is considered as $q$, say by virtue of changing the base field over which $X$ is defined) we may also denote the geometric Frobenius action using notations such as \hl{$\Frob_{\barx, q}$} or \hl{$\Frob_{q}$} to emphasize the size of $\kappa(x)$.
\end{definition}



\begin{definition}[e.g. {\cite[Section A.4]{forey_fresan_kowalski_aftff}}] \label{definition:frobenius_trace_of_a_cohomologically_bounded_complex_of_constructible_ell_adic_sheaves_on_a_finite_type_separated_scheme_over_a_finite_field}
    \TODO{TODO: notate the bounded derived category of constructible sheaves}
    \TODO{TODO: define stalks}
    Let $X$ be a finite type and separated scheme over a finite field $\bbF_q$, let $\ell \neq \Char \Fq$ be a prime number, and let $M \in D_c^b(X,\Qellbar)$. For points $x \in X(\bbF_q)$, write $\barx$ for a geometric point above $x$. 
    
    % The \hyperrefIfExists{definition:arithmetic_and_geometric_frobenius_automorphisms_of_algebraic_extensions_of_finite_fields}{geometric Frobenius automorphism}\CrefIfExists{definition:arithmetic_and_geometric_frobenius_automorphisms_of_algebraic_extensions_of_finite_fields} $\operatorname{Fr}_{q}: \Fqbar \to \Fqbar $ acts on the stalk $M_{\barx}$ and this action is independent of the choice of $\barx$ up to conjugation. 
    The \hldef{Frobenius trace function of $M$ over $\Fqn$} is the function $G(\Fqn) \to \Qellbar$ defined by
    $$\hlin{t_M(x;\Fqn) = \sum_{i \in \bbZ} (-1)^i \operatorname{Tr}(\operatorname{Frob}_{q^n}|H^i(M)_{\barx} )}$$
    where $\Frob_{q^n}$ is the \hyperrefIfExists{definition:geometric_frobenius_action_of_a_derived_object_on_a_scheme_of_characteristic_p_at_a_stalk}{geometric Frobenius action of the sheaf $H^i(M)$ at the stalk at $\barx$}\CrefIfExists{definition:geometric_frobenius_action_of_a_derived_object_on_a_scheme_of_characteristic_p_at_a_stalk}.
    This is independent of the choice of geometric point $\barx$ above $x$. 
\end{definition}

\subsection{Geometric Frobenius homomorphisms/Frobenius correspondences on stalks or derived objects on schemes of prime characteristic}

\begin{lemma} \label{lemma:there_is_an_isomorphism_from_an_etale_sheaf_on_a_scheme_of_characteristic_p_to_its_pushforward_under_frobenius}
    Let $X$ be a scheme of characteristic $p$. Let $n \geq 1$ be an integer. Let $\calF$ be an \'etale sheaf (of sets) on $X$. 
    There is an isomorphism 
    $$\calF \to (\Frob_{X,p^n})_*(\calF)$$
    given, for an \'etale morphism $U \to X$, by the maps
    $$\calF(U) \xrightarrow{\Frob_{U|X, p^n}} \calF(U^{(p^n)/X}) \cong \calF(\Frob_{X,p^n}^{-1}(U)) = (\Frob_{X,p^n})_*(\calF)(U)$$
    where 
    \begin{itemize}
    \item \hyperrefIfExists{definition:absolute_frobenius_morphism_on_a_scheme}{$\Frob_{X, p^n}: X \to X$} is the \hyperrefIfExists{definition:absolute_frobenius_morphism_on_a_scheme}{$p^n$th power Absolute Frobenius morphism of $X$}, and
    \item \hyperrefIfExists{definition:relative_frobenius_morphism_of_a_scheme_over_a_scheme_of_prime_characteristic}{$\Frob_{U|X, p^n}: U \to U^{(p)/X}$}\CrefIfExists{definition:extension_of_scalars_of_a_scheme_of_prime_characteristic_by_frobenius} is the \hyperrefIfExists{definition:relative_frobenius_morphism_of_a_scheme_over_a_scheme_of_prime_characteristic}{$p^n$th power relative Frobenius morphism of $U \to X$}.
    \end{itemize}
\end{lemma}

\begin{definition}
    \TODO{TODO: define etale sheaf}
    \TODO{TODO: defien adjoint morphisms; establish that pullbacks and pushforwards are adjoints}
    Let $X$ be a scheme of characteristic $p$. Let $n \geq 1$ be an integer. Let $\calF$ be an \'etale sheaf on $X$. 
    
    Define a sheaf morphism
    $$\hlin{\Frob_{\calF,p^n}^*: \Frob_{X,p^n}^* \calF \to \calF}$$
    as the \hyperrefIfExists{definition:adjoint_functors_between_categories_unit_counit_of_adjoint_functors}{adjoint morphism} to the isomorphism 
    $$\calF \to (\Frob_{X,p^n})_*(\calF)$$
    of \Cref{lemma:there_is_an_isomorphism_from_an_etale_sheaf_on_a_scheme_of_characteristic_p_to_its_pushforward_under_frobenius}; $\Frob_{\calF,p^n}^*$ is in fact a sheaf isomorphism. 

    More generally, the above construction yields a functorial morphism 
    $$\hlin{\Frob_{K,p^n}^*: \Frob_{X,p^n}^*K \to K}$$
    for any object $K$ in the derived category $D(X) = D(X,\bbZ)$ of sheaves of abelian groups on $X$.

    Such morphisms $\Frob_{\calF,p^n}^*$ and $\Frob_{K,p^n}^*$ are sometimes called the \hldef{$p^n$th power (geometric) Frobenius homomorphism} (e.g. \cite[Chapter II Definition/Lemma 3.8]{kiehl_weissauer_wcps}). A pair $(\Frob_{X,p^n}, \Frob_{\calF,p^n}^*)$ (resp. $(\Frob_{X,p^n}, \Frob_{K,p^n}^*)$) is sometimes called the \hldef{$p^n$th power Frobenius correspondence on $(X, \calF)$ (resp. $(X, K)$)}.

    As usual, when $n = 1$, we might denote the morphisms as \hl{$\Frob_{\calF}^*$} and \hl{$\Frob_K^*$}. 
\end{definition}

\begin{definition}
\TODO{TODO: notate $H_c^i$, cite the adjunction}
    Let $X$ be a scheme of characteristic $p$. Let $n \geq 1$ be an integer. Let $K \in D^+(X)$ be an object in the bounded below derived category of sheaves of abelian groups on $X$. 
    
    Note that the \hyperrefIfExists{definition:adjoint_functors_between_categories_unit_counit_of_adjoint_functors}{unit morphism} for the adjunction between $\Frob_{X,q^n}^*$ and $R\Frob_{X,q^n*}$ yields a morphism 
    $$H_c^i(X, K) \to H_c^i(X, R\Frob_{X,q^n*} \Frob_{X,q^n}^* K)$$
    on compactly supported cohomology groups. The codomain of such a map is naturally isomorphic to $H_c^i(X, \Frob_{X,q^n}^* K)$ because $\Frob_{X,q^n*}$ is proper and hence $R\Frob_{X,q^n*} \cong R\Frob_{X,q^n!}$ and because compactly supported derived pushforwards preserve $H_c^i$. We can then obtain the following endomorphism of $H_c^i(X, K)$:
    $$H_c^i(X, K) \to H_c^i(X, R\Frob_{X,q^n*} \Frob_{X,q^n}^* K) \cong H_c^i(X, \Frob_{X,q^n}^* K) \xrightarrow{\Frob_{K,p^n}^*} H_c^i(X, K).$$
\end{definition}

% \begin{definition}
%     In the case that the absolute Frobenius
%     The geometric Frobenius morphism $$
% \end{definition}


% \begin{definition}
%     Let $R$ be a ring of characteristic $p > 0$. Let $X$ be an $R$ scheme. The \hldef{$p^n$th power Frobenius base change} or \hldef{$p^n$th power Frobenius twist of $X$} is the base change \hl{$X^{(p^n)} = X \times_{\Spec R_1} \Spec R_2$} where $R_1$ and $R_2$ are each copies of $R$ and the scheme morphism $\Spec R_2 \to \Spec R_1$ is given by the $p^n$th power Frobenius endomorphism $R = R_1 \to R_2 = R$ of $R$. The natural morphism $X^{(p^n)} \to X$ is commonly called the \hldef{$p^n$th power geometric Frobenius morphism of $X$} .

%     When $n = 1$, these may be simply referred to as the \hldef{Frobenius base change/Frobenius twist} or \hldef{geometric Frobenius morphism} respectively.
% \end{definition}


% \nocite{*}